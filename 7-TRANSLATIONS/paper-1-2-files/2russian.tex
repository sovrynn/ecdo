\documentclass[10pt,twocolumn,letterpaper]{article}

% My own stuff
\usepackage{booktabs}
% \usepackage{caption}
% \captionsetup[table]{skip=8pt}   % Only affects tables
\usepackage{stfloats}  % Add this to the preamble
\usepackage[T2A]{fontenc}    % Enables Cyrillic fonts
\usepackage[utf8]{inputenc}   % Assumes your source file is in UTF-8 encoding
\usepackage[russian]{babel}   % Loads Russian language support

\usepackage{cvpr}
\usepackage{times}
\usepackage{epsfig}
\usepackage{graphicx}
\usepackage{amsmath}
\usepackage{amssymb}

\makeatletter
\def\cvprsubsection{\@startsection {subsection}{2}{\z@}
    {8pt plus 2pt minus 2pt}{6pt}{\bfseries\normalsize}}
\makeatother

% Include other packages here, before hyperref.

% If you comment hyperref and then uncomment it, you should delete
% egpaper.aux before re-running latex.  (Or just hit 'q' on the first latex
% run, let it finish, and you should be clear).
\usepackage[breaklinks=true,bookmarks=false]{hyperref}

\cvprfinalcopy % *** Uncomment this line for the final submission

\def\cvprPaperID{****} % *** Enter the CVPR Paper ID here
\def\httilde{\mbox{\tt\raisebox{-.5ex}{\symbol{126}}}}

% Pages are numbered in submission mode, and unnumbered in camera-ready
%\ifcvprfinal\pagestyle{empty}\fi
\setcounter{page}{1}
\begin{document}

%%%%%%%%% TITLE
\title{Экспериментальное руководство по данным ECDO Часть 2/2: Исследование научных и исторических аномалий, лучше всего объясняемых «переворотом Земли» по ECDO}

\author{Джунхо\\
Веб-сайт: \href{https://sovrynn.github.io}{sovrynn.github.io}\\
Репозиторий исследований ECDO: \href{https://github.com/sovrynn/ecdo}{github.com/sovrynn/ecdo}\\
{\tt\small junhobtc@proton.me}
}

\maketitle
%\thispagestyle{empty}

%%%%%%%%% ABSTRACT
\begin{abstract}
В мае 2024 года псевдонимный онлайн-автор под именем «Этичный скептик» \cite{0} опубликовал новаторскую теорию, называемую Экзотермическое отслоение ядра и мантии с осцилляцией Джанибекова (ECDO) \cite{1}. Эта теория не только предполагает, что Земля ранее переживала внезапные катастрофические изменения оси вращения, создавая массивное мировое наводнение из-за переноса океанов через континенты под влиянием инерции вращения, но и предлагает объяснительный причинный геофизический процесс с данными, указывающими, что ещё один такой переворот может быть неизбежен. Хотя такие катастрофические наводнения и предсказания конца света не новы, теория ECDO уникально привлекательна благодаря своём научному, современному, многодисциплинарному и основанному на данных подходу.

Данный исследовательский документ представляет собой вторую часть из двух частей конденсированного суммарного изложения 6 месяцев независимых исследований \cite{2,20} по теории ECDO, сосредоточивая внимание на научных и исторических аномалиях, которые лучше всего объясняются катастрофическим «переворотом Земли» по ECDO.

\end{abstract}

%%%%%%%%% BODY TEXT

\section{Введение}

Современная униформитарная геология и история утверждают, что такие крупные геологические ландшафты, как Гранд-Каньон, формировались на протяжении миллионов лет \cite{143}; что соль существует в Долине Смерти (Калифорния), потому что она находилась под океаном сотни миллионов лет назад \cite{144}; что наши предки 150 поколений назад потратили всю свою жизнь на строительство гигантских гробниц \cite{29,70}; и что так называемое «ископаемое топливо» существует сотни миллионов лет \cite{104}. Возможно, самое интересное заключается в том, что считается, что человечеству 300 000 лет \cite{145}, в то время как записанная история и цивилизация насчитывают всего около 5000 лет — эквивалент 150 поколений людей.

Такие аномалии, как мы увидим, лучше всего объясняются катастрофическими геологическими силами.

\section{Мгновенно замороженные мамонты, захороненные в грязи}

\begin{figure}[t]
\begin{center}
   \includegraphics[width=1\linewidth]{jarkov-mammoth.jpg}
\end{center}
   \caption{Мамонт Жарков, 20000-летний идеально сохранившийся сибирский мамонт, найденный в замёрзшей грязи \cite{51}.}
\label{fig:1}
\label{fig:onecol}
\end{figure}

Одна из таких категорий аномалий — это идеально сохранившиеся мгновенно замороженные мамонты, погребенные в иле, которые обычно встречаются в арктических регионах (Рисунок \ref{fig:1}). Березовский мамонт, обнаруженный в Сибири, погребенный в илистой гальке, был настолько хорошо сохранен, что его мясо оставалось съедобным тысячи лет после его смерти. У него также была растительная пища во рту и желудке, что поставило ученых в тупик: как же он мог быть так быстро заморожен, если он пасся на цветущих растениях прямо перед своей смертью \cite{17}. Сообщается, что \textit{"В 1901 году сенсацию произвело открытие полного трупа мамонта возле реки Березовка, так как, казалось, это животное умерло от холода в разгар лета. Содержимое его желудка было хорошо сохранено и включало в себя лютики и цветущие бобы: это означало, что они должны были быть проглочены примерно в конце июля или начале августа. Существо умерло так внезапно, что все еще держало в челюстях пучок трав и цветов. Оно явно было захвачено гигантской силой и выброшено на несколько миль от своего пастбища. Таз и одна нога были сломаны—огромное животное было сбито на колени и затем замерзло на смерть, в то, что обычно является самым жарким временем года"} \cite{18}. Кроме того, \textit{"[Русские ученые] отметили, что даже самая внутренняя оболочка желудка зверя имела идеально сохранившуюся волокнистую структуру, что указывает на то, что его теплообмен был удален каким-то супер-фантастическим природным процессом. Сандерсон, отмечая это один момент, обратился с этой проблемой в Американский институт замороженных продуктов: Что нужно, чтобы заморозить целого мамонта так, чтобы влажное содержание даже самых внутренних частей его тела, даже до внутренней оболочки его желудка, не успело сформировать кристаллы, достаточно большие, чтобы разрушить волокнистую структуру мяса?... Несколько недель спустя институт вернулся к Сандерсону с ответом: Это абсолютно невозможно. С учетом всех наших научных и инженерных знаний абсолютно неизвестен способ удалить тепло из трупа такого размера, как у мамонта, достаточно быстро, чтобы заморозить его без образования больших кристаллов влаги в мясе. Более того, исчерпав научные и инженерные методы, они обратились к природе и пришли к выводу, что не существует известного процесса в природе, который мог бы добиться такого результата"} \cite{19}.

\section{Гранд-Каньон}

Гранд-Каньон, часть Великого Бассейна на юго-западе Северной Америки, является еще одним природным явлением, указывающим на катастрофическое происхождение (Рисунок \ref{fig:2}). Для начала, слои осадочных песчаников и известняков, составляющие Гранд-Каньон, охватывают огромные площади, до 2,4 миллиона км$^2$ \cite{21}. Рисунок \ref{fig:3} показывает распространение слоя песчаника Коконито по западным Соединенным Штатам. Такие массивные горизонтальные слои однородного материала могли быть отложены только сразу.

\begin{figure}[t]
\begin{center}
   \includegraphics[width=1\linewidth]{grand-canyon.jpg}
\end{center}
   \caption{Гранд-Каньон, Аризона, США \cite{49}.}
\label{fig:2}
\label{fig:onecol}
\end{figure}

\begin{figure}[t]
\begin{center}
   \includegraphics[width=1\linewidth]{coconino.jpg}
\end{center}
   \caption{Размер слоя песчаника Коконито в западных Соединенных Штатах \cite{21}.}
\label{fig:3}
\label{fig:onecol}
\end{figure}

Оформляемый более внимательный взгляд на Большой Каньон показывает нам, что отложение этих обширных слоев осадочных пород происходило одновременно с существенными тектоническими силами. Чтобы понять это, мы должны внимательно рассмотреть определенные области в каньоне, где слои осадочных пород были сложены и обнажены. Исследователи из "Ответы в Бытие" \cite{42} провели микроскопическое исследование образцов пород из некоторых из этих складок, таких как Изгиб Монумента, и на основании отсутствия признаков, которые должны были бы присутствовать, если бы складки сформировались на протяжении длительных периодов времени под воздействием тепла и давления, сделали вывод, что слои осадка были сложены тектоническими силами, пока они еще были мягкими, т.е. вскоре после их отложения \cite{43}.

\begin{figure*}
\begin{center}
\includegraphics[width=1\textwidth]{Grand_Staircase-big.jpg}
\end{center}
   \caption{Слои осадочных пород, составляющие Большой Каньон (правая сторона изображения), простираются прямо на север до Сидар Брейкс, Юта (левая сторона изображения), где они все изгибаются вверх \cite{50}.}
\label{fig:4}
\end{figure*}

При более масштабном рассмотрении мы обнаруживаем, что слои, составляющие Большой Каньон, были сложены не только внутри каньона. Слои были сложены на восток в Восточной Каибабской Моноклине \cite{46}, но также и на север в Сидар Брейкс, Юта (Рисунок \ref{fig:4}). Это предполагает, что эти слои могли быть сложены вместе после того, как они были положены друг на друга в быстром последовательности. Для справки, горизонтальные слои Большого Каньона имеют толщину примерно 1700 метров. Масштаб геологических процессов, необходимых для отложения осадочных слоев толщиной в милю, огромен.

Фактическое формирование Большого Каньона является еще одной темой спора в современной геологии. Униформитарная геология предполагает, что Большой Каньон был вырезан рекой Колорадо в течение миллионов лет \cite{47}. Однако исследовательская группа "Ответы в Бытие" считает, что Большой Каньон, вероятно, сформировался за считанные недели из-за эрозии водораздела от древнего озера, прорвавшего свои границы, что привело к удалению массивных объемов осадков при вырезании каньона. Существуют доказательства высокого уровня озера к востоку от Большого Каньона в осадочных отложениях озера и морских ископаемых. Сравнение Большого Каньона с другими примерами эрозии водоразделов, такими как каньон Афтон и гора Святой Елены, выявляет аналогичную топографию и показывает, что большие каньоны могут быть созданы быстро при значительном потоке воды \cite{48}.

Учитывая масштаб геологических процессов, необходимых для отложения осадков на таких обширных участках земли, взаимосвязанность массивных тектонических сил, происходивших вскоре после того, как слои осадков были отложены, и крошечные размеры реки Колорадо по сравнению с огромным масштабом Большого Каньона, это кажется маловероятным, что его формирование было постепенным.

\section{Подземный город Деринкую}

Помимо пирамид, отличным примером древнего инженерного дела является подземный город Деринкую (Рисунок \ref{fig:5}), расположенный в Каппадокии, Турция. Это крупнейшее из более чем 200 подземных убежищ в регионе \cite{54}. Предполагается, что этот подземный город мог вместить до 20 000 человек и простирается на 18 этажей, достигая глубины 85 метров. Хотя его возраст точно не известен, предполагается, что ему по крайней мере 2800 лет. Город был вырезан из мягкой вулканической породы \cite{52, 53}.

\begin{figure}[b]
\begin{center}
   \includegraphics[width=1\linewidth]{derinkuyu.jpeg}
\end{center}
   \caption{Диаграмма подземного города Деринкую \cite{56}.}
\label{fig:5}
\label{fig:onecol}
\end{figure}
Причина, по которой Деринкую интересен, заключается в том, что не ясно, почему какое-либо сообщество решило построить целый город под землей. Чтобы создать жилое пространство под землей, каждую полость нужно вырезать из камня. Грубые формы и текстуры подземных туннелей ясно показывают, что они были вырезаны вручную, а не с помощью электроинструментов, что было бы на порядки сложнее, чем строительство убежищ над землей. На самом деле, не очевидно, почему человек захотел бы постоянно жить под землей в течение своей земной жизни, когда сельское хозяйство, солнечный свет, природа и исследования доступны только над землей. Традиционная "история" утверждает, что Деринкую был создан христианами, которым нужно было уединенное место для практики своей религии \cite{53}. Но здравый смысл подсказывает, что наиболее прямолинейный способ справиться с врагами — это "драться или бежать", а не "вырезать подземный город из камня".

Масштабы, глубина и продуманность дизайна подземного города ясно показывают, что он не был задуман как временная военная оборонительная структура для лучшей борьбы с захватчиками в период напряженности, а скорее как долговременное укрытие для защиты от смертельных сил на поверхности. Деринкую был оснащен не только основными спальнями, кухнями и ванными комнатами, но также конюшнями для животных, резервуарами для воды, хранилищами продуктов, прессами для вина и масла, школами, часовнями, усыпальницами и массивными вентиляционными шахтами (Рисунок \ref{fig:6}). Зачем военному убежищу понадобился бы винный пресс и необходимость быть вырытым на 85 метров в глубину с такой сложностью?

Наиболее правдоподобное объяснение создания Деринкую состоит в острой необходимости подготовки долгосрочного, самодостаточного укрытия для защиты от катастрофических геофизических сил на поверхности Земли.

\begin{figure}[t]
\begin{center}
   \includegraphics[width=1\linewidth]{derinkuyu-air.jpg}
\end{center}
   \caption{Глубокая вентиляционная шахта в Деринкую \cite{53}.}
\label{fig:6}
\label{fig:onecol}
\end{figure}

\section{Аккумуляции биомассы}

Смеси биомассы из различных животных и растений, часто обнаруженные в окаменелом состоянии в слоях осадочных пород, — еще одна загадочная аномалия. В "Reliquoæ Diluvianæ" преподобный Вильям Бакленд описывает находки многочисленных видов фауны, которые не имели объяснимого основания находиться вместе, разбросаны по Британии и Европе, погребены в слоях осадочного 'дилувия' \cite{58}. Такие смеси останков животных также были найдены в пещере Сконгхеллен на острове Валдрой, Норвегия. В этой пещере было найдено более 7,000 костей млекопитающих, птиц и рыб, смешанных в разных осадочных слоях \cite{59}. Еще один пример — Сан Киро, "Пещера гигантов", в Италии. В этой пещере было найдено несколько тонн костей млекопитающих, преимущественно гиппопотамов, в столь свежем состоянии, что их нарезали на украшения и отправили на изготовление сажи для ламп. Сообщается, что кости разных животных были перемешаны, сломаны, разбиты и разбросаны на фрагменты \cite{60,61}. В Древнем Мендесе, Египет, была найдена смесь различных видов костей животных, смешанных с стекловатой (стеклянной) глиной \cite{57}. Такие находки могут показаться загадочными, но их легко объяснить массовыми затоплениями, которые откладывали смеси мертвых животных в осадочных слоях, осаживали животных в пещерах или закапывали их вживую, а в случае стекловатой биомассы в Египте — массовыми электрическими разрядами после перемещения ядра и мантии. На Рисунке \ref{fig:7} изображено типичное обнажение алaскской биомассы 'грязи' \cite{56}.

\begin{figure}[t]
\begin{center}
   \includegraphics[width=1\linewidth]{muck-crop.jpeg}
\end{center}
   \caption{Аляскская 'грязь', составленная из хаотически разбросанных фрагментов деревьев, растений и животных во льду и глинистом иле \cite{146}.}
\label{fig:7}
\label{fig:onecol}
\end{figure}

\section{Древние бункеры}

Наши предки оставили после себя множество высоко инженерных древних сооружений, в которых были найдены человеческие останки. Обычно их интерпретируют как сложные гробницы, но более пристальный взгляд предполагает, что эти сооружения на самом деле могли быть древними бункерами.

\begin{figure}[b]
\begin{center}
   \includegraphics[width=1\linewidth]{ww19.jpg}
\end{center}
   \caption{Ньюгрейндж, Ирландия - посмотрите на посетителей у входа для сравнения масштаба.}
\label{fig:8}
\label{fig:onecol}
\end{figure}

Отличным примером является Ньюгрейндж (Рисунок \ref{fig:8}), главный памятник комплекса Бру-на-Бойн, коллекции древних сооружений, включая так называемые коридорные гробницы. Эти гробницы состоят из одной или нескольких погребальных камер, покрытых землей или камнем, и имеют узкий проход, сделанный из больших камней \cite{70}. Это пример сложной инженерии защищенной структуры, построенной в течение нескольких поколений, предположительно для захоронения нескольких людей, которые даже не были живы, когда началось строительство гробницы. Когда она была заново открыта местным землевладельцем в 1699 году, она была погребена в земле.

Беглый взгляд на структуру раскрывает огромные усилия, затраченные на ее строительство - Ньюгрейндж состоит из примерно 200,000 тонн материала. Внутри нее, \textit{«...коридорная камера, доступ к которой осуществляется через вход на юго-восточной стороне памятника. Коридор простирается на 19 метров (60 футов), или примерно на треть от центра структуры. В конце прохода находятся три небольшие камеры, выходящие из большей центральной камеры с высокой сводчатой крышей... Стены этого прохода состоят из больших каменных плит, называемых ортостатами, двадцать два из которых находятся на западной стороне, и двадцать один – на восточной. Их средняя высота составляет 1,5 метра»} \cite{70}. Также имеются тонкие инженерные детали водонепроницаемости. Например, в крыше: \textit{«Промежутки крыши были заделаны смесью сожженной почвы и морского песка для обеспечения их водонепроницаемости, и из этой смеси были получены две радиоуглеродные даты, сосредоточенные около 2500 года до н.э., для структуры гробницы»} \cite{71}. Кроме того, возможно, для аналогичных целей был использован подъем на внутреннюю камеру: \textit{«Так как пол коридора и камеры гробницы следует за подъемом на холме, на котором построен памятник, существует разница почти в 2 метра в уровне пола между входом и внутренней частью камеры»} \cite{71}.

\begin{figure}[b]
\begin{center}
   \includegraphics[width=1\linewidth]{dolmen.jpg}
\end{center}
   \caption{Дольмен де Сото, Испания \cite{53}.}
\label{fig:9}
\label{fig:onecol}
\end{figure}

Отсутствие человеческих останков внутри также является любопытным моментом. Раскопки выявили сожженные и не сожженные фрагменты костей, представляющие несколько человек, разбросанных по проходу. Строительство Ньюгрейнджа оценивается как занявшее как минимум несколько поколений, исходя из радиоуглеродных датировок материалов внутри. Почему древняя община потратила бы столько усилий, чтобы построить огромную, высокоинженерную гробницу, чтобы только разбросать фрагменты костей нескольких покойников в ее проходе? Намного более правдоподобно, что эти древние и тщательно водонепроницаемые мегалитические структуры были построены как убежища для людей, чтобы защитить их во время периодических катаклизмов Земли.

В Уэльве, на юге Испании, аналогичный пример представляет собой Дольмен де Сото (Рисунок \ref{fig:9}), один из около 200 подобных мест в этом районе \cite{72,32}. Это обтекаемая, высокая инженерная структура, построенная с использованием мегалитических камней и имеющая диаметр 75 метров. Сообщается, что при раскопках было найдено только восемь тел, все захороненные в эмбриональной позе.

\section{Замечательные Упоминания Аномалий}

В этом разделе я кратко отмечу несколько более замечательных аномалий, все из которых хорошо объясняются катастрофой, подобной ECDO.

\subsection{Биологические Аномалии}
\begin{figure}[t]
\begin{center}
   \includegraphics[width=1\linewidth]{bottleneck.jpg}
\end{center}
   \caption{Генетический узкий момент, представляющий выбраковку 95\% самцов около 6000 лет назад \cite{62}.}
\label{fig:10}
\label{fig:onecol}
\end{figure}

Некоторые заметные биологические аномалии включают генетические узкие места и ископаемые останки китов в глубине суши. Цзэн и др. (2018) моделировали 125 последовательностей Y-хромосомы современных людей и, на основе сходств и мутаций в ДНК, выявили узкое место, вызвавшее сокращение мужской популяции на 95\% около 5000-7000 лет назад (Рисунок \ref{fig:10}) \cite{62}. Ископаемые останки китов были найдены на высотах в сотни метров над уровнем моря в Швеции, Мичигане, Вермонте, Канаде, Чили и Египте \cite{63,64,65,66}. Эти киты были найдены в разных состояниях: идеально сохранившимися в болотах, лежащими над ледниковыми отложениями или погребенными в осадке. Число экземпляров на этих местах варьируется от нескольких до более сотни. Киты - глубоководные существа, редко подплывают к берегам. Как же эти киты оказались на таких высоких возвышенностях, часто на большом расстоянии от берега?

На Земле в прошлом произошло множество массовых вымираний, наиболее тщательно изученные из которых называются "Большой пятёркой" фанерозойских событий: позднеордовикское (LOME), позднедевонское (LDME), конец-пермское (EPME), конец-триасовое (ETME) и конец-меловое (ECME) вымирания \cite{88,89}. Любопытно, что несколько из этих вымираний были классифицированы как происходящие в тех же исторических периодах, что и многие из слоёв Гранд-Каньона, а именно пермские и девонские слои.

\subsection{Физические аномалии}

\begin{figure}[b]
\begin{center}
   \includegraphics[width=1\linewidth]{columbia.jpg}
\end{center}
   \caption{Огромные текущие ряби в ледниковом озере Колумбия, штат Вашингтон \cite{80}.}
\label{fig:11}
\label{fig:onecol}
\end{figure}

Существует множество ландшафтов, помимо Гранд-Каньона, которые, вероятно, сформировались под воздействием катастрофических сил. Доказательства массовых континентальных водных потоков можно найти в виде гигантских текущих ряби по всему миру. Такой пример — канализированные утра-земли в Тихоокеанском Северо-Западе. Здесь мы видим не только осадочные пейзажи и хаотические валуны, но и более сотни последовательностей крупных ряби, образовавшихся вследствие мега-текущих потоков \cite{78,79}. Это более крупные версии ряби, формирующихся в песчаных руслах ручьёв. Они встречаются по всему миру во Франции, Аргентине, России и Северной Америке \cite{81}. На рисунке \ref{fig:11} изображены некоторые из этих ряби в штате Вашингтон в США \cite{80}.

\begin{figure}[b]
\begin{center}
% \fbox{\rule{0pt}{2in} \rule{0.9\linewidth}{0pt}}
   \includegraphics[width=1\linewidth]{zhangjiajie.jpg}
\end{center}
   \caption{Массивные каменные столбы в Национальном лесу Чжанцзяцзе, южный Китай.}
\label{fig:12}
\label{fig:onecol}
\end{figure}

\begin{figure}[t]
\begin{center}
   \includegraphics[width=1\linewidth]{hoy.jpg}
\end{center}
   \caption{Морской столб Старик Хоя, Шотландия \cite{83}.}
\label{fig:13}
\label{fig:onecol}
\end{figure}

Структуры эрозии внутри суши также хорошо объясняются переворотом Земли, подобным ECDO. Южный Китай является прекрасным примером массивных карстовых ландшафтов, образованных в результате водной эрозии \cite{82}. Эти ландшафты включают башенный карст, остроконечный карст, конусный карст, природные мосты, ущелья, большие пещерные системы и поноры. Одним из самых поразительных из них является Национальный лес Чжанцзяцзе, который содержит массивные кварцевые песчаниковые столбы (Рисунок \ref{fig:12}) \cite{84}. Эти столбы стоят на средней высоте более 1000 метров, и их более 3100. Более 1000 из них возвышаются выше 120 метров, и 45 достигают более 300 метров \cite{85}. Эти столбы напоминают столбы эрозии моря (Рисунок \ref{fig:13}), которые являются береговыми каменными столбами, образованными в результате обрушения окружающего материала из-за океанских волн. Подобные ландшафты эрозии можно найти в каменных конусах Ургупа, Турция, а также в Ciudad Encantada, Испания, которые также находятся на высоте более 1000 метров над уровнем моря. Все эти места имеют рядом какую-то комбинацию солей и морских осадочных ископаемых, что указывает на прошлые морские вторжения \cite{15,86,87}. Конечно, легенды о потопе \cite{3} упоминают о том, что океан поднимался значительно выше 1000 метров, и это подтверждается наличием соленой воды и массивных солончаковых равнин в Андах и Гималаях на высоте нескольких километров над уровнем моря. Например, солончак Уюни в Боливии достигает 3653 метра над уровнем моря \cite{94}.

\subsection{События Бурного Изменения Климата}

Современная научная литература признает существование событий бурного глобального изменения климата в недавней истории Земли. Два заметных примера — это события 4,2 тысячи и 8,2 тысячи лет, которые совпадают с сокращением населения и нарушениями в обществе на больших географических территориях. Эти события сохраняются как аномалии в осадочных и ледяных ядрах, ископаемом коралле, значениях изотопа O18, записях пыльцы и спелеотемах, а также данных о уровне моря. Предполагаемые изменения климата включают резкое снижение глобальных температур, аридизацию, нарушение меридионального опрокидывающего тока Атлантики, и наступления ледников \cite{90,91,92}. В частности, событие 8,2 тысячи лет происходит одновременно с потенциальным драматическим затоплением соленой водой Черного моря около 6400 лет до н.э. \cite{93}.

\subsection{Археологические Аномалии}

Археологические свидетельства некоторых древних городов показывают несколько слоев, включая захоронение и разрушение, создавая записи о прошлых катастрофических событиях. Древний город Иерихон является одним из таких городов, расположенным в современной Палестине. Он содержит несколько разрушительных слоев, с обрушением каменных конструкций и интенсивными пожарами \cite{96,97}. Хронология, записанная в его слоях, датируется примерно с 9000 года до н.э. до 2000 года до н.э. Особого внимания заслуживает его башня, которая, по-видимому, была срезана и захоронена в осадке около 7400 года до н.э. (Рисунок \ref{fig:14}) \cite{95}. Чатал-Хююк \cite{99}, Грамалоте \cite{98} и минойский дворец Кносс на Крите \cite{100,101} являются подобными примерами археологических мест, содержащих несколько слоев, часто содержащих свидетельства разрушения.

\begin{figure}[b]
\begin{center}
   \includegraphics[width=1\linewidth]{jericho.jpg}
\end{center}
   \caption{Археологическая реконструкция захоронения Башни Иерихона около 7400 года до н.э. \cite{95}.}
\label{fig:14}
\label{fig:onecol}
\end{figure}

Еще одним доказательством крупных катаклизмов, нарушающих человеческую цивилизацию, является изображение Нампа, глиняная кукла, найденная под слоем лавы примерно 100 метров в Айдахо \cite{102,103}. Поток лавы, под которым была найдена фигурка, по оценкам, был отложен в конце третичного или в начале четвертичного периода, предположительно 2 миллиона лет назад. Однако поток лавы в этом регионе выглядит относительно свежо. Такие находки не только указывают на катастрофы, уничтожающие цивилизацию, но и ставят под сомнение современные хронологии датирования.

\section{О Современных Методах Датирования}

Существует значительная причина скептически относиться к современным хронологиям, которые приписывают чрезвычайно длительные возрасты в миллионы или даже сотни миллионов лет различным физическим материалам.

Традиционная версия гласит, что так называемые «ископаемые» топлива, такие как уголь, нефть и природный газ, имеют возраст в сотни миллионов лет \cite{104}. Однако фактическое датирование углеродом нефти в Мексиканском заливе показало возраст примерно 13,000 лет для нефти \cite{105}. Углерод-14 имеет настолько короткий период полураспада (5,730 лет), что предполагается, что он полностью распадается через несколько сотен тысяч лет. Тем не менее, он был найден в угле и ископаемых, которые предположительно в тысячу раз старше \cite{106}. На самом деле, искусственный уголь был произведен в лаборатории в контролируемых условиях, в основном при высокой температуре, за всего 2-8 месяцев \cite{107}.

Методы радиоизотопного датирования, кроме датирования углеродом, также могут быть неточны. Исследовательская группа "Answers in Genesis" обнаружила несоответствия в датах, полученных с помощью таких методов, которые ставят их достоверность под сомнение \cite{108}. Мягкие ткани, содержащие кровяные клетки, сосуды и коллаген, даже были обнаружены в останках динозавров, предположительно возрастом сотни миллионов лет \cite{109,110}. Основываясь на том, что нам известно, возможно, что общепринятые возрасты геологической временной шкалы Земли и физические материалы, такие как камни и ископаемые топлива, могут быть неверными на многие порядки величин.

\section{Вывод}

В этой статье я рассмотрел наиболее убедительные аномалии, которые свидетельствуют о катастрофическом происхождении и лучше всего объясняются переворотом Земли по модели ECDO. Хотя этот набор разнообразен, он неполный - больше аномалий было собрано и доступно в моем исследовательском репозитории на GitHub \cite{2}.

\section{Благодарности}

Спасибо Ethical Skeptic, оригинальному автору тезиса ECDO, за завершение его проницательного, новаторского тезиса и его распространение по всему миру. Его триады тезиса \cite{1} остаются авторитетной работой по теории Развязывания Экзотермического Ядра-Мантии Джанибековской Осцилляции (ECDO) и содержат намного больше информации по данной теме, чем я кратко изложил здесь.

И, конечно, спасибо гигантам, на плечах которых мы стоим; тем, кто провел все исследования и расследования, которые сделали эту работу возможной и трудились, чтобы принести свет человечеству.

{\small
\bibliographystyle{ieee}
\bibliography{egbib}
}

\end{document}

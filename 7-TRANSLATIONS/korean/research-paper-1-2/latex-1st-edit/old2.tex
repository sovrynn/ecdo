\documentclass[10pt,twocolumn,letterpaper]{article}

% My own stuff
\usepackage{booktabs}
% \usepackage{caption}
% \captionsetup[table]{skip=8pt}   % Only affects tables
\usepackage{stfloats}  % Add this to the preamble
\usepackage{kotex}

\usepackage{cvpr}
\usepackage{times}
\usepackage{epsfig}
\usepackage{graphicx}
\usepackage{amsmath}
\usepackage{amssymb}

% Include other packages here, before hyperref.

% If you comment hyperref and then uncomment it, you should delete
% egpaper.aux before re-running latex.  (Or just hit 'q' on the first latex
% run, let it finish, and you should be clear).
\usepackage[breaklinks=true,bookmarks=false]{hyperref}

\cvprfinalcopy % *** Uncomment this line for the final submission

\def\cvprPaperID{****} % *** Enter the CVPR Paper ID here
\def\httilde{\mbox{\tt\raisebox{-.5ex}{\symbol{126}}}}

% Pages are numbered in submission mode, and unnumbered in camera-ready
%\ifcvprfinal\pagestyle{empty}\fi
\setcounter{page}{1}
\begin{document}

%%%%%%%%% TITLE
\title{ECDO 데이터 기반 프라이머 2/2: ECDO "지구 뒤집기"로 가장 잘 설명되는 과학적 및 역사적 이상현상에 대한 조사}

\author{준호\\
웹사이트: \href{https://sovrynn.github.io}{sovrynn.github.io}\\
ECDO 연구 저장소: \href{https://github.com/sovrynn/ecdo}{github.com/sovrynn/ecdo}\\
{\tt\small junhobtc@proton.me}
}

\maketitle
%\thispagestyle{empty}

%%%%%%%%% ABSTRACT
\begin{abstract}
2024년 5월에 "The Ethical Skeptic"이라는 이름의 가명 온라인 저자는 Exothermic Core-Mantle Decoupling Dzhanibekov Oscillation (ECDO)라는 혁신적인 이론을 발표했습니다 \cite{0} \cite{1}. 이 이론은 지구가 이전에 회전 축의 갑작스럽고 파멸적인 변화를 겪었다고 주장하며, 이는 회전 관성으로 인해 대양이 대륙 위로 넘치게 하여 거대한 범세계 홍수를 일으켰으며, 또 다른 그러한 변화가 임박했을 수도 있다는 데이터를 제시하는 설명적 인과 지구물리학적 과정을 제안합니다. 이러한 재앙적 홍수 및 종말 예측은 새로운 것이 아니지만, ECDO 이론은 과학적이고 현대적이며, 다학문적이고 데이터 기반 접근 방식 덕분에 독특하게 설득력이 있습니다.

이 연구 논문은 ECDO 이론에 대한 6개월 간의 독립적 연구 \cite{2,20}의 두 번째 부분 요약으로, 특히 ECDO "지구 뒤집기"로 가장 잘 설명되는 과학적 및 역사적 이상현상에 초점을 맞추고 있습니다.

\end{abstract}

%%%%%%%%% BODY TEXT

\section{서론}

현대 균일주의 지질학과 역사는 그랜드 캐년과 같은 주요 지질학적 풍경이 수백만 년에 걸쳐 형성되었다고 주장합니다 \cite{143}; 데스밸리(캘리포니아)에 염이 존재하는 이유는 수억 년 전 그것이 대양 아래에 있었기 때문이라고 주장합니다 \cite{144}; 150세대 전의 우리의 조상들이 그들의 일생을 거대한 무덤을 짓는 데 보냈다고 주장합니다 \cite{29,70}; 그리고 소위 "화석 연료"가 수억 년 전의 것이라고 주장합니다 \cite{104}. 아마도 가장 흥미로운 것은 인간이 30만 살이라고 믿어지지만 \cite{145}, 기록된 역사와 문명은 약 5,000년, 즉 150세대 정도 전으로만 거슬러 올라간다는 것입니다.

우리가 볼 수 있듯이, 이러한 이상현상은 대홍수적 지질학적 힘에 의해 가장 잘 설명됩니다.

\section{진흙에 묻힌 냉동 매머드}

\begin{figure}[t]
\begin{center}
   \includegraphics[width=1\linewidth]{jarkov-mammoth.jpg}
\end{center}
   \caption{자르코프 매머드, 20,000년 전의 완벽하게 보존된 시베리아 매머드가 얼음 진흙 속에서 발견됐다 \cite{51}.}
\label{fig:1}
\label{fig:onecol}
\end{figure}

이러한 이상현상 중 하나는 주로 북극 지역에서 발견되는 진흙에 묻혀 완벽하게 보존된 냉동 매머드들입니다 (그림 \ref{fig:1}). 실트질 자갈에 묻힌 채 시베리아에서 발견된 베레조프카 매머드는 죽은 지 수천 년이 지난 후에도 고기가 여전히 먹을 수 있을 정도로 완벽하게 보존되었습니다. 게다가 식물성 먹이를 입과 위에서 발견할 수 있었으며, 과학자들은 매머드가 죽기 직전에 꽃이 피어 있는 식물들을 뜯어먹고 있었음을 고려할 때 이것이 어떻게 그렇게 빠르게 얼 수 있었는지 의문을 품었습니다 \cite{17}. 보고에 따르면, \textit{"1901년 베레조프카 강 근처에서 매머드의 전체 사체가 발견되면서 대센세이션을 일으켰습니다. 이 동물은 한여름에 차가워져 죽은 것처럼 보였기 때문입니다. 위의 내용물은 잘 보존되어 있었고, 버터컵과 피어 있는 야생 콩들을 포함하고 있었습니다. 이는 대개 7월 말이나 8월 초쯤에 삼켜졌어야 했음을 의미합니다. 생물체는 갑작스럽게 죽어, 그 입에는 풀과 꽃을 가득 담고 있었고, 그것은 엄청난 힘에 의해 붙잡혀 목초지에서 몇 마일 떨어진 곳에 던져진 것처럼 보였습니다. 골반과 한쪽 다리는 부러졌으며, 거대한 동물은 무릎을 꿇게 되었고, 연도상으로 가장 더운 시기에 얼어 죽게 되었습니다"} \cite{18}. 또한, \textit{"[러시아 과학자들은] 심지어 동물의 위 내부 층이 완벽하게 보존된 섬유 구조를 가지고 있었다고 기록했습니다. 이는 그의 체온이 자연에서 거대한 과정에 의해 제거되었음을 나타냅니다. 샌더슨은 이 하나의 점에 주목하면서 문제를 미국의 냉동식품연구소로 가져갔습니다: 매머드와 같이 큰 사체에서조차 그의 몸의 가장 깊은 부분까지 물의 결정을 형성하기에 충분한 시간이 없도록 체온을 제거하려면 무엇이 필요한가?... 몇 주 후 연구소는 샌더슨에게 이렇게 대답했습니다: 그것은 절대 불가능합니다. 우리의 모든 과학적 및 엔지니어링 지식으로도, 매머드만큼 큰 사체에서 수분 결정이 육의 섬유 구조를 파괴하지 않게 냉동하는 방법은 전혀 알려져 있지 않습니다. 게다가, 과학적 및 엔지니어링 기술을 소진한 후, 그들은 자연에서도 그 업적을 이룰 수 있는 기로 알려지지 않음을 결론지었습니다"} \cite{19}.

\section{그랜드 캐년}

북미 남서부의 그레이트 베이슨의 일부인 그랜드 캐년은 또 다른 자연적 현상으로서 대재앙적인 기원을 가리키고 있습니다 (그림 \ref{fig:2}). 우선, 그랜드 캐년을 구성하는 퇴적 사암 및 석회암 층은 최대 240만 평방킬로미터에 걸쳐 펼쳐져 있습니다 \cite{21}. 그림 \ref{fig:3}은 코코니노 사암 층이 미국 서부 전역에 걸쳐 있는 모습을 보여줍니다. 이러한 거대한 수평의 균일한 물질의 층은 오직 한 번에 모두 놓였을 때만 형성될 수 있습니다.

\begin{figure}[b]
\begin{center}
   \includegraphics[width=1\linewidth]{grand-canyon.jpg}
\end{center}
   \caption{미국 애리조나 주의 그랜드 캐년 \cite{49}.}
\label{fig:2}
\label{fig:onecol}
\end{figure}

\begin{figure}[t]
\begin{center}
   \includegraphics[width=1\linewidth]{coconino.jpg}
\end{center}
   \caption{미국 서부에 있는 코코니노 사암 층의 크기 \cite{21}.}
\label{fig:3}
\label{fig:onecol}
\end{figure}

그랜드 캐년을 자세히 보면, 이 광대한 퇴적층들의 축적이 또한 상당한 지질 구조력과 동시에 발생했음을 알 수 있습니다. 이것을 이해하려면, 우리는 캐년의 특정 지역에서 퇴적층이 접히고 노출된 모습을 가까이서 살펴봐야 합니다. 창조과학회(Answers in Genesis)의 연구자들은 \cite{42} 이러한 접힘 형태의 암석 샘플을 현미경으로 조사하였으며, 접힘이 열과 압력 하에서 오랜 세월에 걸쳐 형성되었다면 존재했어야 할 특징이 결핍된 것을 기반으로 하여, 퇴적층이 아직 부드럽고, 즉 퇴적 후 곧 지질 구조력에 의해 접혔다고 결론 내렸습니다 \cite{43}.

\begin{figure*}
\begin{center}
\includegraphics[width=1\textwidth]{Grand_Staircase-big.jpg}
\end{center}
   \caption{그랜드 캐년(그림의 오른쪽)을 구성하는 퇴적층은 유타주의 세다 브레이크(그림의 왼쪽)까지 북쪽으로 직접 확장되며, 그곳에서 모두 위로 휘어집니다 \cite{50}.}
\label{fig:4}
\end{figure*}

확대해보면, 그랜드 캐니언을 구성하는 지층이 단순히 캐니언 내부에서만 접힌 것이 아님을 알 수 있습니다. 지층은 이스트 카이밥 단층애에서 동쪽으로 접혀 있고 \cite{46}, 또한 유타의 시더 브레이크에서는 북쪽으로도 접혀 있습니다 (그림 \ref{fig:4}). 이는 이러한 지층이 급속도로 연속적으로 쌓인 후 함께 접혔을 가능성을 시사합니다. 참고로, 그랜드 캐니언의 수평 지층은 두께가 약 1700미터에 이릅니다. 1마일 두께의 퇴적층을 형성시키기 위한 지질 과정의 규모는 엄청납니다.

그랜드 캐니언의 실제 형성은 현대 지질학에서 또 다른 논쟁거리가 됩니다. 균일주의 지질학은 그랜드 캐니언이 수백만 년 동안 콜로라도 강에 의해 깎였다고 제안합니다 \cite{47}. 그러나 창세기에 대한 연구팀은 그랜드 캐니언이 고대 호수의 경계를 깨고 넘치는 유수 침식으로 인해 수주일 내에 형성되었을 가능성이 높다고 믿습니다. 이는 캐니언을 깎아내면서 대량의 퇴적물을 운반했기 때문입니다. 그랜드 캐니언 동쪽의 고지대 호수의 퇴적물과 해양 화석에서 증거를 찾을 수 있습니다. 애프턴 캐니언과 세인트 헬렌스 산과 같은 다른 대규모 유수 침식 사례와 그랜드 캐니언을 비교하면 비슷한 지형이 드러나며, 대규모 유수로 인해 큰 캐니언이 빠르게 형성될 수 있음을 보여줍니다 \cite{48}.

이처럼 광대한 육지 위에 퇴적물을 쌓기 위한 지질 과정의 규모, 퇴적층이 쌓인 직후 발생한 대규모 판구조력의 동시 발생, 그리고 거대한 그랜드 캐니언에 비해 아주 작은 콜로라도 강의 크기를 감안할 때, 그 형성 과정에는 점진적인 것이 없었을 가능성이 여부라고 할 수 있습니다.

\section{데린쿠유 지하도시}

\begin{figure}[b]
\begin{center}
% \fbox{\rule{0pt}{2in} \rule{0.9\linewidth}{0pt}}
   \includegraphics[width=1\linewidth]{derinkuyu.jpeg}
\end{center}
   \caption{데린쿠유 지하 도시의 다이어그램 \cite{56}.}
\label{fig:5}
\label{fig:onecol}
\end{figure}

피라미드 외에도 고대 공학의 훌륭한 예는 터키 카파도키아의 데린쿠유 지하도시입니다 (그림 \ref{fig:5}). 이곳은 지역에 있는 200개 이상의 지하 보호 시설 중 가장 큰 곳입니다 \cite{54}. 이 지하도시는 최대 20,000명의 사람들이 거주했으며 18층으로 구성되어 있으며, 깊이는 85미터에 이릅니다. 그 나이는 확실하지 않지만 최소 2800년 이상 되었을 것으로 추정됩니다. 이 도시는 부드러운 화산암을 깎아서 만들어졌습니다 \cite{52, 53}.

데린쿠유가 흥미로운 이유는 왜 한 공동체가 전체 도시를 지하에 건설하려 했는지 명확하지 않기 때문입니다. 지하에 생활공간을 만들기 위해서는 모든 공간이 암석에서 깎여 나와야 합니다. 지하 터널의 거친 형태와 질감은 이런 것이 전동 공구 대신 수작업으로 깎였다는 것을 분명히 보여줍니다. 이는 지상에 대피소를 세우는 것보다 훨씬 더 어려운 일이었을 것입니다. 사실, 지상에서 농업, 햇빛, 자연, 탐구가 가능한데 왜 인간이 지하에서 평생을 살고 싶어할까 하는 의문이 듭니다. 전통적인 "역사"는 데린쿠유가 종교를 실천할 은둔지가 필요했던 기독교인들에 의해 만들어졌다고 제안합니다 \cite{53}. 그러나 상식적으로 생각해보면 적과 맞서 싸우거나 도망가는 것이 가장 직접적인 해결책이지, 바위를 깎아 지하 도시를 만드는 것이 아니라고 결론지을 수 있습니다.

지하 도시의 규모, 깊이, 그리고 설계의 세심함은 이곳이 스트레스 상황에서 침략자와 더 잘 싸우기 위한 일시적 군사 방어 구조로 설계된 것이 아니라, 지상에서의 치명적인 힘에 대비하기 위한 장기적인 대피소로 설계되었음을 분명히 합니다. 데린쿠유에는 기본적인 침실, 주방, 욕실뿐만 아니라 동물을 위한 안정장소, 물탱크, 식량 저장고, 와인 및 오일 압착기, 학교, 예배당, 무덤, 거대한 환기통로가 마련되어 있습니다 (그림 \ref{fig:6}). 군사 대피소가 와인 압착기가 필요한 이유는 무엇이며, 왜 이렇게 복잡하게 85미터 깊이로 파야 했을까요?

데린쿠유를 만든 가장 그럴듯한 설명은 지표면의 재앙적인 지구 물리학적 힘으로부터 보호하기 위한 장기적이고 지속 가능한 대피소를 준비할 긴급한 필요가 있었을 것이라는 점입니다.

\begin{figure}[t]
\begin{center}
   \includegraphics[width=1\linewidth]{derinkuyu-air.jpg}
\end{center}
   \caption{데린쿠유의 깊은 환기구 \cite{53}.}
\label{fig:6}
\label{fig:onecol}
\end{figure}

\section{바이오매스 집합}

종종 퇴적층에서 화석화되어 발견되는 다양한 동물과 식물의 바이오매스 혼합물은 또 다른 수수께끼의 불가사의입니다. "Reliquoæ Diluvianæ"에서, 윌리엄 버클랜드 목사는 영국과 유럽 전역에 무작위로 흩어져 '홍수 퇴적물'층에 묻혀 있는, 함께 발견될 아무 이유가 없는 수많은 종의 동물상을 발견한 내용을 설명합니다 \cite{58}. 이런 동물 유해의 혼합물은 노르웨이 발드로이 섬의 스쾅힐렌 동굴에서도 발견되었습니다. 이 동굴에서는 7,000개 이상의 포유류, 조류, 어류의 뼈가 여러 퇴적층에 혼합되어 발견되었습니다 \cite{59}. 또 다른 예로는 이탈리아의 "거인의 동굴" 산 치로가 있습니다. 이 동굴에서는 수 톤의 포유류 뼈, 대부분 하마,가 너무 신선한 상태로 발견되어 장신구로 잘려나가 검정 램프 제조에 사용되었습니다. 서로 다른 동물의 뼈가 함께 섞이고, 부서지고, 산산조각 나서 흩어져 있었다고 전해집니다 \cite{60,61}. 이집트의 고대 멘데스에서는 유리화된(유리 모양) 점토와 혼합된 여러 종의 동물 뼈가 발견되었습니다 \cite{57}. 이러한 발견은 수수께끼로 보일 수 있지만, 홍수에 의해 퇴적층에 죽은 동물들을 혼합하여 퇴적시키거나, 동물들을 동굴로 쓸어 넣거나 묻은 후, 이집트의 유리화된 바이오매스의 경우처럼 대홍수 후 대규모 전기 방전으로 쉽게 설명될 수 있습니다. 그림 \ref{fig:7}은 알래스카의 '머크'의 전형적인 노출을 보여줍니다 \cite{56}.

\begin{figure}[t]
\begin{center}
   \includegraphics[width=1\linewidth]{muck-crop.jpeg}
\end{center}
   \caption{알래스카의 '머크', 나무, 식물 및 동물 조각이 혼란스럽게 분포된 냉동 실트와 얼음으로 구성되어 있음 \cite{146}.}
\label{fig:7}
\label{fig:onecol}
\end{figure}

\section{고대 벙커}

우리의 조상은 인간 유해가 발견된 많은 고도 엔지니어링된 고대 구조물을 남겼습니다. 이들은 대개 정교한 무덤으로 해석되지만, 자세히 살펴보면 사실 고대 벙커였을 가능성이 있습니다.

\begin{figure}[b]
\begin{center}
   \includegraphics[width=1\linewidth]{ww19.jpg}
\end{center}
   \caption{뉴그레인지, 아일랜드 - 입구에서 보는 방문자들의 크기를 보십시오.}
\label{fig:8}
\label{fig:onecol}
\end{figure}

뛰어난 예 중 하나는 브루 나 보인 단지의 주요 기념물인 뉴그레인지입니다 (그림 \ref{fig:8}). 이 단지는 소위 통로 무덤이라고 불리는 고대 구조물의 모음입니다. 이 무덤들은 하나 이상의 매장실이 토석으로 덮여 있고, 큰 돌로 만든 좁은 출입 통로가 있습니다 \cite{70}. 이는 여러 세대에 걸쳐 지어진 복잡한 보호 구조물의 광범위한 엔지니어링의 예이며, 무덤의 건설이 시작될 때 이미 죽어있던 소수의 사람들을 매장하기 위한 것이라고 여겨집니다. 1699년 지역 토지 소유주에 의해 재발견되었을 때는 흙으로 묻혀 있었습니다.

간단히 구조를 살펴보면 건설에 엄청난 노력이 기울여졌음을 알 수 있다. 뉴그레인지(Newgrange)는 약 200,000톤의 재료로 구성되어 있다. 내부에는 \textit{“...기념물의 남동쪽 측면의 입구를 통해 접근할 수 있는 방이 있는 통로가 있다. 통로는 구조물의 중앙까지 약 19미터 (60피트)로 연장되어 있다. 통로의 끝에는 큰 중앙 방에서 세 개의 작은 방이 있다. 이 통로의 벽은 정교한 건축 기법에 의해 만들어졌으며 방수 기능을 갖추고 있다... 벽은 정면벽이라 불리는 매우 큰 돌판을 맞추어 만들어졌다. 서쪽에 스물두 개, 동쪽에 스물한 개의 돌판이 있으며 평균 높이는 1½미터이다”} \cite{70}. 또한 정교한 방수 공학적 세부사항도 있다. 예를 들어, 지붕에는 \textit{“지붕의 간극은 방수 처리를 위해 소각된 흙과 해사의 혼합물로 메워졌으며, 이 혼합물은 무덤 구조에 대해 기원전 2500년 중심으로 하는 두 개의 방사성 탄소 날짜를 얻었다”} \cite{71}. 또한, 내실로의 상승 구간이 유사한 목적을 위해 구현되었을 수 있다: \textit{“기념물이 세워진 언덕의 경사면에 따라 무덤의 통로와 방의 바닥이 연결되므로, 입구와 방 내부 사이에는 거의 2미터의 바닥 높이 차이가 있다”} \cite{71}.

\begin{figure}[b]
\begin{center}
   \includegraphics[width=1\linewidth]{dolmen.jpg}
\end{center}
   \caption{스페인 돌멘 드 소토(Dolmen de Soto) \cite{53}.}
\label{fig:9}
\label{fig:onecol}
\end{figure}

인간 유해의 부족 또한 호기심을 불러일으킨다. 발굴에서 몇몇 사람들을 대표하는 소수의 소각 및 소각되지 않은 뼛조각이 통로에 흩어져 있는 것이 드러났다. 뉴그레인지를 건설하는 데는 최소 몇 세대가 걸렸을 것으로 추정되며, 이는 내부 재료의 탄소 연대측정에 기초한다. 왜 고대 사회가 그토록 많은 노력을 기울여 거대한, 정교하게 설계된 무덤을 건설하고는 그곳에 소수의 사망자 유해를 흩어놓는 것일까? 오히려 이러한 고대의 정교하게 방수 처리된 거대 구조물은 인류가 지구의 지속적인 재난 동안 자신을 보호하기 위해 지어진 쉼터였다는 것이 더 설득력 있다.

남부 스페인 후엘바(Huelva)에서는 유사한 예시로 약 200여 개의 유적지가 있는 돌멘 드 소토(Dolmen de Soto) (그림 \ref{fig:9})가 있다 \cite{72,32}. 이 구조물은 거석을 사용하여 지어진 유선형의 고도로 설계된 구조물로, 직경이 75미터이다. 발굴당시 태아 자세로 묻힌 8구의 시체만이 발견되었다고 전해진다.

\section{주목할 만한 이상 현상 언급}

이 섹션에서는 ECDO와 같은 대재앙으로 잘 설명되는 몇 가지 주요 이상 현상에 대해 간략히 언급하겠다.

\subsection{생물학적 이상 현상}

\begin{figure}[t]
\begin{center}
   \includegraphics[width=1\linewidth]{bottleneck.jpg}
\end{center}
   \caption{약 6,000년 전 남성의 95\%에 대한 급감 현상을 나타내는 유전적 병목현상 \cite{62}.}
\label{fig:10}
\label{fig:onecol}
\end{figure}

주목할 만한 생물학적 이상 현상으로는 유전적 병목현상과 내륙 고래 화석이 있다. Zeng et al. (2018)은 현대 인류의 125개의 Y 염색체 서열을 모델링하였고, DNA의 유사성과 돌연변이를 기반으로 약 5,000년에서 7,000년 전 남성 인구에서 95\% 인구 감소 병목현상을 확인하였다 (그림 \ref{fig:10}) \cite{62}. 고래 화석은 스웨덴버그, 미시간, 버몬트, 캐나다, 칠레, 이집트 등 해수면 위 수백 미터에 발견되었다 \cite{63,64,65,66}. 이 고래들은 완벽하게 보존된 상태로, 빙하 퇴적물 위의 연못에 놓여 있거나 퇴적물에 묻혀 있는 형태로 발견되었다. 이 지역의 표본 수는 몇 개에서 백 개 이상에 이른다. 고래는 심해 생물로, 해안 근처로 잘 오지 않는다. 이런 고래들이 어떻게 높은 고도에, 특히 육지 깊숙이 침투한 위치에 도달하였을까?

지구 역사상 수많은 대량 멸종이 발생했으며, 가장 철저히 연구된 것은 "빅 파이브" 팬에로조익 사건들이다: 후기 오르도비스기(LOME), 후기 데본기(LDME), 페름기말기(EPME), 트라이아스기말기(ETME), 그리고 백악기말기(ECME) 대멸종 \cite{88,89}. 놀랍게도, 이러한 멸종 사건 중 일부는 그랜드 캐니언의 여러 지층, 특히 페름기와 데본기 지층과 일치하는 역사적 시기에 발생했다고 분류된다.

\subsection{물리적 이상 현상}

\begin{figure}[t]
\begin{center}
   \includegraphics[width=1\linewidth]{columbia.jpg}
\end{center}
   \caption{워싱턴 주 글래이셜 레이크 컬럼비아의 거대한 현재 물살 계단 \cite{80}.}
\label{fig:11}
\label{fig:onecol}
\end{figure}

그랜드 캐니언 이외에도 대재앙적 힘에 의해 형성되었을 가능성이 높은 많은 지형이 있다. 거대한 대륙 간 물 흐름의 증거는 전 세계의 거대한 현재 물살 계단에서 발견될 수 있다. 이러한 예 중 하나가 태평양 북서부의 개천 스캡랜드이다. 여기에는 퇴적층 지형과 불규칙한 큰 바위뿐만 아니라 거대한 물살 흐름에 의해 형성된 100개 이상의 시퀀스가 넘는 큰 모래 언덕도 발견된다 \cite{78,79}. 이들은 하천의 모래 바닥에 형성된 결 계단의 대규모 버전이다. 프랑스, 아르헨티나, 러시아, 북아메리카 등 전 세계에서 찾을 수 있다 \cite{81}. 그림 \ref{fig:11}은 미국 워싱턴 주에 있는 이러한 모래 언덕 중 일부를 묘사한다 \cite{80}.

\begin{figure}[t]
\begin{center}
% \fbox{\rule{0pt}{2in} \rule{0.9\linewidth}{0pt}}
   \includegraphics[width=1\linewidth]{zhangjiajie.jpg}
\end{center}
   \caption{중국 남부 장가계 국립 산림 공원 속 거대한 석기 기둥.}
\label{fig:12}
\label{fig:onecol}
\end{figure}

\begin{figure}[t]
\begin{center}
   \includegraphics[width=1\linewidth]{hoy.jpg}
\end{center}
   \caption{스코틀랜드의 올드 맨 오브 호이(Hoy) 바다 기둥 \cite{83}.}
\label{fig:13}
\label{fig:onecol}
\end{figure}

내륙 침식 구조 또한 ECDO와 같은 지구 뒤집기의 결과로 잘 설명될 수 있다. 남중국은 물 침식에 의해 형성된 거대한 카르스트 지형의 좋은 예이다 \cite{82}. 이러한 지형에는 타워 카르스트, 첨탑 카르스트, 콘 카르스트, 자연 다리, 협곡, 대규모 동굴 시스템 및 싱크홀이 포함된다. 그중에서도 가장 눈에 띄는 것 중 하나는 장가계 국립 산림 공원으로, 이곳에는 거대한 석영 사암 기둥이 있다 (그림 \ref{fig:12}) \cite{84}. 이 기둥들은 평균 고도 1,000미터 이상으로 서 있으며 3,100개가 넘는다. 이 중 1,000개 이상이 120미터 이상 높이 솟아 있고, 45개는 300미터 이상에 달한다 \cite{85}. 이러한 기둥들은 해안 침식 기둥(그림 \ref{fig:13})을 닮았으며, 이는 해양 파도로 인해 주변 물질이 붕괴되어 형성된 해안 바위 기둥이다. 경사면 침식 지형은 터키의 위르괴프 바위뿔이나 스페인의 시우다드 엔칸타다처럼 해발 1,000미터를 넘는 곳에서도 찾을 수 있다. 이 모든 장소들은 가까이에 소금 및 해양 화석을 포함한 조합을 가지고 있으며 이는 과거 해양 침입을 시사한다 \cite{15,86,87}. 물론 홍수 이야기는 \cite{3} 해양이 1,000미터 이상까지 올랐다는 것을 언급했으며, 이는 소금물과 해발 수천 미터에 달하는 안데스 및 히말라야 얼음제해 및 소금 평원이 여러 곳에 존재함으로 확인된다. 볼리비아의 우유니 소금 평야는 예를 들어 해발 3653미터에 이르기도 한다 \cite{94}.

\subsection{급격한 기후 변화 사건}

현대 과학 문헌은 지구의 최근 역사에서 빠른 글로벌 기후 변화 사건의 존재를 인정하고 있다. 두 가지 주목할 만한 예는 4.2 천년 사건과 8.2 천년 사건으로, 둘 다 인구 감소와 광범위한 지리적 지역에서의 사회적 정착지 교란과 일치한다. 이러한 사건들은 퇴적물과 얼음 코어, 화석 산호, O18 동위원소 값, 꽃가루 및 스펠레오템 기록, 해수면 데이터의 이상 현상으로 보존되어 있다. 추론된 기후 변화에는 전 세계 기온의 급격한 하락, 건조화, 대서양 메리디오널 오버터닝 전류의 교란, 그리고 빙하의 전진이 포함된다 \cite{90,91,92}. 특히 8.2 천년 사건은 기원전 6400년경 흑해의 잠재적인 극적인 염수 홍수와 동시에 발생했다 \cite{93}.

\subsection{고고학적 이상 현상}

\begin{figure}[b]
\begin{center}
   \includegraphics[width=1\linewidth]{jericho.jpg}
\end{center}
   \caption{기원전 7400년경 여리고 탑의 매장을 고고학적으로 재구성한 것 \cite{95}.}
\label{fig:14}
\label{fig:onecol}
\end{figure}

일부 고대 도시의 고고학적 증거는 매장 및 파괴를 포함하는 여러 층을 보여주며, 과거의 대격변 사건의 기록을 만든다. 현대 팔레스타인에 위치한 고대 도시 여리고는 그런 도시 중 하나이다. 그것은 돌 구조물의 붕괴와 강한 화재로 인한 여러 파괴 층을 포함한다 \cite{96,97}. 그 층에 기록된 연대기는 기원전 약 9000년부터 기원전 2000년까지 이른다. 특히 주목할 점은 기원전 7400년경 퇴적물에 절단되고 매장된 것으로 보이는 탑이다 (Figure \ref{fig:14}) \cite{95}. 카탈 휘윅 \cite{99}, 그라마로테 \cite{98}, 그리고 크레타의 크노소스 미노아 궁전 \cite{100,101}은 모두 유사한 고고학적 유적지의 예로서 여러 층을 포함하며 종종 파괴의 증거를 포함한다.

인류 문명을 파괴하는 주요 대격변의 또 다른 증거는 아이다호에서 약 100미터의 용암 아래에서 발견된 남파 이미지, 점토 인형이다 \cite{102,103}. 인형이 발견된 용암 흐름은 후기 삼차 또는 초기 제4기의 것으로 추정되며, 약 200만 년 전의 것으로 여겨진다. 그러나 그 지역의 용암 흐름은 비교적 신선한 것으로 보인다. 이러한 발견은 문명을 파괴하는 주요 대격변을 가리킬 뿐만 아니라 현대의 연대 측정 개념을 의문으로 만든다.

\section{현대 연대 측정 방법에 관하여}

필요 이상으로 긴 수백만 년, 혹은 심지어 수억 년의 연대를 다양한 물질에 부여하는 현대 연대기학에 대해 회의적인 이유가 충분히 있다.

전통적인 서사는 석유, 석탄, 천연가스와 같은 소위 "화석 연료"가 수억 년 된 것이라고 주장한다 \cite{104}. 그러나 멕시코 만의 석유에 대한 실제 탄소 연대 측정 결과, 석유의 연대가 약 13,000년으로 판명되었다 \cite{105}. 탄소-14의 반감기는 너무 짧아서 (5,730년) 몇십만 년 후에는 완전히 붕괴되어야 한다. 그러나 이는 석탄과 수천 배로 더 오래된 것으로 여겨지는 화석에서 발견되었다 \cite{106}. 실제로 인공 석탄은 통제된 조건, 주로 높은 열에서 단 2-8개월만에 실험실에서 생산되었다 \cite{107}.

탄소 연대 측정 외의 방사성 동위원소 연대 측정 방법도 정확하지 않을 수 있다. Answers in Genesis 연구 그룹은 그러한 방법으로부터 파생되는 날짜의 불일치를 발견하여 정확성을 의문시하였다 \cite{108}. 심지어 수천만 년 전 것으로 여겨지는 공룡 유해에서 혈액 세포, 혈관, 콜라겐이 포함된 연조직이 발견되었다 \cite{109,110}. 우리가 아는 바에 따르면, 지구의 지질학적 시공간과 암석 및 화석 연료와 같은 물질에 대한 일반적으로 수용되는 연대가 여러 등급으로 잘못된 것일 수 있다.

\section{결론}

이 논문에서는 ECDO 지구 플립에 의해 가장 잘 설명되는 폭력적인 기원을 시사하는 가장 설득력 있는 이상 현상을 다루었다. 다양하지만, 제시된 컬렉션은 불완전하다 - 더 많은 이상 현상이 수집되어 있으며 내 연구 GitHub 저장소에서 공개적으로 제공된다 \cite{2}.

\section{감사의 말씀}

ECDO 논문을 완성하고 이를 세상에 공유한 최초의 ECDO 논문의 저자인 Ethical Skeptic에게 감사드린다. 그의 3부작 논문 \cite{1}은 긴 단편에서 내가 요약한 것보다 훨씬 더 많은 정보를 포함한, 외엄핵-맨틀 분리 잔비코브 진동 (ECDO) 이론의 권위 있는 작품으로 남아 있다.

그리고 당연히 우리가 그의 어깨 위에 서 있는 거대들; 이 작업을 가능하게 만든 모든 연구와 조사를 수행하고 인류에 빛을 가져오려고 노력한 그들에게 감사드린다.

{\small
\bibliographystyle{ieee}
\bibliography{egbib}
}

\end{document}

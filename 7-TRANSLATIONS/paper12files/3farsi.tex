\documentclass[10pt,twocolumn,letterpaper]{article}

% My own stuff
\usepackage{booktabs}
% \usepackage{caption}
% \captionsetup[table]{skip=8pt}   % Only affects tables
\usepackage{stfloats}  % Add this to the preamble

\usepackage[utf8]{inputenc}
\usepackage[LAE]{fontenc}
\usepackage[farsi]{babel}
\TOCLanguage{farsi}

\usepackage{amsmath}




\usepackage{cvpr}
\usepackage{times}
\usepackage{epsfig}
\usepackage{graphicx}
\usepackage{amssymb}

% Include other packages here, before hyperref.

% If you comment hyperref and then uncomment it, you should delete
% egpaper.aux before re-running latex.  (Or just hit 'q' on the first latex
% run, let it finish, and you should be clear).
\usepackage[breaklinks=true,bookmarks=false]{hyperref}

\cvprfinalcopy % *** Uncomment this line for the final submission

\def\cvprPaperID{****} % *** Enter the CVPR Paper ID here
\def\httilde{\mbox{\tt\raisebox{-.5ex}{\symbol{126}}}}

% Pages are numbered in submission mode, and unnumbered in camera-ready
%\ifcvprfinal\pagestyle{empty}\fi
\setcounter{page}{1}
\begin{document}

%%%%%%%%% TITLE
\title{ECDO داده‌محور قسمت ۲/۲: بررسی ناهنجاری‌های علمی و تاریخی که بهترین توضیح برای "چرخش زمین" ECDO است}

\author{جونهو\\
وبسایت: \href{https://sovrynn.github.io}{sovrynn.github.io}\\
مخزن تحقیقاتی ECDO: \href{https://github.com/sovrynn/ecdo}{github.com/sovrynn/ecdo}\\
{\tt\small firstauthor@i1.org}
}

\maketitle
%\thispagestyle{empty}

%%%%%%%%% ABSTRACT
\begin{abstract}
در ماه می ۲۰۲۴، یک نویسنده آنلاین با نام مستعار "شهرآزمای اخلاقی" \cite{0} نظریه‌ای بنیادین بنام جدایش هسته-گوشته اگزوترمیک به همراه نوسان ژانیبکوف (ECDO) منتشر کرد \cite{1}. این نظریه نه تنها پیشنهاد می‌کند که زمین دچار تغییرات ناگهانی و فاجعه‌بار در محور چرخشی خود شده است و باعث ایجاد سیل عظیمی در سراسر جهان با رانش اقیانوس‌ها به روی قاره‌ها به دلیل اینرسی چرخشی می‌شود، بلکه فرایند فیزیکی علّی توضیحی را نیز پیشنهاد می‌کند که داده‌هایی ارائه می‌دهد که نشان می‌دهد ممکن است چرخش دیگری نیز قریب‌الوقوع باشد. در حالی که پیش‌بینی‌های سیل و قیامت فاجعه‌بار جدید نیستند، نظریه ECDO به خاطر رویکرد علمی، مدرن، چند‌رشته‌ای و مبتنی بر داده‌های خود به‌ویژه جذاب است.

این مقاله تحقیقی تشکیل‌دهنده بخش دوم یک خلاصه فشرده دو بخشی از ۶ ماه تحقیق مستقل \cite{2,20} در نظریه ECDO است، که تمرکز ویژه‌ای بر ناهنجاری‌های علمی و تاریخی دارد که بهترین توضیح برای "چرخش زمین" فاجعه‌بار ECDO است.

\end{abstract}

{\small
\bibliographystyle{ieee}
\bibliography{egbib}
}

\end{document}

\documentclass[10pt,twocolumn,letterpaper]{article}

% My own stuff
\usepackage{booktabs}
% \usepackage{caption}
% \captionsetup[table]{skip=8pt}   % Only affects tables
\usepackage{stfloats}  % Add this to the preamble
\usepackage[T1]{fontenc}
\usepackage[utf8]{inputenc}  % Ensure UTF-8 encoding

\usepackage{cvpr}
\usepackage{times}
\usepackage{epsfig}
\usepackage{graphicx}
\usepackage{amsmath}
\usepackage{amssymb}

% Include other packages here, before hyperref.

% If you comment hyperref and then uncomment it, you should delete
% egpaper.aux before re-running latex.  (Or just hit 'q' on the first latex
% run, let it finish, and you should be clear).
\usepackage[breaklinks=true,bookmarks=false]{hyperref}

\cvprfinalcopy % *** Uncomment this line for the final submission

\def\cvprPaperID{****} % *** Enter the CVPR Paper ID here
\def\httilde{\mbox{\tt\raisebox{-.5ex}{\symbol{126}}}}

% Pages are numbered in submission mode, and unnumbered in camera-ready
%\ifcvprfinal\pagestyle{empty}\fi
\setcounter{page}{1}
\begin{document}

%%%%%%%%% TITLE
\title{ECDO Datengetriebene Einführung Teil 2/2: Eine Untersuchung von wissenschaftlichen und historischen Anomalien, die am besten durch einen ECDO „Erdkipp“ erklärt werden}

\author{Junho\\
Website: \href{https://sovrynn.github.io}{sovrynn.github.io}\\
ECDO Forschungsrepository: \href{https://github.com/sovrynn/ecdo}{github.com/sovrynn/ecdo}\\
{\tt\small junhobtc@proton.me}
}

\maketitle
%\thispagestyle{empty}

%%%%%%%%% ABSTRACT
\begin{abstract}
Im Mai 2024 veröffentlichte ein pseudonymer Online-Autor mit dem Namen „The Ethical Skeptic“ \cite{0} eine bahnbrechende Theorie namens Exothermic Core-Mantle Decoupling Dzhanibekov Oscillation (ECDO) \cite{1}. Diese Theorie schlägt nicht nur vor, dass die Erde zuvor plötzliche katastrophale Verschiebungen der Rotationsachse erlitten hat, die eine massive weltweite Flut verursachten, indem sie die Ozeane aufgrund der Rotationsinertie über die Kontinente verschüttete, sondern schlägt auch einen erklärenden ursächlichen geophysikalischen Prozess vor sowie Daten, die darauf hindeuten, dass ein weiterer solcher Kipp unmittelbar bevorstehen könnte. Während solche katastrophalen Flut- und Weltuntergangsvorhersagen nichts Neues sind, ist die ECDO-Theorie aufgrund ihres wissenschaftlichen, modernen, multidisziplinären und datenbasierten Ansatzes einzigartig überzeugend.

Dieses Forschungspaper stellt den zweiten Teil einer zweiteiligen komprimierten Zusammenfassung von 6 Monaten unabhängiger Forschung \cite{2,20} zur ECDO-Theorie dar, mit besonderem Fokus auf die wissenschaftlichen und historischen Anomalien, die am besten durch einen katastrophalen ECDO „Erdkipp“ erklärt werden.

\end{abstract}

%%%%%%%%% BODY TEXT

\section{Einleitung}

Die moderne uniformitaristische Geologie und Geschichte behaupten, dass große geologische Landschaften wie der Grand Canyon über Millionen von Jahren entstanden sind \cite{143}; dass Salz im Death Valley (Kalifornien) existiert, weil es vor Hunderten von Millionen Jahren unter dem Ozean lag \cite{144}; dass unsere Vorfahren vor 150 Generationen ihr ganzes Leben damit verbrachten, gigantische Gräber zu bauen \cite{29,70}; und dass sogenannte „fossile Brennstoffe“ Hunderte von Millionen Jahre alt sind \cite{104}. Vielleicht am faszinierendsten ist, dass angenommen wird, dass Menschen 300.000 Jahre alt sind \cite{145}, während die aufgezeichnete Geschichte und Zivilisation nur etwa 5.000 Jahre zurückreicht – was 150 menschlichen Generationen entspricht.

Solche Anomalien, wie wir sehen werden, werden am besten durch katastrophale geologische Kräfte erklärt.

\section{Blitzgefrorene Mammuts, die im Schlamm vergraben sind}

\begin{figure}[t]
\begin{center}
   \includegraphics[width=1\linewidth]{jarkov-mammoth.jpg}
\end{center}
   \caption{Das Jarkov-Mammut, ein 20.000 Jahre altes perfekt erhaltenes sibirisches Mammut, das in gefrorenem Schlamm gefunden wurde \cite{51}.}
\label{fig:1}
\label{fig:onecol}
\end{figure}

Ein solches Anomalienkategorie sind perfekt erhaltene, tiefgefrorene Mammuts, die im Schlamm begraben sind und häufig in den arktischen Regionen gefunden werden (Abbildung \ref{fig:1}). Das Beresowka-Mammut, das in Sibirien in kiesigem Schlamm begraben gefunden wurde, war so perfekt erhalten, dass sein Fleisch noch Tausende von Jahren nach seinem Tod essbar war. Es hatte auch pflanzliche Nahrung in seinem Mund und Magen, was die Wissenschaftler rätselte, wie es so schnell eingefroren sein konnte, wenn es unmittelbar vor seinem Tod auf Blütenpflanzen weidete \cite{17}. Berichten zufolge, \textit{"Im Jahr 1901 erregte die Entdeckung eines vollständigen Mammut-Kadavers nahe dem Beresowka-Fluss Aufsehen, da dieses Tier scheinbar mitten im Sommer an Kälte gestorben war. Der Inhalt seines Magens war gut erhalten und enthielt Butterblumen und blühende wilde Bohnen: dies bedeutete, dass sie etwa Ende Juli oder Anfang August geschluckt worden sein müssen. Das Tier war so plötzlich gestorben, dass es noch immer einen Mund voll Gräser und Blumen hielt. Es war eindeutig von einer enormen Kraft erfasst und mehrere Meilen von seiner Weidefläche geschleudert worden. Das Becken und ein Bein waren gebrochen - das riesige Tier war auf die Knie geschleudert worden und dann bei dem, was normalerweise die heißeste Zeit des Jahres ist, erfroren."} \cite{18}. Außerdem, \textit{"[Russische Wissenschaftler] stellten fest, dass sogar die innerste Auskleidung des Magens des Tieres eine perfekt erhaltene faserige Struktur aufwies, was darauf hindeutet, dass seine Körperwärme durch einen außergewöhnlichen Prozess in der Natur entfernt worden war. Sanderson, der diesen Punkt besonders beachtete, brachte das Problem zum American Frozen Foods Institute: Was ist nötig, um ein ganzes Mammut einzufrieren, so dass der Feuchtigkeitsgehalt selbst der innersten Teile seines Körpers, sogar bis zur inneren Auskleidung seines Magens, nicht genügend Zeit hat, um Kristalle zu bilden, die groß genug sind, um die faserige Struktur des Fleisches zu zerstören?... Einige Wochen später kehrte das Institut mit der Antwort zu Sanderson zurück: Es ist völlig unmöglich. Mit all unserem wissenschaftlichen und technischen Wissen gibt es absolut keine bekannte Möglichkeit, die Körperwärme von einem Kadaver, der so groß wie ein Mammut ist, schnell genug zu entfernen, um ihn ohne die Bildung großer Feuchtigkeitskristalle im Fleisch einzufrieren. Außerdem stellten sie nach Ausschöpfung der wissenschaftlichen und technischen Techniken fest, dass es keinen bekannten Prozess in der Natur gibt, der dieses Kunststück vollbringen könnte."} \cite{19}.

\section{Der Grand Canyon}

Der Grand Canyon, Teil des Great Basin im Südwesten Nordamerikas, ist ein weiteres Naturphänomen, das auf katastrophale Ursprünge hinweist (Abbildung \ref{fig:2}). Zunächst erstrecken sich die sedimentären Sandstein- und Kalksteinschichten, aus denen der Grand Canyon besteht, über gewaltige Flächen von bis zu 2,4 Millionen km$^2$ \cite{21}. Abbildung \ref{fig:3} zeigt das Ausmaß der Coconino-Sandsteinschicht im Westen der Vereinigten Staaten. Solch gewaltige horizontale Schichten aus einheitlichem Material könnten nur auf einmal abgelagert worden sein.

\begin{figure}[t]
\begin{center}
   \includegraphics[width=1\linewidth]{grand-canyon.jpg}
\end{center}
   \caption{Der Grand Canyon, in Arizona, USA \cite{49}.}
\label{fig:2}
\label{fig:onecol}
\end{figure}

\begin{figure}[t]
\begin{center}
   \includegraphics[width=1\linewidth]{coconino.jpg}
\end{center}
   \caption{Größe der Coconino-Sandsteinschicht im Westen der Vereinigten Staaten \cite{21}.}
\label{fig:3}
\label{fig:onecol}
\end{figure}
Ein genauerer Blick auf den Grand Canyon zeigt uns, dass die Ablagerung dieser ausgedehnten Sedimentschichten auch gleichzeitig mit signifikanten tektonischen Kräften erfolgte. Um dies zu verstehen, müssen wir bestimmte Bereiche im Canyon genau betrachten, in denen die Sedimentschichten gefaltet und freigelegt wurden. Forscher von Answers in Genesis \cite{42} nahmen eine mikroskopische Untersuchung von Gesteinsproben aus einigen dieser Falten, wie etwa der Monument Falte, vor und kamen aufgrund des Fehlens von Merkmalen, die hätten vorhanden sein müssen, wenn die Falten über lange Zeiträume unter Hitze und Druck entstanden wären, zu dem Schluss, dass die Sedimentschichten durch tektonische Kräfte gefaltet wurden, als sie noch weich waren, d.h. kurz nach ihrer Ablagerung \cite{43}.

\begin{figure*}
\begin{center}
\includegraphics[width=1\textwidth]{Grand_Staircase-big.jpg}
\end{center}
   \caption{Die Sedimentschichten, die den Grand Canyon bilden (rechte Seite des Bildes), erstrecken sich direkt nach Norden bis Cedar Breaks, Utah (linke Seite des Bildes), wo sie alle nach oben gebogen sind \cite{50}.}
\label{fig:4}
\end{figure*}

Wenn wir herauszoomen, stellen wir fest, dass die Schichten, die den Grand Canyon bilden, nicht nur im Canyon gefaltet wurden. Die Schichten wurden im Osten in der East Kaibab Monokline \cite{46} gefaltet, aber auch nach Norden in Cedar Breaks, Utah (Abbildung \ref{fig:4}). Dies deutet darauf hin, dass diese Schichten möglicherweise alle zusammen gefaltet wurden, nachdem sie in schneller Folge übereinandergelegt wurden. Zur Orientierung: Die horizontalen Schichten des Grand Canyon sind etwa 1700 Meter dick. Das Ausmaß des geologischen Prozesses, der erforderlich ist, um Sedimentschichten von einer Meile Dicke abzulegen, ist enorm.

Die tatsächliche Entstehung des Grand Canyon ist ein weiteres umstrittenes Thema in der modernen Geologie. Die uniformitaristische Geologie schlägt vor, dass der Grand Canyon über Millionen von Jahren durch den Colorado River geformt wurde \cite{47}. Das Forscherteam von Answers in Genesis glaubt jedoch, dass der Grand Canyon wahrscheinlich in wenigen Wochen durch Abflusserosion entstand, als ein urzeitlicher See seine Grenzen durchbrach, was massive Mengen an Sediment entfernte, während es den Canyon aushöhlte. Es gibt Hinweise auf einen hochgelegenen See östlich des Grand Canyon in Seesedimentablagerungen und Meeresfossilien. Ein Vergleich des Grand Canyons mit anderen groß angelegten Beispielen von Abflusserosion, wie dem Afton Canyon und dem Mount St. Helens, zeigt eine ähnliche Topographie und belegt, dass große Canyons schnell durch große Mengen fließenden Wassers entstehen können \cite{48}.

Angesichts des Umfangs der geologischen Prozesse, die erforderlich sind, um Sediment über solche riesigen Landstriche abzulegen, der Gleichzeitigkeit massiver tektonischer Kräfte kurz nach der Ablagerung der Sedimentschichten und der winzigen Größe des Colorado River im Vergleich zum riesigen Ausmaß des Grand Canyon scheint es, dass an seiner Entstehung möglicherweise nichts allmähliches war.

\section{Derinkuyu Untergrundstadt}

Neben den Pyramiden ist ein großartiges Beispiel für altes Ingenieurwesen die unterirdische Stadt Derinkuyu (Abbildung \ref{fig:5}), die sich in Kappadokien, Türkei, befindet. Sie ist die größte von über 200 unterirdischen Unterkünften in der Region \cite{54}. Diese unterirdische Stadt soll bis zu 20.000 Menschen beherbergt haben und erstreckt sich über 18 Stockwerke und erreicht eine Tiefe von 85 Metern. Während ihr Alter nicht gesichert ist, wird sie auf mindestens 2800 Jahre geschätzt. Die Stadt wurde aus weichem Vulkangestein herausgearbeitet \cite{52, 53}.

\begin{figure}[b]
\begin{center}
   \includegraphics[width=1\linewidth]{derinkuyu.jpeg}
\end{center}
   \caption{Diagramm der unterirdischen Stadt Derinkuyu \cite{56}.}
\label{fig:5}
\label{fig:onecol}
\end{figure}
Der Grund, warum Derinkuyu interessant ist, ist, dass unklar ist, warum eine Gemeinschaft sich entscheiden würde, eine ganze Stadt unter der Erde zu bauen. Um unterirdischen Wohnraum zu schaffen, muss jede Höhlung aus dem Fels gehauen werden. Die groben Formen und Texturen der unterirdischen Tunnel machen deutlich, dass diese von Hand gearbeitet wurden und nicht mit elektrischen Werkzeugen, was um Größenordnungen schwieriger gewesen wäre, als Unterkünfte oberirdisch zu bauen. Tatsächlich ist nicht ersichtlich, warum ein Mensch dauerhaft unterirdisch leben möchte, wenn Ackerbau, Sonnenlicht, Natur und Erkundung nur oberirdisch verfügbar sind, während der Begrenzungen ihres irdischen Lebens. Konventionelle "Geschichte" schlägt vor, dass Derinkuyu von Christen geschaffen wurde, die einen abgelegenen Ort zur Ausübung ihrer Religion benötigten \cite{53}. Aber der gesunde Menschenverstand würde folgern, dass der einfachste Weg, mit Feinden umzugehen, "Kampf oder Flucht" ist, nicht "eine unterirdische Stadt aus dem Fels hauen".

Der Umfang, die Tiefe und die Umsicht des Designs der unterirdischen Stadt machen klar, dass sie nicht als temporäre militärische Verteidigungsstruktur entworfen wurde, um Invasoren in Zeiten der Bedrängnis besser zu bekämpfen, sondern vielmehr als langfristiger Schutzraum, um gegen tödliche Kräfte an der Oberfläche zu schützen. Derinkuyu war nicht nur mit grundlegenden Schlafzimmern, Küchen und Badezimmern ausgestattet, sondern auch mit Ställen für Tiere, Wassertanks, Nahrungsmittellagern, Wein- und Ölpresse, Schulen, Kapellen, Gräbern und massiven Lüftungsschächten (Abbildung \ref{fig:6}). Warum sollte ein militärischer Schutzraum eine Weinpresse erfordern und mit solcher Komplexität 85 Meter tief gegraben werden?

Die plausibelste Erklärung für die Schaffung von Derinkuyu wäre ein dringendes Bedürfnis, einen langfristigen, sich selbst erhaltenden Schutzraum vorzubereiten, um gegen katastrophale geophysikalische Kräfte auf der Erdoberfläche zu schützen.

\begin{figure}[t]
\begin{center}
   \includegraphics[width=1\linewidth]{derinkuyu-air.jpg}
\end{center}
   \caption{Ein tiefer Lüftungsschacht in Derinkuyu \cite{53}.}
\label{fig:6}
\label{fig:onecol}
\end{figure}

\section{Biomassenansammlungen}

Biomischungen aus verschiedenen Arten von Tieren und Pflanzen, die oft fossilisiert in Sedimentschichten gefunden werden, sind eine weitere rätselhafte Anomalie. In "Reliquoæ Diluvianæ" beschreibt Rev. William Buckland Funde zahlreicher Faunaarten, die keinen erklärbaren Grund hatten, zusammen gefunden zu werden, verstreut über Großbritannien und Europa, vergraben in Schichten aus sedimentärem 'Diluvium' \cite{58}. Solche Mischungen von Tierüberresten wurden auch in der Skjonghelleren-Höhle auf der Insel Valdroy, Norwegen, gefunden. In dieser Höhle wurden über 7.000 Knochen von Säugetieren, Vögeln und Fischen gefunden, die über mehrere Sedimentschichten gemischt waren \cite{59}. Ein weiteres Beispiel ist San Ciro, die "Höhle der Riesen", in Italien. In dieser Höhle wurden mehrere Tonnen von Säugetierknochen, hauptsächlich Flusspferde, in einem Zustand so frisch gefunden, dass sie in Ornamente geschnitten und zum Lampenschwarzherstellung verschifft wurden. Die Knochen der verschiedenen Tiere waren Berichten zufolge miteinander vermischt, zerbrochen, zerschmettert und in Fragmenten verstreut \cite{60,61}. In Ancient Mendes, Ägypten, wurde eine Mischung verschiedener Tierknochen zusammen mit verglastem (glasigem) Ton gefunden \cite{57}. Solche Funde mögen rätselhaft erscheinen, aber sie lassen sich leicht durch massive Überschwemmungen erklären, die Mischungen toter Tiere in Sedimentschichten ablagern, Tiere dort hinein oder lebendig in Höhlen vergraben, und im Falle von verglaster Biomasse in Ägypten, durch enorme elektrostatische Entladungen nach einer Verschiebung des Kern-Mantels. Abbildung \ref{fig:7} zeigt eine typische Exposition von alaskischem Biomasse-'Schlamm' \cite{56}.

\begin{figure}[b]
\begin{center}
   \includegraphics[width=1\linewidth]{muck-crop.jpeg}
\end{center}
   \caption{Alaskischer 'Schlamm', bestehend aus chaotisch verteilten Fragmenten von Bäumen, Pflanzen und Tieren in gefrorenem Schluff und Eis \cite{146}.}
\label{fig:7}
\label{fig:onecol}
\end{figure}

\section{Alte Bunker}
Unsere Vorfahren hinterließen viele hochentwickelte antike Strukturen, in denen menschliche Überreste gefunden wurden. Diese werden normalerweise als aufwändige Gräber interpretiert, aber ein genauerer Blick legt nahe, dass es sich tatsächlich um alte Bunker gehandelt haben könnte.

\begin{figure}[b]
\begin{center}
   \includegraphics[width=1\linewidth]{ww19.jpg}
\end{center}
   \caption{Newgrange, Irland - siehe Besucher am Eingang zum Größenvergleich.}
\label{fig:8}
\label{fig:onecol}
\end{figure}

Ein ausgezeichnetes Beispiel ist Newgrange (Abbildung \ref{fig:8}), das Hauptmonument im Brú na Bóinne-Komplex, einer Sammlung von antiken Strukturen einschließlich der sogenannten Ganggräber. Diese Gräber bestehen aus einer oder mehreren Grabkammern, die mit Erde oder Stein bedeckt sind, und verfügen über einen schmalen Zugangspassage aus großen Steinen \cite{70}. Es ist ein Beispiel für die umfangreiche Ingenieurskunst einer komplexen geschützten Struktur, die über mehrere Generationen hinweg gebaut wurde, angeblich um eine Handvoll Menschen zu beerdigen, die nicht einmal am Leben waren, als der Bau des Grabes begann. Als es 1699 von einem örtlichen Landbesitzer wiederentdeckt wurde, war es mit Erde bedeckt.

Ein flüchtiger Blick auf die Struktur offenbart die immensen Anstrengungen, die in den Bau investiert wurden - Newgrange besteht aus etwa 200.000 Tonnen Material. Im Inneren befindet sich \textit{„...ein Kammerdurchgang, der über einen Eingang an der südöstlichen Seite des Monuments zugänglich ist. Der Durchgang erstreckt sich über 19 Meter (60 Fuß), oder etwa ein Drittel des Weges in das Zentrum der Struktur. Am Ende des Durchgangs befinden sich drei kleine Kammern, die von einer größeren zentralen Kammer mit einem hohen Kraggewölbe ausgehen... Die Wände dieses Durchgangs bestehen aus großen Steinplatten, sogenannten Orthostaten, von denen sich zweiundzwanzig auf der Westseite und einundzwanzig auf der Ostseite befinden. Sie sind durchschnittlich 1½ Meter hoch“} \cite{70}. Es gibt auch komplexe Details zur Abdichtung. Beispielsweise wurde im Dach \textit{„Die Zwischenräume des Daches wurden mit einer Mischung aus verbrannter Erde und Meeressand abgedichtet, um sie wasserdicht zu machen, und aus dieser Mischung wurden zwei Radiokarbondaten zentriert auf 2500 v. Chr. für die Struktur des Grabes erhalten“} \cite{71}. Zusätzlich könnte ein Anstieg in der Höhe, der zur inneren Kammer führt, für ähnliche Zwecke implementiert worden sein: \textit{„Da der Boden des Durchgangs und der Kammer des Grabes dem Anstieg des Bodens des Hügels folgt, auf dem das Monument erbaut ist, gibt es einen Unterschied von fast 2 Metern in der Bodenhöhe zwischen dem Eingang und dem Inneren der Kammer“} \cite{71}.

\begin{figure}[b]
\begin{center}
   \includegraphics[width=1\linewidth]{dolmen.jpg}
\end{center}
   \caption{Der Dolmen de Soto, Spanien \cite{53}.}
\label{fig:9}
\label{fig:onecol}
\end{figure}

Das Fehlen menschlicher Überreste im Inneren ist ebenfalls ein merkwürdiger Punkt. Ausgrabungen enthüllten verbrannte und unverbrannte Knochenfragmente, die eine Handvoll Menschen repräsentieren, die verstreut im Durchgang liegen. Der Bau von Newgrange wird aufgrund von Kohlenstoffdaten aus Materialien im Inneren auf mindestens mehrere Generationen geschätzt. Warum sollte eine antike Gemeinschaft so viel Mühe aufwenden, um ein riesiges, hochentwickeltes Grab zu bauen, nur um die Knochenfragmente einiger Verstorbener im Durchgang zu verstreuen? Es ist viel plausibler, dass diese alten und sorgfältig wasserdichten megalithischen Strukturen stattdessen als menschliche Unterkünfte gebaut wurden, um Menschen während wiederkehrender Katastrophen der Erde zu schützen.

In Huelva, Südspanien, gibt es ein ähnliches Beispiel: der Dolmen de Soto (Abbildung \ref{fig:9}), einer von etwa 200 solchen Stätten in der Gegend \cite{72,32}. Es ist eine stromlinienförmige, hochentwickelte Struktur, die mit megalithischen Steinen gebaut wurde und einen Durchmesser von 75 Metern hat. Berichten zufolge wurden bei der Ausgrabung nur acht Leichen gefunden, alle in fötaler Position begraben.

\section{Bemerkenswerte Anomalieerwähnungen}

In diesem Abschnitt erwähne ich kurz einige weitere bemerkenswerte Anomalien, die alle durch eine katastrophenähnliche ECDO erklärt werden können.

\subsection{Biologische Anomalien}
\begin{figure}[b]
\begin{center}
   \includegraphics[width=1\linewidth]{bottleneck.jpg}
\end{center}
   \caption{Ein genetischer Flaschenhals, der eine Reduktion von 95\% der männlichen Population vor etwa 6.000 Jahren zeigt \cite{62}.}
\label{fig:10}
\label{fig:onecol}
\end{figure}

Einige bemerkenswerte biologische Anomalien sind genetische Flaschenhälse und im Landesinneren gefundene Walfossilien. Zeng et al. (2018) modellierten 125 Y-Chromosomen-Sequenzen von modernen Menschen und identifizierten basierend auf Ähnlichkeiten und Mutationen in der DNA einen Flaschenhals mit einer 95\%igen Reduktion in der männlichen Population vor etwa 5.000 bis 7.000 Jahren (Abbildung \ref{fig:10}) \cite{62}. Walfossilien wurden hunderte Meter über dem Meeresspiegel in Schwedenborg, Michigan, Vermont, Kanada, Chile und Ägypten gefunden \cite{63,64,65,66}. Diese Wale wurden in unterschiedlichen Zuständen entdeckt: perfekt erhalten, in Mooren oberhalb von glazialen Ablagerungen liegend oder in Sedimenten vergraben. Die Anzahl der Exemplare an diesen Standorten reicht von wenigen bis zu über hundert. Wale sind Tiefsee-Kreaturen und nähern sich selten den Küsten. Wie kamen diese Wale in solch hohe Lagen, oft weit im Inland?

Zahlreiche Massenaussterben haben in der Vergangenheit der Erde stattgefunden, die am gründlichsten untersuchten sind die "Big Five"-Phanerozoischen Ereignisse: das Spätordovizium (LOME), das Spätdevon (LDME), das Ende-Perm (EPME), das Ende-Trias (ETME) und das Ende-Kreide (ECME) Massenaussterben \cite{88,89}. Merkwürdigerweise fallen mehrere dieser Aussterben in die gleichen historischen Perioden wie viele der Schichten des Grand Canyon, nämlich die Perm- und Devon-Schichten.

\subsection{Physikalische Anomalien}

\begin{figure}[b]
\begin{center}
   \includegraphics[width=1\linewidth]{columbia.jpg}
\end{center}
   \caption{Massive Strömungsrippel im glazialen See Columbia, Bundesstaat Washington \cite{80}.}
\label{fig:11}
\label{fig:onecol}
\end{figure}

Es gibt viele Landschaften außer dem Grand Canyon, die wahrscheinlich durch katastrophale Kräfte geformt wurden. Beweise für massive kontinentale Wasserströme finden sich in gigantischen Strömungsrippeln weltweit. Ein solches Beispiel sind die Channeled Scablands im pazifischen Nordwesten. Hier sehen wir nicht nur Landschaften aus sedimentären Ablagerungen und erratischen Felsbrocken, sondern auch mehr als hundert Abfolgen großer Rippel, die aus Megaströmungen entstanden sind \cite{78,79}. Diese sind großflächigere Versionen von Rippeln, die in den Sandbetten von Strömen geformt werden. Diese können überall auf der Welt in Frankreich, Argentinien, Russland und Nordamerika gefunden werden \cite{81}. Abbildung \ref{fig:11} zeigt einige dieser Rippel im Bundesstaat Washington in den Vereinigten Staaten \cite{80}.

\begin{figure}[b]
\begin{center}
% \fbox{\rule{0pt}{2in} \rule{0.9\linewidth}{0pt}}
   \includegraphics[width=1\linewidth]{zhangjiajie.jpg}
\end{center}
   \caption{Massive Steinsäulen im Zhangjiajie National Forest, Südchina.}
\label{fig:12}
\label{fig:onecol}
\end{figure}

\begin{figure}[b]
\begin{center}
   \includegraphics[width=1\linewidth]{hoy.jpg}
\end{center}
   \caption{Old Man of Hoy Seesäule, Schottland \cite{83}.}
\label{fig:13}
\label{fig:onecol}
\end{figure}

Inländische Erosionsstrukturen werden auch durch ein ECDO-ähnliches Erdflippen gut erklärt. Südchina ist ein hervorragendes Beispiel für massive Karstlandschaften, die durch Wassererosion entstanden sind \cite{82}. Diese Landschaften umfassen Turmkarst, Nadelkarst, Kegelkarst, natürliche Brücken, Schluchten, große Höhlensysteme und Dolinen. Eine der auffälligsten davon ist der Zhangjiajie-Nationalwald, der massive Quarz-Sandsteinpfeiler enthält (Abbildung \ref{fig:12}) \cite{84}. Diese Pfeiler stehen in einer durchschnittlichen Höhe von über 1.000 Metern und zählen mehr als 3.100. Mehr als 1.000 von ihnen ragen über 120 Meter hoch empor, und 45 erreichen über 300 Meter \cite{85}. Diese Pfeiler ähneln Seeerosionspfeilern (Abbildung \ref{fig:13}), die durch den Zusammenbruch des umgebenden Materials aufgrund von Meereswellen gebildete Küstenfelsen sind. Ähnliche Erosionslandschaften finden sich in den Felshüten von Urgup, Türkei, sowie in Ciudad Encantada, Spanien, die beide über 1.000 Meter über dem Meeresspiegel liegen. Alle diese Orte haben eine Kombination aus Salz- und ozeanischen Meeresfossilien in ihrer Nähe, was auf vergangene Meeresvorstöße hinweist \cite{15,86,87}. Natürlich erwähnen die Flutgeschichten \cite{3}, dass das Meer viel höher als 1.000 Meter ging, und dies wird durch das Vorhandensein von Salzwasser und massiven Salzebenen in den Anden und im Himalaya mehrere Kilometer über dem Meeresspiegel bestätigt. Das Uyuni-Salzbecken in Bolivien beispielsweise erreicht 3653 Meter über dem Meeresspiegel \cite{94}.

\subsection{Ereignisse des schnellen Klimawandels}

Die moderne wissenschaftliche Literatur erkennt die Existenz schneller globaler Klimawandelereignisse in der jüngeren Geschichte der Erde an. Zwei bemerkenswerte Beispiele sind die 4,2-Tausendjahr- und die 8,2-Tausendjahr-Ereignisse, die beide mit Bevölkerungsreduktion und der Störung gesellschaftlicher Siedlungen über große geographische Gebiete einhergehen. Diese Ereignisse sind als Anomalien in Sediment- und Eiskernen, fossilen Korallen, O18-Isotopenwerten, Pollen- und Speleothemaufzeichnungen sowie Meeresspiegeldaten erhalten. Die abgeleiteten Klimaveränderungen umfassen einen raschen Rückgang der globalen Temperaturen, Aridifikation, eine Störung der atlantischen meridionalen Umwälzströmung und Gletschervorstöße \cite{90,91,92}. Das 8,2-Tausendjahr-Ereignis fällt insbesondere mit einer möglichen dramatischen Salzwasserflutung des Schwarzen Meeres etwa um 6400 v. Chr. zusammen \cite{93}.

\subsection{Archäologische Anomalien}

Archäologische Beweise einiger antiker Städte zeigen mehrere Schichten mit Beerdigungen und Zerstörungen, die Aufzeichnungen vergangener katastrophaler Ereignisse geschaffen haben. Die antike Stadt Jericho ist eine solche Stadt, die im heutigen Palästina liegt. Sie enthält mehrere Zerstörungsschichten mit Einsturz von Steinstrukturen und intensiven Bränden \cite{96,97}. Die in ihren Schichten aufgezeichnete Chronologie reicht von etwa 9000 v. Chr. bis 2000 v. Chr. Besonders bemerkenswert ist ihr Turm, der um 7400 v. Chr. abgeschert und in Sediment eingetaucht zu sein scheint (Abbildung \ref{fig:14}) \cite{95}. Catal Huyuk \cite{99}, Gramalote \cite{98} und der minoische Palast von Knossos auf Kreta \cite{100,101} sind alle ähnliche Beispiele archäologischer Stätten, die mehrere Schichten mit Zerstörungsbeweisen enthalten.

\begin{figure}[t]
\begin{center}
   \includegraphics[width=1\linewidth]{jericho.jpg}
\end{center}
   \caption{Archäologische Rekonstruktion der Beerdigung des Turms von Jericho um 7400 v. Chr. \cite{95}.}
\label{fig:14}
\label{fig:onecol}
\end{figure}

Ein weiteres Indiz für große Katastrophen, die die menschliche Zivilisation störten, ist das Nampa-Bild, eine Tonpuppe, die unter etwa 100 Metern Lava in Idaho gefunden wurde \cite{102,103}. Der Lavastrom, unter dem die Figur gefunden wurde, soll während des späten Tertiärs oder frühen Quartärs abgelagert worden sein, und wird angeblich auf 2 Millionen Jahre datiert. Der Lavastrom in der Region wirkt jedoch relativ frisch. Solche Funde deuten nicht nur auf große zivildestabilisierende Katastrophen hin, sondern stellen auch moderne Datierungschronologien in Frage.

\section{Zu modernen Datierungsmethoden}
Es gibt beträchtlichen Grund, den modernen Chronologien skeptisch gegenüberzustehen, die extrem lange Zeiträume von Millionen oder sogar bis zu Hunderten von Millionen Jahren verschiedenen physikalischen Materialien zuweisen.

Die konventionelle Erzählung besagt, dass sogenannte „fossile Brennstoffe“ wie Kohle, Öl und Erdgas Hunderte von Millionen Jahren alt sind \cite{104}. Eine tatsächliche Radiokarbon-Datierung von Öl im Golf von Mexiko ergab jedoch ein Alter von etwa 13.000 Jahren für das Öl \cite{105}. Kohlenstoff-14 hat eine so kurze Halbwertszeit (5.730 Jahre), dass es nach einigen Hunderttausend Jahren vollständig zerfallen sein sollte. Dennoch wurde es in Kohle und Fossilien gefunden, die angeblich tausendmal älter sind \cite{106}. Tatsächlich wurde künstliche Kohle unter kontrollierten Bedingungen, hauptsächlich bei hoher Hitze, in einem Labor in nur 2-8 Monaten hergestellt \cite{107}.

Radiometrische Datierungsmethoden, die nicht auf Kohlenstoff-Datierung basieren, können ebenfalls ungenau sein. Die „Answers in Genesis“-Forschungsgruppe fand Inkonsistenzen bei Daten, die aus solchen Methoden abgeleitet wurden, was ihre Glaubwürdigkeit in Frage stellt \cite{108}. Weiches Gewebe, das Blutzellen, Gefäße und Kollagen enthält, wurde sogar in Dinosaurierresten gefunden, die angeblich Hundert Millionen Jahre alt sind \cite{109,110}. Basierend auf dem, was wir wissen, ist es möglich, dass die konventionell akzeptierten Altersangaben der geologischen Zeitskala der Erde und physikalischen Materialien wie Gesteinen und fossilen Brennstoffen um viele Größenordnungen falsch sein können.

\section{Schlussfolgerung}

In diesem Papier habe ich die überzeugendsten Anomalien behandelt, die auf katastrophale Ursprünge hindeuten und am besten durch eine ECDO-Erdumkehrung erklärt werden. Die vorgestellte Sammlung ist zwar vielfältig, aber unvollständig – weitere Anomalien wurden zusammengestellt und sind in meinem Forschungs-GitHub-Repository öffentlich zugänglich \cite{2}.

\section{Danksagungen}

Danke an Ethical Skeptic, den ursprünglichen Autor der ECDO-These, für die Fertigstellung seiner aufschlussreichen, bahnbrechenden Arbeit und das Teilen mit der Welt. Seine dreiteilige Arbeit \cite{1} bleibt das maßgebliche Werk für die Theorie der Exothermen Kern-Mantel-Entkopplung Dzhanibekov-Schwingung (ECDO) und enthält weitaus mehr Informationen zu dem Thema, als ich hier zusammenfassend dargestellt habe.

Und natürlich Dank an die Giganten, auf deren Schultern wir stehen; diejenigen, die all die Forschung und Untersuchungen betrieben haben, die diese Arbeit ermöglichten und daran gearbeitet haben, Licht in die Menschheit zu bringen.

{\small
\bibliographystyle{ieee}
\bibliography{egbib}
}

\end{document}

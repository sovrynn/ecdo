\documentclass[10pt,twocolumn,letterpaper]{article}

% My own stuff
\usepackage{booktabs}
% \usepackage{caption}
% \captionsetup[table]{skip=8pt}   % Only affects tables
\usepackage{stfloats}  % Add this to the preamble

\usepackage{cvpr}
\usepackage{times}
\usepackage{epsfig}
\usepackage{graphicx}
\usepackage{amsmath}
\usepackage{amssymb}

% Include other packages here, before hyperref.

% If you comment hyperref and then uncomment it, you should delete
% egpaper.aux before re-running latex.  (Or just hit 'q' on the first latex
% run, let it finish, and you should be clear).
\usepackage[breaklinks=true,bookmarks=false]{hyperref}

\cvprfinalcopy % *** Uncomment this line for the final submission

\def\cvprPaperID{****} % *** Enter the CVPR Paper ID here
\def\httilde{\mbox{\tt\raisebox{-.5ex}{\symbol{126}}}}

% Pages are numbered in submission mode, and unnumbered in camera-ready
%\ifcvprfinal\pagestyle{empty}\fi
\setcounter{page}{1}
\begin{document}

%%%%%%%%% TITLE
\title{ECDO Data-Driven Primer Part 2/2: An Investigation of Scientific and Historical Anomalies Best Explained by an ECDO “Earth Flip”}

\author{Junho\\
Published February 2025\\
Website (Download papers here): \href{https://sovrynn.github.io}{sovrynn.github.io}\\
ECDO Research Repo: \href{https://github.com/sovrynn/ecdo}{github.com/sovrynn/ecdo}\\
{\tt\small junhobtc@proton.me}
}

\maketitle
%\thispagestyle{empty}

\begin{abstract}
In May 2024, a pseudonymous online author by the name of “The Ethical Skeptic” \cite{0} posted a groundbreaking theory called the Exothermic Core-Mantle Decoupling Dzhanibekov Oscillation (ECDO) \cite{1}. This theory not only proposes that the Earth has previously undergone sudden catastrophic shifts in rotational axis, creating a massive worldwide flood by causing the oceans to spill over the continents due to rotational inertia, but also proposes an explanatory causative geophysical process along with data suggesting that another such flip might be imminent. While such cataclysmic flood and doomsday predictions are not new, the ECDO theory is uniquely compelling due to its scientific, modern, multidisciplinary, and data-based approach.

This research paper constitutes the second part of a two-part condensed summary of 6 months of independent research \cite{2,20} into the ECDO theory, focusing specifically on the scientific and historical anomalies that are best explained by a catastrophic ECDO "Earth flip".

\end{abstract}

\section{Introduction}

Modern uniformitarian geology and history claim that major geological landscapes such as the Grand Canyon were formed over millions of years \cite{143}; that salt exists in Death Valley (California) because it used to be under the ocean hundreds of millions of years ago \cite{144}; that our ancestors from 150 generations ago spent their entire lives building gigantic tombs \cite{29,70}; and that so-called "fossil fuels" are hundreds of millions of years old \cite{104}. Perhaps most fascinating is that humans are believed to be 300,000 years old \cite{145}, yet recorded history and civilization only date back about 5,000 years - the equivalent of 150 human generations.

Such anomalies, as we will see, are best explained by catastrophic geological forces.

\section{Flash-Frozen Mammoths Buried In Mud}

\begin{figure}[t]
\begin{center}
% \fbox{\rule{0pt}{2in} \rule{0.9\linewidth}{0pt}}
   \includegraphics[width=1\linewidth]{jarkov-mammoth.jpg}
\end{center}
   \caption{The Jarkov Mammoth, a 20,000 year old perfectly preserved Siberian mammoth found in frozen mud \cite{51}.}
\label{fig:1}
\label{fig:onecol}
\end{figure}

One such category of anomalies is perfectly preserved flash-frozen mammoths buried in mud, commonly found in the Arctic regions (Figure \ref{fig:1}). The Beresovka mammoth, discovered in Siberia buried in silty gravel, was so perfectly preserved that its meat was still edible thousand of years after its death. It also had vegetative food in its mouth and stomach, puzzling scientists as to how it could have been frozen so quickly if it was grazing on flowering plants right before its death \cite{17}. Reportedly, \textit{"In 1901 a sensation was caused by the discovery of a complete mammoth carcase near the Berezovka river, as this animal seemed to have died of cold in midsummer. The contents of its stomach were well preserved and included buttercups and flowering wild beans: this meant that they must have been swallowed about the end of July or beginning of August. The creature had died so suddenly that it still held in its jaws a mouthful of grasses and flowers. It had clearly been caught up by a tremendous force and hurled several miles from its pasture-ground. The pelvis and one leg were fractured—the huge animal had been knocked to its knees and had then frozen to death, at what is normally the hottest time of the year"} \cite{18}. Additionally, \textit{"[Russian scientists] recorded that even the innermost lining of the beast’s stomach had a perfectly preserved fibrous structure, indicating that his body heat had been removed by some super-prodigious process in nature. Sanderson, taking special notice of this one point, took the problem to the American Frozen Foods Institute: What does it take to freeze an entire mammoth so that the moisture content of even the innermost parts of his body, even to the inner lining of his stomach, do not have time enough to form crystals large enough to destroy the meat’s fibrous structure?... Some weeks later the Institute went back to Sanderson with the answer: It’s utterly impossible. With all of our scientific and engineering knowledge, there is absolutely no known way to remove the body heat from a carcass as big as a mammoth fast enough to freeze it without large moisture crystals forming in the meat. Furthermore, after exhausting the scientific and engineering techniques, they looked to nature and concluded that there is no known process in nature which could accomplish the feat"} \cite{19}.

\section{The Grand Canyon}

The Grand Canyon, part of the Great Basin in southwest North America, is another natural phenomenon suggesting catastrophic origins (Figure \ref{fig:2}). To start, the sedimentary sandstone and limestone layers that make up the Grand Canyon span massive areas of up to 2.4 million km$^2$ \cite{21}. Figure \ref{fig:3} shows the expanse of the Coconino Sandstone layer across the western United States. Such massive horizontal layers of uniform material could only have been laid down all at once.

\begin{figure}[b]
\begin{center}
% \fbox{\rule{0pt}{2in} \rule{0.9\linewidth}{0pt}}
   \includegraphics[width=1\linewidth]{grand-canyon.jpg}
\end{center}
   \caption{The Grand Canyon, in Arizona, USA \cite{49}.}
\label{fig:2}
\label{fig:onecol}
\end{figure}

\begin{figure}[t]
\begin{center}
% \fbox{\rule{0pt}{2in} \rule{0.9\linewidth}{0pt}}
   \includegraphics[width=1\linewidth]{coconino.jpg}
\end{center}
   \caption{Size of the Coconino Sandstone layer in the western United States \cite{21}.}
\label{fig:3}
\label{fig:onecol}
\end{figure}

A closer look at the Grand Canyon tells us that the deposition of these expansive sediment layers also occurred concurrently with significant tectonic forces. To understand this, we must take a close look at certain areas in the canyon where the sediment layers have been folded and exposed. Researchers from Answers in Genesis \cite{42} took a microscopic look at rock samples from some of these folds, such as the Monument Fold, and based on the lack of features that should have been present if the folds formed over long timeframes under heat and pressure, concluded that the sediment layers were folded by tectonic forces while they were still soft, i.e., soon after their deposition \cite{43}.

\begin{figure*}
\begin{center}
% \fbox{\rule{0pt}{2in} \rule{.9\linewidth}{0pt}}
\includegraphics[width=1\textwidth]{Grand_Staircase-big.jpg}
\end{center}
   \caption{The sediment layers making up the Grand Canyon (right side of picture) span directly north to Cedar Breaks, Utah (left side of picture), where they all bend upwards \cite{50}.}
\label{fig:4}
\end{figure*}

Zooming out, we find that the layers making up the Grand Canyon have not just been folded inside the canyon. The layers have been folded east in the East Kaibab Monocline \cite{46}, but also to the north in Cedar Breaks, Utah (Figure \ref{fig:4}). This suggests that these layers may have all been folded together after they were laid down on top of each other in quick succession. For reference, the horizontal layers of the Grand Canyon are approximately 1700 meters in thickness. The scale of geological process required to lay down sediment layers a mile thick is enormous.

The actual formation of the Grand Canyon is another issue of contention in modern geology. Uniformitarian geology proposes that the Grand Canyon was carved by the Colorado River over millions of years \cite{47}. However, the Answers in Genesis research team believes that the Grand Canyon was most likely formed in a matter of weeks due to spillway erosion from an ancient lake breaching its boundaries, which removed massive amounts of sediment as it carved out the canyon. There is evidence of a high-elevation lake east of the Grand Canyon in lake sediment deposits and marine fossils. Comparing the Grand Canyon to other large-scale examples of spillway erosion, such as Afton Canyon and Mount St. Helens, reveals similar topography, and shows that large canyons can be created rapidly through large amounts of flowing water \cite{48}.

Considering the scale of geological processes required to lay down sediment over such massive swathes of land, the concurrency of massive tectonic forces occurring soon after the sediment layers were laid down, and the miniscule size of the Colorado River in comparison to the massive scale of the Grand Canyon, it seems that there may have been nothing gradual about its formation.

\section{Derinkuyu Underground City}

Aside from the pyramids, a great example of ancient engineering is the underground city of Derinkuyu (Figure \ref{fig:5}), located in Cappadocia, Turkey. It is the largest among over 200 underground shelters in the region \cite{54}. This underground city is estimated to have housed up to 20,000 people and spans 18 floors, reaching depths of 85 meters. While its age is not certain, it is estimated to be at least 2800 years old. The city was carved out of soft volcanic rock \cite{52, 53}.

\begin{figure}[b]
\begin{center}
% \fbox{\rule{0pt}{2in} \rule{0.9\linewidth}{0pt}}
   \includegraphics[width=1\linewidth]{derinkuyu.jpeg}
\end{center}
   \caption{Diagram of the Derinkuyu underground city \cite{56}.}
\label{fig:5}
\label{fig:onecol}
\end{figure}

The reason Derinkuyu is interesting is because it's not clear why any community would decide to build an entire city underground. In order to create living space underground, every cavity must be carved out of rock. The rough shapes and textures of the underground tunnels make it clear these were carved with manual labor, rather than with power tools, which would have been orders of magnitude more difficult than building shelters above ground. In fact, it's not apparent why any human would want to permanently live underground during the confines of their earthly life, when agriculture, sunlight, nature, and exploration are only available above ground. Conventional "history" proposes that Derinkuyu was created by Christians who needed a secluded place to practice their religion \cite{53}. But common sense would conclude that the most straightforward way to deal with enemies is "fight or flight", not "carve an underground city out of rock".

The scale, depth, and thoughtfulness of the design of the underground city make it clear that it wasn't designed as a temporary military defensive structure to better fight invaders in times of duress, but rather, a long-term shelter to protect against fatal forces on the surface. Derinkuyu was equipped with not only basic bedrooms, kitchens, and bathrooms, but also stables for animals, water tanks, food storage, wine and oil presses, schools, chapels, tombs, and massive ventilation shafts (Figure \ref{fig:6}). Why would a military shelter require a wine press and need to be be dug 85 meters deep with such complexity?

The most plausible explanation for the creation of Derinkuyu would have been a pressing need to prepare a long-term, self-sustaining shelter to protect against catastrophic geophysical forces on Earth's surface.

\begin{figure}[t]
\begin{center}
% \fbox{\rule{0pt}{2in} \rule{0.9\linewidth}{0pt}}
   \includegraphics[width=1\linewidth]{derinkuyu-air.jpg}
\end{center}
   \caption{A deep ventilation well in Derinkuyu \cite{53}.}
\label{fig:6}
\label{fig:onecol}
\end{figure}

% \section{Additional Anomalies Best Explained By An Earth Flip}

% Before wrapping up, we will mention some additional scientific anomalies that, once viewed in the context of cataclysmic geophysical forces, are well explained.

\section{Biomass Accumulations}

Biomass mixtures of various kinds of animals and plants, often found fossilized in sediment layers, are another puzzling anomaly. In "Reliquoæ Diluvianæ", Rev. William Buckland details findings of numerous species of fauna that had no explicable reason to be found together, scattered across Britain and Europe, buried in layers of sedimentary 'diluvium' \cite{58}. Such mixtures of animal remains were also found in Skjonghelleren Cave on the island of Valdroy, Norway. In this cave, over 7,000 bones of mammals, birds, and fish were found mixed across multiple sediment layers \cite{59}. Another example is San Ciro, the "Cave of the Giants", in Italy. In this cave, several tons of mammal bones, mostly hippopotamus, were found in a state so fresh they were cut into ornaments and shipped out for the manufacture of lamp black. The bones of the different animals were reportedly mixed together, broken, shattered, and dispersed in fragments \cite{60,61}. In Ancient Mendes, Egypt, a mixture of various species of animal bones was found mixed with vitrified (glassy) clay \cite{57}.  Such findings may seem puzzling, but are easily explained by massive flooding laying down mixtures of dead animals in sediment layers, depositing animals into or burying them alive in caves, and in the case of vitrified biomass in Egypt, post-flood massive electrical discharges from a core-mantle displacement. Figure \ref{fig:7} depicts a typical exposure of Alaskan biomass 'muck' \cite{56}.

\begin{figure}[t]
\begin{center}
% \fbox{\rule{0pt}{2in} \rule{0.9\linewidth}{0pt}}
   \includegraphics[width=1\linewidth]{muck-crop.jpeg}
\end{center}
   \caption{Alaskan 'muck', composed of chaotically dispersed fragments of trees, plants and animals in frozen silt and ice \cite{146}.}
\label{fig:7}
\label{fig:onecol}
\end{figure}

\section{Ancient Bunkers}

Our ancestors left behind many highly engineered ancient structures where human remains have been found. These are usually interpreted to be elaborate tombs, but a closer look suggests these may actually have been ancient bunkers.

\begin{figure}[b]
\begin{center}
% \fbox{\rule{0pt}{2in} \rule{0.9\linewidth}{0pt}}
   \includegraphics[width=1\linewidth]{ww19.jpg}
\end{center}
   \caption{Newgrange, Ireland - see visitors at entrance for scale.}
\label{fig:8}
\label{fig:onecol}
\end{figure}

One excellent example is Newgrange (Figure \ref{fig:8}), the main monument in the Brú na Bóinne complex, a collection of ancient structures including so-called passage tombs. These tombs consist of one or more burial chambers covered in earth or stone and have a narrow access passage made of large stones \cite{70}. It is an example of extensive engineering of a complex protected structure, built over multiple generations, supposedly to bury a handful of people, who weren’t even alive when the construction of the tomb began. When it was rediscovered by a local landowner in 1699, it was buried in earth.

A cursory look at the structure reveals the immense effort spent building it - Newgrange consists of about 200,000 tons of material. Inside it, \textit{“...is a chambered passage, which may be accessed by an entrance on the southeastern side of the monument. The passage stretches for 19 metres (60 ft), or about a third of the way into the centre of the structure. At the end of the passage are three small chambers off a larger central chamber with a high corbelled vault roof… The walls of this passage are made up of large stone slabs called orthostats, twenty-two of which are on the western side and twenty-one on the eastern side. They average 1½ metres in height”} \cite{70}. There are also intricate waterproofing engineering details. For example, in the roof, \textit{“The interstices of the roof were caulked with a mixture of burnt soil and sea sand to render them waterproof and from this mixture two radiocarbon dates centering on 2500 BCE were obtained for the structure of the tomb"} \cite{71}. Additionally, an elevation rise leading to the inner chamber may have been implemented for similar purposes: \textit{“Since the floor of the passage and chamber of the tomb follows the rise of ground of the hill on which the monument is built there is a difference of almost 2 meters in floor level between the entrance and the interior of the chamber”} \cite{71}.

\begin{figure}[b]
\begin{center}
% \fbox{\rule{0pt}{2in} \rule{0.9\linewidth}{0pt}}
   \includegraphics[width=1\linewidth]{dolmen.jpg}
\end{center}
   \caption{The Dolmen de Soto, Spain \cite{53}.}
\label{fig:9}
\label{fig:onecol}
\end{figure}

The lack of human remains inside is also a curious point. Excavations revealed burnt and unburnt bone fragments representing a handful of people, scattered about the passage. The construction of Newgrange is estimated to have taken at least several generations based on carbon dates of materials inside. Why would an ancient community spend so much effort to build a massive, highly engineered tomb only to scatter the bone fragments of a few deceased in its passageway? It is much more plausible that these ancient and carefully waterproofed megalithic structures were instead built as human shelters to protect people during Earth’s recurring cataclysms.

In Huelva, southern Spain, a similar example is the Dolmen de Soto (Figure \ref{fig:9}), one of about 200 such sites in the area \cite{72,32}. It is a streamlined, highly engineered structure built using megalithic stones and has a diameter of 75 meters. Reportedly, only eight bodies were found upon excavation, all buried in a fetal position.

\section{Notable Anomaly Mentions}

In this section, I briefly mention some more notable anomalies, all of which are well explained by an ECDO-like cataclysm.

\subsection{Biological Anomalies}

\begin{figure}[b]
\begin{center}
% \fbox{\rule{0pt}{2in} \rule{0.9\linewidth}{0pt}}
   \includegraphics[width=1\linewidth]{bottleneck.jpg}
\end{center}
   \caption{A genetic bottleneck representing a culling of 95\% of males around 6,000 years ago \cite{62}.}
\label{fig:10}
\label{fig:onecol}
\end{figure}

Some notable biological anomalies are genetic bottlenecks and inland whale fossils. Zeng et al. (2018) modeled 125 Y-chromosome sequences from modern humans, and based on the similarities and mutations in the DNA, identified a 95\% population reduction bottleneck in the male population around 5,000 to 7,000 years ago (Figure \ref{fig:10}) \cite{62}. Whale fossils have been found hundreds of meters above sea level, in Swedenborg, Michigan, Vermont, Canada, Chile, and Egypt \cite{63,64,65,66}. These whales were found in varying states: perfectly preserved, in bogs lying above glacial deposits, or buried in sediment. The number of specimens in these sites ranges from a few to over a hundred.  Whales are deep-sea creatures and rarely venture near the shores. How did these whales end up at such high elevations, often at extreme distances inland?

Numerous mass extinctions have occurred in Earth’s past, the most thoroughly studied being the "Big Five"  Phanerozoic events: the Late Ordovician (LOME), Late Devonian (LDME), end-Permian (EPME), end-Triassic (ETME) and end-Cretaceous (ECME) mass extinctions \cite{88,89}. Curiously, several of these extinctions are classified as occurring in the same historical periods as many of the Grand Canyon's layers, namely, the Permian and Devonian layers.

\subsection{Physical Anomalies}

\begin{figure}[b]
\begin{center}
% \fbox{\rule{0pt}{2in} \rule{0.9\linewidth}{0pt}}
   \includegraphics[width=1\linewidth]{columbia.jpg}
\end{center}
   \caption{Massive current ripples in Glacial Lake Columbia, Washington state \cite{80}.}
\label{fig:11}
\label{fig:onecol}
\end{figure}

There are many landscapes other than the Grand Canyon that were likely formed through cataclysmic forces. Evidence of massive continental water flow can be found in giant current ripples worldwide. One such example is the Channeled Scablands in the Pacific Northwest. Here, not only do we see sedimentary deposit landscapes and erratic boulders, but also more than a hundred sequences of large ripples formed from mega current flows \cite{78,79}. These are larger-scale versions of ripples formed in the sand beds of streams. These can be found all over the world in France, Argentina, Russia, and North America \cite{81}. Figure \ref{fig:11} depicts some of these ripples in Washington state in the United States \cite{80}.

\begin{figure}[b]
\begin{center}
% \fbox{\rule{0pt}{2in} \rule{0.9\linewidth}{0pt}}
   \includegraphics[width=1\linewidth]{zhangjiajie.jpg}
\end{center}
   \caption{Massive stone pillars in Zhangjiajie National Forest, south China.}
\label{fig:12}
\label{fig:onecol}
\end{figure}

\begin{figure}[b]
\begin{center}
% \fbox{\rule{0pt}{2in} \rule{0.9\linewidth}{0pt}}
   \includegraphics[width=1\linewidth]{hoy.jpg}
\end{center}
   \caption{Old Man of Hoy sea pillar, Scotland \cite{83}.}
\label{fig:13}
\label{fig:onecol}
\end{figure}

Inland erosion structures are also well-explained by an ECDO-like Earth flip. Southern China is a great example of massive karst landscapes, formed through water erosion \cite{82}. These landscapes include tower karst, pinnacle karst, cone karst, natural bridges, gorges, large cave systems, and sinkholes. One of the most striking of these is the Zhangjiajie National Forest, which contains massive quartz sandstone pillars (Figure \ref{fig:12}) \cite{84}. These pillars stand at an average elevation of over 1,000 meters and number more than 3,100. More than 1,000 of them soar above 120 meters tall, and 45 reach over 300 meters \cite{85}. These pillars resemble sea erosion pillars (Figure \ref{fig:13}), which are coastal rock pillars formed by the collapse of surrounding material due to ocean waves. Similar erosion landscapes can be found in the rock cones of Urgup, Turkey, as well as Ciudad Encantada, Spain, which are both over 1,000 meters above sea level. All these locations have some combination of salt and oceanic marine fossils in close proximity to them, suggesting past marine incursions \cite{15,86,87}. Of course, the flood stories \cite{3} mention the ocean going much higher than 1,000 meters, and this is verified by the presence of saltwater and massive salt flats in the Andes and Himalayas several kilometers above sea level. The Uyuni salt flat in Bolivia, for example, reaches 3653 meters above sea level \cite{94}.

\subsection{Rapid Climate Change Events}

Modern scientific literature recognizes the existence of rapid global climate change events in Earth's recent history. Two notable examples are the 4.2 kiloyear and 8.2 kiloyear events, both coinciding with population reduction and societal settlement disruption over large geographical areas. These events are preserved as anomalies in sediment and ice cores, fossil corals, O18 isotope values, pollen and speleothem records, and sea-level data. The inferred climate changes include a rapid drop in global temperatures, aridification, a disruption of the Atlantic meridional overturning current, and glacial advances \cite{90,91,92}. The 8.2 kiloyear event in particular is concurrent with a potential dramatic saltwater flooding of the Black Sea around 6400 BCE \cite{93}.

\subsection{Archaeological Anomalies}

Archaeological evidence of some ancient cities shows multiple layers involving burial and destruction, creating records of past cataclysmic events. The ancient city of Jericho is one such city, located in modern-day Palestine. It contains multiple destruction layers, with collapse of stone structures and intense fire \cite{96,97}. The chronology recorded in its layers dates from approximately 9000 BCE to 2000 BCE. Of particular note is its tower, which seems to have been sheared off and buried in sediment around 7400 BCE (Figure \ref{fig:14}) \cite{95}. Catal Huyuk \cite{99}, Gramalote \cite{98}, and the Minoan palace of Knossos on Crete \cite{100,101} are all similar examples of archaeological sites containing multiple layers, often containing evidence of destruction.

\begin{figure}[t]
\begin{center}
% \fbox{\rule{0pt}{2in} \rule{0.9\linewidth}{0pt}}
   \includegraphics[width=1\linewidth]{jericho.jpg}
\end{center}
   \caption{Archaeological reconstruction of the burial of the Tower of Jericho circa 7400 BCE \cite{95}.}
\label{fig:14}
\label{fig:onecol}
\end{figure}

Another piece of evidence for major cataclysms disrupting human civilization is the Nampa Image, a clay doll found beneath approximately 100 meters of lava in Idaho \cite{102,103}. The lava flow under which the figurine was found was estimated to be deposited during the Late Tertiary or early Quaternary period, supposedly being 2 million years old. However, the lava flow in the region appears to be relatively fresh. Such finds not only point to major civilization-destroying cataclysms, but also call into question modern dating chronologies.

\section{Regarding Modern Dating Methods}

There is significant reason to be skeptical of modern chronologies, which assign extremely long ages of millions, or even up to hundreds of millions of years to various physical materials.

The conventional narrative states that so-called "fossil fuels" such as coal, oil, and natural gas are hundreds of millions of years old \cite{104}. However, an actual carbon dating of oil in the Gulf of Mexico found an age of approximately 13,000 years for the oil \cite{105}. Carbon-14 has such a short half-life (5,730 years) that it is supposed to completely decay after a few hundred thousand years. However, it has been found in coal and fossils that are supposedly a thousand times older \cite{106}. In fact, artificial coal has been produced in a laboratory under controlled conditions, primarily high heat, in just 2-8 months \cite{107}.

Radioisotope dating methods other than carbon dating may also not be accurate. The Answers in Genesis research group found inconsistencies in dates derived from such methods that call their veracity into question \cite{108}. Soft tissue containing blood cells, vessels, and collagen has even been found in dinosaur remains supposedly a hundred million years old \cite{109,110}. Based on what we know, it is possible that the conventionally accepted ages of Earth's geological timescale and physical materials such as rocks and fossil fuels may be incorrect by many orders of magnitude.

\section{Conclusion}

In this paper, I have covered the most compelling anomalies that suggest catastrophic origins and are best explained by an ECDO Earth flip. While diverse, the presented collection is incomplete - more anomalies have been compiled and are publicly available in my research GitHub repository \cite{2}.

\section{Acknowledgments}

Thanks to Ethical Skeptic, the original author of the ECDO thesis, for completing his insightful, groundbreaking thesis and sharing it with the world. His tri-part thesis \cite{1} remains the authoritative work for Exothermic Core-Mantle Decoupling Dzhanibekov Oscillation (ECDO) theory, and contains much more information on the topic than I have summarized in short here.

And of course, thanks to the giants whose shoulders we stand on; those who have done all the research and investigation that made this work possible and worked to bring light to humanity.

\clearpage
\twocolumn

{\small
\bibliographystyle{ieee}
\bibliography{egbib}
}

\end{document}

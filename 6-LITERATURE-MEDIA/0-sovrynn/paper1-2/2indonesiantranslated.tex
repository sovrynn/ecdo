\documentclass[10pt,twocolumn,letterpaper]{article}

% Barang saya sendiri
\usepackage{booktabs}
% \usepackage{caption}
% \captionsetup[table]{skip=8pt}   % Hanya mempengaruhi tabel
\usepackage{stfloats}  % Tambahkan ini ke preambule

\usepackage{cvpr}
\usepackage{times}
\usepackage{epsfig}
\usepackage{graphicx}
\usepackage{amsmath}
\usepackage{amssymb}

% Sertakan paket lain di sini, sebelum hyperref.

% Jika Anda mengomentari hyperref lalu menghapus komentarnya, Anda harus menghapus
% egpaper.aux sebelum menjalankan latex lagi.  (Atau cukup tekan 'q' pada latex
% pertama, biarkan selesai, dan Anda akan aman).
\usepackage[breaklinks=true,bookmarks=false]{hyperref}

\cvprfinalcopy % *** Uncomment this line for the final submission

\def\cvprPaperID{****} % *** Enter the CVPR Paper ID here
\def\httilde{\mbox{\tt\raisebox{-.5ex}{\symbol{126}}}}

\renewcommand{\figurename}{Gambar}   % or whatever you like instead of "Hình"
\renewcommand{\refname}{Referensi}

\makeatletter
\def\abstract{%
  \centerline{\large\bf Abstrak}% <-- your new label
  \vspace*{12pt}%
  \it%
}
\makeatother

% Halaman diberi nomor dalam mode pengajuan, dan tidak bernomor pada versi siap cetak
%\ifcvprfinal\pagestyle{empty}\fi
\setcounter{page}{1}
\begin{document}

%%%%%%%%% JUDUL
\title{Primer Berbasis Data ECDO Bagian 2/2: Investigasi Anomali Ilmiah dan Sejarah yang Paling Baik Dijelaskan oleh “Pembalikan Bumi” ECDO}

\author{Junho\\
Diterbitkan Februari 2025\\
Situs Web (Unduh makalah di sini): \href{https://sovrynn.github.io}{sovrynn.github.io}\\
Repo Riset ECDO: \href{https://github.com/sovrynn/ecdo}{github.com/sovrynn/ecdo}\\
{\tt\small junhobtc@proton.me}
% Untuk sebuah makalah yang semua penulisnya berasal dari institusi yang sama,
% hilangkan baris-baris berikut hingga penutup ``}''.
% Penulis dan alamat tambahan dapat ditambahkan dengan ``\and'',
% seperti penulis kedua.
% Untuk menghemat ruang, gunakan alamat email atau halaman rumah, bukan keduanya
% \and
% xx
% Institution2\\
% Baris pertama alamat institution2\\
% {\tt\small secondauthor@i2.org}
}

\maketitle
%\thispagestyle{empty}

%%%%%%%%% ABSTRACT
\begin{abstract}
Pada bulan Mei 2024, seorang penulis daring dengan nama samaran “The Ethical Skeptic” \cite{0} memposting sebuah teori terobosan yang disebut Osilasi Dzhanibekov Pelepasan Inti-Mantel Eksotermik (ECDO: Exothermic Core-Mantle Decoupling Dzhanibekov Oscillation) \cite{1}. Teori ini tidak hanya mengemukakan bahwa Bumi sebelumnya pernah mengalami pergeseran mendadak dan katastropik pada sumbu rotasinya—yang menyebabkan banjir besar global akibat lautan tumpah ke daratan karena inersia rotasi, tetapi juga menyajikan mekanisme geofisika yang mendasarinya, berdasarkan data empiris yang mengindikasikan bahwa peristiwa serupa berpotensi terjadi kembali dalam waktu dekat. Walaupun ramalan tentang banjir katastropik dan kiamat bukanlah hal baru, teori ECDO sangat menarik karena pendekatannya yang ilmiah, modern, multidisipliner, dan berbasis data.

Makalah penelitian ini merupakan bagian kedua dari rangkuman singkat dua bagian atas 6 bulan penelitian independen \cite{2,20} tentang teori ECDO, dengan fokus khusus pada anomali ilmiah dan sejarah yang paling masuk akal jika dijelaskan melalui peristiwa "pembalikan Bumi" ECDO yang katastropik.
\end{abstract}

%%%%%%%%% BODY TEXT

\section{Pendahuluan}

Geologi dan sejarah yang berlandaskan prinsip uniformitarianisme modern mengklaim bahwa bentang alam geologis besar seperti Grand Canyon terbentuk selama jutaan tahun \cite{143}; garam yang ada di Death Valley (California) disebabkan tempat itu dulunya berada di bawah laut ratusan juta tahun yang lalu \cite{144}; leluhur kita dari 150 generasi yang lalu menghabiskan seluruh hidup mereka membangun bangunan-bangunan makam raksasa \cite{29,70}; dan apa yang disebut dengan "bahan bakar fosil" berusia hingga ratusan juta tahun \cite{104}. Mungkin yang paling menarik adalah manusia diyakini telah ada selama sekitar 300.000 tahun \cite{145}, namun sejarah tertulis dan peradaban baru muncul sekitar 5.000 tahun yang lalu – setara dengan 150 generasi manusia.

Sebagaimana yang akan kita lihat, keanehan-keanehan ini paling masuk akal bila dijelaskan dengan kekuatan geologi yang bersifat katastropik.

\section{Mamut yang Membeku Seketika dan Terkubur dalam Lumpur}

\begin{figure}[t]
\begin{center}
% \fbox{\rule{0pt}{2in} \rule{0.9\linewidth}{0pt}}
   \includegraphics[width=1\linewidth]{jarkov-mammoth.jpg}
\end{center}
   \caption{Mamut Jarkov, mamut Siberia berusia 20.000 tahun yang terawetkan sempurna dan ditemukan dalam lumpur beku \cite{51}.}
\label{fig:1}

\label{fig:onecol}
\end{figure}

Salah satu kategori anomali tersebut adalah mamut yang terawetkan dengan sempurna, membeku secara tiba-tiba, dan terkubur dalam lumpur, yang kerap ditemukan di wilayah Arktik (Gambar \ref{fig:1}). Mamut Beresovka, yang ditemukan di Siberia telah terkubur dalam kerikil berlumpur, dan terawetkan dengan sangat baik, sampai-sampai dagingnya tetap dapat dimakan meskipun telah berlalu ribuan tahun sejak kematiannya. Mamut tersebut juga masih menyimpan sisa-sisa makanan di mulut dan perutnya, sehingga menimbulkan kebingungan di kalangan ilmuwan bagaimana mungkin hewan ini bisa membeku dengan sangat cepat jika hanya sesaat sebelum kematiannya ia masih merumput pada tanaman berbunga\cite{17}. Dilaporkan, \textit{"Pada tahun 1901 terjadi sensasi akibat penemuan bangkai utuh mamut di dekat sungai Berezovka, karena hewan ini tampaknya mati kedinginan di musim panas. Isi perutnya masih terawetkan dengan baik dan di dalamnya ditemukan bunga yolanda serta bunga kacang liar: ini berarti bahwa tumbuhan tersebut pasti ditelan sekitar akhir Juli atau awal Agustus. Kematian hewan itu terjadi begitu mendadak sampai-sampai rumput dan bunga masih tersisa di mulutnya. Jelas mamut tersebut telah terhempas oleh kekuatan dahsyat dan terlempar beberapa mil dari padang tempatnya merumput. Panggul dan satu kakinya patah—hewan besar itu jatuh berlutut dan mati beku, tepat di masa yang seharusnya merupakan periode terpanas dalam setahun"} \cite{18}. Selain itu, \textit{"[Ilmuwan Rusia] mencatat bahwa bahkan lapisan terdalam dari lambung hewan itu memiliki struktur serat yang terawetkan secara sempurna, yang menunjukkan bahwa panas tubuhnya telah hilang oleh suatu proses alam yang luar biasa. Sanderson, yang memberikan perhatian khusus pada hal ini, membawa masalah ini ke Institut Makanan Beku Amerika: Apa yang dapat membekukan seekor mamut secara menyeluruh sampai kandungan air di bagian terdalam tubuhnya, bahkan sampai lapisan terdalam lambungnya, tidak sempat membentuk kristal besar yang menyebabkan kerusakan struktur serat daging?... Beberapa minggu kemudian, Institut tersebut memberikan tanggapan kepada Sanderson: Ini benar-benar mustahil. Dengan seluruh pengetahuan ilmiah dan teknologi yang kami miliki, sama sekali tidak ada metode yang dapat menghilangkan panas tubuh dari bangkai sebesar mamut dengan cukup cepat untuk bisa membekukannya tanpa membentuk kristal-kristal air yang besar pada dagingnya. Lebih jauh lagi, setelah mencoba berbagai pendekatan ilmiah dan teknologi, mereka meninjau kembali proses-proses alamiah dan menyimpulkan bahwa tidak ada proses alami yang mereka tahu yang dapat melakukan hal itu"} \cite{19}.

\section{Grand Canyon}

Grand Canyon, yang merupakan bagian dari Great Basin di Amerika Utara bagian barat daya, adalah fenomena alam lain yang kemungkinan dibentuk oleh peristiwa katastropik (Gambar \ref{fig:2}). Pertama-tama, lapisan batu pasir dan batu kapur sedimen yang membentuk Grand Canyon membentang sangat luas hingga 2,4 juta km$^2$ \cite{21}. Gambar \ref{fig:3} memperlihatkan luas lapisan batu pasir Coconino di seluruh Amerika Serikat bagian barat. Lapisan horizontal seluas ini, yang tersusun dari material seragam, hanya mungkin terbentuk secara sekaligus dan bukan secara bertahap.

\begin{figure}[t]
\begin{center}
% \fbox{\rule{0pt}{2in} \rule{0.9\linewidth}{0pt}}
   \includegraphics[width=1\linewidth]{grand-canyon.jpg}
\end{center}
   \caption{Grand Canyon, di Arizona, AS \cite{49}.}
\label{fig:2}
\label{fig:onecol}
\end{figure}

\begin{figure}[t]
\begin{center}
% \fbox{\rule{0pt}{2in} \rule{0.9\linewidth}{0pt}}
   \includegraphics[width=1\linewidth]{coconino.jpg}
\end{center}
   \caption{Ukuran lapisan batu pasir Coconino di Amerika Serikat bagian barat\cite{21}.}
\label{fig:3}
\label{fig:onecol}
\end{figure}

Tinjauan lebih dekat terhadap Grand Canyon menunjukkan bahwa pengendapan lapisan sedimen yang luas ini juga terjadi bersamaan dengan gaya tektonik yang signifikan. Untuk memahami hal ini, kita harus mengamati dengan saksama area tertentu di ngarai di mana lapisan sedimen telah terlipat dan terbuka. Peneliti dari Answers in Genesis \cite{42} mengamati secara mikroskopis beberapa sampel batuan dari lipatan-lipatan ini, seperti Monument Fold, dan karena tidak ditemukannya fitur yang seharusnya ada jika lipatan tersebut terbentuk dalam jangka waktu lama di bawah panas dan tekanan, mereka menyimpulkan bahwa lapisan sedimen tersebut terlipat oleh gaya tektonik ketika lapisan itu masih lunak, yaitu seketika setelah terjadi pengendapan \cite{43}.

\begin{figure*}
\begin{center}
% \fbox{\rule{0pt}{2in} \rule{.9\linewidth}{0pt}}
\includegraphics[width=1\textwidth]{Grand_Staircase-big.jpg}
\end{center}
   \caption{Lapisan sedimen yang membentuk Grand Canyon (sisi kanan gambar) membentang langsung ke utara hingga Cedar Breaks, Utah (sisi kiri gambar), di mana semuanya membengkok ke atas \cite{50}.}
\label{fig:4}
\end{figure*}

Jika melihat dari skala yang lebih luas, kami menemukan bahwa lapisan-lapisan yang membentuk Grand Canyon tidak hanya mengalami pelipatan di dalam ngarai itu sendiri. Lapisan-lapisan tersebut juga telah terlipat ke arah timur pada Monoklin Kaibab Timur (East Kaibab Monocline) \cite{46}, dan juga ke utara di Cedar Breaks, Utah (Gambar \ref{fig:4}). Ini menunjukkan bahwa lapisan-lapisan tersebut kemungkinan besar terlipat bersamaan setelah saling tertumpuk di atas satu sama lain dalam waktu yang relatif singkat. Sebagai referensi, lapisan-lapisan horizontal Grand Canyon memiliki ketebalan sekitar 1700 meter. Dibutuhkan proses geologis dengan skala yang luar biasa besar untuk dapat menumpuk lapisan sedimen hingga setebal satu mil.

Pembentukan Grand Canyon yang sesungguhnya juga merupakan isu lain yang menjadi perdebatan dalam geologi modern. Geologi dengan prinsip uniformitarianisme mengemukakan bahwa Grand Canyon dibentuk oleh Sungai Colorado selama jutaan tahun \cite{47}. Namun, tim peneliti Answers in Genesis percaya bahwa Grand Canyon kemungkinan besar terbentuk hanya dalam waktu beberapa minggu oleh erosi saluran air (spillway erosion) dari danau purba yang jebol. Peristiwa ini mengikis sedimen dalam jumlah yang sangat besar saat membentuk ngarai tersebut. Ada bukti keberadaan danau di dataran tinggi di sebelah timur Grand Canyon, dengan melihat endapan sedimen danau dan fosil laut di lokasi itu. Membandingkan Grand Canyon dengan contoh erosi saluran air berskala besar lainnya, seperti Ngarai Afton dan Gunung St. Helens, menunjukkan topografi yang serupa, dan memperlihatkan bahwa ngarai-ngarai besar dapat tercipta dengan cepat oleh aliran air yang besar \cite{48}.

Dengan mempertimbangkan besarnya proses geologis yang diperlukan untuk menumpuk sedimen di atas tanah seluas itu, ditambah adanya gaya tektonik besar tak lama setelah lapisan sedimen tersebut tertumpuk, serta kecilnya ukuran Sungai Colorado bila dibandingkan dengan ukuran Grand Canyon yang luar biasa besar, tampaknya ngarai tersebut tidak mungkin terbentuk secara bertahap.

\section{Kota Bawah Tanah Derinkuyu}

Selain piramida, salah satu contoh luar biasa dari teknik pembangunan kuno adalah kota bawah tanah Derinkuyu (Gambar \ref{fig:5}), yang terletak di Kapadokia, Turki. Struktur ini adalah yang terbesar di antara lebih dari 200 tempat perlindungan bawah tanah di wilayah tersebut \cite{54}. Kota bawah tanah ini diperkirakan pernah menampung hingga 20.000 orang dan memiliki 18 lantai, dengan kedalaman mencapai 85 meter. Walaupun usianya belum pasti, diperkirakan kota ini berumur setidaknya 2800 tahun. Kota ini dipahat dari batuan vulkanik lunak \cite{52, 53}.

\begin{figure}[b]
\begin{center}
% \fbox{\rule{0pt}{2in} \rule{0.9\linewidth}{0pt}}
   \includegraphics[width=1\linewidth]{derinkuyu.jpeg}
\end{center}
   \caption{Diagram kota bawah tanah Derinkuyu \cite{56}.}
\label{fig:5}
\label{fig:onecol}
\end{figure}

Derinkuyu menjadi menarik karena sangat tidak lazim ada sebuah komunitas yang membangun seluruh kota mereka di bawah tanah. Untuk membuat ruang hidup di bawah tanah, setiap ruang harus dipahat dari batu. Bentuk kasar dan tekstur dari terowongan bawah tanah di sini menunjukkan bahwa tempat ini dipahat dengan tenaga kerja manual, bukan dengan alat berat, yang akan jauh lebih sulit dibandingkan membangun tempat tinggal di atas tanah. Faktanya, sulit memikirkan alasan mengapa ada manusia yang ingin tinggal secara permanen di bawah tanah selama hidupnya, apabila pertanian, sinar matahari, alam, dan kesempatan menjelajah ke tempat-tempat lain hanya ada di permukaan. "Sejarah" konvensional mengusulkan bahwa Derinkuyu dibuat oleh orang-orang Kristen yang membutuhkan tempat terlindung untuk menjalankan agama mereka \cite{53}. Tetapi dengan menggunakan akal sehat kita dapat menyimpulkan bahwa cara paling sederhana untuk menghadapi musuh adalah "lawan atau kabur", bukan malah "memahat kota bawah tanah dari batu".

Skala, kedalaman, dan perancangan kota bawah tanah yang begitu matang menunjukkan bahwa tempat ini tidak didesain sebagai struktur pertahanan militer sementara untuk dapat melawan musuh saat perang, melainkan sebagai tempat perlindungan jangka panjang untuk melindungi dari ancaman mematikan di permukaan. Derinkuyu dilengkapi tidak hanya dengan kamar tidur, dapur, dan kamar mandi sederhana, tetapi juga kandang hewan, tangki air, penyimpanan makanan, alat pemeras anggur dan minyak, sekolah, kapel, makam, serta lorong ventilasi raksasa (Gambar \ref{fig:6}). Mengapa tempat perlindungan militer membutuhkan alat pemeras anggur dan harus digali sedalam 85 meter dengan tingkat kompleksitas seperti itu?

Penjelasan yang paling masuk akal untuk pembangunan Derinkuyu adalah kebutuhan mendesak untuk mempersiapkan tempat perlindungan jangka panjang yang mandiri guna melindungi dari kekuatan geofisika katastropik di permukaan Bumi.

\begin{figure}[t]
\begin{center}
% \fbox{\rule{0pt}{2in} \rule{0.9\linewidth}{0pt}}
   \includegraphics[width=1\linewidth]{derinkuyu-air.jpg}
\end{center}
   \caption{Sebuah sumur ventilasi yang dalam di Derinkuyu \cite{53}.}
\label{fig:6}
\label{fig:onecol}
\end{figure}

% \section{Additional Anomalies Best Explained By An Earth Flip}

% Before wrapping up, we will mention some additional scientific anomalies that, once viewed in the context of cataclysmic geophysical forces, are well explained.
\section{Akumulasi Biomassa}

Campuran biomassa dari berbagai jenis hewan dan tumbuhan, yang sering ditemukan dalam bentuk fosil di lapisan sedimen, merupakan anomali yang membingungkan. Dalam "Reliquoæ Diluvianæ", Pendeta William Buckland merinci temuan berbagai spesies fauna yang tidak diketahui dengan jelas mengapa dapat ditemukan bersama-sama, tersebar di seluruh Inggris dan Eropa, terkubur dalam lapisan sedimen 'diluvium' \cite{58}. Campuran sisa-sisa hewan seperti ini juga ditemukan di Gua Skjonghelleren di pulau Valdroy, Norwegia. Di gua ini, lebih dari 7.000 tulang mamalia, burung, dan ikan ditemukan tercampur di beberapa lapisan sedimen \cite{59}. Contoh lain adalah San Ciro, "Gua Para Raksasa", di Italia. Di gua ini, beberapa ton tulang mamalia, terutama kuda nil, ditemukan dalam kondisi begitu segar hingga dipotong untuk dijadikan ornamen dan dikirim untuk pembuatan arang lampu. Tulang-tulang dari berbagai hewan tersebut dilaporkan tercampur, patah, hancur, dan tersebar dalam bentuk fragmen \cite{60,61}. Di Mendes Kuno, Mesir, campuran berbagai spesies tulang hewan ditemukan bercampur dengan tanah liat vitreous (kaca) \cite{57}. Temuan seperti ini mungkin tampak membingungkan, tetapi mudah dijelaskan oleh banjir besar yang membenamkan campuran hewan mati dalam lapisan sedimen, menghanyutkan hewan ke dalam gua atau mengubur mereka hidup-hidup dalam gua, dan dalam hal biomassa ter-vitrifikasi di Mesir, pelepasan tenaga listrik besar pasca-banjir akibat perpindahan inti-mantel. Gambar \ref{fig:7} memperlihatkan paparan khas 'muck' biomassa Alaska \cite{56}.

\begin{figure}[t]
\begin{center}
% \fbox{\rule{0pt}{2in} \rule{0.9\linewidth}{0pt}}
   \includegraphics[width=1\linewidth]{muck-crop.jpeg}
\end{center}
   \caption{'Muck' Alaska, terdiri dari fragmen pohon, tumbuhan, dan hewan yang tersebar dengan kacau dalam lumpur beku dan es \cite{146}.}
\label{fig:7}
\label{fig:onecol}
\end{figure}

\section{Bunker Kuno}

Nenek moyang kita meninggalkan banyak struktur kuno yang dengan tingkat perancangan yang tinggi, di mana di dalamnya telah ditemukan sisa-sisa manusia. Banyak orang mengira bahwa struktur-struktur ini adalah makam yang kompleks, namun jika diamati lebih dekat, struktur-struktur ini mungkin sebenarnya merupakan bunker-bunker kuno.

\begin{figure}[b]
\begin{center}
% \fbox{\rule{0pt}{2in} \rule{0.9\linewidth}{0pt}}
   \includegraphics[width=1\linewidth]{ww19.jpg}
\end{center}
   \caption{Newgrange, Irlandia - lihat pengunjung di pintu masuk untuk perbandingan.}
\label{fig:8}
\label{fig:onecol}
\end{figure}

Salah satu contoh yang sangat baik adalah Newgrange (Gambar \ref{fig:8}), monumen utama di kompleks Brú na Bóinne, yaitu sebuah kumpulan struktur kuno, termasuk apa yang disebut kuburan-kuburan lorong. Kuburan-kuburan ini terdiri dari satu atau lebih ruang pemakaman yang ditutupi tanah atau batu dan memiliki lorong akses sempit yang terbuat dari batu-batu besar \cite{70}. Ini adalah contoh rekayasa bangunan terlindung yang kompleks, yang dibangun selama beberapa generasi, konon untuk menguburkan segelintir orang, yang bahkan belum lahir saat pembangunan makam dimulai. Ketika ditemukan kembali oleh seorang pemilik tanah lokal pada tahun 1699, kuburan tersebut tertimbun tanah.

Dengan melihat sekilas struktur ini, kita dapat melihat usaha besar yang dilakukan dalam pembangunannya - Newgrange terdiri dari sekitar 200.000 ton material. Di dalamnya, \textit{“…ada sebuah lorong berbilik, yang dapat diakses melalui pintu masuk di sisi tenggara monumen. Lorong ini memiliki panjang 19 meter (60 kaki), atau sekitar sepertiga dari jalan menuju pusat struktur. Di ujung lorong terdapat tiga bilik kecil yang bercabang dari sebuah ruang tengah yang lebih besar dengan atap kubah corbel… Dinding lorong ini terdiri dari lempengan-lempengan batu besar yang disebut ortostat, dua puluh dua di antaranya di sisi barat dan dua puluh satu di sisi timur. Tinggi rata-ratanya 1½ meter”} \cite{70}. Ada juga detail rancangan kedap air yang rumit. Misal pada atapnya, \textit{“Celah-celah di atap makam disumbat dengan campuran tanah bakar dan pasir laut agar kedap air. Dari campuran ini, diperoleh dua hasil penanggalan radiokarbon yang menunjukkan usia sekitar tahun 2500 SM untuk struktur makam tersebut.”} \cite{71}. Selain itu, kenaikan tanah menuju ruang dalam mungkin juga dibuat untuk tujuan serupa: \textit{“Karena lantai lorong dan ruang makam mengikuti kenaikan tanah dari bukit tempat monumen itu dibangun, terdapat perbedaan hampir 2 meter pada tinggi lantai antara pintu masuk dengan bagian dalam ruangannya”} \cite{71}.

\begin{figure}[b]
\begin{center}
% \fbox{\rule{0pt}{2in} \rule{0.9\linewidth}{0pt}}
   \includegraphics[width=1\linewidth]{dolmen.jpg}
\end{center}
   \caption{Dolmen de Soto, Spanyol \cite{53}.}
\label{fig:9}
\label{fig:onecol}
\end{figure}

Jumlah sisa-sisa manusia yang sangat sedikit di dalamnya juga menjadi hal yang menarik. Hasil penggalian menunjukkan fragmen-fragmen tulang yang hanya berasal dari beberapa orang, sebagian terbakar dan yang lain tidak, yang tersebar di sepanjang lorong. Pembangunan Newgrange diperkirakan memakan waktu setidaknya beberapa generasi berdasarkan penanggalan karbon dari material di dalamnya. Mengapa sebuah komunitas kuno bersusah payah untuk membangun makam yang besar dengan desain yang rumit, jika tujuannya hanya untuk menyebarkan potongan tulang dari beberapa orang yang telah meninggal di lorongnya? Jauh lebih masuk akal jika struktur megalitikum kuno yang sangat kedap air ini justru dibangun sebagai tempat perlindungan untuk melindungi orang-orang selama terjadinya bencana alam di muka Bumi.

Di Huelva, Spanyol bagian selatan, contoh serupa adalah Dolmen de Soto (Gambar \ref{fig:9}), salah satu dari sekitar 200 situs serupa di daerah tersebut \cite{72,32}. Bangunan ini berstruktur mulus yang dibangun dengan teknik desain yang tinggi menggunakan batu megalitikum dengan diameter 75 meter. Dilaporkan bahwa hanya ada delapan jasad yang ditemukan saat penggalian, semuanya dikuburkan dalam posisi janin.

\section{Penyebutan Anomali Penting}

Pada bagian ini, saya secara singkat menyebutkan beberapa anomali penting lainnya, yang semuanya bisa dijelaskan dengan baik oleh bencana besar seperti ECDO.

\subsection{Anomali Biologis}

\begin{figure}[t]
\begin{center}
% \fbox{\rule{0pt}{2in} \rule{0.9\linewidth}{0pt}}
   \includegraphics[width=1\linewidth]{bottleneck.jpg}
\end{center}
   \caption{Sebuah hambatan genetik (genetic bottleneck) yang menyatakan kemusnahan 95\% laki-laki sekitar 6.000 tahun yang lalu \cite{62}.}
\label{fig:10}
\label{fig:onecol}
\end{figure}

Beberapa anomali biologis yang perlu diperhatikan antara lain hambatan genetik (genetic bottleneck) dan fosil-fosil paus yang ditemukan di daratan. Zeng et al. (2018) memodelkan 125 sekuens kromosom-Y dari manusia modern, dan berdasarkan kesamaan serta mutasi DNA, mengidentifikasi adanya hambatan populasi yang menyebabkan penurunan populasi laki-laki sebesar 95\% sekitar 5.000 hingga 7.000 tahun yang lalu (Gambar \ref{fig:10}) \cite{62}. Fosil paus telah ditemukan ratusan meter di atas permukaan laut, di Swedenborg, Michigan, Vermont, Kanada, Chile, dan Mesir \cite{63,64,65,66}. Fosil-fosil paus ini ditemukan dalam berbagai kondisi: ada yang terawetkan dengan sempurna, ada yang ditemukan di rawa-rawa di atas endapan es, atau terkubur dalam sedimen. Jumlah spesimen di lokasi-lokasi ini berkisar dari beberapa ekor hingga lebih dari seratus ekor. Paus adalah makhluk laut dalam dan jarang mendekat ke pantai. Bagaimana mungkin paus-paus ini terdampar di ketinggian setinggi itu, yang sering kali sangat jauh dari laut?

Sepanjang sejarah geologis Bumi, telah banyak kepunahan massal yang terjadi, dan yang paling banyak diteliti adalah “Lima Besar” peristiwa Fanerozoikum: kepunahan massal Ordovisium akhir, Devonian akhir, akhir Permian, akhir Trias, dan akhir Kapur \cite{88,89}. Menariknya, beberapa dari peristiwa kepunahan ini diklasifikasikan terjadi pada periode sejarah yang sama dengan banyak lapisan Grand Canyon, yaitu lapisan Permian dan Devonian.

\subsection{Anomali Fisik}

\begin{figure}[b]
\begin{center}
% \fbox{\rule{0pt}{2in} \rule{0.9\linewidth}{0pt}}
   \includegraphics[width=1\linewidth]{columbia.jpg}
\end{center}
   \caption{Riak arus besar di Danau Glasial Columbia, negara bagian Washington \cite{80}.}
\label{fig:11}
\label{fig:onecol}
\end{figure}

Ada banyak bentang alam selain Grand Canyon yang kemungkinan besar terbentuk akibat kekuatan katastropik atau bencana besar. Bukti adanya aliran air antar benua yang besar dapat ditemukan pada riak raksasa di seluruh dunia. Salah satu contohnya adalah Channeled Scablands di wilayah barat laut Pasifik. Di sini, kita tidak hanya dapat melihat bentang alam hasil endapan sedimen dan bongkahan batu-batu tak beraturan, tetapi juga lebih dari seratus deret riak besar yang terbentuk dari aliran arus raksasa\cite{78,79}. Ini adalah versi yang lebih besar dari riak pasir yang terbentuk di dasar pasir sungai. Riak besar semacam ini dapat ditemukan di seluruh dunia di Prancis, Argentina, Rusia, dan Amerika Utara \cite{81}. Gambar \ref{fig:11} menunjukkan riak-riak yang terletak di negara bagian Washington di Amerika Serikat \cite{80}.
\begin{figure}[b]
\begin{center}
% \fbox{\rule{0pt}{2in} \rule{0.9\linewidth}{0pt}}
   \includegraphics[width=1\linewidth]{zhangjiajie.jpg}
\end{center}
   \caption{Pilar batu besar di Hutan Nasional Zhangjiajie, Tiongkok selatan.}
\label{fig:12}
\label{fig:onecol}
\end{figure}

\begin{figure}[b]
\begin{center}
% \fbox{\rule{0pt}{2in} \rule{0.9\linewidth}{0pt}}
   \includegraphics[width=1\linewidth]{hoy.jpg}
\end{center}
   \caption{Pilar laut Old Man of Hoy, Skotlandia \cite{83}.}
\label{fig:13}
\label{fig:onecol}
\end{figure}

Struktur erosi daratan juga dijelaskan dengan baik oleh pembalikan Bumi seperti ECDO. Tiongkok bagian selatan adalah contoh hebat dari bentang alam karst yang sangat luas, yang terbentuk melalui erosi air \cite{82}. Bentang alam ini mencakup karst menara, karst runcing, karst berbentuk kerucut, jembatan alami, ngarai, sistem gua besar, dan lubang runtuhan. Salah satu yang paling mencolok adalah Hutan Nasional Zhangjiajie, yang memiliki pilar-pilar besar dari batu pasir kuarsa (Gambar \ref{fig:12}) \cite{84}. Pilar-pilar ini berdiri pada ketinggian rata-rata lebih dari 1.000 meter dan jumlahnya lebih dari 3.100. Lebih dari 1.000 di antaranya menjulang lebih dari 120 meter, dan 45 di antaranya mencapai lebih dari 300 meter \cite{85}. Pilar-pilar ini menyerupai pilar erosi laut (Gambar \ref{fig:13}), yaitu pilar batu pantai yang terbentuk dari runtuhnya material di sekitarnya akibat gelombang laut. Bentang alam erosi yang serupa juga ditemukan di batu kerucut Urgup, Turki, serta Ciudad Encantada, Spanyol, yang keduanya berada lebih dari 1.000 meter di atas permukaan laut. Di sekitar lokasi-lokasi ini ditemukan garam dan fosil laut, yang menandakan adanya gelombang laut di masa lalu \cite{15,86,87}. Tentu saja, cerita-cerita banjir besar \cite{3} menyebutkan bahwa tinggi permukaan laut pernah mencapai lebih dari 1.000 meter, dan hal ini didukung oleh keberadaan air asin dan dataran garam besar di Andes dan Himalaya yang berada beberapa kilometer di atas permukaan laut. Dataran garam Uyuni di Bolivia, misalnya, mencapai ketinggian 3.653 meter di atas permukaan laut \cite{94}.


\subsection{Peristiwa Perubahan Iklim yang Cepat}

Literatur ilmiah modern mengakui keberadaan peristiwa perubahan iklim global yang sangat cepat yang dapat dikatakan baru saja terjadi dalam sejarah Bumi. Dua contoh penting adalah peristiwa 4,2 ribu tahun lalu dan 8,2 ribu tahun lalu, yang keduanya bertepatan dengan penurunan populasi dan gangguan pemukiman masyarakat di wilayah geografis yang luas. Peristiwa ini terekam sebagai anomali dalam sedimen dan inti es, fosil karang, nilai isotop O18, catatan serbuk sari dan speleotem, serta data ketinggian air laut. Perubahan iklim yang dicatat mencakup penurunan suhu global secara cepat, pengeringan, gangguan arus balik meridional Atlantik, dan perkembangan gletser \cite{90,91,92}. Peristiwa 8,2 ribu tahun lalu secara khusus bertepatan dengan banjir besar air asin di Laut Hitam yang kemungkinan pernah terjadi sekitar tahun 6400 SM \cite{93}.

\subsection{Anomali Arkeologis}

Bukti arkeologis dari beberapa kota kuno menunjukkan banyak lapisan yang memiliki bukti-bukti penguburan dan kehancuran, sebagai catatan peristiwa bencana di masa lalu. Kota kuno Yerikho adalah salah satu contohnya, yang sekarang terletak di Palestina. Kota ini memiliki beberapa lapisan kehancuran, dengan runtuhnya bangunan batu dan kebakaran yang hebat \cite{96,97}. Kronologi yang tercatat dalam lapisan-lapisannya bermula dari sekitar 9000 SM hingga 2000 SM. Secara khusus yang paling menonjol adalah menaranya, yang tampaknya telah terpotong dan terkubur dalam sedimen sekitar tahun 7400 SM (Gambar \ref{fig:14}) \cite{95}. Catal Huyuk \cite{99}, Gramalote \cite{98}, dan istana Minoa di Knossos, Kreta \cite{100,101} adalah contoh situs arkeologi serupa yang memiliki banyak lapisan, sering kali memiliki bukti-bukti kehancuran.

\begin{figure}[t]
\begin{center}
% \fbox{\rule{0pt}{2in} \rule{0.9\linewidth}{0pt}}
   \includegraphics[width=1\linewidth]{jericho.jpg}
\end{center}
   \caption{Rekonstruksi arkeologis penguburan Menara Yerikho sekitar tahun 7400 SM \cite{95}.}
\label{fig:14}
\label{fig:onecol}
\end{figure}

Bukti lain tentang bencana besar yang mengganggu peradaban manusia adalah Boneka Nampa, sebuah boneka tanah liat yang ditemukan di bawah sekitar 100 meter lava di Idaho \cite{102,103}. Aliran lava di mana patung kecil itu ditemukan diperkirakan terbentuk pada akhir periode Tersier atau awal periode Kuarter, yang diperkirakan berusia sekitar 2 juta tahun. Namun, aliran lava di wilayah tersebut tampak relatif baru. Penemuan semacam ini tidak hanya menunjukkan adanya bencana besar yang menghancurkan peradaban, tetapi juga mempertanyakan keakuratan kronologi penanggalan modern.

\section{Mengenai Metode Penanggalan Modern}

Ada alasan kuat untuk bersikap kritis terhadap kronologi modern, yang menentukan usia yang sangat tua untuk berbagai jenis benda fisik, bahkan hingga mencapai jutaan atau ratusan juta tahun.

Narasi konvensional menyatakan bahwa apa yang disebut "bahan bakar fosil" seperti batu bara, minyak, dan gas alam berusia ratusan juta tahun \cite{104}. Namun, penanggalan karbon dari minyak di Teluk Meksiko menentukan usia sekitar 13.000 tahun untuk minyak tersebut \cite{105}. Karbon-14 memiliki waktu paruh yang sangat singkat (5.730 tahun) sehingga seharusnya sudah benar-benar terurai setelah beberapa ratus ribu tahun. Namun, karbon-14 telah terbukti ditemukan dalam batu bara dan fosil yang seharusnya seribu kali lebih tua dari itu\cite{106}. Bahkan, batu bara buatan telah diproduksi di laboratorium dalam kondisi terkendali, terutama dengan panas tinggi, hanya dalam waktu 2-8 bulan \cite{107}.

Metode penanggalan radioisotop selain penanggalan karbon juga mungkin memiliki ketidakakuratan. Kelompok riset Answers in Genesis menemukan ketidakkonsistean pada tanggal yang dihasilkan oleh metode tersebut sehingga kredibilitasnya dipertanyakan \cite{108}. Jaringan lunak yang mengandung sel darah, pembuluh, dan kolagen bahkan telah ditemukan dalam sisa-sisa dinosaurus yang seharusnya berusia seratus juta tahun \cite{109,110}. Berdasarkan apa yang kita ketahui, sangat mungkin bahwa usia yang diterima secara konvensional untuk skala waktu geologi Bumi dan benda fisik seperti batuan dan bahan bakar fosil mungkin meleset sangat jauh dari kenyataannya.

\section{Kesimpulan}

Dalam makalah ini, saya telah memaparkan anomali-anomali paling meyakinkan yang mengarah pada asal-usul katastropik dan paling masuk akal jika dijelaskan melalui peristiwa pembalikan Bumi ECDO. Meskipun anomali yang disajikan cukup beragam, koleksi yang disajikan ini belum lengkap - lebih banyak lagi anomali yang telah dikompilasi dan tersedia secara publik di repositori riset GitHub saya \cite{2}.

\section{Ucapan Terima Kasih}

Terima kasih kepada Ethical Skeptic, penulis asli dari tesis ECDO, karena telah menyelesaikan tesisnya yang penuh wawasan dan terobosan serta membagikannya kepada dunia. Tesisnya yang terdiri dari tiga bagian \cite{1} tetap menjadi rujukan utama untuk teori Osilasi Dzhanibekov Pelepasan Inti-Mantel Eksotermik (ECDO), dan memuat jauh lebih banyak informasi tentang topik ini daripada ringkasan singkat saya ini.

Dan tentu saja, terima kasih kepada orang-orang besar yang telah berjasa; orang-orang yang telah melakukan semua penelitian dan penyelidikan yang memungkinkan karya ini dapat ditulis dan telah berupaya membawa pencerahan bagi umat manusia.

\clearpage
\twocolumn

{\small
\bibliographystyle{ieee}
\bibliography{egbib}
}

\end{document}
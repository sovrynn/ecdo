\documentclass[10pt,twocolumn,letterpaper]{article}

% Đồ của tôi
\usepackage{booktabs}
% \usepackage{caption}
% \captionsetup[table]{skip=8pt}   % Chỉ ảnh hưởng đến bảng
\usepackage{stfloats}  % Thêm cái này vào phần tiền đề
\usepackage[T5]{fontenc}

% \usepackage{fontspec}
\usepackage[english]{babel}

% load Lao via babelprovide, turn on "onchar=ids" for automatic shaping
\babelprovide[import,onchar=ids fonts]{vietnamese}

% main (rm) font for Latin
\babelfont{rm}{Noto Serif}

% Lao text in Noto Serif Lao at 1.2× scale
\babelfont[vietnamese]{rm}{Noto Serif}
\babelfont[vietnamese]{sf}{Noto Serif}

% alternate (sans-serif) font for Latin
\babelfont{alt}{Lato}

% Lao text in Noto Serif Lao for the alt family too
\babelfont[vietnamese]{alt}{Noto Serif}

\usepackage{cvpr}
\usepackage{times}
\usepackage{epsfig}
\usepackage{graphicx}
\usepackage{amsmath}
\usepackage{amssymb}

% \makeatletter
% \def\cvprsubsection{\@startsection {subsection}{2}{\z@}
%     {8pt plus 2pt minus 2pt}{6pt}{\bfseries\normalsize}}
% \makeatother

% Bao gồm các gói khác tại đây, trước hyperref.

% Nếu bạn bình luận hyperref rồi sau đó bỏ bình luận nó, bạn nên xóa
% egpaper.aux trước khi chạy lại latex.  (Hoặc chỉ cần nhấn 'q' ở lần chạy latex đầu tiên, để nó hoàn thành, và bạn sẽ ổn).

\usepackage[breaklinks=true,bookmarks=false]{hyperref}

\cvprfinalcopy % *** Bỏ chú thích dòng này cho bản nộp cuối cùng

\def\cvprPaperID{****} % *** Nhập mã số bài báo CVPR ở đây
\def\httilde{\mbox{\tt\raisebox{-.5ex}{\symbol{126}}}}

\renewcommand{\figurename}{Hình}   % or whatever you like instead of "Hình"

% This makes the font slightly bigger than base (10) and bold in Subsection headings rather than using ptmb
% after those:
\addto\captionsenglish{%
  \renewcommand{\figurename}{Hình}%
}

\makeatletter
\def\abstract{%
  \centerline{\large\bf Tóm tắt}% <-- your new label
  \vspace*{12pt}%
  \it%
}
\makeatother

\makeatletter
\def\cvprsubsection{%
  \@startsection{subsection}{2}{\z@}%
    {8pt plus 2pt minus 2pt}{6pt}%
    % {\normalfont\bfseries\selectfont}%
    {\normalfont\bfseries\fontsize{11}{13}\selectfont}%
}
\makeatother

% So this hardcodes the style for the numbers in the section/subsection headings so they're bold
\font\elvbf=ptmb scaled 1100
\font\elvbfs=ptmb scaled 1200
\makeatletter
% Section number: Large + bold
\renewcommand\thesection{%
  {\elvbfs\arabic{section}}%
}

% Subsection number: normalsize + bold + custom punctuation
\renewcommand\thesubsection{%
  {\elvbf
   \arabic{section}.\arabic{subsection}}%
}
\makeatother

% Các trang được đánh số trong chế độ nộp bài, và không đánh số trong bản camera-ready
%\ifcvprfinal\pagestyle{empty}\fi
\setcounter{page}{1}
\begin{document}

%%%%%%%%% TIÊU ĐỀ
\title{Tài liệu hướng dẫn dựa trên dữ liệu về ECDO - Phần 2/2: Khảo cứu các hiện tượng bất thường trong khoa học và lịch sử được lý giải thuyết phục nhất bằng lý thuyết “Trái Đất lật” ECDO}

\author{Junho\\
Xuất bản tháng 2 năm 2025\\
Trang web (Tải bài tại đây): \href{https://sovrynn.github.io}{sovrynn.github.io}\\
Kho Nghiên Cứu ECDO: \href{https://github.com/sovrynn/ecdo}{github.com/sovrynn/ecdo}\\
{\tt\small junhobtc@proton.me}
% Đối với một bài báo mà tất cả tác giả đều thuộc cùng một tổ chức,
% hãy bỏ qua các dòng sau cho đến dấu ``}'' kết thúc.
% Các tác giả và địa chỉ bổ sung có thể được thêm bằng ``\and'',
% giống như tác giả thứ hai.
% Để tiết kiệm không gian, chỉ sử dụng địa chỉ email hoặc trang chủ, không cần cả hai
% \and
% xx
% Tổ chức2\\
% Dòng đầu tiên của địa chỉ tổ chức2\\
% {\tt\small secondauthor@i2.org}
}

\maketitle
%\thispagestyle{empty}

%%%%%%%%% TÓM TẮT
\begin{abstract}
Tháng 5 năm 2024, một tác giả ẩn danh trên mạng có bút danh là “The Ethical Skeptic” \cite{0} đã đăng một lý thuyết đột phá có tên là Hiệu ứng Dzhanibekov do quá trình tách rời tỏa nhiệt giữa lớp lõi và lớp manti của Trái Đất (ECDO — Exothermic Core-Mantle Decoupling Dzhanibekov Oscillation) \cite{1}. Lý thuyết này cho rằng Trái Đất từng trải qua những thay đổi đột ngột và thảm khốc trong trục quay, dẫn đến các trận đại hồng thủy toàn cầu khi đại dương, do quán tính quay, tràn vào lục địa. Ngoài ra, lý thuyết còn trình bày các quá trình địa vật lý giải thích mang tính nhân quả, kèm theo dữ liệu cho thấy lần "Trái Đất lật" tiếp theo có thể sắp xảy ra. Mặc dù các dự đoán về lũ lụt thảm họa và ngày tận thế không phải là điều mới, nhưng lý thuyết ECDO lại đặc biệt có tính thuyết phục nhờ cách tiếp cận mang tính khoa học, hiện đại, đa ngành và dựa trên dữ liệu.

Bài nghiên cứu này là phần thứ hai trong bản tóm tắt súc tích gồm hai phần của 6 tháng nghiên cứu độc lập \cite{2,20} về lý thuyết ECDO, tập trung cụ thể vào những bất thường khoa học và lịch sử được giải thích tốt nhất bằng một sự kiện "Trái Đất lật" thảm khốc theo lý thuyết ECDO.

\end{abstract}

%%%%%%%%% NỘI DUNG CHÍNH

\section{Giới thiệu}

Lịch sử và địa chất hiện đại theo trường phái đồng nhất luận cho rằng các cảnh quan địa chất lớn như hẻm núi Grand Canyon được hình thành trong hàng triệu năm \cite{143}; rằng muối tồn tại ở Thung lũng Chết (California) vì nơi đây từng nằm dưới đại dương cách đây hàng trăm triệu năm \cite{144}; rằng tổ tiên của chúng ta cách đây 150 thế hệ đã dành cả đời để xây các lăng mộ khổng lồ \cite{29,70}; và rằng các lớp đá gọi là "nhiên liệu hóa thạch" đã có tuổi đời hàng trăm triệu năm \cite{104}. Có lẽ điều gây tò mò nhất là con người được cho là đã xuất hiện từ 300.000 năm trước \cite{145}, nhưng lịch sử ghi chép và nền văn minh chỉ mới có từ khoảng 5.000 năm trước — tương đương với 150 thế hệ người.

Những điểm bất thường như vậy, như chúng ta sẽ thấy, được giải thích rõ nhất bởi các lực địa chất mang tính thảm họa.

\section{Voi ma mút bị đóng băng chớp nhoáng chôn trong bùn}

\begin{figure}[b]
\begin{center}
% \fbox{\rule{0pt}{2in} \rule{0.9\linewidth}{0pt}}
   \includegraphics[width=1\linewidth]{jarkov-mammoth.jpg}
\end{center}
   \caption{Voi ma mút Jarkov, một cá thể voi ma mút Siberia 20.000 năm tuổi được bảo quản hoàn hảo trong lớp bùn đóng băng \cite{51}.}
\label{fig:1}
\label{fig:onecol}
\end{figure}

Một trong những dạng bất thường điển hình là những con voi ma mút được bảo quản hoàn hảo, bị đông lạnh và vùi trong bùn, thường được tìm thấy ở các vùng Bắc Cực (Hình \ref{fig:1}). Con voi ma mút Beresovka, được phát hiện ở Siberia bị chôn vùi trong sỏi bùn, đã được bảo quản gần như hoàn hảo đến mức thịt của nó vẫn có thể ăn được sau hàng ngàn năm kể từ khi chết. Thậm chí nó còn có thức ăn thực vật trong miệng và dạ dày, khiến các nhà khoa học bối rối khi không hiểu làm sao nó có thể bị đóng băng nhanh đến vậy nếu như vẫn đang gặm cỏ cây hoa ngay trước khi chết \cite{17}. Theo báo cáo, \textit{"Năm 1901 đã có một chấn động khi phát hiện xác một con voi ma mút nguyên vẹn ở gần sông Berezovka, vì dường như con vật này đã chết vì lạnh giữa mùa hè. Những gì còn lại trong dạ dày của nó được bảo quản rất tốt, có cả hoa bơ và đậu dại đang ra hoa: Nghĩa là những thứ này phải được nuốt vào khoảng cuối tháng 7 hoặc đầu tháng 8. Con vật đã chết đột ngột đến mức vẫn còn giữ trong miệng một mớ cỏ và hoa dại. Rõ ràng nó đã bị cuốn đi bởi một lực cực kỳ mạnh và bị hất văng xa hàng dặm khỏi bãi cỏ của mình. Xương chậu và một chân bị gãy — con vật khổng lồ này đã bị quật ngã rồi chết và bị đóng băng vì lạnh, vào đúng thời điểm mà lẽ ra là nóng nhất trong năm"} \cite{18}. Thêm vào đó, \textit{"[Các nhà khoa học Nga] ghi nhận rằng ngay cả lớp niêm mạc trong cùng của dạ dày nó cũng có cấu trúc sợi được bảo quản hoàn hảo, điều này cho thấy thân nhiệt của nó đã bị loại bỏ bởi một quá trình vô cùng khủng khiếp trong tự nhiên. Sanderson, đặc biệt chú ý đến chi tiết này, đã đem vấn đề đến Viện Công nghiệp Thực phẩm Đông lạnh Hoa Kỳ: Cần điều kiện gì để làm đông lạnh toàn bộ một con voi ma mút tới mức mà hàm lượng độ ẩm ở ngay cả phần bên trong sâu nhất của cơ thể, ngay cả niêm mạc dạ dày, cũng không có đủ thời gian để tạo thành các tinh thể băng lớn làm phá vỡ cấu trúc sợi của thịt?... Vài tuần sau, Viện đã trả lời Sanderson rằng: Điều đó hoàn toàn bất khả thi. Với tất cả kiến thức khoa học và kỹ thuật hiện đại, hoàn toàn không có cách nào để loại bỏ nhiệt ra khỏi một xác chết lớn như voi ma mút đủ nhanh để làm đông nó mà không hình thành các tinh thể băng lớn trong thịt. Hơn nữa, sau khi kiểm tra mọi kỹ thuật khoa học đông lạnh hiện có rồi so sánh với tự nhiên, họ kết luận rằng không có quá trình tự nhiên nào có thể làm được điều này"} \cite{19}.

\section{Hẻm núi Grand Canyon}

Hẻm núi Grand Canyon, thuộc Lưu vực Lớn ở tây nam Bắc Mỹ, là một hiện tượng tự nhiên khác cho thấy nguồn gốc có tính thảm hoạ (Hình \ref{fig:2}). Đầu tiên, các lớp đá trầm tích sa thạch và đá vôi tạo nên Grand Canyon trải dài trên diện tích khổng lồ lên tới 2,4 triệu km$^2$ \cite{21}. Hình \ref{fig:3} minh họa phạm vi của lớp sa thạch Coconino ở khắp miền tây Hoa Kỳ. Những lớp ngang khổng lồ có vật chất đồng nhất như vậy chỉ có thể được hình thành trong cùng một thời điểm.

\begin{figure}[t]
\begin{center}
% \fbox{\rule{0pt}{2in} \rule{0.9\linewidth}{0pt}}
   \includegraphics[width=1\linewidth]{grand-canyon.jpg}
\end{center}
   \caption{Hẻm núi Grand Canyon, tại Arizona, Hoa Kỳ \cite{49}.}
\label{fig:2}
\label{fig:onecol}
\end{figure}

\begin{figure}[t]

\begin{center}
% \fbox{\rule{0pt}{2in} \rule{0.9\linewidth}{0pt}}
   \includegraphics[width=1\linewidth]{coconino.jpg}
\end{center}
   \caption{Kích thước của lớp đá sa thạch Coconino ở miền tây Hoa Kỳ \cite{21}.}
\label{fig:3}
\label{fig:onecol}
\end{figure}

Một cái nhìn kỹ hơn vào Hẻm núi Grand Canyon cho chúng ta biết rằng sự lắng đọng của những lớp trầm tích rộng lớn này cũng xảy ra đồng thời với các lực kiến tạo lớn. Để hiểu được điều này, chúng ta phải quan sát kỹ một số khu vực trong hẻm núi nơi các lớp trầm tích đã bị uốn nếp và lộ ra ngoài. Các nhà nghiên cứu của tổ chức Answers in Genesis \cite{42} đã quan sát các mẫu đá từ một số nếp uốn này dưới kính hiển vi, chẳng hạn như nếp uốn Monument, và dựa trên việc không có các đặc điểm thường xuất hiện nếu các nếp gấp này hình thành qua thời gian dài dưới nhiệt và áp suất, họ kết luận rằng các lớp trầm tích đã bị uốn nếp bởi lực kiến tạo khi chúng vẫn còn mềm, tức là ngay sau khi được lắng đọng \cite{43}.

\begin{figure*}
\begin{center}
% \fbox{\rule{0pt}{2in} \rule{.9\linewidth}{0pt}}
\includegraphics[width=1\textwidth]{Grand_Staircase-big.jpg}
\end{center}
   \caption{Các lớp trầm tích tạo nên Hẻm núi Grand Canyon (bên phải của hình) trải dài trực tiếp về phía bắc đến Cedar Breaks, Utah (bên trái của hình), nơi tất cả đều bị uốn nếp lên trên \cite{50}.}
\label{fig:4}
\end{figure*}

Khi nhìn tổng quát, chúng ta nhận thấy rằng các lớp đá tạo nên Hẻm núi Grand Canyon không chỉ bị uốn nếp bên trong hẻm núi. Các lớp này đã bị uốn nếp sang phía đông ở khu vực East Kaibab Monocline \cite{46}, nhưng cũng bị uốn nếp về phía bắc ở Cedar Breaks, Utah (Hình \ref{fig:4}). Điều này cho thấy có thể các lớp này đã bị uốn nếp cùng nhau sau khi chúng được xếp chồng lên nhau trong khoảng thời gian ngắn. Để tham khảo, các lớp nằm ngang của Grand Canyon dày khoảng 1700 mét. Quy mô của quá trình địa chất cần thiết để lắng đọng các lớp trầm tích dày gần 1 dặm là rất lớn.

Việc hình thành thực sự của Hẻm núi Grand Canyon là một vấn đề tranh cãi khác trong ngành địa chất hiện đại. Thuyết địa chất đồng nhất luận cho rằng Grand Canyon được tạo thành bởi sông Colorado trong hàng triệu năm \cite{47}. Tuy nhiên, nhóm nghiên cứu của Answers in Genesis tin rằng rất có thể Grand Canyon được hình thành chỉ trong vài tuần do hiện tượng xói mòn tràn từ một hồ cổ đại vượt qua ranh giới của nó, loại bỏ lượng lớn trầm tích khi cắt qua hẻm núi. Có bằng chứng về một hồ nước ở độ cao lớn phía đông Grand Canyon trong các trầm tích hồ và hóa thạch biển. Khi so sánh Grand Canyon với các ví dụ quy mô lớn khác về xói mòn tràn bờ, như Hẻm núi Afton và núi St. Helens, người ta thấy chúng có địa hình tương tự và cho thấy rằng các hẻm núi lớn có thể hình thành rất nhanh qua lưu lượng nước chảy lớn \cite{48}.

Xét về quy mô của các quá trình địa chất cần thiết để tích tụ các lớp trầm tích trên diện tích lớn như vậy, sự đồng thời của các lực kiến tạo mạnh xảy ra ngay sau khi các lớp trầm tích được hình thành, và kích thước nhỏ bé của sông Colorado so với quy mô khổng lồ của Grand Canyon, có vẻ như không có gì là “dần dần” trong quá trình hình thành của nó.

\section{Thành phố ngầm Derinkuyu}

Ngoài kim tự tháp, một ví dụ tuyệt vời về kỹ thuật cổ đại là thành phố ngầm Derinkuyu (Hình \ref{fig:5}) nằm ở Cappadocia, Thổ Nhĩ Kỳ. Đây là nơi lớn nhất trong số hơn 200 nơi trú dưới lòng đất trong khu vực \cite{54}. Thành phố ngầm này ước tính từng chứa tới 20.000 người và có 18 tầng, sâu tới 85 mét. Dù tuổi của thành phố chưa được xác định chắc chắn, nhưng nó được ước tính ít nhất đã có 2800 năm tuổi. Thành phố này được khoét vào đá núi lửa mềm \cite{52, 53}.

\begin{figure}[t]
\begin{center}
% \fbox{\rule{0pt}{2in} \rule{0.9\linewidth}{0pt}}
   \includegraphics[width=1\linewidth]{derinkuyu.jpeg}
\end{center}
   \caption{Sơ đồ thành phố ngầm Derinkuyu \cite{56}.}
\label{fig:5}
\label{fig:onecol}
\end{figure}

\begin{figure}[t]
\begin{center}
% \fbox{\rule{0pt}{2in} \rule{0.9\linewidth}{0pt}}
   \includegraphics[width=1\linewidth]{derinkuyu-air.jpg}
\end{center}
   \caption{Một trục thông gió sâu trong thành phố ngầm Derinkuyu \cite{53}.}
\label{fig:6}
\label{fig:onecol}
\end{figure}

Điều khiến Derinkuyu thú vị là vì không rõ tại sao lại có cộng đồng quyết định xây cả một thành phố dưới lòng đất. Để tạo ra không gian sinh sống dưới lòng đất, mọi khoang đều phải được khoét ra từ đá. Hình dạng và kết cấu thô ráp của các đường hầm dưới lòng đất cho thấy rõ ràng những nơi này được khoét bằng sức lao động thủ công thay vì dùng máy móc, điều này khó khăn hơn hàng chục lần so với việc xây nơi trú trên mặt đất. Thực tế, không rõ tại sao con người lại muốn sống vĩnh viễn dưới lòng đất trong suốt cuộc đời, khi mà nông nghiệp, ánh sáng mặt trời, thiên nhiên và khám phá chỉ có trên mặt đất. "Lịch sử" truyền thống cho rằng Derinkuyu được tạo ra bởi những người Cơ đốc giáo cần một nơi ẩn dật để thực hành tôn giáo của họ \cite{53}. Tuy nhiên, theo lẽ thường, cách đối phó trực tiếp và đơn giản nhất với kẻ thù là “chiến đấu hoặc bỏ chạy” chứ không phải là “khoét cả một thành phố ngầm từ đá”.

Quy mô, độ sâu và sự tỉ mỉ trong thiết kế của thành phố ngầm này cho thấy rõ ràng nó không được xây để làm cấu trúc phòng thủ quân sự tạm thời nhằm chiến đấu với quân xâm lược trong thời kỳ nguy nan, mà đúng hơn là nơi trú lâu dài để bảo vệ con người khỏi các thế lực nguy hiểm trên bề mặt. Derinkuyu không chỉ có phòng ngủ, bếp và nhà vệ sinh cơ bản mà còn có cả chuồng trại cho động vật, bồn chứa nước, kho lương thực, máy ép rượu vang và ép dầu, trường học, nhà nguyện, hầm mộ và trục thông gió khổng lồ (Hình \ref{fig:6}). Tại sao nơi trú quân sự lại cần máy ép rượu vang và phải được đào sâu đến 85 mét với sự phức tạp như vậy?

Giải thích hợp lý nhất cho việc xây Derinkuyu chính là sự cần thiết bức thiết phải chuẩn bị một nơi trú lâu dài, tự duy trì, nhằm bảo vệ khỏi các lực lượng địa chất thảm khốc trên bề mặt Trái Đất.

% \section{Các bất thường khác được giải thích rõ nhất bởi lý thuyết "Trái Đất lật"}

% Trước khi kết thúc, chúng tôi sẽ đề cập đến một số bất thường khoa học khác mà, khi xem xét trong bối cảnh các lực lượng địa chất thảm khốc, sẽ được giải thích một cách hợp lý.

\begin{figure}[b]
\begin{center}
% \fbox{\rule{0pt}{2in} \rule{0.9\linewidth}{0pt}}
   \includegraphics[width=1\linewidth]{muck-crop.jpeg}
\end{center}
   \caption{"Bùn" ở Alaska, bao gồm các mảnh vụn của cây cối, thực vật và động vật rải rác hỗn loạn trong phù sa và băng bị đóng băng \cite{146}.}
\label{fig:7}
\label{fig:onecol}
\end{figure}

\section{Sự tích lũy sinh khối}

Hỗn hợp sinh khối của nhiều loài động vật và thực vật khác nhau, thường được tìm thấy dưới dạng hóa thạch trong các lớp trầm tích, là một hiện tượng bí ẩn khác. Trong cuốn "Reliquoæ Diluvianæ" (Tàn tích của Đại Hồng Thủy), Mục sư William Buckland mô tả chi tiết các phát hiện về nhiều loài động vật mà không có lý do nào giải thích được tại sao lại cùng tồn tại, rải rác khắp nước Anh và Châu Âu, bị chôn vùi trong các lớp trầm tích 'diluvium' \cite{58}. Những hỗn hợp xương động vật như vậy cũng được phát hiện ở hang Skjonghelleren trên đảo Valdroy, Na Uy. Trong hang này, hơn 7.000 bộ xương động vật có vú, chim và cá đã được phát hiện trộn lẫn qua nhiều lớp trầm tích \cite{59}. Một ví dụ khác là hang San Ciro, còn được gọi là "Hang của Người Khổng Lồ", ở Ý. Trong hang này, nhiều tấn xương động vật có vú, chủ yếu là hà mã, được phát hiện trong tình trạng tươi đến mức chúng từng được dùng để làm đồ trang trí và xuất khẩu phục vụ việc sản xuất than đèn. Xương của các loài động vật khác nhau được cho là đã bị trộn lẫn, gãy, vỡ vụn và phân tán thành các mảnh nhỏ \cite{60,61}. Ở Mendes cổ đại, Ai Cập, một hỗn hợp các loài xương động vật đã được phát hiện trộn lẫn với đất sét thủy tinh (thủy tinh hóa) \cite{57}. Những phát hiện như vậy có thể gây bối rối, nhưng hoàn toàn có thể được giải thích bằng hiện tượng lũ lụt lớn, cuốn theo xác chết của các loài động vật rồi vùi lấp vào các lớp trầm tích, hoặc đẩy chúng vào trong hang động, và trong trường hợp đất sét thủy tinh hóa ở Ai Cập, là do các đợt phóng điện mạnh sau trận dịch chuyển lớp lõi — lớp manti. Hình \ref{fig:7} mô tả một ví dụ điển hình về "bùn" sinh khối ở Alaska \cite{56}.

\section{Hầm trú cổ đại}

Tổ tiên chúng ta đã để lại nhiều công trình cổ đại được kỹ thuật hóa cao, nơi đã phát hiện ra các di cốt người. Những công trình này thường được cho là các lăng mộ được xây cầu kỳ, nhưng nếu xem xét kỹ hơn, có thể đây thực sự là các hầm trú cổ đại.

\begin{figure}[b]
\begin{center}
% \fbox{\rule{0pt}{2in} \rule{0.9\linewidth}{0pt}}
   \includegraphics[width=1\linewidth]{ww19.jpg}
\end{center}
   \caption{Newgrange, Ireland — Có thể thấy khách tham quan đứng trước lối vào để hình dung về quy mô.}
\label{fig:8}
\label{fig:onecol}
\end{figure}

Một ví dụ tiêu biểu là Newgrange (Hình \ref{fig:8}), tượng đài chính trong quần thể Brú na Bóinne, một tập hợp các công trình cổ đại bao gồm các mộ hành lang. Những ngôi mộ này bao gồm một hoặc nhiều buồng chôn cất được phủ bằng đất hoặc đá và có một lối vào hẹp làm bằng những tảng đá lớn \cite{70}. Đây là ví dụ về kỹ thuật xây dựng rộng lớn cho một công trình phức tạp được bảo vệ, được xây dựng qua nhiều thế hệ, dường như chỉ để chôn cất một số ít người - những người thậm chí không còn sống khi quá trình xây dựng bắt đầu. Khi nó được một chủ đất địa phương phát hiện vào năm 1699, nó đã bị chôn vùi trong đất.

Nếu quan sát sơ qua cấu trúc, có thể thấy nỗ lực to lớn bỏ ra để xây - Newgrange bao gồm khoảng 200.000 tấn vật liệu. Bên trong nó, \textit{"...là một hành lang có buồng, có thể tiếp cận qua một lối vào ở phía đông nam của tượng đài. Hành lang dài 19 mét, tức là khoảng một phần ba chiều dài vào trung tâm của cấu trúc. Ở cuối hành lang có ba buồng nhỏ nối với một buồng trung tâm lớn hơn với mái vòm hình vòm chóp cao... Các bức tường của hành lang này được tạo nên từ những phiến đá lớn gọi là orthostat, với 22 phiến ở phía tây và 21 phiến ở phía đông. Chúng cao trung bình 1½ mét”} \cite{70}. Ngoài ra, còn có những chi tiết kỹ thuật chống thấm cực kỳ tinh vi. Ví dụ, ở phần mái, \textit{“Các khe hở trên mái được trát bằng hỗn hợp đất nung và cát biển để chống thấm nước, từ đó có thể xác định niên đại phóng xạ khoảng năm 2500 TCN cho cấu trúc ngôi mộ này”} \cite{71}. Cũng có một đoạn dốc được thiết kế để dẫn vào gian bên trong, có thể được tạo ra vì mục đích tương tự: \textit{"Do sàn của hành lang và buồng mộ được xây theo độ cao của ngọn đồi nơi công trình đặt trên, nên có sự chênh lệch gần 2 mét giữa lối vào và buồng mộ bên trong"} \cite{71}.

\begin{figure}[b]
\begin{center}
% \fbox{\rule{0pt}{2in} \rule{0.9\linewidth}{0pt}}
   \includegraphics[width=1\linewidth]{dolmen.jpg}
\end{center}
   \caption{Dolmen de Soto, Tây Ban Nha \cite{53}.}
\label{fig:9}
\label{fig:onecol}

\end{figure}

Việc không tìm thấy thi hài bên trong cũng là điểm đáng chú ý. Các cuộc khai quật đã phát hiện các mảnh xương cháy và không cháy, chỉ là tàn tích của một số ít người, rải rác trong hành lang. Công trình Newgrange được cho là mất vài thế hệ mới hoàn thành, dựa vào các mảnh vật liệu bên trong được định tuổi bằng carbon phóng xạ. Tại sao một cộng đồng cổ đại lại bỏ ra rất nhiều công sức để xây ngôi mộ khổng lồ và kỹ thuật cao chỉ để rải rác xương của vài người đã khuất trong lối đi? Sẽ hợp lý hơn nhiều khi cho rằng những cấu trúc đá megalith cổ xưa và được chống thấm cẩn thận này thực ra được xây làm nơi trú cho con người nhằm bảo vệ họ trước những thảm họa định kỳ của Trái Đất.

Tại Huelva, miền nam Tây Ban Nha, một ví dụ tương tự là Dolmen de Soto (Hình \ref{fig:9}), một trong khoảng 200 địa điểm tương tự trong khu vực \cite{72,32}. Đây là cấu trúc tinh gọn, được thiết kế kỹ lưỡng bằng các khối đá megalith và có đường kính 75 mét. Theo báo cáo, chỉ tìm thấy tám bộ hài cốt khi khai quật, tất cả đều được chôn ở tư thế thai nhi.

\section{Đề cập về những bất thường đáng chú ý}

Trong phần này, tôi sẽ đề cập ngắn gọn đến một số bất thường đáng chú ý hơn, tất cả đều được giải thích hợp lý bằng một thảm họa tương tự ECDO.

\subsection{Các bất thường sinh học}

\begin{figure}[b]
\begin{center}
% \fbox{\rule{0pt}{2in} \rule{0.9\linewidth}{0pt}}
   \includegraphics[width=1\linewidth]{bottleneck.jpg}
\end{center}
   \caption{Một nút thắt cổ chai di truyền thể hiện sự sàng lọc 95\% nam giới khoảng 6.000 năm trước \cite{62}.}
\label{fig:10}
\label{fig:onecol}
\end{figure}

Một số bất thường sinh học đáng chú ý bao gồm nút thắt cổ chai di truyền và hóa thạch cá voi trong đất liền. Zeng và cộng sự (2018) đã mô phỏng 125 trình tự nhiễm sắc thể Y từ người hiện đại, và dựa trên các điểm tương đồng và đột biến trong DNA, xác định một nút thắt suy giảm dân số nam giới đến 95\% vào khoảng 5.000 đến 7.000 năm trước (Hình \ref{fig:10}) \cite{62}. Hóa thạch cá voi đã được tìm thấy ở độ cao hàng trăm mét trên mực nước biển, tại Swedenborg, Michigan, Vermont, Canada, Chile và Ai Cập \cite{63,64,65,66}. Các hóa thạch cá voi này được tìm thấy ở nhiều trạng thái khác nhau: Được bảo quản hoàn hảo, nằm trong đầm lầy phía trên các lớp băng tích, hoặc bị chôn vùi trong trầm tích. Số lượng mẫu vật tại những địa điểm này dao động từ vài cái cho đến hơn một trăm cái. Cá voi là loài sinh vật ở biển sâu và hiếm khi tiếp cận gần bờ biển. Làm thế nào mà những con cá voi này lại xuất hiện ở những vị trí có độ cao như vậy, thậm chí ở khoảng cách xa đất liền?

Trong lịch sử Trái Đất đã từng xảy ra nhiều sự kiện tuyệt chủng hàng loạt, được nghiên cứu kỹ lưỡng nhất là "Big Five" (5 sự kiện tuyệt chủng lớn) trong kỷ Phanerozoic: Kết thúc Ordovic muộn (LOME), Devon muộn (LDME), Kết thúc Permi (EPME), Kết thúc Trias (ETME) và Kết thúc Creta (ECME) \cite{88,89}. Thật kỳ lạ, một số sự kiện tuyệt chủng này được cho là xảy ra cùng thời kỳ lịch sử với nhiều lớp đất đá của Hẻm núi Grand Canyon, cụ thể là các lớp đá của kỷ Permi và Devon.

\subsection{Các bất thường vật lý}

\begin{figure}[b]
\begin{center}
% \fbox{\rule{0pt}{2in} \rule{0.9\linewidth}{0pt}}
   \includegraphics[width=1\linewidth]{columbia.jpg}
\end{center}
   \caption{Các gợn sóng dòng chảy khổng lồ ở hồ băng Columbia, bang Washington \cite{80}.}
\label{fig:11}
\label{fig:onecol}
\end{figure}

Có rất nhiều cảnh quan ngoài Hẻm núi Grand Canyon được cho là được hình thành bởi các lực tàn phá dữ dội. Bằng chứng về dòng chảy nước khổng lồ trên lục địa có thể được tìm thấy trong các gợn sóng dòng chảy khổng lồ trên toàn cầu. Một ví dụ là vùng Channeled Scablands ở khu vực Tây Bắc Thái Bình Dương. Ở đây, không chỉ có các hiện tượng cảnh quan trầm tích và các tảng đá trôi dạt kỳ lạ mà còn có hơn một trăm chuỗi gợn sóng lớn hình thành do dòng chảy cực mạnh \cite{78,79}. Đây là những phiên bản quy mô lớn hơn của các gợn sóng trên nền cát của lòng suối. Có thể tìm thấy những gợn sóng này ở khắp nơi trên thế giới như Pháp, Argentina, Nga và Bắc Mỹ \cite{81}. Hình \ref{fig:11} mô tả một số gợn sóng này tại bang Washington của Hoa Kỳ \cite{80}.
\begin{figure}[b]
\begin{center}
% \fbox{\rule{0pt}{2in} \rule{0.9\linewidth}{0pt}}
   \includegraphics[width=1\linewidth]{zhangjiajie.jpg}
\end{center}
   \caption{Những trụ đá khổng lồ ở rừng quốc gia Trương Gia Giới, miền nam Trung Quốc.}
\label{fig:12}
\label{fig:onecol}
\end{figure}

\begin{figure}[b]
\begin{center}
% \fbox{\rule{0pt}{2in} \rule{0.9\linewidth}{0pt}}
   \includegraphics[width=1\linewidth]{hoy.jpg}
\end{center}
   \caption{Cột đá biển Old Man of Hoy, Scotland \cite{83}.}
\label{fig:13}
\label{fig:onecol}
\end{figure}
Các cấu trúc xói mòn nội địa cũng được giải thích hợp lý bằng một sự kiện lật trục trái đất tương tự ECDO. Trung Quốc phía nam là ví dụ điển hình về cảnh quan bị xói mòn bởi nước, hình thành do dòng nước mạnh \cite{82}. Những cảnh quan này bao gồm dạng karst hình tháp, karst dạng chóp, cầu đá tự nhiên, các hẻm núi, hệ thống hang động lớn và các hố sụt. Một trong những dạng nổi bật nhất là rừng quốc gia Trương Gia Giới, nơi chứa các cột đá sa thạch thạch anh khổng lồ (Hình \ref{fig:12}) \cite{84}. Những cột đá này nằm ở độ cao trung bình hơn 1.000 mét và cao nhất hơn 3.100 mét. Hơn 1.000 trong số đó cao hơn 120 mét, 14 cái cao hơn 300 mét \cite{85}. Các cột đá này giống như những cột đá ngoài biển (Hình \ref{fig:13}), vốn là trụ đá ven biển được hình thành do vật liệu xung quanh bị sụp đổ vì sóng biển xói mòn. Các cảnh quan xói mòn tương tự còn xuất hiện ở các khối đá hình nón tại Ürgüp, Thổ Nhĩ Kỳ, cũng như ở Ciudad Encantada, Tây Ban Nha — cả hai đều ở độ cao hơn 1.000 mét so với mực nước biển. Tất cả những nơi này đều có sự kết hợp giữa muối và hóa thạch sinh vật biển nằm gần nhau, cho thấy có thể từng xảy ra sự xâm nhập của nước biển \cite{15,86,87}. Dĩ nhiên, các câu chuyện về đại hồng thủy \cite{3} thường nhắc đến việc nước biển dâng cao hơn 1.000 mét. Điều này được xác nhận bởi sự hiện diện của nước mặn và các cánh đồng muối lớn ở dãy Andes và Himalaya — nằm cách mặt biển hàng kilomet. Ví dụ, cánh đồng muối Uyuni ở Bolivia nằm ở độ cao 3.653 mét so với mực nước biển \cite{94}.

\subsection{Các sự kiện biến đổi khí hậu đột ngột}

Tài liệu khoa học hiện đại công nhận sự tồn tại của các sự kiện thay đổi khí hậu toàn cầu rất nhanh trong lịch sử gần đây của Trái Đất. Hai ví dụ đáng chú ý là sự kiện cách đây 4.200 năm và cách đây 8.200 năm, cả hai đều trùng hợp với sự giảm dân số và rối loạn định cư xã hội trên các khu vực địa lý rộng lớn. Các sự kiện này được lưu giữ dưới dạng các dấu vết bất thường trong lõi trầm tích và băng, san hô hóa thạch, giá trị đồng vị O18, mẫu phấn hoa, măng đá và dữ liệu mực nước biển. Các biến đổi khí hậu được suy luận bao gồm giảm nhanh nhiệt độ toàn cầu, khô hạn, rối loạn dòng hoàn lưu kinh tuyến Đại Tây Dương và sự tiến triển của băng hà \cite{90,91,92}. Đặc biệt, sự kiện cách đây 8.200 năm trùng hợp với khả năng lũ lụt nước mặn dữ dội ở Biển Đen vào khoảng năm 6400 TCN \cite{93}.

\subsection{Những điểm bất thường khảo cổ học}

Bằng chứng khảo cổ về một số thành phố cổ cho thấy nhiều lớp chôn vùi và tàn phá, tạo thành các ghi chép về các sự kiện thảm họa trong quá khứ. Thành cổ Jericho là một ví dụ như vậy, nằm ở Palestine ngày nay. Nó có nhiều lớp tàn phá, với các cấu trúc đá bị sụp đổ và cháy dữ dội \cite{96,97}. Niên đại ghi nhận trong các lớp khảo cổ kéo dài từ khoảng 9000 TCN đến 2000 TCN. Đặc biệt đáng chú ý là tòa tháp của nó, dường như đã bị cắt ngang và chôn vùi trong trầm tích vào khoảng năm 7400 TCN (Hình \ref{fig:14}) \cite{95}. Catal Huyuk \cite{99}, Gramalote \cite{98} và cung điện Minoan ở Knossos trên đảo Crete \cite{100,101} đều là những ví dụ tương tự về các di chỉ khảo cổ học có nhiều lớp, thường có bằng chứng về sự tàn phá.

\begin{figure}[t]
\begin{center}
% \fbox{\rule{0pt}{2in} \rule{0.9\linewidth}{0pt}}
   \includegraphics[width=1\linewidth]{jericho.jpg}
\end{center}
   \caption{Phục dựng khảo cổ về việc chôn vùi Tháp Jericho vào khoảng năm 7400 TCN \cite{95}.}
\label{fig:14}
\label{fig:onecol}
\end{figure}

Một bằng chứng khác cho những thảm họa lớn làm gián đoạn nền văn minh nhân loại là Hình Nampa, một bức tượng nhỏ bằng đất sét được tìm thấy dưới khoảng 100 mét nham thạch ở Idaho \cite{102,103}. Dòng chảy nham thạch nơi phát hiện bức tượng được ước tính là được hình thành vào cuối kỷ Đệ Tam hoặc đầu kỷ Đệ Tứ, được cho là đã 2 triệu năm tuổi. Tuy nhiên, dòng nham thạch ở khu vực này dường như còn khá mới. Những phát hiện như vậy không chỉ cho thấy các thảm họa lớn có khả năng hủy diệt nền văn minh mà còn đặt ra nghi vấn đối với các niên đại hiện đại.

\section{Về phương pháp định tuổi hiện đại}

Có nhiều lý do để hoài nghi về các niên đại hiện đại, vốn gán cho các vật liệu vật lý những độ tuổi rất dài lên đến hàng triệu, thậm chí hàng trăm triệu năm.

Câu chuyện chính thống cho rằng các "nhiên liệu hóa thạch" như than đá, dầu mỏ và khí tự nhiên đã có tuổi thọ hàng trăm triệu năm \cite{104}. Tuy nhiên, xét nghiệm định tuổi bằng carbon thực hiện trên dầu mỏ tại Vịnh Mexico đã cho kết quả tuổi vào khoảng 13.000 năm \cite{105}. Cacbon-14 có chu kỳ bán rã rất ngắn (5.730 năm), nên được cho là sẽ phân rã hoàn toàn sau vài trăm nghìn năm. Tuy vậy, nó đã được tìm thấy trong than đá và hóa thạch được cho là lâu năm hơn gấp cả ngàn lần \cite{106}. Thực tế, người ta còn sản xuất được than nhân tạo trong phòng thí nghiệm dưới điều kiện có kiểm soát, chủ yếu là nhiệt độ cao, chỉ trong 2-8 tháng \cite{107}.

Các phương pháp định tuổi đồng vị phóng xạ khác ngoài định tuổi bằng cacbon cũng có thể không chính xác. Nhóm nghiên cứu Answers in Genesis đã tìm thấy sự không nhất quán trong các kết quả định tuổi từ các phương pháp này, điều này làm dấy lên nghi ngờ về độ tin cậy của chúng \cite{108}. Thậm chí, đã tìm thấy mô mềm chứa tế bào máu, mạch máu và collagen trong hài cốt khủng long được cho là đã hàng trăm triệu năm tuổi \cite{109,110}. Dựa trên những gì chúng ta biết, có khả năng các độ tuổi được chấp nhận rộng rãi liên quan đến niên đại địa chất và các vật liệu vật lý như đá và nhiên liệu hóa thạch của Trái Đất có thể bị sai lệch rất nhiều so với thực tế.

\section{Kết luận}

Trong bài viết này, tôi đã đề cập đến những bất thường có tính thuyết phục nhất cho thấy nguồn gốc thảm họa và được giải thích rõ nhất bởi mô hình Trái Đất lật theo lý thuyết ECDO. Tuy các trường hợp được trình bày rất đa dạng, nhưng bộ sưu tập này vẫn chưa đầy đủ — còn có rất nhiều bất thường khác đã được tổng hợp và công khai trên kho GitHub nghiên cứu của tôi \cite{2}.

\section{Lời cảm ơn}

Cảm ơn The Ethical Skeptic, tác giả ban đầu của lý thuyết ECDO, đã hoàn thành luận án chuyên sâu, đột phá và chia sẻ nó với thế giới. Bộ luận thuyết gồm ba phần của ông \cite{1} vẫn là tài liệu nền tảng cho lý thuyết về ECDO (Hiệu ứng Dzhanibekov do quá trình tách rời tỏa nhiệt giữa lớp lõi và lớp manti của Trái Đất), và chứa đựng nhiều thông tin hơn so với phần trình bày ngắn gọn trong tài liệu này.

Và dĩ nhiên, xin cảm ơn những người khổng lồ mà chúng ta đang đứng trên vai họ — những người đã thực hiện tất cả các nghiên cứu và điều tra giúp cho công trình này trở nên khả thi và đã nỗ lực mang lại ánh sáng cho nhân loại.
\clearpage
\twocolumn

{\small
\renewcommand{\refname}{Tài liệu tham khảo}
\bibliographystyle{ieee}
\bibliography{egbib}
}

\end{document}
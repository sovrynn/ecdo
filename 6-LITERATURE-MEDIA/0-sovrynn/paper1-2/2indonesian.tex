\documentclass[10pt,twocolumn,letterpaper]{article}

% My own stuff
\usepackage{booktabs}
% \usepackage{caption}
% \captionsetup[table]{skip=8pt}   % Only affects tables
\usepackage{stfloats}  % Add this to the preamble

\usepackage{cvpr}
\usepackage{times}
\usepackage{epsfig}
\usepackage{graphicx}
\usepackage{amsmath}
\usepackage{amssymb}

% Include other packages here, before hyperref.

% If you comment hyperref and then uncomment it, you should delete
% egpaper.aux before re-running latex.  (Or just hit 'q' on the first latex
% run, let it finish, and you should be clear).
\usepackage[breaklinks=true,bookmarks=false]{hyperref}

\cvprfinalcopy % *** Uncomment this line for the final submission

\def\cvprPaperID{****} % *** Enter the CVPR Paper ID here
\def\httilde{\mbox{\tt\raisebox{-.5ex}{\symbol{126}}}}

% Pages are numbered in submission mode, and unnumbered in camera-ready
%\ifcvprfinal\pagestyle{empty}\fi
\setcounter{page}{1}
\begin{document}

%%%%%%%%% TITLE
\title{ECDO Pengantar Berbasis Data Bagian 2/2: Sebuah Penyelidikan tentang Anomali Ilmiah dan Sejarah yang Terbaik Dijelaskan oleh “Pembalikan Bumi” ECDO}

\author{Junho\\
Situs web: \href{https://sovrynn.github.io}{sovrynn.github.io}\\
Repo Penelitian ECDO: \href{https://github.com/sovrynn/ecdo}{github.com/sovrynn/ecdo}\\
{\tt\small junhobtc@proton.me}
}

\maketitle
%\thispagestyle{empty}

%%%%%%%%% ABSTRACT
\begin{abstract}
Pada Mei 2024, seorang penulis online dengan nama samaran "The Ethical Skeptic" \cite{0} memposting teori revolusioner yang disebut Exothermic Core-Mantle Decoupling Dzhanibekov Oscillation (ECDO) \cite{1}. Teori ini tidak hanya mengusulkan bahwa Bumi sebelumnya mengalami pergeseran sumbu rotasi yang tiba-tiba dan bencana, menciptakan banjir besar di seluruh dunia dengan menyebabkan lautan meluap di atas benua karena inersia rotasi, tetapi juga mengusulkan proses geofisika penyebab penjelas bersama dengan data yang menunjukkan bahwa pembalikan semacam itu mungkin akan segera terjadi. Sementara prediksi banjir kataklismik dan kiamat semacam itu bukanlah hal baru, teori ECDO sangat menarik karena pendekatannya yang ilmiah, modern, multidisiplin, dan berbasis data.

Makalah penelitian ini merupakan bagian kedua dari rangkuman singkat dua bagian dari 6 bulan penelitian independen \cite{2,20} ke dalam teori ECDO, dengan fokus khusus pada anomali ilmiah dan sejarah yang terbaik dijelaskan oleh "Pembalikan bumi" ECDO yang bencana.

\end{abstract}

%%%%%%%%% BODY TEXT

\section{Pendahuluan}

Geologi dan sejarah uniformitarian modern mengklaim bahwa lanskap geologi utama seperti Grand Canyon terbentuk selama jutaan tahun \cite{143}; bahwa garam ada di Death Valley (California) karena dulu berada di bawah laut ratusan juta tahun yang lalu \cite{144}; bahwa nenek moyang kita dari 150 generasi yang lalu menghabiskan seluruh hidup mereka membangun makam raksasa \cite{29,70}; dan bahwa apa yang disebut "bahan bakar fosil" berusia ratusan juta tahun \cite{104}. Mungkin yang paling menarik adalah bahwa manusia dipercaya berusia 300.000 tahun \cite{145}, namun sejarah dan peradaban yang tercatat hanya berasal dari sekitar 5.000 tahun - setara dengan 150 generasi manusia.

Seperti yang akan kita lihat, anomali semacam itu paling baik dijelaskan oleh kekuatan geologis yang bencana.

\section{Mamut yang Membeku Secepat Kilat Terkubur Dalam Lumpur}

\begin{figure}[t]
\begin{center}
   \includegraphics[width=1\linewidth]{jarkov-mammoth.jpg}
\end{center}
   \caption{Mamut Jarkov, mamut Siberia berusia 20.000 tahun yang terawetkan dengan sempurna ditemukan dalam lumpur beku \cite{51}.}
\label{fig:1}
\label{fig:onecol}
\end{figure}

Satu kategori anomali semacam itu adalah mammoth yang terjaga sempurna dalam keadaan beku kilat yang terkubur dalam lumpur, umumnya ditemukan di wilayah Arktik (Gambar \ref{fig:1}). Mammoth Beresovka, ditemukan di Siberia terkubur dalam kerikil berlempung, begitu terjaga dengan sempurna sehingga dagingnya masih dapat dimakan ribuan tahun setelah kematiannya. Mammoth ini juga memiliki makanan vegetatif di mulut dan perutnya, membingungkan para ilmuwan tentang bagaimana ia bisa dibekukan begitu cepat jika sedang merumput pada tanaman berbunga tepat sebelum kematiannya \cite{17}. Dilaporkan, \textit{"Pada tahun 1901 sebuah sensasi disebabkan oleh penemuan bangkai lengkap mammoth di dekat sungai Berezovka, karena hewan ini tampaknya mati karena kedinginan di pertengahan musim panas. Isi lambungnya terjaga dengan baik dan termasuk bunga buttercup dan kacang liar berbunga: ini berarti mereka harus telah ditelan sekitar akhir Juli atau awal Agustus. Makhluk ini mati begitu tiba-tiba sehingga masih menggenggam segenggam rumput dan bunga di rahangnya. Jelas bahwa ia terjerat oleh kekuatan luar biasa dan terlempar beberapa mil dari padang penggembalaannya. Panggul dan satu kakinya retak—hewan besar ini telah terjatuh berlutut dan kemudian membeku hingga mati, pada waktu yang biasanya adalah waktu terpanas dalam setahun"} \cite{18}. Selain itu, \textit{"[Ilmuwan Rusia] mencatat bahwa bahkan lapisan terdalam dari lambung hewan itu memiliki struktur serat yang terjaga dengan sempurna, menunjukkan bahwa panas tubuhnya telah dihilangkan oleh suatu proses super-prodigious dalam alam. Sanderson, menaruh perhatian khusus pada satu titik ini, membawa masalah ini kepada American Frozen Foods Institute: Apa yang dibutuhkan untuk membekukan seluruh mammoth sehingga kandungan kelembaban dari bagian terdalam tubuhnya, bahkan hingga lapisan dalam lambungnya, tidak memiliki waktu cukup untuk membentuk kristal besar yang dapat menghancurkan struktur serat dagingnya?... Beberapa minggu kemudian, Institut tersebut menghubungi Sanderson dengan jawaban: Ini benar-benar mustahil. Dengan semua pengetahuan ilmiah dan teknik rekayasa kita, sama sekali tidak ada cara yang diketahui untuk menghilangkan panas tubuh dari bangkai sebesar mammoth dengan cepat tanpa membentuk kristal kelembaban besar dalam dagingnya. Lebih lanjut, setelah mencoba semua teknik ilmiah dan rekayasa, mereka melihat ke alam dan menyimpulkan bahwa tidak ada proses yang diketahui di alam yang dapat mencapai prestasi ini"} \cite{19}.

\section{Grand Canyon}

Grand Canyon, bagian dari Great Basin di barat daya Amerika Utara, adalah fenomena alam lain yang menunjukkan asal mula katastrofik (Gambar \ref{fig:2}). Pertama, lapisan batu pasir dan batu kapur sedimen yang membentuk Grand Canyon mencakup area besar hingga mencapai 2,4 juta km$^2$ \cite{21}. Gambar \ref{fig:3} menunjukkan luas lapisan Coconino Sandstone di barat Amerika Serikat. Lapisan horizontal yang begitu besar dan seragam ini hanya dapat terbentuk sekaligus.

\begin{figure}[t]
\begin{center}
   \includegraphics[width=1\linewidth]{grand-canyon.jpg}
\end{center}
   \caption{Grand Canyon, di Arizona, AS \cite{49}.}
\label{fig:2}
\label{fig:onecol}
\end{figure}

\begin{figure}[t]
\begin{center}
   \includegraphics[width=1\linewidth]{coconino.jpg}
\end{center}
   \caption{Ukuran lapisan Coconino Sandstone di barat Amerika Serikat \cite{21}.}
\label{fig:3}
\label{fig:onecol}
\end{figure}

Tinjauan lebih dekat pada Grand Canyon memberi tahu kita bahwa pengendapan lapisan sedimen yang luas ini juga terjadi bersamaan dengan kekuatan tektonik yang signifikan. Untuk memahami hal ini, kita harus melihat dengan cermat pada area-area tertentu di ngarai di mana lapisan sedimen telah dilipat dan terbuka. Peneliti dari Answers in Genesis \cite{42} melihat sampel batuan mikroskopis dari beberapa lipatan ini, seperti Monument Fold, dan berdasarkan kurangnya fitur yang seharusnya ada jika lipatan terbentuk selama kerangka waktu yang lama di bawah panas dan tekanan, menyimpulkan bahwa lapisan sedimen dilipat oleh kekuatan tektonik saat mereka masih lunak, yakni, segera setelah pengendapannya \cite{43}.

\begin{figure*}
\begin{center}
\includegraphics[width=1\textwidth]{Grand_Staircase-big.jpg}
\end{center}
   \caption{Lapisan sedimen yang membentuk Grand Canyon (sisi kanan gambar) meluas langsung ke utara ke Cedar Breaks, Utah (sisi kiri gambar), di mana semua lapisan itu melengkung ke atas \cite{50}.}
\label{fig:4}
\end{figure*}

Dengan memperbesar, kita menemukan bahwa lapisan-lapisan yang membentuk Grand Canyon tidak hanya dilipat di dalam ngarai. Lapisan-lapisan tersebut telah dilipat ke timur di East Kaibab Monocline \cite{46}, tetapi juga ke utara di Cedar Breaks, Utah (Gambar \ref{fig:4}). Ini menunjukkan bahwa lapisan-lapisan ini mungkin semua dilipat bersama setelah mereka diletakkan satu di atas yang lain dalam urutan cepat. Sebagai referensi, lapisan horizontal Grand Canyon kira-kira memiliki ketebalan 1700 meter. Skala proses geologis yang diperlukan untuk meletakkan lapisan sedimen setebal satu mil sangat besar.

Pembentukan sebenarnya dari Grand Canyon adalah masalah perdebatan lain dalam geologi modern. Geologi uniformitarian mengusulkan bahwa Grand Canyon terbentuk oleh Sungai Colorado selama jutaan tahun \cite{47}. Namun, tim peneliti Answers in Genesis percaya bahwa Grand Canyon kemungkinan besar terbentuk dalam hitungan minggu akibat erosi saluran keluar dari danau kuno yang menerobos batasnya, menghilangkan sejumlah besar sedimen saat mengukir ngarai. Terdapat bukti adanya danau berprong tinggi di sebelah timur Grand Canyon dalam deposit sedimen danau dan fosil laut. Membandingkan Grand Canyon dengan contoh erosi saluran keluar berskala besar lainnya, seperti Afton Canyon dan Gunung St. Helens, mengungkapkan topografi yang serupa, dan menunjukkan bahwa ngarai besar dapat dibuat dengan cepat melalui sejumlah besar aliran air \cite{48}.

Mempertimbangkan skala proses geologis yang diperlukan untuk meletakkan sedimen di atas wilayah daratan yang sangat luas, keseragaman kekuatan tektonik yang masif terjadi segera setelah lapisan sedimen terletak, dan ukuran Sungai Colorado yang kecil dibandingkan dengan skala besar Grand Canyon, tampaknya tidak ada yang gradual tentang pembentukannya.

\section{Kota Bawah Tanah Derinkuyu}

Selain piramida, contoh luar biasa dari teknik kuno adalah kota bawah tanah Derinkuyu (Gambar \ref{fig:5}), terletak di Cappadocia, Turki. Ini adalah yang terbesar di antara lebih dari 200 tempat perlindungan bawah tanah di wilayah ini \cite{54}. Kota bawah tanah ini diperkirakan dapat menampung hingga 20.000 orang dan mencakup 18 lantai, mencapai kedalaman 85 meter. Sementara usianya tidak pasti, diperkirakan setidaknya berumur 2800 tahun. Kota ini diukir dari batuan vulkanik lunak \cite{52, 53}.

\begin{figure}[b]
\begin{center}
   \includegraphics[width=1\linewidth]{derinkuyu.jpeg}
\end{center}
   \caption{Diagram kota bawah tanah Derinkuyu \cite{56}.}
\label{fig:5}
\label{fig:onecol}
\end{figure}
Alasan Derinkuyu menarik adalah karena tidak jelas mengapa komunitas manapun memutuskan untuk membangun seluruh kota di bawah tanah. Untuk menciptakan ruang hidup di bawah tanah, setiap rongga harus dipahat dari batu. Bentuk kasar dan tekstur dari terowongan bawah tanah membuatnya jelas bahwa ini dipahat dengan tenaga kerja manual, bukan dengan alat bertenaga, yang akan jauh lebih sulit daripada membangun tempat perlindungan di atas tanah. Faktanya, tidak jelas mengapa ada manusia yang ingin tinggal secara permanen di bawah tanah selama kehidupan duniawinya, ketika pertanian, sinar matahari, alam, dan eksplorasi hanya tersedia di atas tanah. "Sejarah" konvensional mengusulkan bahwa Derinkuyu diciptakan oleh orang Kristen yang membutuhkan tempat terpencil untuk mempraktikkan agama mereka \cite{53}. Namun akal sehat akan menyimpulkan bahwa cara paling sederhana untuk menghadapi musuh adalah "lawan atau lari", bukan "pahat kota bawah tanah dari batu".

Skala, kedalaman, dan ketelitian desain kota bawah tanah membuatnya jelas bahwa itu tidak dirancang sebagai struktur pertahanan militer sementara untuk melawan penyerang di masa krisis, melainkan, tempat perlindungan jangka panjang untuk melindungi dari kekuatan fatal di permukaan. Derinkuyu dilengkapi tidak hanya dengan kamar tidur dasar, dapur, dan kamar mandi, tetapi juga kandang untuk hewan, tangki air, penyimpanan makanan, alat pemeras anggur dan minyak, sekolah, kapel, makam, dan poros ventilasi besar (Gambar \ref{fig:6}). Mengapa tempat perlindungan militer memerlukan alat pemeras anggur dan perlu digali sedalam 85 meter dengan kompleksitas seperti itu?

Penjelasan paling masuk akal untuk penciptaan Derinkuyu adalah kebutuhan mendesak untuk mempersiapkan tempat perlindungan jangka panjang yang mandiri untuk melindungi dari kekuatan geofisika yang dahsyat di permukaan Bumi.

\begin{figure}[t]
\begin{center}
   \includegraphics[width=1\linewidth]{derinkuyu-air.jpg}
\end{center}
   \caption{Sebuah sumur ventilasi dalam di Derinkuyu \cite{53}.}
\label{fig:6}
\label{fig:onecol}
\end{figure}

\section{Akumulasi Biomassa}

Campuran biomassa dari berbagai jenis hewan dan tumbuhan, yang sering ditemukan dalam bentuk fosil di lapisan sedimen, adalah anomali membingungkan lainnya. Dalam "Reliquoæ Diluvianæ", Rev. William Buckland merinci temuan berbagai spesies fauna yang tidak memiliki alasan yang dapat dijelaskan untuk ditemukan bersama, tersebar di seluruh Inggris dan Eropa, terkubur di lapisan 'diluvium' sedimen \cite{58}. Campuran seperti itu dari sisa-sisa hewan juga ditemukan di Gua Skjonghelleren di pulau Valdroy, Norwegia. Di dalam gua ini, lebih dari 7.000 tulang mamalia, burung, dan ikan ditemukan bercampur di berbagai lapisan sedimen \cite{59}. Contoh lain adalah San Ciro, "Gua Para Raksasa", di Italia. Di dalam gua ini, beberapa ton tulang mamalia, kebanyakan kuda nil, ditemukan dalam keadaan begitu segar sehingga dipotong menjadi ornamen dan dikirim untuk produksi lampu hitam. Tulang-tulang dari berbagai hewan tersebut dilaporkan bercampur, pecah, hancur, dan tersebar dalam fragmen \cite{60,61}. Di Mendes Kuno, Mesir, campuran berbagai spesies tulang hewan ditemukan bercampur dengan tanah liat yang tervitrifikasi (berkaca) \cite{57}. Temuan-temuan tersebut mungkin tampak membingungkan, namun dengan mudah dijelaskan oleh banjir besar yang meletakkan campuran hewan mati di lapisan sedimen, mengendapkan hewan ke dalam atau menguburnya hidup-hidup di dalam gua, dan dalam kasus biomassa tervitrifikasi di Mesir, pelepasan listrik besar pasca-banjir dari perpindahan inti-selubung. Gambar \ref{fig:7} menggambarkan eksposur khas 'lumpur' biomassa Alaska \cite{56}.

\begin{figure}[t]
\begin{center}
   \includegraphics[width=1\linewidth]{muck-crop.jpeg}
\end{center}
   \caption{'Lumpur' Alaska, terdiri dari fragmen pohon, tumbuhan, dan hewan yang tersebar secara kacau di lanau beku dan es \cite{146}.}
\label{fig:7}
\label{fig:onecol}
\end{figure}

\section{Bunker Kuno}

Nenek moyang kita meninggalkan banyak struktur kuno yang dirancang dengan sangat baik di mana ditemukan sisa-sisa manusia. Ini biasanya diartikan sebagai makam yang rumit, tetapi jika dilihat lebih dekat menunjukkan bahwa ini mungkin sebenarnya adalah bunker kuno.

\begin{figure}[b]
\begin{center}
   \includegraphics[width=1\linewidth]{ww19.jpg}
\end{center}
   \caption{Newgrange, Irlandia - lihat pengunjung di pintu masuk untuk skala.}
\label{fig:8}
\label{fig:onecol}
\end{figure}

Salah satu contoh yang sangat baik adalah Newgrange (Gambar \ref{fig:8}), monumen utama dalam kompleks Brú na Bóinne, kumpulan struktur kuno termasuk apa yang disebut makam lorong. Makam ini terdiri dari satu atau lebih kamar pemakaman yang ditutup dengan tanah atau batu dan memiliki akses lorong sempit yang terbuat dari batu besar \cite{70}. Ini adalah contoh rekayasa ekstensif dari sebuah struktur terlindungi yang kompleks, dibangun selama beberapa generasi, yang konon untuk mengubur beberapa orang, yang bahkan tidak hidup ketika pembangunan makam dimulai. Ketika ditemukan kembali oleh pemilik tanah setempat pada tahun 1699, itu tertutup tanah.

Sekilas melihat struktur ini mengungkapkan usaha besar yang dihabiskan untuk membangunnya - Newgrange terdiri dari sekitar 200.000 ton material. Di dalamnya, \textit{“…ada lorong berkammer, yang dapat diakses melalui pintu masuk di sisi tenggara monumen. Lorong membentang sepanjang 19 meter (60 kaki), atau sekitar sepertiga dari jalan ke pusat struktur. Di ujung lorong terdapat tiga kamar kecil dari kamar pusat yang lebih besar dengan atap kubah tinggi yang dibangun dengan metode bertumpuk bercabang... Dinding dari lorong ini terbuat dari lempengan batu besar yang disebut ortostat, dua puluh dua di antaranya berada di sisi barat dan dua puluh satu di sisi timur. Rata-rata mereka memiliki tinggi 1½ meter”} \cite{70}. Ada juga rincian teknik kedap air yang rumit. Misalnya, di atap, \textit{“Celahan atap ditutupi dengan campuran tanah bakar dan pasir laut untuk membuatnya kedap air dan dari campuran ini dua tanggal radiokarbon yang berpusat pada 2500 SM diperoleh untuk struktur makam”} \cite{71}. Selain itu, kenaikan elevasi yang mengarah ke kamar dalam mungkin telah diterapkan untuk tujuan serupa: \textit{“Karena lantai lorong dan kamar makam mengikuti kenaikan tanah bukit tempat monumen ini dibangun, ada perbedaan hampir 2 meter dalam tingkat lantai antara pintu masuk dan bagian dalam kamar”} \cite{71}.

\begin{figure}[t]
\begin{center}
   \includegraphics[width=1\linewidth]{dolmen.jpg}
\end{center}
   \caption{Dolmen de Soto, Spanyol \cite{53}.}
\label{fig:9}
\label{fig:onecol}
\end{figure}

Kurangnya sisa-sisa manusia di dalamnya juga merupakan hal yang menarik. Penggalian mengungkapkan fragmen tulang yang dibakar dan tidak dibakar yang mewakili beberapa orang, tersebar di sepanjang lorong. Pembangunan Newgrange diperkirakan memakan waktu setidaknya beberapa generasi berdasarkan tanggal karbon dari bahan di dalamnya. Mengapa komunitas kuno menghabiskan begitu banyak usaha untuk membangun makam besar dan sangat dirancang hanya untuk menyebar fragmen tulang dari beberapa orang yang sudah meninggal di lorongnya? Jauh lebih masuk akal bahwa struktur megalitik kuno ini, yang dipadukan kedap air dengan hati-hati, justru dibangun sebagai tempat perlindungan manusia untuk melindungi orang selama bencana berulang di Bumi.

Di Huelva, Spanyol selatan, contoh serupa adalah Dolmen de Soto (Gambar \ref{fig:9}), salah satu dari sekitar 200 situs semacam itu di daerah tersebut \cite{72,32}. Ini adalah struktur yang ramping dan sangat direkayasa yang dibangun menggunakan batu megalitik dan memiliki diameter 75 meter. Dilaporkan, hanya delapan tubuh yang ditemukan saat penggalian, semuanya terkubur dalam posisi janin.

\section{Penyebutan Anomali Terkemuka}

Di bagian ini, saya secara singkat menyebutkan beberapa anomali terkemuka lainnya, yang semuanya dijelaskan dengan baik oleh bencana seperti ECDO.

\subsection{Anomali Biologis}
\begin{figure}[t]
\begin{center}
   \includegraphics[width=1\linewidth]{bottleneck.jpg}
\end{center}
   \caption{Sebuah hambatan genetik yang mewakili penyusutan 95\% populasi pria sekitar 6.000 tahun lalu \cite{62}.}
\label{fig:10}
\label{fig:onecol}
\end{figure}

Beberapa anomali biologis yang menonjol adalah hambatan genetik dan fosil paus daratan. Zeng et al. (2018) memodelkan 125 urutan kromosom-Y dari manusia modern, dan berdasarkan kesamaan dan mutasi dalam DNA, mereka mengidentifikasi hambatan populasi sebesar 95\% pada populasi pria sekitar 5.000 hingga 7.000 tahun lalu (Gambar \ref{fig:10}) \cite{62}. Fosil paus ditemukan ratusan meter di atas permukaan laut, di Swedenborg, Michigan, Vermont, Kanada, Chili, dan Mesir \cite{63,64,65,66}. Paus-paus ini ditemukan dalam berbagai keadaan: terawetkan sempurna, di rawa-rawa di atas deposit glasial, atau terkubur dalam sedimen. Jumlah spesimen di lokasi-lokasi ini berkisar dari beberapa hingga lebih dari seratus. Paus adalah makhluk laut dalam dan jarang mendekati pantai. Bagaimana paus-paus ini bisa berakhir di ketinggian yang begitu tinggi, seringkali di jarak ekstrem ke daratan?

Banyak kepunahan massal telah terjadi di masa lalu Bumi, yang paling dipelajari secara menyeluruh adalah lima peristiwa "Lima Besar" pada era Fanerozoikum: kepunahan massal Akhir Ordovisium (LOME), Akhir Devon (LDME), akhir Permian (EPME), akhir Triasik (ETME), dan akhir Kapur (ECME) \cite{88,89}. Yang menarik, beberapa dari kepunahan ini diklasifikasikan terjadi dalam periode sejarah yang sama dengan banyak lapisan Grand Canyon, yaitu, lapisan Permian dan Devonian.

\subsection{Anomali Fisik}

\begin{figure}[b]
\begin{center}
   \includegraphics[width=1\linewidth]{columbia.jpg}
\end{center}
   \caption{Riak arus besar di Danau Glasial Columbia, negara bagian Washington \cite{80}.}
\label{fig:11}
\label{fig:onecol}
\end{figure}

Ada banyak lanskap selain Grand Canyon yang kemungkinan terbentuk melalui kekuatan kataklismik. Bukti aliran air kontinental yang besar dapat ditemukan dalam riak arus raksasa di seluruh dunia. Salah satu contohnya adalah Channeled Scablands di Pacific Northwest. Di sini, kita tidak hanya melihat lanskap deposit sedimen dan batu-batu besar yang tersesat, tetapi juga lebih dari seratus urutan riak besar yang terbentuk dari aliran arus mega \cite{78,79}. Ini adalah versi skala besar dari riak yang terbentuk di dasar pasir sungai. Ini dapat ditemukan di seluruh dunia di Prancis, Argentina, Rusia, dan Amerika Utara \cite{81}. Gambar \ref{fig:11} menggambarkan beberapa riak ini di negara bagian Washington di Amerika Serikat \cite{80}.

\begin{figure}[b]
\begin{center}
% \fbox{\rule{0pt}{2in} \rule{0.9\linewidth}{0pt}}
   \includegraphics[width=1\linewidth]{zhangjiajie.jpg}
\end{center}
   \caption{Pilar batu besar di Taman Hutan Nasional Zhangjiajie, selatan Cina.}
\label{fig:12}
\label{fig:onecol}
\end{figure}

\begin{figure}[t]
\begin{center}
   \includegraphics[width=1\linewidth]{hoy.jpg}
\end{center}
   \caption{Old Man of Hoy pilar laut, Skotlandia \cite{83}.}
\label{fig:13}
\label{fig:onecol}
\end{figure}

Struktur erosi pedalaman juga dijelaskan dengan baik oleh pembalikan Bumi seperti ECDO. Cina Selatan adalah contoh yang bagus dari lanskap karts masif, yang terbentuk melalui erosi air \cite{82}. Lanskap ini mencakup karst menara, karst puncak, karst kerucut, jembatan alami, ngarai, sistem gua besar, dan lubang runtuhan. Salah satu yang paling mencolok adalah Hutan Nasional Zhangjiajie, yang berisi pilar batu pasir kuarsa masif (Gambar \ref{fig:12}) \cite{84}. Pilar-pilar ini berdiri pada ketinggian rata-rata lebih dari 1.000 meter dan jumlahnya lebih dari 3.100. Lebih dari 1.000 di antaranya menjulang lebih dari 120 meter, dan 45 mencapai lebih dari 300 meter \cite{85}. Pilar-pilar ini menyerupai pilar erosi laut (Gambar \ref{fig:13}), yang merupakan pilar batu pantai yang terbentuk oleh runtuhnya materi di sekitarnya akibat gelombang laut. Lanskap erosi serupa dapat ditemukan di kerucut batu Urgup, Turki, serta Ciudad Encantada, Spanyol, yang keduanya lebih dari 1.000 meter di atas permukaan laut. Semua lokasi ini memiliki kombinasi fosil laut dan laut yang terletak berdekatan, menunjukkan adanya invasi laut di masa lalu \cite{15,86,87}. Tentu saja, kisah banjir \cite{3} menyebutkan laut naik jauh lebih tinggi dari 1.000 meter, dan ini diverifikasi oleh keberadaan air asin dan dataran garam masif di Andes dan Himalaya beberapa kilometer di atas permukaan laut. Dataran garam Uyuni di Bolivia, misalnya, mencapai 3.653 meter di atas permukaan laut \cite{94}.

\subsection{Peristiwa Perubahan Iklim Cepat}

Literatur ilmiah modern mengakui adanya peristiwa perubahan iklim global yang cepat dalam sejarah Bumi baru-baru ini. Dua contoh yang menonjol adalah peristiwa 4.2 kilotahun dan 8.2 kilotahun, keduanya bertepatan dengan pengurangan populasi dan gangguan pemukiman masyarakat di area geografis yang luas. Peristiwa ini dipertahankan sebagai anomali dalam sedimen dan inti es, fosil koral, nilai isotop O18, catatan serbuk sari dan speleothem, serta data ketinggian permukaan laut. Perubahan iklim yang disimpulkan termasuk penurunan cepat suhu global, pengeringan, gangguan arus balik meridional Atlantik, dan kemajuan glasial \cite{90,91,92}. Peristiwa 8.2 kilotahun khususnya bersamaan dengan potensi banjir dramatis air asin di Laut Hitam sekitar 6400 SM \cite{93}.

\subsection{Anomali Arkeologi}

Bukti arkeologis dari beberapa kota kuno menunjukkan beberapa lapisan yang melibatkan penguburan dan kehancuran, menciptakan catatan peristiwa kataklismik masa lalu. Kota kuno Yerikho adalah salah satu kota tersebut, terletak di wilayah Palestina modern. Ia berisi beberapa lapisan kehancuran, dengan runtuhnya struktur batu dan kebakaran hebat \cite{96,97}. Kronologi yang tercatat dalam lapisannya bertanggal dari sekitar 9000 SM hingga 2000 SM. Yang patut dicatat adalah menaranya, yang tampaknya telah dicukur dan dikubur dalam sedimen sekitar 7400 SM (Gambar \ref{fig:14}) \cite{95}. Catal Huyuk \cite{99}, Gramalote \cite{98}, dan istana Minoan Knossos di Kreta \cite{100,101} semua adalah contoh serupa dari situs arkeologi yang mengandung beberapa lapisan, sering kali berisi bukti kehancuran.

\begin{figure}[t]
\begin{center}
   \includegraphics[width=1\linewidth]{jericho.jpg}
\end{center}
   \caption{Rekonstruksi arkeologi penguburan Menara Yerikho sekitar 7400 SM \cite{95}.}
\label{fig:14}
\label{fig:onecol}
\end{figure}

Bukti lain dari kataklisme besar yang mengganggu peradaban manusia adalah Citra Nampa, sebuah boneka tanah liat yang ditemukan di bawah sekitar 100 meter lava di Idaho \cite{102,103}. Aliran lava di bawah mana figurine tersebut ditemukan diperkirakan didepositkan selama periode Tersier Akhir atau Kuarter Awal, diduga berusia 2 juta tahun. Namun, aliran lava di wilayah tersebut tampaknya relatif baru. Temuan semacam itu tidak hanya menunjuk pada kataklisme yang menghancurkan peradaban besar, tetapi juga mempertanyakan kronologi penanggalan modern.

\section{Mengenai Metode Penanggalan Modern}

Terdapat alasan signifikan untuk meragukan kronologi modern, yang memberikan usia yang sangat panjang hingga jutaan, atau bahkan hingga ratusan juta tahun pada berbagai bahan fisik.

Narasi konvensional menyatakan bahwa apa yang disebut "bahan bakar fosil" seperti batu bara, minyak, dan gas alam berusia ratusan juta tahun \cite{104}. Namun, penanggalan karbon sebenarnya dari minyak di Teluk Meksiko menemukan usia sekitar 13.000 tahun untuk minyak tersebut \cite{105}. Karbon-14 memiliki waktu paruh yang sangat pendek (5.730 tahun) sehingga seharusnya benar-benar meluruh setelah beberapa ratus ribu tahun. Namun, telah ditemukan dalam batu bara dan fosil yang diduga berusia seribu kali lebih tua \cite{106}. Faktanya, batu bara buatan telah diproduksi di laboratorium di bawah kondisi terkendali, terutama panas tinggi, hanya dalam 2-8 bulan \cite{107}.

Metode penanggalan radioisotop selain penanggalan karbon mungkin juga tidak akurat. Kelompok riset Answers in Genesis menemukan ketidaksesuaian dalam tanggal-tanggal yang dihasilkan dari metode tersebut yang memerlukan pertimbangan ulang terhadap keabsahan mereka \cite{108}. Jaringan lunak yang mengandung sel darah, pembuluh, dan kolagen bahkan telah ditemukan dalam sisa-sisa dinosaurus yang diduga berusia seratus juta tahun \cite{109,110}. Berdasarkan apa yang kita ketahui, mungkin usia yang diterima secara konvensional dari skala waktu geologis bumi dan bahan fisik seperti batuan dan bahan bakar fosil mungkin salah hingga banyak tingkat besaran.

\section{Kesimpulan}

Dalam makalah ini, saya telah membahas anomali paling menarik yang menunjukkan asal-usul bencana dan paling baik dijelaskan oleh pembalikan Bumi ECDO. Meskipun beragam, koleksi yang disajikan tidak lengkap - lebih banyak anomali telah dikompilasi dan tersedia untuk umum di repositori GitHub penelitian saya \cite{2}.

\section{Ucapan Terima Kasih}

Terima kasih kepada Ethical Skeptic, penulis asli tesis ECDO, karena telah menyelesaikan tesisnya yang penuh wawasan dan revolusioner serta membagikannya kepada dunia. Tesis tiga bagiannya \cite{1} tetap menjadi karya berwibawa untuk teori Pemisahan Inti-Mantel Eksotermik Oscilasi Dzhanibekov (ECDO), dan berisi lebih banyak informasi tentang topik ini daripada yang saya ringkas secara singkat di sini.

Dan tentu saja, terima kasih kepada para raksasa yang bahunya kami pijak; mereka yang telah melakukan semua penelitian dan penyelidikan yang membuat karya ini mungkin dan bekerja untuk membawa cahaya bagi umat manusia.

{\small
\bibliographystyle{ieee}
\bibliography{egbib}
}

\end{document}

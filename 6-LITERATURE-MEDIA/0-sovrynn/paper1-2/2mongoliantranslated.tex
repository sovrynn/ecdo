\documentclass[10pt,twocolumn,letterpaper]{article}

% Миний өөрийн зүйлс
\usepackage{booktabs}
% \usepackage{caption}
% \captionsetup[table]{skip=8pt}   % Зөвхөн хүснэгтүүдэд нөлөөлнө
\usepackage{stfloats}  % Үүнийг урьдчилсан хэсэгт нэмнэ үү

\usepackage{newunicodechar}
\usepackage{graphicx}

% Define a half-width em-dash
\newcommand{\shortemdash}{\scalebox{0.5}[1]{\textemdash}}

% Map U+2015 to this shorter dash
\newunicodechar{─}{\shortemdash}

\usepackage{fontspec}
\usepackage{ucharclasses}

%–– your fonts ––
\newfontfamily\latinfont{Latin Modern Roman}                % default (Latin)
\newfontfamily\mongscriptfont[Script=Mongolian]{Noto Serif}  
\newfontfamily\cyrillicfont[Script=Cyrillic]{Noto Serif}      % or Noto Serif, etc.

%–– detect and switch ––
\setDefaultTransitions{\latinfont}{}                        
\setTransitionsFor{Mongolian}{\mongscriptfont}{\latinfont} 
\setTransitionsFor{Cyrillic}{\cyrillicfont}{\latinfont}    

\usepackage{cvpr}
\usepackage{times}
\usepackage{epsfig}
\usepackage{graphicx}
\usepackage{amsmath}
\usepackage{amssymb}

% Бусад багцуудыг энд оруулна уу, hyperref-ээс өмнө.

% Хэрвээ та hyperref-г тайлж, дараа нь буцаан идэвхжүүлвэл,
% egpaper.aux-г устгах хэрэгтэй, дараа нь latex-г дахин ажиллуулна уу. (Эсвэл эхний latex 
% ажиллуулах үед 'q'-г дарж болно, дуустал нь хүлээгээд, дараа нь асуудалгүй болно).

\usepackage[breaklinks=true,bookmarks=false]{hyperref}

\cvprfinalcopy % *** Энд финал хувилбарт энэ мөрийг комментоос гаргана уу

\def\cvprPaperID{****} % *** CVPR-ний өгүүллийн дугаарыг энд оруулна уу
\def\httilde{\mbox{\tt\raisebox{-.5ex}{\symbol{126}}}}

\renewcommand{\refname}{Эшлэлүүд}
\renewcommand{\tablename}{Хүснэгт}
\renewcommand{\figurename}{Зураг}   % or whatever you like instead of "Hình"

\makeatletter
\def\abstract{%
  \centerline{\large\bf Хураангуй}% <-- your new label
  \vspace*{12pt}%
  \it%
}
\makeatother

% This makes the font slightly bigger than base (10) and bold in Subsection headings rather than using ptmb
\makeatletter
\def\cvprsubsection{%
  \@startsection{subsection}{2}{\z@}%
    {8pt plus 2pt minus 2pt}{6pt}%
    % {\normalfont\bfseries\selectfont}%
    {\normalfont\bfseries\fontsize{11}{13}\selectfont}%
}
\makeatother

% So this hardcodes the style for the numbers in the section/subsection headings so they're bold
\font\elvbf=ptmb scaled 1100
\font\elvbfs=ptmb scaled 1200
\makeatletter
% Section number: Large + bold
\renewcommand\thesection{%
  {\elvbfs\arabic{section}}%
}

% Subsection number: normalsize + bold + custom punctuation
\renewcommand\thesubsection{%
  {\elvbf
   \arabic{section}.\arabic{subsection}}%
}
\makeatother

% Зөвхөн илгээх горимд хуудасны дугаарлалт хийгдэнэ, эцсийн хувилбарт дугааргүй байна
%\ifcvprfinal\pagestyle{empty}\fi
\setcounter{page}{1}
\begin{document}

%%%%%%%%% ГАРЧИГ
\title{ECDO мэдээлэлд суурилсан гарын авлага 2/2: Шинжлэх ухаан, түүхийн гаж явдлуудыг ECDO “Дэлхийн эргэлт” онолоор тайлбарлах нь}

\author{Жунхо\\
2025 оны хоёрдугаар сард нийтлэв\\
Вэбсайт (Эндээс судалгааг татаж авна уу): \href{https://sovrynn.github.io}{sovrynn.github.io}\\
ECDO судалгааны репо: \href{https://github.com/sovrynn/ecdo}{github.com/sovrynn/ecdo}\\
{\tt\small junhobtc@proton.me}
}

\maketitle
%\thispagestyle{empty}

\begin{abstract}
2024 оны тавдугаар сард “Ёс зүйт сонирхогч” хэмээх нууц нэртэй онлайн зохиогч \cite{0} “Дулаан ялгаруулах цөм-мантийн салалтын джаанибековын хэлбэлзэл” (ECDO) \cite{1} хэмээх хувьсгалт онолыг нийтэлсэн. Энэхүү онолд дэлхий өмнө нь эргэлтийн тэнхлэгийн огцом, сүйрлийн шилжилтэд орж, эргэлтийн инерцийн улмаас далай тэнгисүүд эх газрын дээгүүр даван цутгаж, аварга их үер дэгдэж байсныг дэвшүүлээд зогсохгүй, мөн энэхүү эргэлт дахин ойртсоор байгааг илтгэх баримтат мэдээлэл болон түүнийг үүсгэх геофизикийн тайлбарлагдсан шалтгаант процессыг санал болгож буй. Ийм гамшгийн үер болон дэлхийн сүйрлийн таамаглал шинэ зүйл биш ч, шинжлэх ухааны суурьт, орчин үеийн, олон салбарт тулгуурласан, өгөгдөлд суурилсан арга барилтай ECDO онол нь өөрийн гэсэн онцлогтой бөгөөд үнэмшил төрүүлэхүйц юм.

Энэхүү судалгааны ажил нь сүүлийн 6 сарын хугацаанд бие даан хийсэн судалгааны \cite{2,20} ECDO онолтой холбоотой хоёр хэсгээс бүрдэх товч тайлангийн хоёр дахь хэсэг бөгөөд хүрээлэн буй шинжлэх ухаан болон түүхэнд тохиолдсон онцгой үзэгдлүүдээс ECDO “Дэлхийн эргэлт”-ийн онолын хүрээнд хамгийн оновчтойгоор тайлбарлагдаж болох гамшгийн шинж чанартай үзэгдлүүдийг авч үзнэ.

\end{abstract}

\section{Танилцуулга}

Орчин үеийн жигд өөрчлөлтийн геологи болон түүхийн дагуу Гранд Каньон зэрэг голлох геологийн тогтоц хэдэн сая жилийн турш бүрэлдсэн \cite{143}; Калифорнийн Үхлийн хөндийд давс байдаг нь хэдэн зуун сая жилийн өмнө далайн ёроолд оршиж байсантай холбоотой \cite{144}; 150 үеийн өмнөх манай өвөг дээдэс амьдралынхаа бүх хугацааг аварга том бунхан барихад зарцуулсан \cite{29,70}; мөн “шатах ашигт малтмал”-ууд хэдэн зуун сая жилийн настай \cite{104} гэж сургадаг. Хамгийн сонирхолтой нь хүн төрөлхтөн 300,000 жилийн настай гэж үздэг хэдий ч, бичигдсэн түүх, иргэншил ердөө 5,000 жил буюу зөвхөн 150 хүний үе үргэлжилсэн байдаг.

Дараа нь авч үзэх эдгээр хачирхалтай үзэгдлүүдийг гамшгийн шинж чанартай геологийн хүчээр хамгийн сайн тайлбарлаж болно.

\section{Шууд хөлдсөн хөх мангас шаварт булагджээ}

\begin{figure}[t]
\begin{center}
% \fbox{\rule{0pt}{2in} \rule{0.9\linewidth}{0pt}}
   \includegraphics[width=1\linewidth]{jarkov-mammoth.jpg}
\end{center}
   \caption{Жарковын мамонт, 20,000 жилийн өмнөх мөстэй шаварт бүрэн хадгалагдсан сибирийн мамонт \cite{51}.}
\label{fig:1}
\label{fig:onecol}
\end{figure}

Ийм төрлийн гаж үзэгдлийн нэг нь Арктикийн бүс нутагт ихэвчлэн олддог шаварт булагдсан, бүрэн хадгалагдсан ба маш түргэн хаталттай мамонт юм (Зураг \ref{fig:1}). Березовкагийн мамонт нь Сибирьт шавар ихтэй хайрганд булагдсан байдлаар илэрсэн бөгөөд хэдэн мянган жилийн дараа ч мах нь идэж болохуйц төгс хадгалагдсан байжээ. Мөн ам болон ходоодноос нь ургамлын хоол олдсон нь нэн гайхалтай бөгөөд энэхүү амьтан үхэхээсээ өмнөхөн шинээр цэцэглэсэн ургамлаар идэш тэжээл хийж байхад хэрхэн ингэж хурдан царцах боломжтойг эрдэмтэд гайхаж байна \cite{17}. Мэдээний дагуу, \textit{"1901 онд Березовка голын ойролцоо бүтэн мамонтын сэг олдсон нь шуугиан тарьсан бөгөөд энэ амьтан эд зуны улиралд хөлдөж үхсэн бололтой. Түүний ходоодны агууламж сайн хадгалалдагдсан ба дотор нь цэцэг, зэрлэг вандуй зэрэг байсан нь долоодугаар сарын сүүлээр эсвэл наймдугаар сарын эхээр идсэн байж таарахыг илтгэх юм. Амьтан тун гэнэтээр нас барсан тул амандаа хүртэл өвс, цэцгээр дүүрэн байж. Түүнийг асар их хүчтэй ямар нэгэн зүйл орон нутгаас нь хэдэн милийн зайд шидэж орхисон нь илт байв. Аарцаг болон нэг хөл нь хугарснаар олдсон энэ аварга том амьтан өвдөг дээрээ сөхөрч, дараа нь жилийн хамгийн халуун үеэр хөлдөж үхсэн байна"} \cite{18}. Мөн, \textit{"[Оросын эрдэмтэд] амьтны ходоодны дотоод хана хүртэл бүтцийн хувьд бүрэн хадгалагдсан болохыг тэмдэглэсэн нь, амьтны биеийн дулааныг байгаль дахь ямар нэгэн ер бусын хурдтай үйл явц бүрэн алдагдуулсан болохыг харуулжээ. Сандерсон энэ өнцгийг онцлон авч, түүнийгээ Америкийн Хөлдүү Хүнсний Институтэд тавьсан: Бүхэл бүтэн мамонтын биеийн хамгийн дотоод чийгт хэсэг болон ходоодны дотоод хананд хүртэл талст бий болж, махны бүтцийг эвдэхгүй байхаар цэвдэг болгоход юу шаардлагатай вэ?... Хэдэн долоо хоногийн дараа Институт Сандерсонд хариу ирүүлэв: Энэ бол огт боломжгүй. Бидний бүх шинжлэх ухаан, инженерийн мэдлэгээр, ийм том амьтны сэгийн дулааныг тийм хурдан буулгаж, маханд их хэмжээний чийгийн талст үүсэхгүйгээр хөлдөөх боломж байхгүй. Түүнчлэн, шинжлэх ухаан болон инженерчлэлийн бүх аргыг туршсаны эцэст тэд байгальд ч ийм үйл явц байхгүй гэдэг дүгнэлтэд хүрлээ"} \cite{19}.

\section{Гранд Каньон}

Гранд Каньон, Хойд Америкийн баруун өмнөд хэсгийн Их сав газрын нэг хэсэг, сүйрлийн гарал үүсэлтэй байж болох бас нэгэн байгалийн үзэгдэл юм (Зураг \ref{fig:2}). Юуны түрүүнд, Гранд Каньоныг бүрдүүлдэг тунамал элс, шохойн давхаргууд маш том 2.4 сая км$^2$ талбайд тархсан байдаг \cite{21}. Зураг \ref{fig:3} нь АНУ-ын баруун хэсэгт байрлах Кокононийн элс чулуун давхаргын хүрээг үзүүлжээ. Ийм их хэмжээний, ижил бүтэцтэй хэвтээ давхаргуудыг нэг дор л үүсгэсэн байх боломжтой.

\begin{figure}[b]
\begin{center}
% \fbox{\rule{0pt}{2in} \rule{0.9\linewidth}{0pt}}
   \includegraphics[width=1\linewidth]{grand-canyon.jpg}

\end{center}
   \caption{Гранд Каньон, Аризона, АНУ \cite{49}.}
\label{fig:2}
\label{fig:onecol}
\end{figure}

\begin{figure}[t]
\begin{center}
% \fbox{\rule{0pt}{2in} \rule{0.9\linewidth}{0pt}}
   \includegraphics[width=1\linewidth]{coconino.jpg}
\end{center}
   \caption{АНУ-ын баруун хэсэгт орших Коконино элсэн чулууны үеийн хэмжээ \cite{21}.}
\label{fig:3}
\label{fig:onecol}
\end{figure}

Гранд Каньоныг илүү анхааралтай ажиглавал эдгээр өргөн цар хүрээтэй тунамал давхаргууд хуримтлагдах явцад нөлөө бүхий тектоник хүчнүүд зэрэгцэн үйлчилж байсныг харж болно. Үүнийг ойлгохын тулд хөндийн зарим хэсэгт элсэн үеүд нугаларсан болон ил гарсан байдлыг анхааралтай ажиглах хэрэгтэй. "Answers in Genesis"-ийн судлаачид \cite{42} эдгээр нугалааснуудын заримын, жишээ нь Monument Fold-ийн чулуулгийн дээжийг микроскопоор шинжилж, хэрэв уг нугалааснууд удаан хугацаанд халуун ба даралтын нөлөөгөөр үүссэн бол байх ёстой шинж чанарууд илрээгүй тул тунамал давхаргууд анхалсан даруйдаа буюу зөөлөн байх үедээ тектоник хүчний нөлөөгөөр нугалагдсан гэж дүгнэжээ \cite{43}.

\begin{figure*}
\begin{center}

% \fbox{\rule{0pt}{2in} \rule{.9\linewidth}{0pt}}
\includegraphics[width=1\textwidth]{Grand_Staircase-big.jpg}
\end{center}
   \caption{Гранд Каньоны үелсэн тунамал чулуулгийн давхаргууд (зургийн баруун талд) хойд зүг рүү Сидар Брэйкс, Юта (зургийн зүүн талд) хүртэл чигээрээ үргэлжилдэг бөгөөд тэнд бүгд дээш өргөгдсөн байдаг \cite{50}.}
\label{fig:4}
\end{figure*}

Өргөн хүрээнд авч үзвэл, Гранд Каньоныг бүрдүүлж буй давхаргууд зөвхөн хавцлын доторх нумарсан бүрхүүл төдийхнөөр хязгаарлагдаагүй байна. Энэ давхаргууд зүүн тийш Кайбабын нэг талаар (East Kaibab Monocline) нумарсан \cite{46}, мөн хойд зүгт Сидар Брэйкс, Ютад ч гэсэн (Зураг \ref{fig:4}) ингэж нугарсан байдаг. Энэ нь эдгээр давхаргуудыг хоорондоо хурдтайгаар дараалан үелж үүссэний дараа нэгэн зэрэг эвхэгдсэн байж болзошгүйг харуулж байна. Жишээ нь, Гранд Каньоны хэвтээ давхаргуудын нийт зузаан нь ойролцоогоор 1700 метр байдаг. Нэг милийн зузаантай тунамал давхаргуудыг үүсгэхэд шаардагдах геологийн процессын цар хүрээ асар их юм.

Гранд Каньоны жинхэнэ үүсэл нь орчин үеийн геологийн шинжлэх ухааны маргаант сэдвүүдийн нэг юм. Уламжлалт геологийн үүднээс бол Гранд Каньоныг Колорадо мөрөн олон сая жилийн туршид сийлсэн гэж үздэг \cite{47}. Харин “Answers in Genesis” судалгааны багийнхан Гранд Каньон нь эртний нуурын эргийг сэтэлж гарсан асар их усны урсгалын нөлөөгөөр хэдхэн долоо хоногийн дотор үүссэн байх магадлалтай бөгөөд үүний улмаас их хэмжээний тунамал чулуулаг нэг мөчид зайлуулагдсан гэж үздэг. Зүүн хэсэгт нь орших өндөрлөг байрлалтай нуур байсан шинж тэмдэг нь нуурын тунамал хуримтлал болон далайн гаралтай чулуужсан олдворуудаар батлагдаж байна. Гранд Каньоныг бусад ижил хэмжээний усны түрлэгийн эвдрэлээс үүдэлтэй байгууламжууд болох Афтоны хавцал ба Сэнт Хеленс уултай харьцуулахад, газар зүйн хувьд төстэй дүр төрхтэй бөгөөд их хэмжээний усны урсац богино хугацаанд аварга том хавцлуудыг үүсгэж чадахыг харуулдаг \cite{48}.

Тунамал давхаргуудыг ингэж өргөн уудам нутаг дэвсгэр дээр хуримтлуулахад шаардагдах геологийн процессын цар хүрээ, уг давхаргууд үүссэний дараахан яг зэрэг давхцан тохиолдсон хүчтэй тектоник хөдөлгөөний үр дүн болон Колорадо мөрний бяцхан хэмжээ Гранд Каньоны асар том цар хүрээтэй харьцуулсан зэрэг нь, Гранд Каньон үүссэн явц огтхон ч аажим, удаан хугацаанд хийгдээгүй байх талыг дэмжиж байгаа юм.

\section{Деринкуюгийн газар доорх хот}

Пирамидуудаас гадна, эртний инженерчлэлийн гайхалтай нэгэн жишээ бол Туркийн Каппадоки мужид байрлах Деринкуюгийн газар доорх хот (Зураг \ref{fig:5}) юм. Энэхүү газар доорх хот нь тухайн бүс нутгийн 200 гаруй хоргодох байрны хамгийн том нь бөгөөд \cite{54} нийтдээ 20,000 хүртэлх хүн амьдарч байсан гэж тооцогддог, 18 давхар бүхий, 85 метрийн гүн рүү нэвтэрсэн байгууламж юм. Хэзээ баригдсан нь тодорхойгүй боловч дор хаяж 2800 жилийн настай гэж үздэг. Энэ хотыг галт уулын зөөлөн чулуулгийг ухаж бүтээжээ \cite{52, 53}.

\begin{figure}[b]
\begin{center}
% \fbox{\rule{0pt}{2in} \rule{0.9\linewidth}{0pt}}
\includegraphics[width=1\linewidth]{derinkuyu.jpeg}
\end{center}
   \caption{Дэринкую газрын доорх хотын бүдүүвч зураг \cite{56}.}
\label{fig:5}
\label{fig:onecol}
\end{figure}

Дэринкуюг сонирхолтой нэг тал нь яагаад нэгэн бүлэг хүмүүс газар доор бүхэл бүтэн хот байгуулна гэж шийдсэн нь тодорхой бус байдагт оршдог. Газар доор амьдрах орон зай үүсгэхийн тулд булан тохой бүрийг хад чулуунаас ухаж гаргах шаардлагатай болно. Газрын доорх хонгилуудын ширүүн хэлбэр, бүтэц нь эдгээрийг хүчирхэг багаж хэрэгсэл биш, харин гараар ухсан болохыг илтгэж байгаа бөгөөд энэ нь газар дээрх оромж барихаас хэд дахин илүү төвөгтэй ажил байсан гэсэн үг. Үнэндээ, хүн төрөлхтөн газар дээр нь газар тариалан эрхэлж, нарны гэрэл, байгальд ойрхон байж, эргэн тойрноо сонжихын оронд газрын гүнд бүх амьдралаа өнгөрүүлэхийг яагаад хүсэх нь ойлгомжгүй хэрэг юм. Хэвшмэл “түүх”-ээр бол христийн шашинтнууд шашнаа нууцаар шүтэх аюулгүй газар хэрэгтэй байсан учраас Дэринкуюг барьсан гэж үздэг \cite{53}. Гэтэл эрүүл ухаанаар бодвол, дайсантай тулгарахад хамгийн энгийн арга нь “тэмцэх эсвэл зугтах” болохоос “чулуу ухаж, газар доорх хот байгуулах” биш гэдгийг хялбархан дүгнэж болно.

Газар доорх хотын хэмжээ, гүн, ухаалаг шийдлээс нь харвал энэ байгууламжийг түр зуурын цэргийн хамгаалалтын орогнох газар бус, харин гадаргуу дээрх аюулт хүчин зүйлээс урт хугацаанд хамгаалах орчин болгон бүтээсэн нь илэрхий байдаг. Дэринкуюд энгийн унтлагын өрөө, гал тогоо, ариун цэврийн өрөөнөөс гадна мал амьтны хашаа, усны сав, хүнсний агуулах, дарс бөгөөд тос шахах төхөөрөмж, сургууль, сүм, оршуулгын газар болон асар том агааржуулалтын хонгилууд хүртэл (Зураг \ref{fig:6}) байсан. Цэргийн хоргодох байранд яагаад дарсны шахуурга хэрэгтэй байсан гэж? Мөн яагаад 85 метр гүн ухаж, ийм нарийн төвөгтэйгөөр байгуулсан байх ёстой вэ?

Дэринкуюг үүсгэх хамгийн итгэл төрүүлэх шалтгаан нь дэлхийн гадаргуу дээр үүсэх сүйрлийн геофизикийн хүчин зүйлээс хамгаалах, урт хугацааны, өөрийгөө хангахуйц байр бэлдэх шаардлага байсан байх магадлалтай.

\begin{figure}[t]
\begin{center}
% \fbox{\rule{0pt}{2in} \rule{0.9\linewidth}{0pt}}
   \includegraphics[width=1\linewidth]{derinkuyu-air.jpg}
\end{center}
   \caption{Дэринкую дахь гүн агааржуулалтын худаг \cite{53}.}
\label{fig:6}
\label{fig:onecol}
\end{figure}

% \section{Дэлхийн Эргэлтээр Хамгийн Сайн Тайлбарлагдах Нэмэлт Онцгой Үзэгдлүүд}

% Дуусахаасаа өмнө, бид катастрофик геофизикийн хүчний үүднээс харвал сайн тайлбарлагдах зарим шинжлэх ухааны нэмэлт онцгой үзэгдлийг дурдана.

\section{Биомассын хуримтлал}

Амьтан, ургамлын янз бүрийн холимог биомасс, ихэвчлэн тунамал давхаргад чулуужин олддог нь өөр нэг учир нь тайлагдаагүй оньсого юм. "Reliquoæ Diluvianæ" зохиолд, Уильям Баклэнд пастор Британи болон Европ даяар, ямар ч тодорхой шалтгаангүйгээр хамтдаа олдсон олон төрлийн амьтдын олдворуудыг тунамал "дилувиум" давхаргад булаастайгаар тайлбарласан байдаг \cite{58}. Ийм амьтны үлдэгдлийн холимог нь бас Норвегийн Валдрой арал дахь Скёнхеллерен агуйд илэрсэн. Энэ агуйд 7,000 гаруй хөхтөн амьтан, шувуу, загасны яс олон тунамал давхаргад хоорондоо холилдсон байдалтай олджээ \cite{59}. Өөр нэг жишээ нь Итали дахь Сан Чиро, буюу "Аварга биетнүүдийн агуй" юм. Тус агуйд, хэд хэдэн тонн хөхтөн амьтдын яс, голдуу усны үхрийн яс, маш шинээр олдсон ба бүр уран хээтэй болгон зүссэнээр, дэнгийн нүүрс хийхээр нийлүүлдэг байжээ. Янз бүрийн амьтдын яс нь хоорондоо холилдож, эвдэрч, бутарч, хэлтэрхий болон тараагдсан байсан гэж мэдээлсэн байна \cite{60,61}. Эртний Мендес, Египетэд, янз бүрийн амьтдын үхдэл шилэн шаварт холилдон олдсон \cite{57}. Ийм олдворууд анх харахад төөрөгдүүлмээр санагдаж болох ч, үерийн үлэмж давалгаанд амьтдын сэг тунамал давхаргад холигдон дарагдаж, зарим нь агуйд амьдаараа булагджээ хэмээвээс амархан ойлгогдоно. Харин Египетийн шилэрсэн биомассын хувьд, үерийн дараах цөм-манти давхаргын шилжилтээс үүдэлтэй асар их цахилгаан цэнэглэлтээс болсон байж болох юм. Зураг \ref{fig:7} дээр Аляскийн биомасс "шавар"-ын ил харуулсан жишээ үзүүлсэн байна \cite{56}.

\begin{figure}[t]
\begin{center}
% \fbox{\rule{0pt}{2in} \rule{0.9\linewidth}{0pt}}
   \includegraphics[width=1\linewidth]{muck-crop.jpeg}
\end{center}
   \caption{Аляскийн "шавар"─мод, ургамал, амьтдын үймээнтэй холилдсон хэлтэрхий нь хөлдүү шавар ба мөсөнд агуулагдсан байдалтай \cite{146}.}
\label{fig:7}
\label{fig:onecol}
\end{figure}
\section{Эртний хорогдох байр}

Бидний өвөг дээдэс олон инженерчлэлийн өндөр зэрэг бүхий эртний байгууламжуудыг үлдээсэн бөгөөд эдгээрээс хүний цогцос олдсон байдаг. Ихэвчлэн эдгээрийг нарийн хийцтэй булш гэж тайлбарладаг ч, нарийвчлан харвал эдгээр нь үнэндээ эртний хорогдох байр байж болох магадлалтай.

\begin{figure}[b]
\begin{center}
% \fbox{\rule{0pt}{2in} \rule{0.9\linewidth}{0pt}}
   \includegraphics[width=1\linewidth]{ww19.jpg}
\end{center}
   \caption{Нью Грийнж, Ирланд - орох хаалганы дэргэдэх зочдыг харж хэмжээ харьцуул.}
\label{fig:8}
\label{fig:onecol}
\end{figure}

Нэгэн онцлох жишээ бол Нью Грийнж (Зураг \ref{fig:8}), Бру на Бойнн цогцолборын гол дурсгал, энэ цогцолборт хонгилтой булш зэрэг эртний байгууламжууд багтдаг. Эдгээр булш нь нэг эсвэл хэд хэдэн булшны өрөөнөөс бүрдэх ба тэдгээрийг хөрс эсвэл чулуугаар хучиж, их чулуугаар барьсан нарийн орц, гарцтай байгуулсан байдаг \cite{70}. Энэ нь олон үе дамжин бүтээгдсэн цогцолбор бүхий хамгаалалттай бүтэц байсны жишээ бөгөөд олон хүний асар их хүчин чармайлтаар босож, тухайн бүтээн байгуулалтын ажил эхлэхэд амьд ч байгаагүй хэдхэн хүнийг оршуулах зорилготойгоор бүтээгдсэн гэх. 1699 онд нутгийн газрын эзэн олж илрүүлэх үед газар доор булшлагдсан байжээ.

Бүтцийг өнгөц харвал байгуулахад асар их хөдөлмөр зарцуулсан нь илхэн — Нью Грийнж нь ойролцоогоор 200,000 тонн материалаас бүтдэг. Дотроо, \textit{“…танхимтай хонгил байх бөгөөд баруун өмнөд талаас орж болох нэвтрэх хаалгатай. Хонгил нь 19 метр (60 фут), эсвэл бүтцийн төв хүртэлх замын ойролцоогоор гуравны нэгтэй тэнцэхүйц урт. Хонгилын төгсгөлд төв том танхимтай гурван жижиг өрөө байх бөгөөд корбелл таазан дээврээр өндөрлөн хийгдсэн… Энэ хонгилын ханыг ортостат гэдэг том чулуугаар барьсан, баруун талд хорин хоёр, зүүн талд хорин нэг бий. Эдгээрийн дундаж өндөр 1½ метр”} \cite{70}. Мөн уснаас хамгаалсан нарийн инженерийн шийдэлтэй. Жишээ нь, дээвэрт, \textit{“Дээврийн завсрыг шатаасан хөрс ба далайн элсний холимгоор чигжиж, ус нэвтрэхгүй болгосон бөгөөд энэ холимгоос авсан хоёр радиокарбон огноо булшны бүтэц 2500 МЭӨ үед хамаарахыг харуулсан"} \cite{71}. Түүнчлэн, дотор танхим рүү хүрэх өндрийн ялгаа нь мөн адил зориулалтаар шийдсэн байх магадлалтай: \textit{“Булшны хонгил ба өрөөний шал нь байгууламж барьсан толгойн налуудыг дагасан тул, орц, дотор танхим хоёрын шал хоёрын хооронд бараг 2 метрийн ялгаа бий”} \cite{71}.

\begin{figure}[b]

\begin{center}
% \fbox{\rule{0pt}{2in} \rule{0.9\linewidth}{0pt}}
   \includegraphics[width=1\linewidth]{dolmen.jpg}
\end{center}
   \caption{Испанийн Сото Долмен \cite{53}.}
\label{fig:9}
\label{fig:onecol}
\end{figure}

Дотор нь хүний чулуужсан яс бага олдсон нь бас сонирхолтой юм. Малталтаар шатаагдсан болон шатаагдаагүй хэд хэдэн хүнд хамаарах үлдэгдлүүд буюу ясны хэлтэрхийнүүд, хонгилын дагуу энд тэндгүй тарсан байсан нь олдсон. Ньюгренжийн доторх материалын нүүрстөрөгчийн огноог судалсны дагуу барилгын явц нь хэд хэдэн үе дамнан үргэлжилсэн гэж тооцдог. Эртний нэгэн нийгэм яагаад ийм их хүч хөдөлмөр гарган том хэмжээний, маш нарийн бүтэцтэй булш босгоод ердөө цөөхөн хүний ясны хэлтэрхийг л хонгилоор тараан байрлуулсан байж болох вэ? Усны нэвчилтээс нямбай гэгч нь хамгаалсан эдгээр эртний аварга чулуун байгууламжийг үнэндээ олон дахин давтагдсан дэлхийн гамшгийн үеэр хүмүүсийг хамгаалах байр болгон барьсан нь илүү магадлалтай байж болох юм.

Испанийн өмнөд хэсэгт орших Уэльвад төстэй нэгэн жишээ байдаг нь Сото Долмен (Зураг \ref{fig:9}) бөгөөд энэ бүс нутагт үүнтэй төстэй 200 орчим байгууламж байдгийн нэг нь юм \cite{72,32}. Энэ нь аварга чулууг ашиглан инженерчлэлийн өндөр түвшинд хийгдсэн бөгөөд 75 метр диаметртэй, хялбар болон үр бүтээмжтэй хэлбэрээр баригдсан. Малтлагаар ердөө найман цогцос олдсон бөгөөд бүгдийг нь ургийн хэвтээ байрлалтайгаар оршуулсан байжээ.

\section{Онцлох гаж явдлуудыг дурдах нь}

Энэ хэсэгт, би зарим нэг илүү онцлог гаж үзэгдлийг товч дурдах бөгөөд эдгээр нь бүгд ECDO-той төстэй гамшгаар бүрэн тайлбарлагдах юм.

\subsection{Биологийн гаж явдлууд}

\begin{figure}[b]
\begin{center}
% \fbox{\rule{0pt}{2in} \rule{0.9\linewidth}{0pt}}
   \includegraphics[width=1\linewidth]{bottleneck.jpg}
\end{center}
   \caption{Генетикийн бөглөрлийн үүднээс 6,000 жилийн өмнө эрэгтэйчүүдийн 95\% нь устгагдсан үзэгдлийг илэрхийлэх нь \cite{62}.}
\label{fig:10}
\label{fig:onecol}
\end{figure}

Зарим онцгой биологийн гаж үзэгдлүүдэд генетикийн бөглөрөл болон эх газарт олдсон халимын чулуужсан үлдэгдлүүд юм. Zeng нар (2018) орчин үеийн хүмүүсийн 125 Y-хромосомын дарааллыг судалж, ДНХ-ийн ижил төстэй байдал ба мутацид үндэслэн, 5,000-аас 7,000 жилийн өмнө нийт эрчүүдийн хүн амын 95\%-ийн цөөрөл бий болсон бөглөрлийг илрүүлсэн (Зураг \ref{fig:10}) \cite{62}. Халимын чулуужсан үлдэгдлүүд далайн түвшнээс хэдэн зуун метрийн өндөрт, Шведенборг, Мичиган, Вермонт, Канад, Чили, Египет зэрэг газруудад олдсон байна \cite{63,64,65,66}. Эдгээр халимууд янз бүрийн нөхцөлд: бүрэн бүтэн хадгалагдсан, мөсөн үеийн давхаргын дээр намагт хэвтэж байсан болон хөрсөнд булшлагдсан байдалтайгаар олдсон байна. Эдгээр газруудад байгаа чулуужмалын тоо хэдхэнээс хэдэн зуу хүртэл өөр өөр бий. Халим бол далайн гүн усны амьтан бөгөөд эрэг рүү бараг ойртдоггүй. Эдгээр халим яаж ийм өндөрлөг цэгт, эх газраас хол зайнд очсон юм бол?

Дэлхийн түүхэнд тэр чигээрээ устгалд орсон тохиолдлууд хэд хэд бий бөгөөд хамгийн их судлагдсан нь Фанерозойн "Их Тав"-ын массын устгалууд: Ордовикийн сүүлийн үеийн (LOME), Девоны сүүлийн үеийн (LDME), Пермийн төгсгөлийн (EPME), Триасын төгсгөлийн (ETME), Цэрдийн төгсгөлийн (ECME) массын устгалууд юм \cite{88,89}. Сонирхолтой нь, эдгээрийн хэд хэд нь Их Хөндийн олон давхаргын үед буюу Пермийн болон Девоны үеийн давхаргатай нэг цаг хугацаанд тохиолдсон нь тогтоогдсон байна.

\subsection{Физикийн гаж явдлууд}

\begin{figure}[b]
\begin{center}
% \fbox{\rule{0pt}{2in} \rule{0.9\linewidth}{0pt}}
   \includegraphics[width=1\linewidth]{columbia.jpg}
\end{center}

\caption{Вашингтон мужийн Колумбын мөсөн нуурт үүссэн аварга давлагааны мөрүүд \cite{80}.}
\label{fig:11}
\label{fig:onecol}
\end{figure}

Гранд Каньоноос гадна олон газар нутаг сүйрлийн хүчний нөлөөгөөр бий болсон байх магадлалтай. Эх газрын усны үер дэгдэж байсны нотолгоо дэлхий даяар аварга давлагааны мөрүүдээр илэрнэ. Ийм жишээний нэг нь Номхон далайн баруун хойд бүсэд байрлах Чаннелед Скабландс юм. Энд бид тунамал чулуулаг бүхий газар нутаг, тохиолдлын аварга хад чулуу төдийгүй мега урсгалын үр дүнд бий болсон зуу гаруй удаашралттай том давлагааны ул мөрүүдийг харж болно \cite{78,79}. Эдгээр нь горхины элсэрхэг ёроолд бий болдог жижигхэн давлагааны илүү том хэмжээний хувилбарууд юм. Эдгээрийг Франц, Аргентин, Орос, Хойд Америк зэрэг дэлхийн олон газарт харж болно \cite{81}. Зураг \ref{fig:11} нь АНУ-ын Вашингтон муж дахь эдгээр давлагаанаас харуулж байна \cite{80}.

\begin{figure}[b]
\begin{center}
% \fbox{\rule{0pt}{2in} \rule{0.9\linewidth}{0pt}}
   \includegraphics[width=1\linewidth]{zhangjiajie.jpg}
\end{center}
   \caption{Өмнөд Хятадын Жанжиажие Үндэсний ойн цэцэрлэгт хүрээлэнгийн аварга чулуун баганууд.}
\label{fig:12}
\label{fig:onecol}
\end{figure}

\begin{figure}[b]
\begin{center}
% \fbox{\rule{0pt}{2in} \rule{0.9\linewidth}{0pt}}
\includegraphics[width=1\linewidth]{hoy.jpg}
\end{center}
   \caption{Хой тэнгисийн багана, Шотланд \cite{83}.}
\label{fig:13}
\label{fig:onecol}
\end{figure}

Далд газрын элэгдлийн бүтцүүд мөн ECDO-той төстэй дэлхийн эргэлтийн онолоор сайн тайлбарлагддаг. Өмнөд Хятад бол усан элэгдлээр бүрэлдсэн маш том карстын ландшафтын тод жишээ юм \cite{82}. Эдгээр газар нутагт цамхаг карст, оргил карст, конус карст, байгалийн гүүр, хавцал, том агуйн систем, болон цөөрөм уналтууд багтана. Эдгээрээс хамгийн гайхамшигтай нь Жанжиажие Үндэсний ойн цогцолбор бөгөөд асар том кварцын элсэн чулууны багануудтай (Зураг \ref{fig:12}) \cite{84}. Эдгээр багануудын дундаж өндөр нь 1000 метрээс давсан бөгөөд нийт 3100 гаруй байдаг. Үүний 1000 гаруй нь 120 метрээс өндөр, 45 нь 300 метрээс давсан өндөртэй \cite{85}. Эдгээр баганууд далайн элэгдлийн багануудтай (Зураг \ref{fig:13}) ижилхэн бөгөөд тэдгээр нь далайн давлагаанаар эргэн тойрны чулуулгууд унаж, эвдэрснээр үүсдэг. Ижил элэгдлийн ландшафтуудыг Туркийн Үргүпийн чулуужсан конус болон Испанийн Сьюдад Энкантадад олж болно. Эдгээр хоёр газар хоёулаа далайн түвшнээс 1000 метрээс дээш өндөрт оршдог. Эдгээр бүх газруудын эргэн тойронд давс болон далайн гаралтай чулуужсан үлдэгдэл илэрдэг нь эртний далайн түрэмгийлэл болсныг харуулна \cite{15,86,87}. Мэдээж, үерийн тохиолдлууд \cite{3} далайн ус 1000 метрээс давсан өндөрт хүрсэн талаар өгүүлдэг ба үүнийг хэдэн километрийн өндөрт орших Анд болон Гималайн нуруун дахь давстай устай цөл болон том давст талбайнууд нотолдог. Жишээ нь, Боливийн Уюни давст тал далайн түвшнээс 3653 метрийн өндөрт оршино \cite{94}.

\subsection{Уур амьсгалын гэнэт өөрчлөгдсөн тохиолдлууд}

Орчин үеийн шинжлэх ухааны судалгаанд дэлхийн ойрын түүхэнд гэнэтийн, хурдан уур амьсгалын өөрчлөлт тохиолдсоныг хүлээн зөвшөөрсөн. Үүний хоёр чухал жишээ нь 4.2 мянган жил ба 8.2 мянган жилийн үзэгдлүүд бөгөөд эдгээр нь хүн ам буурах, нийгмийн суурьшил алдагдах үзэгдэлтэй давхацсан байдаг. Эдгээр үйл явдлууд тунадас болон мөсний цөм, чулуужсан шүрэн, O18 изотопын хэмжээ, тоос, сталактит, сталагмитны бичлэг, далайн түвшний мэдээнд тусгагдсан. Таамагласан уур амьсгалын өөрчлөлтөд дэлхийн дулааны огцом уналт, хуурайшилт, Атлантын меридиональ урсгал тасалдал, мөсөн голын үргэлжлэл зэрэг багтдаг \cite{90,91,92}. Ялангуяа 8.2 мянган жилийн үзэгдэл нь МЭӨ 6400 оны үед Хар тэнгис давстай усаар хүчтэй үерлэсэнтэй давхацсан байж болзошгүй \cite{93}.

\subsection{Археологийн гаж үзэгдлүүд}

Эртний зарим хотын археологийн олдворууд олон давхарга бүхий булш болон сүйрлийн ул мөр илэрдэг нь эртний гамшигт үйл явдлыг баталдаг. Ийм хотуудын нэг нь одоогийн Палестинд байрлах Иерихон юм. Энэ хотод чулуун бүтцийн нуралт болон хүчтэй галын улмаас үүссэн олон давхар гамшгийн ул мөр илэрчээ \cite{96,97}. Эдгээр давхаргыг судлан цаг үеийг шинжвээс МЭӨ 9000-2000 оны хооронд таарна. Ялангуяа энэ хотын цамхаг МЭӨ 7400 оны үед огцом сүйрч, тунадастай хамт булшлагдсан байна (Зураг \ref{fig:14}) \cite{95}. Чатал Хүйүк \cite{99}, Грамалотэ \cite{98}, мөн Крит дэх Миносын Кноссын ордон \cite{100,101} зэрэг нь аливаа гамшигт үзэгдэл тохиолдсоныг гэрчлэх баримт бүхий олон давхаргат археологийн жишээнүүд юм.

\begin{figure}[t]
\begin{center}
% \fbox{\rule{0pt}{2in} \rule{0.9\linewidth}{0pt}}

   \includegraphics[width=1\linewidth]{jericho.jpg}
\end{center}
   \caption{Ерихын цамхгийн булшны археологийн сэргээсэн дүрслэл, МЭӨ 7400 он орчим \cite{95}.}
\label{fig:14}
\label{fig:onecol}
\end{figure}

Хүн төрөлхтний соёл иргэншлийг бусниулсан томоохон сүйрлийн өөр нэгэн нотолгоо бол Нампагийн дүрслэл юм. Энэ нь Идахо мужид ойролцоогоор 100 метр галт уулын хайлмагийн дороос олдсон шавран хүүхэлдэй юм \cite{102,103}. Хүүхэлдэй олдсон хайлмагийн давхаргыг хожуу гуравдагч галав эсвэл эрт дөрөвдүгээр галавын үед бий болсон гэж тооцоолж, 2 сая жилийн настай гэж үзсэн байдаг. Гэвч тус бүс нутгийн хайлмагийн давхарга харьцангуй шинэ байх магадлалтай. Ийнхүү олдворууд зөвхөн соёл иргэншлийг устгасан гамшиг тохиолдсон гэдгийг харуулаад зогсохгүй, орчин үеийн он цагийн тооллын үнэн зөв байдалд эргэлзээ төрүүлж байна.

\section{Орчин үеийн он цагийн аргын тухай}

Орчин үед хэрэглэгдэж буй он цагийн тооллын аргууд нь хэдэн сая, заримдаа хэдэн зуугаас хэдэн зуун сая жилээр тооцоолох болсон нь эргэлзээ төрүүлэх шалтгаан ихтэй.

Уламжлалт ойлголтоор бол "чулуужсан түлш" болох нүүрс, газрын тос, байгалийн хийг хэдэн зуун сая жилийн настай гэж үздэг \cite{104}. Гэвч Мексикийн булангаас авсан газрын тосыг нүүрстөрөгчийн аргаар шинжлэхэд ойролцоогоор 13,000 жилийн настай гэж тооцоологдсон байна \cite{105}. Нүүрстөрөгч-14 нь маш богино хагас задралын хугацаатай (5,730 жил) бөгөөд хэдэн зуун мянган жилийн дараа бүрэн задрах учиртай. Гэвч үүнээс мянга дахин өндөр настай гэж тооцоологдсон нүүрс болон чулуужсан олдворуудад нүүрстөрөгч-14 илэрсэн байдаг \cite{106}. Үнэндээ хиймэл нүүрсийг лабораторийн нөхцөлд, гол төлөв өндөр халуун ашиглан ердөө 2-8 сарын дотор гарган авсан байна \cite{107}.

Нүүрстөрөгчийн аргаас өөр радиоизотопын аргууд ч бас яг нарийн биш байх магадлалтай. Answers in Genesis судалгааны бүлэг ийм аргуудаар гарсан огноо хоорондын зөрүүг илрүүлж, үүний үнэн зөвийг эсэхийг тунгаахад хүргэсэн байдаг \cite{108}. Түүнчлэн, зуун сая жилийн настай гэж тооцогдсон үлэг гүрвэлийн олдвороос цусны эс, судас, коллаген агуулсан зөөлөн эдүүд илэрсэн тохиолдол бий \cite{109,110}. Бидний мэдэж байгаагаар, Дэлхийн геологийн цаг хугацааны хуваарь болон чулуулаг, чулуужсан түлш зэрэг материалуудын уламжлалт настай холбоотой ойлголтууд нь бодит байдлаас хол зөрүүтэй байж болох юм.

\section{Дүгнэлт}

Энэ судалгааны ажилд би гамшгийн гаралтай гэж үзэж болохуйц, ECDO онолын дэлхийн эргэлтийн тайлбарт хамгийн зохистой нийцэх онцгой гаж үзэгдлүүдийн заримыг танилцууллаа. Танилцуулсан цуглуулга олон төрлийг хамарч буй боловч бүрэн гүйцэд биш — илүү олон гаж үзэгдлийг судалгааны GitHub санд орж үзэх боломжтой \cite{2}.

\section{Талархал}

ECDO-ийн онолын бүтээлийн анхны зохиогч болох "Ёс зүйт сонирхогч"-д, түүний гүн, шинэлэг бүтээлээ дэлхийтэй хуваалцсанд талархал илэрхийлье. Түүний гурван хэсэг бүхий онолын бүтээл \cite{1} нь Дулаан ялгаруулах цөм-мантийн салалтын джаанибековын хэлбэлзэл (ECDO) онолын хамгийн нэр хүндтэй бүтээл бөгөөд энэ сэдвээр миний энд товчлон өгүүлснээс илүү их мэдээллийг агуулдаг.

Мөн биднээс өмнө бүх судалгаа болон шинжилгээг хийж олон нийтэд түгээсэн, хүн төрөлхтөнд гэрэл авчрах гэж зорьсон хэн бүхэнд талархал илэрхийлье. Та бүхэн л энэхүү судалгааны ажил орших боломжийг олгож байгаа ба энэхүү аваргуудын мөрөн дээр зогсож байгаадаа баяртай байна.

\clearpage
\twocolumn

{\small
\renewcommand{\refname}{Эшлэлүүд}
\bibliographystyle{ieee}
\bibliography{egbib}
}

\end{document}
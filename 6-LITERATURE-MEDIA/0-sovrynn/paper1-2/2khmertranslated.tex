\documentclass[10pt,twocolumn,letterpaper]{article}

% ខ្ញុំបន្ថែមផ្ទាល់ខ្លួន
\usepackage{booktabs}
% \usepackage{caption}
% \captionsetup[table]{skip=8pt}   % មានផលប៉ះពាល់តែតារាងប៉ុណ្ណោះ
\usepackage{stfloats}  % បន្ថែមវានៅក្នុង preamble

%–– ICU line-breaking for Khmer ––
\XeTeXlinebreaklocale "km"
\XeTeXlinebreakskip = 0pt plus 0pt minus 0pt
% \XeTeXlinebreakskip = 0pt plus 1pt

\usepackage{fontspec}

%–– define your two fonts ––
\newfontfamily\latinfont{Latin Modern Roman}        % for all Latin text
\newfontfamily\khmerfont[Script=Khmer]{Noto Sans Khmer} % for all Khmer text

%–– load ucharclasses to auto‐detect Unicode blocks ––
\usepackage{ucharclasses}

% default (everything outside Khmer) → Latin font
\setDefaultTransitions{\latinfont}{}

% when entering the Khmer Unicode block → switch to Khmer font,
% and when leaving it → switch back to Latin
\setTransitionsFor{Khmer}{\khmerfont}{\latinfont}

% 1) Choose your desired fixed leading:
\renewcommand\baselinestretch{1.2}  % or 1.3, 1.1…  adjust to taste

% 2) Force TeX to *always* use \baselineskip, never fall back to \lineskip:
\makeatletter
  \setlength\lineskiplimit{-\maxdimen} % always allow baselineskip
  \setlength\lineskip{0pt}             % no extra glue ever
\makeatother


\usepackage{cvpr}
\usepackage{times}
\usepackage{epsfig}
\usepackage{graphicx}
\usepackage{amsmath}
\usepackage{amssymb}

% សូមបញ្ចូល package ផ្សេងទៀតនៅទីនេះ មុនពេល hyperref។

% ប្រសិនបើអ្នកមកនិយាយពី hyperref ហើយបន្ទាប់មកមកដោះសោវា វា គួរតែលុប
% egpaper.aux មុនពេលប្ដូរឡើងវិញ latex។  (ឬតែចុច 'q' នៅលើ latex ដំបូង
% អោយវាបញ្ចប់ ហើយអ្នកគួរតែស្អាត)។
\usepackage[breaklinks=true,bookmarks=false]{hyperref}

\cvprfinalcopy % *** ដោះស្រាយបន្ទាត់នេះសម្រាប់ការដាក់ស្នើចុងក្រោយ

\def\cvprPaperID{****} % *** បញ្ចូលលេខសម្គាល់សៀវភៅ CVPR នៅទីនេះ
\def\httilde{\mbox{\tt\raisebox{-.5ex}{\symbol{126}}}}


\renewcommand{\tablename}{តារាង}
\renewcommand{\figurename}{រូប}   % or whatever you like instead of "Hình"
\renewcommand{\refname}{ឯកសារយោង}

\makeatletter
\def\abstract{%
  \centerline{\large\bf សេចក្តីសង្ខេប}% <-- your new label
  \vspace*{12pt}%
  \it%
}
\makeatother

\makeatletter
\def\cvprsubsection{%
  \@startsection{subsection}{2}{\z@}%
    {8pt plus 2pt minus 2pt}{6pt}%
    % {\normalfont\bfseries\selectfont}%
    {\normalfont\bfseries\fontsize{11}{13}\selectfont}%
}
\makeatother

% So this hardcodes the style for the numbers in the section/subsection headings so they're bold
\font\elvbf=ptmb scaled 1100
\font\elvbfs=ptmb scaled 1200
\makeatletter
% Section number: Large + bold
\renewcommand\thesection{%
  {\elvbfs\arabic{section}}%
}

% Subsection number: normalsize + bold + custom punctuation
\renewcommand\thesubsection{%
  {\elvbf
   \arabic{section}.\arabic{subsection}}%
}
\makeatother

% ទំព័រត្រូវបានដាក់លេខនៅក្នុងរបៀបដាក់ស្នើ ហើយមិនមានលេខនៅក្នុងរបៀបបោះពុម្ព​ចុងក្រោយ
%\ifcvprfinal\pagestyle{empty}\fi
\setcounter{page}{1}
\begin{document}

%%%%%%%%% ចំណងជើង
\title{ECDO Data-Driven Primer ភាគ 2/2: ការស្រាវជ្រាវអំពីភាពមិនប្រក្រតីវិទាសាស្ត្រនិងប្រវត្តិសាស្ត្រដែលពន្យល់បានយ៉ាងល្អដោយ "ក្រឡាប់ផែនដី" របស់  ECDO}

\author{Junho\\
បោះពុម្ពផ្សាយ ខែកុម្ភៈ ឆ្នាំ2025\\
គេហទំព័រ (ទាញយកអត្ថបទនៅទីនេះ): \href{https://sovrynn.github.io}{sovrynn.github.io}\\
ប្រព័ន្ធស្រាវជ្រាវ ECDO: \href{https://github.com/sovrynn/ecdo}{github.com/sovrynn/ecdo}\\
{\tt\small junhobtc@proton.me}
}

\maketitle
%\thispagestyle{empty}

\begin{abstract}
នៅក្នុងខែឧសភា ឆ្នាំ2025 អ្នកនិពន្ធអនាមិកអនឡាញម្នាក់ឈ្មោះ "The Ethical Skeptic"\cite{0} បានផ្សព្វផ្សាយទ្រឹស្ដីបំប្លែងមួយដែលមានឈ្មោះថា ការបំបែកលំយោលស្រទាប់ក្រឡាប់ក្តៅក្នុងផែនដី (ECDO)\cite{1}។ ទ្រឹស្ដីនេះមិនត្រឹមតែបានផ្តល់នូវយោបល់ថាផែនដីកាលពីមុនបានបម្លែងអ័ក្សវិលដ៏សាហាវ ធ្វើអោយមានទឹកជំនន់ដ៏ធំទូទាំងពិភពលោក ដែលបណ្តាលអោយមហាសមុទ្រលេចលើទ្វីបនានា មិនត្រឹមតែដោយសារនិចលភាពនៃការវិលប៉ុណ្ណោះទេ តែថែមទាំងផ្ដល់ការបកស្រាយនៃដំណើរការភូមិវិទ្យាដែលផ្តោតលើទិន្នន័យហេតុផលដែលបានបង្ហាញថាការផ្លាស់ប្តូរផែនដីអាចនឹងកើតមានម្តងទៀត។ ទោះបីជាការព្យាករណ៍ទឹកជំនន់និងគ្រោះមហន្តរាយមិនមែនជារឿងថ្មីក៏ដោយ ទ្រឹស្ដី ECDO គឺមានភាពទាក់ទាញយ៉ាងខ្លាំងដោយសារតែការយោងទៅលើវិទ្យាសាស្ត្រ ពហុបច្ចេកទេស និងមានមូលដ្ឋានដែលផ្អែកទៅលើទិន្នន័យ។

ក្រដាសស្រាវជ្រាវនេះជាផ្នែកទី2នៃការសង្ខេបដែលមានពីរផ្នែក នៃការស្រាវជ្រាវឯករាជ្យដែលមានរយៈពេល6ខែ\cite{2,20} ទៅក្នុងទ្រឹស្ដី ECDO ដោយផ្តោតទៅលើភាពមិនប្រក្រតីនៃផ្នែកវិទ្យាសាស្ត្រ និងប្រវត្តិសាស្ត្រដែលត្រូវបានពន្យល់យ៉ាងល្អដោយ "ក្រឡាប់ផែនដី" តាមទ្រឹស្ដីបែប ECDO។

\end{abstract}

\section{សេក្តីណែនាំ}

ភូគព្ភវិទ្យា និងប្រវត្តិសាស្ត្ររួមសម័យនេះអះអាងថា តំបន់ធំៗដូចជា Grand Canyon ត្រូវបានបង្កើតឡើងក្នុងរយៈពេលរាប់លានឆ្នាំ\cite{143} ថាមានអំបិលនៅក្នុង Death Valley (California) ព្រោះតំបន់នោះធ្លាប់ស្ថិតនៅក្រោមមហាសមុទ្ររយៈពេលរាប់រយលានឆ្នាំកន្លងមក \cite{144} បុព្វបុរសរបស់យើងប្រហែល150ជំនាន់កន្លងមកបានចំណាយជីវិតសាងសង់ផ្នូរធំៗ\cite{29,70} និងបានរៀបរាប់ថា "ប្រេងឥន្ធនៈផូសុីល" គឺមានអាយុរាប់រយលានឆ្នាំមកហើយ\cite{104}។ អ្វីដែលគួរអោយគត់សំគាល់នោះគឺ គេជឿជាក់ថាមនុស្សមានអាយុ300000ឆ្នាំ\cite{145} ប៉ុន្តែប្រវត្តិសាស្ត្រដែលចងក្រងគឺត្រឹមតែ5000ឆ្នាំប៉ុណ្ណោះ ស្មើរនឹងមនុស្ស150ជំនាន់។

ភាពមិនប្រក្រតីទាំងនេះ នឹងត្រូវបានពន្យល់យ៉ាងក្បោះក្បាយដោយកម្លាំងមហន្តរាយភូគព្ភវិទ្យា។

\section{ដុំសត្វម៉ាមម៉ុស្ដ្រត្រូវបានបោសកកក្នុងភក់}
\begin{figure}[t]
\begin{center}
% \fbox{\rule{0pt}{2in} \rule{0.9\linewidth}{0pt}}
   \includegraphics[width=1\linewidth]{jarkov-mammoth.jpg}
\end{center}
   \caption{សត្វម៉ាមម៉ុត់យ៉ាកក៉ូវ (Jarkov Mammoth) អាយុ20000ឆ្នាំដែលស្តិតក្នុងស្ថានភាពយ៉ាងល្អគឺជាសត្វម៉ាមម៉ុត់នៅតំបន់ស៊ីបេរីដែលបានរកឃើញនៅក្នុងភក់ដែលគ្របដណ្តប់ដោយទឹកកក\cite{51}។}
\label{fig:1}
\label{fig:onecol}
\end{figure}

ភាពមិនប្រក្រតីមួយនៃបាតុភូតនេះ គឺជាសត្វម៉ាមម៉ុត់កក ដែលនៅមានស្ថានភាពយ៉ាងល្អដែលបានកប់នៅក្នុងភក់ ត្រូវគេប្រទះឃើញនៅតំបន់អាកទិក (រូបភាពទី \ref{fig:1})។ សត្វម៉ាមម៉ុត់បេរេសូវកា (Beresovka mammoth) ដែលបានរកឃើញនៅតំបន់ស៊ីបេរីដែលបានកប់នៅក្នុងកំហាប់ថ្ម មានសភាពយ៉ាងល្អសូម្បីតែសាច់របស់វាក៏អាចបរិភោគបានដែរ ទោះបីវាមានអាយុរាប់ពាន់ឆ្នាំក៏ដោយ។ សត្វនេះនៅមានអាហាររុក្ខជាតិនៅក្នុងមាត់និងពោះរបស់វា ដែលបណ្ដាលអោយវេជ្ជបណ្ឌិតជាច្រើនមានចម្ងល់ថា ហេតុអ្វីបានជាវាកកយ៉ាងឆាប់រហ័សបែបនេះ ខណៈពេលវាកំពុងតែសុីស្មៅមុនពេលវាស្លាប់\cite{17}។ មានរបាយការណ៍មួយបានសរសេរថា\textit{"នៅឆ្នាំ1901 មានការភ្ញាក់ផ្អើលយ៉ាងខ្លាំងពេលរកឃើញសាកសពសត្វម៉ាមម៉ុត់ដែលមានស្ថានភាពយ៉ាងល្អនៅជិតទន្លេបេរេហ្សូវកា ដោយសារសត្វនេះមើលទៅដូចបានសម្លាប់ដោយសារទឹកកកនៅចំរដូវក្តៅទៅវិញ។ អាហារនៅក្នុងពោះរបស់វាស្ថិតក្នុងស្ថានភាពយ៉ាងល្អ ហើយមានទាំងផ្ការសណ្តែកព្រៃផងដែរ នេះមានន័យផ្ការទាំងនេះត្រូវបានវាសុីនៅចុងខែកក្កដាឬដើមខែសីហា។ សត្វនេះបានស្លាប់យ៉ាងឆាប់រហ័សសូម្បីតែស្មៅនិងផ្ការក៏មាននៅក្នុងថ្គាមវាដែរ។ ហាក់បីដូចជាមានកម្លាំងមួយដ៏ខ្លាំងខ្លាបានវាប្រហារ និងបោះវាពីរបីគីឡូម៉ែត្រពីតំបន់ដែលវាសុីចំណី។ ត្រគាកនិងជើងមួយចំហៀងរបស់វាបានបាក់ ប្រហែលជាសត្វដ៏ធំមួយនេះត្រូវបានកករហូតដល់ស្លាប់ នៅក្នុងអំឡុងពេលដែលក្តៅទៅវិញ"} \cite{18}។ ម្យ៉ាងទៀត\textit{"[អ្នកវិទ្យាសាស្ត្ររុស្ស៊ី] បានកត់ត្រាថាស្រទាប់ពោះខាងក្នុងបំផុតរបស់សត្វនេះមានរចនាសម្ព័ន្ធសរសៃនៅល្អនៅឡើយ មានន័យថាកម្តៅនៅក្នុងរាងកាយរបស់វាត្រូវបានយកចេញដោយដំណើរការធម្មជាតិដ៏អស្ចារ។ លោក Sanderson បានចាប់អារម្មណ៍យ៉ាងខ្លាំងចំពោះបញ្ហានេះ គាត់ក៏បានបានទាក់ទងទៅស្ថាប័នអាហារត្រជាក់សហរដ្ឋអាមេរិក(AFFI)៖ តើហេតុអ្វីបានជាសត្វម៉ាមម៉ុត់កកទាំងស្រុង សូម្បីតែជាតិសំណើមនៅក្នុងស្រទាប់ពោះក្នុងបំផុតរបស់វាគ្មានពេលសម្រាប់បង្កើតគ្រីស្តាល់ធំល្មមដែលអាចបំផ្លាញរចនាសម្ព័ន្ធនៃសរសៃសាច់?... ប៉ុន្មានសប្តាហ៍ក្រោយមកស្ថាប័ននេះបានឆ្លើយតបទៅលោក Sanderson ថា៖ វាមិនអាចទៅរួចនោះទេ។ ចំណេះដឹងនឹងបច្ចេកវិទ្យាទាំងអស់ដែលយើងមាន គ្មានវិធីណាមួយដែលអាចយកកម្ដៅចេញពីសាកសពដ៏ធំដូចម៉ាមម៉ុត់បានលឿនគ្រប់គ្រាន់ដើម្បីអោយវាកកដោយគ្មានការបង្កើតដំគ្រីស្តាល់ធំៗនៅក្នុងសាច់វានោះទេ។ ម្យ៉ាងទៀតទោះបីខិតខំប្រើបច្ចេកទេសវិទ្យាសាស្រ្តនិងវិស្វកម្មពួកគេសង្គេតទៅធម្មជាតិហើយបានបញ្ជាក់ថា តាមការយល់ដឹងរបស់មនុស្សសព្វថ្ងៃគ្មានរបៀបណាមួយដែលអាចធ្វើវាបាននោះឡើយ"}\cite{19}។

\section{ (Grand Canyon)}

The Grand Canyon ដែលជាផ្នែកមួយនៃតំបន់ Great Basin នៅភាគនិរតីអាមេរិកខាងជើង គឺជាបាតុភូតធម្មជាតិមួយផ្សេងទៀតដែលបញ្ជាក់ពីប្រភពមហន្តរាយ(រូបភាព \ref{fig:2})។ ជាដំបូង ស្រទាប់ថ្មខ្សាច់ និងថ្មកំបោរដែលបង្កើត Grand Canyon មានផ្ទៃដីរហូតដល់2.4លាន គ.ម $^2$ \cite{21}។ រូបភាពទី \ref{fig:3} បង្ហាញអំពីផ្ទៃដីនៃស្រទាប់ខ្សាច់ Coconino នៃផ្នែកខាងលិចសហរដ្ឋអាមេរិក។ ស្រទាប់ផ្តេកដ៏ធំទូលាយនេះដែលមានរូបធាតុដូចគ្នាហាក់បីដូចជាត្រូវបានបង្កើតឡើងក្នុងពេលតែមួយដង។

\begin{figure}[b]
\begin{center}
% \fbox{\rule{0pt}{2in} \rule{0.9\linewidth}{0pt}}
   \includegraphics[width=1\linewidth]{grand-canyon.jpg}

\end{center}
   \caption{The (Grand Canyon), នៅរដ្ឋអារហ្សូណា សហរដ្ឋអាមេរិក \cite{49}.}
\label{fig:2}
\label{fig:onecol}
\end{figure}

\begin{figure}[t]
\begin{center}
% \fbox{\rule{0pt}{2in} \rule{0.9\linewidth}{0pt}}
   \includegraphics[width=1\linewidth]{coconino.jpg}
\end{center}
   \caption{ទំហំស្រទាប់ថ្មខ្សាច់ Coconino នៅភាគខាងលិចសហរដ្ឋអាមេរិក\cite{21}.}
\label{fig:3}
\label{fig:onecol}
\end{figure}

បើយើងសង្កេតទៅលើ Grand Canyon វាអាចពន្យល់យើងអំពីកំណនៃស្រទាប់ដីល្បាប់ ក៏បានកើតឡើងជាដោយសារតែកម្លាំងធូរតិចតូនិចផងដែរ។ ដើម្បីស្វែងយល់អំពីរឿងនេះយើងត្រូវសង្កេតតំបន់ខ្លះៗនៃ Canyon អោយបានច្បាស់លាស់ទៅលើស្រទាប់ដីល្បាប់ដែលបានបត់បែន។ អ្នកស្រាវជ្រាវមកពី Answers in Genesis\cite{42} បានសិក្សាខ្នាតមីក្រូស្កូពិកទៅលើសំណំថ្មមួយចំនួនដូចជា Monument Fold ហើយដោយសារតែខ្វះខាតនៃលក្ខណៈពិសេសដែលគួរតែមាន បើសិនជាសំណុំថ្មបត់បែនទាំងនេះបានកើតនៅក្នុងរយៈពេលយូរនៅក្រោមកម្តៅនិងសំពាធ។ អ្នកស្រាវជ្រាវទាំងនេះបានសន្និដ្ឋានថា ស្រទាប់ដីល្បាប់ត្រូវបានបត់បែនដោយកម្លាំងធូរតិចតូនិច នៅពេលដែលវានៅទន់ នោះគឺបន្ទាប់ពីវាបានកើតជាកំណរ\cite{43}។

\begin{figure*}
\begin{center}
% \fbox{\rule{0pt}{2in} \rule{.9\linewidth}{0pt}}
\includegraphics[width=1\textwidth]{Grand_Staircase-big.jpg}
\end{center}
   \caption{ស្រទាប់ដីល្បាប់ដែលបង្កើតទៅជា Grand Canyon (រូបភាពខាងស្តាំ) ស្របទៅរកតំបន់ Cedar Breaks, Utah (រូបភាពខាងឆ្វេង) បានបង្ហាញថាថ្មទាំនេះបត់ឡើងលើ\cite{50}.}
\label{fig:4}
\end{figure*}

នៅពេលដែលយើងពង្រីកមើល ឃើញថាស្រទាប់ដែលបង្កើត Grand Canyon មិនត្រឹមតែបត់កោងនៅក្នុង Grand Canyon នោះទេ។ ស្រទាប់ទាំងនេះបានបត់កោងទៅភាគខាងកើតនៃតំបន់ East Kaibab Monocline\cite{46} ហើយក៏បត់ទៅជើងនៃ Cedar Breaks រដ្ឋ Utah ផងដែរ (រូបភាពទី \ref{fig:4})។ មានន័យថាស្រទាប់ទាំងនេះត្រូវបានបត់លើគ្នាយ៉ាងលឿនដោយកម្លាំងតេតូនិច។ យើងអាចយោងទៅតាមស្រទាប់ផ្តេកនៃ Grand Canyon ដែលមានកម្រាស់ប្រហែល 1700 ម៉ែត្រនេះបាន។ ទំហំនៃដំណើរការភូគព្ភសាស្ត្រដែលចំបាច់សម្រាប់ដាក់ស្រទាប់ដីល្បាប់ដែលមានកម្រាស់ជាងមួយគីឡីម៉ែត្រនេះគឺធំធេងណាស់។ 

ពត៍មានជាក់លាក់នៃ Grand Canyon គឺជាបញ្ហាមួយទៀតនៃការពិភាក្សាសម្រាប់វិទ្យាសាស្ត្រសម័យទំនើប។ ទ្រឹស្តីវិទ្យាសាស្ត្រតាមលំនាំបានបង្ហាញថា Grand Canyon ត្រូវបានឆ្លាក់ដោយទន្លេ Colorado រាប់លានឆ្នាំមកហើយ\cite{47}។ យ៉ាងណាក៏ដោយក្រុមស្រាវជ្រាវរបស់ Answers in Genesis ជឿថា Grand Canyon ត្រូវបានបង្កើតឡើងក្នុងរយៈពេលតែពីរបីសប្តាហ៍ដោយសារច្រោះដីហៀរចេញពីបឹងបុរាណមួយដែលបណ្តាលអោយនាំចេញដីល្បាប់ជាច្រើន ខណៈដែលវាបានឆ្លាក់ Canyon ទាំងនេះ។ ភស្តុតាងជាក់លាក់អំពីបឹងដែលនៅលើតំបន់ខ្ពស់ខាងកើត Grand Canyon គឺអាចរកឃើញតាមរយៈដីល្បាប់និងផូសុីល។ ការប្រៀបធៀប Grand Canyon ជាមួយនិងឧទាហរណ៍ខ្នាតធំៗផ្សេងទៀតនៃការច្រោះដីហៀរចេញដូចជា Afton Canyon និងភ្នំ St. Helens បង្ហាញពីភាពស្រដៀងគ្នានៃភូមិសាស្ត្រ និងបង្ហាញថា Canyon ធំៗអាចបង្កើតបានដោយសារហូរទឹកទ្រង់ទ្រាយធំ\cite{48}។

បើសិនជាយើងពិចារណាមូលដ្ឋាននៃដំណើរការភូមិសាស្ត្រដែលចំបាច់សម្រាប់បង្ហូរដីល្បាប់ទៅលើទំហំដីដ៏សែនមហិមានេះ ជាមួយនិងទំហំដ៏សែនតូចនៃទន្លេ Colorado និងទំហំដ៏មហិមានៃ Grand Canyon យើងអាចសន្និឋានបានថាវាមិនត្រូវបានបង្កើតបន្តិចម្តងៗនោះទេ។

\section{ទីក្រុងក្រោមដី Derinkuyu}

ក្រៅពីរ៉ាមីត ឧទាហរណ៍ដ៏ល្អមួយនៃវិស្វកម្មបុរាណគឺទីក្រុងក្រោមដី Derinkuyu (រូបភាពទី\ref{fig:5}) ដែលស្ថិតនៅក្នុងតំបន់ Cappadocia ប្រទេសតួកគី។ វាជាទីក្រុងដែលធំជាងគេក្នុងចំណោមជម្រកក្រោមដីជាង200កន្លែងនៅតំបន់នេះ\cite{54}។ ទីក្រុងក្រោមដីនេះត្រូវបានគេប៉ាន់ប្រមាណថាអាចមានប្រជាជនរស់នៅដល់ទៅ20000នាក់និងមាន១៨ជាន់ ហើយវាមានជ្រៅរហូតដល់85ម៉ែត្រ។ ទោះយើងមិនដឹងពីអាយុពិតរបស់វាក៏ដោយ បន្តែគេបានប៉ាន់ប្រមាណថាវាមានអាយុយ៉ាងតិច2800ឆ្នាំ។ ទីក្រុងនេះត្រូវបានគេឆ្លាក់ចេញពីថ្មភ្នុំភ្លើងទន់ \cite{52, 53}។

\begin{figure}[b]
\begin{center}
% \fbox{\rule{0pt}{2in} \rule{0.9\linewidth}{0pt}}

   \includegraphics[width=1\linewidth]{derinkuyu.jpeg}
\end{center}
   \caption{គំនូនៃទីក្រុងក្រោមដី Derinkuyu\cite{56}.}
\label{fig:5}
\label{fig:onecol}
\end{figure}

មូលហេតុដែល Derinkuyu គួរអោយចាប់អារម្មណ៍គឺដោយសារតែមិនដឹងថាមូលហេតុអ្វីបានជាមានសហគមន៍មួយ បានសម្រេចចិត្តសាងសង់ទីក្រុងក្រោមដីនេះទាំងមូល។ ដើម្បីបង្កើតកន្លែងរស់នៅក្រោមដី គេត្រូវឆ្លាក់រូងទាំងអស់ចេញពីថ្ម។ សភាពរូងភ្នុំបង្ហាញយ៉ាងច្បាស់ថាវាត្រូវបានគេឆ្លាក់ដោយកម្លាំងកាយ មិនមែនប្រើឧបករណ៍ជំនួយនោះទេ ដែលធ្វើអោយវាមានការពិបាកយ៉ាងខ្លាំងណាស់បើប្រៀបធៀបទៅនិងការសាងសាងលំនៅឋ្ឋាននៅលើដី។ ជាក់ស្តែង យើងមិនទាន់យល់ច្បាស់នោះទេថាហេតុអ្វីបានជាមានមនុស្សសម្រេចចិត្តរស់នៅក្រោមដីទៅវិញ ខណៈដែលកសិកម្ម ពន្លឺថ្ងៃ ធម្មជាតិ និងការស្វែងរកមានតែនៅលើផ្ទៃដីប៉ុណ្ណោះ។ តាមទ្រឹស្តី "ប្រវត្តិសាស្រ្ត" ធម្មតាបានសន្និដ្ឋានថា Derinkuyu ត្រូវបានសាងសង់ឡើងដោយគ្រិស្ដសាសនិកដែលត្រូវការទីកន្លែងសម្ងាត់សម្រាប់ប្រតិបត្តិជំនឿរបស់ខ្លួន\cite{53}។ ជាធម្មតាវាមិនងាយស្រួលនោះទេ ព្រោះសម្រាប់មនុស្សយើងទូទៅនៅពេលមានសត្រូវគឺយើងត្រូវ "ប្រយុទ្ធឬរត់" មិនមែន "ឆ្លាក់ទីក្រុងក្រោមដីចេញពីថ្ម" នោះទេ។

ទំហំ ជម្រៅ និងការពិចារណាទៅលើការរចនានៃទីក្រុងក្រោមដីនេះបង្ហាញថាវាមិនត្រូវបានសាងសង់ដើម្បីជាបន្ទាយទាហានបណ្តោះអាសន្នសម្រាប់ការពារពួកឈ្លានពាននាគ្រាមានអាសន្ននោះទេ ប៉ុន្តែគឺដើម្បីជាជម្រកការពារពីគ្រោះមហន្ថរាយនៅលើផ្ទៃដីទៅវិញ។ Derinkuyu មិនត្រឹមតែមានបន្ទប់គេង ផ្ទះបាយ បន្ទប់ទឹកនោះទេ មានទាំងរោងសត្វ ពាងទឹក បន្ទប់ផ្ទុកអាហារ ឧបករណ៍ចម្រោះស្រា និងប្រេង សាលា វិហារ ផ្នូរ និងរន្ធសម្រាប់ខ្យល់ចេញចូលផងដែរ (រូបភាពទី\ref{fig:6})។ បើសិនវាជាបន្ទាយទាហានមែននោះ ចាំបាច់រូវការឧបករណ៍ចម្រោះស្រា ហើយត្រូវសាងសង់ជ្រៅរហូតដល់85ម៉ែត្រយ៉ាងស្មុគស្មាញបែបនេះទៅវិញ?

ការពន្យល់ដែលគួរអោយជឿជាក់បំផុតនៃការបង្កើត Derinkuyu គឺអាចសន្និដ្ឋានបានថា វាជាការរៀបចំរយៈពេលវែងនិងជាជម្រកដែលអាចអាស្រ័យផលបានដើម្បីការពារពីកម្លាំងមហន្តរាយនៃភូគព្ភសាស្ត្រលើផ្ទៃផែនដី។

\begin{figure}[t]
\begin{center}
% \fbox{\rule{0pt}{2in} \rule{0.9\linewidth}{0pt}}
   \includegraphics[width=1\linewidth]{derinkuyu-air.jpg}
\end{center}
   \caption{រណ្តៅខ្យល់ជ្រៅមួយនៅក្នុង Derinkuyu\cite{53}.}
\label{fig:6}
\label{fig:onecol}
\end{figure}

% \section{អាណោមាលីបន្ថែមដែលបានអថិប្បាយយ៉ាងល្អបំផុតដោយការបង្វិលផែនដី}

% មុនពេលបញ្ចប់ យើងនឹងលើកឡើងពីអាណោមាលីវិទ្យាសាស្ត្របន្ថែមដែលនៅពេលបានមើលក្នុងបរិបទកម្លាំងភូមិសាស្ត្រដ៏អស្ចារ្យ វាត្រូវបានបកស្រាយយ៉ាងល្អ។ 

\section{ការប្រមូលផ្ដុំជីវម៉ាស់}

ការប្រមូលផ្តុំជីវម៉ាស់នៃប្រភេទសត្វនិងរុក្ខជាតិផ្សេងៗគ្នា ជាធម្មតាត្រូវបានរកឃើញជាផូសុីលនៅក្នុងស្រទាប់ដីល្បាប់ ដែលជារឿងដ៏មិនប្រក្រតីមួយផ្សេងទៀត។ នៅក្នុង "Reliquoæ Diluvianæ" លោក Rev. William Buckland បានកត់ត្រាអំពីការរកឃើញនូវសត្វជាច្រើនប្រភេទដែលមិនគួររកឃើញក្នុងពេលតែមួយ ដែលបានខ្ចាតខ្ចាយពេញប្រទេសអង់គ្លេសនិងអឺរ៉ុប ដែលបានកប់នៅស្រទប់ដីល្បាប់ 'diluvium'\cite{58}។ ការប្រមូលផ្តុំនៃសាកសព្វសត្វបែបនេះត្រូវបានរកឃើញនៅក្នុងរូងភ្នុំ Skjonghelleren នៅកោះ Valdroy ប្រទេសន័រវែស។ នៅក្នុងរូងនេះមានឆ្អឹងលើសពី7000 នៃថនិកសត្វ បក្សុី និងត្រី ដែលត្រូវបានរកឃើញនៅក្នុងស្រទាប់ជាច្រើននៃដីល្បាប់\cite{59}។ ឧទាហរណ៍មួយទៀតគឺនៅតំបន់ San Ciro "Cave of the Giants" នៅប្រទេសអុីតាលី។ នៅក្នុងរូងនេះគេបានរកឃើញឆ្អឹងថនិកសត្វជាច្រើនតោន ជាពិសេសសត្វដំរីទឹកដែលនូវមានសភាពល្អយ៉ាងខ្លាំងដែលត្រូវបានគេកាត់ធ្វើជាគ្រឿងលម្អ និងនាំចេញដើម្បីផលិតចង្កៀងខ្មៅ (Lamp black)។ តាមការរាយការណ៍ ឆ្អឹងសត្វផ្សេងៗទាំងនោះត្រូវបានលាយគ្នា និងបាក់បែកបាក់ជាបំណែកៗ\cite{60,61}។ នៅក្នុងទីក្រុងបុរាណ Mendes ប្រទេសអេហ្ស៊ីប ក៏បានរកឃើញឆ្អឹងសត្វប្រភេទផ្សេងៗដែលបានលាយច្របល់ជាមួយគ្នាទៅនឹងដីឥដ្ឋកែវ (glassy clay)\cite{57}។ ការរកឃើញបែបនេះអាចធ្វើអោយមានចម្ងល់ ប៉ុន្តែអាចពន្យល់បានយ៉ាងងាយស្រួល គឺដោយសារមានទឹកជំនន់ដ៏ធំមហិមាដែលបង្ករអោយឆ្នឹងសត្វគ្រប់ប្រភេទហូរឬកប់ទាំងរស់ទៅក្នុងស្រទាប់ដីល្បាប់ឬនៅក្នុងរូងភ្នុំ ហើយសម្រាប់ករណីជីវម៉ាស់ដែលក្លាយជាកែវ (vitrified biomass) នៅអេហ្ស៊ីបគឺកើតឡើងដោយចំហាយអគ្គិសនីនៅក្នុងស្រទាប់ម៉ង់តូលបន្ទាប់ពីទឹកជំនន់ដ៏ធំមហិមា។ រូបភាពទី\ref{fig:7} បង្ហាញពីជីវម៉ាស់ 'muck' នៅរដ្ឋអាឡាស្កា\cite{56}។

\begin{figure}[t]
\begin{center}
% \fbox{\rule{0pt}{2in} \rule{0.9\linewidth}{0pt}}
   \includegraphics[width=1\linewidth]{muck-crop.jpeg}
\end{center}
   \caption{ជីវម៉ាស់អាឡាស្កា 'muck' ដែលផ្សុំពីបំណែកឈើ រុក្ខជាតិ និងសត្វផ្សេងៗនៅក្នុងក្សាច់ឬទឹកកក\cite{146}។}
\label{fig:7}
\label{fig:onecol}
\end{figure}
\section{ត្រង់សេបុរាណ}

បុព្វជនរបស់យើងបានបន្សល់ទុកនូវសំណង់វិស្វកម្មបុរាណជាច្រើន ហើយនៅក្នុងនោះយើងបានរកឃើញសាកសពមនុស្សដែលជាទូទៅត្រូវបានចាត់ទុកថាជាផ្នូរ ប៉ុន្តែបើយើងពិនិត្យមើលអោយច្បាស់លាស់វាអាចជាត្រង់សេបុរាណក៏ថាបាន។

\begin{figure}[b]
\begin{center}
% \fbox{\rule{0pt}{2in} \rule{0.9\linewidth}{0pt}}
   \includegraphics[width=1\linewidth]{ww19.jpg}
\end{center}
   \caption{Newgrange ប្រទេសអៀរឡង់ - សូមពិនិត្យមើលទំហំភ្ញៀវទេសចរណ៍នៅច្រកចូលសម្រាប់ការប្រៀបធៀប។}
\label{fig:8}
\label{fig:onecol}
\end{figure}

ឧទាហរណ៍ដ៏ល្អមួយគឺនៅក្នុងតំបន់ Newgrange (រូបភាពទី\ref{fig:8}), ដែលជាវិមានធំជាងគេនៅក្នុងតំបន់ Brú na Bóinne មានសំណង់បុរាណជាច្រើនរួមទាំងផ្នូរបុរាណផងដែរ។ ផ្នូរទាំងនេះមានបន្ទប់បញ្ចុះសពមួយឬច្រើនកន្លែងដែលគ្របដណ្តប់ដោយដី ឬថ្ម ដែលមានផ្លូវចូលតូចៗសាងសង់ដោយថ្មធំៗ\cite{70}។ វាជាឧទាហរណ៍មួយនៃរចនាសម្ព័ន្ធវិស្វកម្មការពារដ៏ធំមហិមា ដែលបានសាងសង់អស់ជាច្រើនជំនន់ដើម្បីធ្វើជាផ្នូរអោយមនុស្សពីរបីនាក់ដែលបានស្លាប់មុនផ្នូរបុរាណទាំងនេះត្រូវបានចាប់ផ្តើមសាងសង់។ នៅពេលដែលកន្លែងនេះត្រូវបានរកឃើញឡើងវិញដោយម្ចាស់ដីក្នុងឆ្នាំ1699 កន្លែងនេះគឺត្រូវបានកប់ដោយដីទាំងស្រុង។

គ្រាន់តែពិនិត្យមើលសំណង់នេះតែបន្តិចក៏ដឹងដែរថា ការសាងសង់វាគឺចំណាយកម្លាំងមិនតិចនោះទេ។ ដើម្បីសាងសង់ Newgrange គេត្រូវចំណាយសម្ភារ:រហូតទៅដល់ 200000តោន។ ខាងក្នុងនោះ\textit{“... មានផ្លូវបន្ទប់សាកសពដែលអាចចេញចូលបានតាមច្រកអាគ្នេយ៍។ ផ្លូវនេះមានប្រវែងប្រហែល19ម៉ែត្រ ឬ1/3នៃផ្លូវសរុបទៅកាន់ផ្នែកកណ្ដាល។ នៅចំនុចចុងផ្លូវមានបន្ទប់តូចចំនួនបីដែលជាប់នឹងបន្ទប់ធំមួយទៀតហើយមានដំបូលក្កោបខ្ពស់ធ្វើពីសិលា... ជញ្ជាំងនៃផ្លូវនេះ សាងសង់ដោយដុំថ្មធំៗហៅថា orthostat ដែលមាន22នៅខាងលិចនិង21នៅខាងកើត។ វាមានកម្ពស់មធ្យម1.5ម៉ែត្រ”}\cite{70}។ សព្វថ្ងៃនៅមានស្នាកស្មាមអំពីវិស្វកម្មការពារទឹកយ៉ាងល្អផងដែរ។ ជាឧទាហរណ៍នៅលើដំបូល\textit{“នៅតាមចន្លោះដំបូលត្រូវបានបិតដោយល្បាយដីដុតនិងដីខ្សាច់ដើម្បីការពារទឹកភ្លៀង ហើយបើតាមរយៈកាលបរិច្ឆេទកាបូន (radiocarbon dating) ល្បាយផ្នូរនេះបានរកឃើញថាវាត្រូវបាងសាងសង់នៅចន្លោះឆ្នាំ2500មុនគ្រឹស្តសករាជ"}\cite{71}។ ណាមួយទៀត ការដែលដំឡើងកំពស់ទៅកាន់បន្ទប់កណ្តាលគឺត្រូវបានធ្វើឡើងសម្រាប់គោលបំណងដូចគ្នា៖\textit{“ព្រោះថា\_កម្រាល\_នៃផ្លូវនិងបន្ទប់សាកសពនេះបានធ្វើតាមកំពស់ដីដែលសំណង់នេះត្រូវបានកសាងឡើង គឺខុសគ្នាជិត2ម៉ែត្ររវាងច្រកចូលនិងផ្ទៃខាងក្នុងនៃបន្ទប់”}\cite{71}។

\begin{figure}[b]
\begin{center}
% \fbox{\rule{0pt}{2in} \rule{0.9\linewidth}{0pt}}
   \includegraphics[width=1\linewidth]{dolmen.jpg}
\end{center}
   \caption{Dolmen de Soto, អ៉េស្ប៉ាញ \cite{53}.}
\label{fig:9}
\label{fig:onecol}
\end{figure}

អ្វីដែលគួរអោយមានចម្ងល់នោះគឺមិនសូវមានកាកសំណល់នៃសាកសព្វនោះឡើយ។ តាមការស្រាយជ្រាវបានបង្ហាញថា មានបំណែកឆ្អឹងឆេះនិងឆ្អឹងធម្មតាដែលជារបស់មនុស្សតែពីរបីនាក់តែប៉ុន្នោះដែលរាយប៉ាយតាមផ្លូវសំណង់មួយនេះ។ បើយោងទៅតាមកាលបរិច្ឆេទកាបូននៃសម្ភារៈនៅខាងក្នុងបានបង្ហាញថា ការសាងសង់ Newgrange គឺត្រូវបានចំណាយពេលយ៉ាងតិចពិរបីជំនាន់។ ហេតុអ្វីបានជាសហគមន៍បុរាណមួយនេះប្រឹងប្រែងសាងសង់ផ្នូរបុរាណពពេញទៅដោយវិស្វកម្មដ៏មហិមានេះ ប៉ុន្តែបានត្រឹមរាយប៉ាយបំណែកឆ្អឹងសាកសព្វតែពីរបីនាក់ទៅវិញ? មានន័យថាសំណង់បុរាណដែលបានសាងសង់មិនអោយជ្រាបទឹកទាំងនេះគឺកសាងដើម្បីជាជម្រកការពារគ្រោះមហន្ថរាយលើផែនដីដែលតាំងតែកើតមានឡើង។

ឧទាហរណ៍មួយទៀតគឺនៅ Dolmen de Soto នៃតំបន់ Huelva ផ្នែខាងត្បូងនៃអ៉េស្ប៉ាញ (រូបភាពទី \ref{fig:9}) ដែលជាតំបន់មួយក្នុងចំណោ200តំបន់ផ្សេងទៀត\cite{72,32} វាជាសំណង់វិស្វកម្មដ៏ឆ្នើម សាងសង់ឡើងដោយប្រើប្រាស់ដុំថ្មបុរាណធំៗមានអង្កត់ផ្ចិត75ម៉ែត្រ។ យោងតាមការស្រាវជ្រាវ មានតែសាកសពចំនួនប្រាំបីនាក់នោះទេដែលត្រូវបានរកឃើញ សាកសព ទាំងនេះគឺត្រូវបានគេបត់ដូចពេលទារកនៅក្នុងស្បូនមុនពេលគេកប់។

\section{ការលើកឡើងអំពីភាពមិនប្រក្រតីដែលគួរអោយគត់សំគាល់}

នៅក្នុងផ្នែកនេះ ខ្ញុំបានសង្ខេបនូវការលើកឡើងអំពីភាពមិនប្រក្រតីដែលគួរអោយគត់សម្គាល់មួយចំនួនដែលបានបកស្រាយយ៉ាងក្បោះក្បាយនូវមហន្តរាយដូចនិង ECDO ផងដែរ។

\subsection{ភាពមិនប្រក្រតីនៃជីវវិទ្យា}

\begin{figure}[b]
\begin{center}
% \fbox{\rule{0pt}{2in} \rule{0.9\linewidth}{0pt}}
   \includegraphics[width=1\linewidth]{bottleneck.jpg}
\end{center}
   \caption{ការបង្ហាញអំពីការថយចុះនៃចំនួនបុរស95\% ប្រហែល6000ឆ្នាំមុន\cite{62}.}
\label{fig:10}
\label{fig:onecol}
\end{figure}

ភាពមិនប្រក្រតីជីវវិទ្យាដែលគួរអោយគត់សំគាល់មួយចំនួន គឺការកាត់បន្តយនៃហ្សែននិងផូសុីលត្រីប៉ាឡែន។ Zeng et al (2018) បើយោងទៅតាមគំរូលំដាប់ក្រូម៉ូសូម Y-125 ពីមនុស្សសម័យក្រោយ និងផ្តោតទៅលើភាពស្រដៀងនិងការប្រែប្រួលនៅក្នុង DNA បានបង្ហាញថា ចំនួនប្រជាជនបុរសប្រហែល95\% ត្រូវបានកាត់បន្តយកាលពី5000ទៅ7000ឆ្នាំមុន (រូបភាពទី \ref{fig:10})\cite{62}។ ផូសុីលត្រីប៉ាឡែនត្រូវបានរកឃើញនូវកំពស់រាប់រយម៉ែត្រលើសពីកំពស់សមុទ្រនៅរដ្ឋ Swedenborg, Michigan, Vermont ប្រទេសកាណាដា ឈីលី និងអ៊ីហ្ស៊ីប\cite{63,64,65,66}។ ត្រីប៉ាឡែនទាំងនេះត្រូវបានរកឃើញស្ថិតនៅក្នុងស្ថានភាពខុសៗគ្នា៖ ខ្លះមានរូបរាងពេញលេញ នៅក្នុងភក់ដែលស្ថិតនៅលើផែនទឹកកក ឬក៏កប់នៅក្នុងស្រទាប់ល្បាប់ដី។ ចំនួនសត្វនៅតំបន់ទាំងនេះមានចាប់ពីពីរបីក្បាលរហូតដល់រាប់រយ។ ត្រីប៉ាឡានជាសត្វសមុទ្រដែលកម្រមកជិតឆ្នេរណាស់ មូលហេតុអ្វីបានជាត្រីប៉ាឡែនទាំងនេះត្រូវបានប្រទះឃើញនៅលើតំបន់ខ្ពស់ៗនិងឆ្ងាយពីសមុទ្របែបនេះទៅវិញ?

ការវិនាសសាបសូន្យលើផែនដីបានកើតឡើងជាច្រើនដងកាលពីមុន ហើយការវិនាសសាបសូន្យធំៗដែលយើងបានស្រាវជ្រាវបានច្រើនបំផុតគឺ "ប្រាំធំៗ" ប្រព្រឹត្តិការណ៍ Phanerozoic៖ ប្រព្រឹត្តិការណ៍ចុងសម័យ Late Ordovician (LOME), Late Devonian (LDME), en-Permian (EPME), en-Triassic (ETME) និង en-Cretaceous (ECME)\cite{88,89}។ អ្វីដ៏គួរអោយចាប់អារម្មណ៍បំផុតនោះ ការវិនាសសាបសូន្យមួយចំនួនទាំងនេះត្រូវបានគេចាត់ទុកថាបានកើតឡើងស្របពេលនិងការបង្កើតស្រទាប់ Grand Canyon ផងដែរ ជាពិសេសគឺនៅក្នុងស្រទាប់ Permian និង Devonian។

\subsection{ភាពមិនប្រក្រតីរូបរាង}

\begin{figure}[b]
\begin{center}
% \fbox{\rule{0pt}{2in} \rule{0.9\linewidth}{0pt}}
   \includegraphics[width=1\linewidth]{columbia.jpg}
\end{center}
   \caption{ចលនារលកធំៗនៅក្នុង Glacial Lake Columbia នៅរដ្ឋវ៉ាស៊ីងតោន\cite{80}។}
\label{fig:11}
\label{fig:onecol}
\end{figure}

ក្រៅពី Grand Canyon មានទីតាំងផ្សេងៗជាច្រើនទៀតដែលបានបង្កើតឡើងដោយកម្លាំងមហន្តរាយនៃធម្មជាតិ។ ភស្តុតាងអំពីការហូរទឹកដ៏ធំធេងត្រូវបានគេរកឃើញតាមចរន្តស្នាមហូរទូទាំងពិភពលោក។ ឧទាហរណ៍គឺនៅតំបន់ Channeled Scablands នៅរដ្ឋ Northwest។ នៅទីនេះយើងមិនត្រឹមតែឃើញទីតាំងល្បាយដីខ្សាច់និងដុំថ្មផ្នត់ៗនោះទេ មានទាំងលំដាប់ស្នាមហូររាប់រយដែលបានបង្កើតឡើងដោយចរន្តទឹកធំៗ\cite{78,79}។ វាជាស្នាមចរន្តទឹកហូរធំៗដែលបានបង្កើតឡើងនៅក្នុងស្រទាប់ដីខ្សាច់នៃទឹកហូរ។ ស្នាមចរន្តទឹកហូរទាំងនេះយើងអាចរកវាបានទូទាំងពិភពលោក ដូចជានៅប្រទេសប៉ារាំង អ៉ាស្សង់ទីន រ៉ូស្សី និង អាមេរិកខាងជើង\cite{81}។ រូបភាពទី\ref{fig:11} បង្ហាញអំពីស្នាមចរន្តទឹកហូរនៅក្នុងរដ្ឋ Washington\cite{80}។

\begin{figure}[b]
\begin{center}
% \fbox{\rule{0pt}{2in} \rule{0.9\linewidth}{0pt}}
   \includegraphics[width=1\linewidth]{zhangjiajie.jpg}
\end{center}
   \caption{ថ្មបញ្ឈរធំៗនៃព្រៃអភិរក្សនៅ Zhangjiajie ខាងត្បូងប្រទេសចិន។}
\label{fig:12}
\label{fig:onecol}
\end{figure}

\begin{figure}[b]
\begin{center}
% \fbox{\rule{0pt}{2in} \rule{0.9\linewidth}{0pt}}

   \includegraphics[width=1\linewidth]{hoy.jpg}
\end{center}
   \caption{ថ្មបញ្ឈរឆ្នេរសមុទ្រនៃ Old Man of Hoy ប្រទេសស្កុតឡង់ដ៍\cite{83}.}
\label{fig:13}
\label{fig:onecol}
\end{figure}

រចនាសម្ព័ន្ធការហូរច្រោះក៏ត្រូវបានពន្យល់យ៉ាងច្បាស់ដូចនិងទ្រឹស្តី "ក្រឡាប់ផែនដី" របស់ ECDO ផងដែរ។ ឧទាហរណ៍ដ៏ល្អមួយនៃការហូរច្រោះ គឺនៅកន្លែងថ្មស្រួចផ្នែកខាងត្បូងប្រទេសចិន\cite{82}។ ទីតាំងទាំងនេះមានទាំង ប៉ុមថ្មស្រួច កំពូលថ្មស្រួច ស្ពានធម្មជាតិ ជ្រលងភ្នុំ ប្រព័ន្ធរូងភ្នុំធំៗ និងរន្តៅ។ ក្នុងចំណោមទីតាំងដ៏ពិសេសទាំងនេះ គឺព្រៃអភិរក្ស Zhangjiajie ដែលមានសសរក្វាតថ្មខ្សាច់ធំៗ(រូបភាពទី \ref{fig:12})\cite{84}។ សសរថ្មខ្សាច់ទាំងនេះមានកំពស់រហូតដល់1000ម៉ែត្រ ហើយមានចំនួនជាង3100ឯនោះ។ ក្នុងចំណោមថ្មទាំងនេះ ថ្មដែលមានកំពស់លើស120ម៉ែត្រគឺមានរហូតដល់ទៅ1000 ហើយសម្រាប់កំពស់300ម៉ែត្រគឺមានចំនួន45\cite{85}។ សសរថ្មទាំងនេះមានរូបរាងដូចសសរថ្មដែលបង្កើតឡើងដោយសសរច្រោះនៃសមុទ្រ(រូបភាពទី \ref{fig:13}) ដែលជាថ្មឆ្នេរបង្កើតឡើងដោយការបាក់បែកជំវិញ ដែលបង្ករឡើងដោយរលកសមុទ្រ។ យើងអាចស្វែងរកទីតាំងនៃការច្រោះប្រភេទបែបនេះបាននៅ Urgup ប្រទេសតួកគី និងនៅ Ciudad Encantada ប្រទេសអេស្ប៉ាញ ដែលមានកម្ពស់លើសសមុទ្ររហូតដល់1000ម៉ែត្រ។ ទីតាំងទាំងនេះមានទាំងល្បាយអំបិលនិងផូសុីលសត្វសមុទ្រដែលនៅជិតៗទីនោះ ដែលបង្ហាញអំពីការលុកលុយនៃសត្វសមុទ្រកាលពីមុន\cite{15,86,87}។ រឿងព្រេងទឹកជំនន់\cite{3}បានរៀបរាប់ថារលកសមុទ្រមានកំពស់ខ្ពស់ជាង1000ម៉ែត្រ ដែលយើងទទួលស្គាល់បានតាមរយៈអំបិលនិងលានអំបិលធំៗនៅលើភ្នុំ Andes និង Himalayas ដែលមានកំពស់ខ្ពស់ជាងសមុទ្រជាច្រើនគីឡូម៉ែត្រ។ ឧទាហរណ៍វាលអំបិល Uyuni នៅប្រទេសបុូលីវីមានកំពស់ខ្ពស់ជាងសមុទ្រដល់ទៅ3653ម៉ែត្រ\cite{94}។

\subsection{ព្រឹត្តិការណ៍ផ្លាស់ប្ដូរអាកាសធាតុ}

អត្ថបទវិទ្យាសាស្ត្រនាសម័យថ្មីទទួលស្គាល់ថាមានព្រឹត្តិការណ៍ផ្លាស់ប្ដូរអាកាសធាតុពិតមែន បើយោងទៅតាមប្រវត្តថ្មីៗនៃភពផែនដី។ ឧទាហរណ៍សំខាន់ៗចំនួនពីរគឺព្រឹត្តិការណ៍4200និង8200ឆ្នាំ ដែលកើតមានក្នុងពេលតែមួយនៃការថយចុះចំនួនប្រជាជន និងការប៉ះពាល់ដល់ការរស់នៅទៅលើផ្ទៃដីដ៏ធំ។ ព្រឹត្តិការណ៍ទាំងនេះត្រូវបានរក្សាទុកជាភាពមិនប្រក្រតីក្នុងស្រទាប់ទឹកកកនិងដីល្បាប់ ផូសុីលផ្ការថ្ម អុីសូតូប018 លំអង និងទិន្នន័យស្ពេលូថឹមនិងកម្រិតទឹកសមុទ្រ។ តាមការបកស្រាយនៃព្រឹត្តិការណ៍ផ្លាស់ប្ដូរអាកាសធាតុរួមមានការថយចុះនៃសីតុណ្ហភាពផែនដី ភាពរាំងស្ងួត ការផ្លាស់ប្តូរនៃចរន្តទឹកសមុទ្រអ៉ាត្លង់តិចផ្នែកខាងត្បូងនិងការកើនឡើងនៃផែនទឹកកក\cite{90,91,92}។ ព្រឹត្តិការណ៍8200ឆ្នាំគឺកើតឡើងក្នុងពេលតែមួយជាមួយនិងទឹកជំនន់សមុទ្រនៃតំបន់សមុទ្រខ្មៅអំលុងពេល6400ឆ្នាំមុនគ្រឹស្តសករាជ\cite{93}។

\subsection{ភាពមិនប្រក្រតីនៃបុរាណវិទ្យា}

យោងតាមភស្ថុតាំងនៃបុរាណវិទ្យា មានទីក្រុងបុរាណមួយចំនួនបានបង្ហាញអំពីស្រទាប់បូជានិងបំផ្លិតបំផ្លាញ ដែលបង្ហាញអំពីព្រឹត្តិការណ៍គ្រោះមហន្តរាយកាលពីមុន។ ជាឧទាហរណ៍គឺទីក្រុងបុរាណ Jericho ដែលជាទីក្រុងមួយដែលស្ថិតនៅប្រទេសប៉ាលេស្ទីននាពេលបច្ចុប្បុន្ន។ ទីក្រុងនេះមានស្រទាប់បំផ្លាញជាច្រើន ដោយមានការឆាបឆេះជាខ្លាំងនិងការរលំនៃសំណង់ថ្ម\cite{96,97}។ កំណត់ត្រាប្រវត្តិសាស្ត្រនៅក្នុងស្រទាប់ទាំងនេះចាប់តាំងពី9000ទៅ2000ឆ្នាំមុនគ្រឹស្តសករាជ។ អ្វីដែលគួរអោយគត់សម្គាល់នោះគឺប៉មនៃសំណង់ទាំងនេះ វាហាក់បីដូចជាត្រូវគេកាត់ចោលហើយកប់ក្នុងដីល្បាប់កាលពី7400ឆ្នាំមុនគ្រឹស្តសករាជ (រូបភាពទី \ref{fig:14})\cite{95}។ Catal Huyuk \cite{99} Gramalote \cite{98} និងរាជវាំងរបស់ Minoan នៃ Knossos នៅលើកោះ Crete\cite{100,101} គឺសុទ្ធតែជាឧទាហរណ៍នៃទីតាំងបុរាណដែលមានភស្ថុតាងក្នុងស្រទាប់នៃការបំផ្លិតបំផ្លាញ។

\begin{figure}[t]
\begin{center}
% \fbox{\rule{0pt}{2in} \rule{0.9\linewidth}{0pt}}

   \includegraphics[width=1\linewidth]{jericho.jpg}
\end{center}
   \caption{គំនូបុរាណនៃប៉មកំពូលរបស់ Jericho កាលពី7400ឆ្នាំមុនគ្រឹស្តសករាជ\cite{95}.}
\label{fig:14}
\label{fig:onecol}
\end{figure}

ភស្តុតាងដែលបញ្ជាក់ពីព្រឹត្តិការណ៍គ្រោះមហន្តរាយដ៏ធំមួយទៀតដែលរំខានដល់ការអភិវឌ្ឍន៍របស់មនុស្សលោកគឺរូប Nampa ជារូបធ្វើមកពីដីឥដ្ឋដែលត្រូវបានគេរកឃើញជំរៅ100ម៉ែត្រខាងក្រោមកម្អែភ្នុំភ្លើងនៃរដ្ឋ Idaho \cite{102,103}។ លំហូរនៃកម្អែភ្នំភ្លើងដែលបានរកឃើញរូបនេះ ត្រូវបានគេប៉ាន់ស្មានថាកើតឡើងនាអំឡុងពេលចុងសម័យទីបីឬទីបួន ប្រហែលជា2លានឆ្នាំមុន។ ទោះជាយ៉ាងណាកម្អែភ្នំភ្លើងក្នុងតំបន់នេះអាចកើតឡើងនាពេលថ្មីៗនេះក៏ថាបាន។ ការរកឃើញបែបនេះមិនត្រឹមតែបញ្ជាក់ពីព្រឹត្តិការណ៍គ្រោះមហន្ថរាយដែលបំផ្លាញនៃការអភិវឌ្ឍន៍មនុស្សនោះទេ វាថែមទាំងផ្ទុយទៅនឹងការធ្វើកាលបរិច្ឆេទកាបូននាពេលបច្ចុប្បន្នថែមទៀត។

\section{អំពីវិធីសាស្ត្រការធ្វើកាលបរិច្ឆេទកាបូននាបច្ចុប្បន្ន}

មានមូលហេតុសំខាន់ៗជាច្រើនដែលធ្វើអោយយើងមិនមានជឿទៅលើកាលវិទ្យានាពេលបុច្ចប្បន្ន ដែលប្រើប្រាស់ពេលវេលារាប់លានឬរាប់រយលានឆ្នាំឯនោះ។

របាយការណ៍ធម្មតាបានបញ្ជាក់ថា "ប្រេងឥន្ធនៈផូសុីល" ដូចជាធ្យូងថ្មនិងឧស្ម័នធម្មជាតិមានអាយុរាប់រយលានឆ្នាំ\cite{104}។ ផ្ទុយទៅវិញ បើតាមការធ្វើកាលបរិច្ឆេទកាបូនជាក់ស្តែងសម្រាប់ប្រេងឥន្ធនៈនៃឈូងសមុទ្រមិចស៊ិកូ បានរកឃើញថាវាមានអាយុត្រឹមតែ13000ឆ្នាំតែប៉ុណ្ណោះ\cite{105}។ អាយុកាលពាក់កណ្តាល (half-life) នៃកាបូន-14 គឺមានរយៈពេលខ្លីណាស់(5730ឆ្នាំ) ធ្វើអោយវាពុកផុយទាំងស្រុកបន្ទាប់ពីរយៈពេល2-3សែនឆ្នាំ។ យ៉ាងណាក៏ដោយ គេក៏បានរកឃើញថានៅក្នុងធ្យងថ្មនិងផូសុីលអាចមានអាយុរាប់ពាន់ដងយូរជាងនឹងទៅទៀត\cite{106}។ ជាការពិតណាស់ធ្យងថ្មសិប្បនិម្មិតត្រូវបានគេផលិតឡើងនៅក្នុងមន្ទីរពិសោធន៍ ការផលិតនេះអាស្រ័យទៅលើកម្តៅខ្ពស់ ហើយគេអាចផលិតវាបានក្នុងរយៈពេលត្រឹមតែ2-8ខែតែប៉ុណ្ណោះ\cite{107}។

វិធីសាស្ត្រធ្វើកាលបរិច្ឆេទអុីសូតូបក៏មិនប្រាកដថាត្រឹមត្រូវនោះដែរ។ ក្រុមស្រាវជ្រាវ Answers in Genesis បានរកឃើញថាទិន្នន័យដែលប្រើប្រាស់វិធីទាំងនេះគឺមិនដូចគ្នានោះទេ និងថែមទាំងសួរដេញដោលទៅលើភាពស្មោះត្រង់របស់អ្នកស្រាវជ្រាវពីមុនថែមទៀត\cite{108}។ ជាលិកាទន់ដែលមានកោសិកាឈាម សសៃឈាម និងកូឡាជែនត្រូវបានរកឃើញក្នុងសាកសព្វសត្វដាយណូស័រដែលថាមានអាយុរហូតដល់រាប់រយលានឆ្នាំឯណោះ\cite{109,110}។ តាមអ្វីដែលយើងដឹង ការទទួលស្គាល់អំពីអាយុភូគព្ភសាស្ត្រនៃថ្មនិងប្រេងឥន្ធនៈផូសុីលរបស់ផែនដីសព្វថ្ងៃ គឺអាចខុសជាខ្លាំង។ 

\section{ការសន្និដ្ឋាន}

ក្នុងអត្ថបទនេះ ខ្ញុំបានលើកឡើងនូវភាពមិនប្រក្រតីដែលដែលអាចផ្តល់ជាយោបល់អំពីប្រភពដើមនៃគ្រោះមហន្ថរាយដែលអាចពន្យល់បានយ៉ាងក្បោះក្បាយតាមរបៀប "ក្រឡាប់ផែនដី" របស់ ECDO។ ទោះបីវាខុសគ្នាក៏ដោយ ទិន្នន័យដែលបានបង្ហាញគឺនៅខ្វះខាតនៅឡើយ - ភាពមិនប្រក្រតីផ្សេងៗទៀតត្រូវបានប្រមូលផ្តុំរួចរាល់ ហើយអាចស្វែងរកវាបាននៅក្នុង Github របស់ខ្ញុំបាន\cite{2}។
\section{សេចក្ដីអរគុណ}

សូមអរគុណចំពោះ Ethical Skeptic ដែលជាអ្នកនិពន្ធដំបូងនៃនិក្ខេបទ ECDO ដែលបានបំពេញនូវនិក្ខេបទដ៏ជ្រាលជ្រៅមួយនេះ ហើយនិងបានចែករំលែកវាទៅកាន់មនុស្សទាំងអស់នៅលើពិភពលោក។ និក្ខេបទទាំងបីផ្នែករបស់លោក\cite{1} នៅតែជាច្បាប់ដើមសម្រាប់ទ្រឹស្តី "ការបំបែកលំយោលស្រទាប់ក្រឡាប់ក្តៅក្នុងផែនដី(ECDO)" ដែលមានព័ត៌មានច្រើនជាអ្វីដែលខ្ញុំបានសង្ខេតនៅក្នុងអត្ថបទនេះ។

ចុងបញ្ចប់សូមអរគុណទៅដល់អ្នកដែលបានជួយជ្រាជ្រែង អ្នកស្រាវជ្រាវដែលបានប្រមូលពត៍មានទាំងអស់ដែលធ្វើអោយយើងមានលទ្ធភាពបង្កើតអត្ថបទនេះដើម្បីជួយទៅកាន់មនុស្សទូទាំងពិភពលោក។

\clearpage
\twocolumn

{\small
\bibliographystyle{ieee}
\bibliography{egbib}
}

\end{document}
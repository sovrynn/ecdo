\documentclass[10pt,twocolumn,letterpaper]{article}

% Mis propias cosas
\usepackage{booktabs}
% \usepackage{caption}
% \captionsetup[table]{skip=8pt}   % Solo afecta a las tablas
\usepackage{stfloats}  % Añade esto al preámbulo

\usepackage{cvpr}
\usepackage{times}
\usepackage{epsfig}
\usepackage{graphicx}
\usepackage{amsmath}
\usepackage{amssymb}

% Incluye otros paquetes aquí, antes de hyperref.

% Si comentas hyperref y luego lo descomentás, debes eliminar
% egpaper.aux antes de volver a ejecutar latex. (O simplemente presiona 'q' en la primera ejecución de latex, deja que termine, y deberías estar bien).
\usepackage[breaklinks=true,bookmarks=false]{hyperref}

\cvprfinalcopy % *** Descomente esta línea para la versión final

\def\cvprPaperID{****} % *** Ingrese el ID del artículo CVPR aquí
\def\httilde{\mbox{\tt\raisebox{-.5ex}{\symbol{126}}}}

% Las páginas están numeradas en modo de envío, y sin numerar en la versión final
%\ifcvprfinal\pagestyle{empty}\fi
\setcounter{page}{1}
\begin{document}

%%%%%%%%% TÍTULO
\title{Cartilla Basada en Datos sobre ECDO Parte 2/2: Una Investigación de Anomalías Científicas e Históricas Mejor Explicadas por un “Vuelco de la Tierra” ECDO}

\author{Junho\\
Publicado en febrero de 2025\\
Sitio web (Descarga los artículos aquí): \href{https://sovrynn.github.io}{sovrynn.github.io}\\
Repositorio de investigación ECDO: \href{https://github.com/sovrynn/ecdo}{github.com/sovrynn/ecdo}\\
{\tt\small junhobtc@proton.me}
% Para un artículo cuyos autores están todos en la misma institución,
% omita las siguientes líneas hasta el cierre de ``}''.
% Se pueden agregar autores y direcciones adicionales con ``\and'',
% al igual que con el segundo autor.
% Para ahorrar espacio, use ya sea la dirección de correo electrónico o la página web, no ambas
% \and
% xx
% Institución2\\
% Primera línea de la dirección de la institución2\\
% {\tt\small secondauthor@i2.org}
}

\maketitle
%\thispagestyle{empty}

%%%%%%%%% ABSTRACT
\begin{abstract}
En mayo de 2024, un autor en línea seudónimo bajo el nombre de “The Ethical Skeptic” \cite{0} publicó una teoría innovadora llamada Desacoplamiento Exotérmico Núcleo-Manto con Oscilación Dzhanibekov (ECDO) \cite{1}. Esta teoría no solo propone que la Tierra ha experimentado anteriormente cambios catastróficos y repentinos en su eje de rotación, creando una inundación mundial masiva al hacer que los océanos se derramaran sobre los continentes debido a la inercia rotacional, sino que también propone un proceso geofísico causal y explicativo junto con datos que sugieren que otro de estos vuelcos podría ser inminente. Aunque predicciones de inundaciones catastróficas y del fin del mundo no son nuevas, la teoría ECDO es singularmente convincente debido a su enfoque científico, moderno, multidisciplinario y basado en datos.

Este artículo de investigación constituye la segunda parte de un resumen de dos partes de seis meses de investigación independiente \cite{2,20} sobre la teoría ECDO, enfocándose específicamente en las anomalías científicas e históricas que se explican mejor por un catastrófico "volteo" terrestre ECDO.

\end{abstract}

%%%%%%%%% BODY TEXT

\section{Introducción}

La geología y la historia uniformitarianas modernas afirman que los principales paisajes geológicos como el Gran Cañón se formaron a lo largo de millones de años \cite{143}; que la sal existe en el Valle de la Muerte (California) porque solía estar bajo el océano hace cientos de millones de años \cite{144}; que nuestros antepasados de hace 150 generaciones pasaron toda su vida construyendo tumbas gigantescas \cite{29,70}; y que los llamados “combustibles fósiles” tienen cientos de millones de años \cite{104}. Quizás lo más fascinante es que se cree que los seres humanos tienen 300.000 años \cite{145}, pero la historia registrada y la civilización solo datan de hace unos 5.000 años, el equivalente a 150 generaciones humanas.

Como veremos, tales anomalías se explican mejor por fuerzas geológicas catastróficas.

\section{Mamuts Congelados Rápidamente Enterrados en Lodo}

\begin{figure}[t]
\begin{center}
% \fbox{\rule{0pt}{2in} \rule{0.9\linewidth}{0pt}}
   \includegraphics[width=1\linewidth]{jarkov-mammoth.jpg}
\end{center}
   \caption{El Mamut de Jarkov, un mamut siberiano perfectamente conservado de 20.000 años encontrado en barro congelado \cite{51}.}
\label{fig:1}
\label{fig:onecol}
\end{figure}

Una de estas categorías de anomalías son los mamuts perfectamente conservados y congelados, enterrados en lodo, comúnmente encontrados en las regiones árticas (Figura \ref{fig:1}). El mamut de Beresovka, descubierto en Siberia enterrado en grava limosa, estaba tan perfectamente conservado que su carne aún era comestible miles de años después de su muerte. Además, tenía alimento vegetal en la boca y el estómago, lo que desconcertó a los científicos respecto a cómo pudo haberse congelado tan rápidamente si estaba pastando plantas con flores justo antes de morir \cite{17}. Según se informa, \textit{"En 1901 se causó sensación por el descubrimiento de un cadáver completo de mamut cerca del río Berezovka, ya que este animal parecía haber muerto de frío en pleno verano. El contenido de su estómago estaba bien conservado e incluía ranúnculos y frijoles silvestres en flor: esto significaba que debían haber sido ingeridos cerca de finales de julio o principios de agosto. La criatura había muerto tan repentinamente que aún conservaba en la boca un bocado de pasto y flores. Claramente había sido llevada por una fuerza tremenda y arrojada varios kilómetros lejos de su pastizal. La pelvis y una pierna estaban fracturadas—el enorme animal había sido derribado de rodillas y luego congelado hasta la muerte, en lo que normalmente es la época más calurosa del año"} \cite{18}. Además, \textit{"[Los científicos rusos] registraron que incluso el revestimiento más interno del estómago del animal tenía una estructura fibrosa perfectamente preservada, lo que indica que su calor corporal había sido eliminado por algún proceso sobrehumano de la naturaleza. Sanderson, fijándose especialmente en este punto, llevó el problema al American Frozen Foods Institute: ¿Qué se necesita para congelar un mamut entero de manera que el contenido de humedad incluso de las partes más internas de su cuerpo, hasta el revestimiento interno de su estómago, no tenga tiempo suficiente para formar cristales lo suficientemente grandes como para destruir la estructura fibrosa de la carne?... Unas semanas después, el Instituto volvió con la respuesta: Es absolutamente imposible. Con todo nuestro conocimiento científico y de ingeniería, no existe absolutamente ningún método conocido para eliminar el calor corporal de un cadáver tan grande como un mamut lo suficientemente rápido para congelarlo sin que se formen grandes cristales de humedad en la carne. Además, tras agotar las técnicas científicas y de ingeniería, miraron a la naturaleza y concluyeron que no existe ningún proceso conocido en la naturaleza que pueda lograr tal hazaña"} \cite{19}.

\section{El Gran Cañón}

El Gran Cañón, parte de la Gran Cuenca en el suroeste de Norteamérica, es otro fenómeno natural que sugiere orígenes catastróficos (Figura \ref{fig:2}). Para empezar, las capas sedimentarias de arenisca y caliza que conforman el Gran Cañón abarcan áreas masivas de hasta 2.4 millones de km$^2$ \cite{21}. La Figura \ref{fig:3} muestra la extensión de la capa de la arenisca Coconino a lo largo del oeste de los Estados Unidos. Capas horizontales tan masivas de material uniforme solo podrían haberse depositado de una sola vez.

\begin{figure}[t]
\begin{center}
% \fbox{\rule{0pt}{2in} \rule{0.9\linewidth}{0pt}}
   \includegraphics[width=1\linewidth]{grand-canyon.jpg}
\end{center}
   \caption{El Gran Cañón, en Arizona, EE. UU. \cite{49}.}
\label{fig:2}
\label{fig:onecol}
\end{figure}

\begin{figure}[t]
\begin{center}
% \fbox{\rule{0pt}{2in} \rule{0.9\linewidth}{0pt}}
   \includegraphics[width=1\linewidth]{coconino.jpg}
\end{center}
   \caption{Tamaño de la capa de arenisca Coconino en el oeste de los Estados Unidos \cite{21}.}
\label{fig:3}
\label{fig:onecol}
\end{figure}

Una observación más detallada del Gran Cañón nos indica que la deposición de estas extensas capas de sedimentos también ocurrió de manera concurrente con fuerzas tectónicas significativas. Para entender esto, debemos observar cuidadosamente ciertas zonas en el cañón donde las capas de sedimentos han sido plegadas y expuestas. Investigadores de Answers in Genesis \cite{42} examinaron microscópicamente muestras de roca de algunos de estos pliegues, como el Monument Fold, y, basándose en la ausencia de características que deberían estar presentes si los pliegues se hubieran formado durante largos períodos bajo calor y presión, concluyeron que las capas de sedimento fueron plegadas por fuerzas tectónicas mientras aún estaban blandas, es decir, poco después de su deposición \cite{43}.

\begin{figure*}
\begin{center}
% \fbox{\rule{0pt}{2in} \rule{.9\linewidth}{0pt}}
\includegraphics[width=1\textwidth]{Grand_Staircase-big.jpg}
\end{center}
   \caption{Las capas de sedimentos que conforman el Gran Cañón (lado derecho de la imagen) se extienden directamente hacia el norte hasta Cedar Breaks, Utah (lado izquierdo de la imagen), donde todas se doblan hacia arriba \cite{50}.}
\label{fig:4}
\end{figure*}

Al ampliar la perspectiva, encontramos que las capas que conforman el Gran Cañón no solo han sido dobladas dentro del cañón. Las capas han sido dobladas hacia el este en la Monoclina Este de Kaibab \cite{46}, pero también hacia el norte en Cedar Breaks, Utah (Figura \ref{fig:4}). Esto sugiere que estas capas pueden haber sido dobladas juntas después de haber sido depositadas una encima de la otra en rápida sucesión. Como referencia, las capas horizontales del Gran Cañón tienen aproximadamente 1700 metros de grosor. La magnitud del proceso geológico requerido para depositar capas de sedimentos de una milla de espesor es enorme.

La formación real del Gran Cañón es otro tema de debate en la geología moderna. La geología uniformista propone que el Gran Cañón fue tallado por el río Colorado durante millones de años \cite{47}. Sin embargo, el equipo de investigación de Answers in Genesis cree que el Gran Cañón probablemente se formó en cuestión de semanas debido a la erosión de un vertedero de un antiguo lago que rompió sus límites, lo cual eliminó grandes cantidades de sedimento al excavar el cañón. Hay evidencia de un lago de gran altitud al este del Gran Cañón en depósitos de sedimentos lacustres y fósiles marinos. La comparación del Gran Cañón con otros ejemplos a gran escala de erosión por vertedero, como el Cañón de Afton y el Monte Santa Helena, revela una topografía similar y muestra que grandes cañones pueden ser creados rápidamente a través de grandes volúmenes de agua fluyendo \cite{48}.

Considerando la magnitud de los procesos geológicos requeridos para depositar sedimentos sobre extensiones tan enormes de terreno, la concurrencia de fuerzas tectónicas masivas ocurriendo poco después de que se depositaron las capas de sedimentos, y el tamaño diminuto del río Colorado en comparación con la inmensidad del Gran Cañón, parece que no hubo nada gradual en su formación.

\section{Ciudad Subterránea de Derinkuyu}

Aparte de las pirámides, un gran ejemplo de ingeniería antigua es la ciudad subterránea de Derinkuyu (Figura \ref{fig:5}), ubicada en Capadocia, Turquía. Es la más grande entre más de 200 refugios subterráneos en la región \cite{54}. Se estima que esta ciudad subterránea albergó hasta 20.000 personas y abarca 18 niveles, alcanzando una profundidad de 85 metros. Aunque su antigüedad no es segura, se estima que tiene al menos 2800 años. La ciudad fue excavada en roca volcánica blanda \cite{52, 53}.

\begin{figure}[b]
\begin{center}
% \fbox{\rule{0pt}{2in} \rule{0.9\linewidth}{0pt}}
   \includegraphics[width=1\linewidth]{derinkuyu.jpeg}
\end{center}
   \caption{Diagrama de la ciudad subterránea de Derinkuyu \cite{56}.}
\label{fig:5}
\label{fig:onecol}
\end{figure}

La razón por la que Derinkuyu es interesante es porque no está claro por qué alguna comunidad decidiría construir una ciudad entera bajo tierra. Para crear espacio habitable bajo tierra, cada cavidad debe ser excavada en la roca. Las formas y texturas irregulares de los túneles subterráneos dejan claro que estos fueron excavados con trabajo manual, en lugar de herramientas eléctricas, lo cual habría sido órdenes de magnitud más difícil que construir refugios en la superficie. De hecho, no es evidente por qué algún ser humano quisiera vivir permanentemente bajo tierra durante su vida terrenal, cuando la agricultura, la luz del sol, la naturaleza y la exploración solo están disponibles en la superficie. La "historia" convencional propone que Derinkuyu fue creado por cristianos que necesitaban un lugar aislado para practicar su religión \cite{53}. Pero el sentido común concluiría que la forma más directa de enfrentar a los enemigos es "luchar o huir", no "excavar una ciudad subterránea en la roca".

La escala, la profundidad y la minuciosidad del diseño de la ciudad subterránea dejan claro que no fue diseñada como una estructura defensiva militar temporal para combatir mejor a los invasores en momentos de apuro, sino más bien como un refugio a largo plazo para proteger contra fuerzas fatales en la superficie. Derinkuyu estaba equipada no solo con dormitorios, cocinas y baños básicos, sino también con establos para animales, depósitos de agua, almacenamiento de alimentos, prensas de vino y aceite, escuelas, capillas, tumbas y enormes pozos de ventilación (Figura \ref{fig:6}). ¿Por qué un refugio militar requeriría una prensa de vino y necesitaría ser excavado a 85 metros de profundidad con tal grado de complejidad?

La explicación más plausible para la creación de Derinkuyu habría sido una necesidad imperiosa de preparar un refugio autosuficiente y a largo plazo para protegerse contra fuerzas geofísicas catastróficas en la superficie de la Tierra.

\begin{figure}[t]
\begin{center}
% \fbox{\rule{0pt}{2in} \rule{0.9\linewidth}{0pt}}
   \includegraphics[width=1\linewidth]{derinkuyu-air.jpg}
\end{center}
   \caption{Un pozo profundo de ventilación en Derinkuyu \cite{53}.}
\label{fig:6}
\label{fig:onecol}
\end{figure}

% \section{Anomalías adicionales mejor explicadas por un volteo de la Tierra}

% Antes de concluir, mencionaremos algunas anomalías científicas adicionales que, una vez vistas en el contexto de fuerzas geofísicas catastróficas, se explican bien.
\section{Acumulaciones de Biomasa}

Mezclas de biomasa de varios tipos de animales y plantas, a menudo encontradas fosilizadas en capas de sedimento, son otra anomalía desconcertante. En "Reliquoæ Diluvianæ", el reverendo William Buckland detalla hallazgos de numerosas especies de fauna que no tenían una razón explicable para encontrarse juntas, esparcidas por Gran Bretaña y Europa, enterradas en capas de 'diluvium' sedimentario \cite{58}. Tales mezclas de restos animales también se encontraron en la cueva Skjonghelleren en la isla de Valdroy, Noruega. En esta cueva, se encontraron más de 7.000 huesos de mamíferos, aves y peces mezclados a través de múltiples capas de sedimento \cite{59}. Otro ejemplo es San Ciro, la "Cueva de los Gigantes", en Italia. En esta cueva, se encontraron varias toneladas de huesos de mamíferos, en su mayoría hipopótamos, en un estado tan fresco que fueron cortados para hacer adornos y enviados para la manufactura de negro de carbón. Se informa que los huesos de los diferentes animales estaban mezclados, rotos, fragmentados y dispersos en pedazos \cite{60,61}. En la antigua Mendes, Egipto, se encontró una mezcla de varias especies de huesos de animales mezclados con arcilla vitrificada (vidriosa) \cite{57}. Tales hallazgos pueden parecer desconcertantes, pero se explican fácilmente por inundaciones masivas que depositaron mezclas de animales muertos en capas de sedimento, depositando animales o enterrándolos vivos en cuevas, y en el caso de biomasa vitrificada en Egipto, descargas eléctricas masivas posteriores al diluvio provenientes de un desplazamiento núcleo-manto. La Figura \ref{fig:7} muestra una exposición típica de 'muck' de biomasa de Alaska \cite{56}.

\begin{figure}[t]
\begin{center}
% \fbox{\rule{0pt}{2in} \rule{0.9\linewidth}{0pt}}
   \includegraphics[width=1\linewidth]{muck-crop.jpeg}
\end{center}
   \caption{'Muck' de Alaska, compuesto por fragmentos dispersos caóticamente de árboles, plantas y animales en limo y hielo congelados \cite{146}.}
\label{fig:7}
\label{fig:onecol}
\end{figure}

\section{Búnkeres Antiguos}

Nuestros antepasados dejaron muchas estructuras antiguas altamente sofisticadas donde se han encontrado restos humanos. Usualmente se interpretan como elaboradas tumbas, pero una inspección más cercana sugiere que en realidad pudieron haber sido antiguos búnkeres.

\begin{figure}[t]
\begin{center}
% \fbox{\rule{0pt}{2in} \rule{0.9\linewidth}{0pt}}
   \includegraphics[width=1\linewidth]{ww19.jpg}
\end{center}
   \caption{Newgrange, Irlanda - observe a los visitantes en la entrada para una idea de la escala.}
\label{fig:8}
\label{fig:onecol}
\end{figure}

Un excelente ejemplo es Newgrange (Figura \ref{fig:8}), el monumento principal en el complejo Brú na Bóinne, una colección de estructuras antiguas que incluye las llamadas tumbas de corredor. Estas tumbas consisten en una o más cámaras funerarias cubiertas de tierra o piedra y tienen un estrecho pasaje de acceso hecho de grandes piedras \cite{70}. Es un ejemplo de ingeniería extensiva de una estructura compleja protegida, construida a lo largo de varias generaciones, supuestamente para enterrar a un puñado de personas que ni siquiera estaban vivas cuando comenzó la construcción de la tumba. Cuando fue redescubierta por un terrateniente local en 1699, estaba enterrada bajo tierra.

Un vistazo superficial a la estructura revela el inmenso esfuerzo invertido en su construcción: Newgrange consta de unas 200.000 toneladas de material. En su interior, \textit{“...hay un pasaje con cámaras, al que se puede acceder por una entrada en el lado sureste del monumento. El pasaje se extiende por 19 metros (60 pies), o aproximadamente un tercio del camino hacia el centro de la estructura. Al final del pasaje hay tres pequeñas cámaras alrededor de una cámara central más grande con un techo alto de bóveda de ménsula... Las paredes de este pasaje están formadas por grandes losas de piedra llamadas ortostatos, veintidós en el lado oeste y veintiuna en el lado este. Su altura promedio es de metro y medio”} \cite{70}. También hay intrincados detalles de ingeniería impermeabilizante. Por ejemplo, en el techo, \textit{“Los intersticios del techo fueron calafateados con una mezcla de tierra quemada y arena de mar para hacerlos impermeables y de esta mezcla se obtuvieron dos dataciones por radiocarbono centradas en el 2500 a.C. para la estructura de la tumba"} \cite{71}. Además, una elevación que conduce a la cámara interior pudo haber sido implementada con un propósito similar: \textit{“Dado que el suelo del pasaje y de la cámara de la tumba sigue la subida del terreno de la colina sobre la que se construyó el monumento, hay una diferencia de casi 2 metros en el nivel del suelo entre la entrada y el interior de la cámara”} \cite{71}.

\begin{figure}[t]
\begin{center}
% \fbox{\rule{0pt}{2in} \rule{0.9\linewidth}{0pt}}
   \includegraphics[width=1\linewidth]{dolmen.jpg}
\end{center}
   \caption{El Dolmen de Soto, España \cite{53}.}
\label{fig:9}
\label{fig:onecol}
\end{figure}

La falta de restos humanos en el interior también es un punto curioso. Las excavaciones revelaron fragmentos de huesos quemados y no quemados que representan solo a unas pocas personas, dispersos a lo largo del pasaje. Se estima que la construcción de Newgrange tomó al menos varias generaciones según las fechas de carbono de los materiales encontrados en su interior. ¿Por qué una comunidad antigua dedicaría tanto esfuerzo a construir una tumba masiva y altamente tecnificada solo para esparcir los fragmentos óseos de unos pocos difuntos en su pasadizo? Es mucho más plausible que estas antiguas y cuidadosamente impermeabilizadas estructuras megalíticas fueran en realidad construidas como refugios humanos para proteger a las personas durante los cataclismos recurrentes de la Tierra.

En Huelva, sur de España, un ejemplo similar es el Dolmen de Soto (Figura \ref{fig:9}), uno de unos 200 sitios similares en la zona \cite{72,32}. Es una estructura estilizada, altamente sofisticada, construida con piedras megalíticas y que posee un diámetro de 75 metros. Según informes, solo se encontraron ocho cuerpos durante la excavación, todos enterrados en posición fetal.

\section{Menciones de Anomalías Notables}

En esta sección, menciono brevemente algunas anomalías más notables, todas ellas bien explicadas por un cataclismo de tipo ECDO.

\subsection{Anomalías Biológicas}

\begin{figure}[b]
\begin{center}
% \fbox{\rule{0pt}{2in} \rule{0.9\linewidth}{0pt}}
   \includegraphics[width=1\linewidth]{bottleneck.jpg}
\end{center}
   \caption{Un cuello de botella genético que representa una reducción del 95\% de los varones hace unos 6.000 años \cite{62}.}
\label{fig:10}
\label{fig:onecol}
\end{figure}

Algunas anomalías biológicas notables son los cuellos de botella genéticos y los fósiles de ballenas tierra adentro. Zeng et al. (2018) modelaron 125 secuencias del cromosoma Y de humanos modernos y, basándose en las similitudes y mutaciones en el ADN, identificaron un cuello de botella poblacional del 95\% en la población masculina hace unos 5.000 a 7.000 años (Figura \ref{fig:10}) \cite{62}. Se han encontrado fósiles de ballenas a cientos de metros sobre el nivel del mar, en Swedenborg, Michigan, Vermont, Canadá, Chile y Egipto \cite{63,64,65,66}. Estas ballenas se encontraron en distintos estados: perfectamente preservadas, en turberas situadas sobre depósitos glaciares o enterradas en sedimentos. El número de ejemplares en estos sitios varía desde unos pocos hasta más de cien. Las ballenas son criaturas de aguas profundas y rara vez se acercan a las costas. ¿Cómo terminaron estas ballenas a tales altitudes, a menudo a distancias extremas tierra adentro?

Numerosas extinciones masivas han ocurrido en el pasado de la Tierra, siendo las más estudiadas las "Cinco Grandes" del Fanerozoico: las extinciones masivas del Ordovícico Tardío (LOME), Devónico Tardío (LDME), final del Pérmico (EPME), final del Triásico (ETME) y final del Cretácico (ECME) \cite{88,89}. Curiosamente, varias de estas extinciones se clasifican como ocurridas en los mismos períodos históricos que muchas de las capas del Gran Cañón, concretamente, las capas Pérmicas y Devónicas.

\subsection{Anomalías Físicas}

\begin{figure}[b]
\begin{center}
% \fbox{\rule{0pt}{2in} \rule{0.9\linewidth}{0pt}}
   \includegraphics[width=1\linewidth]{columbia.jpg}
\end{center}
   \caption{Ondas de corriente masivas en el Lago Glacial Columbia, estado de Washington \cite{80}.}
\label{fig:11}
\label{fig:onecol}
\end{figure}

Existen muchos paisajes además del Gran Cañón que probablemente se formaron por fuerzas cataclísmicas. La evidencia de un flujo de agua continental masivo puede encontrarse en gigantescas ondas de corriente en todo el mundo. Un ejemplo de esto es el Channeled Scablands en el Noroeste del Pacífico. Aquí, no solo vemos paisajes de depósitos sedimentarios y rocas erráticas, sino también más de cien secuencias de grandes ondas formadas por mega flujos de corriente \cite{78,79}. Estas son versiones a mayor escala de las ondas formadas en los lechos arenosos de arroyos. Se pueden encontrar en todo el mundo, en Francia, Argentina, Rusia y América del Norte \cite{81}. La Figura \ref{fig:11} muestra algunas de estas ondas en el estado de Washington, en Estados Unidos \cite{80}.

\begin{figure}[b]
\begin{center}
% \fbox{\rule{0pt}{2in} \rule{0.9\linewidth}{0pt}}
   \includegraphics[width=1\linewidth]{zhangjiajie.jpg}
\end{center}
   \caption{Masivos pilares de piedra en el Bosque Nacional de Zhangjiajie, sur de China.}
\label{fig:12}
\label{fig:onecol}
\end{figure}

\begin{figure}[t]
\begin{center}
% \fbox{\rule{0pt}{2in} \rule{0.9\linewidth}{0pt}}
   \includegraphics[width=1\linewidth]{hoy.jpg}
\end{center}
   \caption{Pilar marino Old Man of Hoy, Escocia \cite{83}.}
\label{fig:13}
\label{fig:onecol}
\end{figure}

Estructuras de erosión tierra adentro también son bien explicadas por un volcamiento de la Tierra como ECDO. China meridional es un gran ejemplo de masivos panoramas kársticos, formados mediante la erosión del agua. \cite{82}. Estos panoramas incluyen torres kársticas, pináculos kársticos, conos kársticos, puentes naturales, cañones, sistemas de grandes cuevas, y sumideros. Uno de los más impactantes de estos es el Bosque Nacional Zhangjiajie National Forest, el cual contiene masivos pilares de cuarzoarenita (Figure \ref{fig:12}) \cite{84}. Estos pilares tienen una elevación promedio de 1.000 metros y número de más de 3.100. Más de 1.000 de ellos se elevan por encima de los 120 metros de alto, y 45 superan los 300 metros. \cite{85}. Estos pilares se asemejan a pilares de erosión marina (Figure \ref{fig:13}), los cuales son pilares de roca costera formados por el colapso de material alrededor a causa de olas marinas. Panoramas de erosión similares pueden ser encontrados en los conos rocosos de Urgup, Turquía, así como en  Ciudad Encantada, España, de los cuales ambos son están 1.000 metros por encima del nivel del mar. Todas estas ubicaciones tienen una combinación de sal y fósiles marinos oceánicos muy cerca de ellas, sugiriendo incursiones marinas pasadas \cite{15,86,87}. Por supuesto, los relatos de inundación \cite{3} mencionan el océano superando mucho más allá de los 1.000 metros, y esto es verificado por la presencia de agua marina en salinas masivas en los Andes e Himalayas varios kilómetros por encima del nivel del mar. La salina Uyuni en Bolivia, por ejemplo, alcanza los 3.653 metros por encima del nivel del mar \cite{94}.

\subsection{Eventos Rápidos de Cambio Climático}

La literatura científica moderna reconoce la existencia de eventos de cambio climático global rápido en la historia reciente de la Tierra. Dos ejemplos notables son los eventos de 4,2 mil años y 8,2 mil años, ambos coincidiendo con reducciones poblacionales y disrupciones en asentamientos sociales sobre grandes áreas geográficas. Estos eventos están preservados como anomalías en núcleos de sedimentos y hielo, corales fósiles, valores de isótopos O18, registros de polen y espeleotemas, y datos sobre niveles del mar. Los cambios climáticos inferidos incluyen una rápida caída de las temperaturas globales, aridificación, una alteración de la corriente de retorno meridional del Atlántico y avances glaciares \cite{90,91,92}. El evento de 8,2 mil años es particularmente contemporáneo con una posible dramática inundación de agua salada en el Mar Negro alrededor del 6400 a.C. \cite{93}.

\subsection{Anomalías Arqueológicas}

La evidencia arqueológica de algunas ciudades antiguas muestra múltiples capas que involucran entierro y destrucción, creando registros de eventos cataclísmicos pasados. La antigua ciudad de Jericó es una de ellas, situada en la actual Palestina. Contiene múltiples capas de destrucción, con colapso de estructuras de piedra e intensos incendios \cite{96,97}. La cronología registrada en sus capas data aproximadamente desde el 9000 a.C. hasta el 2000 a.C. Es especialmente relevante su torre, que parece haber sido cortada y enterrada en sedimentos alrededor del 7400 a.C. (Figura \ref{fig:14}) \cite{95}. Catal Huyuk \cite{99}, Gramalote \cite{98}, y el palacio minoico de Cnosos en Creta \cite{100,101} son ejemplos similares de sitios arqueológicos que contienen múltiples capas, a menudo con evidencia de destrucción.

\begin{figure}[t]
\begin{center}
% \fbox{\rule{0pt}{2in} \rule{0.9\linewidth}{0pt}}
   \includegraphics[width=1\linewidth]{jericho.jpg}
\end{center}
   \caption{Reconstrucción arqueológica del entierro de la Torre de Jericó circa 7400 a.C. \cite{95}.}
\label{fig:14}
\label{fig:onecol}
\end{figure}

Otra pieza de evidencia de cataclismos mayores perturbando la civilización humana es el Nampa Image, un muñeco de barro encontrado enterrado a aproximadamente 100 metros de lava en Idaho. \cite{102,103}. El flujo de lava bajo el cual se encontró la figura que se estimó que fue depositado durante el período Cenozoico o Pleistoceno Temprano, supuestamente teniendo 2 millones de años. Sin embargo, el flujo de lava en la región parece ser relativamente fresco. Tales descubrimientos no solo señalan cataclismos mayores destructores de civilización, también llaman a cuestionar las cronologías modernas.

\section{Sobre los métodos modernos de datación}

Hay razones significativas para ser escépticos sobre las cronologías modernas, que asignan edades extremadamente largas de millones, o incluso cientos de millones de años, a diversos materiales físicos.

La narrativa convencional sostiene que los llamados "combustibles fósiles" como el carbón, el petróleo y el gas natural tienen cientos de millones de años \cite{104}. Sin embargo, una datación real por carbono del petróleo en el Golfo de México arrojó una antigüedad de aproximadamente 13.000 años para el petróleo \cite{105}. El carbono-14 tiene una vida media tan corta (5.730 años) que se supone debe decaer completamente después de unos cientos de miles de años. Sin embargo, se ha encontrado en carbón y fósiles que supuestamente tienen mil veces más antigüedad \cite{106}. De hecho, se ha producido carbón artificial en laboratorios bajo condiciones controladas, principalmente con alta temperatura, en tan solo 2-8 meses \cite{107}.

Los métodos de datación por radioisótopos distintos a la datación por carbono tampoco pueden ser precisos. El grupo de investigación Answers in Genesis encontró inconsistencias en las fechas derivadas de estos métodos que ponen en duda su veracidad \cite{108}. Incluso se ha encontrado tejido blando que contiene células sanguíneas, vasos y colágeno en restos de dinosaurios que supuestamente tienen cien millones de años \cite{109,110}. Según lo que sabemos, es posible que las edades convencionalmente aceptadas de la escala de tiempo geológico de la Tierra y de materiales físicos como rocas y combustibles fósiles sean incorrectas por muchos órdenes de magnitud.

\section{Conclusión}

En este artículo, he cubierto las anomalías más convincentes que sugieren orígenes catastróficos y que se explican mejor por un vuelco de la Tierra ECDO. Aunque diversa, la colección presentada está incompleta: se han recopilado más anomalías y están disponibles públicamente en mi repositorio de investigación en GitHub \cite{2}.

\section{Agradecimientos}

Gracias a Ethical Skeptic, el autor original de la tesis ECDO, por completar su perspicaz y revolucionaria tesis y compartirla con el mundo. Su tesis tripartita \cite{1} sigue siendo la obra autorizada sobre la teoría de la Oscilación de Desacoplamiento Núcleo-Manto Exotérmica Dzhanibekov (ECDO), y contiene mucha más información sobre el tema de la que he resumido brevemente aquí.

Y, por supuesto, gracias a los gigantes sobre cuyos hombros estamos; aquellos que han realizado toda la investigación e indagación que hicieron posible este trabajo y trabajaron para traer luz a la humanidad.

\clearpage
\twocolumn

{\small
\renewcommand{\refname}{Referencias}
\bibliographystyle{ieee}
\bibliography{egbib}
}

\end{document}
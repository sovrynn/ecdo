\documentclass[10pt,twocolumn,letterpaper]{article}

% ကိုယ်ပိုင်အရာများ
\usepackage{booktabs}
% \usepackage{caption}
% \captionsetup[table]{skip=8pt}   % ဇယားများအတွက်သာ သက်ရောက်သည်။
\usepackage{stfloats}  % ၎င်းကို preamble တွင် ထည့်ပါ။

% \usepackage{fontspec}
\usepackage[english]{babel}

% load Lao via babelprovide, turn on "onchar=ids" for automatic shaping
\babelprovide[import,onchar=ids fonts]{burmese}

% main (rm) font for Latin
\babelfont{rm}{Noto Serif}

% % Lao text in Noto Serif Lao at 1.2× scale
% \babelfont[burmese]{rm}{Noto Serif Myanmar}
% \babelfont[burmese]{sf}{Noto Serif Myanmar}
% % Lao text in Noto Serif Lao for the alt family too
% \babelfont[burmese]{alt}{Noto Serif Myanmar}


\babelfont[burmese]{rm}{Noto Sans Myanmar}
\babelfont[burmese]{sf}{Noto Sans Myanmar}
% Lao text in Noto Serif Lao for the alt family too
\babelfont[burmese]{alt}{Noto Sans Myanmar}

% alternate (sans-serif) font for Latin
\babelfont{alt}{Lato}

% GPT: LuaTeX doesn’t have built-in Lao line-breaking rules, but babel can assimilate line breaks to hyphenation if you supply some simple “patterns” for common syllable boundaries. Add this to your preamble:
\babelpatterns[burmese]{%
  1က 1ခ 1ဂ 1ဃ 1င 1စ 1ဆ 1ဇ 1ဈ 1ဉ 1ည 1ဋ 1ဌ 1ဍ 1ဎ 1ဏ
  1တ 1ထ 1ဒ 1ဓ 1န 1ပ 1ဖ 1ဗ 1ဘ 1မ 1ယ 1ရ 1လ 1ဝ 1သ 1ဟ
  1ဢ 1ါ 1ာ 1ိ 1ီ 1ု 1ူ 1ေ 1ဲ 1ဲ
  1ျ 1ြ 1ွ 1ှ
  1္ 1့ 1း 1ံ 1်%
}

\usepackage{cvpr}
\usepackage{times}
\usepackage{epsfig}
\usepackage{graphicx}
\usepackage{amsmath}
\usepackage{amssymb}

% တခြား package များကိုလည်း ဒီမှာ (hyperref မပြည့်မီ) ထည့်ပါ။

% မင်းက hyperref ကို comment လုပ်ပြီးပြန်ဖွင့်မယ်ဆိုရင်၊ egpaper.aux ကို ဖျက်ပြီး latex ကို ပြန်လည်ရောက်ပါ။  (သို့မဟုတ် ပထမဆုံး latex run မှာ 'q' ကို နှိပ်ပြီး၊ ၎င်းအပြီးတာနီးသော်လည်း အသစ်ပြင်ပြီးသားဖြစ်လာမည်။)

\usepackage[breaklinks=true,bookmarks=false]{hyperref}

\makeatletter
\def\cvprsubsection{\@startsection {subsection}{2}{\z@}
    {8pt plus 2pt minus 2pt}{6pt}{\bfseries\normalsize}}
\makeatother

\cvprfinalcopy % *** အဆုံးခံတင်သွင်းမှုအတွက် ဒီလိုင်းကို uncomment လုပ်ပါ

\def\cvprPaperID{****} % *** ဒီမှာ CVPR စာတမ်းအမှတ်ကို ထည့်ပါ
\def\httilde{\mbox{\tt\raisebox{-.5ex}{\symbol{126}}}}

% တင်သွင်းမှုအဆင့်မှာ စာမျက်နှာနံပါတ် ရှိပြီး၊ camera-ready မှာ နံပါတ်မရှိပါ
%\ifcvprfinal\pagestyle{empty}\fi
\setcounter{page}{1}
\begin{document}

%%%%%%%%% ခေါင်းစဥ်
\title{ECDO ဒေတာအခြေပြု အကြိုသိရမည့်အချက်များ အပိုင်း ၂/၂: ECDO “Earth Flip” ဖြင့် ပိုမိုကောင်းစွာရှင်းပြနိုင်သည့် သိပ္ပံနှင့် သမိုင်းဆိုင်ရာ ထူးခြားမှုများကို စူးစမ်းသုံးသပ်ခြင်း}

\author{Junho\\
၂၀၂၅ ဖေဖော်ဝါရီမှာ ထုတ်ဝေသည်\\
ဝက်ဘ်ဆိုဒ် (စာတမ်းများကို ဒီမှာ ဆွဲယူပါ): \href{https://sovrynn.github.io}{sovrynn.github.io}\\
ECDO သုတေသန repo: \href{https://github.com/sovrynn/ecdo}{github.com/sovrynn/ecdo}\\
{\tt\small junhobtc@proton.me}
}

\maketitle
%\thispagestyle{empty}

\begin{abstract}
၂၀၂၄ မေလတွင်၊ “The Ethical Skeptic” \cite{0} ဟု အမည်မသိ အွန်လိုင်းစာရေးသူတစ်ဦးက Exothermic Core-Mantle Decoupling Dzhanibekov Oscillation (ECDO) \cite{1} ဟုခေါ်သော ဆန်းသစ်ထူးခြားသော သီအိုရီတစ်ခုတင်ပြခဲ့သည်။ ဤသီအိုရီတွင်မြေကမ္ဘာသည် ယခင်က များစွာကြိမ်ကြိမ် အလွန်အကောင်းဆုံးသော လှည့်ပတ်ဝင်ရိုး ပြောင်းလဲမှုကြီးကြပ်ကို ကြုံတွေ့ဖူးကြောင်း၊ ထိုကြောင့် ပင်လယ်များသည် မြေကြီးပေါ်သို့ မတော်တဆ နှိမ့်ကျရေနံဆန်မှုကြောင့် ပြန့်ကျဲခဲ့ပြီး ဒဏ္ဍာရီကြီးသောရေဆှာင့်သဘောတူဘီလာကြီးတစ်ခု ဖြစ်ပေါ်ခဲ့ကြောင်း၊ ထို့အပြင် ထိုသို့ ပြန့်ကျဲမှုတစ်ခုကို ဖြစ်စေသော မြေဗေဒဆိုင်ရာ လုပ်ငန်းစဉ်နှင့် ထပ်မံ flip ပြုလုပ်မှုတစ်ခုပင် ခန့်မှန်းရနိုင်မည့် ဒေတာများပါ သဘောတူတင်ပြထားသည်။ ထိုသို့သော မေတ္တာတရားရေဆှာင့်ဖြစ်ပေါ်မှုနှင့် ကြမ္ဘာကုန်ဆုံးသံသယများသည် အသစ်မဟုတ်သော်လည်း၊ ECDO သီအိုရီသည် သိပ္ပံရေးရာ၊ နောက်ဆုံးပေါ်၊ စုံစမ်းတူသော သဘောတူနည်းလမ်းနှင့် ဒေတာအခြေခံဖြစ်မှုကြောင့် ထူးခြားစွာ စိတ်ဝင်စားဖွယ် ဖြစ်သည်။

ဤသုတေသနစာတမ်းသည် လွတ်လပ်စွာ ၆ လတာ သုတေသန (နှစ်ပိုင်းထဲမှ) ဒုတိယပိုင်း အကျဉ်းချုပ် ဖြစ်ပြီး၊ ECDO သီအိုရီနှင့်ပတ်သက်၍ သိပ္ပံနှင့် သမိုင်းဖြတ်အထူးလက္ခဏာအနုစိတ်များအား ECDO "မြေကမ္ဘာ အလှည့်" ကြီးဖြင့် အကောင်းဆုံး ရှင်းပြနိုင်ကြောင်းကို အထူးအာရုံစိုက်ပါသည်။ 

\end{abstract}

\section{နိဒါန်း}

ခေတ်သစ်တောင်တက်ဘောင် သီအိုရီနှင့် သမိုင်းစာတမ်းများအရ Grand Canyon ကဲ့သို့သော ဦးတည်သည့် မြေညီထုကြီးများသည် သန်းခေါင်ရာစုပေါင်းများစွာကြာမြင့်စွာဖွဲ့တစ်နည်းတစ်ထား ဖြစ်ပေါ်လာသည်ဟုဆိုကြသည် \cite{143}။ Death Valley (California) တွင် ဆားတွေတည်ရှိရာမှာ သန်းခေါင်ရာစုပေါင်းများခန့် မြစ်နှင့်ပင်လယ်အောက်တွင်ပါတယ်ဆိုသည့် အကြောင်းကြားချက်ခြင်း \cite{144}၊ လူတို့၏ ယခင်ညီမျှ၍၂၅၀ကြိမ်လောက် မျိုးဆက်ရာကခန့်သောပုံပြင်တွင် လူ့ဘိုးဘွားများသည် များပြားသည့်သင်္ချိုင်းကြီးများ တည်ဆောက်ရန် ခေတ်စာအုပ်၏အားလုံးကိုသုံးစွဲခဲ့ကြသည်ဟု ယူဆကြသည် \cite{29,70}။ ထို့ပြင် "သဲလွင့်သေတ္တာရေနံ" ဟုခေါ်သော သယောင်ထုတ်သော နက်ရိုက်ရေနံအရင်းအမြစ်များသည် သန်းခေါင်ရာစုပေါင်းများစွာအဟောင်းကြောင်း \cite{104} ဆိုကြသည်။ အလွန်စိတ်မဝင်စားနိုင်တာကလည်း လူသားမှာ ၃၀၀,၀၀၀ နှစ် သက်တမ်းရှိကြောင်း ယူဆကြသော်လည်း မှတ်တမ်းတင် သမိုင်းနှင့် ယဉ်ကျေးမှုသည် ၅,၀၀၀ နှစ်ခန့် - လူ့မျိုးနွယ် တစ်မျိုး ၁၅၀ ကြိမ်စာသာ ရှိသည်။

ထိုကဲ့သို့ ထူးခြားသောအချက်များသည် ပျက်စီးကြမ်းတမ်းသော မြေဗေဒ ရုန်းရင်းအားများဖြင့် အကောင်းဆုံး ရှင်းလင်းနိုင်သည်ကို ကျွန်ုပ်တို့မြင်တွေ့နိုင်သည်။

\section{ရေအေးတစ်ပြိုင်နက် ချက်ချင်းသေဆုံးသွားသော မမ်မုတ်များ အမွေအနှစ်တည်နေရာ၌ မြေဆီလွှာအောက်သို့ရုပ်သိမ်းခြင်း}


\begin{figure}[t]
\begin{center}
% \fbox{\rule{0pt}{2in} \rule{0.9\linewidth}{0pt}}
   \includegraphics[width=1\linewidth]{jarkov-mammoth.jpg}
\end{center}
   \caption{Jarkov မမ်မာ့သစ်သားသည် နှစ် ၂၀,၀၀၀ ကျော်အရွယ်ရှိ ချေးရှင်းစွာသိမ်းဆည်းထားနိုင်ခဲ့သော စိုင်းဘေးရီးယား မမ်မာ့နွားတစ်ကောင်ကို တည်ငြိမ်သော ရေခဲပေါ်တွင် ရှာဖွေတွေ့ရှိခဲ့ခြင်းဖြစ်သည် \cite{51}။}
\label{fig:1}
\label{fig:onecol}
\end{figure}

ဤအထူးအမျိုးအစားထဲမှ တစ်ခုမှာ အတိအကျ သိမ်းဆည်းထားနိုင်သည့် ရေခဲတောင်ဖြင့် ချက်ခြင်းသေဆုံးသွားသော မမ်မာ့နွားများဖြစ်ပြီး Arctic ဒေသများတွင် ရှာဖွေတွေ့ရှိနိုင်သည် (ပုံ \ref{fig:1})။ Beresovka မမ်မာ့သည် စိုင်းဘးရီးယားတွင် ငဲသောကျောက်ပြားအောက်မှ ရှာဖွေတွေ့ရှိခဲ့ပြီး ချက်ခြင်းသေဆုံးပြီးနောက် နှစ်ပေါင်းများစွာကြာမြင့်သော်လည်း ၎င်း၏သားလည်း အစားအသောက်အဖြစ် အတောက်ပကျန်ရှိနေဆဲဖြစ်သည်။ ထို့အပြင်၊ ၎င်း၏ ပါးစပ်နှင့် ဗိုက်ထဲတွင် အပင်စားအစာများပါရှိနေသောကြောင့် သက်ဆိုင်ရာသိပ္ပံပညာရှင်များသည် ၎င်းသည် ပန်းပွင့်များကို သေဆုံးမီ တစ်ပြိုင်နက်ပဲ စားနေခဲ့သည့်အခါကန့်ကွက်ချက်ရှိနေသည်ဟု ဆိုကြသည် \cite{17}။ ဆိုသော်၊ \textit{"၁၉၀၁ ခုနှစ်တွင် Berezovka မြစ်အနီးတွင် မမ်မာ့နွားတစ်ကောင်လုံး ရှာဖွေတွေ့ရှိမည်ဖြစ်ပြီး ၎င်းသည် မျောသမျှပင်တစ်နေရာတည်းတင်သေဆုံးသွားကြောင်း ထင်ဟပ်စေခဲ့သည်။ ၎င်း၏ဗိုက်ထဲ အာဟာရများကောင်းစွာသိမ်းဆည်းထားနိုင်ခဲ့ပြီး ထဲတွင် နွားအိုပန်းနှင့် ပန်းတစ်ပါးခန့်ပါရှိသည်။ ၎င်းကို ဇူလိုင်လကုန်၊ သို့မဟုတ် သြဂုတ်လအစအဝေးများလောက်မှာ သောက်ခဲ့ရမည်ဖြစ်သည်။ ၎င်းအသားတင်သေသွားခဲ့ပြီးတော့ပါးစပ်ထဲလည်း အပွင့်အပင်များကို မဖျက်အောင်ထားရှိနေသည်။ ၎င်းသည် တစ်အားကြီးထွက်သွားသည့်အခါကြီးကောင်းမှပင် ဖမ်းဆီးခံရခြင်းနှင့် ခုတင်ဖိထားခံရခြင်းသည် ဖြစ်နိုင်သည်။ ၎င်း၏ အရိုးနှင့် ခြေတစ်ဖက် အပ်ကွဲခံရသည်- မမ်မာ့ကြီးသည် ဒိုးခုံပေါ်သို့ ထိုင်သွားလိုက်၍ ပြီးတော့ မိုးရာသီအပူမြင့်ချိန်တွင်တင်ပင် ရေခဲခံခဲ့ရသည်"} \cite{18}။ ထို့အပြင်၊ \textit{"[ရုရှားသိပ္ပံပညာရှင်များက] မမ်မာ့နွား၏ ဗိုက်အတွင်းပိုင်းထပ်သားမှာပါ အတိအကျသိမ်းဆည်းထားနိုင်ပြီး အရောင်ထူသောဖွဲ့စည်းမှုရှိကြောင်း မှတ်တမ်းပြုထားသည်။ ၎င်းသည် ကျန်ရှိသည့်အပူအား အလွန်ကြီးမားသော သဘာဝဖြစ်စဥ်တစ်ခုမှ ဆုံးဖြတ်ရေးရှိသွားခဲ့သည်ဟု သက်သေအနေနှင့် ဖြစ်သည်။ Sanderson သည် ဤအချက်ကို အထူးဂရုစိုက်ပြီး၊ အမေရိကန် အအေးခံစားသူလေ့လာရေးဌာနသို့ မေးခွန်းတင်ပြခဲ့သည်- မမ်မာ့နွားတစ်ကောင်လုံး၏ ခန္ဓာကိုယ်အတွင်းအမှုန်အစိတ်အပိုင်းများထိအပါအဝင် အလွန်အေးမြစွာ (မရေခဲခင်နောက်ဆုံးရုပ်ထုအနိမ့်သည့် သားအမျိုးအစားအထိ) ချက်ခြင်းသတ္တုတည်အောင်သိမ်းဆည်းရန် ဘယ်လို လုပ်ရမလဲ?... အချိန်အတစ်ချိန်အတွင်းမှာ အဆိုပါဌာနက Sanderson ထံသို့ ပြန်သွားပြောခဲ့သည်- အလုံးစုံမဖြစ်နိုင်ပါ။ ကျွန်ုပ်တို့ရဲ့ သိပ္ပံနည်းပညာများ၊ အင်ဂျင်နီယာနည်းပညာများထဲမှာ မမ်မာ့အရွယ်ကွယ်သောသတ္တုငယ်တစ်ခုလုံးကို အလျင်အမြန်အေးမြစေဘို့ နည်းလမ်းမရှိပါ။ ထို့အပြင်၊ သိပ္ပံနှင့် အင်ဂျင်နီယာနည်းပညာအား အကုန်အလွန်ကြိုးစား၍နောက်ဆုံးတွင် သဘာဝထဲတွင်လည်း ထိုမျိုးအလုပ်ကို နိုင်ငံတော်အဆင့်မပြည့်မီဖြစ်နိုင်မည့် သဘာဝဖြစ်စဉ်မတစ်ခုမှ ရှာမတွေ့ပေ"} \cite{19}။

\section{Grand Canyon}

Grand Canyon သည် မြောက်အမေရိက၏ တောင်ညာဘက်၊ Great Basin ၏တစ်စိတ်တစ်ပိုင်းအပါအဝင် သဘာဝအံ့သြဖွယ်အရာတစ်ခုဖြစ်ပြီး အကြီးစား ပြင်းထန်သော သဘာဝဖြစ်စဉ်မှ စတင်ပေါ်ပေါက်ခဲ့ကြောင်းအထင်ရစေသည် (ပုံ \ref{fig:2})။ အစပိုင်းတွင် Grand Canyon ၏ ယေဘုယျမြေဆီလွှာနှင့် သဲလွှာ၊ နှင့် ကျောက်သားလွှာများသည် ၂.၄ သန်း/စတုရန်း ကီလိုမီတာပေါ် အသွားအလာရှိနေသည်ဟု တွက်ချက်နိုင်သည် \cite{21}။ ပုံ \ref{fig:3} တွင် Coconino သဲလွှာသည် အနောက်အမေရိက တစ်လျှောက်တွင် ဘယ်လောက်တိုးချဲ့နေသည်ကို ပြသထားသည်။ ဤသို့ စီမံတည်ဆောက်ထားသည့် တန်းပြားနီးပါးကျောက်ထုလွှာကြီးများသည် တစ်ချိန်တည်းတည်း အစုလိုက်အပြုံလိုက် တည်ဆောက်ထားသည်သာဖြစ်နိုင်သည်။

\begin{figure}[b]
\begin{center}
% \fbox{\rule{0pt}{2in} \rule{0.9\linewidth}{0pt}}
   \includegraphics[width=1\linewidth]{grand-canyon.jpg}

\end{center}
   \caption{အာရီဇိုနာပြည်နယ်၊ အမေရိကန်ပြည်ထောင်စုရှိ ဂရန့်ဒ်ကပျ်ယွန်း \cite{49}။}
\label{fig:2}
\label{fig:onecol}
\end{figure}

\begin{figure}[t]
\begin{center}
% \fbox{\rule{0pt}{2in} \rule{0.9\linewidth}{0pt}}
   \includegraphics[width=1\linewidth]{coconino.jpg}
\end{center}
   \caption{အနောက်ပိုင်းအမေရိကန်ပြည်ထောင်စုရှိ Coconino သဲသားတန်းအလွှာ၏ အရွယ်အစား \cite{21}။}
\label{fig:3}
\label{fig:onecol}
\end{figure}

ဂရန့်ဒ်ကပျ်ယွန်းကို မြင့်မြင့်နက်နက်ကြည့်လေ့လာပါက ဤကျယ်ပြန့်သဲသောင်တန်းအလွှာများ တည်နေထိုင်သည်မှာ တချိန်တည်းတွင် အရေးပါသော ပင်လယ်ပြင်လှုပ်ရှားမှုကြီးကြပ်မှုများနှင့်တကွ ဖြစ်ပွားခဲ့သည့်အကြောင်း သိနိုင်ပါသည်။ ဤအကြောင်းကို နားလည်ရန်အတွက် ကပျ်ယွန်းအတွင်း သဲသောင်တန်းအလွှာများ ချိုးကွေ့ထားပြီး မဖုံးအုပ်သေးသည့် အချို့နေရာများကို နီးကပ်စွာလေ့လာရန်လိုအပ်ပါသည်။ Answers in Genesis မှ သုတေသနပြုသူများ \cite{42} သည် Monument Fold ကဲ့သို့သော ချိုးကွေ့နေတဲ့တည်နေရာများမှ တော်တော်အသေးစိတ် ကျောက်တုံးနမူနာများကို မိုက်ခရိုစကုပ်ဖြင့် ကြည့်ရင်း၊ ချိုးကွေ့မှုသည် အချိန်ကာလရှည်ကြာပြီး ဓာတ်ငွေ့နှင့် ဖိအားအောက်တွင်ဖြစ်ပေါ်ခဲ့သည်ဆိုပါက တွေ့ရှိသင့်သည့် လက္ခဏာများ မရှိခြင်းကြောင့်၊ သဲသောင်တန်းအလွှာများသည် တည်နေထိုင်မှုတိုက်ရိုက်ပြီးနောက်မကြာခင်၊ ပင်လယ်ပြင်လှုပ်ရှားမှုကြီးများကြောင့် မသန်အောင် ချိုးကွေ့ခြင်းဖြစ်သည်ဟု သတ်မှတ်ခဲ့သည် \cite{43}။

\begin{figure*}
\begin{center}

% \fbox{\rule{0pt}{2in} \rule{.9\linewidth}{0pt}}
\includegraphics[width=1\textwidth]{Grand_Staircase-big.jpg}
\end{center}
   \caption{Grand Canyon (ဓာတ်ပုံ၏ညာဘက်ခြမ်း) တွင် ဖြစ်ပေါ်သည့် သဲခြင်းအလွှာများသည် တိုက်ရိုက်ကာ ဘရိတ်စ်မြစ်၊ ယူတာ (ဓာတ်ပုံ၏ ဘယ်ဘက်ခြမ်း) သို့ မြောက်ဘက်တွင် တိုးစီးနေပြီး၊ အလွှာအားလုံးသည် အပေါ်သို့ မောက်တက်နေသည်ဟု တွေ့ရသည် \cite{50}။}
\label{fig:4}
\end{figure*}

အနည်းငယ် ဝေးကပ်၍ ကြည့်သော်၊ Grand Canyon တွင်ရှိသည့် လွှာများသည် ချုံအတွင်း၌သာ မဖန်တီးခဲ့သည့် လုပ်ငန်းဖြစ်ကြောင်း တွေ့နိုင်သည်။ အလွှာများသည် အရှေ့ Kaibab Monocline \cite{46} တွင် အရှေ့ဘက်သို့ ဖိသိပ်ပြီး၊ Cedar Breaks, Utah (ပုံ \ref{fig:4}) တွင်လည်း မြောက်ဘက်သို့ ဖိသိပ်မှုရှိသည်။ ၎င်းသည် အလွှာအားလုံးကို တတိယထပ်၍ ချင်းချင်းအပေါ်သို့ မြန်မြန်ဆန်ဆန် ထပ်တူဖန်တီးပြီးနောက် တစ်ပြိုင်တည်း ဖိသိပ်မှုရှိနိုင်ကြောင်း ဉာဏ်သွေးပေးသည်။ ဖြေဆိုချက်များအနက် Grand Canyon တွင်ရှိသည့် အလှည့်အပြောင်းမရှိသော သဲနှင့်ထွဲအလွှာများသည် ခန့်မှန်းခြေ မီတာ ၁၇၀၀ ခန့် ထူသည်။ သဲစာအလွှာများကို မိုင်တစ်မိုင်ခန့် ထူသွားရန် လိုအပ်သည့် ဓာတုဗေဒလုပ်ငန်းစဉ်သည် အလွန်ကြီးမားလှသည်။

Grand Canyon ဖွဲ့စည်းမှုအမှန်သည် ခေတ်သစ် ဓာတုဗေဒတွင် သဘောထားကွဲပြားမှုတစ်ခုပင် ဖြစ်သည်။ Uniformitarian ဓာတုဗေဒသည် Grand Canyon ကို Colorado မြစ်က နှစ်နဲ့နှစ်ပေါင်းများစွာ ပွတ်တိုက်မှုဖြင့် ဖြတ်တောက်ဖန်တီးသည်ဟု ဆိုသည် \cite{47}။ သို့သော် Answers in Genesis သုတေသနအဖွဲ့သည် Grand Canyon ကို ယခင်ကီန်လက်တစ်ခုဟောင်းမှ ရေပေါက်ထွက်၍ တစ်ပတ်အတွင်းတင် ဖြစ်ပေါ်လာခဲ့နိုင်သည်ဟု ယုံကြည်သည်၊ ထိုပြင် ချုံဖန်တီးစဉ် အတွင်း သဲနှင့်လောင်းထူထပ်များကို အလွန်အမင်း ဖယ်ရှားခဲ့သည်။ Grand Canyon အရှေ့ဘက်ရှိ မြင့်မားသောကိန်း lake တွင် သဲနှင့်အထည်အခါများ၊ ပင်လယ်ဇီဝများနှင့်ထင်ရှားမွတ်သိပ်မှုတွေ့ရသည်။ Grand Canyon ကို spillway erosion တစ်ခုဖြစ်သော Afton Canyon နှင့် Mount St. Helens ကဲ့သို့ ငွေကြေးပါးသော နမူနာများနှင့် နှိုင်းယှဉ်ကြည့်လျှင် မြန်လှမြန်မြန် ရေကြီးစီးမှုကြောင့် ချုံကြီးများကို မြန်မြန်ဖြစ်နိုင်ကြောင်း တွေ့ရသည် \cite{48}။

ဓာတုဗေဒလုပ်ငန်းများ၏ အတိုင်းအတာကြီးသည် မျက်နှာပြင်အကြီးအကျယ်တွင် သဲအလွှာ ထပ်မံအောက်ထားသည့်နောက်၊ အလွှာထပ်ထားပြီးတိုင်း နောက်ဆက်တွဲ ဒြပ်စင်အကြီးစား tectonic ဖိအားဖြစ်ပွားမှု၊ Grand Canyon ၏ၾကီးမားချက်နှင့်နှိုင်းယှဥ်၍ Colorado မြစ်၏ အရပ်ငယ်မှုကို တိုင်းတာလိုက်သောအခါ၌ ချုံဖွဲ့စည်းပုံသည် တဖြည်းဖြည်းဖြစ်လာခဲ့ပြီလားဆိုသည်မှာ သတိထားစရာများ ရှိနေသည်။

\section{Derinkuyu မြေအောက်မြို့ကြီး}

ပYRမစ်များအပြင်၊ နောက်ထပ် တစ်ခုထူးခြားသည့် ခေတ်အဟောင်း အင်ဂျင်နီယာလုပ်ငန်းတစ်ခုမှာ คาပဒိုစီးယား၊ တူရကီရှိ Derinkuyu မြေအောက်မြို့ကြီး (ပုံ \ref{fig:5}) ဖြစ်သည်။ ၎င်းသည် ထိုဒေသရှိ မြေအောက်အ။
ားအပါး shelter ၂၀၀ ကျော်အားလုံးတွင် အကြီးဆုံးဖြစ်သည် \cite{54}။ မြေအောက်မြို့ကြီးသည် လူ ၂၀,၀၀၀ ထိ နေရေးအလျာလောက်ရှိပြီး အထပ် ၁၈ ထပ်ကျော်အထိ တိုင်ပင်သည်။ အနည်းဆုံး နှစ် ၂၈၀၀ ခန့်ရှေးမြေပေ့သည်ဟု ခန့်မှန်းနိုင်သည်။ မြို့ကြီးကို ပျော့ပြောင်းသော ပူဇလပ်တောင်ကျောက်ဖြင့် တူးထုတ်ပြုလုပ်ခဲ့သည် \cite{52, 53}။

\begin{figure}[b]
\begin{center}
% \fbox{\rule{0pt}{2in} \rule{0.9\linewidth}{0pt}}
   \includegraphics[width=1\linewidth]{derinkuyu.jpeg}
\end{center}
   \caption{Derinkuyu မြေအောက်မြို့၏ ပုံပြင် \cite{56}.}
\label{fig:5}
\label{fig:onecol}
\end{figure}

Derinkuyu ကို စိတ်ဝင်စားစရာကောင်းစေသော အကြောင်းက မည်သည့် အသိုင်းအဝိုင်းက မည်သည့်အကြောင်းပြချက်ကြောင့် မြို့တစ်မြို့လုံးကို မြေအောက်မှာ ဆောက်ရန် ဆုံးဖြတ်ခဲ့သည့်အကြောင်း မထင်ရှားသည့်အတွက် ဖြစ်သည်။ အောက်မြေအတွင်း နေထိုင်နိုင်သော နေရာများဖန်တီးရန် ရှစ်သည့် အခန်းတိုင်းကို ကျောက်တုံးများမှ ခုတ်တူးပေးရမည် ဖြစ်သည်။ မြေအောက်လမ်းဦးများ၏ ကြမ်းတမ်းသော သွင်ပြင်နှင့် အထူအပါးများကြောင့် ဓာတ်အားသုံးကိရိယာများ မသုံးပဲ ကိုယ်တိုင် လက်နက်ဖြင့် ခုတ်တူးလုပ်ဆောင်ခဲ့ကြောင်း သိသာသည်၊ ၎င်းသည် မြေပေါ်တွင် အိမ်ယာများတည်ဆောက်မှုထက် အဆ များစွာ ပိုမိုပြင်းထန်သော အလုပ်ဖြစ်သည်။ တကယ်တော့ မြေကြီးထဲတွင် အမြဲတမ်း နေထိုင်ရန် လူသားဘယ်သူမျှ မကြိုက်နိုင်ဟု သဘောပေါက်နိုင်သည်၊ အတွက်မှာ တောတွင်းခြံခြံ၊ နေရောင်ခြည်၊ သဘာဝနှင့် စူးစမ်းဖော်ထုတ်ခြင်းတို့သည် အပေါ်မြေတွင်သာ ရနိုင်သော အရာများဖြစ်သည်။ ပုံမှန် "သမိုင်း" မှာ Derinkuyu ကို ကိုယ်ပိုင်ဘာသာရေးကို လှုပ်ရှားရန် နယ်သာလန်နေရာလိုအပ်သော ခရစ်ယာန်များက တည်ထောင်ခဲ့ကြောင်း အဆိုပြုသည် \cite{53}။ သို့သော် ဆင်ခြင်လိုက်လျှင် အတိအကျကိုယ်တိုင် ပြိုင်၍တိုက်ခိုက်ခြင်း သို့မဟုတ် ထွက်ပြေးခြင်းသည် မိမိ၏ အတားအဆီးများကို ဖြေရှင်းရာတွင် ပိုမိုလုံခြုံသော နည်းလမ်းဖြစ်သည်။ ကျောက်တုံးများထဲမှ မြေအောက်မြို့တစ်မြို့လုံး ခုတ်တူးခြင်း ဆိုသည်မှာ အနည်းဆုံး မူဉ်းထားရမည့် ဆုံးဖြတ်ချက်မျိုး မဟုတ်ပါ။

မြေအောက်မြို့၏ စံနှုန်း၊ နက်လယ်မှုနှင့် ခုပြုကောင်းမှုတို့ ကြည့်ကြည့်လျှင် ၎င်းသည် ရုတ်တရက် စစ်ရေးကာကွယ်မှုလိုအပ်ချက်အတွက် တည်ဆောက်သည့် ယာယီအဆောက်အအုံမဟုတ်ပဲ၊ မြေပေါ်တွင် ဖြစ်နိုင်သော အန္တရာယ်ကြီးကြီးမားမားများမှ ကာကွယ်ရန်အစီအစဉ်လှပြင်ထားသည့် ရေရှည်နေထိုင်ရေးအိမ်ရာ ဖြစ်သည်ဟု သိသာသည်။ Derinkuyu တွင် မိမိလိုအပ်သော အိပ်ခန်း၊ မီးဖို၊ သန့်စင်ခန်းများသာမက တိရစ္ဆာန်များအတွက် ပျံရုံ၊ ရေသိုလှောင်ကန်၊ အစားအသောက် သိမ်းဆည်းရာ၊ စပါးနယ်၊ ၀ိုင်နှင့် ဆီချစ်စက်များ၊ ကျောင်း၊ ဘုရား၊ သေဆုံးသူအခန်း နှင့် လေဝင်လေထွက်ကြီးမားသည့် လေထုတိုင်များ (ပုံ \ref{fig:6}) တို့ထည့်သွင်းထားသည်။ စစ်ရေးခိုလှုံရာတစ်ခုသည် ၀ိုင်ချစ်စက်လိုအပ်သလား၊ နက်ရှိုင်းမှု ၈၅ မီတာအထိ ခုတ်တူးရရန် လိုအပ်သလားဆိုသည့် သေးမဲ့မေးခွန်းနှင့် ကြုံတွေ့ရသည်။

Derinkuyu ကို ဖန်တီးလိုက်ရခြင်းအတွက် အထိရောက်ဆုံး အကြောင်းပြချက်မှာ လွန်စွာ ပြင်းထန်ရာပါ၃၍ မြေပေါ်တွင် ဖြစ်နိုင်သော သစ်တော၊ မြေမျက်နှာသစ် မတည့်နိုင်အောင် ရေရှည်တည်တံ့နိုင်သော ကိုယ်ပိုင်နေထိုင်ရန်ခိုလှုံရာ တစ်ခု ဖန်တီးရန်လိုအပ်မှု ဖြစ်နိုင်သည်။

\begin{figure}[t]
\begin{center}
% \fbox{\rule{0pt}{2in} \rule{0.9\linewidth}{0pt}}
   \includegraphics[width=1\linewidth]{derinkuyu-air.jpg}
\end{center}
   \caption{Derinkuyu အတွင်းရှိ နက်ရှိုင်းလှသော လေဝင်လေထွက်လယ်တိုင်တစ်ခု \cite{53}.}
\label{fig:6}

\label{fig:onecol}
\end{figure}

% \section{ကမ္ဘာဘက်ပြောင်းမှုဖြင့် အကောင်းဆုံးရှင်းပြနိုင်သော ထပ်ဆောင်းထူးခြားဆန်း وضعیتများ}

% သတ်မှတ်ချက်မြောက်လာခြင်းမတိုင်မီ၊ သြေကြာပြင်းထန်သော ရုပ်ပိုင်းဆိုင်ရာအင်အားများ၏ အမြင်မှ ကြည့်လျှင် ကောင်းစွာရှင်းပြနိုင်သည့် ထပ်ဆောင်းသိပ္ပံဆိုင်ရာထူးခြားဆန်း وضعیتအချို့ကို ဖော်ပြပါမည်။

\section{ဇီဝထု ပေါင်းဆုံမှုများ}

တိရစ္ဆာန်နှင့် အပင်အမျိုးမျိုး၏ ဇီဝထု ပေါင်းစပ်မှုများကို မကြာခဏ သဲရှင်းအလွှာများ၌ ကျောက်ဖြစ်အဖြစ် တွေ့ရပါသည်။ ၎င်းသည် ထူးထူးခြားခြားသော သဗ္ဗာ နောက်ထပ်ထူးခြားဆန်း وضعیتတစ်ခုဖြစ်သည်။ "Reliquoæ Diluvianæ" မှာ Rev. William Buckland သည် မတူညီသော တိရစ္ဆာန်အမျိုးစားအများအပြားသည် အတူတကွ တွေ့ရမှုအကြောင်း ရှင်းပြထားပြီး၊ ၎င်းတို့သည် ဘာကြောင့် အတူတူ ပေါင်းစပ်နေပါသလဲဆိုသည်ကို ရှင်းလင်းစေရန် အကြောင်းအရာမရှိကြောင်း ဖော်ပြထားသည်။ ၎င်းအရာများသည် ဗြိတိန်နှင့် ဥရောပတစ်လျှောက်တွင် စိန့်ကျေသော 'diluvium' သဲလွှာထဲတွင် သိမ်းဆည်းထားသည် \cite{58}။ ထိုမျိုးစုံတိရစ္ဆာန်အရေအတွက်များကို နော်ဝေ၏ Valdroy ကျွန်း၊ Skjonghelleren ဂူတွင်လည်း တွေ့ရှိခဲ့သည်။ ဤဂူတွင် သားသတ်တိရစ္ဆာန်၊ ငှက်နှင့် ငလျင်များ၏ အရိုး ၇,၀၀၀ ကျော်သည် သဲလွှာအမျိုးမျိုးအတွင်း တွဲဖက်ရှပ်ထားခြင်း အဖြစ် တွေ့ရှိခဲ့သည် \cite{59}။ နောက်ထပ် ဥပမာတစ်ခုမှာ အီတလီနိုင်ငံမှ San Ciro၊ "မဟာဘုရင်များ၏ ဂူ" ဖြစ်သည်။ ဤဂူအတွင်းတွင် သားသတ်တိရစ္ဆာန် အရိုးများ တစ်ထွေထွေ အတွင်းမှ hippopotamus များဖြင့်ဖွဲ့စည်းထားခဲ့သည်။ ဤအရိုးများသည် အသစ်အသစ်နှင့်တူအောင်သာ၊ အလှလုပ်ရန် မှုတ်သုတ်၊ မီးပျောက်မှုလုပ်ငန်းများအတွက် ပို့ချပေးခဲ့သည်။ တိရစ္ဆာန်မျိုးစုံ၏အရိုးများသည် တစ်တူတူ ရောပြီး ဖျက်ဆီးပျက်ယွင်းကွဲကွင်းခဲ့ကြောင်း သိရှိရသည် \cite{60,61}။ ရှေးဟောင်း Mendes, Egypt တွင်တောင် တိရစ္ဆာန်အမျိုးမျိုး၏ အရိုးများသည် အလှူပစ္စည်းပုံစံနှင့် ဖြစ်သော vitrified (မွန်မြှုပ်) လေဖြင့် တွဲဖက်ထားရှိသည် \cite{57}။  ထိုမျိုးသောတွေ့ရှိမှုများသည် ထူးခြားသော်လည်း ဧရိယာကြီးမားသောရေလျှံမှု၊ သေဆုံးသွားသောတိရစ္ဆာန်တူရာသဲလွှာများထဲသို့ ထည့်သွင်းခြင်း၊ သို့မဟုတ် ဂူထဲသို့ ရှင်းလင်းရှင်းလင်းသယ်ယူသွားခြင်းကဲ့သို့ ရေဘေးရှည်ကြာသည့်သက်ရောက်မှုများကြောင့် ဖြစ်နိုင်သည်ကို မလွယ်ကူသော်လည်း ရှင်းပြနိုင်သည်။ အီဂျစ်ရှိ vitrified ဇီဝထုအခန်းတွင်လည်း ရေပျံအပြီးတွင် မြေထုလှုပ်ရှားမှုကြီးကြပ်မှုကြောင့် ဓာတ်အားပြင်းထန်စွာ ပျံ့နှံ့မှု ဖြစ်ပေါ်နိုင်သည်။ ဓာတ်ပုံ \ref{fig:7} သည် အလက်စကာ ဇီဝထု 'muck' အထွေထွေများအား ပါဝင်ပုံကို ပြထားသည် \cite{56}။

\begin{figure}[t]
\begin{center}
% \fbox{\rule{0pt}{2in} \rule{0.9\linewidth}{0pt}}
   \includegraphics[width=1\linewidth]{muck-crop.jpeg}
\end{center}
   \caption{အလက်စကာ 'muck' သည် သစ်ပင်၊ အပင်နှင့် တိရစ္ဆာန်အစိတ်အပိုင်းများကို ရောထူထပ်စွာ ရေခဲနှင့် သံပေါင်းသံဖြင့် တွဲဖက်ထားသည့်အရာ ဖြစ်သည် \cite{146}။}
\label{fig:7}
\label{fig:onecol}
\end{figure}
\section{အစဉ်အဟုန်ရာဇဝင် မတ်တပ်ရပ် အိမ်သိုက်များ}

ကျွန်ုပ်တို့၏ ဘိုးဘွားဘိုးဘြားများသည် လူသားအလောင်းအလိမ့်များကို ရှာဖွေတွေ့ရှိရသော အတော်လေးကြီးမားသည့် အင်ဂျင်နီယာအတတ်ပါရှိသည့် ရှေးဟောင်းအဆောက်အအုံများစွာကို ကျန်ထားခဲ့သည်။ ထိုအဆောက်အအုံများကို တော်တော်များများသည် ခမ်းနားသည့် သေတ္တာအိမ်အဖြစ် ဘာသာပြန်လေ့လာကြသည်။ သို့သော် ပိုမိုနက်နက်ရှိုင်းရှိုင်းကြည့်လေ့လာသည်နှင့် ထိုနေရာများသည် အစောပိုင်းကာကွယ်ရေးအိမ်အဖြစ် တကယ်တမ်းအသုံးပြုခဲ့ကြမည်ဟု သောကောင်း သေချာမူရှိသည့်အထောက်အထားများပါဝင်သည်။

\begin{figure}[b]
\begin{center}
% \fbox{\rule{0pt}{2in} \rule{0.9\linewidth}{0pt}}
   \includegraphics[width=1\linewidth]{ww19.jpg}
\end{center}
   \caption{Newgrange, အိုင်ယာလန် - ဝင်ခွင့်မြှောက်မှ ဧည့်သွားများကို စိန့်ဆာအရွယ်အစားအတွက် ကြည့်ပါ။}
\label{fig:8}
\label{fig:onecol}
\end{figure}

ထူးချွန်သော နမူနာတစ်ခုမှာ Newgrange (ပုံ \ref{fig:8}) ဖြစ်ပြီး၊ Brú na Bóinne စုစည်းမှုတွင် အဓိကအမှတ်အသားဖြစ်သည်။ ၎င်းသည် သေတ္တာလျှောက်လမ်းများဟု ခေါ်ဆိုကြသည့် ရှေးဟောင်းအဆောက်အအုံအစုရွေ့တွင် ပါဝင်သည်။ ထိုသေတ္တာများတွင် မြေအောက်သို့မဟုတ် ကျောက်တုံးများဖြင့် ဖုံးအုပ်ထားသည့် အကွက်တခုပေါင်းသို့မဟုတ် တစ်ခုပေါင်းပါတဲ့ အလောင်းသိုလှောင်ခန်းများပါဝင်သည်။ ထိုသလောက် အကြီးစားကျောက်တုံးများဖြင့်ဖန်တီးထားသည့် တိုတောင်းသော ဝင်ပေါက်လမ်းကြောင်းတစ်ခုရှိသည် \cite{70}။ ၎င်းသည် စီးကွင်းစဉ်ဆက်ဖြစ်စေရန်အတွက် မျိုးဆက်များစွာ ချိတ်ဆက်တည်ဆောက်ထားရသော ကာကွယ်ရေးအဆောက်အအုံအကြီးစားတစ်ခုကို အလွန်တန်ဖိုးကြီးစွာ တည်ဆောက်ထားသည့် ဥပမာတစ်ခုဖြစ်သည်။ သေဆုံးပြီးသည့် အနည်းငယ်သော လူအလောင်းများအတွက်သာ တိုက်ရိုက်တည်ဆောက်ခဲ့သည်ဟုထင်မြင်သော်လည်း၊ ထိုသေတ္တာတည်ဆောက်ချိန်တွင် သက်ရှိမရှိဖြစ်သူများကိုတောင် မသိရှိဘဲဖြစ်နိုင်သည်။ ၁၆၉၉ ခုနှစ်တွင် ဒေသခံမြေရှင်တစ်ဦးက ပြန်လည်ရှာဖွေတွေ့ရှိသည့်အခါ၊ မြေဆောင်ပြီးထားခဲ့သည်။

အဆောက်အအုံကို မစိမ်းခေါက်ကြည့်လျှင် တည်ဆောက်မှုတွင် အလွန်ကြီးမားသည့် ကြိုးစားအားထုတ်မှုများကို တွေ့ရှိနိုင်သည် - Newgrange တွင် တန်ချိန် ၂၀၀,၀၀၀ ခန့်ရှိသော တည်ဆောက်ပစ္စည်း များပါဝင်သည်။ ထိုအတွင်းတွင် \textit{“...ပေါက်ပေါက်ခန်းတစ်ခုရှိပြီး၊ အဆိုပါအထိမ်းအမှတ်အဆောက်အအုံ၏ အရှေ့တောင်ဘက်ဝင်ပေါက်မှ ဝင်ရောက်နိုင်သည်။ ထိုလမ်းကြောင်းသည် မီတာ ၁၉ (အပေါ် ၆၀) ခန့်ရှိပြီး အဆောက်အအုံ၏ အလယ်ဗဟိုသို့ သုံးပိုင်းတစ်ပိုင်းခန့်တိတိဖြတ်သွားသည်။ လမ်းကြောင်းပြီးဆုံးရာ၌ မျက်နှာကျက်ကျယ်သော အလယ်စင်ခန်းမှ အပေါ်သို့ပထမ ဆည်ခေါင်ခန်းသုံးခန်းခွဲရှိပြီး၊ အထက်ကာပူဖုံးအုပ်ထားသည့် မိုးသည်းတိမ်မြန်သော ထက်မြင့်ကြေးမုံရှိသည်... ထိုလမ်းကြောင်း၏ နံရံသည် orthostats ဟုခေါ်သည့် မှီခိုထားသည့် ကြီးမားသည့်ကျောက်တုံးများဖြင့် ပြုလုပ်ထားသည်။ ပြင်ပဘက်တွင် ကြောက်တုံး ၂၂ ချောင်း၊ အရှေ့ဘက်တွင် ၂၁ ချောင်းရှိသည်။ တစ်ခုပျမ်းမျှ မီတာ ၁.၅ ရှည်သည်”} \cite{70}။ လည်းကောင်း ရေစိုခံအင်ဂျင်နီယာလုပ်ငန်းဆိုင်ရာ အသေးစိတ်ပါရှိသည်။ ဥပမာအားဖြင့်၊ မျက်နှာထပ်အပေါ်တွင် \textit{“အထပ်တစ်ထပ်၏ အထဲဆွဲဟာ မီးဖြင့်ချက်သည့်မြေဆီနှင့် ပင်လယ်သဲဖြင့် စပ်ချက်ထားသည့်အကြောင့် ရေအေးစိုခံနိုင်စေခဲ့ရုံသာမက ထို့ကြောင့် ထိုသေတ္တာတည်ဆောက်မှုအတွက် ကရေ့ဗွန်နဲ့ခန့်မှန်းသည့် နှစ်မှာ ခရစ်တော်မပေါ်မီ ၂၅၀၀ ခုနှစ်ကို ဦးတည်သည်"} \cite{71}။ ထို့ပြင် အတွင်းခန်းသို့တက်ရာ၌ မြေမျက်နှာပြင်မြင့်တက်မှုတစ်ခု ဖန်ဆင်းခြင်းသည်လည်း အလားတူရည်ရွယ်ချက်အတွက် ဖြစ်နိုင်သည်။ \textit{“သေတ္တာ၏ လမ်းကြောင်းနှင့် အခန်း၏ အောက်ခြေသည် ထိုအထိမ်းအမှတ်အဆောက်အအုံတည်ရှိသည့် တောင်စောင်းမြေမျက်နှာပြင်ကို လိုက်နာသော်လည်း ဝင်ပေါက်မှ အခန်းအတွင်းသို့ အမြင့်အနည်းဆုံး မီတာ ၂ ခန့်ကွာခြားသည်”} \cite{71}။

\begin{figure}[b]

\begin{center}
% \fbox{\rule{0pt}{2in} \rule{0.9\linewidth}{0pt}}
   \includegraphics[width=1\linewidth]{dolmen.jpg}
\end{center}
   \caption{ဒေါ်မင် ဒ စိုတို၊ စပိန် \cite{53}။}
\label{fig:9}
\label{fig:onecol}
\end{figure}

အတွင်းတွင် လူအလောင်းအရုပ်များ မတွေ့ရခြင်းသည်လည်း ထူးဆန်းစရာ အချက်တစ်ခုဖြစ်သည်။ တူးဖော်မှုများအတွင်း မီးလောင်သည့်၊ မလောင်သည့် အရိုးအုပ်စုသေးသမားများနှင့်လူအနည်းငယ် ၏ အရိုးအုပ်စုများ ချစ်စေ့ချစ်စေ့ သံစေ့ပေါ်၌ ဖြန့်ချထားရန် တွေ့ရှိခဲ့သည်။ Newgrange ၏ ဆောက်လုပ်မှုအချိန်ကို အတွင်းရှိပစ္စည်းများမှ ကာဗွန်နေ့စွဲချက်များအရ အနည်းဆုံးမျိုးဆက်အသီးသီး ကြာမြင့်ခဲ့သည်ဟု ခန့်မှန်းခဲ့သည်။ နန်းျဖဴအုပ်စုချင်း သာေသာကြီးမား၍ အတုအယောင်မရှိ အထူးစီမံကျဆင်း၍ ဆောက်လုပ်ထားသည့် သုသာန်ကြီးတစ်ခုသို့ လူနည်းငယ်၏ အရိုးအုပ်စုအားလုံးကိုသာ လမ်းကြောင်းတွင် ဖြန့်ချခြင်းအား ထိုဟုဆောက်လုပ်ကြမည်လား? ယခင်ခေတ် ထာဝရနှင့် သန့်ရှင်းစွာရေအောင် ပြုလုပ်ထားသော မေဂါလစ်သံတုဆောက်လုပ်မှုကြီးများသည် မြေကြီး၏ ထပ်ပြန်ဖြစ်သော သဘာဝအန္တရာယ်များကာကွယ်ရန် လူသားများ နေထိုင်ရာအိမ်အဖြစ် ဆောက်လုပ်ခဲ့ကြသည်ဟု သက်သာလွယ်ကူစွာ ယူဆနိုင်သည်။

Huelva၊ တောင်ပိုင်း စပိန်တွင် ထပ်တူတူသော ဥပမာတစ်ခုမှာ ဒေါ်မင် ဒ စိုတို (ပုံ \ref{fig:9}) ဖြစ်ပြီး၊ ယင်းနယ်မြေတွင် ယင်းကဲ့သို့သော နေရာ ၂၀၀ ခန့် ရှိသည်ဟု ဆိုသည် \cite{72,32}။ ၎င်းသည် မေဂါလစ်ခဲတုံးများ အသုံးပြုပြီး အထူးစီမံကိန်းဖြင့် ဆောက်လုပ်ထားသော ပြီးပြည့်စုံသည့်တည်ဆောက်မှုတစ်ခုဖြစ်ပြီး ချင်းဝိုင်းအချင်းအဝိုင်း ၇၅ မီတာရှိသည်။ တူးဖော်သောအခါ တွေ့ရသည့် လူ့အလောင်း ရှစ်လောင်းသာ ရှိကြောင်း၊ ထိုအလောင်းအားလုံးကို သားအိမ်ထိုးနည်းဖြင့် မြေအောက်သိုင့်ထားခဲ့ကြသည်ဟု ဆိုသည်။

\section{ထူးခြားသော အမှတ်တရများ}

ဤအခန်းတွင်၊ ထူးခြားဖွယ် အဖြစ်အပျက်များထပ်မံ၍ တင်ပြမည်ဖြစ်ပြီး၊ အားလုံးသည် ECDO-သဘာဝပေါက်ကွဲဖြစ်စဉ်တစ်ခုဖြင့် ကောင်းစွာ ရှင်းပြနိုင်သည်။

\subsection{ဇီဝဘောင်ထူးခြားမှုများ}

\begin{figure}[b]
\begin{center}
% \fbox{\rule{0pt}{2in} \rule{0.9\linewidth}{0pt}}
   \includegraphics[width=1\linewidth]{bottleneck.jpg}
\end{center}
   \caption{ယေဘုယျ လူသားအမျိုးအစား၏အထက်မံသားအစု ၁၉၅\% ပြတ်တောက်မှု ဖြစ်ပွားခဲ့သည့် ဂျင်နက္တစ် ချိတ်ကပ်မှုကို နမူနာပြထားသည် (အ​လူ ၆,၀၀၀ နှစ် အကြာခန့်) \cite{62}။}
\label{fig:10}
\label{fig:onecol}
\end{figure}

သိသိသာသာ ကြောငျးကြောငျးရှိသော ဇီဝလက္ခဏာဆိုးဝါးများထဲတွင် ဂျင်နက္တစ် bottlenecks များနှင့် ခရိုင်အတွင်းဝင် တံငါဘဲရဲ့ ကျောက်ဖြစ်ကျောက်များပါဝင်သည်။ Zeng et al. (2018) မှ လူသားတို့၏ ယခုခေတ် Y-ခရိုမိုဆိုင် ၁၂၅ ခုအား တိုက်နှိုင်းသုံးသပ်ကာ DNA အတွင်းရှိ ဟုံတူသောနေရာများနှင့် မျိုးကွဲခြားမှုများကိုအခြေခံ၍ ယခင် ၅,၀၀၀ မှ ၇,၀၀၀ နှစ်အတွင်း အထီး လူဦးရေ ၉၅% လျော့နည်းသွားသော ဂျင်နက္တစ် ချိတ်ကပ်မှုတစ်ခု တွေ့ရှိခဲ့သည် (ပုံ \ref{fig:10}) \cite{62}။ တံငါဘဲကျောက်များအား ပင်လယ်ရေမတက်နိုင်သည့်အမြင့်ရာများ၊ Swedenborg၊ မီချီဂန်၊ Vermont၊ ကနေဒါ၊ ချီလီနှင့် ဧဂျစ်တို့တွင် တွေ့ရှိနိုင်သည် \cite{63,64,65,66}။ ဤတံငါဘဲများ၏အခြေအနေသည်အမျိုးမျိုးပြောင်းလဲသည် - အလုံးစုံကောင်းစွာတည်ရှိခြင်း၊ ရေနွံထဲတွင် ရေခဲကုန်းရုပ်မြေအပေါ် တည်ရှိခြင်း သို့မဟုတ် သဲမြေထဲတွင် ဖုံးအုပ်ခြင်း။ ဤနေရာများတွင်တံငါဘဲများ၏အရေအတွက်သည် တစ်ချို့အနည်းငယ်မှ တစ်ရာကျော်အထိရှိသည်။ တံငါဘဲများသည် ပင်လယ်ရှည်နက်တွင်သာနေထိုင်ပြီး နေလွတ်ကမ်းပါးသို့ နည်းနည်းသာလာတတ်သည်။ ထိုတံငါဘဲများက ယင်းလို အမြင့်ဆုံးနေရာများ၊ ခရိုင်အတွင်းအလွန်ဝေးကွာသောနေရာများတွင် ဘယ်လိုရောက်ရှိလာနိုင်ခဲ့သနည်း?

မြေကြီးသမိုင်းအတွင်း အစုလိုက်အပြုံလိုက် မျိုးပျောက်မှုကြီးကြပ်စွာဖြစ်ပွားခဲ့ကြသည်။ ထိုထဲမှ သိပ္ပံပညာရှင်များအနေနှင့် အကောင်းဆုံးလေ့လာခဲ့သည်များမှာ "အကြီးဆုံး ငါးခု" ဟုခေါ်သည့် Phanerozoic အခန်းအထိပ်ဆုံးနှင့်ပတ်သက်သော မျိုးပျောက်မှုဖြစ်ရပ်များဖြစ်လေသည်။ ထိုသို့ဖြစ်သည့် မျိုးပျောက်မှုများမှာ နောက်ဆုံး Ordovician (LOME), နောက်ဆုံး Devonian (LDME), နောက်ဆုံး Permian (EPME), နောက်ဆုံး Triassic (ETME) နှင့် နောက်ဆုံး Cretaceous (ECME) မျိုးပျောက်မှုကြီးများဖြစ်သည် \cite{88,89}။ စိတ်ဝင်စားစရာကောင်းသည့်အချက်မှာ ဤမျိုးပျောက်မှုကြီးများအနက်တချို့သည် Grand Canyon တွင်တွေ့ရှိနိုင်သည့် ကျောက်တန်းအချို့ဖြစ်သည့် Permian နှင့် Devonian အလွှာများရှိသည့် သမိုင်းကာလအတူတူတွင် ဖြစ်ပွားခဲ့သည်ဟု သတ်မှတ်ထားပါသည်။

\subsection{ရုပ်ပိုင်းဆိုင်ရာ ကန့်သတ်ချက်များ}

\begin{figure}[b]
\begin{center}
% \fbox{\rule{0pt}{2in} \rule{0.9\linewidth}{0pt}}
   \includegraphics[width=1\linewidth]{columbia.jpg}
\end{center}
\caption{Washington ပြည်နယ်၊ Glacial Lake Columbia တွင် တွေ့ရှိသော ကြီးမားသော လှိုင်းပုံစံများ \cite{80} ။}
\label{fig:11}
\label{fig:onecol}
\end{figure}

Grand Canyon ကို မဖွဲ့စည်းခဲ့သော်လည်း၊ မတည်ရှိတဲ့အခြားသော သိသာထင်ရှားသော မြေပြင်ပုံစံများသည်လည်း တရားဝင်ပါဝင်သော မတည့်တည်းသည့် အကြမ်းခံစွမ်းအင်ကြောင့် ဖြစ်ပေါ်လာခဲ့သည်ဟု ယုံကြည်မှုရှိသည်။ များပြားလှသော ကြီးမားသော တောအုပ်ရေစီးကြောင်း၏ သက်သေများကို ကမ္ဘာတစ်ဝှမ်းလုံးတွင် မြွေ့နေသော ကြီးမားသော လှိုင်းပုံစံများမှ တွေ့ရသည်။ ဥပမာတစ်ခုမှာ Pacific Northwest တွင် ရှိသော Channeled Scablands ဖြစ်သည်။ ဤနေရာတွင် ကျောက်ဆက် တင်သွင်းမှုများနှင့် ကြီးမားသော ကျောက်တုံးများကိုသာမက၊ mega current flows ကြောင့်ဖွဲ့စည်းထားသော ကြီးမားသော လှိုင်းပုံစံများကို မည်သည့် hundred sequences ထက်မပိုအောင် တွေ့ရှိနိုင်သည် \cite{78,79} ။ ၎င်းသည် သာမန်ရေကြောင်းများ၏ သဲအိမ်၌ဖွဲ့စည်းသည့် လှိုင်းပုံစံများ၏ ကြီးမားသည့်ပုံစံဖြစ်သည်။ ထိုလှိုင်းပုံစံများကို France၊ Argentina၊ Russia နှင့် North America တို့တွင်လည်း တွေ့ရှိနိုင်သည် \cite{81} ။ Fig \ref{fig:11} တွင် အမေရိကန်ပြည်ထောင်စု၏ Washington ပြည်နယ်ရှိ လှိုင်းပုံစံများကို ဖော်ပြထားသည် \cite{80}။

\begin{figure}[b]
\begin{center}
% \fbox{\rule{0pt}{2in} \rule{0.9\linewidth}{0pt}}
   \includegraphics[width=1\linewidth]{zhangjiajie.jpg}
\end{center}
   \caption{တောင်တန့်တိုင်ကျောက်တိုင်ကြီးများကို တောင်တန်းတောင်တန်း၊ တောင်တောင်တောင်တောအရပ်၊ တောင်တောင်တောင်တောအရပ်၊ တောင်တောင်တောင်တောအရပ်၊ တောင်တောင်တောင်တောအရပ်၊ တောင်တောင်တောင်တောအရပ်၊ တောင်တောင်တောင်တောအရပ်၊ တောင်တောင်တောင်တောအရပ်၊ တောင်တောင်တောင်တောင်တောင်တောင်တောင်တောင်တောင်တောင်တောင်တောင်တောင်တောင်တောင်တောင်တောင်တောင်တောင်တောင်တောင်တောင်တောင်တောင်တောင်တောင်တောင်တောင်တောအရေးတွင် တွေ့ရသည်။}
\label{fig:12}
\label{fig:onecol}
\end{figure}

\begin{figure}[b]
\begin{center}
% \fbox{\rule{0pt}{2in} \rule{0.9\linewidth}{0pt}}

   \includegraphics[width=1\linewidth]{hoy.jpg}
\end{center}
   \caption{ဟိုက်ကမ်းစွန်းရေငုပ်တန်း၊ စကော့တလန် \cite{83}.}
\label{fig:13}
\label{fig:onecol}
\end{figure}

မြေတွင်းငယ်ယင်းဖျက်စီးမှု အခြေဆောင် အဆောက်အအုံများသည် ECDO- နည်းလမ်းနှင့် သက်တမ်းတက်မြေကမ္ဘာလှည့် ပြောင်းလဲမှုဖြင့်လည်း ကောင်းစွာ ရှင်းပြနိုင်သည်။ တရုတ်တောင်ပိုင်းသည် ရေအသွယ်ဖြင့် ပေါ်ပေါက်လာသည့် ကြီးမားသော ကားစဖွဲ့ပုံမြေရုပ်များအတွက် အလွန်တော်သောနမူနာ တစ်ခုဖြစ်သည် \cite{82}။ ဤမြေရုပ်များတွင် တောင်စည်ကားစ၊ ထိပ်တော်ကားစ၊ စပြီး ကုန်းကားစ၊ သဘာဝတံတားများ၊ မြစ်ခေးများ၊ အကြီးစားဂူစနစ်များနှင့် မြေအောက်ပေါက်ပေါက်များပါဝင်သည်။ ဤတွင် အထူးထင်ရှားသည့် ဥပမာအနက် တစ်ခုမှာ Zhangjiajie အမျိုးသားသစ်တော အုပ်ချုပ်မှုနေရာဖြစ်ပြီး၊ ဦးတည်သောကျောက်သကာခဲတိုင်ကြီးများ ပါဝင်သည် (ပုံ \ref{fig:12}) \cite{84}။ ဤကျောက်တိုင်များသည် ပျမ်းမျှ မြင့်ချင်းမီတော် ၁၀၀၀ မီတာကျော်ရှိပြီး၊ စုစုပေါင်း ၃,၁၀၀ ကျော်ရှိသည်။ ၎င်းတို့အနက် ၁,၀၀၀ ကျော်သည် ၁၂၀ မီတာကျော် မြင့်မားသော်လည်း၊ ၄၅ ခုသည် ၃၀၀ မီတာထက်လည်း ပိုမြင့်သည် \cite{85}။ ဤကျောက်တိုင်များသည် ပင်လယ်ရေခြစ်ကြောင့် အနားထောင်လိုက်သည့် ပင်လယ်မွန်တိုင်များနှင့် ဆင်တူသည် (ပုံ \ref{fig:13})။ ဤပရိသတ်ငယ်များသည် သင်္ဘောတင်စိုက်ထားသည့် ဂူများ၊ အာ့ဂျုပ် (တူရကီ) နှင့် Ciudad Encantada (စပိန်) တို့တွင်လည်း တွေ့နိုင်ပြီး၊ များစွာသောနေရာများသည် ပင်လယ်ရေထွက်စပါးများနှင့် သဲဓာတ်ပါဘူးမျိုးကို နီးစပ်သည့်နေရာတွင် တွေ့နိုင်သည် \cite{15,86,87}။ သဘာဝပျော်ရွှင်စရာ ရေအထုတ်ပုံပြင်များ \cite{3} တွင် ပင်လယ်ရေသည် ၁,၀၀၀ မီတာထက်လည်း မြင့်တက်ခဲ့သည်ဟု ဖော်ပြထား၍၊ ဤသည်ကို ပင်လယ်ဆားရေ နှင့် မြင့်မားသည့် အန်ဒီဇတွင် ထွက်ရှိသည့် ဆားဝပ်ကြီးများမှလည်း သက်သေပြနိုင်သည်။ ဥပမာအားဖြင့် ဗိုလီးဗီးယားရှိ Uyuni ဆားဝပ်သည် ပင်လယ်ရေနိမ့်မျက်နှာပြင်ထက် ၃၆၅၃ မီတာရှိသည် \cite{94}။

\subsection{အလွန်မြန်သော ရာသီဥတု ပြောင်းလဲမှုဖြစ်ရပ်များ}

ခေတ်သစ် သိပ္ပံစာတမ်းများတွင် မြေကြီးသမိုင်းနောက်ဆုံးကာလအတွင်း အလွန်မြန်မြန် ပြောင်းလဲသည့် ကမ္ဘာလုံးဆိုင်ရာ ရာသီဥတုအပြောင်းအလဲ ဖြစ်ရပ်များ ရှိကြောင်း ချီးမြှောက်သည်။ ထင်ရှားသော ဥပမာ နှစ်ခုမှာ နှစ် ၄,၂၀၀ နှင့် ၈,၂၀၀ ပြည့်သိုင်း အဖြစ်ရပ်များ ဖြစ်ပြီး၊ ဤအချိန်များတွင် လူဦးရေ လျော့နည်းမှုနှင့် လူမှုအိမ်ရာအုပ်စုကြီးများ ပျက်စီးမှုတို့ တွေ့ရသည်။ ဗစ်သန့်နှင့် ရေခဲအသားတင်ထုများ၊ ကျောက်မီးခိုးဓာတ်သမိုင်းအကြောင်းအရာများ၊ O18 နစ်ဉာဏ်တန်ဖိုးများ၊ ဖရဲကသျှ စုထားနဲ့ တောင်အောက်ရေများမှတ်တမ်း၊ ပင်လယ်ရေမျက်နှာပြင်အချက်အလက်တို့တွင် ထူးခြားချက်များအဖြစ် သိမ်းဆည်းထားသည်။ ပြောင်းလဲလာသည့် ရာသီဥတုများတွင် ကမ္ဘာလုံးဆိုင်ရာအအေးအတက် မြန်မြန်ဖြစ်ခြင်း၊ မိုးရွာသက်ခြင်း လျော့နည်းခြင်း၊ Atlantic ဆားသွယ်စီးကြောင်း ပြတ်တောက်ခြင်းနှင့် ရေခဲတောင် တိုးမြင့်ခြင်း \cite{90,91,92} တို့ ပါဝင်သည်။ ၈,၂၀၀ ပြည့်သိုင်းဖြစ်ရပ်မှာ အထူးသဖြင့် ဘလက်စီးရေကန်တွင် ဆားရေဘက်ထိုးဝင်မှု ကြီးမားစွာဖြစ်ပွားခဲ့နိုင်ခြင်းနှင့် တကွ မတူကွဲပြားသည့် မြန်မြန်ဆုံးဖြစ်ရပ်တစ်ခုဖြစ်သည် (ခန့်မှန်းခြေ မှာ တစ်ဆယ်မြောက် BCE ၆၄၀၀) \cite{93}။

\subsection{သမိုင်းနှင့်ထူးခြားမှုများ}

ရှေးဟောင်းနယ်မြေတချို့တွင် ရုပ်ပုံအလွှာအမျိုးမျိုးတွင် သင်္ဂြိုဟ်ခြင်းနှင့် ဖျက်သိမ်းခြင်းပါဝင်မှုများ တွေ့ရပြီး၊ အတိတ်ကာလ ကြီးမားသည့် ပြင်းထန်သော ဖြစ်စဉ်များကို မှတ်တမ်းတင်ထားသည်။ ဤနယ်မြေတွင် ပထမဦးဆုံးဖြစ်သူ တစ်ဦးမှာ ယေရီခိုမြို့ဟောင်းဖြစ်ပြီး၊ ယနေ့ပလက်စတိုင်းတွင် တည်ရှိသည်။ ဤမြို့တွင် အဖွဲ့လိုက်ဖျက်သိမ်းခြင်းအလွှာများရှိပြီး၊ ကျောက်တည်ဆောက်ပုံ သစ်ပျက်ဆီးမှုနှင့် မီးလောင်သည့်သက်သေများပါဝင်သည် \cite{96,97}။ ၎င်း၏အလွှာများတွင် မှတ်တမ်းတင်ထားသောကာလသည် စုစုပေါင်း BCE ၉,၀၀၀ မှ BCE ၂,၀၀၀ တွင် ဖြစ်သည်။ အထူးဖော်ပြစရာမှာ၊ ၎င်းတွင် တောင်တန်းတစ်ခုသည် BCE ၇,၄၀၀ ခန့်တွင် ဖြတ်ကျွတ်ဖျက်ခံခဲ့၍၊ သဲမြေတွင် သျှံ့သိမ်းခဲ့သည်ကို တွေ့ရသည် (ပုံ \ref{fig:14}) \cite{95}။ Catal Huyuk \cite{99}၊ Gramalote \cite{98}၊ နှင့် ကရီတကျွန်းရှိ မီနိုအန်နန်းတော် Knossos \cite{100,101} တို့တွင်လည်း ထပ်တူသက်သေခံနိုင်သည့် ဖျက်ဆီးမှုအလွှာအမျိုးမျိုးပါဝင်သည့် ရှေးဟောင်းနေရာများ ဖြစ်ကြသည်။

\begin{figure}[t]
\begin{center}
% \fbox{\rule{0pt}{2in} \rule{0.9\linewidth}{0pt}}

\includegraphics[width=1\linewidth]{jericho.jpg}
\end{center}
   \caption{ခေတ်ပထမ ငါးထောင်ခုနစ်ရာခုနှစ်ဒီလီယမ်ခန့် (ရခေတ်စတုရန်း ထာဝ်ယေရိခု မြုပ်နှံသည့်နေရာသမိုင်းလေ့လာမှု ပြုလုပ်ချက်) \cite{95}။}
\label{fig:14}
\label{fig:onecol}
\end{figure}

လူ့အဖွဲ့အစည်းဆိုးရွားစွာဖျက်ဆီးသွားသည့် သက်သေထောက်ခံချက်တစ်ခုမှာ နန်ပါပံုတူပန်းတလင်းဖြစ်သည်။ အဆိုပါပန်းတလင်းကို အီဒါအိုပြည်နယ်တွင် မီးတောင်ခြောက်ကျတင်သော ပန်းတလင်းလျှောက်မီတာ ၁၀၀ ခန့်အောက်တွင် တွေ့ရှိခဲ့သည် \cite{102,103}။ အဆိုပါပန်းတလင်း တွေ့ရှိရာ မီးတောင်ခြောက်ကျသည် နောက်ဆုံးတတိယခေတ် (Late Tertiary) သို့မဟုတ် နိမ့်စုတောင်ခေတ်အစောပိုင်း (Early Quaternary) တွင် တင်ပွဲတင်ကြောင်း ခန့်မှန်းသည်။ အသက်နှစ်သန်းနှစ်ပေါင်း ၂ သန်းကြာခဲ့သည့် မီးတောင်ခြောက်ကျဟုဆိုသော်လည်း၊ ဒေသတွင် ယခုလက်ရှိ မီးတောင်ခြောက်များဟုထင်ရသည်။ တောင့်တင်းသောအဖြစ်များသည် လူ့အဖွဲ့အစည်းကိုဖျက်ဆီးနိုင်သော သဘာဝသဘာဝဘေးအန္တရာယ်ကြီးကြပ်မှုကိုပြသသည့်အပြင် ယနေ့ခေတ် အသက်သက်တမ်းခန့်မှန်းပုံစံများကိုလည်း အားနည်းစေသည်။

\section{ခေတ်သစ် အသက်ခန့်မှန်းနည်းလမ်းများအကြောင်း}

ခေတ်သစ် အသက်ခန့်မှန်းချက်များသည် သက်တမ်းနှစ်ပေါင်း သန်းနှစ်များ၊ သို့မဟုတ် ရာနှစ်ပေါင်း သန်းများအထိ ခန့်မှန်းတတ်ပါသည်။ ထိုအကြောင်းအရ သံသယချပို့ရန် အရေးကြီးသောအကြောင်းပြချက်များရှိသည်။

ပုံမှန်အယူအဆအရ "ဇီဝဖိုးဆိုသော" မီးသွေးသစ်လွင်၊ ဆီနှင့် သဘာဝဓာတ်ငွေ့တို့သည် ရာနှစ်ပေါင်း သန်းများရှိသည်ဟုဆိုသည် \cite{104}။ သို့သော် မက္ကဆီကိုခြမ်းရေနျမစ်တွင် ရရှိသော ဆီကို ကာဗွန်-၁၄ နည်းဖြင့် အသက်ခန့်တွက်သောအခါ နှစ် ၁၃,၀၀၀ ခန့် ဖြစ်သည်ဟုတွေ့ရှိသည် \cite{105}။ ကာဗွန်-၁၄ ၏ သက်တမ်းခွဲအချိန်မှာ (၅,၇၃၀ နှစ်) သာဖြစ်သည့်အတွက် သောင်းနှစ်ပေါင်း ရာနှစ်များအထိ ပြီးဆုံးပျောက်ဆုံးသွားသည့်အထိ ကုန်ဆုံးသွားသင့်သည်။ သို့သော် ထိုခုနှစ်ထက် တစ်ထောင်ဆ ပိုအသက်ကြီးသည့် မီးသွေးသစ်လွင်နှင့် ဇီဝဖိုးပိုင်းများတွင်ပါရှိနေသည့်အချက်သည် သက်တမ်းခန့်မှန်းမှုစနစ်များအား မေးခွန်းထုတ်စေသည် \cite{106}။ ထို့အပြင် သက်တမ်းတိုအလွန်တွင် ထိန်းချုပ်ထားသော အပူချိန်မြင့်အနေအထားတစ်ခုတွင် စက်႐ုံအတွင်း မီးသွေးသစ်လွင်ကို လလ ၂-၈ အတွင်း ထုတ်လုပ်နိုင်ကြောင်းလည်း အတည်ပြုချက်ရှိသည် \cite{107}။

ကာဗွန်-၁၄ သက်တမ်းခန့်မှန်းခြင်းအပြင် အခြား ရေဒီယိုအိုဆိုတိုပ် (Radioisotope) သက်တမ်းခန့်မှန်းနည်းလမ်းများလည်း မှန်ကန်မှုမရှိနိုင်သည်။ Answers in Genesis သုတေသနအဖွဲ့သည် ထိုနည်းလမ်းများအရရရှိလာသည့် အသက်ခန့်မှန်းမှုကိန်းဂဏန်းများတွင် တစ်သက်တစ်သက်ပြဿနာများရှိကြောင်း တွေ့ရှိခဲ့သည် \cite{108}။ သန်းတစ်ရာနှစ်အနှစ်ရှိကြောင်းဆိုသော ဒိုင်နိုဆောအရုပ်များတွင် သွေးကောလဟ်၊ သွေးကြော၊ ကိုလလဂျင်ပါဝင်သော ပျော့ပျော့သားများထပ်မံတွေ့ရှိနိုင်ခဲ့သည် \cite{109,110}။ ဤကဲ့သို့လေ့လာချက်များအရ၊ ကမ္ဘာ့ဂეოလိုဂျီရုပ်ပြဇယားနှင့် သက်တမ်းခန့်မှန်းခံထားသည့် လူ့အဖွဲ့အစည်းသုံးရုပ်ကြွေများအပါအဝင်ရုပ်ပိုင်းဒြပ်စင်များ၏ အသက်လေးစားမှုသည် အလွန်အမင်း မှားယွင်းနေနိုင်ပါသည်။

\section{နိဂုံးချုပ်}

ဤစာတမ်းတွင်၊ သဘာဝဘေးအန္တရာယ်ကြီးကြပ်မှုကြောင့် ဖြစ်လာသည်ဟု မွန်းဆနိုင်သည့် သက်သေများအနက်မှာ အကြောင်းအများဆုံးနှင့် ထင်ရှားသမျှ အယူအဆများကို တင်ပြခဲ့ပါသည်။ မျိုးစုံသော သက်သေများကို တင်ပြထားသော်လည်း၊ ဤစာတမ်းတွင် ဖော်ပြထားသည်မှာ မပြည့်စုံပါ။ ထပ်မံသိရှိလိုပါက research GitHub repository တွင် သက်သေများစုစည်းတင်ထားပါသည် \cite{2}။
\section{အမှတ်တရများ}

ECDO သီအိုရီ၏မူရင်းစာတမ်းစာရေးသူ Ethical Skeptic ကို ဗဟုသုတအရင်းအမြစ်ကြီး၊ စိတ်ဝင်စားဖွယ်နဲ့ လောကျီစွာကမ္ဘာသို့မျှဝေခဲ့တဲ့သူအဖြစ် ကျေးဇူးတင်ပါတယ်။ သူ၏ သုံးပိုင်းသီအိုရီ \cite{1} သည် Exothermic Core-Mantle Decoupling Dzhanibekov Oscillation (ECDO) သီအိုရီအတွက် အာဏာတည်သော လက်စွဲစာအုပ်ဖြစ်ပြီး ဒီမှာအကျဉ်းချုပ်ထားသောအထက်မက ပိုမိုအသေးစိတ်တဲ့ အချက်အလက်များပါဝင်သည်။

ဒါတင်မကဘဲ ယနေ့ကျွန်ုပ်တို့ရပ်တည်ရာ ဦးခေါင်းရပ်တည်နိုင်စေသော မဟာမင်းတို့အားလုံးကိုလည်း ကျေးဇူးတင်ပါတယ်။ သူတို့၏ သုတေသနနှင့် စူးစမ်းလေ့လာမှုများကြောင့် စာတမ်းအလုပ်ရှုပ်သည် ဖြစ်လာခဲ့သော်လည်း လူသားအပေါ် အလင်းရောင် မီးမောင်းထိုးပေးလာကြသည်။

\clearpage
\twocolumn

{\small
\renewcommand{\refname}{ရင်းမြစ်များ}
\bibliographystyle{ieee}
\bibliography{egbib}
}

\end{document}
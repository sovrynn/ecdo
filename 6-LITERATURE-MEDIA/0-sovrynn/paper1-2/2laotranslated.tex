\documentclass[10pt,twocolumn,letterpaper]{article}

% My own stuff
\usepackage{booktabs}
% \usepackage{caption}
% \captionsetup[table]{skip=8pt}   % Only affects tables
\usepackage{stfloats}  % Add this to the preamble

%–– line-breaks for Lao instead of Thai ––
\XeTeXlinebreaklocale "lo"
\XeTeXlinebreakskip = 0pt plus 1pt

\usepackage{fontspec}
\usepackage{ucharclasses}

%–– define your two fonts ––
\newfontfamily\latinfont{Latin Modern Roman}         % for all non-Lao (e.g. Latin) text
\newfontfamily\laofont[Script=Lao]{Noto Serif Lao}   % for all Lao text
% \newfontfamily\laofont[Script=Lao]{FreeSerif}   % for all Lao text

%–– ucharclasses auto-detects Unicode blocks ––
\setDefaultTransitions{\latinfont}{}                   % outside Lao → Latin
\setTransitionsFor{Lao}{\laofont}{\latinfont}          % Lao block → Lao font, then back

\usepackage{cvpr}
\usepackage{times}
\usepackage{epsfig}
\usepackage{graphicx}
\usepackage{amsmath}
\usepackage{amssymb}
\usepackage[breaklinks=true,bookmarks=false]{hyperref}

% Include other packages here, before hyperref.
 
% If you comment hyperref and then uncomment it, you should delete
% egpaper.aux before re-running latex.  (Or just hit 'q' on the first latex
% run, let it finish, and you should be clear\usepackage[breaklinks=true,bookmarks=false]{hyperref}

\cvprfinalcopy % *** Uncomment this line for the final submission

\def\cvprPaperID{****} % *** Enter the CVPR Paper ID here
\def\httilde{\mbox{\tt\raisebox{-.5ex}{\symbol{126}}}}

% 1) Choose your desired fixed leading:
\renewcommand\baselinestretch{1.2}  % or 1.3, 1.1…  adjust to taste

% 2) Force TeX to *always* use \baselineskip, never fall back to \lineskip:
\makeatletter
  \setlength\lineskiplimit{-\maxdimen} % always allow baselineskip
  \setlength\lineskip{0pt}             % no extra glue ever
\makeatother

% \renewcommand{\tablename}{ตาราง}
\renewcommand{\figurename}{ຮູບພາບ}   % or whatever you like instead of "Hình"
\renewcommand{\refname}{ເອກະສານອ້າງອີງ}

\makeatletter
\def\abstract{%
  \centerline{\large\bf ບົດສະຫຼຸບຫຍໍ້}% <-- your new label
  \vspace*{12pt}%
  \it%
}
\makeatother

% This makes the font slightly bigger than base (10) and bold in Subsection headings rather than using ptmb
\makeatletter
\def\cvprsubsection{%
  \@startsection{subsection}{2}{\z@}%
    {8pt plus 2pt minus 2pt}{6pt}%
    % {\normalfont\bfseries\selectfont}%
    {\normalfont\bfseries\fontsize{11}{13}\selectfont}%
}
\makeatother

% So this hardcodes the style for the numbers in the section/subsection headings so they're bold
\font\elvbf=ptmb scaled 1100
\font\elvbfs=ptmb scaled 1200
\makeatletter
% Section number: Large + bold
\renewcommand\thesection{%
  {\elvbfs\arabic{section}}%
}

% Subsection number: normalsize + bold + custom punctuation
\renewcommand\thesubsection{%
  {\elvbf
   \arabic{section}.\arabic{subsection}}%
}
\makeatother

% Pages are numbered in submission mode, and unnumbered in camera-ready
%\ifcvprfinal\pagestyle{empty}\fi
\setcounter{page}{1}
\begin{document}

%%%%%%%%% TITLE
\title{ຄູ່ມື ຖານຂໍ້ມູນ ECDO ຕອນທີ 2/2: ການກວດສອບຄວາມຜິດປົກກະຕິທາງວິທະຍາສາດ ແລະ ປະຫວັດສາດທີ່ອະທິບາຍໄດ້ດີທີ່ສຸດດ້ວຍ "ການພິກໂລກ" ຂອງ ECDO}

\author{Junho\\
ເຜີ່ຍແຜ່ເດືອນກຸມພາ ປີ 2025\\
ເວັບໄຊ (ດາວໂຫຼດເອກະສານໄດ້ທີ່ນີ້): \href{https://sovrynn.github.io}{sovrynn.github.io}\\
ຄັງຂໍ້ມູນການຄົ້ນຄວ້າຂອງ ECDO: \href{https://github.com/sovrynn/ecdo}{github.com/sovrynn/ecdo}\\
{\tt\small junhobtc@proton.me}
% For a paper whose authors are all at the same institution,
% omit the following lines up until the closing ``}''.
% Additional authors and addresses can be added with ``\and'',
% just like the second author.
% To save space, use either the email address or home page, not both
% \and
% xx
% Institution2\\
% First line of institution2 address\\
% {\tt\small secondauthor@i2.org}
}
\maketitle
%\thispagestyle{empty}

%%%%%%%%% ABSTRACT
\begin{abstract}
ໃນເດືອນພຶດສະພາ ປີ 2024, ຜູ້ຂຽນອອນລາຍທີ່ໃຊ້ນາມແຝງວ່າ “The Ethical Skeptic” \cite{0} ໄດ້ໂພສທິດສະດີໃໝ່ທີ່ເອີ້ນວ່າ ການແຍກຕົວຂອງແກນໂລກເພື່ອປົດປ່ອຍຄວາມຮ້ອນດ້ວຍການສັ່ນສະເທືອນ (Exothermic Core-Mantle Decoupling Dzhanibekov Oscillation - ECDO) \cite{1}. ທິດສະດີນີ້ບໍ່ພຽງແຕ່ສະເໜີວ່າ ໂລກເຄີຍປະສົບກັບການປ່ຽນແປງແກນໝຸນຢ່າງກະທັນຫັນ ແລະ ຮຸນແຮງ ເຊິ່ງກໍ່ໃຫ້ເກີດນ້ຳຖ້ວມຄັ້ງໃຫຍ່ທົ່ວໂລກ ໂດຍເຮັດໃຫ້ມະຫາສະໝຸດໄຫຼລົ້ນທະວີບຕ່າງໆ ເນື່ອງຈາກຄວາມຕ້ານທານຂອງການໝຸນ, ແຕ່ຍັງສະເໜີຂະບວນການທາງທໍລະນີຟິຊິກ ເຊິ່ງອະທິບາຍພ້ອມຂໍ້ມູນທີ່ບົ່ງຊີ້ວ່າການປ່ຽນແປງດັ່ງກ່າວອາດຈະເກີດຂຶ້ນໃນໄວໆນີ້. ເຖິງແມ່ນວ່າ ການພະຍາກອນນ້ຳຖ້ວມຄັ້ງໃຫຍ່ ແລະ ວັນສິ້ນໂລກຈະບໍ່ແມ່ນເລື່ອງໃໝ່, ແຕ່ທິດສະດີ ECDO ກໍຍັງໜ້າສົນໃຈເປັນພິເສດ ເນື່ອງຈາກເປັນແນວທາງ ເນື່ອງຈາກເປັນແນວທາງທາງວິທະຍາສາດ, ທັນສະໄໝ, ຫຼາກຫຼາຍສາຂາວິຊາ ແລະ ມີແນວທາງອີງຕາມຂໍ້ມູນ.

ເອກະສານຄົ້ນຄວ້ານີ້ແມ່ນສ່ວນທີສອງຂອງເນື້ອຫາສັງລວມໂດຍຫຍໍ້ສອງສ່ວນ ຈາກການຄົ້ນຄວ້າອິດສະຫຼະ 6 ເດືອນ \cite{2,20} ກ່ຽວກັບທິດສະດີ ECDO, ໂດຍການເນັ້ນສະເພາະຄວາມຜິດປົກກະຕິທາງວິທະຍາສາດ ແລະ ປະຫວັດສາດທີ່ອະທິບາຍໄດ້ດີທີ່ສຸດດ້ວຍ "ການພິກໂລກ" ຂອງທິດສະດີ ECDO.

\end{abstract}

%%%%%%%%% BODY TEXT

\section{ຄຳນຳ}

ທໍລະນີສາດ ແລະ ປະຫວັດສາດແບບເອກຮູບສະໄໝໃໝ່ອ້າງວ່າ ພູມສັນຖານທາງທໍລະນີສາດທີ່ສຳຄັນ ເຊັ່ນ: ແກຣນແຄນຢອນ ໄດ້ກໍ່ຕົວຂຶ້ນເມື່ອຫຼາຍລ້ານປີກ່ອນ \cite{143}; ເກືອມີຢູ່ໃນ ເດດວັນເລ (ແຄລິຟໍເນຍ) ເນື່ອງຈາກເຄີຍຢູ່ກ້ອງທະເລເມື່ອຫຼາຍຮ້ອຍລ້ານປີກ່ອນ \cite{144}; ບັນພະບຸລຸດຂອງພວກເຮົາເມື່ອ 150 ຊົ່ວຄົນກ່ອນ ໃຊ້ຊີວິດທັງໝົດຂອງພວກເຂົາໃນການສ້າງສຸສານຂະໜາດຍັກ \cite{29,70}; ແລະ ສິ່ງທີ່ເອີ້ນວ່າ "ເຊື້ອເພີງຊາກດຶກດຳບັນ" ແມ່ນມີອາຍຸຫຼາຍຮ້ອຍລ້ານປີ \cite{104}. ສິ່ງທີ່ໜ້າສົນໃຈທີ່ສຸດກໍຄື ເຊື່ອວ່າມະນຸດມີອາຍຸ 300,000 ປີ \cite{145}, ແຕ່ປະຫວັດສາດ ແລະ ອາລິຍະທຳທີ່ບັນທຶກໄວ້ມີອາຍຸພຽງແຕ່ປະມານ 5,000 ປີ - ເຊິ່ງທຽບເທົ່າກັບມະນຸດ 150 ຊົ່ວຄົນ.

ຄວາມຜິດປົກກະຕິດັ່ງກ່າວທີ່ພວກເຮົາຈະເຫັນນັ້ນ ແມ່ນສາມາດອະທິບາຍໄດ້ດີທີ່ສຸດດ້ວຍພະລັງໄພພິບັດທາງທໍລະນີສາດ.

\section{ແມັມມັອດທີ່ຖືກແຊ່ແຂງຢ່າງກະທັນຫັນຝັງຢູ່ໃນດິນຕົມ}

\begin{figure}[b]
\begin{center}
% \fbox{\rule{0pt}{2in} \rule{0.9\linewidth}{0pt}}
   \includegraphics[width=1\linewidth]{jarkov-mammoth.jpg}
\end{center}
   \caption{ແມັມມັອດ Jarkov, ແມັມມັອດໄຊບີເຣຍ ອາຍຸ 20,000 ປີ ທີ່ໄດ້ຮັບການອະນຸລັກໄວ້ຢ່າງສົມບູນແບບ ພົບເຫັນຢູ່ໃນດິນຕົມທີ່ແຂງໂຕ \cite{51}.}
\label{fig:1}
\label{fig:onecol}
\end{figure}

ຄວາມຜິດປົກກະຕິປະເພດໜຶ່ງດັ່ງກ່າວແມ່ນແມັມມັອດທີ່ແຂງໂຕຈົນສົມບູນ ເຊິ່ງຝັງຢູ່ໃນດິນຕົມ, ເຊິ່ງມັກຈະພົບເຫັນຢູ່ເຂດພາກພື້ນອາກຕິກ (ຮູບທີ \ref{fig:1}). ແມັມມັອດ Beresovka, ເຊິ່ງຄົ້ນພົບໃນໄຊບີເຣຍ ຖືກຝັງໄວ້ໃນຫີນແຮ່ທີ່ມີຕະກອນ ເຊິ່ງໄດ້ຮັບການເກັບຮັກສາໄວ້ເປັນຢ່າງດີ ຈົນຊີ້ນຂອງມັນຍັງສາມາດກິນໄດ້ ເຖິງແມ່ນວ່າຈະຕາຍໄປແລ້ວຫຼາຍພັນປີກໍຕາມ. ນອກຈາກນີ້ ມັນຍັງມີອາຫານທີ່ເປັນພືດຢູ່ໃນປາກ ແລະ ກະເພາະ, ເຮັດໃຫ້ບັນດານັກວິທະຍາສາດປະຫຼາດໃຈວ່າເປັນຫຍັງມັນຈຶ່ງຖືກແຊ່ແຂງໄດ້ຢ່າງໄວວາ ຖ້າມັນຍັງກິນພືດດອກຢູ່ກ່ອນທີ່ມັນຈະຕາຍ \cite{17}. ມີການລາຍງານວ່າ, \textit{"ໃນປີ 1901 ມີຄວາມຕື່ນເຕັ້ນເກີດຂຶ້ນ ເມື່ອພົບເຫັນຊາກແມັມມັອດທີ່ສົມບູນໝົດທັງໂຕ ຢູ່ໃກ້ກັບແມ່ນ້ຳ Berezovka, ເນື່ອງຈາກສັດໂຕນີ້ເບິ່ງຄືວ່າຈະຕາຍຍ້ອນຄວາມໜາວເຢັນໃນຊ່ວງກາງລະດູຮ້ອນ. ສິ່ງທີ່ຢູ່ໃນກະເພາະຂອງມັນຍັງຄົງສະພາບດີຢູ່ ແລະ ລວມເຖິງດອກບັດເຕີຄັບ ແລະ ໝາກຖົ່ວປ່າທີ່ອອກດອກ: ນັ້ນໝາຍຄວາມວ່າ ພວກມັນຕ້ອງຖືກກືນລົງໄປໃນຊ່ວງປາຍເດືອນກໍລະກົດ ຫຼື ຕົ້ນເດືອນສິງຫາ. ສັດໂຕນີ້ຕາຍຢ່າງກະທັນຫັນຫຼາຍ ຈົນຍັງຄາບຫຍ້າ ແລະ ດອກໄມ້ໄວ້ໃນປາກ. ເຫັນໄດ້ຢ່າງຊັດເຈນວ່າ ພວກມັນຖືກແຮງອັນມະຫາສານພັດພາໄປ ແລະ ຟົ້ງໄປໄກຈາກທົ່ງຫຍ້າຫຼາຍໄມລ໌. ກະດູກຊາມ ແລະ ຂາຂ້າງໜຶ່ງຫັກ ສັດໂຕໃຫຍ່ໂຕນັ້ນໄດ້ລົ້ມລົງ ແລະ ຖືກແຊ່ແຂງຈົນຕາຍ, ເຊິ່ງໂດຍປົກກະຕິແລ້ວຈະແມ່ນໄລຍະທີ່ຮ້ອນທີ່ສຸດຂອງປີ"} \cite{18}. ນອກຈາກນີ້, \textit{"[ນັກວິທະຍາສາດຣັດເຊຍ] ໄດ້ບັນທຶກໄວ້ວ່າ ແມ້ແຕ່ເນື້ອເຍື່ອຊັ້ນໃນສຸດຂອງກະເພາະຂອງສັດຮ້າຍກໍຍັງມີໂຄງສ້າງເສັ້ນໃຍທີ່ຍັງຄົງສະພາບໄວ້ໄດ້ຢ່າງສົມບູນແບບ, ເຊິ່ງບົ່ງຊີ້ວ່າ ຄວາມຮ້ອນໃນຮ່າງກາຍຂອງມັນໄດ້ຖືກກຳຈັດອອກໄປດ້ວຍຂະບວນການທີ່ໜ້າອັດສະຈັນບາງຢ່າງໃນທຳມະຊາດ. Sanderson ເຊິ່ງໃຫ້ຄວາມສົນໃຈເປັນພິເສດໃນປະເດັນນີ້ ໄດ້ນຳເອົາບັນຫາດັ່ງກ່າວໄປປຶກສາກັບສະຖາບັນອາຫານແຊ່ງແຂງຂອງສະຫະລັດ: ຈະຕ້ອງເຮັດແນວໃດຈຶ່ງຈະແຊ່ແຂງແມັມມັອດໝົດທັງໂຕໄດ້ ຈົນຄວາມຊື້ນໃນຊັ້ນໃນສຸດຂອງຮ່າງກາຍຂອງມັນ ແມ້ແຕ່ເນື້ອເຍື່ອຊັ້ນໃນຂອງກະເພາະ ກໍບໍ່ມີເວລາພຽງພໍທີ່ຈະກໍ່ໂຕເປັນຜະລຶກຂະໜາດໃຫຍ່ພໍທີ່ຈະທຳລາຍໂຄງສ້າງເສັ້ນໃຍຂອງຊີ້ນໄດ້?... ບໍ່ພໍເທົ່າໃດອາທິດຕໍ່ມາ ສະຖາບັນໄດ້ກັບມາຫາ Sanderson ພ້ອມຄຳຕອບວ່າ: ມັນເປັນໄປບໍ່ໄດ້ແທ້ໆ ດ້ວຍຄວາມຮູ້ທາງວິທະຍາສາດ ແລະ ວິສະວະກຳທັງໝົດຂອງພວກເຮົາ ບໍ່ມີວິທີໃດທີ່ຈະຮູ້ໄດ້ຢ່າງຊັດເຈນເລີຍວ່າ ຈະກຳຈັດຄວາມຮ້ອນໃນຮ່າງກາຍອອກຈາກຊາກສັດທີ່ມີຄວາມໃຫຍ່ຂະໜາດໂຕແມັມມັອດໄດ້ໄວພໍທີ່ຈະແຊ່ແຂງ ໂດຍບໍ່ໃຫ້ຜະລຶກຄວາມຊື້ນຂະໜາດໃຫຍ່ກໍ່ຕົວໃນຊີ້ນ. ນອກຈາກນີ້ ຫຼັງຈາກໃຊ້ເຕັກນິກທາງວິທະຍາສາດ ແລະ ວິສະວະກຳຈົນໝົດແລ້ວ, ພວກເຂົາກໍໄດ້ຫັນໄປເບິ່ງທຳມະຊາດ ແລະ ສະຫຼຸບວ່າ ບໍ່ມີຂະບວນການໃດໃນທຳມະຊາດທີ່ຈະຮູ້ໄດ້ຢ່າງຊັດເຈນ ທີ່ຈະສາມາດບັນລຸຜົນສຳເລັດດັ່ງກ່າວໄດ້"} \cite{19}.

\section{ແກຣນແຄນຢອນ}

ແກຣນແຄນຢອນ ເຊິ່ງເປັນສ່ວນໜຶ່ງຂອງແອ່ງ ເກຣທເບຊິນ ທາງຕາເວັນຕົກສຽງໃຕ້ຂອງອາເມຣິກາເໜືອ ເປັນປາກົດການທາງທຳມະຊາດອີກປະການໜຶ່ງ ທີ່ບົ່ງບອກເຖິງຕົ້ນກຳເນີດຂອງຄວາມຫາຍະນະ (ຮູບທີ \ref{fig:2}). ເລີ່ມຈາກຊັ້ນຫີນຊາຍ ແລະ ຫີນປູນຕະກອນທີ່ປະກອບເປັນແກຣນແຄນຢອນ ເຊິ່ງມີເນື້ອທີ່ກວ້າງໃຫຍ່ເຖິງ 2.4 ລ້ານ km$^2$ \cite{21}. ຮູບທີ \ref{fig:3} ສະແດງໃຫ້ເຫັນຄວາມກວ້າງຂອງຊັ້ນຫີນຊາຍ ໂຄໂຄນິໂນ ທີ່ແຜ່ຂະຫຍາຍໄປທົ່ວທາງຕາເວັນຕົກຂອງສະຫະລັດອາເມຣິກາ. ຊັ້ນຫີນຊາຍຂະໜາດໃຫຍ່ທີ່ມີວັດສະດຸຄືກັນໃນລວງນອນດັ່ງກ່າວ ແມ່ນສາມາດເກີດຂຶ້ນໄດ້ພ້ອມກັນທັງໝົດເທົ່ານັ້ນ.

\begin{figure}[t]
\begin{center}
% \fbox{\rule{0pt}{2in} \rule{0.9\linewidth}{0pt}}
   \includegraphics[width=1\linewidth]{grand-canyon.jpg}
\end{center}
   \caption{ແກຣນແຄນຢອນ ໃນອາຣິໂຊນາ, ສະຫະລັດອາເມຣິກາ \cite{49}.}
\label{fig:2}
\label{fig:onecol}
\end{figure}

\begin{figure}[t]
\begin{center}
% \fbox{\rule{0pt}{2in} \rule{0.9\linewidth}{0pt}}
   \includegraphics[width=1\linewidth]{coconino.jpg}
\end{center}
   \caption{ຂະໜາດຂອງຊັ້ນຫີນຊາຍ Coconino ທາງຕາເວັນຕົກຂອງສະຫະລັດອາເມຣິກາ \cite{21}.}
\label{fig:3}
\label{fig:onecol}
\end{figure}

ເມື່ອພິຈາລະນາ ແກຣນແຄນຢອນ ຢ່າງໃກ້ຊິດ ພວກເຮົາຈະພົບວ່າ ການທັບຖົມຂອງຊັ້ນຕະກອນທີ່ຂະຫຍາຍຕົວເຫຼົ່ານີ້ເກີດຂຶ້ນພ້ອມກັນກັບຄວາມແຮງທາງທໍລະນີສາດທີ່ສຳຄັນ. ເພື່ອທຳຄວາມເຂົ້າໃຈເລື່ອງນີ້, ພວກເຮົາຕ້ອງພິຈາລະນາເນື້ອທີ່ບາງສ່ວນໃນຮ່ອມພູຢ່າງໃກ້ຊິດ ເຊິ່ງຊັ້ນຕະກອນຖືກພັບຊ້ອນ ແລະ ເປີດອອກ. ນັກຄົ້ນຄວ້າຈາກ Answers in Genesis \cite{42} ໄດ້ກວດເບິ່ງຕົວຢ່າງຫີນຈາກຮອຍພັບເຫຼົ່ານີ້ໃນລະດັບຈຸນລະພາກ ເຊັ່ນ ຮອຍພັບ Monument ແລະ ຈາກການຂາດຄຸນສົມບັດທີ່ຄວນມີຢູ່ ຖ້າຮອຍພັບເກີດຂຶ້ນໃນໄລຍະເວລາອັນຍາວນານ ພາຍໃຕ້ຄວາມຮ້ອນ ແລະ ແຮງກົດດັນ, ຈຶ່ງສະຫຼຸບໄດ້ວ່າ ຊັ້ນຕະກອນຖືກພັບໂດຍຄວາມແຮງທາງທໍລະນີສາດ ໃນຂະນະທີ່ຍັງອ່ອນຢູ່ ນັ້ນກໍຄື ບໍ່ດົນຫຼັງຈາກການຕົກຕະກອນ \cite{43}.

\begin{figure*}
\begin{center}
% \fbox{\rule{0pt}{2in} \rule{.9\linewidth}{0pt}}
\includegraphics[width=1\textwidth]{Grand_Staircase-big.jpg}
\end{center}
   \caption{ຊັ້ນຕະກອນທີ່ປະກອບເປັນ ແກຣນແຄນຢອນ (ດ້ານຂວາຂອງຮູບ) ຢຽດຍາວໄປທາງເໜືອໂດຍກົງ ສູ່ ເຊດາ ເບຣກ ລັດຢູທາ (ດ້ານຊ້າຍຂອງຮູບ) ເຊິ່ງຊັ້ນຕະກອນທັງໝົດຈະໂຄ້ງຂຶ້ນໄປ \cite{50}.}
\label{fig:4}
\end{figure*}

ເມື່ອຊູມອອກ, ພວກເຮົາຈະພົບວ່າ ຊັ້ນຕ່າງໆທີ່ປະກອບກັນເປັນແກຣນແຄນຢອນແມ່ນບໍ່ໄດ້ຖືກພັບຊ້ອນແຕ່ພາຍໃນຮ່ອມພູເທົ່ານັ້ນ, ຊັ້ນຕ່າງໆຍັງຖືກພັບຊ້ອນໄປທາງທິດຕາເວັນອອກໃນ Kaibab Monocline \cite{46}, ແລະກໍຍັງໄປທາງທິດເໜືອໃນ ເຊດາ ເບຣກ, ລັດຢູທາ (ຮູບທີ \ref{fig:4}). ເຊິ່ງສະແດງໃຫ້ເຫັນວ່າຊັ້ນຕ່າງໆເຫຼົ່ານີ້ ອາດຈະຖືກພັບຊ້ອນເຂົ້າໃສ່ກັນຫຼັງຈາກທີ່ຖືກວາງທັບຊ້ອນຢ່າງວ່ອງໄວ. ເພື່ອອ້າງອີງ, ຊັ້ນລວງນອນຂອງແກຣນແຄນຢອນມີຄວາມໜາປະມານ 1700 ແມັດ. ຂະບວນການທາງທໍລະນີສາດທີ່ຈຳເປັນໃນການວາງຊັ້ນຕະກອນທີ່ມີຄວາມໜາ 1 ໄມລ໌ນັ້ນແມ່ນມີຂະໜາດໃຫຍ່ຫຼາຍ.

ການກໍ່ໂຕຂຶ້ນຂອງແກຣນແຄນຢອນແມ່ນອີກປະເດັນໜຶ່ງທີ່ຖົກຖຽງກັນໃນທໍລະນີສາດສະໄໝໃໝ່. ທໍລະນີສາດແບບເອກະພາບສະເໜີວ່າ ແກຣນແຄນຢອນຖືກກັດເຊາະໂດຍແມ່ນ້ຳ Colorado ເປັນເວລາຫຼາຍລ້ານປີ \cite{47}. ຢ່າງໃດກໍຕາມ, ທີມຄົ້ນຄວ້າ Answers in Genesis ເຊື່ອວ່າ ແກຣນແຄນຢອນໜ້າຈະກໍ່ໂຕຂຶ້ນພາຍໃນເວລາບໍ່ພໍເທົ່າໃດອາທິດ ເນື່ອງຈາກການກັດເຊາະທາງລະບາຍນ້ຳຈາກທະເລສາບບູຮານທີ່ລ້ຳຂອບເຂດ, ເຊິ່ງເຮັດໃຫ້ຕະກອນຈຳນວນຫຼາຍຖືກກັດເຊາະຈົນເປັນຮ່ອມພູ. ມີຫຼັກຖານຂອງທະເລສາບທີ່ຢູ່ສູງທາງຕາເວັນອອກຂອງແກຣນແຄນຢອນຢູ່ໃນຕະກອນຂອງທະເລສາບ ແລະ ຊາກດຶກດຳບັນທາງທະເລ. ເມື່ອປຽບທຽບແກຣນແຄນຢອນກັບຕົວຢ່າງການກັດເຊາະທາງລະບາຍນ້ຳຂະໜາດໃຫຍ່ຕົວຢ່າງອື່ນໆ ເຊັ່ນ ຮ່ອມພູ ອາຟຕັນ ແລະ ພູເຂົາ ເຊັນ ເຮເລນ, ຈະເຫັນລັກສະນະທາງພູມສັນຖານທີ່ຄ້າຍຄືກັນ ແລະ ສະແດງໃຫ້ເຫັນວ່າ ຮ່ອມພູຂະໜາດໃຫຍ່ສາມາດເກີດຂຶ້ນໄດ້ຢ່າງວ່ອງໄວ ຈາກນ້ຳທີ່ໄຫຼໃນປະລິມານຫຼາຍ \cite{48}. 

ເມື່ອພິຈາລະນາເຖິງຂະໜາດຂອງຂະບວນການທາງທໍລະນີສາດທີ່ຈຳເປັນໃນການທັບຖົມຕະກອນໃນພື້ນທີ່ຂະໜາດໃຫຍ່, ຄວາມແຮງທາງທໍລະນີສາດອັນມະຫາສານທີ່ເກີດຂຶ້ນໃນເວລາບໍ່ດົນຫຼັງຈາກຊັ້ນຕະກອນທັບຖົມ ແລະ ຂະໜາດທີ່ນ້ອຍໆຂອງແມ່ນ້ຳ Colorado ເມື່ອທຽບກັບຂະໜາດມະຫຶມາຂອງແກຣນແຄນຢອນ, ເບິ່ງຄືວ່າ ການກໍ່ຕົວຂອງແມ່ນ້ຳອາດຈະບໍ່ໄດ້ມີລັກສະນະຄ່ອຍເປັນຄ່ອຍໄປແຕ່ຢ່າງໃດ. 

\section{ເມືອງໃຕ້ດິນ ເດີຣິນກູຢູ}

ນອກຈາກປີຣາມິດແລ້ວ ຕົວຢ່າງທີ່ຍິ່ງໃຫຍ່ຂອງວິສະວະກຳສາດບູຮານກໍຄື ເມືອງໃຕ້ດິນ ເດີຣິນກູຢູ (ຮູບທີ \ref{fig:5}), ເຊິ່ງຕັ້ງຢູ່ໃນ ແຄບພາໂດເຊຍ, ປະເທດຕຸລະກີ. ເມືອງໃຕ້ດິນແຫ່ງນີ້ແມ່ນເມືອງໃຕ້ດິນທີ່ໃຫຍ່ທີ່ສຸດໃນບັນດາເມືອງໃຕ້ດິນ 200 ກວ່າແຫ່ງໃນພາກພື້ນນີ້ \cite{54}. ຄາດຄະເນວ່າເມືອງໃຕ້ດິນແຫ່ງນີ້ເຄີຍເປັນທີ່ຢູ່ອາໄສຂອງປະຊາຊົນຫຼາຍກວ່າ 20,000 ຄົນ ແລະ ຢຽດຍາວ 18 ຊັ້ນ ແລະ ເລິກໄປເຖິງ 85 ແມັດ. ເຖິງແມ່ນວ່າຈະບໍ່ຮູ້ຊັດເຈນວ່າເມືອງນີ້ມີອາຍຸເທົ່າໃດ ແຕ່ຄາດຄະເນວ່າ ໜ້າຈະມີອາຍຸຢ່າງໜ້ອຍ 2800 ປີ. ເມືອງນີ້ຖືກແກະສະຫຼັກຈາກຫີນພູເຂົາໄຟອ່ອນ \cite{52, 53}.

\begin{figure}[b]
\begin{center}
% \fbox{\rule{0pt}{2in} \rule{0.9\linewidth}{0pt}}
   \includegraphics[width=1\linewidth]{derinkuyu.jpeg}
\end{center}
   \caption{ແຜນຜັງເມືອງໃຕ້ດິນ ເດີຣິນກູຢູ \cite{56}.}
\label{fig:5}
\label{fig:onecol}
\end{figure}

ເຫດຜົນທີ່ເມືອງເດີຣິນກູຢູ ມີຄວາມໜ້າສົນໃຈ ກໍເພາະວ່າ ມັນບໍ່ຊັດເຈນວ່າເປັນຫຍັງຈຶ່ງມີຊຸມຊົນຕັດສິນໃຈສ້າງເມືອງໃຕ້ດິນທັງໝົດ. ເພື່ອຈະສ້າງພື້ນທີ່ຢູ່ອາໄສຢູ່ໃຕ້ດິນ ຈຳເປັນຕ້ອງຂຸດທຸກຊ່ອງຈາກຫີນໃຫ້ເປັນຜົ້ງ. ຮູບຮ່າງ ແລະ ພື້ນຜິວທີ່ຫຍາບຂອງອຸໂມງໃຕ້ດິນເຮັດໃຫ້ເຫັນຢ່າງຊັດເຈນວ່າຕ້ອງແມ່ນຂຸດດ້ວຍແຮງງານຄົນ ບໍ່ແມ່ນໃຊ້ເຄື່ອງມືໄຟຟ້າ, ເຊິ່ງຍາກກວ່າການສ້າງບ່ອນພັກອາໄສຢູ່ເທິງພື້ນດິນຫຼາຍເທົ່າ. ໃນຄວາມເປັນຈິງ, ບໍ່ຊັດເຈນວ່າເປັນຫຍັງ ມະນຸດຈຶ່ງຢາກອາໄສຢູ່ໃຕ້ດິນຢ່າງຖາວອນໃນຊ່ວງຊີວິດເທິງໂລກນີ້ ເມື່ອການກະເສດ, ແສງແດດ, ທຳມະຊາດ ແລະ ການສຳຫຼວດແມ່ນມີໃຫ້ສະເພາະຢູ່ເທິງພື້ນດິນເທົ່ານັນ. "ປະຫວັດສາດ" ຕາມທຳນຽມສະເໜີວ່າ ເມືອງ Derinkuyu ຖືກສ້າງຂຶ້ນໂດຍຊາວຄຣິສຕຽນທີ່ຕ້ອງການສະຖານທີ່ງຽບສະຫງົບ ເພື່ອປະກອບພິທີກຳທາງສາສະໜາຂອງຕົນ \cite{53}. ສາມັນສຳນຶກຈະສະຫຼຸບໄດ້ວ່າ ວິທີທີ່ກົງໄປກົງມາທີ່ສຸດໃນການຈັດການກັບສັດຕູແມ່ນ "ຕໍ່ສູ້ ຫຼື ໜີ", ບໍ່ແມ່ນ "ຂຸດເມືອງໃຕ້ດິນຈາກຫີນ".

ຂະໜາດ, ຄວາມເລິກ ແລະ ຄວາມໃສ່ໃຈໃນການອອກແບບເມືອງໃຕ້ດິນ ເຮັດໃຫ້ເຫັນຊັດເຈນວ່າ ບໍ່ໄດ້ອອກແບບມາໃຫ້ເປັນໂຄງສ້າງປ້ອງກັນທາງທະຫານຊົ່ວຄາວ ເພື່ອຕໍ່ສູ້ກັບຜູ້ຮຸກຮານໃນເວລາຄັບຂັນ ແຕ່ເປັນບ່ອນພັກອາໄສໄລຍະຍາວ ເພື່ອປ້ອງກັນກອງກຳລັງທີ່ກໍ່ອັນຕະລາຍເທິງໜ້າດິນ. ເມືອງ Derinkuyu ບໍ່ພຽງແຕ່ມີຫ້ອງນອນ, ເຮືອນຄົວ ແລະ ຫ້ອງນ້ຳພື້ນຖານເທົ່ານັ້ນ, ແຕ່ຍັງມີຄອກສັດ, ຖັງເກັບນ້ຳ, ໂຮງເກັບອາຫານ, ໂຮງຄັ້ນວາຍ ແລະ ນ້ຳມັນ, ໂຮງຮຽນ, ໂບດ, ສຸສານ ແລະ ຊ່ອງລະບາຍອາການຂະໜາດໃຫຍ່ (ຮູບທີ \ref{fig:6}). ເປັນຫຍັງບ່ອນພັກອາໄສທາງການທະຫານຈຶ່ງຕ້ອງມີເຄື່ອງຄັ້ນວາຍ ແລະ ຕ້ອງຂຸດເລິກລົງໄປ 85 ແມັດ ເຊິ່ງມີຄວາມຊັບຊ້ອນແບບນີ້?

ຄຳອະທິບາຍທີ່ໜ້າເຊື່ອຖືທີ່ສຸດສຳລັບການສ້າງເມືອງເດີຣິນກູຢູ ກໍຄືຄວາມຈຳເປັນຮີບດ່ວນໃນການກະກຽມບ່ອນພັກອາໄສໄລຍະຍາວ ທີ່ສາມາດເພິ່ງພາຕົນເອງໄດ້ ເພື່ອປ້ອງກັນກອງກຳລັງທາງທໍລະນີຟິສິກທີ່ໂຫດຮ້າຍເທິງພື້ນໂລກ.

\begin{figure}[t]
\begin{center}
% \fbox{\rule{0pt}{2in} \rule{0.9\linewidth}{0pt}}
   \includegraphics[width=1\linewidth]{derinkuyu-air.jpg}
\end{center}
   \caption{ບໍ່ນ້ຳລະບາຍອາກາດເລິກໃນເມືອງເດີຣິນກູຢູ \cite{53}.}
\label{fig:6}
\label{fig:onecol}
\end{figure}

% \section{Additional Anomalies Best Explained By An Earth Flip}

% Before wrapping up, we will mention some additional scientific anomalies that, once viewed in the context of cataclysmic geophysical forces, are well explained.

\section{ການສະສົມຊີວະມວນ}

ສ່ວນປະສົມຂອງຊີວະມວນຂອງສັດ ແລະ ພືດຕ່າງໆ ເຊິ່ງມັກພົບເຫັນເປັນຊາກດຶກດຳບັນໃນຊັ້ນຕະກອນແມ່ນເປັນຄວາມຜິດປົກກະຕິທີ່ໜ້າປະຫຼາດໃຈອີກຢ່າງໜຶ່ງ. ໃນໜັງສື "Reliquoæ Diluvianæ", Rev. William Buckland ໄດ້ໃຫ້ລາຍລະອຽດກ່ຽວກັບການຄົ້ນພົບສັດຫຼາຍສາຍພັນທີ່ບໍ່ມີເຫດຜົນອະທິບາຍໄດ້ວ່າ ເປັນຫຍັງຈຶ່ງພົບຢູ່ນຳກັນ, ກະຈາຍໄປທົ່ວອັງກິດ ແລະ ຢູໂຣບ, ຝັງຢູ່ໃນຊັ້ນຕະກອນ 'ດີລູວີອູມ' \cite{58}. ສ່ວນປະສົມຂອງຊາກສັດດັ່ງກ່າວຍັງພົບໃນຖ້ຳ ຊົງເຮເລເລນ (Skjonghelleren) ຢູ່ເທິງເກາະ ວາລດຣອຍ (Valdroy), ປະເທດນໍເວ ອີກດ້ວຍ. ໃນຖ້ຳນີ້ ພົບກະດູກສັດລ້ຽງລູກດ້ວຍນ້ຳນົມ, ນົກ ແລະ ປາ 7,000 ທ່ອນ ປະປົນກັນໃນຊັ້ນຕະກອນຫຼາຍຊັ້ນ \cite{59}. ອີກຕົວຢ່າງໜຶ່ງແມ່ນຖ້ຳ ຊານ ຊີໂຣ, "ຖ້ຳແຫ່ງຍັກ", ໃນອີຕາລີ. ໃນຖ້ຳນີ້ ພົບກະດູກສັດລ້ຽງລູກດ້ວຍນ້ຳນົມຫຼາຍໂຕນ, ເຊິ່ງສ່ວນໃຫຍ່ແມ່ນກະດູກຮິບໂປໂປເຕມັສ ຢູ່ໃນສະພາບສົດໃໝ່ຫຼາຍຈົນຕ້ອງຕັດເປັນເຄື່ອງປະດັບ ແລະ ສົ່ງອອກໄປເຮັດຕະກຽງດຳ. ມີລາຍງານວ່າ ກະດູຂອງສັດຕ່າງຊະນິດແມ່ນຖືກປະສົມເຂົ້າໃສ່ກັນ, ຫັກ, ແຫຼກສະຫຼາຍ ແລະ ກະຈັດກະຈາຍເປັນທ່ອນເລັກທ່ອນນ້ອຍ \cite{60,61}. ໃນເມືອງ ແມນເດສ ບູຮານ, ປະເທດອີຢິບ, ພົບກະດູກສັດຫຼາຍຊະນິດປະສົມກັບດິນໜຽວເຜົາ (ຄືແກ້ວ) \cite{57}.  ການຄົ້ນພົບດັ່ງກ່າວອາດຈະໜ້າສົງໄສ ແຕ່ສາມາດອະທິບາຍໄດ້ງ່າຍໆ ໂດຍນ້ຳຖ້ວມຄັ້ງໃຫຍ່ເຮັດໃຫ້ສັດຕາຍທັບຖົມກັນເປັນຊັ້ນຕະກອນ, ທັບສັດ ຫຼື ຝັງພວກມັນທັງເປັນໃນຖ້ຳ ແລະ ໃນກໍລະນີຊີວະມວນທີ່ຜ່ານການເຜົາແລ້ວໃນອີຢິບ, ກະແສໄຟຟ້າຈຳນວນມະຫາສານທີ່ປ່ອຍອອກມາຈາກແກນໂລກທີ່ເຄື່ອນຕົວໄປມາຈາກຊັ້ນເນື້ອໂລກຫຼັງຈາກນ້ຳຖ້ວມ. ຮູບທີ \ref{fig:7} ສະແດງການສຳຜັດກັບ 'ດິນຕົມ'ຊີວະມວນໃນອາລາສກ້າໂດຍທົ່ວໄປ \cite{56}.

\begin{figure}[t]
\begin{center}
% \fbox{\rule{0pt}{2in} \rule{0.9\linewidth}{0pt}}
   \includegraphics[width=1\linewidth]{muck-crop.jpeg}
\end{center}
   \caption{'ດິນຕົມ' ຂອງອາລາສກ້າປະກອບດ້ວຍເສດຊາກຕົ້ນໄມ້, ພືດ ແລະ ສັດທີ່ກະຈັດກະຈາຍຢ່າງສັບສົນໃນຕະກອນນ້ຳແຂງທີ່ແຂງຕົວ \cite{146}.}
\label{fig:7}
\label{fig:onecol}
\end{figure}

\section{ຫຼຸມຫຼົບໄພບູຮານ}

ບັນພະບຸລຸດຂອງພວກເຮົາໄດ້ຖິ້ມໂຄງສ້າງບູຮານທີ່ໄດ້ຮັບການອອກແບບຢ່າງປະນີດໄວ້ຢ່າງຫຼວງຫຼາຍ ເຊິ່ງໄດ້ພົບຊາກສົບຂອງມະນຸດ. ໂດຍທົ່ວໄປແລ້ວ ໂຄງສ້າງເຫຼົ່ານີ້ມັກຈະຖືກຕີຄວາມໝາຍວ່າເປັນສຸສານທີ່ມີຄວາມຊັບຊ້ອນ ແຕ່ເມື່ອພິຈາລະນາໃຫ້ລະອຽດຂຶ້ນກໍຈະພົບວ່າ ທີ່ຈິງແລ້ວອາດຈະແມ່ນຫຼຸມຫຼົບໄພບູຮານ.

\begin{figure}[b]
\begin{center}
% \fbox{\rule{0pt}{2in} \rule{0.9\linewidth}{0pt}}
   \includegraphics[width=1\linewidth]{ww19.jpg}
\end{center}
   \caption{ສຸສານຫີນ ນິວເກຣນຈ໌, ປະເທດໄອແລນ - ເບິ່ງນັກທ່ອງທ່ຽວຢູ່ທາງເຂົ້າ ເພື່ອທຽບຂະໜາດ.}
\label{fig:8}
\label{fig:onecol}
\end{figure}

ຕົວຢ່າງທີ່ດີຫຼາຍຢ່າງໜຶ່ງແມ່ນ ນິສເກຣນຈ໌ (ຮູບທີ \ref{fig:8}), ເຊິ່ງເປັນອະນຸສອນສະຖານຫຼັກໃນກຸ່ມບຣູນາບອນ, ເຊິ່ງເປັນກຸ່ມອາຄານບູຮານທີ່ມີສິ່ງທີ່ເອີ້ນວ່າສຸສານແບບມີທາງຍ່າງ. ສຸສານເຫຼົ່ານີ້ປະກອບດ້ວຍຫ້ອງຝັງສົບໜຶ່ງຫ້ອງຂຶ້ນໄປ ເຊິ່ງປົກຄຸມດ້ວຍດິນ ຫຼື ຫີນ ແລະ ມີທາງເຂົ້າແຄບໆທີ່ເຮັດຈາກຫີນຂະໜາດໃຫຍ່ \cite{70}. ນີ້ແມ່ນຕົວຢ່າງຂອງການອອກແບບໂຄງສ້າງທີ່ຊັບຊ້ອນ ເຊິ່ງໄດ້ຮັບການປົກປ້ອງເຊິ່ງສ້າງຂຶ້ນມາຫຼາຍຊົ່ວອາຍຸຄົນ, ໂດຍສັນນິຖານວ່າ ໃຊ້ຝັງຄົນພຽງບໍ່ພໍເທົ່າໃດຄົນທີ່ຍັງບໍ່ເກີດຊ້ຳ ເມື່ອເລີ່ມສ້າງສຸສານ. ເມື່ອເຈົ້າຂອງດິນໃນທ້ອງຖິ່ນຄົ້ນພົບໃໝ່ໃນປີ 1699 ຂຸມຝັງສົບກໍຖືກຝັງຢູ່ໃນດິນແລ້ວ.

ເມື່ອພິຈາລະນາໂຄງສ້າງຢ່າງຜິວເຜີນຈະເຫັນວ່າ ຕ້ອງໃຊ້ຄວາມພະຍາຍາມຢ່າງຫຼວງຫຼາຍໃນການສ້າງຂຶ້ນມາ - ນິວເກຣນຈ໌ ປະກອບດ້ວຍວັດສະດຸປະມານ 200,000 ໂຕນ. ພາຍໃນ, \textit{“...ມີທາງຍ່າງເປັນຫ້ອງ ເຊິ່ງສາມາດເຂົ້າເຖິງໄດ້ຈາກທາງເຂົ້າທາງດ້ານຕາເວັນອອກສຽງໃຕ້ຂອງອະນຸສອນສະຖານ. ທາງຍ່າງຢຽດຍາວ 19 ແມັດ (60 ຟຸດ) ຫຼື ປະມານໜຶ່ງສ່ວນສາມຂອງໄລຍະທາງຈາກໃຈກາງຂອງໂຄງສ້າງ. ຢູ່ປາຍທາງຍ່າງມີຫ້ອງນ້ອຍສາມຫ້ອງ ຈາກຫ້ອງກາງຂະໜາດໃຫຍ່ທີ່ມີຫຼັງຄາໂຄ້ງສູງ ແບບໂຄ້ງງໍ… ຝາຜະໜັງຂອງທາງຍ່າງນີ້ປະກອບດ້ວຍແຜ່ນຫີນຂະໜາດໃຫຍ່ທີ່ເອີ້ນວ່າ ອໍໂທສແທດ, ເຊິ່ງມີຊາວສອງແຜ່ນຢູ່ທາງດ້ານຕາເວັນຕົກ ແລະ ຊາວເອັດແຜ່ນຢຸ່ທາງດ້ານຕາເວັນອອກ. ແຜ່ນຫີນເຫຼົ່ານີ້ມີຄວາມສູງສະເລ່ຍ 1½ ແມັດ”} \cite{70}. ນອກຈາກນີ້ຍັງມີລາຍລະອຽດທາງວິສະວະກຳການກັນຊຶມທີ່ຊັບຊ້ອນອີກດ້ວຍ. ຕົວຢ່າງ ເທິງຫຼັງຄາ, \textit{“ຊ່ອງວ່າງລະຫວ່າງຫຼັງຄາແມ່ນຖືກອັດດ້ວຍສ່ວນປະສົມຂອງດິນເຜົາ ແລະ ດິນຊາຍທະເລ ເພື່ອໃຫ້ກັນນ້ຳໄດ້ ແລະ ຈາກສ່ວນປະສົມນີ້ເອງ ພວກເຮົາຈຶ່ງສາມາດຊອກຫາອາຍຸຂອງກາກບອນກຳມັນຕະລັງສີໄດ້ສອງຊ່ວງ ເຊິ່ງມີສູນກາງຢູ່ທີ່ 2500 ປີກ່ອນ ຄ.ສ. ສຳລັບໂຄງສ້າງຂອງສຸສານ"} \cite{71}. ນອກຈາກນີ້ ອາດຈະມີການສ້າງທາງຍົກລະດັບ ເພື່ອນຳໄປສູ່ຫ້ອງດ້ານໃນ ເພື່ອຈຸດປະສົງທີ່ຄ້າຍຄືກັນ: \textit{“ເນື່ອງຈາກພື້ນຂອງທາງຍ່າງ ແລະ ຫ້ອງຂອງສຸສານລຽງຕາມເນີນດິນທີ່ສ້າງອະນຸສອນສະຖານຂຶ້ນ ຈຶ່ງມີລະດັບພື້ນຕ່າງກັນເກືອບ 2 ແມັດ ລະຫວ່າງທາງເຂົ້າ ແລະ ພາຍໃນຫ້ອງ”} \cite{71}.

\begin{figure}[b]
\begin{center}
% \fbox{\rule{0pt}{2in} \rule{0.9\linewidth}{0pt}}
   \includegraphics[width=1\linewidth]{dolmen.jpg}
\end{center}
   \caption{ອະນຸສອນສະຖານໂດເມນເດີໂຊໂຕ, ປະເທດສະເປນ \cite{53}.}
\label{fig:9}
\label{fig:onecol}
\end{figure}

ການທີ່ບໍ່ມີຊາກສົບມະນຸດຢູ່ພາຍໃນກໍຖືເປັນຈຸດທີ່ໜ້າສົນໃຈເຊັ່ນກັນ. ການຂຸດຄົ້ນພົບຊິ້ນສ່ວນກະດູກທີ່ຖືກເຜົາ ແລະ ບໍ່ໄດ້ເຜົາ ເຊິ່ງເປັນຕົວແທນຂອງຄົນຈຳນວນໜຶ່ງທີ່ກະຈັດກະຈາຍຢູ່ຕາມທາງຍ່າງ. ການກໍ່ສ້າງສຸສານຫີນນິວເກຣນຈ໌ ຄາດຄະເນວ່າໃຊ້ເວລາຢ່າງໜ້ອຍຫຼາຍຊົ່ວອາຍຸຄົນ ໂດຍພິຈາລະນາຈາກອາຍຸກາກບອນຂອງວັດສະດຸພາຍໃນ. ເປັນຫຍັງຊຸມຊົນບູຮານຈຶ່ງໃຊ້ຄວາມພະຍາຍາມຫຼາຍໃນການສ້າງສຸສານຂະໜາດໃຫຍ່ທີ່ມີການອອກແບບທາງວິສະວະກຳຢ່າງພິຖີພິຖັນ ພຽງເພື່ອກະຈາຍຊິ້ນສ່ວນກະດູກຂອງຜູ້ເສຍຊີວິດພຽງບໍ່ພໍເທົ່າໃດຄົນໃນທາງຍ່າງ? ມັນເປັນໄປໄດ້ຫຼາຍວ່າ ໂຄງສ້າງຫີນຂະໜາດໃຫຍ່ບູຮານທີ່ກັນນ້ຳຢ່າງລະມັດລະວັງເຫຼົ່ານີ້ຖືກສ້າງຂຶ້ນເພື່ອເປັນບ່ອນພັກອາໄສຂອງມະນຸດ ເພື່ອປົກປ້ອງຄົນໃນຊ່ວງທີ່ເກີດໄພພິບັດຊ້ຳແລ້ວຊ້ຳອີກໃນໂລກ.

ໃນເມືອງຮູເອວວາ, ທາງຕອນໃຕ້ຂອງປະເທດສະເປນ, ຕົວຢ່າງທີ່ຄ້າຍກັນແມ່ນອະນຸສອນສະຖານ ໂດເມນເດີໂຊໂຕ (ຮູບທີ \ref{fig:9}), ເຊິ່ງເປັນໜຶ່ງໃນປະມານ 200 ແຫ່ງໃນພື້ນທີ່ \cite{72,32}. ໂຄງສ້າງນີ້ໄດ້ຮັບການອອກແບບທາງວິສະວະກຳຢ່າງພິຖີພິຖັນໂດຍໃຊ້ກ້ອນຫີນຂະໜາດໃຫຍ່ ແລະ ມີເສັ້ນຜ່າໃຈກາງ 75 ແມັດ. ມີລາຍງານວ່າ ພົບເຫັນສົບພຽງແຕ່ແປດສົບໃນລະຫວ່າງການຂຸດຄົ້ນ, ໂດຍທັງໝົດຖືກຝັງໃນທ່າຄືແອນ້ອຍໃນທ້ອງແມ່.

\section{ການກ່າວເຖິງຄວາມຜິດປົກກະຕິທີ່ສຳຄັນ}

ໃນພາກສ່ວນນີ້, ຂ້າພະເຈົ້າຈະກ່າວເຖິງຄວາມຜິດປົກກະຕິທີ່ສຳຄັນອື່ນໆໂດຍສັງເຂບ, ເຊິ່ງທັງໝົດໄດ້ຮັບການອະທິບາຍຢ່າງດີດ້ວຍໄພພິບັດແບບ ECDO.

\subsection{ຄວາມຜິດປົກກະຕິທາງຊີວະພາບ}

\begin{figure}[t]
\begin{center}
% \fbox{\rule{0pt}{2in} \rule{0.9\linewidth}{0pt}}
   \includegraphics[width=1\linewidth]{bottleneck.jpg}
\end{center}
   \caption{ຄໍຂວດທາງພັນທຸກຳ ທີ່ສະແດງເຖິງການຄັດເລືອກເພດຊາຍ 95\% ເມື່ອປະມານ 6,000 ປີກ່ອນ \cite{62}.}
\label{fig:10}
\label{fig:onecol}
\end{figure}

ຄວາມຜິດປົກກະຕິທາງຊີວະສາດທີ່ເຫັນໄດ້ຢ່າງຊັດເຈນບາງປະການ ໄດ້ແກ່ ຄໍຂວດທາງພັນທຸກຳ ແລະ ຊາກດຶກດຳບັນວານນ້ຳຈືດ. Zeng et al (2018) ໄດ້ສ້າງແບບຈຳລອງລຳດັບໂຄຣໂມໂຊມ Y ຈຳນວນ 125 ລຳດັບຈາກມະນຸດຍຸກປະຈຸບັນ ແລະ ຈາກຄວາມຄ້າຍຄືກັນ ແລະ ການກາຍພັນໃນ DNA, ລະບຸຄໍຂວດການຫຼຸດຈຳນວນປະຊາກອນລົງ 95\% ໃນປະຊາກອນເພດຊາຍ ເມື່ອປະມານ 5,000 ເຖິງ 7,000 ປີກ່ອນ (ຮູບທີ \ref{fig:10}) \cite{62}. ຊາກດຶກດຳບັນວານຖືກຄົ້ນພົບຫຼາຍຮ້ອຍແມັດເໜືອລະດັບນ້ຳທະເລໃນ ສະວີເດັນເບີກ, ລັດມີຊິແກນ, ລັດເວີມອນ, ແຄນນາດາ, ຊີລີ ແລະ ອີຢິບ \cite{63,64,65,66}. ວານເຫຼົ່ານີ້ຖືກພົບໃນສະພາບຕ່າງໆ: ຢູ່ໃນສະພາບສົມບູນ ໃນໜອງບຶງທີ່ຢູ່ເໜືອຕະກອນທານນ້ຳແຂງ ຫຼື ຝັງຢູ່ໃນຕະກອນ. ຈຳນວນຂອງຕົວຢ່າງໃນແຫຼ່ງເຫຼົ່ານີ້ມີຕັ້ງແຕ່ບໍ່ພໍເທົ່າໃດໂຕ ໄປຈົນເຖິງຮ້ອຍກວ່າໂຕ. ວານແມ່ນສັດທະເລນ້ຳເລິກ ແລະ ບໍ່ຄ່ອຍກ້າສ່ຽງເຂົ້າໃກ້ຊາຍຝັ່ງ. ວານເຫຼົ່ານີ້ໄປຢູ່ໃນລະດັບທີ່ສູງນັ້ນໄດ້ແນວໃດ ເຊິ່ງມັກຈະຢູ່ຫ່າງໄກຈາກແຜ່ນດິນຫຼາຍ?

ໃນອະດີດ ໂລກເຄີຍເກີດການສູນພັນຄັ້ງໃຫຍ່ຫຼາຍຄັ້ງ ເຊິ່ງເຫດການທີ່ໄດ້ຮັບການສຶກສາຢ່າງລະອຽດທີ່ສຸດແມ່ນເຫດການ "ຫ້າເຫດການໃຫຍ່"  ຂອງ ຟາເນີໂຣໂຊອີກ ໄດ້ແກ່ ການສູນພັນຄັ້ງໃຫຍ່ໃນຍຸກອໍໂດວິຊຽນຕອນປາຍ (LOME), ຍຸກດີໂວນຽນຕອນປາຍ (LDME), ຍຸກເພີມຽນຕອນປາຍ (EPME), ຍຸກໄທຣແອສຊິກຕອນປາຍ (ETME) ແລະ ຍຸກຄຣີເທຊຽສຕອນປາຍ (ECME) \cite{88,89}. ທີ່ໜ້າສົນໃຈແມ່ນ ການສູນພັນຫຼາຍຄັ້ງເຫຼົ່ານີ້ຈັດຢູ່ໃນປະເພດດຽວກັນທີ່ເກີດຂຶ້ນໃນໄລຍະເວລາທາງປະຫວັດສາດດຽວກັນກັບຊັ້ນຕ່າງໆຂອງແກຣນແຄນຢອນ ນັ້ນແມ່ນຊັ້ນ ເພີມຽນ ແລະ ດີໂວນຽນ.

\subsection{ຄວາມຜິດປົກກະຕິທາງກາຍະພາບ}

\begin{figure}[t]
\begin{center}
% \fbox{\rule{0pt}{2in} \rule{0.9\linewidth}{0pt}}
   \includegraphics[width=1\linewidth]{columbia.jpg}
\end{center}
   \caption{ກະແສນ້ຳຮອບໃຫຍ່ໃນທະເລສາບນ້ຳແຂງໂຄລຳເບຍ, ລັດວໍຊິງຕັນ \cite{80}.}
\label{fig:11}
\label{fig:onecol}
\end{figure}

ມີພູມສັນຖານອື່ນໆອີກຫຼາຍແຫ່ງທີ່ໜ້າຈະກໍ່ໂຕຂຶ້ນຈາກພະລັງແຫ່ງຫາຍະນະ. ຫຼັກຖານຂອງການໄຫຼຂອງນ້ຳໃນທະວີບຂະໜາດໃຫຍ່ສາມາດພົບເຫັນໄດ້ໃນກະແສນ້ຳທີ່ໄຫຼແຮງທົ່ວໂລກ. ຕົວຢ່າງໜຶ່ງແມ່ນ ຊ່ອງສະຄາບແລນ ໃນປາຊິຟິກຕາເວັນຕົກສຽງເໜືອ. ຢູ່ທີ່ນີ້ ພວກເຮົາບໍ່ພຽງແຕ່ເຫັນພູມສັນຖານຕະກອນທັບຖົມ ແລະ ຫີນກ້ອນໃຫຍ່ທີ່ເຄື່ອນໂຕບໍ່ແນ່ນອນເທົ່ານັ້ນ, ແຕ່ຍັງເຫັນລຳດັບຂອງຮອບນ້ຳຂະໜາດໃຫຍ່ຮ້ອຍປາຍລຳດັບທີ່ເກີດຈາກກະແສນ້ຳຂະໜາດໃຫຍ່ \cite{78,79} ອີກດ້ວຍ. ຮອບນ້ຳເຫຼົ່ານີ້ແມ່ນຮອບນ້ຳຂະໜາດໃຫຍ່ໃນຊັ້ນຊາຍຂອງລຳທານ. ສາມາດພົບເຫັນໄດ້ທົ່ວໂລກ ໃນ ຝຣັ່ງ, ອາເຈນຕິນາ, ຣັດເຊຍ ແລະ ອາເມຣິກາເໜືອ \cite{81}. ຮູບທີ \ref{fig:11} ສະແດງຮອບນ້ຳບາງສ່ວນໃນລັດວໍຊິງຕັນ ສະຫະລັດອາເມຣິກາ \cite{80}.

\begin{figure}[t]
\begin{center}
% \fbox{\rule{0pt}{2in} \rule{0.9\linewidth}{0pt}}
   \includegraphics[width=1\linewidth]{zhangjiajie.jpg}
\end{center}
   \caption{ເສົາຫີນຂະໜາດໃຫຍ່ໃນປ່າສະຫງວນແຫ່ງຊາດ ຈາງເຈຍເຈ້ຍ ທາງຕອນໃຕ້ຂອງຈີນ.}
\label{fig:12}
\label{fig:onecol}
\end{figure}

\begin{figure}[t]
\begin{center}
% \fbox{\rule{0pt}{2in} \rule{0.9\linewidth}{0pt}}
   \includegraphics[width=1\linewidth]{hoy.jpg}
\end{center}
   \caption{ເສົາຫີນ ໂອນແມນອັອຟຮອຍ, ສະກັອດແລນ \cite{83}.}
\label{fig:13}
\label{fig:onecol}
\end{figure}

ໂຄງສ້າງການກັດເຊາະພາຍໃນແຜ່ນດິນນັ້ນອະທິບາຍໄດ້ດີເຊັ່ນກັນ ໂດຍ ການພິກໂລກແບບ ECDO. ປະເທດຈີນຕອນໃຕ້ແມ່ນຕົວຢ່າງທີ່ດີຂອງພູມສັນຖານຫີນປູນຂະໜາດໃຫຍ່ທີ່ເກີດຈາກການກັດເຊາະຂອງນ້ຳ \cite{82}. ພູມສັນຖານເຫຼົ່ານີ້ ໄດ້ແກ່ ຫີນປູນຮູບຫໍຄອຍ, ຫີນປູນຮູບຍອດແຫຼມ, ຫີນປູນຮູບຈວຍ, ຂົວທຳມະຊາດ, ຮ່ອມພູ, ລະບົບຖ້ຳຂະໜາດໃຫຍ່ ແລະ ຂຸມຍຸບ. ໜຶ່ງໃນພູມສັນຖານທີ່ສະດຸດຕາທີ່ສຸດແມ່ນປ່າສະຫງວນແຫ່ງຊາດ ຈາງເຈຍເຈ້ຍ, ເຊິ່ງມີເສົາຫີນຊາຍຄວອດຂະໜາດໃຫຍ່ (ຮູບທີ \ref{fig:12}) \cite{84}. ເສົາເຫຼົ່ານີ້ຕັ້ງຢູ່ເທິງລະດັບຄວາມສູງສະເລ່ຍກວ່າ 1,000 ແມັດ ແລະ ມີຈຳນວນຫຼາຍກວ່າ 3,100 ຕົ້ນ. ເຊິ່ງຫຼາຍກວ່າ 1,000 ຕົ້ນມີຄວາມສູງກວ່າ 120 ແມັດ ແລະ 45 ຕົ້ນມີຄວາມສູງກວ່າ 300 \cite{85}. ເສົາເຫຼົ່ານີ້ມີລັກສະນະຄ້າຍຄືເສົາຫີນຖືກນ້ຳທະເລກັດເຊາະ (ຮູບທີ \ref{fig:13}), ເຊິ່ງເປັນເສົາຫີນຊາຍຝັ່ງທີ່ເກີດຈາກການພັງທະລາຍຂອງວັດສະດຸໂດຍຮອບ ອັນເນື່ອງມາຈາກຄື້ນທະເລ. ພູມສັນຖານທີ່ເກີດການກັດເຊາະທີ່ຄ້າຍຄືກັນນີ້ ສາມາດພົບເຫັນໄດ້ໃນຈວຍຫີນຂອງເມືອງ ອຸກູບ, ປະເທດຕຸລະກີ, ລວມເຖິງ ຄິວແດດ ເອນກັນຕາດາ, ປະເທດສະເປນ, ເຊິ່ງທັງສອງແຫ່ງແມ່ນຢູ່ເໜືອລະດັບນ້ຳທະເລຫຼາຍກວ່າ 1,000 ແມັດ. ສະຖານທີ່ເຫຼົ່ານີ້ ທັງໝົດມີຊາກດຶກດຳບັນຂອງທະເລ ແລະ ເກືອໃນບໍລິເວນໃກ້ຄຽງກັນ ເຊິ່ງຊີ້ໃຫ້ເຫັນເຖິງການບຸກລຸກຂອງທະເລໃນອະດີດ \cite{15,86,87}. ແນ່ນອນວ່າ ເລື່ອງລາວກ່ຽວກັບນ້ຳຖ້ວມ \cite{3} ໄດ້ກ່າວເຖິງມະຫາສະໝຸດທີ່ສູງກວ່າ 1,000 ແມັດຫຼາຍ ແລະ ສິ່ງນີ້ກໍໄດ້ຮັບການພິສູດ ໂດຍການປາກົດໂຕຂອງນ້ຳເຄັມ ແລະ ແອ່ງເກືອຂະໜາດໃຫຍ່ໃນເທືອກເຂົາ Andes ແລະ ເທືອກເຂົາຫິມະໄລ ເຊິ່ງຢູ່ເໜືອລະດັບນ້ຳທະເລຫຼາຍກິໂລແມັດ. ຕົວຢ່າງເຊັ່ນ ແອ່ງເກືອ ອູຢູນີ ໃນປະເທດໂບລີເວຍ ມີຄວາມສູງເໜືອລະດັບນ້ຳທະເລ 3653 ແມັດ \cite{94}.

\subsection{ເຫດການການປ່ຽນແປງສະພາບພູມອາກາດຢ່າງວ່ອງໄວ}

ວັນນະກຳທາງວິທະຍາສາດສະໄໝໃໝ່ຍອມຮັບການມີຢູ່ຂອງເຫດການການປ່ຽນແປງສະພາບອາກາດສະພາບພູມອາກາດໂລກຢ່າງວ່ອງໄວ ໃນປະຫວັດສາດໂລກເມື່ອບໍ່ດົນມານີ້. ຕົວຢ່າງທີ່ໜ້າສັງເກດສອງກໍລະນີແມ່ນເຫດການ 4.2 ກິໂລປີ ແລະ 8.2 ກິໂລປີ ເຊິ່ງທັງສອງກໍລະນີເກີດຂຶ້ນພ້ອມກັນກັບການຫຼຸດລົງຂອງຈຳນວນປະຊາກອນ ແລະ ການຢຸດຊະງັກຂອງການຕັ້ງຖິ່ນຖານໃນສັງຄົມໃນພື້ນທີ່ທາງພູມສັນຖານຂະໜາດໃຫຍ່. ເຫດການເຫຼົ່ານີ້ໄດ້ຮັບການເກັບຮັກສາໄວ້ເປັນຄວາມຜິດປົກກະຕິໃນຕະກອນ ແລະ ແກນນ້ຳກ້ອນ, ຊາກດຶກດຳບັນປະກາລັງ, ຄ່າໄອໂຊໂທບ O18, ບົດບັນທຶກລະອອງເລນູ ແລະ ຫີນງອກ, ແລະ ຂໍ້ມູນລະດັບນ້ຳທະເລ. ການປ່ຽນແປງສະພາບພູມອາກາດທີ່ສົມມຸດຂຶ້ນໄດ້ ລວມເຖິງອຸນຫະພູມໂລກທີ່ຫຼຸດລົງຢ່າງວ່ອງໄວ, ພາວະແຫ້ງແລ້ງ, ການຢຸດຊະງັກຂອງກະແສນ້ຳໄປກັບໃນມະຫາສະໝຸດແອດແລນຕິກ ແລະ ການເຄື່ອນໂຕຂອງທານນ້ຳແຂງ \cite{90,91,92}. ໂດຍສະເພາະຢ່າງຍິ່ງ ເຫດການ 8.2 ກິໂລປີນັ້ນ ເກີດຂຶ້ນພ້ອມກັນກັບນ້ຳຖ້ວມທະເລດຳຄັ້ງໃຫຍ່ ທີ່ອາດຈະເກີດຂຶ້ນໄດ້ເມື່ອປະມານ 6400 ປີກ່ອນ ຄ.ສ. \cite{93}.

\subsection{ຄວາມຜິດປົກກະຕິທາງບູຮານຄະດີ}

ຫຼັກຖານທາງບູຮານຄະດີຂອງເມືອງບູຮານບາງແຫ່ງແມ່ນສະແດງໃຫ້ເຫັນຊັ້ນຕ່າງໆຫຼາຍຊັ້ນທີ່ກ່ຽວຂ້ອງກັບການຝັງສົບ ແລະ ການທຳລາຍລ້າງ ເຊິ່ງສ້າງບົດບັນທຶກເຫດການຫາຍະນະໃນອະດີດ. ເມືອງບູຮານ Jericho ເປັນເມືອງໜຶ່ງທີ່ຕັ້ງຢູ່ໃນ ປາເລສໄຕ ໃນປະຈຸບັນ. ເມືອງນີ້ມີຊັ້ນຫີນຫຼາຍຊັ້ນທີ່ພັງທະລາຍລົງມາ ແລະ ໄຟໄໝ້ຮຸນແຮງ \cite{96,97}. ລຳດັບເຫດການທີ່ບັນທຶກໄວ້ໃນຊັ້ນຫີນເຫຼົ່ານີ້ມີອາຍຸຕັ້ງແຕ່ປະມານ 9000 ປີກ່ອນ ຄ.ສ. ເຖິງ 2000 ປີກ່ອນ ຄ.ສ. ສິ່ງທີ່ໜ້າສົນໃຈເປັນພິເສດແມ່ນຫໍຄອຍ, ເຊິ່ງເບິ່ງຄືຈະຖືກຕັດອອກ ແລະ ຝັງຢູ່ໃນຕະກອນ ເມື່ອປະມານ 7400 ປີກ່ອນ ຄ.ສ. (ຮຸບທີ \ref{fig:14}) \cite{95}. ຄາທັລ ຮູຢຸກ \cite{99}, ກຣາມມາໂລດ \cite{98}, ແລະ ພະລາຊະວັງ ມິໂນອັນ ແຫ່ງ ນອສຊອສ ຢູ່ເທິງເກາະຄຣີດ \cite{100,101} ລ້ວນແມ່ນຕົວຢ່າງທີ່ຄ້າຍຄືກັນຂອງແຫຼ່ງບູຮານຄະດີທີ່ມີຊັ້ນຫີນຫຼາຍຊັ້ນ, ເຊິ່ງມັກຈະມີຫຼັກຖານຂອງການທຳລາຍລ້າງຢູ່ນຳ.

\begin{figure}[t]
\begin{center}
% \fbox{\rule{0pt}{2in} \rule{0.9\linewidth}{0pt}}
   \includegraphics[width=1\linewidth]{jericho.jpg}
\end{center}
   \caption{ການບູລະນະທາງບູຮານຄະດີຂອງການຝັງສົບໃນຫໍຄອຍແຫ່ງເມືອງເຈຣິໂຄ ປະມານ 7400 ປີກ່ອນ ຄ.ສ. \cite{95}.}
\label{fig:14}
\label{fig:onecol}
\end{figure}

ຫຼັກຖານອີກຢ່າງໜຶ່ງທີ່ບົ່ງຊີ້ເຖິງໄພພິບັດຄັ້ງໃຫຍ່ທີ່ທຳລາຍອາລະຍະທຳຂອງມະນຸດແມ່ນ ນາມປາ ອິເມຈ, ຕຸກກະຕາດິນເຜົາທີ່ພົບເຫັນໃຕ້ລາວາເລິກປະມານ 100 ແມັດໃນລັດໄອດາໂຮ \cite{102,103}. ລາວາທີ່ໄຫຼຜ່ານຮູບປັ້ນນີ້ຄາດຄະເນວ່າ ຖືກທັບຖົມລົງໃນຊ່ວງປາຍຍຸກເທີເຊຍລີ ຫຼື ຕົ້ນຍຸກຄວາເທີນາຣີ, ເຊິ່ງຄາດວ່າມີອາຍຸປະມານ 2 ລ້ານປີ. ຢ່າງໃດກໍຕາມ, ລາວາໄຫຼໃນພາກພື້ນນີ້ເບິ່ງຄືວ່າຈະຍັງຂ້ອນຂ້າງໃໝ່. ການຄົ້ນພົບດັ່ງກ່າວບໍ່ພຽງແຕ່ຊີ້ໃຫ້ເຫັນເຖິງໄພພິບັດຄັ້ງໃຫຍ່ທີ່ທຳລາຍອາລະຍະທຳເທົ່ານັ້ນ ແຕ່ຍັງຕັ້ງຄຳຖາມເຖິງລຳດັບເວລາຂອງການຊອກຫາອາຍຸສະໄໝໃໝ່ອີກດ້ວຍ.

\section{ກ່ຽວກັບວິທີການຊອກຫາອາຍຸສະໄໝໃໝ່}

ມີເຫດຜົນສຳຄັນທີ່ຈະຕ້ອງສົງໄສລຳດັບເວລາສະໄໝໃໝ່, ເຊິ່ງກຳນົດອາຍຸຂອງວັດສະດຸທາງກາຍະພາບຕ່າງໆໄວ້ຍາວນານຫຼາຍ ເຖິງຫຼາຍລ້ານປີ ຫຼື ອາດຈະເຖິງຫຼາຍຮ້ອຍລ້ານປີ.

ເລື່ອງເລົ່າທົ່ວໄປລະບຸວ່າ "ເຊື້ອເພີງຊາກດຶກດຳບັນ" ເຊັ່ນ: ຖ່ານຫີນ, ນ້ຳມັນ ແລະ ກາສທຳມະຊາດແມ່ນມີອາຍຸຫຼາຍຮ້ອຍລ້ານປີ \cite{104}. ຢ່າງໃດກໍຕາມ, ການຊອກຫາອາຍຸດ້ວຍກາກບອນຂອງນ້ຳມັນໃນອ່າວເມັກຊີໂກ ເຫັນວ່າອາຍຸຂອງນ້ຳມັນຢູ່ທີ່ປະມານ 13,000 ປີ \cite{105}. ກາກບອນ-14 ມີເຄິ່ງຊີວິດທີ່ສັ້ນຫຼາຍ (5,730 ປີ) ຈຶ່ງຄາດຄະເນວ່າ ມັນຈະສະຫຼາຍຕົວໝົດພາຍໃນເວລາບໍ່ພໍເທົ່າໃດແສນປີ. ຢ່າງໃດກໍຕາມ, ມີການຄົ້ນພົບຖ່ານຫີນ ແລະ ຊາກດຶກດຳບັນທີ່ຄາດຄະເນວ່າມີອາຍຸເກົ່າແກ່ກວ່າພັນເທື່ອ \cite{106}. ໃນຄວາມເປັນຈິງ, ຖ່ານຫີນທຽມໄດ້ຖືກຜະລິດຂຶ້ນໃນຫ້ອງປະຕິບັດການພາຍໃຕ້ສະພາວະຄວບຄຸມ, ໂດຍສະເພາະຄວາມຮ້ອນສູງ, ໃນເວລາພຽງ 2-8 ເດືອນ \cite{107}.

ວິທີການຊອກຫາອາຍຸດ້ວຍໄອໂຊໂທບລັງສີອື່ນໆ ນອກເໜືອຈາກການຊອກຫາອາຍຸດ້ວຍກາກບອນກໍອາດຈະບໍ່ແມ້ນຢຳເຊັ່ນກັນ. ກຸ່ມນັກຄົ້ນຄວ້າ Answers in Genesis ຄົ້ນພົບວ່າ ວັນທີທີ່ໄດ້ຈາກວິທີການດັ່ງກ່າວບໍ່ສອດຄ່ອງກັນ ເຊິ່ງເຮັດໃຫ້ເກີດຄຳຖາມເຖິງຄວາມຖືກຕ້ອງຂອງວິທີການດັ່ງກ່າວ \cite{108}. ເນື້ອເຍື່ອອ່ອນທີ່ມີເຊວເມັດເລືອດ, ຫຼອດເລືອດ ແລະ ຄໍລາເຈນຍັງພົບເຫັນໃນຊາກໄດໂນເສົາ ເຊິ່ງຄາດຄະເນວ່າມີອາຍຸຫຼາຍກວ່າຮ້ອຍລ້ານປີ \cite{109,110}. ຈາກສິ່ງທີ່ພວກເຮົາຮູ້ ເປັນໄປໄດ້ວ່າອາຍຸທີ່ຍອມຮັບກັນໂດຍທົ່ວໄປຂອງໄລຍະເວລາທາງທໍລະນີສາດ ແລະ ວັດສະດຸທາງກາຍະພາບ ເຊັ່ນ ຫີນ ແລະ ເຊື້ອເພີງຊາກດຶກດຳບັນຂອງໂລກອາດຈະບໍ່ຖືກຕ້ອງຫຼາຍເທົ່າ.

\section{ບົດສະຫຼຸບ}

ໃນບົດຄວາມນີ້ ຂ້າພະເຈົ້າໄດ້ກ່າວເຖິງຄວາມຜິດປົກກະຕິທີ່ໜ້າສົນໃຈທີ່ສຸດ ເຊິ່ງບົ່ງຊີ້ເຖິງຕົ້ນກຳເນີດຂອງຫາຍະນະ ແລະ ອະທິບາຍໄດ້ດີທີ່ສຸດໂດຍການພິກໂລກຂອງ ECDO. ເຖິງວ່າຈະມີຄວາມຫຼາກຫຼາຍ ແຕ່ຂໍ້ມູນເກັບກຳທີ່ນຳສະເໜີນັ້ນຍັງບໍ່ສົມບູນ - ຄວາມຜິດປົກກະຕິເພີ່ມຕື່ມນັ້ນແມ່ນໄດ້ຮັບການເກັບກຳ ແລະ ເຜີຍແຜ່ສູ່ສາທາລະນະໃນຄັງຂໍ້ມູນຄົ້ນຄວ້າ GitHub ຂອງຂ້າພະເຈົ້າ \cite{2}.

\section{ຄຳຂອບໃຈ}

ຂໍຂອບໃຈ Ethical Skeptic, ຜູ້ຂຽນຕົ້ນສະບັບຂອງວິທະຍານິພົນ ECDO, ສຳລັບວິທະຍານິພົນທີ່ເລິກເຊິ່ງ, ລ້ຳສະໄໝຂອງລາວທີ່ສຳເລັດສົມບູນ ແລະ ແບ່ງປັນໃຫ້ໂລກໄດ້ຮັບຮູ້. ວິທະຍານິພົນສາມສ່ວນຂອງລາວ \cite{1} ຍັງຄົງເປັນຜົນງານທີ່ໜ້າເຊື່ອຖື ສຳລັບທິດສະດີ ການແຍກຕົວຂອງແກນໂລກເພື່ອປົດປ່ອຍຄວາມຮ້ອນດ້ວຍການສັ່ນສະເທືອນ (ECDO) ແລະ ຍັງມີຂໍ້ມູນກ່ຽວກັບຫົວຂໍ້ນີ້ ຫຼາຍກວ່າທີ່ຂ້າພະເຈົ້າໄດ້ສະຫຼຸບສັ້ນໆໄວ້ໃນທີ່ນີ້ອີກດ້ວຍ.

ແລະແນ່ນອນ, ຂໍຂອບໃຈຍັກໃຫຍ່ທີ່ໃຫ້ຄວາມຊ່ວຍເຫຼືອພວກເຮົາ ຜູ້ທີ່ໄດ້ເຮັດການຄົ້ນຄວ້າ ແລະ ການສືບສວນທັງໝົດ ທີ່ເຮັດໃຫ້ການເຮັດວຽກນີ້ເປັນໄປໄດ້ ແລະ ເຮັດວຽກເພື່ອນຳເອົາແສງສະຫວ່າງມາສູ່ມວນມະນຸດ.

\clearpage
\twocolumn

{\small
\bibliographystyle{ieee}
\bibliography{egbib}
}

\end{document}

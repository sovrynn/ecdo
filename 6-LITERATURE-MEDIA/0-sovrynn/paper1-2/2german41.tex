\documentclass[10pt,twocolumn,letterpaper]{article}

% Meine eigenen Sachen
\usepackage{booktabs}
% \usepackage{caption}
% \captionsetup[table]{skip=8pt}   % Betrifft nur Tabellen
\usepackage{stfloats}  % Fügen Sie dies dem Vorspann hinzu
\usepackage[T1]{fontenc}
\usepackage[utf8]{inputenc}  % Ensure UTF-8 encoding

\usepackage{cvpr}
\usepackage{times}
\usepackage{epsfig}
\usepackage{graphicx}
\usepackage{amsmath}
\usepackage{amssymb}

% Weitere Pakete hier einfügen, vor hyperref.

% Wenn Sie hyperref auskommentieren und dann wieder einkommentieren, sollten Sie
% egpaper.aux löschen, bevor Sie latex erneut ausführen. (Oder drücken Sie einfach 'q' beim ersten latex-
% Durchlauf, lassen Sie ihn fertig laufen, und alles sollte in Ordnung sein.)

\usepackage[breaklinks=true,bookmarks=false]{hyperref}

\cvprfinalcopy % *** Uncomment this line for the final submission

\def\cvprPaperID{****} % *** Enter the CVPR Paper ID here
\def\httilde{\mbox{\tt\raisebox{-.5ex}{\symbol{126}}}}

% Seiten sind im Einreichungsmodus nummeriert und in der druckfertigen Version unnummeriert
%\ifcvprfinal\pagestyle{empty}\fi
\setcounter{page}{1}
\begin{document}

%%%%%%%%% TITLE
\title{ECDO Datengetriebener Leitfaden Teil 2/2: Eine Untersuchung wissenschaftlicher und historischer Anomalien, die am besten durch einen ECDO "Erdumschwung" erklärt werden können}

\author{Junho\\
Veröffentlicht im Februar 2025\\
Webseite (Hier können die Arbeiten heruntergeladen werden): \href{https://sovrynn.github.io}{sovrynn.github.io}\\
ECDO Forschungs-Repo: \href{https://github.com/sovrynn/ecdo}{github.com/sovrynn/ecdo}\\
{\tt\small junhobtc@proton.me}
% Für eine Arbeit, deren Autoren alle an derselben Institution sind,
% lassen Sie die folgenden Zeilen bis zum abschließenden ``}'' weg.
% Weitere Autoren und Adressen können mit ``\and'' hinzugefügt werden,
% genau wie beim zweiten Autor.
% Um Platz zu sparen, verwenden Sie entweder die E-Mail-Adresse oder die Homepage, nicht beides
% \and
% xx
% Institution2\\
% Erste Zeile der Adresse der Institution2\\
% {\tt\small secondauthor@i2.org}
}

\maketitle
%\thispagestyle{empty}

%%%%%%%%% ABSTRACT
\begin{abstract}
Im Mai 2024 veröffentlichte ein pseudonymer Online-Autor mit dem Namen „The Ethical Skeptic“ \cite{0} eine bahnbrechende Theorie namens Exothermic Core-Mantle Decoupling Dzhanibekov Oscillation (ECDO) \cite{1}. Diese Theorie schlägt nicht nur vor, dass die Erde in der Vergangenheit plötzliche katastrophale Verschiebungen ihrer Rotationsachse durchgemacht hat, wodurch es durch die Trägheit der Rotation zu einer massiven weltweiten Flut kam, weil die Ozeane über die Kontinente schwappten, sondern sie schlägt auch einen erklärenden, ursächlichen geophysikalischen Prozess vor und präsentiert Daten, die darauf hindeuten, dass eine weitere solche Umkehr unmittelbar bevorstehen könnte. Obwohl solche katastrophalen Flut- und Weltuntergangsvorhersagen nicht neu sind, ist die ECDO-Theorie aufgrund ihres wissenschaftlichen, modernen, multidisziplinären und datengestützten Ansatzes einzigartig überzeugend.

Dieses Forschungspapier bildet den zweiten Teil einer zweiteiligen komprimierten Zusammenfassung von 6 Monaten unabhängiger Forschung \cite{2,20} zur ECDO-Theorie, wobei der Fokus speziell auf den wissenschaftlichen und historischen Anomalien liegt, die am besten durch einen katastrophalen „Earth flip“ gemäß ECDO erklärt werden können.
\end{abstract}

%%%%%%%%% BODY TEXT

\section{Einleitung}

Die moderne uniformitaristische Geologie und Geschichtsschreibung behauptet, dass große geologische Landschaften wie der Grand Canyon über Millionen von Jahren entstanden sind \cite{143}; dass Salz im Death Valley (Kalifornien) existiert, weil es vor Hunderten von Millionen Jahren unter dem Ozean lag \cite{144}; dass unsere Vorfahren vor 150 Generationen ihr ganzes Leben damit verbrachten, gigantische Gräber zu errichten \cite{29,70}; und dass sogenannte „fossile Brennstoffe“ Hunderte von Millionen Jahren alt sind \cite{104}. Am faszinierendsten ist vielleicht, dass der Mensch angeblich 300.000 Jahre alt sein soll \cite{145}, aufgezeichnete Geschichte und Zivilisation jedoch erst etwa 5.000 Jahre zurückreichen – was 150 Menschengenerationen entspricht.

Solche Anomalien, wie wir sehen werden, lassen sich am besten durch katastrophale geologische Kräfte erklären.

\section{Blitzgefrorene Mammuts, die im Schlamm begraben wurden}

\begin{figure}[t]
\begin{center}
% \fbox{\rule{0pt}{2in} \rule{0.9\linewidth}{0pt}}
   \includegraphics[width=1\linewidth]{jarkov-mammoth.jpg}
\end{center}
   \caption{Das Jarkov-Mammut, ein 20.000 Jahre altes, perfekt erhaltenes sibirisches Mammut, das im gefrorenen Schlamm gefunden wurde \cite{51}.}
\label{fig:1}
\label{fig:onecol}
\end{figure}

Eine solche Kategorie von Anomalien sind perfekt erhaltene, schockgefrostete Mammuts, die im Schlamm vergraben sind und häufig in den arktischen Regionen gefunden werden (Abbildung \ref{fig:1}). Das Beresowka-Mammut, das in Sibirien in schluffigem Kies begraben entdeckt wurde, war so gut erhalten, dass sein Fleisch noch tausende Jahre nach seinem Tod essbar war. Es hatte zudem pflanzliche Nahrung im Maul und Magen, was Wissenschaftler vor ein Rätsel stellte, wie es so schnell gefrieren konnte, wenn es direkt vor seinem Tod noch blühende Pflanzen fraß \cite{17}. Berichten zufolge \textit{"Im Jahr 1901 sorgte die Entdeckung eines vollständigen Mammutkadavers am Fluss Berezowka für Aufsehen, da dieses Tier scheinbar mitten im Sommer an Kälte starb. Der Inhalt seines Magens war gut erhalten und enthielt Butterblumen und blühende Wildbohnen: Das bedeutete, dass sie etwa Ende Juli oder Anfang August verschluckt worden sein mussten. Das Tier war so plötzlich gestorben, dass es noch eine Maulvoll Gräser und Blumen in den Kiefern hielt. Es war offensichtlich von einer gewaltigen Kraft erfasst und mehrere Meilen von seiner Weide fortgeschleudert worden. Das Becken und ein Bein waren gebrochen—das riesige Tier war auf die Knie geschlagen worden und dann erfroren, und das zur normalerweise heißesten Zeit des Jahres."} \cite{18}. Außerdem \textit{"[Russische Wissenschaftler] stellten fest, dass selbst die innerste Auskleidung des Magens der Tiere eine vollkommen erhaltene faserige Struktur hatte, was darauf hinwies, dass die Körperwärme durch einen äußerst außergewöhnlichen Vorgang in der Natur entzogen worden war. Sanderson, der diesem Punkt besondere Aufmerksamkeit widmete, wandte sich mit dem Problem an das American Frozen Foods Institute: Was ist erforderlich, um ein ganzes Mammut so zu gefrieren, dass selbst im innersten Körperbereich, sogar an der inneren Magenschleimhaut, die Feuchtigkeit keine Zeit hat, Kristalle zu bilden, die groß genug sind, um die Faserstruktur des Fleisches zu zerstören? ... Einige Wochen später kam das Institut auf Sanderson mit der Antwort zurück: Es ist völlig unmöglich. Mit all unserem wissenschaftlichen und technischen Wissen gibt es absolut keine bekannte Möglichkeit, einem Kadaver von der Größe eines Mammuts die Körperwärme so schnell zu entziehen, dass das Fleisch gefriert, ohne dass große Eiskristalle entstehen. Darüber hinaus kamen sie, nachdem sie alle wissenschaftlichen und technischen Methoden geprüft hatten, auch zu dem Schluss, dass es keinen bekannten Prozess in der Natur gibt, der dies bewerkstelligen könnte."} \cite{19}.

\section{Der Grand Canyon}

Der Grand Canyon, Teil des Great Basin im Südwesten Nordamerikas, ist ein weiteres Naturphänomen, das auf katastrophale Ursprünge hindeutet (Abbildung \ref{fig:2}). Zunächst erstrecken sich die sedimentären Sandstein- und Kalksteinschichten, aus denen der Grand Canyon besteht, über gewaltige Flächen von bis zu 2,4 Millionen km$^2$ \cite{21}. Abbildung \ref{fig:3} zeigt die Ausdehnung der Coconino-Sandsteinschicht über den Westen der USA. Solch riesige, horizontale Schichten aus einheitlichem Material konnten nur auf einmal abgelagert worden sein.

\begin{figure}[b]
\begin{center}
% \fbox{\rule{0pt}{2in} \rule{0.9\linewidth}{0pt}}
   \includegraphics[width=1\linewidth]{grand-canyon.jpg}
\end{center}
   \caption{Der Grand Canyon, in Arizona, USA \cite{49}.}
\label{fig:2}
\label{fig:onecol}
\end{figure}

\begin{figure}[t]
\begin{center}
% \fbox{\rule{0pt}{2in} \rule{0.9\linewidth}{0pt}}
   \includegraphics[width=1\linewidth]{coconino.jpg}
\end{center}
   \caption{Größe der Coconino-Sandsteinschicht im Westen der Vereinigten Staaten \cite{21}.}
\label{fig:3}
\label{fig:onecol}
\end{figure}

Ein genauerer Blick auf den Grand Canyon zeigt uns, dass die Ablagerung dieser ausgedehnten Sedimentschichten ebenfalls gleichzeitig mit bedeutenden tektonischen Kräften stattfand. Um dies zu verstehen, müssen wir bestimmte Bereiche im Canyon genauer betrachten, in denen die Sedimentschichten gefaltet und freigelegt wurden. Forscher von Answers in Genesis \cite{42} untersuchten Gesteinsproben aus einigen dieser Falten, wie zum Beispiel der Monument Fold, unter dem Mikroskop und kamen aufgrund des Fehlens von Merkmalen, die vorhanden sein sollten, wenn die Falten über lange Zeiträume unter Hitze und Druck entstanden wären, zu dem Schluss, dass die Sedimentschichten durch tektonische Kräfte gefaltet wurden, während sie noch weich waren, d.\,h. kurz nach ihrer Ablagerung \cite{43}.

\begin{figure*}
\begin{center}
% \fbox{\rule{0pt}{2in} \rule{.9\linewidth}{0pt}}
\includegraphics[width=1\textwidth]{Grand_Staircase-big.jpg}
\end{center}
   \caption{Die Sedimentschichten, aus denen der Grand Canyon besteht (rechte Seite des Bildes), erstrecken sich direkt nach Norden bis nach Cedar Breaks, Utah (linke Seite des Bildes), wo sie alle nach oben gebogen sind \cite{50}.}
\label{fig:4}
\end{figure*}

Zoomt man heraus, stellt man fest, dass die Schichten, aus denen der Grand Canyon besteht, nicht nur innerhalb des Canyons gefaltet wurden. Die Schichten wurden im Osten in der East Kaibab-Monokline gefaltet \cite{46}, aber auch im Norden in Cedar Breaks, Utah (Abbildung \ref{fig:4}). Dies deutet darauf hin, dass all diese Schichten möglicherweise gemeinsam gefaltet wurden, nachdem sie in rascher Folge übereinandergeschichtet wurden. Zur Orientierung: Die horizontalen Schichten des Grand Canyon sind etwa 1700 Meter mächtig. Das Ausmaß des geologischen Prozesses, der erforderlich ist, um Sedimentschichten von fast einer Meile Dicke abzulagern, ist enorm.

Die tatsächliche Entstehung des Grand Canyon ist ein weiteres Streitthema in der modernen Geologie. Die uniformitaristische Geologie schlägt vor, dass der Grand Canyon über Millionen von Jahren durch den Colorado River geformt wurde \cite{47}. Das Forschungsteam von Answers in Genesis hingegen glaubt, dass der Grand Canyon höchstwahrscheinlich innerhalb weniger Wochen durch Erosionsprozesse entstanden ist, nachdem ein urzeitlicher See ausgebrochen war und riesige Sedimentmengen beim Ausschneiden des Canyons abtrug. Es gibt Hinweise auf einen hochgelegenen See östlich des Grand Canyon in Form von Seesedimentablagerungen und Meeresfossilien. Der Vergleich des Grand Canyon mit anderen großmaßstäblichen Beispielen für Erosionsprozesse, wie dem Afton Canyon und Mount St. Helens, zeigt eine ähnliche Topographie und belegt, dass große Canyons sehr schnell durch große Mengen fließenden Wassers entstehen können \cite{48}.

Angesichts des Ausmaßes der geologischen Prozesse, die erforderlich sind, um Sedimente über so riesige Landflächen abzulagern, des gleichzeitigen Auftretens massiver tektonischer Kräfte kurz nach der Ablagerung der Sedimentschichten und der winzigen Größe des Colorado River im Vergleich zur gewaltigen Ausdehnung des Grand Canyon scheint es, dass an seiner Entstehung möglicherweise nichts allmählich war.

\section{Untergrundstadt Derinkuyu}

Abgesehen von den Pyramiden ist ein großartiges Beispiel antiker Ingenieurskunst die unterirdische Stadt Derinkuyu (Abbildung \ref{fig:5}), gelegen in Kappadokien, Türkei. Sie ist die größte von über 200 unterirdischen Schutzanlagen in der Region \cite{54}. Diese unterirdische Stadt soll bis zu 20.000 Menschen beherbergt haben und erstreckt sich über 18 Stockwerke bis in Tiefen von 85 Metern. Ihr genaues Alter ist nicht bekannt, wird aber auf mindestens 2800 Jahre geschätzt. Die Stadt wurde aus weichem vulkanischem Gestein herausgehauen \cite{52, 53}.

\begin{figure}[b]
\begin{center}
% \fbox{\rule{0pt}{2in} \rule{0.9\linewidth}{0pt}}
   \includegraphics[width=1\linewidth]{derinkuyu.jpeg}
\end{center}
   \caption{Diagramm der unterirdischen Stadt Derinkuyu \cite{56}.}
\label{fig:5}
\label{fig:onecol}
\end{figure}
The reason Derinkuyu is interesting is because it's not clear why any community would decide to build an entire city underground. In order to create living space underground, every cavity must be carved out of rock. The rough shapes and textures of the underground tunnels make it clear these were carved with manual labor, rather than with power tools, which would have been orders of magnitude more difficult than building shelters above ground. In fact, it's not apparent why any human would want to permanently live underground during the confines of their earthly life, when agriculture, sunlight, nature, and exploration are only available above ground. Conventional "history" proposes that Derinkuyu was created by Christians who needed a secluded place to practice their religion \cite{53}. But common sense would conclude that the most straightforward way to deal with enemies is "fight or flight", not "carve an underground city out of rock".

The scale, depth, and thoughtfulness of the design of the underground city make it clear that it wasn't designed as a temporary military defensive structure to better fight invaders in times of duress, but rather, a long-term shelter to protect against fatal forces on the surface. Derinkuyu was equipped with not only basic bedrooms, kitchens, and bathrooms, but also stables for animals, water tanks, food storage, wine and oil presses, schools, chapels, tombs, and massive ventilation shafts (Figure \ref{fig:6}). Why would a military shelter require a wine press and need to be be dug 85 meters deep with such complexity?

The most plausible explanation for the creation of Derinkuyu would have been a pressing need to prepare a long-term, self-sustaining shelter to protect against catastrophic geophysical forces on Earth's surface.

\begin{figure}[t]
\begin{center}
% \fbox{\rule{0pt}{2in} \rule{0.9\linewidth}{0pt}}
   \includegraphics[width=1\linewidth]{derinkuyu-air.jpg}
\end{center}
   \caption{Ein tiefer Belüftungsschacht in Derinkuyu \cite{53}.}
\label{fig:6}
\label{fig:onecol}
\end{figure}

% \section{Additional Anomalies Best Explained By An Earth Flip}

% Before wrapping up, we will mention some additional scientific anomalies that, once viewed in the context of cataclysmic geophysical forces, are well explained.
\section{Biomasseanreicherungen}

Biomassemischungen verschiedener Arten von Tieren und Pflanzen, die oft fossiliert in Sedimentschichten gefunden werden, stellen eine weitere rätselhafte Anomalie dar. In "Reliquoæ Diluvianæ" beschreibt Rev. William Buckland Funde zahlreicher Tierarten, für deren gemeinsames Vorkommen es keine erklärbare Ursache gab, verstreut über Großbritannien und Europa, begraben in Schichten von sedimentärem 'Diluvium' \cite{58}. Solche Mischungen von Tierüberresten wurden auch in der Skjonghelleren-Höhle auf der Insel Valdroy, Norwegen, gefunden. In dieser Höhle wurden über 7.000 Knochen von Säugetieren, Vögeln und Fischen vermischt über mehrere Sedimentschichten entdeckt \cite{59}. Ein weiteres Beispiel ist San Ciro, die "Höhle der Giganten", in Italien. In dieser Höhle wurden mehrere Tonnen Säugetierknochen, meist Flusspferde, in einem so frischen Zustand gefunden, dass sie zu Schmuck verarbeitet und für die Herstellung von Lampenschwarz exportiert wurden. Die Knochen der verschiedenen Tiere sollen miteinander vermischt, zerbrochen, zerschlagen und in Fragmenten verstreut gewesen sein \cite{60,61}. Im antiken Mendes, Ägypten, wurde eine Mischung verschiedenartiger Tierknochen zusammen mit glasartig geschmolzenem Ton gefunden \cite{57}. Solche Funde mögen rätselhaft erscheinen, lassen sich jedoch leicht durch massive Überschwemmungen erklären, die Mischungen toter Tiere in Sedimentschichten ablagern, Tiere in Höhlen spülen oder lebendig begraben, und im Fall der verglasten Biomasse in Ägypten durch gewaltige elektrische Entladungen nach einer Kern-Mantel-Verschiebung. Abbildung \ref{fig:7} zeigt eine typische Freilegung des alaskischen Biomasse-'Mucks' \cite{56}.

\begin{figure}[t]
\begin{center}
% \fbox{\rule{0pt}{2in} \rule{0.9\linewidth}{0pt}}
   \includegraphics[width=1\linewidth]{muck-crop.jpeg}
\end{center}
   \caption{Alaskischer 'Muck', bestehend aus chaotisch verteilten Fragmenten von Bäumen, Pflanzen und Tieren in gefrorenem Schlamm und Eis \cite{146}.}
\label{fig:7}
\label{fig:onecol}
\end{figure}

\section{Antike Bunker}

Unsere Vorfahren hinterließen viele hoch entwickelte antike Bauwerke, in denen menschliche Überreste gefunden wurden. Diese werden üblicherweise als aufwändige Gräber interpretiert, aber ein genauerer Blick legt nahe, dass es sich tatsächlich um antike Bunker handeln könnte.

\begin{figure}[b]
\begin{center}
% \fbox{\rule{0pt}{2in} \rule{0.9\linewidth}{0pt}}
   \includegraphics[width=1\linewidth]{ww19.jpg}
\end{center}
   \caption{Newgrange, Irland – sehen Sie die Besucher am Eingang zum Größenvergleich.}
\label{fig:8}
\label{fig:onecol}
\end{figure}

Ein ausgezeichnetes Beispiel ist Newgrange (Abbildung \ref{fig:8}), das Hauptmonument im Brú na Bóinne-Komplex, einer Ansammlung uralter Bauwerke, darunter sogenannte Ganggräber. Diese Gräber bestehen aus einer oder mehreren Grabkammern, die mit Erde oder Steinen bedeckt sind, und haben einen schmalen Zugangsgang aus großen Steinen \cite{70}. Es handelt sich um ein Beispiel für aufwändige Ingenieurskunst eines komplexen geschützten Bauwerks, das über mehrere Generationen hinweg gebaut wurde, angeblich um eine Handvoll Menschen zu bestatten, die nicht einmal mehr lebten, als der Bau des Grabes begann. Als es 1699 von einem örtlichen Grundbesitzer wiederentdeckt wurde, war es von Erde bedeckt.

Ein flüchtiger Blick auf die Struktur offenbart den immensen Aufwand, der in den Bau gesteckt wurde – Newgrange besteht aus etwa 200.000 Tonnen Material. Im Inneren befindet sich \textit{„...eine Gangkammer, die über einen Eingang an der Südostseite des Monuments zugänglich ist. Der Gang erstreckt sich über 19 Meter (60 Fuß), also etwa ein Drittel des Weges ins Zentrum der Struktur. Am Ende des Ganges befinden sich drei kleine Kammern, die von einer größeren zentralen Kammer mit einem hohen Kraggewölbe abgehen… Die Wände dieses Ganges bestehen aus großen Steinplatten, den sogenannten Orthostaten, von denen zweiundzwanzig auf der Westseite und einundzwanzig auf der Ostseite liegen. Sie sind im Durchschnitt 1½ Meter hoch.“} \cite{70}. Es gibt auch ausgeklügelte wasserabweisende Ingenieursdetails. Zum Beispiel im Dach: \textit{„Die Fugen des Daches wurden mit einer Mischung aus verbrannter Erde und Meersand abgedichtet, um sie wasserdicht zu machen, und aus dieser Mischung wurden zwei Radiokarbondaten erhalten, die etwa auf 2500 v. Chr. datiert wurden und sich auf die Struktur des Grabes beziehen.“} \cite{71}. Zusätzlich wurde womöglich eine Erhöhung zum Zugang zur inneren Kammer aus ähnlichen Gründen implementiert: \textit{„Da der Boden des Ganges und der Kammer dem Anstieg des Geländes des Hügels folgt, auf dem das Monument steht, ergibt sich ein Höhenunterschied von fast 2 Metern zwischen dem Eingang und dem Inneren der Kammer.“} \cite{71}.

\begin{figure}[b]
\begin{center}
% \fbox{\rule{0pt}{2in} \rule{0.9\linewidth}{0pt}}
   \includegraphics[width=1\linewidth]{dolmen.jpg}
\end{center}
   \caption{Der Dolmen de Soto, Spanien \cite{53}.}
\label{fig:9}
\label{fig:onecol}
\end{figure}

Das Fehlen menschlicher Überreste im Inneren ist ebenfalls ein bemerkenswerter Punkt. Ausgrabungen förderten verbrannte und unverbrannte Knochenfragmente zutage, die nur eine Handvoll Menschen repräsentieren und über den Gang verstreut waren. Die Errichtung von Newgrange wird anhand der Kohlenstoffdaten des verwendeten Materials auf mindestens mehrere Generationen geschätzt. Warum sollte eine antike Gemeinschaft so viel Mühe aufwenden, um ein massives, hoch entwickeltes Grab zu errichten, nur um die Knochenfragmente einiger weniger Verstorbener im Gang zu verstreuen? Es ist viel plausibler, dass diese uralten und sorgfältig abgedichteten megalithischen Bauwerke stattdessen als menschliche Schutzräume gebaut wurden, um Menschen während wiederkehrender Katastrophen der Erde zu schützen.

In Huelva, Südspanien, ist ein ähnliches Beispiel der Dolmen de Soto (Abbildung \ref{fig:9}), einer von etwa 200 derartigen Stätten in der Gegend \cite{72,32}. Es handelt sich um eine stromlinienförmige, hoch entwickelte Struktur aus megalithischen Steinen mit einem Durchmesser von 75 Metern. Berichten zufolge wurden bei den Ausgrabungen nur acht Leichen gefunden, alle in embryonaler Haltung bestattet.

\section{Bemerkenswerte Hinweise auf Anomalien}

In diesem Abschnitt erwähne ich kurz einige weitere bemerkenswerte Anomalien, die alle durch eine ECBO-ähnliche Katastrophe gut erklärt werden können.

\subsection{Biologische Anomalien}

\begin{figure}[b]
\begin{center}
% \fbox{\rule{0pt}{2in} \rule{0.9\linewidth}{0pt}}
   \includegraphics[width=1\linewidth]{bottleneck.jpg}
\end{center}
   \caption{Ein genetischer Flaschenhals, der eine Auslöschung von 95\% der Männer vor etwa 6.000 Jahren darstellt \cite{62}.}
\label{fig:10}
\label{fig:onecol}

\end{figure}

Einige bemerkenswerte biologische Anomalien sind genetische Flaschenhälse und Walfossilien im Landesinneren. Zeng et al. (2018) modellierten 125 Y-Chromosom-Sequenzen von heutigen Menschen und identifizierten auf der Grundlage von Ähnlichkeiten und Mutationen in der DNA einen 95\%igen Bevölkerungsrückgang in der männlichen Bevölkerung vor etwa 5.000 bis 7.000 Jahren (Abbildung \ref{fig:10}) \cite{62}. Walfossilien wurden Hunderte Meter über dem Meeresspiegel gefunden, in Schwedenborg, Michigan, Vermont, Kanada, Chile und Ägypten \cite{63,64,65,66}. Diese Wale wurden in unterschiedlichen Zuständen gefunden: perfekt erhalten, in Mooren über glazialen Ablagerungen liegend oder im Sediment vergraben. Die Anzahl der Exemplare an diesen Fundstellen reicht von wenigen bis über hundert. Wale sind Tiefseetiere und nähern sich selten den Küsten. Wie gelangten diese Wale auf solch hohe Höhen, oftmals tief ins Landesinnere hinein?

Zahlreiche Massenaussterben haben in der Erdgeschichte stattgefunden, die am gründlichsten untersuchten sind die „Großen Fünf“ phanerozoischen Ereignisse: das spätordovizische (LOME), spätdevonische (LDME), permische (EPME), triasische (ETME) und kreidezeitliche (ECME) Massenaussterben \cite{88,89}. Interessanterweise werden mehrere dieser Aussterben in dieselben historischen Perioden eingeordnet wie viele der Schichten des Grand Canyon, nämlich die permischen und devonischen Schichten.

\subsection{Physikalische Anomalien}

\begin{figure}[b]
\begin{center}
% \fbox{\rule{0pt}{2in} \rule{0.9\linewidth}{0pt}}
   \includegraphics[width=1\linewidth]{columbia.jpg}
\end{center}
   \caption{Massive Strömungsrippeln im Glazialsee Columbia, Bundesstaat Washington \cite{80}.}
\label{fig:11}
\label{fig:onecol}
\end{figure}

Es gibt viele Landschaften neben dem Grand Canyon, die wahrscheinlich durch katastrophale Kräfte entstanden sind. Hinweise auf massive kontinentale Wasserflüsse finden sich in riesigen Strömungsrippeln weltweit. Ein solches Beispiel sind die Channeled Scablands im pazifischen Nordwesten. Hier sieht man nicht nur Landschaften aus sedimentären Ablagerungen und verstreute Findlinge, sondern auch mehr als hundert Abfolgen großer Rippeln, die durch Mega-Strömungsfluten gebildet wurden \cite{78,79}. Diese sind großformatige Versionen der Rippeln, die in Sandbetten von Bächen entstehen. Solche Erscheinungen finden sich weltweit in Frankreich, Argentinien, Russland und Nordamerika \cite{81}. Abbildung \ref{fig:11} zeigt einige dieser Rippeln im US-Bundesstaat Washington \cite{80}.
\begin{figure}[b]
\begin{center}
% \fbox{\rule{0pt}{2in} \rule{0.9\linewidth}{0pt}}
   \includegraphics[width=1\linewidth]{zhangjiajie.jpg}
\end{center}
   \caption{Massive Steinsäulen im Zhangjiajie-Nationalwald, Südchina.}
\label{fig:12}
\label{fig:onecol}
\end{figure}

\begin{figure}[b]
\begin{center}
% \fbox{\rule{0pt}{2in} \rule{0.9\linewidth}{0pt}}
   \includegraphics[width=1\linewidth]{hoy.jpg}
\end{center}
   \caption{Old Man of Hoy Meersäule, Schottland \cite{83}.}
\label{fig:13}
\label{fig:onecol}
\end{figure}
Inländische Erosionsstrukturen werden ebenfalls gut durch einen ECDO-ähnlichen Erdflip erklärt. Südchina ist ein hervorragendes Beispiel für riesige Karstlandschaften, die durch Wassererosion entstanden sind \cite{82}. Diese Landschaften umfassen Turmkarst, Pinnakelnkarst, Konuskarst, natürliche Brücken, Schluchten, große Höhlensysteme und Dolinen. Eine der auffälligsten davon ist der Zhangjiajie National Forest, der massive Quarzsandsteinpfeiler enthält (Abbildung \ref{fig:12}) \cite{84}. Diese Pfeiler stehen auf einer durchschnittlichen Höhe von über 1.000 Metern und es gibt mehr als 3.100 davon. Mehr als 1.000 davon ragen über 120 Meter hoch, und 45 erreichen über 300 Meter \cite{85}. Diese Pfeiler ähneln Seeerosionspfeilern (Abbildung \ref{fig:13}), die Küstenfelsenpfeiler sind und durch den Einsturz des umliegenden Materials aufgrund von Meereswellen entstehen. Ähnliche Erosionslandschaften finden sich in den Felsnadeln von Ürgüp, Türkei, sowie in der Ciudad Encantada, Spanien, beide über 1.000 Meter über dem Meeresspiegel. All diese Orte haben eine Kombination aus Salz und marinen Fossilien in unmittelbarer Nähe, was auf frühere Meereseinbrüche hindeutet \cite{15,86,87}. Natürlich erwähnen die Flutgeschichten \cite{3}, dass der Ozean viel höher als 1.000 Meter stieg, und dies wird durch das Vorkommen von Salzwasser und riesigen Salzpfannen in den Anden und im Himalaya, mehrere Kilometer über dem Meeresspiegel, bestätigt. Die Uyuni-Salzwüste in Bolivien beispielsweise erreicht 3.653 Meter über dem Meeresspiegel \cite{94}.

\subsection{Ereignisse rapider Klimaveränderung}

Die moderne wissenschaftliche Literatur erkennt die Existenz rapider globaler Klimaveränderungsereignisse in der jüngeren Erdgeschichte an. Zwei bemerkenswerte Beispiele sind das 4,2-Kilojahr- und das 8,2-Kilojahr-Ereignis, beide in Zusammenhang mit Bevölkerungsrückgang und gesellschaftlicher Siedlungsunterbrechung über große geografische Gebiete. Diese Ereignisse sind als Anomalien in Sediment- und Eiskernen, fossilen Korallen, O18-Isotopenwerten, Pollendaten und Stalagmiten-Aufzeichnungen sowie Meeresspiegeldaten erhalten. Die daraus abgeleiteten Klimaveränderungen umfassen einen rapiden Abfall der globalen Temperaturen, Aridifizierung, eine Störung der Atlantischen meridionalen Umwälzströmung und Gletschervorstöße \cite{90,91,92}. Das 8,2-Kilojahr-Ereignis fällt insbesondere mit einer möglichen dramatischen Salzwasserflutung des Schwarzen Meeres um 6400 v. Chr. zusammen \cite{93}.

\subsection{Archäologische Anomalien}

Archäologische Beweise einiger antiker Städte zeigen mehrere Schichten von Begrabung und Zerstörung, die Aufzeichnungen vergangener katastrophaler Ereignisse schaffen. Die antike Stadt Jericho ist eine solche Stadt, gelegen im heutigen Palästina. Sie enthält mehrere Zerstörungsschichten mit dem Einsturz von Steinstrukturen und intensiven Bränden \cite{96,97}. Die in ihren Schichten überlieferte Chronologie reicht von etwa 9000 v. Chr. bis 2000 v. Chr. Besonders bemerkenswert ist ihr Turm, der offenbar abgeschoren und um 7400 v. Chr. im Sediment vergraben wurde (Abbildung \ref{fig:14}) \cite{95}. Catal Huyuk \cite{99}, Gramalote \cite{98} und der minoische Palast von Knossos auf Kreta \cite{100,101} sind alles ähnliche Beispiele archäologischer Stätten mit mehreren Schichten, die oft Hinweise auf Zerstörung enthalten.

\begin{figure}[t]
\begin{center}
% \fbox{\rule{0pt}{2in} \rule{0.9\linewidth}{0pt}}
   \includegraphics[width=1\linewidth]{jericho.jpg}
\end{center}
   \caption{Archäologische Rekonstruktion der Begrabung des Turms von Jericho um 7400 v. Chr. \cite{95}.}
\label{fig:14}
\label{fig:onecol}
\end{figure}
Another piece of evidence for major cataclysms disrupting human civilization is the Nampa Image, a clay doll found beneath approximately 100 meters of lava in Idaho \cite{102,103}. The lava flow under which the figurine was found was estimated to be deposited during the Late Tertiary or early Quaternary period, supposedly being 2 million years old. However, the lava flow in the region appears to be relatively fresh. Such finds not only point to major civilization-destroying cataclysms, but also call into question modern dating chronologies.

Ein weiteres Indiz für große Katastrophen, die die menschliche Zivilisation unterbrochen haben, ist das Nampa-Bild, eine Tonpuppe, die unter etwa 100 Metern Lava in Idaho gefunden wurde \cite{102,103}. Der Lavastrom, unter dem die Figur gefunden wurde, soll während des späten Tertiärs oder frühen Quartärs abgelagert worden sein und angeblich 2 Millionen Jahre alt sein. Allerdings wirkt der Lavastrom in der Region relativ frisch. Solche Funde deuten nicht nur auf große, zivilisationszerstörende Katastrophen hin, sondern stellen auch moderne Datierungschronologien in Frage.

\section{Zu modernen Datierungsmethoden}

Es gibt gute Gründe, bei modernen Chronologien skeptisch zu sein, die verschiedenen physikalischen Materialien extrem lange Zeitspannen von Millionen oder sogar Hunderten von Millionen Jahren zuschreiben.

Die konventionelle Erzählung besagt, dass sogenannte „fossile Brennstoffe“ wie Kohle, Öl und Erdgas Hunderte Millionen Jahre alt sind \cite{104}. Eine tatsächliche Radiokohlenstoffdatierung von Öl im Golf von Mexiko ergab jedoch ein Alter von etwa 13.000 Jahren für das Öl \cite{105}. Kohlenstoff-14 hat eine so kurze Halbwertszeit (5.730 Jahre), dass es nach ein paar hunderttausend Jahren vollständig zerfallen sein soll. Dennoch wurde es in Kohle und Fossilien gefunden, die angeblich tausendmal älter sind \cite{106}. Tatsächlich wurde im Labor unter kontrollierten Bedingungen, hauptsächlich großer Hitze, künstliche Kohle in nur 2-8 Monaten hergestellt \cite{107}.

Auch radioisotopische Datierungsmethoden außer der Radiokohlenstoffdatierung sind möglicherweise nicht zuverlässig. Die Forschungsgruppe „Answers in Genesis“ stellte Inkonsistenzen bei den mit solchen Methoden ermittelten Daten fest, die ihre Zuverlässigkeit in Frage stellen \cite{108}. Weiches Gewebe mit Blutzellen, Gefäßen und Kollagen wurde sogar in Dinosaurierfunden entdeckt, die angeblich hundert Millionen Jahre alt sind \cite{109,110}. Nach dem Stand unseres Wissens ist es möglich, dass die allgemein akzeptierten Altersangaben der geologischen Zeitskala der Erde und von physikalischen Materialien wie Gestein und fossilen Brennstoffen um viele Größenordnungen falsch sind.

\section{Fazit}

In dieser Arbeit habe ich die überzeugendsten Anomalien zusammengetragen, die auf katastrophale Ursprünge hindeuten und am besten durch einen ECDO-Erdflip erklärt werden. Die präsentierte Sammlung ist zwar vielfältig, aber unvollständig – weitere Anomalien sind zusammengetragen und in meinem GitHub-Forschungsarchiv öffentlich zugänglich \cite{2}.

\section{Danksagung}

Mein Dank gilt Ethical Skeptic, dem ursprünglichen Autor der ECDO-These, für die Fertigstellung seiner aufschlussreichen, bahnbrechenden Arbeit und ihre Weitergabe an die Welt. Seine dreiteilige Abhandlung \cite{1} bleibt das maßgebliche Werk zur Theorie der Exothermic Core-Mantle Decoupling Dzhanibekov Oscillation (ECDO) und enthält weitaus mehr Informationen zu diesem Thema, als ich hier kurz zusammengefasst habe.

Und natürlich danke ich den Riesen, auf deren Schultern wir stehen; jenen, die all die Forschung und Untersuchungen geleistet haben, die diese Arbeit möglich gemacht und dazu beigetragen haben, Licht in die Menschheit zu bringen.

\clearpage
\twocolumn

{\small
\renewcommand{\refname}{References}
\bibliographystyle{ieee}
\bibliography{egbib}
}

\end{document}
\documentclass[10pt,twocolumn,letterpaper]{article}

% Barang saya sendiri
\usepackage{booktabs}
% \usepackage{caption}
% \captionsetup[table]{skip=8pt}   % Hanya mempengaruhi tabel
\usepackage{stfloats}  % Tambahkan ini ke preambule

\usepackage{cvpr}
\usepackage{times}
\usepackage{epsfig}
\usepackage{graphicx}
\usepackage{amsmath}
\usepackage{amssymb}

% Sertakan paket lain di sini, sebelum hyperref.

% Jika Anda mengomentari hyperref lalu menghapus komentarnya, Anda harus menghapus
% egpaper.aux sebelum menjalankan latex lagi.  (Atau cukup tekan 'q' pada latex
% pertama, biarkan selesai, dan Anda akan aman).
\usepackage[breaklinks=true,bookmarks=false]{hyperref}

\cvprfinalcopy % *** Uncomment this line for the final submission

\def\cvprPaperID{****} % *** Enter the CVPR Paper ID here
\def\httilde{\mbox{\tt\raisebox{-.5ex}{\symbol{126}}}}

% Halaman diberi nomor dalam mode pengajuan, dan tidak bernomor pada versi siap cetak
%\ifcvprfinal\pagestyle{empty}\fi
\setcounter{page}{1}
\begin{document}

%%%%%%%%% JUDUL
\title{Primer Berbasis Data ECDO Bagian 2/2: Investigasi Anomali Ilmiah dan Sejarah yang Paling Baik Dijelaskan oleh “Pembalikan Bumi” ECDO}

\author{Junho\\
Diterbitkan Februari 2025\\
Situs Web (Unduh makalah di sini): \href{https://sovrynn.github.io}{sovrynn.github.io}\\
Repo Riset ECDO: \href{https://github.com/sovrynn/ecdo}{github.com/sovrynn/ecdo}\\
{\tt\small junhobtc@proton.me}

```latex
% Untuk sebuah makalah yang semua penulisnya berasal dari institusi yang sama,
% hilangkan baris-baris berikut hingga penutup ``}''.
% Penulis dan alamat tambahan dapat ditambahkan dengan ``\and'',
% seperti penulis kedua.
% Untuk menghemat ruang, gunakan alamat email atau halaman rumah, bukan keduanya
% \and
% xx
% Institution2\\
% Baris pertama alamat institution2\\
% {\tt\small secondauthor@i2.org}
}

\maketitle
%\thispagestyle{empty}

%%%%%%%%% ABSTRACT
\begin{abstract}
Pada bulan Mei 2024, seorang penulis online dengan nama samaran “The Ethical Skeptic” \cite{0} memposting sebuah teori terobosan yang disebut Exothermic Core-Mantle Decoupling Dzhanibekov Oscillation (ECDO) \cite{1}. Teori ini tidak hanya mengusulkan bahwa Bumi sebelumnya pernah mengalami pergeseran mendadak dan katastropik pada sumbu rotasinya, menciptakan banjir besar di seluruh dunia akibat lautan tumpah ke daratan karena inersia rotasi, tetapi juga mengusulkan proses geofisika penyebab yang bersifat penjelasan beserta data yang menunjukkan bahwa peristiwa semacam itu mungkin akan segera terjadi. Walaupun ramalan tentang banjir katastropik dan kiamat bukanlah hal baru, teori ECDO sangat menarik karena pendekatannya yang ilmiah, modern, multidisipliner, dan berbasis data.

Makalah penelitian ini merupakan bagian kedua dari rangkuman ringkas dua bagian dari 6 bulan penelitian independen \cite{2,20} tentang teori ECDO, dengan fokus khusus pada anomali ilmiah dan sejarah yang paling baik dijelaskan oleh peristiwa "pembalikan Bumi" ECDO yang katastropik.
\end{abstract}
```
\end{abstract}

%%%%%%%%% BODY TEXT

\section{Pendahuluan}

Geologi dan sejarah uniformitarian modern mengklaim bahwa lanskap geologi utama seperti Grand Canyon terbentuk selama jutaan tahun \cite{143}; bahwa garam yang ada di Death Valley (California) disebabkan tempat itu dulunya berada di bawah laut ratusan juta tahun yang lalu \cite{144}; bahwa leluhur kita dari 150 generasi yang lalu menghabiskan seluruh hidupnya membangun makam-makam raksasa \cite{29,70}; dan bahwa yang disebut "bahan bakar fosil" berusia ratusan juta tahun \cite{104}. Mungkin yang paling menarik adalah bahwa manusia diyakini berusia 300.000 tahun \cite{145}, namun sejarah tertulis dan peradaban hanya ada sekitar 5.000 tahun yang lalu – setara dengan 150 generasi manusia.

Keanehan-keanehan seperti ini, seperti akan kita lihat, paling baik dijelaskan oleh kekuatan geologi yang bersifat katastrofik.

\section{Mamut yang Membeku Seketika dan Terkubur dalam Lumpur}

\begin{figure}[t]
\begin{center}
% \fbox{\rule{0pt}{2in} \rule{0.9\linewidth}{0pt}}
   \includegraphics[width=1\linewidth]{jarkov-mammoth.jpg}
\end{center}
   \caption{Jarkov Mammoth, mamut Siberia berusia 20.000 tahun yang terawetkan sempurna dan ditemukan dalam lumpur beku \cite{51}.}
\label{fig:1}

\label{fig:onecol}
\end{figure}

Salah satu kategori anomali tersebut adalah mamut yang diawetkan secara sempurna yang membeku secara mendadak dan terkubur di lumpur, yang biasa ditemukan di wilayah Arktik (Gambar \ref{fig:1}). Mamut Beresovka, yang ditemukan di Siberia terkubur di kerikil berlumpur, diawetkan dengan sangat baik sehingga dagingnya masih dapat dimakan ribuan tahun setelah kematiannya. Ia juga memiliki sisa makanan tumbuhan di mulut dan perutnya, yang membingungkan para ilmuwan bagaimana mungkin hewan ini bisa membeku begitu cepat jika ia sedang merumput pada tanaman berbunga sesaat sebelum kematiannya \cite{17}. Dilaporkan, \textit{"Pada tahun 1901 terjadi sensasi akibat penemuan bangkai utuh mamut di dekat sungai Berezovka, karena hewan ini tampaknya mati kedinginan di musim panas. Isi perutnya terawetkan dengan baik dan ditemukan bunga buttercup serta kacang liar berbunga: ini berarti bahwa bahan tersebut pasti ditelan sekitar akhir Juli atau awal Agustus. Hewan itu mati begitu tiba-tiba sehingga masih menggenggam rumput dan bunga di mulutnya. Jelas ia telah terhempas oleh kekuatan dahsyat dan terlempar beberapa mil dari padang tempatnya merumput. Panggul dan satu kakinya patah—hewan besar itu jatuh berlutut dan mati beku, pada saat yang biasanya merupakan waktu terpanas sepanjang tahun"} \cite{18}. Selain itu, \textit{"[Ilmuwan Rusia] mencatat bahwa bahkan lapisan terdalam dari lambung hewan itu memiliki struktur serat yang terawetkan secara sempurna, menunjukkan bahwa panas tubuhnya telah diambil oleh suatu proses luar biasa di alam. Sanderson, memberikan perhatian khusus pada satu hal ini, membawa masalah ini ke American Frozen Foods Institute: Apa yang dibutuhkan untuk membekukan seekor mamut secara keseluruhan sehingga kadar air pada bagian terdalam tubuhnya, bahkan sampai lapisan terdalam lambungnya, tidak punya cukup waktu untuk membentuk kristal besar yang bisa merusak struktur serat daging?... Beberapa minggu kemudian, Institut tersebut memberikan jawaban kepada Sanderson: Ini benar-benar mustahil. Dengan seluruh pengetahuan ilmiah dan teknologi kami, sama sekali tidak ada metode yang diketahui untuk menghilangkan panas tubuh dari bangkai sebesar mamut cukup cepat untuk membekukannya tanpa membentuk kristal-kristal air yang besar pada dagingnya. Lebih jauh lagi, setelah kehabisan segala teknik ilmiah dan teknologi, mereka melihat ke alam dan menyimpulkan bahwa tidak ada proses alami yang diketahui dapat melakukan hal itu"} \cite{19}.

\section{Grand Canyon}

Grand Canyon, bagian dari Great Basin di barat daya Amerika Utara, adalah fenomena alam lain yang mengisyaratkan asal-usul yang bersifat katastrofis (Gambar \ref{fig:2}). Sebagai permulaan, lapisan batu pasir dan batu kapur sedimen yang membentuk Grand Canyon membentang sangat luas hingga 2,4 juta km$^2$ \cite{21}. Gambar \ref{fig:3} memperlihatkan luas lapisan Coconino Sandstone di seluruh bagian barat Amerika Serikat. Lapisan horizontal masif dari material yang seragam ini hanya dapat diendapkan dalam satu waktu.

\begin{figure}[b]
\begin{center}
% \fbox{\rule{0pt}{2in} \rule{0.9\linewidth}{0pt}}
   \includegraphics[width=1\linewidth]{grand-canyon.jpg}
\end{center}
   \caption{Grand Canyon, di Arizona, AS \cite{49}.}
\label{fig:2}
\label{fig:onecol}
\end{figure}

\begin{figure}[t]
\begin{center}
% \fbox{\rule{0pt}{2in} \rule{0.9\linewidth}{0pt}}
   \includegraphics[width=1\linewidth]{coconino.jpg}
\end{center}
   \caption{Ukuran lapisan Coconino Sandstone di bagian barat Amerika Serikat \cite{21}.}
\label{fig:3}
\label{fig:onecol}
\end{figure}

Tinjauan lebih dekat terhadap Grand Canyon menunjukkan bahwa pengendapan lapisan sedimen yang luas ini juga terjadi bersamaan dengan gaya tektonik yang signifikan. Untuk memahami hal ini, kita harus mengamati dengan saksama area tertentu di ngarai di mana lapisan sedimen telah terlipat dan terbuka. Peneliti dari Answers in Genesis \cite{42} mengamati secara mikroskopis sampel batuan dari beberapa lipatan ini, seperti Monument Fold, dan berdasarkan tidak adanya fitur yang seharusnya ada jika lipatan tersebut terbentuk dalam jangka waktu lama di bawah panas dan tekanan, menyimpulkan bahwa lapisan sedimen tersebut terlipat oleh gaya tektonik ketika masih lunak, yaitu segera setelah pengendapannya \cite{43}.

\begin{figure*}
\begin{center}
% \fbox{\rule{0pt}{2in} \rule{.9\linewidth}{0pt}}
\includegraphics[width=1\textwidth]{Grand_Staircase-big.jpg}
\end{center}
   \caption{Lapisan sedimen yang membentuk Grand Canyon (sisi kanan gambar) membentang langsung ke utara hingga Cedar Breaks, Utah (sisi kiri gambar), di mana semuanya membengkok ke atas \cite{50}.}
\label{fig:4}
\end{figure*}
Zooming out, kita menemukan bahwa lapisan-lapisan yang membentuk Grand Canyon tidak hanya terlipat di dalam ngarai tersebut. Lapisan-lapisan tersebut telah terlipat ke timur di East Kaibab Monocline \cite{46}, tetapi juga ke utara di Cedar Breaks, Utah (Gambar \ref{fig:4}). Hal ini menunjukkan bahwa lapisan-lapisan tersebut mungkin semuanya telah terlipat bersama setelah mereka tertumpuk satu sama lain dalam waktu singkat. Sebagai referensi, lapisan-lapisan horizontal Grand Canyon memiliki ketebalan sekitar 1700 meter. Skala proses geologi yang dibutuhkan untuk menumpuk lapisan sedimen setebal satu mil sangatlah besar.

Pembentukan Grand Canyon yang sebenarnya merupakan isu lain yang menjadi perdebatan dalam geologi modern. Geologi uniformitarian mengemukakan bahwa Grand Canyon dipahat oleh Sungai Colorado selama jutaan tahun \cite{47}. Namun, tim peneliti Answers in Genesis percaya bahwa Grand Canyon kemungkinan besar terbentuk hanya dalam hitungan minggu akibat erosi spillway dari danau kuno yang meluap melewati batasnya, yang menghilangkan sejumlah besar sedimen saat membentuk ngarai tersebut. Ada bukti keberadaan danau dengan elevasi tinggi di sebelah timur Grand Canyon pada endapan sedimen danau dan fosil laut. Membandingkan Grand Canyon dengan contoh erosi spillway berskala besar lainnya, seperti Afton Canyon dan Gunung St. Helens, menunjukkan topografi yang serupa, dan memperlihatkan bahwa ngarai besar dapat tercipta secara cepat melalui aliran air yang besar \cite{48}.

Dengan mempertimbangkan skala proses geologi yang diperlukan untuk menumpuk sedimen di atas tanah seluas itu, terjadinya gaya tektonik besar tak lama setelah lapisan sedimen tersebut tertumpuk, serta kecilnya ukuran Sungai Colorado dibandingkan skala Grand Canyon yang masif, tampaknya tidak ada yang bersifat bertahap tentang pembentukan ngarai tersebut.

\section{Kota Bawah Tanah Derinkuyu}

Selain piramida, contoh luar biasa rekayasa kuno adalah kota bawah tanah Derinkuyu (Gambar \ref{fig:5}), yang terletak di Cappadocia, Turki. Ini adalah yang terbesar di antara lebih dari 200 tempat perlindungan bawah tanah di wilayah tersebut \cite{54}. Kota bawah tanah ini diperkirakan pernah menampung hingga 20.000 orang dan memiliki 18 lantai, dengan kedalaman mencapai 85 meter. Walaupun usianya tidak pasti, diperkirakan kota ini berumur setidaknya 2800 tahun. Kota ini dipahat dari batuan vulkanik yang lunak \cite{52, 53}.

\begin{figure}[b]
\begin{center}
% \fbox{\rule{0pt}{2in} \rule{0.9\linewidth}{0pt}}
   \includegraphics[width=1\linewidth]{derinkuyu.jpeg}
\end{center}
   \caption{Diagram kota bawah tanah Derinkuyu \cite{56}.}
\label{fig:5}
\label{fig:onecol}
\end{figure}
The reason Derinkuyu is interesting is because it's not clear why any community would decide to build an entire city underground. In order to create living space underground, every cavity must be carved out of rock. The rough shapes and textures of the underground tunnels make it clear these were carved with manual labor, rather than with power tools, which would have been orders of magnitude more difficult than building shelters above ground. In fact, it's not apparent why any human would want to permanently live underground during the confines of their earthly life, when agriculture, sunlight, nature, and exploration are only available above ground. Conventional "history" proposes that Derinkuyu was created by Christians who needed a secluded place to practice their religion \cite{53}. But common sense would conclude that the most straightforward way to deal with enemies is "fight or flight", not "carve an underground city out of rock".

Alasan Derinkuyu menarik adalah karena tidak jelas mengapa suatu komunitas memutuskan untuk membangun seluruh kota di bawah tanah. Untuk menciptakan ruang hidup di bawah tanah, setiap ruang harus dipahat dari batu. Bentuk kasar dan tekstur terowongan bawah tanah menunjukkan bahwa tempat ini dipahat dengan tenaga kerja manual, bukan dengan alat berat, yang akan jauh lebih sulit daripada membangun tempat berlindung di atas tanah. Faktanya, tidak jelas mengapa ada manusia yang ingin tinggal secara permanen di bawah tanah selama hidupnya, ketika pertanian, sinar matahari, alam, dan penjelajahan hanya tersedia di atas permukaan tanah. "Sejarah" konvensional mengusulkan bahwa Derinkuyu dibuat oleh orang Kristen yang membutuhkan tempat terpencil untuk menjalankan agama mereka \cite{53}. Tetapi akal sehat akan menyimpulkan bahwa cara paling langsung untuk menghadapi musuh adalah "melawan atau melarikan diri", bukan "memahat kota bawah tanah dari batu".

The scale, depth, and thoughtfulness of the design of the underground city make it clear that it wasn't designed as a temporary military defensive structure to better fight invaders in times of duress, but rather, a long-term shelter to protect against fatal forces on the surface. Derinkuyu was equipped with not only basic bedrooms, kitchens, and bathrooms, but also stables for animals, water tanks, food storage, wine and oil presses, schools, chapels, tombs, and massive ventilation shafts (Figure \ref{fig:6}). Why would a military shelter require a wine press and need to be be dug 85 meters deep with such complexity?

Skala, kedalaman, dan perancangan kota bawah tanah yang begitu matang menunjukkan bahwa tempat ini tidak didesain sebagai struktur pertahanan militer sementara untuk melawan penjajah di masa genting, melainkan sebagai tempat perlindungan jangka panjang untuk melindungi dari kekuatan mematikan di permukaan. Derinkuyu dilengkapi tidak hanya dengan kamar tidur, dapur, dan kamar mandi dasar, tetapi juga kandang hewan, tangki air, penyimpanan makanan, alat pemeras anggur dan minyak, sekolah, kapel, makam, serta lorong ventilasi raksasa (Gambar \ref{fig:6}). Mengapa tempat perlindungan militer membutuhkan alat pemeras anggur dan harus digali sedalam 85 meter dengan tingkat kompleksitas seperti itu?

The most plausible explanation for the creation of Derinkuyu would have been a pressing need to prepare a long-term, self-sustaining shelter to protect against catastrophic geophysical forces on Earth's surface.

Penjelasan paling masuk akal untuk pembangunan Derinkuyu kemungkinan adalah kebutuhan mendesak untuk mempersiapkan tempat perlindungan jangka panjang yang mandiri guna melindungi dari kekuatan geofisika katastropik di permukaan Bumi.

\begin{figure}[t]
\begin{center}
% \fbox{\rule{0pt}{2in} \rule{0.9\linewidth}{0pt}}
   \includegraphics[width=1\linewidth]{derinkuyu-air.jpg}
\end{center}
   \caption{Sebuah sumur ventilasi yang dalam di Derinkuyu \cite{53}.}
\label{fig:6}
\label{fig:onecol}
\end{figure}

% \section{Additional Anomalies Best Explained By An Earth Flip}

% Before wrapping up, we will mention some additional scientific anomalies that, once viewed in the context of cataclysmic geophysical forces, are well explained.
\section{Akumulasi Biomassa}

Campuran biomassa dari berbagai jenis hewan dan tumbuhan, yang sering ditemukan dalam bentuk fosil di lapisan sedimen, merupakan anomali yang membingungkan. Dalam "Reliquoæ Diluvianæ", Pendeta William Buckland merinci temuan berbagai spesies fauna yang tidak memiliki alasan jelas untuk ditemukan bersama, tersebar di seluruh Inggris dan Eropa, terkubur dalam lapisan 'diluvium' sedimen \cite{58}. Campuran sisa-sisa hewan seperti ini juga ditemukan di Gua Skjonghelleren di pulau Valdroy, Norwegia. Di gua ini, lebih dari 7.000 tulang mamalia, burung, dan ikan ditemukan tercampur di beberapa lapisan sedimen \cite{59}. Contoh lain adalah San Ciro, "Gua Para Raksasa", di Italia. Di gua ini, beberapa ton tulang mamalia, terutama kuda nil, ditemukan dalam kondisi sangat segar hingga dipotong untuk dijadikan ornamen dan dikirim untuk pembuatan arang lampu. Tulang-tulang dari berbagai hewan tersebut dilaporkan tercampur, patah, hancur, dan tersebar dalam bentuk fragmen \cite{60,61}. Di Mendes Kuno, Mesir, campuran berbagai spesies tulang hewan ditemukan bercampur dengan tanah liat vitreous (kaca) \cite{57}. Temuan seperti ini mungkin tampak membingungkan, tetapi mudah dijelaskan oleh banjir besar yang membenamkan campuran hewan mati dalam lapisan sedimen, mendepositkan hewan ke dalam atau mengubur mereka hidup-hidup di gua, dan dalam kasus biomassa vitreous di Mesir, pelepasan listrik besar pasca-banjir akibat perpindahan inti-mantel. Gambar \ref{fig:7} memperlihatkan paparan khas 'muck' biomassa Alaska \cite{56}.

\begin{figure}[t]
\begin{center}
% \fbox{\rule{0pt}{2in} \rule{0.9\linewidth}{0pt}}
   \includegraphics[width=1\linewidth]{muck-crop.jpeg}
\end{center}
   \caption{'Muck' Alaska, terdiri dari fragmen pohon, tumbuhan, dan hewan yang tersebar secara kacau dalam lumpur beku dan es \cite{146}.}
\label{fig:7}
\label{fig:onecol}
\end{figure}

\section{Bunker Kuno}

Nenek moyang kita meninggalkan banyak struktur kuno yang sangat terencana, di mana sisa-sisa manusia telah ditemukan. Struktur-struktur ini biasanya dianggap sebagai makam yang rumit, namun jika diamati lebih dekat, struktur ini mungkin sebenarnya merupakan bunker kuno.

\begin{figure}[b]
\begin{center}
% \fbox{\rule{0pt}{2in} \rule{0.9\linewidth}{0pt}}
   \includegraphics[width=1\linewidth]{ww19.jpg}
\end{center}
   \caption{Newgrange, Irlandia - lihat pengunjung di pintu masuk untuk skala.}
\label{fig:8}
\label{fig:onecol}
\end{figure}

Salah satu contoh yang sangat baik adalah Newgrange (Gambar \ref{fig:8}), monumen utama di kompleks Brú na Bóinne, kumpulan struktur kuno termasuk apa yang disebut kuburan lorong. Kuburan ini terdiri dari satu atau lebih ruang pemakaman yang ditutupi tanah atau batu dan memiliki lorong akses sempit yang terbuat dari batu-batu besar \cite{70}. Ini adalah contoh rekayasa ekstensif dari struktur kompleks yang dilindungi, dibangun selama beberapa generasi, konon untuk menguburkan segelintir orang, yang bahkan sudah tidak hidup lagi saat pembangunan makam dimulai. Ketika ditemukan kembali oleh seorang pemilik tanah lokal pada tahun 1699, kuburan tersebut tertimbun tanah.

Sekilas melihat struktur ini mengungkapkan usaha besar yang dilakukan dalam pembangunannya - Newgrange terdiri dari sekitar 200.000 ton material. Di dalamnya, \textit{“…ada sebuah lorong berkamah, yang dapat diakses melalui pintu masuk di sisi tenggara monumen. Lorong ini membentang sepanjang 19 meter (60 kaki), atau sekitar sepertiga dari jalan menuju pusat struktur. Di ujung lorong terdapat tiga ruangan kecil yang bercabang dari sebuah ruang tengah yang lebih besar dengan atap kubah corbel… Dinding lorong ini terdiri dari lempengan-lempengan batu besar yang disebut ortostat, dua puluh dua di antaranya di sisi barat dan dua puluh satu di sisi timur. Tinggi rata-ratanya 1½ meter”} \cite{70}. Ada juga detail teknik kedap air yang rumit. Misalnya, pada atapnya, \textit{“Rongga-rongga atapnya diisi dengan campuran tanah bakar dan pasir laut untuk membuatnya kedap air dan dari campuran ini dua penanggalan radiokarbon berpusat pada 2500 SM diperoleh untuk struktur makam tersebut”} \cite{71}. Selain itu, elevasi yang naik menuju ruang dalam mungkin juga diterapkan untuk tujuan serupa: \textit{“Karena lantai lorong dan ruang makam mengikuti kenaikan tanah dari bukit tempat monumen itu dibangun, terdapat perbedaan hampir 2 meter pada tinggi lantai antara pintu masuk dengan bagian dalam ruangannya”} \cite{71}.

\begin{figure}[b]
\begin{center}
% \fbox{\rule{0pt}{2in} \rule{0.9\linewidth}{0pt}}
   \includegraphics[width=1\linewidth]{dolmen.jpg}
\end{center}
   \caption{Dolmen de Soto, Spanyol \cite{53}.}
\label{fig:9}
\label{fig:onecol}
\end{figure}

Kurangnya sisa-sisa manusia di dalamnya juga merupakan poin yang menarik. Penggalian mengungkapkan fragmen tulang yang terbakar dan tidak terbakar, mewakili segelintir orang, tersebar di sepanjang lorong. Pembangunan Newgrange diperkirakan memakan waktu setidaknya beberapa generasi berdasarkan penanggalan karbon dari material di dalamnya. Mengapa sebuah komunitas kuno menghabiskan begitu banyak usaha untuk membangun makam besar yang sangat terancang hanya untuk menyebarkan fragmen tulang dari beberapa orang yang telah meninggal di lorongnya? Jauh lebih masuk akal jika struktur megalitikum kuno yang sangat kedap air ini justru dibangun sebagai tempat berlindung manusia untuk melindungi orang-orang selama bencana alam berulang di Bumi.

Di Huelva, Spanyol bagian selatan, contoh serupa adalah Dolmen de Soto (Gambar \ref{fig:9}), salah satu dari sekitar 200 situs serupa di daerah tersebut \cite{72,32}. Ini adalah struktur yang ramping dan sangat terancang yang dibangun menggunakan batu megalitikum dan memiliki diameter 75 meter. Dilaporkan, hanya delapan jasad yang ditemukan saat penggalian, semuanya dikuburkan dalam posisi janin.

\section{Penyebutan Anomali Penting}

Pada bagian ini, saya secara singkat menyebutkan beberapa anomali penting lainnya, yang semuanya dijelaskan dengan baik oleh bencana mirip ECDO.

\subsection{Anomali Biologis}

\begin{figure}[b]
\begin{center}
% \fbox{\rule{0pt}{2in} \rule{0.9\linewidth}{0pt}}
   \includegraphics[width=1\linewidth]{bottleneck.jpg}
\end{center}
   \caption{Sebuah bottleneck genetik yang mewakili pemusnahan 95\% laki-laki sekitar 6.000 tahun yang lalu \cite{62}.}
\label{fig:10}
\label{fig:onecol}
\end{figure}

Beberapa anomali biologis yang penting adalah bottleneck genetik dan fosil paus di daratan. Zeng et al. (2018) memodelkan 125 sekuens kromosom-Y dari manusia modern, dan berdasarkan kesamaan serta mutasi DNA, mengidentifikasi adanya bottleneck yang menyebabkan penurunan populasi laki-laki sebesar 95\% sekitar 5.000 hingga 7.000 tahun yang lalu (Gambar \ref{fig:10}) \cite{62}. Fosil paus telah ditemukan ratusan meter di atas permukaan laut, di Swedenborg, Michigan, Vermont, Kanada, Chili, dan Mesir \cite{63,64,65,66}. Fosil-fosil paus ini ditemukan dalam kondisi bervariasi: terawetkan dengan sempurna, di rawa-rawa di atas deposit glasial, atau terkubur dalam sedimen. Jumlah spesimen di lokasi-lokasi ini berkisar dari beberapa hingga lebih dari seratus. Paus adalah makhluk laut dalam dan jarang mendekat ke pantai. Bagaimana mungkin paus-paus ini berakhir di ketinggian seperti itu, sering kali sangat jauh dari laut?

Banyak kepunahan massal telah terjadi dalam sejarah Bumi, di mana yang paling banyak dipelajari adalah “Lima Besar” peristiwa Fanerozoikum: kepunahan massa Ordovisium Akhir (LOME), Devon Akhir (LDME), akhir Permian (EPME), akhir Trias (ETME), dan akhir Kapur (ECME) \cite{88,89}. Menariknya, beberapa dari kepunahan ini diklasifikasikan terjadi pada periode sejarah yang sama dengan banyak lapisan Grand Canyon, yaitu, lapisan Permian dan Devon.

\subsection{Anomali Fisik}

\begin{figure}[b]
\begin{center}
% \fbox{\rule{0pt}{2in} \rule{0.9\linewidth}{0pt}}
   \includegraphics[width=1\linewidth]{columbia.jpg}
\end{center}
   \caption{Riak arus besar di Danau Glasial Columbia, negara bagian Washington \cite{80}.}
\label{fig:11}
\label{fig:onecol}
\end{figure}

Ada banyak lanskap selain Grand Canyon yang kemungkinan besar terbentuk melalui kekuatan kataklismik. Bukti aliran air kontinental besar dapat ditemukan pada riak arus raksasa di seluruh dunia. Salah satu contohnya adalah Channeled Scablands di Pacific Northwest. Di sini, kita tidak hanya dapat melihat lanskap endapan sedimen dan bongkahan batu besar, tetapi juga lebih dari seratus deret riak besar yang terbentuk dari aliran arus besar \cite{78,79}. Ini adalah versi berskala lebih besar dari riak yang terbentuk di dasar pasir sungai. Riak arus besar semacam ini dapat ditemukan di seluruh dunia di Prancis, Argentina, Rusia, dan Amerika Utara \cite{81}. Gambar \ref{fig:11} menggambarkan beberapa riak tersebut di negara bagian Washington di Amerika Serikat \cite{80}.
\begin{figure}[b]
\begin{center}
% \fbox{\rule{0pt}{2in} \rule{0.9\linewidth}{0pt}}
   \includegraphics[width=1\linewidth]{zhangjiajie.jpg}
\end{center}
   \caption{Pilar batu besar di Hutan Nasional Zhangjiajie, Tiongkok selatan.}
\label{fig:12}
\label{fig:onecol}
\end{figure}

\begin{figure}[b]
\begin{center}
% \fbox{\rule{0pt}{2in} \rule{0.9\linewidth}{0pt}}
   \includegraphics[width=1\linewidth]{hoy.jpg}
\end{center}
   \caption{Pilar laut Old Man of Hoy, Skotlandia \cite{83}.}
\label{fig:13}
\label{fig:onecol}
\end{figure}
Inland erosion structures are also well-explained by an ECDO-like Earth flip. Southern China is a great example of massive karst landscapes, formed through water erosion \cite{82}. These landscapes include tower karst, pinnacle karst, cone karst, natural bridges, gorges, large cave systems, and sinkholes. One of the most striking of these is the Zhangjiajie National Forest, which contains massive quartz sandstone pillars (Figure \ref{fig:12}) \cite{84}. These pillars stand at an average elevation of over 1,000 meters and number more than 3,100. More than 1,000 of them soar above 120 meters tall, and 45 reach over 300 meters \cite{85}. These pillars resemble sea erosion pillars (Figure \ref{fig:13}), which are coastal rock pillars formed by the collapse of surrounding material due to ocean waves. Similar erosion landscapes can be found in the rock cones of Urgup, Turkey, as well as Ciudad Encantada, Spain, which are both over 1,000 meters above sea level. All these locations have some combination of salt and oceanic marine fossils in close proximity to them, suggesting past marine incursions \cite{15,86,87}. Of course, the flood stories \cite{3} mention the ocean going much higher than 1,000 meters, and this is verified by the presence of saltwater and massive salt flats in the Andes and Himalayas several kilometers above sea level. The Uyuni salt flat in Bolivia, for example, reaches 3653 meters above sea level \cite{94}.

\subsection{Peristiwa Perubahan Iklim yang Cepat}

Literatur ilmiah modern mengenali keberadaan peristiwa perubahan iklim global yang cepat dalam sejarah terbaru Bumi. Dua contoh penting adalah peristiwa 4,2 ribu tahun dan 8,2 ribu tahun, yang keduanya bertepatan dengan penurunan populasi dan gangguan pemukiman masyarakat di wilayah geografis yang luas. Peristiwa ini terekam sebagai anomali dalam sedimen dan inti es, fosil karang, nilai isotop O18, catatan serbuk sari dan speleotem, serta data tingkat laut. Perubahan iklim yang disimpulkan mencakup penurunan suhu global yang cepat, pengeringan, gangguan arus balik meridional Atlantik, dan kemajuan gletser \cite{90,91,92}. Peristiwa 8,2 ribu tahun secara khusus bertepatan dengan kemungkinan banjir air asin yang dramatis di Laut Hitam sekitar tahun 6400 SM \cite{93}.

\subsection{Anomali Arkeologis}

Bukti arkeologis dari beberapa kota kuno menunjukkan banyak lapisan yang melibatkan penguburan dan kehancuran, menciptakan catatan peristiwa bencana di masa lalu. Kota kuno Yerikho adalah salah satu contohnya, terletak di Palestina modern. Kota ini memiliki beberapa lapisan kehancuran, dengan runtuhnya struktur batu dan kebakaran hebat \cite{96,97}. Kronologi yang terekam dalam lapisan-lapisannya bermula dari sekitar 9000 SM hingga 2000 SM. Secara khusus menonjol adalah towernya, yang tampaknya telah terpotong dan terkubur dalam sedimen sekitar tahun 7400 SM (Gambar \ref{fig:14}) \cite{95}. Catal Huyuk \cite{99}, Gramalote \cite{98}, dan istana Minoa di Knossos, Kreta \cite{100,101} adalah contoh serupa situs arkeologi yang memiliki banyak lapisan, sering kali mengandung bukti kehancuran.

\begin{figure}[t]
\begin{center}
% \fbox{\rule{0pt}{2in} \rule{0.9\linewidth}{0pt}}
   \includegraphics[width=1\linewidth]{jericho.jpg}
\end{center}
   \caption{Rekonstruksi arkeologis penguburan Menara Yerikho sekitar tahun 7400 SM \cite{95}.}
\label{fig:14}
\label{fig:onecol}
\end{figure}

Another piece of evidence for major cataclysms disrupting human civilization is the Nampa Image, a clay doll found beneath approximately 100 meters of lava in Idaho \cite{102,103}. The lava flow under which the figurine was found was estimated to be deposited during the Late Tertiary or early Quaternary period, supposedly being 2 million years old. However, the lava flow in the region appears to be relatively fresh. Such finds not only point to major civilization-destroying cataclysms, but also call into question modern dating chronologies.

\section{Mengenai Metode Penanggalan Modern}

Ada alasan signifikan untuk bersikap skeptis terhadap kronologi modern, yang menetapkan usia sangat panjang jutaan, bahkan hingga ratusan juta tahun untuk berbagai material fisik.

Narasi konvensional menyatakan bahwa apa yang disebut "bahan bakar fosil" seperti batu bara, minyak, dan gas alam berusia ratusan juta tahun \cite{104}. Namun, penanggalan karbon nyata dari minyak di Teluk Meksiko menemukan usia sekitar 13.000 tahun untuk minyak tersebut \cite{105}. Karbon-14 memiliki waktu paruh yang sangat singkat (5.730 tahun) sehingga seharusnya benar-benar terurai setelah beberapa ratus ribu tahun. Namun, karbon-14 telah ditemukan dalam batu bara dan fosil yang seharusnya seribu kali lebih tua \cite{106}. Faktanya, batu bara buatan telah diproduksi di laboratorium di bawah kondisi terkendali, terutama dengan panas tinggi, hanya dalam waktu 2-8 bulan \cite{107}.

Metode penanggalan radioisotop selain penanggalan karbon juga mungkin tidak akurat. Kelompok riset Answers in Genesis menemukan inkonsistensi pada tanggal yang dihasilkan oleh metode tersebut sehingga kredibilitasnya dipertanyakan \cite{108}. Jaringan lunak yang mengandung sel darah, pembuluh, dan kolagen bahkan telah ditemukan dalam sisa-sisa dinosaurus yang seharusnya berusia seratus juta tahun \cite{109,110}. Berdasarkan apa yang kita ketahui, sangat mungkin bahwa usia yang diterima secara konvensional untuk skala waktu geologi Bumi dan material fisik seperti batuan dan bahan bakar fosil mungkin salah sangat jauh dari kenyataannya.

\section{Kesimpulan}

Dalam makalah ini, saya telah memaparkan anomali-anomali paling meyakinkan yang menunjukkan asal-usul bencana dan paling baik dijelaskan oleh ECDO Earth flip. Meskipun beragam, koleksi yang disajikan ini belum lengkap - lebih banyak anomali telah dikompilasi dan tersedia secara publik di repositori GitHub penelitian saya \cite{2}.

\section{Ucapan Terima Kasih}

Terima kasih kepada Ethical Skeptic, penulis asli dari tesis ECDO, atas penyelesaian tesisnya yang penuh wawasan dan terobosan serta membagikannya kepada dunia. Tesis trilogi miliknya \cite{1} tetap menjadi karya otoritatif untuk teori Exothermic Core-Mantle Decoupling Dzhanibekov Oscillation (ECDO), dan memuat jauh lebih banyak informasi tentang topik ini daripada yang telah saya ringkas secara singkat di sini.

Dan tentu saja, terima kasih kepada para raksasa yang bahunya kita pijak; mereka yang telah melakukan semua penelitian dan penyelidikan yang memungkinkan karya ini dan telah berupaya membawa pencerahan bagi umat manusia.
{\small
\renewcommand{\refname}{References}
\bibliographystyle{ieee}
\bibliography{egbib}
}

\end{document}
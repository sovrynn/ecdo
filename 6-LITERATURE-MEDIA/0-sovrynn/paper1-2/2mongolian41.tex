\documentclass[10pt,twocolumn,letterpaper]{article}

% Миний өөрийн зүйлс
\usepackage{booktabs}
% \usepackage{caption}
% \captionsetup[table]{skip=8pt}   % Зөвхөн хүснэгтүүдэд нөлөөлнө
\usepackage{stfloats}  % Үүнийг урьдчилсан хэсэгт нэмнэ үү

% \usepackage{fontspec}
\usepackage[english]{babel}

% load Lao via babelprovide, turn on "onchar=ids" for automatic shaping
\babelprovide[import,onchar=ids fonts]{mongolian}

% main (rm) font for Latin
\babelfont{rm}{Noto Serif}
% alternate (sans-serif) font for Latin
\babelfont{alt}{Lato}

% Lao text in Noto Serif Lao at 1.2× scale
\babelfont[mongolian]{rm}{Noto Serif}
\babelfont[mongolian]{sf}{Noto Serif}
% Lao text in Noto Serif Lao for the alt family too
\babelfont[mongolian]{alt}{Noto Serif}

\usepackage{cvpr}
\usepackage{times}
\usepackage{epsfig}
\usepackage{graphicx}
\usepackage{amsmath}
\usepackage{amssymb}

% Бусад багцуудыг энд оруулна уу, hyperref-ээс өмнө.

% Хэрвээ та hyperref-г тайлж, дараа нь буцаан идэвхжүүлвэл,
% egpaper.aux-г устгах хэрэгтэй, дараа нь latex-г дахин ажиллуулна уу. (Эсвэл эхний latex 
% ажиллуулах үед 'q'-г дарж болно, дуустал нь хүлээгээд, дараа нь асуудалгүй болно).

\usepackage[breaklinks=true,bookmarks=false]{hyperref}

\cvprfinalcopy % *** Энд финал хувилбарт энэ мөрийг комментоос гаргана уу

\makeatletter
\def\cvprsubsection{\@startsection {subsection}{2}{\z@}
    {8pt plus 2pt minus 2pt}{6pt}{\bfseries\normalsize}}
\makeatother

\def\cvprPaperID{****} % *** CVPR-ний өгүүллийн дугаарыг энд оруулна уу
\def\httilde{\mbox{\tt\raisebox{-.5ex}{\symbol{126}}}}

% Зөвхөн илгээх горимд хуудасны дугаарлалт хийгдэнэ, эцсийн хувилбарт дугааргүй байна
%\ifcvprfinal\pagestyle{empty}\fi
\setcounter{page}{1}
\begin{document}

%%%%%%%%% ГАРЧИГ
\title{ECDO Мэдээлэлд суурилсан гарын авлага 2/2: ШУ, Түүхийн Гаж Явдлуудыг ECDO “Дэлхийн Эргэлтээр” Тайлбарлах Судалгаа}

\author{Жунхо\\
2025 оны хоёрдугаар сард нийтлэв\\
Вэбсайт (Эндээс өгүүллийг татаж авна уу): \href{https://sovrynn.github.io}{sovrynn.github.io}\\
ECDO судалгааны репо: \href{https://github.com/sovrynn/ecdo}{github.com/sovrynn/ecdo}\\
{\tt\small junhobtc@proton.me}
}

\maketitle
%\thispagestyle{empty}

\begin{abstract}
2024 оны тавдугаар сард “Ёс зүйн Скептик” нэртэй танигдаагүй онлайн зохиогч \cite{0} “Экзотермик Цөм-Бүрхүүлийн Салалтын Джаанибековын Үелзэл” (ECDO) \cite{1} хэмээх хувьсгалт онолыг нийтэлсэн. Энэхүү онолд Дэлхий өмнө нь эргэлтийн тэнхлэгийн огцом, сүйрлийн шилжилтэд орж, эргэлтийн инерцийн улмаас далай тэнгисүүд эх газрыг дээгүүр даван цутгаж, аварга их үер бий болгож байсныг дэвшүүлээд зогсохгүй, мөн ийм “эргэлтийн солилт” дахин ойртсоор байгааг илтгэх өгөгдөл болон түүнийг үүсгэх геофизикийн тайлбарлагдсан шалтгаант процессыг санал болгож буй. Ийм гамшгийн үер болон дэлхийн сүйрлийн таамаглал шинэ зүйл биш ч, шинжлэх ухааны, орчин үеийн, олон салбарт тулгуурласан, өгөгдөлд суурилсан арга барилаар ECDO онол нь өөрийн гэсэн онцлог, үнэмшилтэй.

Энэхүү судалгааны өгүүлэл нь сүүлийн 6 сарын хугацаанд бие даан хийсэн судалгааны \cite{2,20} ECDO онолтай холбоотой хоёр хэсгээс бүрдэх товч тайлангийн хоёр дахь хэсэг бөгөөд хүрээлэн буй шинжлэх ухаан болон түүхийн хамгийн сайн тайлбарлагдах онцгой үзэгдлийг гамшгийн шинж чанартай ECDO “Дэлхийн эргэлтийн эргэлт”-ийн хүрээнд авч үзнэ.

\end{abstract}

\section{Танилцуулга}

Орчин үеийн жигд өөрчлөлтийн геологи, түүх нь Гранд Каньон зэрэг голлох геологийн тогтоц хэдэн сая жилийн турш бүрэлдсэн \cite{143}; Калифорнийн Үхлийн Хөндийд давс байдаг нь хэдэн зуун сая жилийн өмнө далайн ёроолд оршиж байсантай холбоотой \cite{144}; 150 үеийн өмнөх манай өвөг дээдэс амьдралынхаа бүх хугацааг аварга том бунхан барихад зарцуулсан \cite{29,70}; мөн “шатах ашигт малтмал”-ууд хэдэн зуун сая жилийн настай \cite{104} гэж сургадаг. Хамгийн сонирхолтой нь хүн төрөлхтөн 300,000 жилийн настай гэж үздэг хэдий ч, бичигдсэн түүх, иргэншил ердөө 5,000 жил буюу 150 хүний үе л үргэлжилсэн байдаг.

Дараа нь авч үзэх эдгээр хачирхалтай үзэгдлүүдийг гамшгийн шинж чанартай геологийн хүчээр хамгийн сайн тайлбарлаж болно.

\section{Шууд Хөлдсөн Хөх Мангас Шаварт Булагджээ}

\begin{figure}[t]
\begin{center}
% \fbox{\rule{0pt}{2in} \rule{0.9\linewidth}{0pt}}
   \includegraphics[width=1\linewidth]{jarkov-mammoth.jpg}
\end{center}
   \caption{Жарковын мамонт, 20,000 жилийн өмнөх мөстэй шаварт бүрэн хадгалагдсан Сибирийн мамонт \cite{51}.}
\label{fig:1}
\label{fig:onecol}
\end{figure}

Ийм төрлийн гаж үзэгдлийн нэг нь Арктикийн бүс нутагт ихэвчлэн олддог шаварт булагдсан, бүрэн хадгалагдсан хурдан хатсан мамонт юм (Зураг \ref{fig:1}). Березовкагийн мамонт нь Сибирьт шавар ихтэй хайрганд булсан байдлаар илэрсэн бөгөөд тэр үеийнхээ дараа хэдэн мянган жилийн турш мах нь иддэг байсан төгс хадгалагдсан байжээ. Мөн ам болон ходоодондоо ургамлын хоол идэж байсан нь нэн гайхалтай бөгөөд энэхүү амьтан үхэхээсээ өмнө цэцэглэсэн ургамлаар идэш тэжээл хийж байхад ингэж хурдан царцах боломжийг эрдэмтэд гайхаж байна \cite{17}. Мэдээлснээр, \textit{"1901 онд Березовка голын ойролцоо бүтэн мамонтын сэг олдсон нь шуугиан тарьсан бөгөөд энэ амьтан зун дунд хүйтнээс үхсэн бололтой. Түүний ходоодны агууламж сайн хадгалагдаж, дотор нь цэцэг, зэрлэг вандуй байжээ: энэ нь долдугаар сарын сүүлээр эсвэл наймдугаар сарын эхээр идсэн байх ёстойг харуулна. Амьтан тийм гэнэт нас барсан тул амандаа хүртэл өвс, цэцгээр дүүрэн байлаа. Түүнийг асар их хүчтэй ямар нэгэн зүйл орон нутгаас нь хэдэн милийн зайд шидэж орхисон нь илт байв. Таз яс болон нэг хөл нь хугарсан — аварга том амьтныг өвдөг дээр нь унагаж, дараа нь хүйтэнд үхсэн бололтой, энэ бол жил дэх хамгийн халуун үе юм"} \cite{18}. Мөн, \textit{"[Оросын эрдэмтэд] амьтны ходоодны дотоод хана хүртэл бүтцийн хувьд бүрэн хадгалагдсан болохыг тэмдэглэсэн нь, амьтны биеийн дулааныг байгаль дахь ямар нэгэн ер бусын хурдтай үйл явц бүрэн авч гарсан болохыг харуулжээ. Сандерсон энэ онцлогийг онцлон авч, асуудлыг Америкийн Хөлдүү Хүнсний Институтэд тавьсан: Бүхэл бүтэн мамонтыг түүний биеийн хамгийн дотоод хэсгийн чийг хир удаан хөлдөж талст бий болж, махны бүтэцийг эвдэхгүй байхаар хэр хурдан цэвдэг болгох шаардлагатай вэ?... Хэдэн долоо хоногийн дараа Институт Сандерсонд хариу ирүүлэв: Энэ бол огт боломжгүй. Бидний бүх шинжлэх ухаан, инженерийн мэдлэгээр, ийм том амьтны сэгийн дулааныг тийм хурдан авч гарч, маханд их хэмжээний чийгийн талст үүсэхгүйгээр хөлдөөх боломж байхгүй. Түүнчлэн, шинжлэх ухаан болон инженерчлэлийн бүх аргыг туршсаны эцэст тэд байгальд ч ийм үйл явц байхгүй гэдэг дүгнэлтэд хүрсэн"} \cite{19}.

\section{Их Хөндий}

Их Хөндий, Хойд Америкийн баруун өмнөд хэсгийн Их сав газрын нэг хэсэг, сүйрлийн гарал үүсэлтэй байж болох бас нэгэн байгалийн үзэгдэл юм (Зураг \ref{fig:2}). Юуны түрүүнд, Их Хөндийг бүрдүүлдэг тунамал элс, шохойн давхаргууд маш том 2.4 сая км$^2$ талбайд тархсан байдаг \cite{21}. Зураг \ref{fig:3} нь АНУ-ын баруун хэсэгт байрлах Кокононийн элс чулуун давхаргын хүрээг үзүүлжээ. Ийм их хэмжээний, ижил бүтэцтэй ус хэвтээ давхаргуудыг нэг дор үүсгэсэн байх боломжтой.

\begin{figure}[b]
\begin{center}
% \fbox{\rule{0pt}{2in} \rule{0.9\linewidth}{0pt}}
   \includegraphics[width=1\linewidth]{grand-canyon.jpg}

\end{center}
   \caption{Гранд Каньон, Аризона, АНУ \cite{49}.}
\label{fig:2}
\label{fig:onecol}
\end{figure}

\begin{figure}[t]
\begin{center}
% \fbox{\rule{0pt}{2in} \rule{0.9\linewidth}{0pt}}
   \includegraphics[width=1\linewidth]{coconino.jpg}
\end{center}
   \caption{АНУ-ын баруун хэсэгт орших Коконино элсэн чулууны үеийн хэмжээ \cite{21}.}
\label{fig:3}
\label{fig:onecol}
\end{figure}

Гранд Каньоныг илүү анхааралтай ажиглахад эдгээр өргөн цар хүрээтэй тунамал давхаргууд давхар хурдаслагдан хуримтлагдах явцад мөн чухал тектоник хүчүүд зэрэгцэн үйлчилж байсныг бидэнд хэлдэг. Үүнийг ойлгохын тулд Каньоны зарим хэсэгт элсэн үеүүд нугалсан болон ил гарсан байдлыг анхааралтай ажиглах хэрэгтэй. "Answers in Genesis"-ийн судлаачид \cite{42} эдгээр нугалаасуудын заримын, жишээ нь Monument Fold-ийн чулуулгийн дээжийг микроскопоор шинжилж, хэрэв уг нугалаасууд удаан хугацаанд халуун ба даралтын нөлөөгөөр үүссэн бол байх ёстой шинж чанарууд илрээгүй тул тунамал давхаргууд анхласан даруйдаа буюу зөөлөн байх үедээ тектоник хүчний нөлөөгөөр нугалсан гэж дүгнэжээ \cite{43}.

\begin{figure*}
\begin{center}

% \fbox{\rule{0pt}{2in} \rule{.9\linewidth}{0pt}}
\includegraphics[width=1\textwidth]{Grand_Staircase-big.jpg}
\end{center}
   \caption{Их хавцалын үелсэн тунамал чулуулагийн давхаргууд (зурагны баруун талд) хойд зүг рүү Сидар Брэйкс, Юта (зурагны зүүн талд) хүртэл шууд үргэлжилдэг бөгөөд тэнд бүгд дээш өргөгдсөн байдлаар харагдана \cite{50}.}
\label{fig:4}
\end{figure*}

Өргөн хүрээнд авч үзвэл, Их хавцлыг бүрдүүлж буй давхаргууд зөвхөн хавцлын доторхи нумарсан бүрхүүл төдийхнөөр хязгаарлагдаагүй байна. Энэ давхаргууд зүүн тийш Кайбабын нэг талаар (East Kaibab Monocline) нумарсан \cite{46}, мөн хойд зүгт Сидар Брэйкс, Ютад ч гэсэн (Зураг \ref{fig:4}) ингэж нугарсан байдаг. Энэ нь эдгээр давхаргуудыг хоорондоо хурдан дараалан үелж үүссэний дараа нэгэн зэрэг эвхэгдсэн байж болзошгүйг харуулж байна. Жишээ нь, Их хавцлын хэвтээ давхаргуудын нийт зузаан нь ойролцоогоор 1700 метр байдаг. Нэг милийн зузаантай тунамал давхаргуудыг үүсгэхэд шаардагдах геологийн процессын цар хүрээ асар их юм.

Их хавцлын жинхэнэ үүсэл нь орчин үеийн геологийн шинжлэх ухааны маргааны нэг сэдэв юм. Уламжлалт (Uniformitarian) геологийн үүднээс бол Их хавцалыг Колорадо мөрөн олон сая жилийн туршид сийлсэн гэж үздэг \cite{47}. Харин “Answers in Genesis” судалгааны багийнхан Их хавцал нь эртний нуурын эргийг сэтэлж гарсан асар их усны урсгалын нөлөөгөөр хэдхэн долоо хоногийн дотор үүссэн байх магадлалтай гэж үздэг бөгөөд үүний улмаас их хэмжээний тунамал чулуулаг нэг мөчид зайлуулагдсан гэж үздэг. Их хавцлын зүүн хэсэгт орших өндөрлөг байрлалтай нуур байсан шинж тэмдэг нь нуурын тунамал хуримтлал болон далайн гаралтай чулуужсан олдворуудаар батлагдаж байна. Их хавцлыг бусад ижил хэмжээний усны түрлэгийн эвдрэлээс үүдэлтэй байгууламжууд болох Афтоны хавцал ба Сэнт Хеленс уултай харьцуулахад, газар зүйн хувьд төстэй дүр төрхтэй бөгөөд их хэмжээний усны урсац богино хугацаанд аварга том хавцлуудыг гаргаж чаддагийг харуулдаг \cite{48}.

Тунамал давхаргуудыг ингэж өргөн уудам нутаг дэвсгэр дээр хуримтлуулахад шаардагдах геологийн процессын цар хүрээ, тунамал давхаргууд үүссний дараахан яг зэрэг давхцан тохиолдсон хүчтэй тектоник хөдөлгөөний үр дүн болон Харин Колорадо мөрний өчүүхэн хэмжээ Их хавцлын асар том цар хүрээтэй харьцуулахад, энэ бүхэн Их хавцлын үүслийн явц огтхон ч аажим, удаан хугацаанд хийгдээгүй байх магадлалтайг санал болгож байна.

\section{Деринкуюгийн газар доорх хот}

Пирамидуудаас гадна, эртний инженерчлэлийн гайхалтай жишээ бол Туркийн Каппадоки мужид байрлах Деринкуюгийн газар доорх хот (Зураг \ref{fig:5}) юм. Энэхүү газар доорх хот нь тухайн бүс нутгийн 200 гаруй хоргодох байрны хамгийн том нь бөгөөд \cite{54} нийтдээ 20,000 хүртэлх хүн амьдарч байсан гэж тооцогддог, 18 давхар бүхий, 85 метрийн гүн рүү нэвтэрсэн байгууламж юм. Нас нь тодорхойгүй боловч дор хаяж 2800 жилийн настай гэж үздэг. Энэ хотыг зөөлөн галт уулын чулуулгийг ухаж бүтээжээ \cite{52, 53}.

\begin{figure}[b]
\begin{center}
% \fbox{\rule{0pt}{2in} \rule{0.9\linewidth}{0pt}}
\includegraphics[width=1\linewidth]{derinkuyu.jpeg}
\end{center}
   \caption{Дэринкую газрын доорх хотын бүдүүвч зураг \cite{56}.}
\label{fig:5}
\label{fig:onecol}
\end{figure}

Дэринкуюг сонирхолтой болгодог шалтгаан нь ямар ч олон нийт бүхэл бүтэн хотыг газар доор байгуулна гэж яагаад шийдсэн нь тодорхой бус байдагт оршдог. Газар доор амьдрах орон зай үүсгэхийн тулд орон бүрийг хад чулуунаас ухаж гаргах шаардлагатай болно. Газрын доорх хонгилуудын ширүүн хэлбэр, бүтэц нь эдгээрийг хүчирхэг багаж хэрэгсэл биш, харин гараар ухсан болохыг илтгэж байгаа бөгөөд энэ нь газар дээрх байр барихаас хэд дахин илүү төвөгтэй ажил байсан гэсэн үг. Үнэндээ, хүн төрөлхтөн газар дээр нь газар тариалан эрхэлж, нарны туяанд хүрч, байгаль, нээлт хийж болох байхад газрын гүнд бүх амьдралаа өнгөрүүлэхийг яагаад хүсэх нь ойлгомжгүй юм. Хэвшмэл “түүх”-ээр бол Дэринкую христийн шашинтнууд шашнаа нууцаар шүтэх аюулгүй газар хэрэгтэй байсан учраас үүссэн гэж үздэг \cite{53}. Гэтэл эрүүл ухаанаар бодвол, дайсантай тулгарахад хамгийн шууд арга нь “тэмцэх эсвэл зугтах” болохоос “газрын чулууг ухаж, газар доорх хот байгуулах” биш гэдгийг хялбархан дүгнэж болно.

Газар доорх хотын хэмжээ, гүн, ухаалаг шийдэл нь энэ байгууламжийг түр зуурын цэргийн хамгаалалтын байгууламж бус, харин гадаргуу дээрх аюулт хүчин зүйлээс урт хугацаанд хамгаалах орогнол болгон бүтээсэн нь илэрхий болгодог. Дэринкуюд энгийн унтлагын өрөө, гал тогоо, ариун цэврийн өрөөнөөс гадна мал амьтны хашаа, усны сав, хүнсний агуулах, дарс, тос шахах төхөөрөмж, сургууль, сүм, оршуулгын газар болон аварга том агааржуулалтын хонгилууд (Зураг \ref{fig:6}) байсан. Цэргийн хоргодох байранд яагаад дарсны шахуурга хэрэгтэй байсан ба яагаад ийм гүн (85 метр) ухаж, ийм нарийн төвөгтэйгээр байгуулсан байх ёстой вэ?

Дэринкуюг үүсгэх хамгийн итгэмжлэгдсэн шалтгаан нь дэлхийн гадаргуу дээр үүсэх сүйрлийн геофизикийн хүчин зүйлээс хамгаалах, урт хугацааны, өөрийгөө хангах байр бэлдэх шаардлага байсан байх магадлалтай.

\begin{figure}[t]
\begin{center}
% \fbox{\rule{0pt}{2in} \rule{0.9\linewidth}{0pt}}
   \includegraphics[width=1\linewidth]{derinkuyu-air.jpg}
\end{center}
   \caption{Дэринкую дахь гүн агааржуулалтын худаг \cite{53}.}
\label{fig:6}
\label{fig:onecol}
\end{figure}

% \section{Дэлхийн Эргэлтээр Хамгийн Сайн Тайлбарлагдах Нэмэлт Онцгой Үзэгдлүүд}

% Дуусахаасаа өмнө, бид катастрофик геофизикийн хүчний үүднээс харвал сайн тайлбарлагдах зарим шинжлэх ухааны нэмэлт онцгой үзэгдлийг дурдана.

\section{Биомассын Хуримтлал}

Амьтан, ургамлын янз бүрийн холимог биомасс, ихэвчлэн тунамал давхаргад чулуужин олддог нь өөр нэг учир нь тайлагдаагүй оньсого юм. "Reliquoæ Diluvianæ" зохиолд, Уильям Баклэнд пастор Британи болон Европ даяар, ямар ч ойлгомжтой шалтгаангүйгээр хамтдаа олдсон олон төрлийн амьтдын олдворуудыг тунамал "дилувиум" давхаргад булаастайгаар тайлбарласан байдаг \cite{58}. Ийм амьтны үлдэгдлийн холимог нь бас Норвегийн Валдрой арал дахь Скёнхеллерен агуйд илэрсэн. Энэ агуйд 7,000 гаруй хөхтөн амьтан, шувуу, загасны яс олон тунамал давхаргад хоорондоо холилдсон байдалтай олдсон \cite{59}. Өөр нэг жишээ нь Итали дахь Сан Чиро, буюу "Аварга биетнүүдийн агуй" юм. Тус агуйд, хэд хэдэн тонн хөхтөн амьтдын яс, голдуу усны одосны яс, маш шинэ хэвээрээ байж гоёл чимэглэл болгон зүсэж, дэнгийн нүүрс хийхээр гаргадаг байжээ. Янз бүрийн амьтдын яс нь хоорондоо холилдож, эвдэрч, бутарч, хэлтэрхий болон тараагдсан байсан гэж мэдээлсэн \cite{60,61}. Хуучин Мендес, Египетэд, янз бүрийн амьтдын үхдэл шилэн (шил мэт болсон) шаварт холилдон олдсон \cite{57}. Ийм олдворууд анх харахад гайхалтай санагдаж болох ч, үерийн үлэмж давалгаанд үхсэн амьтдыг тунамал давхаргад холилдуулан дарж, зарим нь агуйд шууд оруулж булсан эсвэл амьдаар нь булсан тайлбараар амархан ойлгогдоно. Харин Египетийн шилэрсэн биомассын хувьд, үерийн дараах цөм-манти давхаргын шилжилтээс үүдэлтэй асар их цахилгаан цэнэглэлтээс болсон байж болох юм. Зураг \ref{fig:7} дээр Аляскийн биомасс "шавар"-ын ил харагдацын жишээ үзүүлсэн байна \cite{56}.

\begin{figure}[t]
\begin{center}
% \fbox{\rule{0pt}{2in} \rule{0.9\linewidth}{0pt}}
   \includegraphics[width=1\linewidth]{muck-crop.jpeg}
\end{center}
   \caption{Аляскийн "шавар"─мод, ургамал, амьтдын үймээнтэй холилдсон хэлтэрхий нь хөлдүү шавар ба мөсөнд агуулагдсан байдалтай \cite{146}.}
\label{fig:7}
\label{fig:onecol}
\end{figure}
\section{Эртний бункерууд}

Бидний өвөг дээдэс олон өндөр инженерчлэлтэй эртний байгууламжуудыг үлдээсэн бөгөөд эдгээрээс хүний ясан олддог. Ихэвчлэн эдгээрийг тансаг булш гэж тайлбарладаг ч, нарийвчлан харвал эдгээр нь үнэндээ эртний бункер байж болох шинжтэй.

\begin{figure}[b]
\begin{center}
% \fbox{\rule{0pt}{2in} \rule{0.9\linewidth}{0pt}}
   \includegraphics[width=1\linewidth]{ww19.jpg}
\end{center}
   \caption{Шинэ Грийнж, Ирланд - орох хаалганы дэргэдэх зочдыг харж хэмжээ харьцуул.}
\label{fig:8}
\label{fig:onecol}
\end{figure}

Нэгэн онцлох жишээ бол Шинэ Грийнж (Зураг \ref{fig:8}), Бру на Бойнн цогцолборын гол дурсгал, энэ цогцолборт т.н. хонгилтой булш зэрэг эртний байгууламжууд багтдаг. Эдгээр булш нь нэг эсвэл хэд хэдэн булшны өрөөнөөс бүрдэх ба тэдгээрийг хөрс эсвэл чулуугаар хучиж, их чулуугаар барьсан нарийн орц гарцтай байдаг \cite{70}. Энэ нь олон үеэр бүтээгдсэн цогцолбор бүтэц бүхий хамгаалалттай бүтэц байсны жишээ бөгөөд олон хүний хүчин чармайлтаар хэдхэн хүнийг оршуулах гэж байгуулсан, тухайн булш баригдаж байх үед эдгээр хүмүүс амьд байгаагүй ч байж магадгүй. 1699 онд орон нутгийн газар эзэмшигч олж илрүүлэх үед газар доор булшлагдсан байжээ.

Бүтэцийг өнгөц харвал байгуулахад асар их хөдөлмөр зарцуулсан нь илхэн — Шинэ Грийнж нь ойролцоогоор 200,000 тонн материалаас бүтдэг. Дотроо, \textit{“…танхимтай хонгил байх бөгөөд баруун өмнөд талаас орж болох нэвтрэх гарцтай. Хонгил нь 19 метр (60 фут), эсвэл бүтцийн төв хүртэлх замын ойролцоогоор гуравны нэгтэй тэнцэнэ. Хонгилын төгсгөлд төв том танхимтай гурван жижиг өрөө байх бөгөөд корбелл таазан дээврээр өндөрлөн хийгдсэн… Энэ хонгилын ханыг ортостат гэдэг том чулуугаар барьсан, баруун талд хорин хоёр, зүүн талд хорин нэг бий. Эдгээрийн дундаж өндөр 1½ метр”} \cite{70}. Мөн уснаас хамгаалсан нарийн инженерийн шийдэл бий. Жишээ нь, дээвэрт, \textit{“Дээврийн завсрыг шатаасан хөрс ба далайн элсний холимогоор чигжиж, ус нэвтрэхгүй болгосон бөгөөд энэ холимогоос авсан хоёр радиокарбон огноо булшны бүтэц 2500 МЭӨ үед хамаарахыг харуулсан"} \cite{71}. Нэмж дурдахад, дотор өрөө рүү хүрэх өндрийн ялгаа нь мөн адил зориулалтаар шийдсэн байх магадлалтай: \textit{“Булшны хонгил ба өрөөний шал нь байгууламж барьсан толгойн налуудыг дагасан тул, орц, дотор өрөө хоёрын шал хоёрын хооронд бараг 2 метрийн ялгаа бий”} \cite{71}.

\begin{figure}[b]

\begin{center}
% \fbox{\rule{0pt}{2in} \rule{0.9\linewidth}{0pt}}
   \includegraphics[width=1\linewidth]{dolmen.jpg}
\end{center}
   \caption{Испанийн Сото Долмен \cite{53}.}
\label{fig:9}
\label{fig:onecol}
\end{figure}

Дотор нь хүний чулуужсан яс бага байгаа нь бас сонирхолтой юм. Малталтаар шатаагдсан болон шатаагдаагүй ясны хэлтэрхийнүүд, хэд хэдэн хүнийг төлөөлөн, хонгилын дагуу энд тэндгүй тарсан байжээ. Ньюгренжийн байгууламжийг бүтээснийг доторх материалын нүүрстөрөгчийн огноогоор хэд хэдэн үе үргэлжилсэн гэж тооцдог. Эртний нэгэн нийгэм яагаад ийм их хүч хөдөлмөр гарган том хэмжээний, маш нарийн бүтэцтэй булш босгоод ердөө цөөхөн хүний ясны хэлтэрхийг л хонгилоор тараан байрлуулсан байж болох вэ? Эдгээр эртний, маш нямбайгаар усны нэвчилтээс хамгаалсан мегалит байгууламжийг үнэндээ олон дахин давтагдсан дэлхийн гамшгийн үеэр хүмүүсийг хамгаалах байр болгон барьсан нь илүү магадлалтай байж болох юм.

Испанийн өмнөд хэсэгт орших Уэльвад төстэй нэгэн жишээ бол Сото Долмен (Зураг \ref{fig:9}) бөгөөд энэ бүс нутагт ийм 200 орчим байгууламж байдгийн нэг нь юм \cite{72,32}. Энэ нь мегалит чулуунуудыг ашиглан маш инженерчлэлтэй, урсгасан хэлбэртэй бүтэцтэй бөгөөд 75 метр диаметртэй. Малтлагаар ердөө найман цогцос гарсан бөгөөд бүгдийг нь ургийн байрлалтайгаар оршуулсан байжээ.

\section{Онцлох Гаж Явдлуудын Дурдах}

Энэ хэсэгт, би зарим нэг илүү онцлог гаж үзэгдлийг товч дурьдах бөгөөд эдгээр нь бүгд ECDO-той төстэй гамшгаар бүрэн сайн тайлбарлагдаж байна.

\subsection{Биологийн Гаж Явдлууд}

\begin{figure}[b]
\begin{center}
% \fbox{\rule{0pt}{2in} \rule{0.9\linewidth}{0pt}}
   \includegraphics[width=1\linewidth]{bottleneck.jpg}
\end{center}
   \caption{Генетикийн бөглөрөл нь 6,000 жилийн өмнө эрэгтэйчүүдийн 95\% нь устгагдсан үзэгдлийг илэрхийлж байна \cite{62}.}
\label{fig:10}
\label{fig:onecol}
\end{figure}

Зарим онцгой биологийн гаж үзэгдлүүдэд генетикийн бөглөрөл болон эх газарт олдсон халимны чулуужмалууд орно. Zeng нар (2018) орчин үеийн хүмүүсийн 125 Y-хромосомын дарааллыг судалж, ДНХ-ийн ижил төстэй байдал ба мутацид үндэслэн, 5,000-аас 7,000 жилийн өмнө эрэгтэйчүүдийн популяцид 95\%-ийн цөөрөл бий болсон бөглөрлийг илрүүлсэн (Зураг \ref{fig:10}) \cite{62}. Халимны чулуужмалууд далайн түвшнээс хэдэн зуун метрийн өндөрт, Шведенборг, Мичиган, Вермонт, Канад, Чили, Египет зэрэг газруудад олдсон байна \cite{63,64,65,66}. Эдгээр халимнууд янз бүрийн нөхцөлд: бүрэн бүтэн хадгалагдсан, мөсөн үеийн давхаргын дээр намагт хэвтэж байсан буюу хөрсөнд булшлагдсан байдалтайгаар олдсон байна. Эдгээр газруудад байгаа чулуужмалын тоо хэдээс хэдэн зуу хүрдэг. Халимнууд бол далайн гүн усны амьтад бөгөөд эрэг рүү бараг ойртдоггүй. Эдгээр халим яаж ийм өндөрлөг газар, ихэвчлэн эх газраас хол зайнд очсон юм бол?

Дэлхийн өнгөрсөн түүхэнд олон удаагийн бөөн устгалууд болсон бөгөөд хамгийн их судлагдсан нь Фанерозойн "Их Тав"-ын (Big Five) массын устгалууд: Ордовикийн сүүлийн үеийн (LOME), Девоны сүүлийн үеийн (LDME), Пермийн төгсгөлийн (EPME), Триасын төгсгөлийн (ETME), Цэрдийн төгсгөлийн (ECME) массын устгалууд юм \cite{88,89}. Сонирхолтой нь, эдгээр устгалуудын хэд хэд нь Гранд Каньоны олон давхаргын үед буюу Пермийн болон Девоны үеийн давхаргатай нэг цаг хугацаанд тохиолддог нь тогтоогдсон байна.

\subsection{Физикийн Гаж Явдлууд}

\begin{figure}[b]
\begin{center}
% \fbox{\rule{0pt}{2in} \rule{0.9\linewidth}{0pt}}
   \includegraphics[width=1\linewidth]{columbia.jpg}
\end{center}

\caption{Вашингтон мужийн Колумбын мөсөн нуурт үүссэн аварга давалгааны мөрүүд \cite{80}.}
\label{fig:11}
\label{fig:onecol}
\end{figure}

Гранд Каньоноос гадна олон газар нутаг сүйрлийн хүчний нөлөөгөөр бий болсон байх магадлалтай. Асар их эх газрын усыг урсгаж байсны нотолгоо дэлхий даяар аварга давалгааны мөрүүдээр илэрнэ. Ийм жишээний нэг нь Номхон далайн баруун хойд бүсэд байрлах Чаннелед Скабландс юм. Энд бид тунамал чулуулаг бүхий газар нутаг, тохиолдлын аварга хад чулуу төдийгүй мега урсгалын үр дүнд бий болсон зуу гаруй удаашралттай том давалгааны ул мөрүүдийг харж болно \cite{78,79}. Эдгээр нь горхины элсэрхэг ёроолд бий болдог жижигхэн давалгааны илүү том хэмжээний хувилбарууд юм. Эдгээрийг Франц, Аргентин, Орос, Хойд Америк зэрэг дэлхийн олон газарт харж болно \cite{81}. Зураг \ref{fig:11} нь АНУ-ын Вашингтон муж дахь эдгээр давалгааны заримыг харуулж байна \cite{80}.

\begin{figure}[b]
\begin{center}
% \fbox{\rule{0pt}{2in} \rule{0.9\linewidth}{0pt}}
   \includegraphics[width=1\linewidth]{zhangjiajie.jpg}
\end{center}
   \caption{Өмнөд Хятадын Жанжиажие Үндэсний Ойн цэцэрлэгт хүрээлэнгийн аварга чулуун баганууд.}
\label{fig:12}
\label{fig:onecol}
\end{figure}

\begin{figure}[b]
\begin{center}
% \fbox{\rule{0pt}{2in} \rule{0.9\linewidth}{0pt}}
\includegraphics[width=1\linewidth]{hoy.jpg}
\end{center}
   \caption{Хой тэнгисийн багана, Шотланд \cite{83}.}
\label{fig:13}
\label{fig:onecol}
\end{figure}

Далд газрын элэгдлийн бүтэцүүд мөн ECDO-той төстэй Дэлхийн хэлтгэлтийн онолоор сайн тайлбарлагддаг. Өмнөд Хятад бол усан элэгдлээр бүрэлдсэн маш том карстын ландшафтын сайхан жишээ юм \cite{82}. Эдгээр газар нутагт цамхаг карст, оргил карст, конус карст, байгалийн гүүр, хавцал, том агуйн систем, болон цөөрөм уналтууд багтана. Эдгээрээс хамгийн гайхамшигтай нь Жанжиажие Үндэсний Ойн цогцолбор бөгөөд асар том кварцын элсэн чулууны багануудтай (Зураг \ref{fig:12}) \cite{84}. Эдгээр багануудын дундаж өндөр нь 1000 метраас давсан бөгөөд нийт 3100 гаруй байдаг. Үүний 1000 гаруй нь 120 метрээс өндөр, 45 нь 300 метрээс давсан өндөртэй \cite{85}. Эдгээр баганууд далайн элэгдлийн багануудтай (Зураг \ref{fig:13}) ижилхэн бөгөөд тэдгээр нь далайн давалгаанаар эргэн тойрны чулуулгууд унаж, эвдэрснээр үүсдэг. Ижил элэгдлийн ландшафтуудыг Туркийн Үргүпийн чулуужсан конус болон Испанийн Сьюдад Энкантадад олж болно. Эдгээр хоёр газар хоёулаа далайн түвшнээс 1000 метраас дээш өндөрт оршдог. Эдгээр бүх газруудын эргэн тойронд давс болон далайн гаралтай чулуужсан үлдэгдэл илэрдэг нь эртний далайн түрэмгийлэл байсныг харуулна \cite{15,86,87}. Мэдээж, үерийн тухай түүхүүдэд \cite{3} далайн ус 1000 метраас илүү өндөр хүрсэн гэж өгүүлдэг ба үүнийг хэдэн километрийн өндөрт орших Анд болон Гималайн нуруун дахь давстай устай цөл болон том давст талбайнууд нотолдог. Жишээ нь, Боливийн Уюни давст тал далайн түвшнээс 3653 метрийн өндөрт оршино \cite{94}.

\subsection{Гэнэтийн уур амьсгалын өөрчлөлтийн үйл явдлууд}

Орчин үеийн шинжлэх ухааны судалгаанд Дэлхийн шинэ түүхэнд гэнэтийн, хурдан уур амьсгалын өөрчлөлтийн үйл явдлуудыг хүлээн зөвшөөрсөн. Хоёр чухал жишээ нь 4.2 мянган жил ба 8.2 мянган жилийн үйл явдлууд бөгөөд эдгээр нь хүн ам буурах, нийгмийн суурьшил алдагдах үзэгдэлтэй давхцдаг. Эдгээр үйл явдлууд тунадас болон мөсний цөм, чулуужсан шүрэн, O18 изотопын хэмжээ, тоос, сталактит, сталагмитны бичлэг, далайн түвшний мэдээнд тусгагдсан. Таамагласан уур амьсгалын өөрчлөлтөд дэлхийн дулааны огцом уналт, хуурайшилт, Атлантын меридиональ урсгал тасалдал, мөсөн голын үргэлжлэл багтдаг \cite{90,91,92}. Ялангуяа 8.2 мянган жилийн үйл явдал нь МЭӨ 6400 оны үед Хар тэнгис рүү далайн ус хүчтэй цутгасан байж болзошгүй \cite{93}.

\subsection{Археологийн гаж үзэгдлүүд}

Эртний зарим хотын археологийн олдворууд олон давхарга бүхий булшилт, сүйрлийн ул мөртэй байдаг нь эртний сүйрлийн үйл явдлыг баталдаг. Ийм хотуудын нэг нь одоогийн Палестинд байрлах Иерихон юм. Энэ хотод чулууны барилга нуралт ба хүчтэй галын ул мөртэй олон устгалын давхарга илэрчээ \cite{96,97}. Эдгээрийн үеийн он цаг нь МЭӨ 9000-2000 оны хооронд таарна. Ялангуяа түүний цамхаг МЭӨ 7400 оны үед огцом тайрагдаж, тунадастай хамт булшлагдсан байна (Зураг \ref{fig:14}) \cite{95}. Чатал Хүйүк \cite{99}, Грамалотэ \cite{98}, мөн Крит дэх Миносын Кноссын ордон \cite{100,101} зэрэг нь археологийн олон давхаргатай, ихэвчлэн сүйрлийн нотолгоотой олдворын төстэй жишээнүүд юм.

\begin{figure}[t]
\begin{center}
% \fbox{\rule{0pt}{2in} \rule{0.9\linewidth}{0pt}}

   \includegraphics[width=1\linewidth]{jericho.jpg}
\end{center}
   \caption{Ерихын цамхгийн булшны археологийн сэргээсэн дүрслэл, МЭӨ 7400 он орчим \cite{95}.}
\label{fig:14}
\label{fig:onecol}
\end{figure}

Хүн төрөлхтний соёл иргэншлийг сүйтгэсэн томоохон сүйрлийн өөр нэгэн нотолгоо бол Нампагийн дүрслэл юм. Энэ нь Идахо мужид ойролцоогоор 100 метр лавын дор шавраар хийсэн хүүхэлдэй олдсон тохиолдол юм \cite{102,103}. Хүүхэлдэй олдсон лавын давхаргыг Хожуу Гуравдагч галав эсвэл Эрт дөрөвдүгээр галавын үед бий болсон гэж тооцоолж, 2 сая жилийн настай гэж үзсэн байдаг. Гэвч тус бүс нутгийн лавын давхарга харьцангуй шинэ байж магадгүй харагдаж байна. Ийм олдворууд зөвхөн соёл иргэншлийг устгасан сүйрэл байсан гэдгийг харуулаад зогсохгүй, орчин үеийн он цагийн тооллын үнэн зөв байдалд эргэлзээ төрүүлж байна.

\section{Орчин үеийн он цагийн аргын тухай}

Орчин үед хэрэглэгдэж буй он цагийн тооллын аргууд нь хэдэн сая, заримдаа хэдэн зуугаас хэдэн зуун сая жилээр тооцоолох болсон нь эргэлзээ төрүүлэх шалтгаан ихтэй.

Уламжлалт ойлголтоор бол "чулуужсан түлш" болох нүүрс, нефть, байгалийн хийг хэдэн зуун сая жилийн настай гэж үздэг \cite{104}. Гэвч Мексикийн булангаас авсан газрын тосыг нүүрстөрөгчийн аргаар шинжлэхэд түүний нас ойролцоогоор 13,000 жил гэж гарсан байна \cite{105}. Нүүрстөрөгч-14 нь маш богино хагас задралын хугацаатай (5,730 жил) бөгөөд хэдэн зуун мянган жилийн дараа бүрэн задрах ёстой. Гэвч мянга дахин өндөр настай гэж тооцоологдсон нүүрс болон чулуужсан олдворуудад нүүрстөрөгч-14 илэрсэн байдаг \cite{106}. Үнэндээ хиймэл нүүрсийг лабораторийн нөхцөлд, гол төлөв өндөр халуун ашиглан ердөө 2-8 сарын дотор гарган авсан байна \cite{107}.

Нүүрстөрөгчийн аргаас өөр радиоизотопын аргууд ч бас яг нарийн биш байх магадлалтай. Answers in Genesis хэмээх судалгааны бүлэг ийм аргуудаар гарсан огноо хоорондын зөрүүг илрүүлж, үүний үнэн зөвийг асуултын тэмдэгт оруулсан байдаг \cite{108}. Түүнчлэн, зуун сая жилийн настай гэж тооцогдсон үлэг гүрвэлийн олдвороос цусны эс, судас, коллаген агуулсан зөөлөн эдүүд илэрсэн тохиолдол бий \cite{109,110}. Бидний мэдэж байгаагаар, Дэлхийн геологийн цаг хугацааны хуваарь болон чулуулаг, чулуужсан түлш зэрэг материалуудын уламжлалт настай холбоотой ойлголтууд нь магадгүй маш ихээр алдаатай байж болох юм.

\section{Дүгнэлт}

Энэ өгүүлэлд би сүйрлийн гаралтай гэж үзэж болохуйц, ECDO онолын дэлхийн эргэлтийн тайлбарт хамгийн зохистой нийцэх онцгой гаж үзэгдлүүдийн заримыг танилцууллаа. Танилцуулсан цуглуулга олон төрөлтэй боловч бүрэн гүйцэд биш — илүү олон гаж үзэгдлийг судалгааны GitHub санхад түргэн олж үзэх боломжтой \cite{2}.
\section{Талархал}

ECDO-ийн онолын бүтээлийн анхны зохиогч Ethical Skeptic-д түүний гүн, шинэлэг бүтээлээ дуусгаж, дэлхийтэй хуваалцсанд талархал илэрхийлье. Түүний гурван хэсэг бүхий онолын бүтээл \cite{1} нь Гадаадын Дулааны Гол-Мантлын Салалт Жанибековын Осцилляцийн (ECDO) онолын хамгийн нэр хүндтэй бүтээл бөгөөд энэ сэдвээр миний энд товчлон өгүүлсэнээс 훨 илүү их мэдээллийг агуулдаг.

Мөн бидний зогсоод байгаа аваргуудын мөрөн дээр суусан гэдгээ мэдэж, энэ ажлыг боломжтой болгосон, хүн төрөлхтөнд гэрэл авчирсан бүх судалгаа, шинжилгээг хийсэн хүмүүст талархал илэрхийлье.

\clearpage
\twocolumn

{\small
\renewcommand{\refname}{Эшлэлүүд}
\bibliographystyle{ieee}
\bibliography{egbib}
}

\end{document}
\documentclass[10pt,twocolumn,letterpaper]{article}

% ของฉันเอง
\usepackage{booktabs}
% \usepackage{caption}
% \captionsetup[table]{skip=8pt}   % มีผลเฉพาะกับตาราง
\usepackage{stfloats}  % เพิ่มอันนี้ในพรีแอมเบิล

\usepackage{babel}

\babelprovide[main,import,maparabic]{thai}

\babelfont{rm}{FreeSerif}

\usepackage{microtype}

\usepackage{cvpr}
\usepackage{times}
\usepackage{epsfig}
\usepackage{graphicx}
\usepackage{amsmath}
\usepackage{amssymb}

% รวมแพ็กเกจอื่นที่นี่ ก่อน hyperref

% ถ้าคุณคอมเมนต์ hyperref แล้วมา uncomment ใหม่ คุณควรลบ
% egpaper.aux ก่อนรัน latex ใหม่  (หรือแค่กด 'q' ตอนรัน latex ครั้งแรก แล้วปล่อยมันรันจนจบ ก็จะเคลียร์)

\usepackage[breaklinks=true,bookmarks=false]{hyperref}

\cvprfinalcopy % *** ยกเลิกการคอมเมนต์บรรทัดนี้สำหรับการส่งฉบับสุดท้าย

\makeatletter
\def\cvprsubsection{\@startsection {subsection}{2}{\z@}
    {8pt plus 2pt minus 2pt}{6pt}{\bfseries\normalsize}}
\makeatother

\def\cvprPaperID{****} % *** ใส่รหัส CVPR Paper ที่นี่
\def\httilde{\mbox{\tt\raisebox{-.5ex}{\symbol{126}}}}

% หน้าถูกใส่หมายเลขในโหมดส่งงาน และไม่มีหมายเลขในกล้องพร้อมพิมพ์
%\ifcvprfinal\pagestyle{empty}\fi
\setcounter{page}{1}
\begin{document}

%%%%%%%%% TITLE
\title{เอกสารแนะนำข้อมูลขับเคลื่อนโดยข้อมูล ECDO ตอนที่ 2/2: การตรวจสอบความผิดปกติทางวิทยาศาสตร์และประวัติศาสตร์ที่อธิบายได้ดีที่สุดโดย “การพลิกโลก” ของ ECDO}

\author{จุนโฮ\\
เผยแพร่ กุมภาพันธ์ 2025\\
เว็บไซต์ (ดาวน์โหลดเอกสารที่นี่): \href{https://sovrynn.github.io}{sovrynn.github.io}\\
ศูนย์วิจัย ECDO: \href{https://github.com/sovrynn/ecdo}{github.com/sovrynn/ecdo}\\
{\tt\small junhobtc@proton.me}
}

\maketitle
%\thispagestyle{empty}

\begin{abstract}
ในเดือนพฤษภาคม ค.ศ. 2024 ผู้เขียนออนไลน์นามแฝงชื่อว่า “The Ethical Skeptic” \cite{0} ได้เผยแพร่ทฤษฎีใหม่ที่เปลี่ยนแปลงสำคัญ เรียกว่า Exothermic Core-Mantle Decoupling Dzhanibekov Oscillation (ECDO) \cite{1} ทฤษฎีนี้ไม่เพียงแต่เสนอว่าโลกเคยประสบกับการเปลี่ยนแปลงแกนหมุนอย่างกะทันหันจนอาจเกิดน้ำท่วมโลกขึ้น เพราะทะเลหลากล้นแผ่นดินเนื่องด้วยแรงเฉื่อยการหมุน แต่ยังเสนอกระบวนการธรณีฟิสิกส์ที่นำไปสู่เหตุการณ์เหล่านั้นพร้อมข้อมูลที่บ่งชี้ว่าอาจกำลังเกิดเหตุการณ์ลักษณะเดียวกันอีกครั้ง แม้คำทำนายเรื่องวันโลกาวินาศและน้ำท่วมมหันตภัยจะไม่ใช่เรื่องใหม่ แต่ทฤษฎี ECDO มีความน่าสนใจเป็นพิเศษด้วยแนวทางทางวิทยาศาสตร์ ทันสมัย สหวิทยาการ และอิงข้อมูล

งานวิจัยนี้เป็นส่วนที่ 2 ของบทสรุปสั้นสองตอนจากการวิจัยอิสระเป็นเวลา 6 เดือน \cite{2,20} เกี่ยวกับทฤษฎี ECDO ซึ่งเน้นเฉพาะความผิดปกติทางวิทยาศาสตร์และประวัติศาสตร์ที่อธิบายได้ดีที่สุดโดยปรากฏการณ์ “Earth flip” อันเป็นผลจาก ECDO แบบหายนะ

\end{abstract}

\section{บทนำ}

ภูมิธรณีวิทยาสมัยใหม่และประวัติศาสตร์แนวยูนิฟอร์มิทาเรียนระบุว่า ภูมิประเทศขนาดใหญ่ เช่น แกรนด์แคนยอน ถูกสร้างขึ้นในช่วงเวลาหลายล้านปี \cite{143}; เกลือในหุบเขามรณะ (แคลิฟอร์เนีย) นั้นเกิดขึ้นเพราะพื้นที่นี้เคยอยู่ใต้ทะเลเมื่อนับร้อยล้านปีก่อน \cite{144}; บรรพบุรุษของเราเมื่อ 150 รุ่นก่อน ใช้ชีวิตทั้งชีวิตในการสร้างสุสานขนาดมหึมา \cite{29,70}; และที่มาของ “เชื้อเพลิงฟอสซิล” ก็มีอายุหลายร้อยล้านปี \cite{104} สิ่งที่น่าสนใจยิ่งกว่านั้นคือมนุษย์ถูกเชื่อว่ามีอายุราว 300,000 ปี \cite{145} แต่ประวัติศาสตร์ที่มีการบันทึกและอารยธรรมมีเพียงประมาณ 5,000 ปี หรือเทียบเท่า 150 รุ่นมนุษย์เท่านั้น

ความผิดปกติเหล่านี้ อย่างที่เราจะได้เห็นกันต่อไป ล้วนสามารถอธิบายได้ดีที่สุดโดยอิทธิพลจากแรงธรณีวิทยาแบบหายนะ

\section{ช้างแมมมอธถูกแช่แข็งเฉียบพลันฝังในโคลน}
\begin{figure}[t]
\begin{center}
% \fbox{\rule{0pt}{2in} \rule{0.9\linewidth}{0pt}}
   \includegraphics[width=1\linewidth]{jarkov-mammoth.jpg}
\end{center}
   \caption{แมมมอธจาร์คอฟ แมมมอธไซบีเรียอายุ 20,000 ปีที่ถูกรักษาสภาพไว้อย่างสมบูรณ์ในโคลนเยือกแข็ง \cite{51}.}
\label{fig:1}
\label{fig:onecol}
\end{figure}

หมวดหมู่หนึ่งของสิ่งผิดปกติเหล่านี้คือแมมมอธที่ถูกแช่แข็งอย่างสมบูรณ์และถูกฝังอยู่ในโคลน มักพบในเขตอาร์กติก (ดูรูปที่ \ref{fig:1}) แมมมอธเบเรโซฟกา ที่ถูกค้นพบในไซบีเรียใต้กรวดโคลน มีสภาพสมบูรณ์มากจนเนื้อของมันยังสามารถรับประทานได้หลังจากมันตายไปแล้วนับพันปี มันยังคงมีเศษพืชอยู่ในปากและกระเพาะอาหาร ทำให้นักวิทยาศาสตร์สงสัยว่ามันถูกแช่แข็งอย่างรวดเร็วได้อย่างไร ทั้งที่มันยังหากินพืชดอกอยู่ก่อนตาย \cite{17} มีรายงานว่า \textit{"เมื่อปี 1901 ได้เกิดความฮือฮาจากการค้นพบซากแมมมอธสมบูรณ์ใกล้แม่น้ำเบเรโซฟกา เนื่องจากสัตว์ตัวนี้ดูเหมือนจะตายเพราะความเย็นจัดในช่วงกลางฤดูร้อน สิ่งที่อยู่ในกระเพาะของมันยังถูกรักษาไว้เป็นอย่างดีและมีทั้งบัตเตอร์คัพและถั่วป่าออกดอก ซึ่งหมายความว่ามันกลืนกินสิ่งเหล่านี้ในปลายเดือนกรกฎาคมหรือต้นสิงหาคม สัตว์ตัวนี้ตายอย่างกะทันหันมากจนมันยังมีหญ้าและดอกไม้คาอยู่ในปาก เห็นได้ชัดว่ามันต้องถูกแรงมหาศาลพัดพาไปไกลจากทุ่งหญ้าของมัน กระดูกเชิงกรานและขาข้างหนึ่งหัก—สัตว์ขนาดใหญ่นี้ถูกเหวี่ยงล้มลงและจากนั้นก็ถูกแช่แข็งจนตายในช่วงเวลาที่ตามปกติแล้วจะเป็นช่วงที่ร้อนที่สุดของปี"} \cite{18} นอกจากนี้ \textit{"[นักวิทยาศาสตร์รัสเซีย] รายงานด้วยว่าแม้แต่ผนังชั้นในสุดของกระเพาะสัตว์นั้นก็ยังมีโครงสร้างเส้นใยที่สมบูรณ์ดี แสดงว่าความร้อนในร่างกายของมันถูกดึงออกไปด้วยกระบวนการธรรมชาติที่รุนแรงมาก แซนเดอร์สันให้ความสนใจกับประเด็นข้อนี้โดยเฉพาะ และนำปัญหานี้ไปสอบถามสถาบันอาหารแช่แข็งแห่งอเมริกา: ต้องใช้วิธีใดในการแช่แข็งแมมมอธทั้งตัวจนแม้แต่ความชื้นในส่วนลึกสุดของร่างกาย รวมถึงผนังในสุดของกระเพาะก็ไม่มีเวลานานพอที่จะเกิดผลึกน้ำขนาดใหญ่จนทำลายเส้นใยเนื้อ?... ไม่กี่สัปดาห์ต่อมาสถาบันก็กลับมาแจ้งกับแซนเดอร์สันว่า มันเป็นไปไม่ได้โดยสิ้นเชิง ด้วยความรู้วิทยาศาสตร์และวิศวกรรมทั้งหมดที่มีอยู่ ไม่มีวิธีใดเลยที่จะดึงความร้อนออกจากซากสัตว์ใหญ่เท่าแมมมอธได้รวดเร็วมากพอจนไม่ก่อให้เกิดผลึกน้ำขนาดใหญ่ในเนื้อสัตว์ นอกจากนี้ หลังจากที่ใช้เทคนิคทางวิทยาศาสตร์และวิศวกรรมจนหมดแล้ว พวกเขายังมองไปที่กระบวนธรรมชาติก็ยังสรุปว่าไม่มีวิธีทางธรรมชาติที่รู้จักใด ๆ ที่จะสามารถทำเช่นนั้นได้"} \cite{19}.

\section{แกรนด์แคนยอน}

แกรนด์แคนยอน ซึ่งเป็นส่วนหนึ่งของเกรตเบซินในภูมิภาคตะวันตกเฉียงใต้ของทวีปอเมริกาเหนือ เป็นอีกหนึ่งปรากฏการณ์ทางธรรมชาติที่บ่งบอกถึงที่มาแบบหายนะ (ดูรูปที่ \ref{fig:2}) เริ่มต้นจากชั้นหินทรายและหินปูนตะกอนที่ประกอบกันเป็นแกรนด์แคนยอนซึ่งขยายตัวเป็นบริเวณกว้างถึง 2.4 ล้านกิโลเมตร$^2$ \cite{21} รูปที่ \ref{fig:3} แสดงการแผ่ขยายของชั้นหินทรายโคโคนิโนทั่วสหรัฐอเมริกาตะวันตก ชั้นทางราบขนาดมหึมาเช่นนี้ที่มีลักษณะเหมือนกันโดยทั่วคงมีเพียงแค่ถูกสะสมทับถมพร้อมกันเท่านั้น

\begin{figure}[b]
\begin{center}
% \fbox{\rule{0pt}{2in} \rule{0.9\linewidth}{0pt}}
   \includegraphics[width=1\linewidth]{grand-canyon.jpg}

\end{center}
   \caption{แกรนด์แคนยอน ที่รัฐแอริโซนา ประเทศสหรัฐอเมริกา \cite{49}.}
\label{fig:2}
\label{fig:onecol}
\end{figure}

\begin{figure}[t]
\begin{center}
% \fbox{\rule{0pt}{2in} \rule{0.9\linewidth}{0pt}}
   \includegraphics[width=1\linewidth]{coconino.jpg}
\end{center}
   \caption{ขนาดของชั้นหินทรายโคโคไนนโน (Coconino Sandstone) ในสหรัฐอเมริกาตะวันตก \cite{21}.}
\label{fig:3}
\label{fig:onecol}
\end{figure}

การสำรวจแกรนด์แคนยอนอย่างใกล้ชิดบอกเราว่าการสะสมตัวของชั้นตะกอนขนาดใหญ่เหล่านี้เกิดขึ้นพร้อมกับแรงทางเทคโทนิกที่สำคัญด้วย เพื่อที่จะเข้าใจสิ่งนี้ เราต้องดูพื้นที่บางส่วนในแคนยอนอย่างใกล้ชิด ซึ่งชั้นตะกอนเหล่านี้ถูกพับงอและถูกเปิดเผย นักวิจัยจาก Answers in Genesis \cite{42} ได้ศึกษาตัวอย่างหินจากบริเวณรอยพับเหล่านี้ในระดับจุลทรรศน์ เช่นที่ Monument Fold และจากการขาดลักษณะบางอย่างที่ควรจะเกิดขึ้นหากรอยพับเหล่านี้เกิดขึ้นในช่วงเวลานานภายใต้ความร้อนและแรงกดดัน จึงสรุปได้ว่าชั้นตะกอนเหล่านี้ถูกพับโดยแรงทางเทคโทนิกในขณะที่ยังคงอ่อนตัวอยู่ กล่าวคือ เกิดขึ้นไม่นานหลังจากการสะสมตัวของตะกอน \cite{43}.

\begin{figure*}
\begin{center}
% \fbox{\rule{0pt}{2in} \rule{.9\linewidth}{0pt}}
\includegraphics[width=1\textwidth]{Grand_Staircase-big.jpg}
\end{center}
   \caption{ชั้นตะกอนที่ประกอบกันเป็นแกรนด์แคนยอน (ด้านขวามือของภาพ) ทอดยาวไปทางทิศเหนือถึง Cedar Breaks, Utah (ด้านซ้ายมือของภาพ) ซึ่งทุกชั้นจะโค้งขึ้น \cite{50}.}
\label{fig:4}
\end{figure*}

เมื่อมองในมุมกว้างขึ้น เราพบว่าชั้นหินที่ประกอบกันเป็นแกรนด์แคนยอนไม่ได้ถูกโค้งเพียงแค่ภายในหุบเขาเท่านั้น ชั้นหินเหล่านี้ถูกโค้งไปทางตะวันออกใน East Kaibab Monocline \cite{46} และยังโค้งไปทางเหนือที่ Cedar Breaks, Utah (ดูรูป \ref{fig:4}) ด้วย ข้อสังเกตนี้บ่งชี้ว่าชั้นหินเหล่านี้อาจถูกโค้งทั้งหมดพร้อมกันหลังจากที่มันถูกวางทับซ้อนกันอย่างรวดเร็ว เพื่อให้เห็นภาพ ชั้นหินแนวนอนของแกรนด์แคนยอนมีความหนาประมาณ 1700 เมตร ขนาดของกระบวนการทางธรณีวิทยาที่ต้องใช้ในการทับถมตะกอนหนาถึงหนึ่งไมล์นั้นมีความยิ่งใหญ่มาก

การเกิดขึ้นจริงของแกรนด์แคนยอนยังเป็นข้อถกเถียงสำคัญในทางธรณีวิทยาสมัยใหม่ ธรณีวิทยากระแสหลักเสนอว่าแกรนด์แคนยอนถูกกัดเซาะโดยแม่น้ำโคโลราโดเป็นล้านๆ ปี \cite{47} อย่างไรก็ตาม ทีมวิจัย Answers in Genesis เชื่อว่าแกรนด์แคนยอนน่าจะเกิดขึ้นภายในเวลาไม่กี่สัปดาห์จากการกัดเซาะแบบ spillway (ธารน้ำล้น) อันเนื่องมาจากทะเลสาบโบราณที่เกิดการทะลักล้น พัดพาตะกอนออกจำนวนมากขณะขุดผ่านหุบเขา มีหลักฐานของทะเลสาบที่อยู่สูงทางตะวันออกของแกรนด์แคนยอนทั้งในชั้นตะกอนของทะเลสาบและซากดึกดำบรรพ์สัตว์ทะเล เมื่อเปรียบเทียบแกรนด์แคนยอนกับตัวอย่างอื่นของการกัดเซาะแบบ spillway ขนาดใหญ่ เช่น Afton Canyon และ Mount St. Helens พบว่ามีภูมิประเทศที่คล้ายคลึงกัน และแสดงให้เห็นว่าหุบเขาขนาดใหญ่สามารถเกิดขึ้นได้อย่างรวดเร็วโดยน้ำจำนวนมหาศาล \cite{48}

เมื่อพิจารณาถึงขนาดของกระบวนการทางธรณีวิทยาที่ต้องใช้ในการวางทับซ้อนตะกอนบนพื้นที่กว้างใหญ่เช่นนี้ การเกิดแรงทางเทคโตนิกอย่างรุนแรงในเวลาไล่เลี่ยหลังจากการวางชั้นตะกอน และความเล็กของแม่น้ำโคโลราโดเมื่อเทียบกับขนาดมหึมาของแกรนด์แคนยอน ดูเหมือนว่าการก่อตัวของแกรนด์แคนยอนอาจไม่ได้เกิดขึ้นอย่างค่อยเป็นค่อยไป

\section{เมืองใต้ดินเดรินกูยู}

นอกเหนือจากพีระมิด ตัวอย่างทางวิศวกรรมโบราณที่โดดเด่นคือเมืองใต้ดินเดรินกูยู (ดูรูป \ref{fig:5}) ตั้งอยู่ที่ Cappadocia ประเทศตุรกี เป็นเมืองใต้ดินที่ใหญ่ที่สุดในบรรดาที่พักพิงใต้ดินมากกว่า 200 แห่งในภูมิภาคนี้ \cite{54} เมืองใต้ดินแห่งนี้เชื่อว่ารองรับประชากรได้สูงสุดถึง 20,000 คน และมีทั้งหมด 18 ชั้น ลึกถึง 85 เมตร แม้อายุของเมืองจะไม่แน่ชัด แต่คาดว่ามีอายุน้อยที่สุด 2,800 ปี เมืองนี้ถูกขุดออกมาจากหินภูเขาไฟเนื้ออ่อน \cite{52, 53}

\begin{figure}[b]
\begin{center}
% \fbox{\rule{0pt}{2in} \rule{0.9\linewidth}{0pt}}

   \includegraphics[width=1\linewidth]{derinkuyu.jpeg}
\end{center}
   \caption{แผนผังของเมืองใต้ดิน Derinkuyu \cite{56}.}
\label{fig:5}
\label{fig:onecol}
\end{figure}

เหตุผลที่ Derinkuyu น่าสนใจก็คือยังไม่ชัดเจนว่าทำไมชุมชนใดจึงจะตัดสินใจสร้างทั้งเมืองไว้ใต้ดิน เพื่อจะสร้างพื้นที่อยู่อาศัยใต้ดิน ทุกโพรงจะต้องสกัดจากหิน รูปร่างและลักษณะที่หยาบของอุโมงค์ใต้ดินทำให้เห็นชัดว่าเกิดจากการใช้แรงงานมือ ไม่ใช่เครื่องมือกลไฟฟ้า ซึ่งจะยิ่งยากกว่าการสร้างที่พักอาศัยอยู่บนผิวดินอย่างมาก ที่จริงแล้ว ยังไม่ชัดเจนว่าทำไมมนุษย์ถึงอยากอาศัยอยู่อย่างถาวรใต้ดินในระหว่างชีวิตบนโลกของตน เมื่อการเกษตร แสงแดด ธรรมชาติ และการสำรวจสามารถพบได้เพียงบนผิวดิน “ประวัติศาสตร์” ตามแบบแผนเสนอว่า Derinkuyu ถูกสร้างโดยชาวคริสต์ที่ต้องการสถานที่ที่สงบเงียบเพื่อปฏิบัติศาสนกิจของตน \cite{53} แต่สามัญสำนึกจะสรุปได้ว่าวิธีที่ตรงไปตรงมาที่สุดในการรับมือกับศัตรูก็คือ “สู้หรือหนี” ไม่ใช่ “สกัดเมืองใต้ดินออกมาจากหิน”

ขนาด, ความลึก, และความรอบคอบของการออกแบบเมืองใต้ดินทำให้เห็นได้ชัดว่าไม่ได้ถูกสร้างขึ้นเพื่อเป็นโครงสร้างทางการทหารชั่วคราวสำหรับต่อสู้ข้าศึกในยามคับขัน หากแต่เป็นที่หลบภัยระยะยาวเพื่อป้องกันภัยถึงชีวิตบนพื้นผิว Derinkuyu ไม่ได้มีแค่ห้องนอน ห้องครัว ห้องน้ำ แต่ยังมีคอกสัตว์ ถังเก็บน้ำ ที่เก็บอาหาร โรงบีบไวน์และน้ำมัน โรงเรียน โบสถ์ สุสาน และปล่องระบายอากาศขนาดใหญ่ (รูป \ref{fig:6}) ที่หลบภัยของทหารจะมีโรงบีบไวน์ไปทำไม และต้องขุดลงไปลึกถึง 85 เมตรด้วยความซับซ้อนอย่างนี้เพื่ออะไร?

คำอธิบายที่มีเหตุผลที่สุดในการสร้าง Derinkuyu ก็คือมีความจำเป็นเร่งด่วนต่อการเตรียมที่หลบภัยระยะยาวที่พึ่งพาตนเองได้ เพื่อป้องกันอันตรายทางธรณีวิทยาครั้งใหญ่บนผิวโลก

\begin{figure}[t]
\begin{center}
% \fbox{\rule{0pt}{2in} \rule{0.9\linewidth}{0pt}}
   \includegraphics[width=1\linewidth]{derinkuyu-air.jpg}
\end{center}
   \caption{ปล่องระบายอากาศลึกใน Derinkuyu \cite{53}.}
\label{fig:6}
\label{fig:onecol}
\end{figure}

% \section{ความผิดปกติเพิ่มเติมที่อธิบายได้ดีที่สุดโดยการพลิกโลก}

% ก่อนจะสรุป เราจะกล่าวถึงความผิดปกติทางวิทยาศาสตร์เพิ่มเติมบางประการที่เมื่อมองในบริบทของพลังงานธรณีวิทยาที่เกิดมหันตภัยแล้ว จะได้รับการอธิบายอย่างดี

\section{การสะสมของชีวมวล}

ส่วนผสมของชีวมวลซึ่งประกอบด้วยสัตว์และพืชหลากหลายชนิด โดยมักจะพบในสภาพเป็นซากดึกดำบรรพ์ในชั้นตะกอน เป็นความผิดปกติที่น่าฉงนอีกประการหนึ่ง ใน "Reliquoæ Diluvianæ" บาทหลวงวิลเลียม บัคลันด์ ได้อธิบายถึงการค้นพบสัตว์หลายสายพันธุ์จำนวนมากที่ไม่มีเหตุผลใด ๆ ที่ควรจะพบอยู่ร่วมกัน กระจายอยู่ทั่วบริเตนและยุโรป ฝังอยู่ในชั้น 'ดิลูเวียม' ตะกอนน้ำท่วม \cite{58} ส่วนผสมของซากสัตว์เช่นนี้ยังถูกพบในถ้ำ Skjonghelleren บนเกาะ Valdroy ประเทศนอร์เวย์ ในถ้ำแห่งนี้ มีกระดูกของสัตว์เลี้ยงลูกด้วยนม นก และปลา กว่า 7,000 ชิ้น ถูกพบว่าปะปนกันอยู่ในชั้นตะกอนต่าง ๆ \cite{59} ตัวอย่างอีกที่หนึ่งคือที่ San Ciro "ถ้ำยักษ์" ประเทศอิตาลี ในถ้ำนี้ พบกระดูกสัตว์เลี้ยงลูกด้วยนมหลายตัน ส่วนใหญ่เป็นฮิปโปโปเตมัส ซึ่งอยู่ในสภาพสดใหม่มากจนถูกนำไปตัดทำเป็นเครื่องประดับ และส่งออกนำไปใช้ผลิตเขม่าดำสำหรับทำโคมไฟ โดยมีกระดูกสัตว์ชนิดต่าง ๆ ปะปนกัน แตกหัก กระจัดกระจายเป็นชิ้นเล็กชิ้นน้อย \cite{60,61} ที่ Mendes โบราณ ประเทศอียิปต์ ยังพบส่วนผสมของกระดูกสัตว์หลากหลายชนิดปะปนกับดินเหนียวที่กลายเป็นแก้ว \cite{57} การค้นพบในลักษณะดังกล่าวอาจดูแปลกประหลาด แต่สามารถอธิบายได้ง่ายโดยอุทกภัยขนาดใหญ่ซึ่งพัดพาเอาซากสัตว์ไปทับถมกันในชั้นตะกอน นำสัตว์เข้าไปหรือฝังพวกมันทั้งเป็นในถ้ำ และในกรณีของชีวมวลที่กลายเป็นแก้วในอียิปต์ เกิดขึ้นจากไฟฟ้าขนาดมหึมาหลังภัยพิบัติซึ่งเกิดจากการเคลื่อนตัวของแก่นโลก รูปที่ \ref{fig:7} แสดงตัวอย่างชั้นชีวมวล 'โคลน' ในอลาสก้า \cite{56}

\begin{figure}[t]
\begin{center}
% \fbox{\rule{0pt}{2in} \rule{0.9\linewidth}{0pt}}
   \includegraphics[width=1\linewidth]{muck-crop.jpeg}
\end{center}
   \caption{'โคลน' ของอลาสก้า ประกอบด้วยเศษซากของต้นไม้ พืช และสัตว์ที่กระจัดกระจายอย่างไร้ระเบียบในตะกอนน้ำแข็งและน้ำแข็งแข็งตัว \cite{146}.}
\label{fig:7}
\label{fig:onecol}
\end{figure}
\section{บังเกอร์โบราณ}

บรรพบุรุษของเราได้ทิ้งโครงสร้างโบราณที่มีวิศวกรรมขั้นสูงไว้มากมาย ซึ่งพบว่ามีซากศพมนุษย์อยู่ในนั้น โดยทั่วไป โครงสร้างเหล่านี้มักถูกตีความว่าเป็นสุสานที่วิจิตรตระการตา แต่เมื่อสังเกตอย่างละเอียดแล้ว กลับชวนให้คิดว่าโครงสร้างเหล่านี้อาจเป็นบังเกอร์โบราณ

\begin{figure}[b]
\begin{center}
% \fbox{\rule{0pt}{2in} \rule{0.9\linewidth}{0pt}}
   \includegraphics[width=1\linewidth]{ww19.jpg}
\end{center}
   \caption{นิวเกรนจ์ ประเทศไอร์แลนด์ - ดูนักท่องเที่ยวที่ทางเข้าเพื่อเปรียบเทียบขนาด}
\label{fig:8}
\label{fig:onecol}
\end{figure}

ตัวอย่างที่ดีเยี่ยมคือ นิวเกรนจ์ (รูปที่ \ref{fig:8}) อนุสรณ์สถานหลักในกลุ่มบรู นา โบอินน์ ซึ่งเป็นกลุ่มโครงสร้างโบราณรวมถึงสิ่งที่เรียกว่าสุสานทางเดิน สุสานเหล่านี้ประกอบด้วยห้องฝังศพหนึ่งห้องหรือมากกว่าซึ่งถูกคลุมด้วยดินหรือหิน และมีทางเดินเข้าแคบที่สร้างจากหินก้อนใหญ่ \cite{70} นี่คือตัวอย่างของวิศวกรรมขั้นสูงในการสร้างโครงสร้างป้องกันที่ซับซ้อน สร้างขึ้นหลายชั่วอายุคน โดยเชื่อกันว่าเพื่อฝังศพผู้คนน้อยมาก ที่แม้แต่ตัวพวกเขาเองก็ยังไม่ได้เกิดเมื่อเริ่มสร้างสุสานนี้ เมื่อมีการค้นพบใหม่โดยเจ้าของที่ดินท้องถิ่นในปี 1699 สุสานนี้ถูกฝังอยู่ใต้ดิน

เมื่อดูคร่าว ๆ จะเห็นความพยายามมากมายที่ทุ่มเทไปกับการก่อสร้าง – นิวเกรนจ์ประกอบด้วยวัสดุประมาณ 200,000 ตัน ภายในนั้น \textit{“…เป็นทางเดินที่ต่อเนื่องไปยังห้องกลาง ซึ่งสามารถเข้าสู่ได้จากทางเข้าด้านตะวันออกเฉียงใต้ของอนุสรณ์ ทางเดินนี้ยาว 19 เมตร (60 ฟุต) หรือประมาณหนึ่งในสามของทางเข้าสู่ศูนย์กลางโครงสร้าง ที่ท้ายของทางเดินจะมีห้องเล็ก 3 ห้องแยกออกจากห้องกลางขนาดใหญ่ที่มีหลังคาแบบห้องโถงสูง… ผนังของทางเดินนี้ประกอบด้วยแผ่นหินขนาดใหญ่ที่เรียกว่าออโธสแตต มีอยู่ยี่สิบสองแผ่นด้านตะวันตกและยี่สิบเอ็ดแผ่นด้านตะวันออก โดยมีความสูงเฉลี่ย 1½ เมตร”} \cite{70} นอกจากนี้ยังมีรายละเอียดด้านวิศวกรรมกันน้ำที่ซับซ้อน เช่น ที่หลังคา \textit{“ช่องว่างระหว่างแผ่นหินที่หลังคาถูกอัดด้วยส่วนผสมของดินเผาและทรายทะเลเพื่อกันน้ำ และจากส่วนผสมนี้ได้ผลการหาอายุคาร์บอนที่มีศูนย์กลางอยู่ที่ 2500 ปีก่อนคริสต์ศักราชสำหรับโครงสร้างสุสาน”} \cite{71} นอกจากนี้ยังอาจมีการยกระดับพื้นที่นำไปสู่ห้องกลางเพื่อจุดประสงค์เดียวกัน: \textit{“เนื่องจากพื้นของทางเดินและห้องกลางของสุสานนี้ปฏิบัติตามความเอียงของเนินที่อนุสรณ์สร้างอยู่ จึงทำให้ระดับพื้นระหว่างทางเข้าและภายในห้องต่างกันเกือบ 2 เมตร”} \cite{71}

\begin{figure}[b]

\begin{center}
% \fbox{\rule{0pt}{2in} \rule{0.9\linewidth}{0pt}}
   \includegraphics[width=1\linewidth]{dolmen.jpg}
\end{center}
   \caption{Dolmen de Soto, สเปน \cite{53}.}
\label{fig:9}
\label{fig:onecol}
\end{figure}

การขาดซากศพมนุษย์ภายในก็เป็นจุดที่น่าสงสัยเช่นกัน จากการขุดค้นพบชิ้นส่วนกระดูกที่ถูกเผาและไม่ได้เผาซึ่งเป็นตัวแทนของคนเพียงไม่กี่คน กระจายอยู่ตามทางเดิน การก่อสร้างนิวเกรนจ์คาดว่าต้องใช้เวลาหลายชั่วอายุคนตามอายุคาร์บอนของวัสดุที่พบภายใน ทำไมชุมชนในอดีตจึงต้องใช้ความพยายามอย่างมากในการสร้างสุสานขนาดใหญ่ที่ออกแบบอย่างดี เพียงเพื่อโปรยกระดูกของผู้ตายเพียงไม่กี่คนไว้ในทางเดิน? เป็นไปได้มากกว่าว่าโครงสร้างหินขนาดใหญ่ที่สร้างมาอย่างพิถีพิถันและกันน้ำเหล่านี้ถูกสร้างขึ้นเพื่อใช้เป็นที่หลบภัยของมนุษย์ในช่วงที่เกิดหายนะซ้ำ ๆ บนโลก

ที่อูเอลบา ทางตอนใต้ของสเปน ตัวอย่างที่คล้ายกันคือ Dolmen de Soto (รูปที่ \ref{fig:9}) ซึ่งเป็นหนึ่งในสถานที่ประมาณ 200 แห่งในพื้นที่นี้ \cite{72,32} เป็นโครงสร้างที่ออกแบบอย่างล้ำสมัยโดยใช้แท่นหินขนาดใหญ่ และมีเส้นผ่าศูนย์กลาง 75 เมตร มีรายงานว่าพบศพเพียงแปดร่างเมื่อขุดค้น โดยทั้งหมดถูกฝังในท่าคู้ตัว

\section{การเอ่ยถึงความผิดปกติที่โดดเด่น}

ในส่วนนี้ ข้าพเจ้าจะกล่าวถึงความผิดปกติที่โดดเด่นเพิ่มเติมอีกเล็กน้อย ซึ่งล้วนแล้วแต่สามารถอธิบายได้ดีโดยหายนะคล้าย ECDO

\subsection{ความผิดปกติทางชีววิทยา}

\begin{figure}[b]
\begin{center}
% \fbox{\rule{0pt}{2in} \rule{0.9\linewidth}{0pt}}
   \includegraphics[width=1\linewidth]{bottleneck.jpg}
\end{center}
   \caption{คอขวดทางพันธุกรรมที่แสดงถึงการลดจำนวนประชากรชายลง 95\% เมื่อประมาณ 6,000 ปีก่อน \cite{62}.}
\label{fig:10}
\label{fig:onecol}
\end{figure}

ความผิดปกติทางชีววิทยาที่น่าสนใจได้แก่ คอขวดทางพันธุกรรมและซากวาฬที่พบในแผ่นดินลึก Zeng และคณะ (2018) ได้จำลองลำดับโครโมโซม Y จำนวน 125 ตัวอย่างจากมนุษย์ยุคใหม่ และจากความคล้ายคลึงและการกลายพันธุ์ในดีเอ็นเอ ระบุถึงคอขวดทางประชากรที่ลดลง 95\% ของประชากรชาย เมื่อประมาณ 5,000 ถึง 7,000 ปีก่อน (รูปที่ \ref{fig:10}) \cite{62} ซากวาฬถูกพบเหนือระดับน้ำทะเลหลายร้อยเมตร ในสวีเดนบอร์ก มิชิแกน เวอร์มอนต์ แคนาดา ชิลี และอียิปต์ \cite{63,64,65,66} วาฬเหล่านี้ถูกพบในสภาพที่แตกต่างกันไป: สภาพสมบูรณ์มาก, ในบึงเหนือชั้นตะกอนธารน้ำแข็ง หรือฝังอยู่ในตะกอน จำนวนตัวอย่างที่พบในแต่ละพื้นที่มีตั้งแต่ไม่กี่ตัวจนถึงกว่าร้อยตัว วาฬเป็นสัตว์ทะเลน้ำลึกและแทบจะไม่เข้ามาใกล้ชายฝั่ง วาฬเหล่านี้ไปจบอยู่ที่ระดับความสูงมากเช่นนี้ และห่างไกลจากทะเลได้อย่างไร?

การสูญพันธุ์ครั้งใหญ่เกิดขึ้นบนโลกหลายครั้ง โดยที่ได้รับการศึกษามากที่สุดคือ "บิ๊กไฟว์" เหตุการณ์มหาสูญพันธุ์แห่งมหายุคฟาเนอโรโซอิก: ปลายออร์โดวิเชียน (LOME), ปลายดีโวเนียน (LDME), ปลายเปอร์เมียน (EPME), ปลายไทรแอสซิก (ETME) และปลายครีเทเชียส (ECME) \cite{88,89} ที่น่าสนใจก็คือ หลายเหตุการณ์ถูกจัดว่าเกิดขึ้นในช่วงเวลาเดียวกับชั้นหินยุคต่าง ๆ ของแกรนด์แคนยอน คือ ชั้นหินยุคเปอร์เมียนและดีโวเนียน

\subsection{ความผิดปกติทางกายภาพ}

\begin{figure}[b]
\begin{center}
% \fbox{\rule{0pt}{2in} \rule{0.9\linewidth}{0pt}}
   \includegraphics[width=1\linewidth]{columbia.jpg}
\end{center}
   \caption{ระลอกคลื่นขนาดใหญ่จากกระแสน้ำในทะเลสาบโคลัมเบีย น้ำแข็งละลาย รัฐวอชิงตัน \cite{80}.}
\label{fig:11}
\label{fig:onecol}
\end{figure}

มีภูมิประเทศอีกมากมายที่ไม่ใช่แค่แกรนด์แคนยอนซึ่งมีแนวโน้มว่าจะก่อตัวขึ้นจากพลังงานอันมหาศาล หลักฐานของการไหลของน้ำขนาดใหญ่ระดับทวีปสามารถพบได้จากระลอกคลื่นขนาดยักษ์ทั่วโลก ตัวอย่างหนึ่งก็คือบริเวณ Channeled Scablands ในแถบแปซิฟิกตะวันตกเฉียงเหนือ ที่นี่ เราไม่เพียงแต่จะพบภูมิประเทศฝากตะกอนและก้อนหินขนาดใหญ่ที่ผิดปกติ แต่ยังมีระลอกคลื่นขนาดใหญ่ที่เกิดจากกระแสน้ำมหาศาลมากกว่าร้อยลำดับ \cite{78,79} ระลอกคลื่นเหล่านี้มีขนาดใหญ่กว่าแบบที่เกิดในท้องทรายตามลำธารมาก ระลอกคลื่นประเภทนี้สามารถพบได้ทั่วโลกในฝรั่งเศส อาร์เจนตินา รัสเซีย และอเมริกาเหนือ \cite{81} รูปที่ \ref{fig:11} แสดงระลอกคลื่นบางส่วนเหล่านี้ในรัฐวอชิงตัน สหรัฐอเมริกา \cite{80}

\begin{figure}[b]
\begin{center}
% \fbox{\rule{0pt}{2in} \rule{0.9\linewidth}{0pt}}
   \includegraphics[width=1\linewidth]{zhangjiajie.jpg}
\end{center}
   \caption{เสาหินขนาดมหึมาในอุทยานแห่งชาติจางเจียเจี้ย ภาคใต้ของประเทศจีน}
\label{fig:12}
\label{fig:onecol}
\end{figure}

\begin{figure}[b]
\begin{center}
% \fbox{\rule{0pt}{2in} \rule{0.9\linewidth}{0pt}}

   \includegraphics[width=1\linewidth]{hoy.jpg}
\end{center}
   \caption{เสาหินเก่าแก่ โอลด์แมนแห่งฮอย ประเทศสกอตแลนด์ \cite{83}.}
\label{fig:13}
\label{fig:onecol}
\end{figure}

โครงสร้างการกัดเซาะภายในแผ่นดินก็อธิบายได้ดีโดยโมเดล ECDO ที่คล้ายการพลิกโลก ภาคใต้ของประเทศจีนเป็นตัวอย่างที่ยอดเยี่ยมของภูมิประเทศคาสต์ขนาดใหญ่ที่เกิดจากการกัดเซาะของน้ำ \cite{82} ภูมิประเทศเหล่านี้รวมถึงคาสต์รูปหอคอย คาสต์ยอดแหลม คาสต์รูปกรวย สะพานหินธรรมชาติ ช่องเขา ระบบถ้ำขนาดใหญ่ และปล่องภูเขา หนึ่งในจุดที่โดดเด่นที่สุดคืออุทยานป่าไม้จางเจียเจี้ย ซึ่งมีเสาทรายควอตซ์ขนาดมหึมา (รูป \ref{fig:12}) \cite{84} เสาเหล่านี้ตั้งอยู่ที่ความสูงเฉลี่ยมากกว่า 1,000 เมตร และมีจำนวนมากกว่า 3,100 ต้น มากกว่า 1,000 ต้นสูงเกิน 120 เมตร และ 45 ต้นสูงเกิน 300 เมตร \cite{85} เสาเหล่านี้มีลักษณะคล้ายกับเสาหินชายฝั่ง (รูป \ref{fig:13}) ซึ่งเป็นเสาหินตามชายฝั่งที่เกิดจากการพังทลายของวัสดุรอบ ๆ ด้วยแรงคลื่นทะเล ภูมิประเทศที่ถูกกัดเซาะลักษณะเดียวกันนี้สามารถพบได้ในกรวยหินที่เออร์กุป ประเทศตุรกี เช่นเดียวกับ Ciudad Encantada ประเทศสเปน ซึ่งทั้งสองแห่งตั้งอยู่สูงกว่าระดับน้ำทะเล 1,000 เมตร สถานที่ทั้งหมดนี้มีร่องรอยของเกลือและซากดึกดำบรรพ์สิ่งมีชีวิตทะเลอยู่ใกล้เคียง สะท้อนถึงการบุกรุกของน้ำทะเลในอดีต \cite{15,86,87} เรื่องเล่าน้ำท่วม \cite{3} ก็กล่าวถึงทะเลที่ขึ้นสูงกว่าระดับ 1,000 เมตรในอดีต และได้รับการยืนยันโดยการพบเกลือน้ำทะเลและแหล่งเกลือขนาดใหญ่ในเทือกเขาแอนดีสและหิมาลัยซึ่งสูงจากระดับน้ำทะเลหลายกิโลเมตร ตัวอย่างเช่น ทะเลเกลืออูยูนีในโบลิเวียมีความสูงถึง 3,653 เมตรจากระดับน้ำทะเล \cite{94}

\subsection{เหตุการณ์เปลี่ยนแปลงสภาพภูมิอากาศอย่างรวดเร็ว}

วรรณกรรมทางวิทยาศาสตร์สมัยใหม่ยอมรับถึงการมีอยู่ของเหตุการณ์เปลี่ยนแปลงสภาพภูมิอากาศโลกอย่างรวดเร็วในประวัติศาสตร์โลกที่ผ่านมา ตัวอย่างสำคัญสองเหตุการณ์คือ เหตุการณ์ 4.2 พันปีก่อน และ 8.2 พันปีก่อน ซึ่งตรงกับช่วงที่ประชากรลดลงและการหยุดชะงักของการตั้งถิ่นฐานของสังคมในพื้นที่กว้าง เหตุการณ์เหล่านี้ถูกบันทึกเป็นความผิดปกติในแกนตะกอนและแกนน้ำแข็ง ซากปะการังซากดึกดำบรรพ์ ค่าระดับไอโซโทป O18 บันทึกเกสรพืชและหินงอกหินย้อย และข้อมูลระดับน้ำทะเล การเปลี่ยนแปลงสภาพภูมิอากาศที่สันนิษฐาน ได้แก่ การลดลงอย่างรวดเร็วของอุณหภูมิโลกโดยรวม ความแห้งแล้ง การหยุดชะงักของกระแสน้ำหมุนเวียนแอตแลนติก และการรุกคืบของธารน้ำแข็ง \cite{90,91,92} เหตุการณ์ 8.2 พันปีก่อนสอดคล้องกับเหตุการณ์น้ำเค็มท่วมทะเลดำขนาดใหญ่ราว 6,400 ปีก่อนคริสตกาล \cite{93}

\subsection{ความผิดปกติทางโบราณคดี}

หลักฐานทางโบราณคดีของบางเมืองโบราณแสดงให้เห็นชั้นการฝังกลบและการทำลายล้างหลายชั้น ซึ่งสร้างบันทึกเหตุการณ์หายนะในอดีต เมืองโบราณเยรีโคเป็นหนึ่งในตัวอย่างนั้น ตั้งอยู่ในปาเลสไตน์ปัจจุบัน เมืองนี้มีชั้นการทำลายล้างหลายชั้น รวมถึงการถล่มของสิ่งก่อสร้างหินและไฟไหม้รุนแรง \cite{96,97} ลำดับชั้นในเมืองบันทึกการอยู่อาศัยตั้งแต่ประมาณ 9000 ปีก่อนคริสตกาลถึง 2000 ปีก่อนคริสตกาล สิ่งที่น่าสนใจเป็นพิเศษคือหอคอยของเมืองซึ่งดูเหมือนว่าจะถูกเฉือนและฝังในตะกอนราว 7,400 ปีก่อนคริสตกาล (รูป \ref{fig:14}) \cite{95} ชาตาลฮิวยุก \cite{99} กรามาโลเต \cite{98} และพระราชวังมิโนอันแห่งคโนซอสบนเกาะครีต \cite{100,101} ล้วนเป็นตัวอย่างของแหล่งโบราณคดีที่มีชั้นหลักฐานหลายชั้น โดยมักพบหลักฐานการทำลายล้างร่วมด้วย

\begin{figure}[t]
\begin{center}
% \fbox{\rule{0pt}{2in} \rule{0.9\linewidth}{0pt}}

\includegraphics[width=1\linewidth]{jericho.jpg}
\end{center}
   \caption{การจำลองทางโบราณคดีของการฝังหอคอยเยรีโคประมาณ 7400 ปีก่อนคริสต์ศักราช \cite{95}.}
\label{fig:14}
\label{fig:onecol}
\end{figure}

หลักฐานอีกชิ้นหนึ่งที่แสดงให้เห็นถึงหายนะครั้งใหญ่ที่ทำลายอารยธรรมมนุษย์คือภาพนัมปา รูปปั้นตุ๊กตาดินที่พบใต้ลาวาประมาณ 100 เมตรในไอดาโฮ \cite{102,103} ลาวาที่พบรูปร่างนี้ถูกประเมินว่าถูกสะสมในช่วงปลายยุคเทอร์เชียรีหรือยุคควอเทอร์นารีตอนต้น ซึ่งคิดว่าอายุประมาณ 2 ล้านปี อย่างไรก็ตาม ลาวาในภูมิภาคนั้นดูค่อนข้างใหม่ การค้นพบเช่นนี้ไม่เพียงแต่ชี้ให้เห็นถึงหายนะที่ทำลายอารยธรรมใหญ่หลวง แต่ยังกระตุ้นให้เกิดข้อสงสัยต่อการกำหนดอายุทางโบราณคดีในยุคปัจจุบันด้วย

\section{เกี่ยวกับวิธีการกำหนดอายุสมัยใหม่}

มีเหตุผลสำคัญในการตั้งข้อสงสัยต่อการกำหนดอายุในยุคปัจจุบัน ซึ่งมักจะกำหนดอายุของวัสดุทางกายภาพว่ายาวนานเป็นล้านปี หรือแม้แต่หลายร้อยล้านปี

เรื่องเล่าแบบดั้งเดิมกล่าวว่าสิ่งที่เรียกว่า "เชื้อเพลิงฟอสซิล" เช่น ถ่านหิน น้ำมัน และก๊าซธรรมชาติ มีอายุหลายร้อยล้านปี \cite{104} อย่างไรก็ตาม ผลการตรวจอายุคาร์บอนของน้ำมันในอ่าวเม็กซิโก พบว่าน้ำมันมีอายุประมาณ 13,000 ปี \cite{105} คาร์บอน-14 มีครึ่งชีวิตสั้นมาก (5,730 ปี) ซึ่งควรจะสลายตัวจนหมดภายในเวลาไม่กี่แสนปี แต่ก็ยังถูกค้นพบในถ่านหินและซากดึกดำบรรพ์ที่อ้างว่ามีอายุเก่าแก่กว่าเป็นพันเท่า \cite{106} นอกจากนี้ ถ่านหินสังเคราะห์ยังสามารถผลิตได้ในห้องทดลองภายใต้สภาวะควบคุมที่มีความร้อนสูง ใช้เวลาเพียง 2-8 เดือน \cite{107}

วิธีการหาอายุด้วยไอโซโทปรังสีนอกเหนือจากคาร์บอนก็อาจไม่ถูกต้องเช่นกัน กลุ่มวิจัย Answers in Genesis พบความไม่สอดคล้องกันของอายุที่ได้จากวิธีเหล่านี้ ซึ่งทำให้เกิดข้อสงสัยในความถูกต้องของผลลัพธ์ \cite{108} แม้แต่เนื้อเยื่ออ่อนที่มีเซลล์เลือด เส้นเลือด และคอลลาเจนก็ยังถูกค้นพบในซากไดโนเสาร์ที่อ้างว่ามีอายุร้อยล้านปี \cite{109,110} จากสิ่งที่เราทราบ เป็นไปได้ว่าอายุที่ได้รับการยอมรับในทางธรณีวิทยาและวัสดุทางกายภาพต่าง ๆ เช่น หินและเชื้อเพลิงฟอสซิล อาจมีความคลาดเคลื่อนจากความจริงอย่างมาก

\section{บทสรุป}

ในบทความนี้ ข้าพเจ้าได้นำเสนอความผิดปกติที่น่าสนใจที่สุดซึ่งบ่งบอกถึงต้นกำเนิดที่เกิดหายนะและอธิบายได้ดีที่สุดด้วยทฤษฎีการเปลี่ยนขั้วโลกของโลก ECDO แม้ว่าตัวอย่างที่นำเสนอจะมีความหลากหลายแต่ก็ยังไม่สมบูรณ์ - ยังมีความผิดปกติอื่น ๆ อีกมากซึ่งถูกรวบรวมและเผยแพร่สาธารณะไว้ใน GitHub สำหรับงานวิจัยของข้าพเจ้า \cite{2}.

\section{กิตติกรรมประกาศ}

ขอขอบคุณ Ethical Skeptic ผู้เขียนต้นฉบับของวิทยานิพนธ์ ECDO สำหรับการจัดทำวิทยานิพนธ์ที่ลึกซึ้งและบุกเบิกขึ้นมา พร้อมแบ่งปันให้กับโลก วิทยานิพนธ์ทั้งสามส่วนของเขา \cite{1} ยังคงเป็นงานอ้างอิงหลักสำหรับทฤษฎีการแยกตัวของแกนกับเนื้อโลกแบบดูดซับความร้อนและการสั่นสะเทือนแบบจานิเบคอฟ (ECDO) และมีข้อมูลเกี่ยวกับหัวข้อนี้มากกว่าที่ข้าพเจ้าได้สรุปไว้โดยสังเขปที่นี่

และแน่นอน ขอบคุณยักษ์ใหญ่ทั้งหลายที่เป็นรากฐานให้เราก้าวต่อไป; ผู้ซึ่งได้ทำการวิจัยและสืบสวนทั้งหมดจนทำให้งานนี้บังเกิดขึ้นได้และทำงานเพื่อนำแสงสว่างมาสู่มนุษยชาติ

\clearpage
\twocolumn

{\small
\renewcommand{\refname}{บรรณานุกรม}
\bibliographystyle{ieee}
\bibliography{egbib}
}

\end{document}
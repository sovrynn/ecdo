\documentclass[10pt,twocolumn,letterpaper]{article}

% ខ្ញុំបន្ថែមផ្ទាល់ខ្លួន
\usepackage{booktabs}
% \usepackage{caption}
% \captionsetup[table]{skip=8pt}   % មានផលប៉ះពាល់តែតារាងប៉ុណ្ណោះ
\usepackage{stfloats}  % បន្ថែមវានៅក្នុង preamble

% \usepackage{fontspec}
\usepackage[english]{babel}

\directlua{
  luaotfload.add_fallback("myfallback", {
    "FreeSerif:mode=harf;",
  })
}

% load Lao via babelprovide, turn on "onchar=ids" for automatic shaping
\babelprovide[import,onchar=ids fonts]{khmer}

% main (rm) font for Latin
\babelfont{rm}[RawFeature={fallback=myfallback}]{Noto Serif}

% Lao text in Noto Serif Lao at 1.2× scale
\babelfont[khmer]{rm}[RawFeature={fallback=myfallback}]{Noto Sans Khmer}
\babelfont[khmer]{sf}[RawFeature={fallback=myfallback}]{Noto Sans Khmer}

% alternate (sans-serif) font for Latin
\babelfont{alt}{Lato}

% Lao text in Noto Serif Lao for the alt family too
\babelfont[khmer]{alt}[RawFeature={fallback=myfallback}]{Noto Sans Khmer}

\babelpatterns[khmer]{%
  % all Khmer consonants (U+1780–U+17A2)
  1ក 1ខ 1គ 1ឃ 1ង 1ច 1ឆ 1ជ 1ឈ 1ញ
  1ដ 1ឋ 1ឌ 1ឍ 1ណ 1ត 1ថ 1ទ 1ធ 1ន
  1ប 1ផ 1ព 1ភ 1ម 1យ 1រ 1ល 1វ 1ឝ
  1ឞ 1ស 1ហ 1ឡ 1អ
  % all modern independent vowels (U+17A5–U+17B3)
  1ឥ 1ឦ 1ឧ 1ឨ 1ឩ 1ឪ 1ឫ 1ឬ 1ឭ 1ឮ
  1ឯ 1ឰ 1ឱ 1ឲ 1ឳ%
}

\usepackage{cvpr}
\usepackage{times}
\usepackage{epsfig}
\usepackage{graphicx}
\usepackage{amsmath}
\usepackage{amssymb}

% សូមបញ្ចូល package ផ្សេងទៀតនៅទីនេះ មុនពេល hyperref។

% ប្រសិនបើអ្នកមកនិយាយពី hyperref ហើយបន្ទាប់មកមកដោះសោវា វា គួរតែលុប
% egpaper.aux មុនពេលប្ដូរឡើងវិញ latex។  (ឬតែចុច 'q' នៅលើ latex ដំបូង
% អោយវាបញ្ចប់ ហើយអ្នកគួរតែស្អាត)។
\usepackage[breaklinks=true,bookmarks=false]{hyperref}

\makeatletter
\def\cvprsubsection{\@startsection {subsection}{2}{\z@}
    {8pt plus 2pt minus 2pt}{6pt}{\bfseries\normalsize}}
\makeatother

\cvprfinalcopy % *** ដោះស្រាយបន្ទាត់នេះសម្រាប់ការដាក់ស្នើចុងក្រោយ

\def\cvprPaperID{****} % *** បញ្ចូលលេខសម្គាល់សៀវភៅ CVPR នៅទីនេះ
\def\httilde{\mbox{\tt\raisebox{-.5ex}{\symbol{126}}}}

% ទំព័រត្រូវបានដាក់លេខនៅក្នុងរបៀបដាក់ស្នើ ហើយមិនមានលេខនៅក្នុងរបៀបបោះពុម្ព​ចុងក្រោយ
%\ifcvprfinal\pagestyle{empty}\fi
\setcounter{page}{1}
\begin{document}

%%%%%%%%% ចំណងជើង
\title{ECDO Data-Driven Primer ភាគ ២/២៖ ការស្រាវជ្រាវអំពីភាពខុសប្លែកវិទាសាស្ត្រនិងប្រវត្តិសាស្ត្រ ដែលពន្យល់បានល្អជាងដោយ "ការប្រែប្រួលផែនដី" របស់ ECDO}

\author{Junho\\
ប្រកាសខែកុម្ភៈ ឆ្នាំ២០២៥\\
គេហទំព័រ (ទាញយកអត្ថបទនៅទីនេះ): \href{https://sovrynn.github.io}{sovrynn.github.io}\\
ឃ្លាំងស្រាវជ្រាវ ECDO: \href{https://github.com/sovrynn/ecdo}{github.com/sovrynn/ecdo}\\
{\tt\small junhobtc@proton.me}
}

\maketitle
%\thispagestyle{empty}

\begin{abstract}
នៅខែឧសភា ឆ្នាំ២០២៤ អ្នកនិពន្ធអនាមិកម្នាក់ឈ្មោះ "The Ethical Skeptic" \cite{0} បានផ្សព្វផ្សាយទ្រឹស្ដីបំប្លែងមួយដែលមានឈ្មោះថា Exothermic Core-Mantle Decoupling Dzhanibekov Oscillation (ECDO) \cite{1}។ ទ្រឹស្ដីនេះមិនត្រឹមតែបានផ្តល់យោបល់ថាផែនដីមុននេះរង់ចាំបានបម្លែងអ័ក្សវិលជាមួយភាពផ្លាស់ប្តូរដ៏សាហាវជាបន្ទាន់ ធ្វើឲ្យមានទឹកជំនន់ធំជាសកលដោយបណ្តាលឲ្យមហាសមុទ្រលេចលើទ្វីបដោយអុសទីយ៉ានកម្មនៃការទ្រាំវិលទេប៉ុណ្ណោះទេ តែថែមទាំងផ្ដល់ពន្យល់នៃដំណើរការជាធាតុភូមិវិទ្យាដែលមានគ្រាប់ហេតុ និងទិន្នន័យបង្ហាញថា អាចនឹងមានបម្លែងថ្១ទៀតកំពុងកើតឡើង។ ទោះបីជាការព្យាករណ៍ទឹកជំនន់ និងអាសន្នស្រាប់ជាពាក្យចាស់ក៏ដោយ ទ្រឹស្ដី ECDO នេះគឺមានភាពទាក់ទាញពិសេសដោយសារតែការរួបរួមវិទ្យាសាស្ត្រសាមញ្ញទំនើបពហុបច្ចេកទេស និងមានមូលដ្ឋានលើទិន្នន័យ។

ក្រដាសស្រាវជ្រាវនេះជាផ្នែកទី២នៃសង្ខេបផ្នែកពីរនៃខ្សែស្ដីពីការស្រាវជ្រាវឯករាជ្យរយៈពេល ៦ ខែ \cite{2,20} ទល់ទៅនឹងទ្រឹស្ដី ECDO ដោយផ្តោតលើភាពខុសប្លែកផ្នែកវិទ្យាសាស្ត្រ និងប្រវត្តិសាស្ត្រដែលត្រូវបានពន្យល់ប្រសើរបំផុតដោយ "ការប្រែប្រួលភូគព្ភសាស្ត្រដ៏កក្រើក" តាមបែបទ្រឹស្ដី ECDO។

\end{abstract}

\section{ការណែនាំ}

ភូគព្ភវិទ្យា និងប្រវត្តិសាស្ត្ររួមសម័យនេះអះអាងថា ទំនាបធំបំផុតដូចជា Grand Canyon ត្រូវបានបង្កើតឡើងក្នុងរយៈពេលរាប់លានឆ្នាំ \cite{143}។ ថា អំបិលនៅក្នុង Death Valley (California) មានស្រាប់ព្រោះតំបន់នោះធ្លាប់ស្ថិតក្រោមមហាសមុទ្ររយៈពេលរាប់រយលានឆ្នាំកន្លងមក \cite{144}។ ថា បុព្វបុរសរបស់យើងកន្លងមក១៥០ជំនាន់បានចំណាយជីវិតសាងសង់បុរាណសពធំនានា \cite{29,70}។ ហើយថា "ឥន្ធនៈហ្សៃវះ" គឺមានអាយុរាប់រយលានឆ្នាំ \cite{104}។ ដែលគួរឲ្យឆិកក៏នោះគឺនៅពេលដែលជំនាន់មនុស្សត្រូវគេជឿថាមានអាយុ៣០០,000 ឆ្នាំ \cite{145} ប៉ុន្តែប្រវត្តិសាស្ត្រដែលមានការចងកំណត់ត្រឹមតែ ៥,០០០ ឆ្នាំប៉ុណ្ណោះ ជាសមមូលនឹង១៥០ជំនាន់មនុស្ស។

ភាពខុសប្លែកទាំងនេះ ដូចដែលយើងនឹងឃើញខាងក្រោយ ត្រូវបានពន្យល់បានល្អបំផុតដោយកម្លាំងភូគព្ភវិទ្យាដែលកើតមានអាសន្ន។

\section{ដុំសត្វម៉ាមម៉ុស្ដ្រត្រូវបានបោសកកក្នុងផ្តិល}
\begin{figure}[t]
\begin{center}
% \fbox{\rule{0pt}{2in} \rule{0.9\linewidth}{0pt}}
   \includegraphics[width=1\linewidth]{jarkov-mammoth.jpg}
\end{center}
   \caption{សត្វម៉ាមម៉ុត់យ៉ាកក៉ូវ (Jarkov Mammoth) អាយុ ២០,០០០ ឆ្នាំ ស្ថិតក្នុងសភាពថែទាំបានដ្បិតល្អ ជាសត្វម៉ាមម៉ុត់ស៊ីបេរី ដែលរកឃើញនៅក្នុងភូមិដីកខ្ចក់ដៃត្រជាក់ \cite{51}។}
\label{fig:1}
\label{fig:onecol}
\end{figure}

ប្រភេទភាពចម្លែកមួយនៃបាតុភូតនេះ គឺសត្វម៉ាមម៉ុត់ត្រជាក់ដែលត្រូវបានថែទាំគ្រប់លក្ខណៈនៅក្នុងដីកខ្ចក់ តែងតែប្រទះឃើញនៅតំបន់អាρκទិក (រូបភាព \ref{fig:1})។ សត្វម៉ាមម៉ុត់បេរេសូវកា ដែលរកឃើញនៅស៊ីបេរីនៅក្នុងដីកកម្សាន្ត មានសភាពថែទាំល្អដល់ថ្នាក់សាច់របស់វានៅអាចបរិភោគបានស្រួលៗ ចាប់តាំងពីពាន់ឆ្នាំបន្ទាប់ពីវាស្លាប់។ វានៅមានអាហារដំណាំនៅក្នុងមាត់ និងពោះរបស់វា ដែលបណ្ដាលให​វេជ្ជបណ្ឌិតចម្រេីនចិត្តថាតើធ្វើដូចម្តេចបានសត្វនេះត្រូវកកទ័រយ៉ាងរហ័សបែបនេះ ខណៈពេលវាកំពុងចែងល្មនលើរុក្ខជាតិផ្កាហើយមុនពេលវាស្លាប់ \cite{17}។ តាមរបាយការណ៍ បានសរសេរថា \textit{"នៅឆ្នាំ ១៩០១ មានការភ្ញាក់ផ្អើលពេលរកឃើញសាកសពសត្វម៉ាមម៉ុត់គាំទ្របានល្អនៅជិតទន្លេបេរេហ្សូវកា ដោយសារសត្វនេះសម្លាប់ដោយសារក្តៅក្រឡឹកក្នុងរដូវក្តៅ។ អាហារនៅក្នុងពោះរបស់វាត្រូវបានថែទាំនៅសភាពល្អ និងមានផ្កាបូតិកាបនិងសណ្តែកព្រៃកំពកផ្កា នេះមានន័យថាវានៅថ្មីៗបានបរិភោគបញ្ចប់នៅចុងខែកក្កដាឬដើមសីហា។ សត្វនេះស្លាប់ជាឈឺចាប់ដល់ថ្នាក់វានៅកាន់ស្លឹកបានយ៉ាងតឹងល្អបំផុត។ វាត្រូវបានចាប់ដៃដរាប់ទាំងមូល និងបំពាក់បានខ្លាំងៗពីទីលំនៅដើមរបស់វាច្រើនគីឡូម៉ែត្រ។ សាច់ត្រចៀកនិងជើងមួយបាក់ខូច—សត្វធំធាត់បានជ្រុលទៅលើជង្គង់ ហើយបន្តែត្រជាក់ដល់ស្លាប់ នៅក្នុងអំឡុងពេលដែលជាឆ្នាំក្តៅបំផុត"} \cite{18}។ បន្ថែមទៅលើនេះ \textit{"[វិទ្យាសាស្ត្រ​ រុស្ស៊ី] បានកត់សម្គាល់ថា ស្រទាប់ខាងក្នុងជាទីស្នាក់នៅក្នុងពោះសត្វនេះ មានរចនាសម្ព័ន្ធសរសៃថែទាំល្អបំផុត សំគាល់ថាបំណងតែសំណើមផ្ទៃក្នុងសត្វត្រូវបានយកចេញដោយដំណើរការដ៏អស្ចារ្យក្នុងធម្មជាតិ។ សែនដឺសុន ទទួលក្តីចាប់អារម្មណ៍យ៉ាងខ្លាំងចំពោះបញ្ហានេះ ហើយបញ្ជូនទៅស្ថាប័នអាហារត្រជាក់សហរដ្ឋអាមេរិក៖ តើត្រូវរបៀបអ្វីដែរ ដើម្បីត្រជាក់សត្វម៉ាមម៉ុត់ដែលធំដល់ថ្នាក់ ឲ្យសំណើមផ្ទៃក្នុងសត្វ ទៅដល់ស្រទាប់ក្នុងជើងផងពោះផង ឲ្យកកដោយគ្មានសរសៃទឹកកកធំៗណាមួយបំផ្លាញរចនាសម្ព័ន្ធសាច់?... ប៉ុន្មានសប្តាហ៍ក្រោយមក ស្ថាប័ននេះបានឆ្លើយតបទៅសែនដឺសុនថា៖ វាអស្ត្រា្ច ច្រើនខ្លាំង។ ជាមួយចំណេះដឹងនឹងបច្ចេកវិទ្យាទាំងអស់របស់យើង គ្មានវិធីណាមួយទេដែលអាចយកកម្ដៅពីសាកសពធំៗដូចម៉ាមម៉ុត់បានយ៉ាងលឿន ទៅកកដោយស្រួលដែលគ្មានទឹកកកវែងៗបង្កើតនៅក្នុងសាច់ឡើយ។ ចំពោះបច្ចេកវិទ្យានិងវិទ្យាសាស្ត្រទាំងអស់ពិតជាមិនអាចបំពេញបានទេ ហើយពេលស្រាវជ្រាវទៅការប្រព្រឹត្តិខ្លាំង ហេតុអ្វីដែរគ្មានដំណើរការធម្មជាតិណាមួយអាចធ្វើអោយបំពេញបំណងនេះបាន"} \cite{19}។

\section{កោះធំ (Grand Canyon)}

កោះធំ ដែលជាផ្នែកមួយនៃតំបន់ Great Basin នៅភាគនិរតីអាមេរិកខាងជើង គឺជាបាតុភូតធម្មជាតិមួយផ្សេងទៀតដែលបញ្ជាក់ពីប្រភពបង្កដោយស្ថានភាពគ្រោះថ្នាក់ធម្មជាតិ (រូបភាព \ref{fig:2})។ ដំបូង ចំនួនស្រទាប់ថ្មខ្សាច់ និងថ្មជ័រ ដែលបង្កើតបានកោះធំ មានផ្ទៃដីធំដល់ ២.៤លាន គ.ម $^2$ \cite{21}។ រូបភាព \ref{fig:3} បង្ហាញអំពីកំណត់ផ្ទៃធំទូលាយនៃស្រទាប់ថ្មខ្សាច់កូកូនីណូ (Coconino Sandstone) នៅក្នុងភាគខាងលិចសហរដ្ឋអាមេរិក។ ស្រទាប់ធំទូលាយទាំងនេះនៃសារធាតុជាក់លាក់តែមួយអាចត្រូវបានបង្កើតឡើងតែមួយដងប៉ុណ្ណោះ។

\begin{figure}[b]
\begin{center}
% \fbox{\rule{0pt}{2in} \rule{0.9\linewidth}{0pt}}
   \includegraphics[width=1\linewidth]{grand-canyon.jpg}

\end{center}
   \caption{ចហ្រ្វានដ៍ខានយូន (Grand Canyon), នៅអារហ្សូណា សហរដ្ឋអាមេរិក \cite{49}.}
\label{fig:2}
\label{fig:onecol}
\end{figure}

\begin{figure}[t]
\begin{center}
% \fbox{\rule{0pt}{2in} \rule{0.9\linewidth}{0pt}}
   \includegraphics[width=1\linewidth]{coconino.jpg}
\end{center}
   \caption{ទំហំស្រទាប់ថ្មខ្សាច់ Coconino នៅភាគខាងលិចសហរដ្ឋអាមេរិក \cite{21}.}
\label{fig:3}
\label{fig:onecol}
\end{figure}

ការពិនិត្យដោយមានភាពជិតស្និទ្ធទៅលើចហ្រ្វានដ៍ខានយូន បង្ហាញថាការតម្កល់ស្រទាប់អ័ព្ទសេឌីម៉ង់យ៉ាងធំធេងទាំងនេះ​ក៏បានកើតឡើងជាមួយគ្នាជាមួយកម្លាំងធូរតិចតូនិចសំខាន់ផងដែរ។ ដើម្បីយល់អំពីសកម្មភាពនេះ យើងត្រូវសង្កេតយ៉ាងល្អទៅតំបន់ខ្លះៗក្នុងចហ្រ្វានដ៍ ដែលនៅតំបន់នោះស្រទាប់សេឌីម៉ង់បានត្រូវបត់ និងបង្ហាញខ្លួនចេញ។ អ្នកស្រាវជ្រាវពី Answers in Genesis \cite{42} បានសិក្សាខ្នាតមីក្រូស្ពិកលើគំរូថ្មពីសំណុំបត់មួយចំនួនដូចជា Monument Fold ហើយដោយផ្អែកលើការខ្វះលក្ខណៈពិសេសដែលគួរតែមាន ប្រសិនបើសំណុំបត់ទាំងនេះបានកើតមានអស់ពេលយូរជាមួយកម្តៅនិងសំពាធ បានសន្និដ្ឋានថា ស្រទាប់សេឌីម៉ង់ត្រូវបានបត់ដោយកម្លាំងធូរតិចតូនិច ខណៈពេលដែលវានៅស្រាល នៅមិនទាន់រឹងគឺបន្ទាប់ពីវាត្រូវបានតម្កល់ ថ្មីៗ \cite{43}។

\begin{figure*}
\begin{center}
% \fbox{\rule{0pt}{2in} \rule{.9\linewidth}{0pt}}
\includegraphics[width=1\textwidth]{Grand_Staircase-big.jpg}
\end{center}
   \caption{ជាប់ជានិច្ចប្លង់ស្រទាប់សដិផៅដែលបង្កើតឡើងជាគ្រោងហ្បូងGrand Canyon (ខាងស្តាំនៃរូបភាព)ធ្វើការស្របតាមទិសជើងទៅរកCedar Breaks, Utah (ខាងឆ្វេងរូបភាព) ដែលបង្រ្កាបឡើងលើខ្លួនឡើងវិញ \cite{50}.}
\label{fig:4}
\end{figure*}

នៅពេលពង្រីកមើលយើងរកឃើញថាស្រទាប់ដែលបង្កើតGrand Canyon មិនត្រឹមតែកោងឡើងនៅក្នុងលំហនេះប៉ុណ្ណោះទេ។ ស្រទាប់ទាំងនេះត្រូវបានបង្រ្កាបទៅភាគខាងកើតនៅEast Kaibab Monocline \cite{46} ផងដែរ ក្នុងកំណត់តំបន់ជើងនៅCedar Breaks, Utah (រូបភាព \ref{fig:4})។ អ្វីនេះបង្ហាញថាស្រទាប់ទាំងនេះប្រហែលជាត្រូវបានបង្រ្កាបជាមួយគ្នាបន្ទាប់ពីបានត្រូវដាក់សន្ធឹកលើគ្នាដោយល្បឿនលឿន។ សម្រាប់យោង ស្រទាប់ផត់កំណត់ផ្តេកនៃGrand Canyon មានកម្រាស់ប្រហែល ១៧០០ម៉ែត្រ។ វិសាលភាពនៃដំណើរការជីវភាគនេះដើម្បីតម្កល់ស្រទាប់សដិផៅជាច្រើនបណ្តោយមួយម៉ាយគឺធំនៅមួយ។

ការបង្កើតពេលប្រាកដនៃGrand Canyon គឺជាបញ្ហាមួយនៃការពិភាក្សានៅវិទ្យាសាស្ត្រព្រំដែនស័យសម័យទំនើប។ វិទ្យាសាស្ត្រលំនាំធម្មតាបានបង្ហាញថា Grand Canyon ត្រូវបានក៏ព្រំផត់ដោយទន្លេ Colorado ជាយូរឆ្នាំរាប់លាន \cite{47}។ ទោះជាយ៉ាងណាក្រុមស្រាវជ្រាវរបស់ Answers in Genesis ព្យាយាមឲ្យគេជឿថាGrand Canyon ប្រហែលជាត្រូវបានបង្កើតក្នុងរយៈពេលពីរបីសប្តាហ៍ដោយសារការទប់ស្ទឹងហៀរចេញពីបឹងបុរាណមួយដោយកប៉ាល់សាំញ៉ាំភាគខាងកើត និងបានបកកាប់ជាច្រើននៅពេលវានាំឲ្យកើតជារន្ធភ្នំ។ មានភស្តុតាងពីបឹងដែលមានកំពស់ខ្ពស់ខាងកើត Grand Canyon នៅក្នុងស្រទាប់នៅលើទីតាំងនៃបឹងនិងអថ្សសមុទ្រល្បឿន។ ការប្រៀបធៀបGrand Canyon ជាមួយឧទាហរណ៍នៃការបណ្តេញទឹកធំៗដូចជាAfton Canyon និងភ្នំ St. Helens បង្ហាញពីភាពស្រដៀងគ្នានៃចំការប្រវត្តិសាស្ត្រ និងបង្ហាញថា រន្ធភ្នំធំៗអាចត្រូវបានបង្កើតឡើងយ៉ាងឆាប់រហ័សដោយសារទឹករត់ច្រើនណាស់ \cite{48}។

ពិចារណាអំពីវិសាលភាពនៃដំណើរការជីវភាគដែលទាមទារដើម្បីធ្វើការតម្កល់ស្រទាប់សដិផៅនៅលើផ្ទៃដីឯកន្ទភាពធំៗយ៉ាងហួសប្រមាណ កម្លាំងតោងភាគធំដែលកើតឡើងជាប់ស្របបន្ទាប់ពីស្រទាប់សដិផៅត្រូវបានដាក់សន្ធឹកឲ្យគ្នា និងទំហំតូចចង្អៀតនៃទន្លេ Colorado ជាប់នឹងវិសាលភាពធំទូលាយនៃGrand Canyon វាធ្វើឲ្យទំនងថាប្រតិបត្តិការ​នៃការបង្កើតវា មិនមែនកើតឡើងយ៉ាងដិតដៃនោះទេ។

\section{Derinkuyu Underground City}

ក្រៅពីសុពលភាពនៃប៉េរមីត ឧទាហរណ៍ឧបករណ៍សំណង់បុរាណល្អមួយគឺទីក្រុងក្រោមដីDerinkuyu (រូបភាព \ref{fig:5}) ដែលស្ថិតនៅក្នុងតំបន់ Cappadocia ប្រទេសតួកគី។ វាជាធំជាងគេក្នុងចំណោមជម្រកក្រោមដីជាង២០០កន្លែងនៅតំបន់នេះ \cite{54}។ ទីក្រុងក្រោមដីនេះត្រូវបានគេប៉ាន់ប្រមាណថាអាចមានប្រជាជនរស់នៅដល់ទៅ២០,០០០នាក់ និងមានជាន់១៨ជាន់ ឈានចូលទៅជ្រៅ៨៥ម៉ែត្រ។ ទោះបីវាអាយុប៉ុន្មានមិនប្រាកដនោះក៏ដោយ វាត្រូវបានគេប៉ាន់ប្រមាណថាមានអាយុយ៉ាងតិច២៨០០ឆ្នាំ។ ទីក្រុងនេះបានខាងភាគចេញពីថ្មភ្លាក់ភ្លើងដ៏ទន់ \cite{52, 53}។

\begin{figure}[b]
\begin{center}
% \fbox{\rule{0pt}{2in} \rule{0.9\linewidth}{0pt}}

   \includegraphics[width=1\linewidth]{derinkuyu.jpeg}
\end{center}
   \caption{ប្លង់នៃទីក្រុងក្រោមដី Derinkuyu \cite{56}.}
\label{fig:5}
\label{fig:onecol}
\end{figure}

មូលហេតុដែល Derinkuyu គួរឱ្យចាប់អារម្មណ៍គឺដោយសារ វាមិនច្បាស់ថាតើ​សហគមន៍ណាមួយនឹងសម្រេចចិត្តសាងសង់ទីក្រុងមួយទាំងមូលនៅក្រោមដីមកពីមូលហេតុអ្វី។ ដើម្បីបង្កើតកន្លែងរស់នៅក្រោមដី គ្រប់រន្ធទាំងអស់ត្រូវតែត្រូវប៉ាន់យកចេញពីថ្ម។ រាង និងសភាពរាងកាយរបស់ឆងផ្លូវក្រោមដីបង្ហាញយ៉ាងច្បាស់ថាវាត្រូវបានចម្លាក់ដោយកម្លាំងកាយ ដោយមិនប្រើឧបករណ៍ធ្វើការដែលប្រើថាមពលនោះទេ ដែលវានឹងពិបាកជាងការសាងសង់ជាលក្ខណៈសាំញ៉ាំនៅលើផ្ទៃដី។ ជាក់ស្តែង វាមិនច្បាស់ថាមនុស្សណាម្នាក់នឹងចង់រស់នៅក្រោមដីរហូតទៅក្នុងអំឡុងជីវិតយូរហូតពេលនៅលើផែនដី ខណៈដែលកសិកម្ម ពន្លឺថ្ងៃ ធម្មជាតិ និងការរុករកមានតែផ្ទៃដីប៉ុណ្ណោះ។ ប្រវត្តិសាស្ត្រធម្មតា "history" សន្និដ្ឋានថា Derinkuyu ត្រូវបានបង្កើតឡើងដោយគ្រិស្ដសាសនិកដែលត្រូវការទីកន្លែងសម្ងាត់សម្រាប់អនុវត្តសាសនានៃខ្លួន \cite{53}។ ប៉ុន្តែវាងាយស្រួលព្រាងយល់ថាវិធីសម្បូរបំផុតដើម្បីអប់រំជាមួយសត្រូវគឺ "ប្រយុទ្ធឬគេច" មិនមែន "ចម្លាក់ទីក្រុងក្រោមដីចេញពីថ្ម" ទេ។

វាស់វែងទៅតាមវិមាត្រ បណ្តោយ និងការរចនាដែលពិចារណាជ្រាលជ្រៅនៃទីក្រុងក្រោមដីនេះបង្ហាញថាវាមិនត្រូវបានរចនា ដើម្បីជាគ្រឿងសម្ភារៈការពារសម្រាប់យុទ្ធសាស្រ្តយោធាវេលាបង្រ្កាបសត្រូវនោះទេ ប៉ុន្តែជាស្ថានភាពសុវត្ថិភាពរយៈពេលវែងដើម្បីការពារកម្សាន្តខ្លាំងបំផុតលើផ្ទៃដី។ Derinkuyu មានបំពាក់មិនត្រឹមតែបន្ទប់គេង ផ្ទះបាយ បន្ទប់ទឹកទេ តែមានផងដែរស្ទាបលសត្វធំៗ ចំណុះទឹក ទុកអាហារ ស្ថានីយ៍ច្រោះស្រា និងប្រេង ទីសិក្សា សាលាប្រជុំ សាលាកាពិក ផ្ទាំងបញ្ចុកសព និងរន្ធបូមខ្យល់តំលៃធំ (រូបភាព \ref{fig:6})។ តើគ្រោងការពារយោធាអ្វីមួយត្រូវការប្រើស្ថានីយ៍ច្រោះស្រា និងត្រូវសាងសង់ជ្រៅ ៨៥ ម៉ែត្រដោយប្រើបច្ចេកវិជ្ជាស្មុគស្មាញបែបនេះដូចម្តេច?

ការពន្យល់ដែលគួរឱ្យទុកចិត្តបំផុតសម្រាប់ការបង្កើត Derinkuyu ក៏អាចត្រូវបានសន្និដ្ឋានថា ជាការចាំបាច់ត្រូវរៀបចំស្ថានីយ៍សុវត្ថិភាពវែងប្រៀបដូចជាកន្លែងរស់នៅដែលខ្លួនឯងអាចរកស៊ី ប្រើប្រាស់បានយូរដើម្បីការពារពីបញ្ហាធម្មជាតិធ្ងន់ធ្ងរលើផ្ទៃផែនដី។

\begin{figure}[t]
\begin{center}
% \fbox{\rule{0pt}{2in} \rule{0.9\linewidth}{0pt}}
   \includegraphics[width=1\linewidth]{derinkuyu-air.jpg}
\end{center}
   \caption{រន្ធបូមខ្យល់ជ្រៅក្នុង Derinkuyu \cite{53}.}
\label{fig:6}
\label{fig:onecol}
\end{figure}

% \section{អាណោមាលីបន្ថែមដែលបានអថិប្បាយយ៉ាងល្អបំផុតដោយការបង្វិលផែនដី}

% មុនពេលបញ្ចប់ យើងនឹងលើកឡើងពីអាណោមាលីវិទ្យាសាស្ត្របន្ថែមដែលនៅពេលបានមើលក្នុងបរិបទកម្លាំងភូមិសាស្ត្រដ៏អស្ចារ្យ វាត្រូវបានបកស្រាយយ៉ាងល្អ។ 

\section{ការប្រមូលផ្ដុំសរីរាង្គ}

ការប្រមាណសរីរាង្គនៃប្រភេទសត្វនិងរុក្ខជាតិផ្សេងៗគ្នា ដែលជាញឹកញាប់ត្រូវបានរកឃើញជាអណ្តែតនៅក្នុងស្រទាប់ស្ពាន់ជាតិ ជាបញ្ហាប្រលោកមួយផ្សេងទៀត។ នៅក្នុង "Reliquoæ Diluvianæ" លោក Rev. William Buckland បានលើកឡើងអំពីការរកឃើញសត្វជាច្រើនប្រភេទដែលគ្មានមូលហេតុអ្វីដែលឲ្យឧត្តមានជាមួយគ្នា បែកចេញតាមចំណោលប្រទេសអង់គ្លេសនិងអឺរ៉ុប ដែលបានប៉ុនប៉ងក្នុងស្រទាប់ 'diluvium' \cite{58}។ ការងារដូចនេះក៏ត្រូវបានរកឃើញនៅក្នុងរូង Skjonghelleren នៅកោះ Valdroy ប្រទេសន័រវែស។ នៅក្នុងរូងនេះ មានឆ្អឹងសត្វមានទឹកដោះ សត្វស្រែក និងត្រីចំនួនលើស ៧,០០០ ត្រូវបានរកឃើញ ក្រឡាប់គ្នាតាមស្រទាប់ស្ពាន់ជាតិតិចតួច \cite{59}។ ឧទាហរណ៍មួយទៀតគឺ San Ciro, "រូងអ្នកធំ", នៅប្រទេសអ៊ីតាលី។ នៅក្នុងរូងនេះ គេបានរកឃើញឆ្អឹងសត្វមានទឹកដោះជាច្រើនតោន ជាពិសេសសត្វហ៊ីប៉ូដែលមានសភាពស្រស់ប៉ុន្មានដល់ថ្នាក់ត្រូវបានកាត់មកធ្វើជាអលង្ការហើយនាំចេញដើម្បីផលិតឧត្ថរៈវិទ្យា Lamp black។ ឆ្អឹងសត្វផ្សេងៗត្រូវបានរាយការណ៍ថាលាយគ្នា បាក់ បែកបាក់ និងរំលាយជាបំណែកៗ \cite{60,61}។ នៅក្នុងទីក្រុងបុរាណ Mendes ប្រទេសអេហ្ស៊ីប ក៏បានរកឃើញឆ្អឹងសត្វប្រភេទផ្សេងៗលាយច្របល់ជាមួយដីភ្លឺសំយោគ (glassy clay) \cite{57}។ ការរកឃើញទាំងនេះត្រូវមានស្រាប់ប្រហែលបំណុល ប៉ុន្តែក្រោមការពន្យល់ធំធេងដោយទឹកជំនន់ធំដែលបង្ហាប់សារពាង្គកាយសត្វស្លាប់លាយគ្នានៅក្នុងស្រទាប់ស្ពាន់ជាតិ ដាក់សត្វចូលក្នុងរូង ឬបំផ្លាញវាឱ្យរស់នៅក្នុងរូង ហើយសំរាប់សារពាង្គកាយវិទ្យាសាស្ត្រដែលមានសភាពភ្លឺនៅអេហ្ស៊ីប កើតឡើងដោយចំហាយអគ្គិសនីធំធេងក្រោយទឹកជំនន់ដែលកើតពីការផ្លាស់ទីស្នូល-ស្រទាប់ខាំថ្ម។ រូបភាព \ref{fig:7} បង្ហាញពីសភាពធម្មតានៃ 'muck' សហរដ្ឋអាឡាស្កា \cite{56}។

\begin{figure}[t]
\begin{center}
% \fbox{\rule{0pt}{2in} \rule{0.9\linewidth}{0pt}}
   \includegraphics[width=1\linewidth]{muck-crop.jpeg}
\end{center}
   \caption{'muck' អាឡាស្កា ដែលច្នៃប្រឌិតពីបំណែកឈើ រុក្ខជាតិ និងសត្វផ្សេងៗត្រូវបានរាលដាលយ៉ាងចម្រិតក្នុងសែលតនិងទឹកកក \cite{146}។}
\label{fig:7}
\label{fig:onecol}
\end{figure}
\section{ធ្វើឱ្យមានសុវត្ថិភាពចាស់ៗ}

បុព្វជនរបស់យើងបានទុកចោលសំណង់បុរាណដ៏ម៉ត់ចត់ជាច្រើន ដែលរកឃើញសាកសពមនុស្ស។ ទូទៅត្រូវបានបកស្រាយថា​ជា​បสุវន្តយ៉ាងស្រស់ស្អាត ប៉ុន្តែបើមើលជិតស្និទ្ធ មានន័យថា​វា​អាច​ជាកន្លែងសុវត្ថិភាពបុរាណ៕

\begin{figure}[b]
\begin{center}
% \fbox{\rule{0pt}{2in} \rule{0.9\linewidth}{0pt}}
   \includegraphics[width=1\linewidth]{ww19.jpg}
\end{center}
   \caption{Newgrange, អៀរឡង់ - សូមមើលភ្ញៀវនៅផ្លូវចូលសម្រាប់ប្រៀបធៀបរង្វាស។}
\label{fig:8}
\label{fig:onecol}
\end{figure}

ឧទាហរណ៍ល្អមួយគឺ Newgrange (រូបភាព \ref{fig:8}), ដែល​ជា​ប្រាសាទ​មេ​នៅក្នុងក្រុម Brú na Bóinne ជាឃ្លុំសំណង់បុរាណរួមមានអ្វីដែលហៅថា​បរអធិប្បាយផងដែរ។ បរអធិប្បាយទាំងនេះមានបន្ទប់បញ្ចុះសាកសពមួយឬច្រើនដែលគ្របដោយដី ឬថ្ម ហើយមានផ្លូវចូលតូចៗបង្កើតពីថ្មធំៗ \cite{70}។ វាជាឧទាហរណ៍នៃវិស្វកម្មដ៏សម្បើមនៃរចនាសម្ព័ន្ធការពារ ជាសំណង់ដែលសង់រយៈពេលជាច្រើនជំនាន់ ថែមទាំងសម្រាប់បញ្ចុះសាកសពមនុស្សតិចរបស់ពួកគេដែលមិនទាន់រស់នៅពេលសាងសង់ផ្សែងនោះ។ នៅពេលវាត្រូវបានរកឃើញឡើងវិញដោយម្ចាស់ដីមួយក្នុងឆ្នាំ ១៦៩៩ វាត្រូវបានគ្របដោយដី។

ពេលមើលលឿនទៅលើសំណង់ នោះអ្នកនឹងឃើញការខិតខំយ៉ាងខ្លាំងក្នុងការសាងសង់វា - Newgrange មានសម្ភារៈប្រហែល ២០០,០០០ តោន។ ខាងក្នុងវា \textit{“... មានផ្លូវបន្ទប់សាកសពដែលអាចចូលពីផ្លូវចូលខាងអាគ្នេយ៍នៃសំណង់នោះ។ ផ្លូវនេះវែងប្រហែល ១៩ ម៉ែត្រ (៦០ ជើង) ឬជាភាគបីនៃផ្លូវទៅកណ្ដាលសំណង់។ ចុងផ្លូវនោះមានបន្ទប់តូចបីបណ្តាលពីបន្ទប់កណ្តាលធំមួយដែលមានដំបូលសិលាក្ដោបខ្ពស់... ជញ្ជាំងបង្វិលទាំងនោះកសាងដោយថ្មធំៗហៅថា orthostat ដែលមាន២២នៅខាងលិចនិង២១ នៅខាងកើត។ កម្ពស់មធ្យម ១.៥ ម៉ែត្រ”} \cite{70}។ មានព័ត៌មានអំពីវិស្វកម្មការពារទឹកជ្រាបយ៉ាងប្រុងប្រយ័ត្ន។ ឧទាហរណ៍នៅដំបូល \textit{“រន្ធក្នុងកម្រិតរុក្ខជាតិដែងកោសដោយការលាយនឹងដីឆេះនិងផ្សំព្រៃដើម្បីទប់ទឹកឱ្យជ្រាប និងពីសំណុំសំណុំនេះ បានប្រើកាលបរិច្ឆេទធូរកាបូន២គេលើផ្នែករចនាសម្ព័ន្ធនៃបរអធិប្បាយដែលមានកាលបរិច្ឆេទទៅវិញទៅមកជាង ២៥០០ ឆ្នាំមុន​គ.ស."} \cite{71}។ ភាពខ្ពស់បន្ថែមដឹកនាំទៅកាន់បន្ទប់ខាងក្នុងវាអាចចាត់ទុកថាជាមូលហេតុដូចគ្នា៖ \textit{“ព្រោះថា\_floor\_ ផ្លូវ និងបន្ទប់សាកសពនោះ ស្ថិតតាមថ្ងៃនៃគំនូសរក្សាទំហំរបស់បរ វិញវាមានភាពខុសគ្នាប្រហែល ២ ម៉ែត្រពីច្រកចូលទៅក្នុងបន្ទប់”} \cite{71}។

\begin{figure}[b]
\begin{center}
% \fbox{\rule{0pt}{2in} \rule{0.9\linewidth}{0pt}}
   \includegraphics[width=1\linewidth]{dolmen.jpg}
\end{center}
   \caption{Dolmen de Soto, ស្ប៉ាញ \cite{53}.}
\label{fig:9}
\label{fig:onecol}
\end{figure}

ការខ្វះខាតនៃក្តមរបស់មនុស្សនៅខាងក្នុងក៏ជាចំណុចចម្លែកផងដែរ។ ការជីករកឃើញចំណិតឆ្អឹងដែលឆេះ និងមិនឆេះ បង្ហាញពីមនុស្សប៉ុន្មាននាក់ ដែលសំណព្វជុំវិញផ្លូវកណ្តាល។ ការសង់ Newgrange ត្រូវបានប៉ាន់ប្រមាណថាចំណាយពេលយ៉ាងតិចជីវិតជាច្រើនជំនាន់ ទាញយកពីកាលបរិច្ឆេទកាបូននៃសម្ភារៈនៅខាងក្នុង។ ហេតុអ្វីសហគមន៍បុរាណមួយធ្វើការខិតខំដ៏ធំធេងដើម្បីសាងសង់ហៅតំបន់បូជា ណាស់ៗមួយដ៏ធំនិងត្រឹមត្រូវបែបនិស្ស័យនិងយកឆ្អឹងរបស់អ្នកស្លាប់តិចមករាយបាលនៅផ្លូវចេញចូលតែម្តង? ភាពមានហេតុផលបំផុតគឺរចនាសម្ព័ន្ធមេហ្គាលីតបែបបុរាណបះបោនទឹកនិងប្រុងប្រយ័ត្នក្នុងការសាងសង់នោះ ពិតជា ត្រូវបានបង្កើតឡើងនិងប្រើសម្រាប់ការឯកោរបស់មនុស្ស ដើម្បីការពារមនុស្សក្រោមមហន្តរាយធម្មជាតិធំធេងដែលកើតឡើងប្រចាំពិភពលោក។

នៅ Huelva ខាងត្បូងនៃស្ប៉ាញ គំរូដូចគ្នាដោយសង្ខេបបានជាពីរជាន់គឺ Dolmen de Soto (រូប \ref{fig:9}) ដែលជាតំបន់មួយក្នុងចំណោមប្រមាណ២០០ ក្នុងតំបន់នោះ \cite{72,32}។ វាជាស្ថានសង់ស្ដើងប្រែប្រួល ប្រើថ្មយក្សក្នុងការសាងសង់ ហើយមានអង្កត់ផ្ចិត ៧៥ ម៉ែត្រ។ ហើយតាមការចងក្រងតាមការជីករកថា មានតែលាសពចំនួនប្រាំបីតែប៉ុណ្ណោះដែលត្រូវបានរកឃើញ និងទាំងអស់សុទ្ធតែស្ថិតនៅសភាពដេកមូល។

\section{ការចង្អុលបង្ហាញអנומាលីពិសេសៗ}

នៅក្នុងភាគនេះ ខ្ញុំសង្ខេបនូវអנומាលីដែលគួរបញ្ជាក់បន្ថែមមួយចំនួន ដូចជា ទាំងអស់ត្រូវបានបកស្រាយយ៉ាងច្បាស់ដោយមហន្តរាយស្រដៀង ECDO។

\subsection{អនុមាត្រ ជីវវិទ្យា}

\begin{figure}[b]
\begin{center}
% \fbox{\rule{0pt}{2in} \rule{0.9\linewidth}{0pt}}
   \includegraphics[width=1\linewidth]{bottleneck.jpg}
\end{center}
   \caption{ការបង្អាក់ប្រភេទក្រុមហ៊ុនសណ្ឋានសរសៃស្ត្រីបុរសប្រហែល ៩៥\% ប្រហែលមកពី ៦,០០០ ឆ្នាំមុន \cite{62}.}
\label{fig:10}
\label{fig:onecol}
\end{figure}

ប异常ជីវៈឯកតាដ៏សំខាន់មួយៗ មានការបង្អាក់ប្រភេទក្រុមហ៊ុនសណ្ឋានសរសៃ និងរំដួលឆ្លងទន្លេវ៉ាលឆ្លង។ Zeng et al. (2018) បានចុះគន្លង ១២៥ រចនាសម្ព័ន្ធ Y-chromosome ពីមនុស្សសម័យទំនើប ហើយអាស្រ័យលើភាពស្រដៀងនិងការប្រែប្រួលក្នុង DNA ទទួលបានការខាតបន្ថយប្រជាជនប្រហែល ៩៥\% នៅក្នុងប្រជាជនបុរសប្រហែល ៥,០០០ ទៅ ៧,០០០ ឆ្នាំមុន (រូប \ref{fig:10}) \cite{62}។ រំដួលវាលទន្លេត្រូវបានរកឃើញនៅជាន់រូឡើងជាងផ្ទៃសមុទ្ររយៈម៉ែត្ររយៗ នៅស៊្វេដែនប៊ក មិគីហ្គាន វឺម៉ុន កាណាដា ឆីឡេ និងអ៊ីហ្ស៊ីប \cite{63,64,65,66}។ ជនសត្វទាំងនេះត្រូវបានរកឃើញនៅក្នុងសភាពផ្សេងៗគ្នា៖ មានរូបរាងរឹងមាំ ឬសពញឹកញាប់ ស្ថិតនៅលើសំណល់ទឹកកក ឬប៉ុន្តែក៏មានសពបិទមិនឃើញក្នុងស្រទាប់ដី។ ចំនួនសត្វនៅតំបន់ទាំងនេះមានចាប់ពីប៉ុន្មានដល់ជាងរយ។ ត្រីវាលទន្លេជាសត្វរស់នៅជ្រៅសមុទ្រនិងកម្រស្ថិតនៅជិតឆ្នេរនាវា។ តើសត្វទាំងនេះមកដល់កំពូលជួរភ្នំនេះដោយរបៀបណា? ហើយនៅចម្ងាយឆ្ងាយពីសមុទ្រប៉ុន្មាន?

មានការបាត់បង់ជីវចម្រើនយ៉ាងច្រើនកើតឡើងក្នុងអតីតកាលផែនដី ដែលស្រាវជ្រាវច្រើនបំផុតគឺ "ប្រាំធំ" របស់សម័យ Phanerozoic៖ ព្រឹត្តិកត្តិសំខាន់ក្នុងសម័យ Late Ordovician (LOME), Late Devonian (LDME), បញ្ចប់-Permian (EPME), បញ្ចប់-Triassic (ETME) និងបញ្ចប់-Cretaceous (ECME) \cite{88,89}។ គួរឲ្យចាប់អារម្មណ៍ពីការបាត់បង់ជីវចម្រើនមួយចំនួននេះត្រូវបានចាត់ថាជាកើតឡើងក្នុងយុគសម័យដូចគ្នាជាមួយស្រទាប់ជាច្រើនរបស់តំបន់ក្រហមធំ (Grand Canyon) គឺស្រទាប់ Permian និង Devonian។

\subsection{អសកម្មភាពរូបវន្ត}

\begin{figure}[b]
\begin{center}
% \fbox{\rule{0pt}{2in} \rule{0.9\linewidth}{0pt}}
   \includegraphics[width=1\linewidth]{columbia.jpg}
\end{center}
   \caption{រលកចលនា​ធំៗ​នៅក្នុង​បារាយណ៍​ទឹកកក Columbia រដ្ឋវ៉ាស៊ីងតោន \cite{80}.}
\label{fig:11}
\label{fig:onecol}
\end{figure}

មាន​ផ្ទៃដី​ជាច្រើន​ក្រៅពី​ប្រាសាទ Grand Canyon ដែល​ទំនងជា​បានបង្កើត​ឡើង​ដោយ​កម្លាំង​ធម្មជាតិ​ធំធេង។​ ភស្តុតាង​អំពី​ការហូរទឹក​ធំទូលាយ​ក្នុង​សេរ៉ាណា​អាណាចក្រ​អាចឃើញ​រលក​ចលនា​ធំៗ​នៅជុំវិញ​ពិភពលោក។ ตัวอย่าง​មួយ​គឺ Scablands ដែល​មាន​ផ្លូវឆ្លង​បណ្ដោយ​នៅ​ភាគ​ខាងជើង​អាស៊ីប៉ាស៊ីហ្វិក។ នៅទីនេះ មិនត្រឹមតែមាន​ផ្ទៃដី​សមុទ្រដ្រាប​សមាសន៍ និង​ថ្ម​ធំ​ទម្លាក់ព្រោះ​ទេ ប៉ុន្តែ​ត្រូវ​បានរកឃើញ​រលក​ធំជាច្រើន​ជាង​មួយរយជួរ ដែលកើតពី​កម្លាំងហូរទឹក​ធំ \cite{78,79}។ គឺជា​កំណែក្រិត​ធំជាង​នៃ​រលក​ដែលកើតឡើងលើ​ដំបកខ្សាច់របស់​ទន្លេ។ រលកទាំងនេះអាច​រកឃើញ​នៅថ្នល់ជាច្រើន​បរទេស​ដូចជា​បារាំង អាហ្សង់ទីន រូបស៊ី និង​កោះអាមេរិកខាង​ជើង \cite{81}។ រូបទី \ref{fig:11} បង្ហាញ​រលក​មួយ​ចំនួន​នៅ​រដ្ឋវ៉ាស៊ីងតោន សហរដ្ឋអាមេរិក \cite{80}។

\begin{figure}[b]
\begin{center}
% \fbox{\rule{0pt}{2in} \rule{0.9\linewidth}{0pt}}
   \includegraphics[width=1\linewidth]{zhangjiajie.jpg}
\end{center}
   \caption{ស្នាដៃ​ថ្មធំនៅ​ព្រៃជាតិ Zhangjiajie ភាគខាងត្បូង​ចិន។}
\label{fig:12}
\label{fig:onecol}
\end{figure}

\begin{figure}[b]
\begin{center}
% \fbox{\rule{0pt}{2in} \rule{0.9\linewidth}{0pt}}

   \includegraphics[width=1\linewidth]{hoy.jpg}
\end{center}
   \caption{ស្ថប់ឆ្នេរសមុទ្រ Old Man of Hoy, ស្កុតឡង់ដ៍ \cite{83}.}
\label{fig:13}
\label{fig:onecol}
\end{figure}

រចនាសម្ព័ន្ធកូរ៉ូសិនក្នុងដីក៏ត្រូវបានពន្យល់យ៉ាងល្អដោយការបង្រែផែនដីស្រដៀងกับ ECDO។ ប្រទេសចិនខាងត្បូងជាគំរូល្អនៃទេសភាពកាស្ត៍ធំទូលាយ ដែលកើតឡើងដោយការច្រេះនៃទឹក \cite{82}។ ទេសភាពទាំងនេះរួមមានតួកាស្ត៍ (tower karst), ជើងស្នាមកាស្ត៍ (pinnacle karst), ពងកាស្ត៍ (cone karst), ស្ពានធម្មជាតិ, ច្រាំងទន្លេ, គ្រឿងប្រព័ន្ធផ្លូវអណ្តើកធំ, និងរន្ធធ្លាក់។ យ៉ាងពិសេសមួយក្នុងចំណោមទាំងនេះគឺឧទ្យានជាតិស្វាយហ្សាប់ជី (Zhangjiajie National Forest) ដែលមានជើងស្ពានកវ៉ាតសសាន់ដ៍ស enormes (Figure \ref{fig:12}) \cite{84}។ ជើងស្ពានទាំងនេះមានកម្ពស់ជាមធ្យមលើស ១,០០០ ម៉ែត្រ ហើយចំនួនលើស 3,100។ ច្រើនជាង ១,០០០ ក្នុងនោះកម្ពស់លើស ១២០ ម៉ែត្រ ហើយ ៤៥ មានកម្ពស់លើស ៣០០ម៉ែត្រ \cite{85}។ ជើងស្ពានទាំងនេះមានសភាពស្រដៀងនឹងជើងស្ពានសមុទ្រ (sea erosion pillars) (Figure \ref{fig:13}) ដែលជាឈើស្រឡាញ់ថ្មនៅឆ្នេរសមុទ្របង្កើតឡើងដោយការរលំជុំវិញដោយចលនារលកសមុទ្រ។ ទេសភាពជ្រោះកូរ៉ូសិនស្រដៀងគ្នាគេអាចរកឃើញនៅក្នុងតង់ថ្មនៅ Urgup ប្រទេសតួកគី និងនៅ Ciudad Encantada ប្រទេសអេស្ប៉ាញ ដែលទាំងពីរមានកម្ពស់លើស ១,០០០ ម៉ែត្រ លើសមុទ្រ។ ទីតាំងទាំងអស់នេះមានសតវត្សរ៍ជាមួយសមាសធាតុអំបិលនិងអាណាពន្ធសមុទ្រដ៏ជិតស្និទ្ធបង្ហាញពីការចូលមករបស់សមុទ្រពីអតីតកាល \cite{15,86,87}។ យ៉ាងណាមិញ រឿងព្រេងទឹកជំនន់ \cite{3} បានរាយការណ៍ថាសមុទ្រឡើងខ្ពស់ជាង ១,០០០ ម៉ែត្រ ហើយយើងទទួលស្គាល់ច្បាស់លាស់ដោយវត្តមានស្រូវនិងវាលអំបិលធំៗ នៅ Andes និង Himalayas ដែលខ្ពស់ជាច្រើនគីឡូម៉ែត្រលើសមុទ្រ។ ឧទាហរណ៍វាលអំបិល Uyuni នៅបុូលីវីមានកំពស់ដល់ទៅ ៣៦៥៣ ម៉ែត្រ លើសមុទ្រ \cite{94}។

\subsection{ព្រឹត្តិការណ៍ផ្លាស់ប្ដូរអាកាសធាតុយ៉ាងលឿន}

អត្ថបទវិទ្យាសាស្ត្រសម័យទំនើបទទួលស្គាល់ពីការប្រព្រឹត្ដនៃព្រឹត្តិការណ៍ផ្លាស់ប្ដូរអាកាសធាតុជាសកលយ៉ាងលឿនក្នុងប្រវត្តិសាស្ត្រថ្មីៗនៃផែនដី។ ឧទាហរណ៍ចម្បងពីរគឺព្រឹត្តិការណ៍ 4,200 ឆ្នាំ និង 8,200 ឆ្នាំ ដែលទាំងពីរចម្លងស្រដៀងជាមួយការថយចុះប្រជាជន និងការរំខានដល់សង្គមនៅលើផ្ទៃដីធំបរិយាបន្ន។ ព្រឹត្តិការណ៍ទាំងនេះត្រូវបានរក្សាទុកជាការបំផ្លាញក្នុងស្នោត sediment និងគន្លឹះទឹកកក, កំបោរ fossil, តម្លៃអេសូតូប O18,កំណត់ត្រាប៉ូលេននិងស្ពេលូថឹម, និងទិន្នន័យកម្រិតទឹកសមុទ្រ។ ការប្រែប្រួលអាកាសធាតុសន្និដ្ឋានមានfree drop សីតុណ្ហភាពនៃផែនដីជាសាកល, ការរងួត, ការរំខានដល់ស្រទាប់លំដាប់ផ្ទៃទឹកតំបន់ Atlantic, និងការទម្លាក់ទឹកកក \cite{90,91,92}។ ព្រឹត្តិការណ៍ 8,200 ឆ្នាំជាពិសេសកើតឡើងជាមួយនឹងការដាក់សំពាធដោយអំបិលទៅលើសមុទ្រ Black កន្លែងប្រហែលឆ្នាំ ៦៤០០ មុនគ.ស. \cite{93}។

\subsection{គម្លាតបុរាណវិទ្យា}

សក្ខីកម្មបុរាណវិទ្យានៃទីក្រុងបុរាណខ្លះបង្ហាញពីស្រទាប់ជាច្រើននៃការបាំងនិងការបំផ្លាញ បង្កើតកំណត់ត្រានៃព្រឹត្តិការណ៍ធម្មជាតិធំៗ។ ទីក្រុងបុរាណ Jericho គឺជាឧទាហរណ៍មួយ ស្ថិតនៅប្រទេសប៉ាលេស្ទីនសម័យទំនើប។ វាមានស្រទាប់បំផ្លាញជាច្រើន ដោយមានការរលំទំាងសងខាងនៃរចនាសម្ព័ន្ធថ្មនិងអគ្គិភ័យខ្លាំង \cite{96,97}។ ប្រវត្តិសាស្ត្រកំណត់ត្រានៅក្នុងស្រទាប់ទាំងនេះចាប់ពីប្រហែល ៩,០០០ មុនគ.ស. ដល់ ២,០០០ មុនគ.ស.។ ជាពិសេសគឺប៉មរបស់វាជាមុខងារត្រូវបានកាត់បន្ថយនិងបុណ្យថ្មជិតប្រហែលឆ្នាំ 7400 មុនគ.ស. (Figure \ref{fig:14}) \cite{95}។ Catal Huyuk \cite{99}, Gramalote \cite{98}, និងសាសន្យមិនូអាន Knossos នៅកោះ Crete \cite{100,101} ទាំងអស់សុទ្ធសឹងតែជាឧទាហរណ៍នៃបុរាណវិទ្យាមានស្រទាប់ច្រើន ច្រើនពាក្យមានសក្ខីកម្មនៃការបំផ្លាញ។

\begin{figure}[t]
\begin{center}
% \fbox{\rule{0pt}{2in} \rule{0.9\linewidth}{0pt}}

   \includegraphics[width=1\linewidth]{jericho.jpg}
\end{center}
   \caption{ការស្តារ​បុរាណវិទ្យា​នៃ​ការ​បញ្ចុះ​សព​ផ្ទៃតួកំពូល Jericho ប្រហែលឆ្នាំ ៧៤០០ មុន​គ.ស. \cite{95}.}
\label{fig:14}
\label{fig:onecol}
\end{figure}

ភស្តុតាងម្ដងទៀតសម្រាប់ព្រឹត្តិការណ៍ធំៗដែលបំផ្លាញអភិវឌ្ឍន៍មនុស្សគឺ​រូបភាព Nampa ដែលជារូបអ្នកតោនភិលានៃសិប្បកម្មតំរូវត្រូវបានរកឃើញនៅក្រោមមន្ទិលជើងប្រហែល២០០ម៉ែត្រនៃភក់ភ្នំភ្លើង​នៅ​រដ្ឋ Idaho \cite{102,103}។ ភក់ភ្នំភ្លើងដែលរកឃើញរូបចម្លាក់នេះត្រូវបានប៉ាន់ស្មានថាត្រូវបានដាក់កំលាំងពេលអំឡុងកសិកម្មពពួក Tertiary ចុង ឬ Quaternary ដំបូង មានអាយុប្រហែល២លានឆ្នាំ។ ទោះយ៉ាងណា ភក់ភ្នំភ្លើងក្នុងតំបន់នេះតែមើលទៅថាជាក់ស្តែងថាមានសភាពថ្មី។ ការរកឃើញបែបនេះមិនត្រឹមតែបញ្ជាក់ពីព្រឹត្តិការណ៍ធំបំផ្លាញអភិវឌ្ឍន៍មនុស្សទេ ប៉ុន្ដែហៅឱ្យមានការសង្ស័យចំពោះវិធីសាស្ត្រចែងកាលវិភាគសព្វថ្ងៃផងដែរ។

\section{អំពីវិធីសាស្ត្រកំណត់អាយុបច្ចុប្បន្ន}

មានមូលហេតុសំខាន់ដែលត្រូវសង្ស័យនូវកាលវិភាគសព្វថ្ងៃ ដោយយកអាយុយូរណាស់រហូតដល់លាន ឬនេតសិប្បនិម្មិតរយៈពេលរយៈពាន់លានឆ្នាំដល់សម្ភារៈផ្ទាល់ខ្លួនផ្សេងៗ។

ប្រាប់តាមរបាយការណ៍គគ្រឹកគតិ គេនិយាយថាឥន្ធនៈមាត់ដុំៗដូចជាឥន្ធនៈថ្មខ្មៅ ប្រេងនិងឧស្ម័នធម្មជាតិ មានអាយុរយៈពាន់លានឆ្នាំ \cite{104}។ ប៉ុន្តែការធ្វើតេស្តកាបូនលើប្រេងនៅឈូងសមុទ្រមិចស៊ិកាក៏អោយអាយុប្រហែល ១៣,០០០ ឆ្នាំសម្រាប់ប្រេងនោះផងដែរ \cite{105}។ កាបូន-១៤ មានអាយុកាលពាក់កណ្តាលខ្លីណាស់ (៥,៧៣០ ឆ្នាំ) ដូច្នេះវាគួរតែពុំមានទេបន្ទាប់ពីរយៈពេលរយៈសត្វពាន់ឆ្នាំ។ ទោះជាយ៉ាងណា វាក៏ត្រូវបានរកឃើញនៅក្នុងថ្មខ្មៅ និងអណ្តើកដ៏សន្មតថាមានអាយុជាងពាក់កណ្តាលពាន់ដង \cite{106}។ ជាករណីពិត កាបូនត្រូវបានផលិតក្លែងបន្លំក្នុងមន្ទីរពិសោធន៍ក្រោមល្បឿនកម្ដៅខ្ពស់ក្នុងរយៈពេលត្រឹមតែ២-៨ខែ \cite{107}។

វិធីសាស្ត្រកំណត់អាយុកាលដោយរ៉ាឌីយ៉ូអ៊ីសូតូបក្រៅពីកាបូនក៏អាចមិនត្រឹមត្រូវដែរ។ ក្រុមស្រាវជ្រាវ Answers in Genesis បានរកឃើញភាពមិនស៊ិចផ្សេងៗគ្នាក្នុងកាលបរិច្ឆេទដែលទាញយកពីវិធីសាស្ត្រនេះដែលធ្វើឱ្យមានសំណួរសម្រាប់ភាពត្រឹមត្រូវរបស់វា \cite{108}។ សាច់ទន់ដែលមានកោសិកាឈាម បណ្តាញឈាម និងកូឡាជែនត្រូវបានរកឃើញក្នុងសំបូរមាន់សូរាដែលសន្មតថាមានអាយុរយៈសត្វពាន់លានឆ្នាំ \cite{109,110}។ តាមអ្វីដែលយើងដឹង ប្រហែលជាវាអាចបង្ហាញថាកាលបរិច្ឆេទត្រឹមត្រូវទាំងអស់នៃវគ្គបរិយាកាសផែនដី និងសម្ភារៈរឹងថ្មនិងអ្វីៗដែលទាក់ទងនឹងឥន្ធនៈអាចមិនត្រឹមត្រូវនោះដែរ។ 

\section{និស្សិត}

ក្នុងអត្ថបទនេះ ខ្ញុំបានយកឯកសារដ៏ជូតជាច្រើនបំផុតដែលបញ្ជាក់ពីប្រភពគ្រោះថ្នាក់ធ្ងន់ធ្ងរដែលពុំអាចបកស្រាយបាន ដោយសមស្របបំផុតជាមួយមូលដ្ឋានបក្សពួកប្រែផែនដី ECDO។ ទោះបីចម្រុះបែបណាក៏ដោយ ការប្រមូលផ្ដុំនេះមិនទាន់ពេញលេញឡើយ - មានបាតុភូតចម្លែកបន្ថែមទៀត ត្រូវបានប្រមូលផ្ដុំ ហើយអាចរកបានសាធារណៈនៅក្នុងធនធាន GitHub ស្រាវជ្រាវរបស់ខ្ញុំផងដែរ \cite{2}។
\section{សេចក្ដីអរគុណ}

អរគុណចំពោះ Ethical Skeptic, អ្នកនិពន្ធដើមនៃធិសេស ECDO សម្រាប់បញ្ចប់នូវធិសេសដ៏ជ្រៅលំនឹងនិងបង្កើតអ្វីថ្មីរបស់លោក ហើយចែករំលែកវាជាមួយពិភពលោក។ ធិសេសបីផ្នែករបស់លោក \cite{1} នៅតែជាការងារដែលមានអាជ្ញាធរច្បាស់លាស់សម្រាប់ទ្រឹស្តី Exothermic Core-Mantle Decoupling Dzhanibekov Oscillation (ECDO) ហើយមានព័ត៌មានច្រើនជាងនេះទៀតអំពីប្រធានបទនេះជាងដែលខ្ញុំបានសង្ខេបខ្លីនៅទីនេះ។

ហើយជាក់ស្តែងថា អរគុណចំពោះយក្សដែលយើងឈរលើប៉ុនៗរបស់ពួកគេ ទាំង​អ្នក​ដែល​បានធ្វើការស្រាវជ្រាវ​និងស៊ើបអង្កេតទាំងអស់ដែលបានធ្វើឲ្យការងារនេះអាចប្រព្រឹត្តទៅ និងបានខិតខំដើម្បីនាំអោយមានពន្លឺដល់មនុស្សជាតិ។

\clearpage
\twocolumn

{\small
\renewcommand{\refname}{តំណាងអត្ថបទយោង}
\bibliographystyle{ieee}
\bibliography{egbib}
}

\end{document}
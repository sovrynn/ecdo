\documentclass[10pt,twocolumn,letterpaper]{article}

% Мои собственные вещи
\usepackage{booktabs}
% \usepackage{caption}
% \captionsetup[table]{skip=8pt}   % Влияет только на таблицы
\usepackage{stfloats}  % Добавьте это в преамбулу


\usepackage[T2A]{fontenc}    % Enables Cyrillic fonts
\usepackage[utf8]{inputenc}   % Assumes your source file is in UTF-8 encoding
\usepackage[russian]{babel}   % Loads Russian language support

% \makeatletter
% \def\cvprsubsection{\@startsection {subsection}{2}{\z@}
%     {8pt plus 2pt minus 2pt}{6pt}{\bfseries\normalsize}}
% \makeatother

\usepackage{cvpr}
\usepackage{times}
\usepackage{epsfig}
\usepackage{graphicx}
\usepackage{amsmath}
\usepackage{amssymb}

% Включайте другие пакеты здесь, перед hyperref.

% Если вы закомментируете hyperref, а затем раскомментируете его, вы должны удалить
% egpaper.aux перед повторным запуском latex. (Или просто нажмите 'q' при первом запуске latex,
% дайте ему завершиться, и всё будет чисто).

\usepackage[breaklinks=true,bookmarks=false]{hyperref}

\cvprfinalcopy % *** Раскомментируйте эту строку для финальной подачи

\def\cvprPaperID{****} % *** Введите здесь ID статьи для CVPR
\def\httilde{\mbox{\tt\raisebox{-.5ex}{\symbol{126}}}}

\renewcommand{\tablename}{Таблица}
\renewcommand{\figurename}{Рисунок}   % or whatever you like instead of "Hình"
\renewcommand{\refname}{Список литературы}

\makeatletter
\def\abstract{%
  \centerline{\large\bf абстракт}% <-- your new label
  \vspace*{12pt}%
  \it%
}
\makeatother

% This makes the font slightly bigger than base (10) and bold in Subsection headings rather than using ptmb
\makeatletter
\def\cvprsubsection{%
  \@startsection{subsection}{2}{\z@}%
    {8pt plus 2pt minus 2pt}{6pt}%
    % {\normalfont\bfseries\selectfont}%
    {\normalfont\bfseries\fontsize{11}{13}\selectfont}%
}
\makeatother

% So this hardcodes the style for the numbers in the section/subsection headings so they're bold
\font\elvbf=ptmb scaled 1100
\font\elvbfs=ptmb scaled 1200
\makeatletter
% Section number: Large + bold
\renewcommand\thesection{%
  {\elvbfs\arabic{section}}%
}

% Subsection number: normalsize + bold + custom punctuation
\renewcommand\thesubsection{%
  {\elvbf
   \arabic{section}.\arabic{subsection}}%
}
\makeatother

% Страницы нумеруются в режиме подачи, и не нумеруются в финальной версии
%\ifcvprfinal\pagestyle{empty}\fi
\setcounter{page}{1}
\begin{document}

%%%%%%%%% TITLE
\title{ECDO, руководство на основе данных, Часть 2/2: Исследование научных и исторических аномалий, наилучшим образом объясняемых “переворотом Земли” по ECDO}

\author{Джунхо\\
Опубликовано в феврале 2025 г.\\
Веб-сайт (здесь можно скачать статьи): \href{https://sovrynn.github.io}{sovrynn.github.io}\\
Репозиторий исследования ECDO: \href{https://github.com/sovrynn/ecdo}{github.com/sovrynn/ecdo}\\
{\tt\small junhobtc@proton.me}
% Для статьи, авторы которой все находятся в одном учреждении,
% пропустите следующие строки до закрывающей ``}''.
% Дополнительные авторы и адреса могут быть добавлены с помощью ``\and'',
% так же, как и второй автор.
% Для экономии места используйте либо адрес электронной почты, либо домашнюю страницу, но не оба
% \and
% xx
% Institution2\\
% Первая строка адреса institution2\\
% {\tt\small secondauthor@i2.org}
}

\maketitle
%\thispagestyle{empty}

%%%%%%%%% ABSTRACT
\begin{abstract}
В мае 2024 года анонимный онлайн-автор под псевдонимом “The Ethical Skeptic” \cite{0} опубликовал революционную теорию под названием Экзотермическое разделение ядра-мантии и эффект Джанибекова (ECDO) \cite{1}. Эта теория не только утверждает, что Земля ранее переживала внезапные катастрофические сдвиги вращательной оси, вызывая всемирный потоп из-за того, что океаны переливались через континенты по инерции вращения, но также предлагает объясняющий причинно-геофизический процесс наряду с данными, указывающими на то, что подобный переворот может быть неизбежен в ближайшее время. Хотя подобные предсказания катастрофических потопов и конца света не являются новыми, теория ECDO уникально убедительна благодаря своему научному, современному, междисциплинарному и основанному на данных подходу.

Данная научная статья составляет вторую часть двухчастного краткого обзора 6 месяцев независимых исследований \cite{2,20} теории ECDO, с особым вниманием к научным и историческим аномалиям, которые лучше всего объясняются катастрофическим «переворотом Земли» по ECDO.

\end{abstract}

%%%%%%%%% BODY TEXT

\section{Введение}

Современная униформитарная геология и история утверждают, что такие крупные геологические ландшафты, как Гранд-Каньон, сформировались за миллионы лет \cite{143}; что соль существует в Долине Смерти (Калифорния), потому что когда-то она находилась под океаном сотни миллионов лет назад \cite{144}; что наши предки в 150 поколениях назад всю жизнь строили гигантские гробницы \cite{29,70}; и что так называемые "ископаемое топливо" образовалось сотни миллионов лет назад \cite{104}. Возможно, наиболее поразительно то, что считается, будто человечеству 300000 лет \cite{145}, хотя письменная история и цивилизация насчитывают лишь около 5000 лет — это эквивалентно 150 поколениям человека.

Как увидим далее, такие аномалии лучше всего объясняются катастрофическими геологическими силами.

\section{Мгновенно замороженные мамонты, погребённые в грязи}

\begin{figure}[t]
\begin{center}
% \fbox{\rule{0pt}{2in} \rule{0.9\linewidth}{0pt}}
   \includegraphics[width=1\linewidth]{jarkov-mammoth.jpg}
\end{center}
   \caption{Мамонт Жаркова, идеально сохранившийся сибирский мамонт возрастом 20000 лет, найденный в замёрзшей грязи \cite{51}.}
\label{fig:1}
\label{fig:onecol}
\end{figure}

Одна из таких категорий аномалий — идеально сохранившиеся мамонты, мгновенно замороженные и погребённые в грязи, которые часто встречаются в арктических регионах (рисунок \ref{fig:1}). Берёзовский мамонт, обнаруженный в Сибири, погребённый в иловом гравии, был настолько хорошо сохранён, что его мясо оставалось съедобным спустя тысячи лет после его смерти. В его рту и желудке также была найдена растительная пища, что поставило учёных в тупик — как он мог быть заморожен настолько быстро, если он пасся на цветущих растениях прямо перед смертью \cite{17}. Сообщается, что \textit{"В 1901 году сенсацией стало обнаружение полного трупа мамонта возле реки Берёзовка, так как казалось, что это животное погибло от холода в середине лета. Содержимое его желудка было хорошо сохранено и включало лютики и цветущие дикие бобы: это означало, что они были проглочены примерно в конце июля или в начале августа. Существо погибло настолько внезапно, что у него в челюсти ещё оставалась горсть трав и цветов. Очевидно, оно было захвачено огромной силой и отброшено на несколько миль от своего пастбища. Таз и одна нога были сломаны — огромное животное было сбито с ног и затем замёрзло до смерти, в то время, которое обычно считается самым жарким в году"} \cite{18}. Кроме того, \textit{"[Российские учёные] зафиксировали, что даже самая внутренняя оболочка желудка животного имела прекрасно сохранившуюся волокнистую структуру, что говорит о том, что тепло тела было удалено каким-то сверхмощным природным процессом. Сандерсон, обратив особое внимание на этот один момент, обратился к Американскому институту замороженных продуктов с вопросом: что нужно, чтобы заморозить целого мамонта так, чтобы даже во внутренних частях его тела, даже во внутренней оболочке желудка, содержавшаяся там влага не успела бы образовать кристаллы такого размера, которые бы разрушили волокнистую структуру мяса?... Через несколько недель институт дал Сандерсону ответ: Это абсолютно невозможно. Со всем нашим научным и инженерным знанием не существует ни одного известного способа удалить тепло из туши размером с мамонта настолько быстро, чтобы заморозить её без образования крупных кристаллов влаги в мясе. Более того, испробовав все научные и инженерные методы, они обратились к природе и пришли к выводу, что не существует ни одного известного природного процесса, способного добиться такого достижения"} \cite{19}.

\section{Гранд-Каньон}

Гранд-Каньон, часть Большого бассейна на юго-западе Северной Америки, — ещё одно природное явление, указывающее на катастрофические происхождения (рисунок \ref{fig:2}). Для начала, осадочные слои песчаника и известняка, составляющие Гранд-Каньон, охватывают огромные площади до 2,4 миллиона км$^2$ \cite{21}. На рисунке \ref{fig:3} показан масштаб распространения слоя песчаника Коконино по западным штатам США. Такие массивные горизонтальные слои однородного материала могли быть отложены только одновременно.

\begin{figure}[b]
\begin{center}
% \fbox{\rule{0pt}{2in} \rule{0.9\linewidth}{0pt}}
   \includegraphics[width=1\linewidth]{grand-canyon.jpg}
\end{center}
   \caption{Гранд-Каньон, Аризона, США \cite{49}.}
\label{fig:2}
\label{fig:onecol}
\end{figure}

\begin{figure}[t]
\begin{center}
% \fbox{\rule{0pt}{2in} \rule{0.9\linewidth}{0pt}}
   \includegraphics[width=1\linewidth]{coconino.jpg}
\end{center}
   \caption{Размер слоя песчаника Коконтино на западе США \cite{21}.}
\label{fig:3}
\label{fig:onecol}
\end{figure}

Более пристальный взгляд на Гранд-Каньон показывает, что отложение этих обширных слоев осадочных пород также происходило одновременно с воздействием значительных тектонических сил. Чтобы понять это, мы должны внимательно рассмотреть определённые участки каньона, где слои осадочных пород были изогнуты и обнажены. Исследователи из Answers in Genesis \cite{42} провели микроскопическое исследование образцов пород из некоторых складок, таких как Monument Fold, и, основываясь на отсутствии признаков, которые должны были бы присутствовать, если бы складки образовывались в течение длительного времени под воздействием тепла и давления, пришли к выводу, что слои осадочных пород были изогнуты тектоническими силами, пока они еще были мягкими, то есть вскоре после их отложения \cite{43}.

\begin{figure*}
\begin{center}
% \fbox{\rule{0pt}{2in} \rule{.9\linewidth}{0pt}}
\includegraphics[width=1\textwidth]{Grand_Staircase-big.jpg}
\end{center}
   \caption{Слои осадочных пород, образующих Гранд-Каньон (правая сторона изображения), простираются прямо на север до Сидар-Брэйкс, Юта (левая сторона изображения), где они все изгибаются вверх \cite{50}.}
\label{fig:4}
\end{figure*}

Уменьшая масштаб, мы обнаруживаем, что слои, составляющие Гранд-Каньон, были сложены не только внутри каньона. Слои были сложены на востоке в моноклинали Ист-Кайбаб \cite{46}, но также и на севере в Сидар-Брейкс, Юта (рисунок \ref{fig:4}). Это говорит о том, что все эти слои могли быть сложены вместе после того, как они были отложены друг на друга в быстрой последовательности. Для справки, горизонтальные слои Гранд-Каньона имеют толщину приблизительно 1700 метров. Масштаб геологического процесса, необходимого для отложения осадочных слоев толщиной в милю, огромен.

Истинное образование Гранд-Каньона является еще одним спорным вопросом в современной геологии. Униформистская геология предполагает, что Гранд-Каньон был вырезан рекой Колорадо в течение миллионов лет \cite{47}. Однако исследовательская группа "Answers in Genesis" полагает, что Большой каньон, скорее всего, образовался в течение нескольких недель из-за водосбросной эрозии древнего озера, нарушившего его границы, что удалило огромное количество осадка, вырезая каньон. Существуют доказательства нахождения высокогорного озера к востоку от Гранд-Каньона в отложениях озерных осадков и морских ископаемых. Сравнение Гранд-Каньона с другими крупномасштабными примерами водосбросной эрозии, такими как каньон Афтон и гора Сент-Хеленс, выявляет схожую топографию и показывает, что большие каньоны могут быть созданы быстро за счет большого количества текущей воды \cite{48}.

Учитывая масштаб геологических процессов, необходимых для отложения осадочных пород на таких огромных участках суши, одновременное воздействие мощных тектонических сил, возникших вскоре после отложения осадочных слоев, и крошечные размеры реки Колорадо по сравнению с огромными масштабами Гранд-Каньона, кажется, что в его формировании не было ничего постепенного.

\section{Подземный город Деринкую}

Помимо пирамид, отличным примером древней инженерии является подземный город Деринкую (рис. \ref{fig:5}), расположенный в Каппадокии, Турция. Это самый крупный из более чем 200 подземных убежищ в данном регионе \cite{54}. Этот подземный город, по оценкам, мог вмещать до 20000 человек и насчитывает 18 этажей, достигая глубины 85 метров. Хотя его возраст точно неизвестен, считается, что ему не менее 2800 лет. Город был вырезан с мягкой вулканической породой \cite{52, 53}.

\begin{figure}[b]
\begin{center}
% \fbox{\rule{0pt}{2in} \rule{0.9\linewidth}{0pt}}
   \includegraphics[width=1\linewidth]{derinkuyu.jpeg}
\end{center}
   \caption{Схема подземного города Деринкую \cite{56}.}
\label{fig:5}
\label{fig:onecol}
\end{figure}

Причина, по которой Деринкую интересен, заключается в том, что непонятно, почему какое-либо сообщество решило построить целый город под землей. Чтобы создать жилое пространство под землей, каждая полость должна быть вырезана в скале. Грубые формы и текстуры подземных туннелей ясно показывают, что они были вырезаны вручную, а не с помощью электроинструментов, что было бы на порядок сложнее, чем строительство убежищ над землей. На самом деле, непонятно, почему какой-либо человек хотел бы постоянно жить под землей в пределах своей земной жизни, когда сельское хозяйство, солнечный свет, природа и исследования доступны только над землей. Традиционная «история» предполагает, что Деринкую был создан христианами, которым нужно было уединенное место для исповедания своей религии \cite{53}. Но здравый смысл подсказывает, что самый простой способ справиться с врагами — «сражаться или бежать», а не «вырезать подземный город из скалы».

Масштаб, глубина и продуманность дизайна подземного города ясно показывают, что он не был спроектирован как временное военное оборонительное сооружение для более эффективной борьбы с захватчиками в тяжелые времена, а скорее как долгосрочное убежище для защиты от фатальных сил на поверхности. Деринкую был оборудован не только базовыми спальнями, кухнями и ванными комнатами, но и конюшнями для животных, резервуарами для воды, хранилищами продуктов, винными и масляными прессами, школами, часовнями, гробницами и массивными вентиляционными шахтами (рисунок \ref{fig:6}). Зачем военному убежищу нужен был винный пресс и его нужно было вырыть на глубине 85 метров с такой сложностью?

Наиболее правдоподобным объяснением создания Деринкую могла бы стать острая необходимость в подготовке долгосрочного, самодостаточного убежища для защиты от катастрофических геофизических сил на поверхности Земли.

\begin{figure}[t]
\begin{center}
% \fbox{\rule{0pt}{2in} \rule{0.9\linewidth}{0pt}}
   \includegraphics[width=1\linewidth]{derinkuyu-air.jpg}
\end{center}
   \caption{Глубокий вентиляционный колодец в Деринкую \cite{53}.}
\label{fig:6}
\label{fig:onecol}
\end{figure}

% \section{Additional Anomalies Best Explained By An Earth Flip}

% Before wrapping up, we will mention some additional scientific anomalies that, once viewed in the context of cataclysmic geophysical forces, are well explained.

% --- TRANSLATED VERSION BELOW ---

% Причина, по которой Деринкую интересен, заключается в том, что непонятно, почему какое-либо сообщество решило построить целый город под землей. Чтобы создать жилое пространство под землей, каждая полость должна быть вырезана в скале. Грубые формы и текстуры подземных туннелей ясно показывают, что они были вырезаны вручную, а не с помощью электроинструментов, что было бы на порядок сложнее, чем строительство убежищ над землей. На самом деле, непонятно, почему какой-либо человек хотел бы постоянно жить под землей в пределах своей земной жизни, когда сельское хозяйство, солнечный свет, природа и исследования доступны только над землей. Традиционная «история» предполагает, что Деринкую был создан христианами, которым нужно было уединенное место для исповедания своей религии \cite{53}. Но здравый смысл подсказывает, что самый простой способ справиться с врагами — «сражаться или бежать», а не «вырезать подземный город из скалы».

% Масштаб, глубина и продуманность дизайна подземного города ясно показывают, что он не был спроектирован как временное военное оборонительное сооружение для более эффективной борьбы с захватчиками в тяжелые времена, а скорее как долгосрочное убежище для защиты от фатальных сил на поверхности. Деринкую был оборудован не только базовыми спальнями, кухнями и ванными комнатами, но и конюшнями для животных, резервуарами для воды, хранилищами продуктов, винными и масляными прессами, школами, часовнями, гробницами и массивными вентиляционными шахтами (рисунок \ref{fig:6}). Зачем военному убежищу нужен был винный пресс и его нужно было вырыть на глубине 85 метров с такой сложностью?

% Наиболее правдоподобным объяснением создания Деринкую могла бы стать острая необходимость в подготовке долгосрочного, самодостаточного убежища для защиты от катастрофических геофизических сил на поверхности Земли.

\begin{figure}[t]
\begin{center}
% \fbox{\rule{0pt}{2in} \rule{0.9\linewidth}{0pt}}
   \includegraphics[width=1\linewidth]{derinkuyu-air.jpg}
\end{center}
   \caption{Глубокий вентиляционный колодец в Деринкую \cite{53}.}
\label{fig:6}
\label{fig:onecol}
\end{figure}

% \section{Дополнительные аномалии, которые лучше всего объясняются переворотом Земли}

% Прежде чем закончить, упомянем ещё некоторые научные аномалии, которые, если рассматривать их в контексте катастрофических геофизических сил, получают разумное объяснение.
\section{Биомассовые накопления}

Смеси биомассы различных видов животных и растений, часто встречающиеся в виде ископаемого в слоях осадочных пород, представляют собой еще одну загадочную аномалию. В "Reliquoæ Diluvianæ" преподобный Уильям Бакленд описывает находки многочисленных видов фауны, для которых не было объяснимой причины находиться вместе, разбросанных по Британии и Европе, погребённых в слоях осадочного "дилювия" \cite{58}. Подобные смеси остатков животных также были обнаружены в пещере Ськьонгхеллерен на острове Вальдрой, Норвегия. В этой пещере более 7000 костей млекопитающих, птиц и рыб были найдены перемешанными в нескольких слоях осадков \cite{59}. Другой пример — Сан Чиро, "Пещера гигантов", в Италии. В этой пещере было найдено несколько тонн костей млекопитающих, преимущественно гиппопотамов, в настолько свежем состоянии, что их использовали для изготовления украшений и отправляли на производство сажи для ламп. Кости различных животных, как сообщается, были размешаны, сломаны, раздроблены и рассеяны фрагментами \cite{60,61}. В Древнем Мендесе, Египет, смесь костей различных видов животных была найдена смешанной с остеклованной (стекловидной) глиной \cite{57}. Такие находки могут показаться загадочными, но легко объясняются масштабными наводнениями, укладывающими смеси мёртвых животных в осадочные слои, заносящими животных в пещеры или закапывающими их заживо, а в случае остеклованной биомассы в Египте — постпотопными мощнейшими электрическими разрядами из-за смещения ядра и мантии. На рисунке \ref{fig:7} показан типичный разрез аллювиального "ила" Аляски \cite{56}.

\begin{figure}[t]
\begin{center}
% \fbox{\rule{0pt}{2in} \rule{0.9\linewidth}{0pt}}
   \includegraphics[width=1\linewidth]{muck-crop.jpeg}
\end{center}
   \caption{Аляскинский "ил", состоящий из хаотично разбросанных фрагментов деревьев, растений и животных в замёрзшем иле и льду \cite{146}.}
\label{fig:7}
\label{fig:onecol}
\end{figure}

\section{Древние бункеры}

Наши предки оставили после себя множество высокотехнологичных древних сооружений, в которых были обнаружены человеческие останки. Обычно их интерпретируют как сложные гробницы, однако пристальное рассмотрение показывает, что на самом деле это могли быть древние бункеры.

\begin{figure}[b]
\begin{center}

% \fbox{\rule{0pt}{2in} \rule{0.9\linewidth}{0pt}}
   \includegraphics[width=1\linewidth]{ww19.jpg}
\end{center}
   \caption{Ньюгрейндж, Ирландия — смотрите на посетителей у входа для понимания масштаба.}
\label{fig:8}
\label{fig:onecol}
\end{figure}

Одним из прекрасных примеров служит Ньюгрейндж (рисунок \ref{fig:8}), главный памятник в комплексе Брю-на-Бойн, собрании древних сооружений, включающих так называемые коридорные гробницы. Эти гробницы состоят из одной или нескольких погребальных камер, покрытых землей или камнем, и имеют узкий проход, сделанный из больших камней \cite{70}. Это пример масштабной инженерии сложного защищённого сооружения, построенного за несколько поколений, предположительно для захоронения небольшой группы людей, которые даже не были живы, когда началось строительство гробницы. Когда она была вновь обнаружена местным землевладельцем в 1699 году, она была засыпана землёй.

Беглый взгляд на сооружение раскрывает огромные усилия, затраченные на его возведение — Ньюгрейндж состоит примерно из 200000 тонн материала. Внутри него, \textit{«...находится коридор с камерами, к которому можно попасть через вход с юго-восточной стороны памятника. Коридор протягивается на 19 метров (60 футов), что примерно составляет треть пути к центру сооружения. В конце прохода расположены три небольшие камеры, отходящие от большой центральной камеры с высоким сводчатым потолком... Стены этого коридора состоят из крупных каменных плит, называемых ортостатами, двадцать две из которых находятся на западной стороне и двадцать одна — на восточной. Их средняя высота составляет 1½ метра»} \cite{70}. Имеются также сложные инженерные детали по водонепроницаемости. Например, в крыше, \textit{«Щели в крыше были заделаны смесью обожжённой земли и морского песка, чтобы сделать их водонепроницаемыми, и из этой смеси были получены два радиоуглеродных датирования, относящихся примерно к 2500 году до нашей эры, для структуры гробницы»} \cite{71}. Кроме того, могла быть реализована подъёмная высота, ведущая к внутренней камере, для аналогичных целей: \textit{«Поскольку пол коридора и камеры гробницы повторяет подъём земли холма, на котором построен памятник, существует разница почти в 2 метра по уровню пола между входом и внутренней частью камеры»} \cite{71}.

\begin{figure}[b]
\begin{center}
% \fbox{\rule{0pt}{2in} \rule{0.9\linewidth}{0pt}}
   \includegraphics[width=1\linewidth]{dolmen.jpg}
\end{center}
   \caption{Дольмен де Сото, Испания \cite{53}.}
\label{fig:9}
\label{fig:onecol}
\end{figure}

Отсутствие человеческих останков внутри — тоже любопытный момент. Раскопки выявили обожжённые и необожжённые фрагменты костей, представляющие  небольшое количество людей, разбросанные по проходу. Строительство Ньюгрейнджа примерно заняло по меньшей мере несколько поколений, основываясь на углеродном датировании материалов внутри. Почему древнее сообщество потратило бы столько усилий на возведение массивной, технически сложной гробницы, чтобы разбросать фрагменты костей нескольких умерших в её проходе? Гораздо более правдоподобно, что эти древние и тщательно гидроизолированные мегалитические сооружения были построены как убежища для защиты людей во время повторяющихся катаклизмов Земли.

В Уэльве, на юге Испании, похожий пример — это Дольмен де Сото (рис. \ref{fig:9}), один из примерно 200 подобных объектов в этой местности \cite{72,32}. Это обтекаемая, технически сложная структура, построенная из мегалитических камней и имеет диаметр с 75 метров. Сообщалось, что при раскопках было найдено лишь восемь тел, все погребённые в позе эмбриона.

\section{Известные упоминания аномалий}

В этом разделе я кратко упоминаю ещё некоторые известные аномалии, все из которых хорошо объясняются катастрофой, похожей на ЕCDO.

\subsection{Биологические аномалии}

\begin{figure}[b]
\begin{center}
% \fbox{\rule{0pt}{2in} \rule{0.9\linewidth}{0pt}}
   \includegraphics[width=1\linewidth]{bottleneck.jpg}
\end{center}
   \caption{Генетическое сужение, отражающее уничтожение 95\% мужчин примерно 6000 лет назад \cite{62}.}
\label{fig:10}
\label{fig:onecol}
\end{figure}

Некоторые заметные биологические аномалии включают генетические сужения и ископаемые киты на побережье. Цзэн и др. (2018) смоделировали 125 последовательностей Y-хромосомы современных людей и, на основе сходств и мутаций в ДНК, выявили сужающее сокращение мужского населения на 95\% около 5000 – 7000 лет назад (рисунок \ref{fig:10}) \cite{62}. Останки китов были обнаружены на сотни метров выше уровня моря в Шведенборге, Мичигане, Вермонте, Канаде, Чили и Египте \cite{63,64,65,66}. Эти киты были найдены в разном состоянии: идеально сохранившимися в болотах, лежащими над ледниковыми отложениями, или погребёнными в осадочных породах. Количество экземпляров в этих местах варьируется от нескольких до более сотни. Киты — морские глубинные обитатели и редко приближаются к берегам. Как же эти киты оказались на таких больших высотах, часто на значительном удалении от побережья?

Множество массовых вымираний произошло на земле, наиболее изучены из них "Большая пятёрка" фанерозойских событий: позднеордовикское (LOME), позднедевонское (LDME), пермское (EPME), триасовое (ETME) и меловое (ECME) массовые вымирания \cite{88,89}. Любопытно, что несколько из этих вымираний произошли в те же исторические периоды, что и многие слои Гранд-Каньона, в частности пермские и девонские слои.

\subsection{Физические аномалии}

\begin{figure}[b]
\begin{center}
% \fbox{\rule{0pt}{2in} \rule{0.9\linewidth}{0pt}}
   \includegraphics[width=1\linewidth]{columbia.jpg}
\end{center}
   \caption{Массивные волны течения в ледниковом озере Колумбия, штат Вашингтон \cite{80}.}
\label{fig:11}
\label{fig:onecol}
\end{figure}

Существует множество ландшафтов, помимо Гранд-Каньона, которые, вероятно, были сформированы катастрофическими силами. Свидетельства масштабных континентальных потоков воды можно найти в виде гигантских волнообразных форм рельефа по всему миру. Один из таких примеров — район Чаннелд-Скаблендс на северо-западе США. Здесь мы видим не только ландшафты из осадочных отложений и валуны ледника, но и более ста последовательностей крупных волн, сформированных мега-потоками \cite{78,79}. Это более крупномасштабные версии ряби, формируемой в песчаных руслах ручьёв. Подобные образования можно встретить по всему миру: во Франции, Аргентине, России и Северной Америке \cite{81}. Рисунке \ref{fig:11} показывает некоторые из этих волн в штате Вашингтон, США \cite{80}.

\begin{figure}[b]
\begin{center}
% \fbox{\rule{0pt}{2in} \rule{0.9\linewidth}{0pt}}
   \includegraphics[width=1\linewidth]{zhangjiajie.jpg}
\end{center}
   \caption{Массивные каменные столбы в национальном лесу Чжанцзяцзе, южный Китай.}
\label{fig:12}
\label{fig:onecol}
\end{figure}

\begin{figure}[b]
\begin{center}
% \fbox{\rule{0pt}{2in} \rule{0.9\linewidth}{0pt}}
   \includegraphics[width=1\linewidth]{hoy.jpg}
\end{center}
   \caption{Морской столб Старик Хой, Шотландия \cite{83}.}
\label{fig:13}
\label{fig:onecol}
\end{figure}

Структуры внутренней эрозии также хорошо объясняются ECDO-подобным переворотом Земли. Южный Китай является прекрасным примером массивных карстовых ландшафтов, образованных водной эрозией \cite{82}. Эти ландшафты включают башенный карст, пиковый карст, конический карст, естественные мосты, ущелья, большие пещерные системы и карстовые воронки. Одним из самых ярких из них является Национальный лес Чжанцзяцзе, в котором находятся массивные столбы из кварцевого песчаника (рисунок \ref{fig:12}) \cite{84}. Эти столбы находятся на средней высоте более 1000 метров, а их число превышает 3100. Более 1000 из них возвышаются более чем на 120 метров, а 45 достигают более 300 метров \cite{85}. Эти столбы напоминают столбы морской эрозии (рисунок \ref{fig:13}), которые представляют собой прибрежные скальные столбы, образованные обрушением окружающего материала из-за океанских волн. Похожие эрозионные ландшафты можно найти в скальных конусах Ургупа, Турция, а также Сьюдад-Энкантада, Испания, которые находятся на высоте более 1000 метров над уровнем моря. Во всех этих местах есть некоторое сочетание соли и океанических морских окаменелостей в непосредственной близости от них, что предполагает прошлые морские вторжения \cite{15,86,87}. Конечно, истории о потопах \cite{3} упоминают как океан поднялся намного выше 1000 метров, и это подтверждается наличием соленой воды и массивных солончаков в Андах и Гималаях на высоте нескольких километров над уровнем моря. Например, солончак Уюни в Боливии достигает 3653 метров над уровнем моря \cite{94}.

\subsection{События стремительных изменений климата}

Современная научная литература признает существование быстрых глобальных изменений климата в недавней истории Земли. Два заметных примера — это события 4,2 килотысячных и 8,2 килотысячных лет, оба из которых совпадают с сокращением населения и разрушением поселений на больших географических территориях. Эти события сохраняются как аномалии в осадочных и ледяных кернах, ископаемых кораллах, значениях изотопа кислорода O18, поленовых и спелеотемных записях, а также данных об уровне моря. Предполагаемые климатические изменения  включают резкое понижение глобальных температур, аридизацию, нарушение меридиональной циркуляции Атлантики и продвижение ледников \cite{90,91,92}. Особенно событие 8,2 килотысячных лет совпадает с возможным драматическим затоплением Черного моря солёной водой примерно в 6400 году до нашей эры \cite{93}.

\subsection{Археологические аномалии}

Археологические свидетельства некоторых древних городов показывают множество слоёв, связанных с захоронениями и разрушениями, что создаёт летопись прошлых катастрофических событий. Древний город Иерихон — один из таких городов, находящийся на территории современной Палестины. Он содержит несколько слоёв разрушения, с обрушением каменных конструкций и значительными следами огня \cite{96,97}. Хронология, записанная в его слоях, простирается примерно с 9000 года до нашей эры по 2000 год до нашей эры. Особого внимания заслуживает его башня, которая, по-видимому, была срезана и погребена в осадочных породах около 7400 года до нашей эры (Рисунок \ref{fig:14}) \cite{95}. Чатал-Гуюк \cite{99}, Грамалоте \cite{98} и минойский дворец Кносс на Крите \cite{100,101} — все это схожие примеры археологических местоположений, содержащих многослойные отложения, часто с признаками разрушения.

\begin{figure}[t]
\begin{center}
% \fbox{\rule{0pt}{2in} \rule{0.9\linewidth}{0pt}}
   \includegraphics[width=1\linewidth]{jericho.jpg}
\end{center}
   \caption{Археологическая реконструкция захоронения башни Иерихона около 7400 года до н.э. \cite{95}.}
\label{fig:14}
\label{fig:onecol}
\end{figure}

Еще одним доказательством крупных катастроф, разрушающих человеческую цивилизацию, является Нампа Имидж, глиняная кукла, найденная примерно под 100 метрами лавы в Айдахо \cite{102,103}. Поток лавы, под которым была найдена фигурка, был отложен в позднетретичный или раннечетвертичный период, предположительно ему 2 миллиона лет. Однако поток лавы в этом регионе, по-видимому, относительно свежий. Такие находки не только указывают на крупные катастрофы, разрушающие цивилизацию, но и ставят под сомнение современные хронологии датирования.

\section{Относительно современных методов датирования}

Есть серьезные основания скептически относиться к современным хронологиям, которые приписывают чрезвычайно большие возраста в миллионы или даже сотни миллионов лет различным физическим материалам.

Согласно распространённому мнению, так называемые "ископаемые виды топлива" — уголь, нефть и природный газ — имеют возраст сотни миллионов лет \cite{104}. Однако фактическое углеродное датирование нефти, добытой в Мексиканском заливе, показало возраст примерно 13000 лет \cite{105}. Период полураспада углерода-14 настолько короток (5730 лет), что он должен полностью распасться за несколько сотен тысяч лет. Тем не менее, C-14 был обнаружен в угле и ископаемых останках, которые, как считается, на тысячу раз старше \cite{106}. Более того, искусственный уголь был получен в лаборатории под контролируемыми условиями, в первую очередь при высокой температуре, всего за 2–8 месяцев \cite{107}.

Радиоизотопные методы датирования, кроме углеродного, также могут быть ненадёжными. Исследовательская группа Answers in Genesis обнаружила несоответствия в результатах, полученных этими методами, что ставит их достоверность под сомнение \cite{108}. Мягкие ткани, содержащие эритроциты, сосуды и коллаген, были обнаружены даже в окаменелостях динозавров, которым якобы сто миллионов лет \cite{109,110}. На основании имеющихся данных можно предположить, что общепринятые возраста геологической шкалы Земли и материалов, таких как породы и ископаемое топливо, могут быть ошибочными на порядки величины.

\section{Заключение}

В этой работе я рассмотрел наиболее убедительные аномалии, указывающие на катастрофическое происхождение и наилучшим образом объясняющиеся моделью ECDO-инверсии Земли. Несмотря на разнообразие, представленный тут перечень неполон — больше аномалий собрано и находится в открытом доступе в моём научном репозитории на GitHub \cite{2}.

\section{Благодарности}

Благодарю Ethical Skeptic, оригинального автора тезиса ECDO, за завершение его проницательной, новаторской работы и за то, что он поделился ею с миром. Его трёхчастный труд \cite{1} остаётся авторитетной работой по теории Экзотермического разделения ядра-мантии и эффекта Джанибекова (ECDO) и содержит гораздо больше информации по теме, чем я смог вкратце изложить здесь.

И, конечно, спасибо гигантам, на плечах которых мы стоим; тем, кто провёл все исследования и расследования, благодаря которым эта работа стала возможной, и кто старался принести свет человечеству.

\clearpage
\twocolumn

{\small
\bibliographystyle{ieee}
\bibliography{egbib}
}

\end{document}
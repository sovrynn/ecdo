\documentclass[10pt,twocolumn,letterpaper]{article}

% My own stuff
\usepackage{booktabs}
% \usepackage{caption}
% \captionsetup[table]{skip=8pt}   % Only affects tables
\usepackage{stfloats}  % Add this to the preamble

%–– ICU line-breaking for Khmer ––
%–– enable proper line-breaking for Burmese, and 0 space between characters? ––
\XeTeXlinebreaklocale "my"
\XeTeXlinebreakskip = 0pt plus 0pt minus 0pt
%–– you can experiment with some stretch:
% \XeTeXlinebreakskip = 0pt plus 1pt

\usepackage{fontspec}

%–– define your two fonts ––
\newfontfamily\latinfont{Latin Modern Roman}              % for all Latin text
\newfontfamily\burmesefont[Script=Myanmar]{Noto Sans Myanmar}        % for all Burmese text

%–– load ucharclasses to auto-switch when the script block changes ––
\usepackage{ucharclasses}

% default (everything outside Myanmar block) → Latin font
\setDefaultTransitions{\latinfont}{}

% when entering the Myanmar block → switch to Burmese font,
% and when exiting → switch back to Latin
\setTransitionsFor{Myanmar}{\burmesefont}{\latinfont}

% This makes the font slightly bigger than base (10) and bold in Subsection headings rather than using ptmb
\makeatletter
\def\cvprsubsection{%
  \@startsection{subsection}{2}{\z@}%
    {8pt plus 2pt minus 2pt}{6pt}%
    % {\normalfont\bfseries\selectfont}%
    {\normalfont\bfseries\fontsize{11}{13}\selectfont}%
}
\makeatother

% So this hardcodes the style for the numbers in the section/subsection headings so they're bold
\font\elvbf=ptmb scaled 1100
\font\elvbfs=ptmb scaled 1200
\makeatletter
% Section number: Large + bold
\renewcommand\thesection{%
  {\elvbfs\arabic{section}}%
}

% Subsection number: normalsize + bold + custom punctuation
\renewcommand\thesubsection{%
  {\elvbf
   \arabic{section}.\arabic{subsection}}%
}
\makeatother

\usepackage{cvpr}
\usepackage{times}
\usepackage{epsfig}
\usepackage{graphicx}
\usepackage{amsmath}
\usepackage{amssymb}

% Include other packages here, before hyperref.

% If you comment hyperref and then uncomment it, you should delete
% egpaper.aux before re-running latex.  (Or just hit 'q' on the first latex
% run, let it finish, and you should be clear).
\usepackage[breaklinks=true,bookmarks=false]{hyperref}

\cvprfinalcopy % *** Uncomment this line for the final submission

\def\cvprPaperID{****} % *** Enter the CVPR Paper ID here
\def\httilde{\mbox{\tt\raisebox{-.5ex}{\symbol{126}}}}

% % 1) Choose your desired fixed leading:
% \renewcommand\baselinestretch{1.2}  % or 1.3, 1.1…  adjust to taste

% % 2) Force TeX to *always* use \baselineskip, never fall back to \lineskip:
% \makeatletter
%   \setlength\lineskiplimit{-\maxdimen} % always allow baselineskip
%   \setlength\lineskip{0pt}             % no extra glue ever
% \makeatother


\renewcommand{\tablename}{ဇယား}
\renewcommand{\figurename}{ပုံ}   % or whatever you like instead of "Hình"
\renewcommand{\refname}{ကိုးကားချက်များ}

\makeatletter
\def\abstract{%
  \centerline{\large\bf အကျဉ်းချုပ်}% <-- your new label
  \vspace*{12pt}%
  \it%
}
\makeatother

% This makes the font slightly bigger than base (10) and bold in Subsection headings rather than using ptmb
\makeatletter
\def\cvprsubsection{%
  \@startsection{subsection}{2}{\z@}%
    {8pt plus 2pt minus 2pt}{6pt}%
    % {\normalfont\bfseries\selectfont}%
    {\normalfont\bfseries\fontsize{11}{13}\selectfont}%
}
\makeatother

% So this hardcodes the style for the numbers in the section/subsection headings so they're bold
\font\elvbf=ptmb scaled 1100
\font\elvbfs=ptmb scaled 1200
\makeatletter
% Section number: Large + bold
\renewcommand\thesection{%
  {\elvbfs\arabic{section}}%
}

% Subsection number: normalsize + bold + custom punctuation
\renewcommand\thesubsection{%
  {\elvbf
   \arabic{section}.\arabic{subsection}}%
}
\makeatother

\renewcommand{\refname}{References}

% Pages are numbered in submission mode, and unnumbered in camera-ready
%\ifcvprfinal\pagestyle{empty}\fi
\setcounter{page}{1}
\begin{document}

%%%%%%%%% TITLE
\title{ECDO ဒေတာအခြေပြု စာတမ်း အပိုင်း ၂/၂- ECDO “ကမ္ဘာမြေကြီးရွေ့လျားမှု" အရ အကောင်းဆုံး ရှင်းပြထားသည့် သိပ္ပံနှင့် သမိုင်းဆိုင်ရာ ထူးခြားမှုများအကြောင်း သုတေသန}

\author{Junho\\
၂၀၂၅ ဖေဖော်ဝါရီတွင် ထုတ်ပြန်သည်\\
ဝက်ဘ်ဆိုက် (စာတမ်းများကို ဒီမှာဒေါင်းလုတ်ယူပါ)- \href{https://sovrynn.github.io}{sovrynn.github.io}\\
ECDO သုတေသန စုစည်းထားသည့်စာရင်း- \href{https://github.com/sovrynn/ecdo}{github.com/sovrynn/ecdo}\\
{\tt\small junhobtc@proton.me}
% For a paper whose authors are all at the same institution,
% omit the following lines up until the closing ``}''.
% Additional authors and addresses can be added with ``\and'',
% just like the second author.
% To save space, use either the email address or home page, not both
% \and
% xx
% Institution2\\
% First line of institution2 address\\
% {\tt\small secondauthor@i2.org}
}

\maketitle
%\thispagestyle{empty}

%%%%%%%%% အကျဉ်းချုပ်
\begin{abstract}
၂၀၂၄ ခုနှစ် မေလတွင် “ကျင့်ဝတ် ဝေဖန်သူ” \cite{0} ဟု အမည်ရသော ကလောင်အမည်ဖြင့် အွန်လိုင်းစာရေးသူတစ်ဦးက အပူထုတ်လွှတ်သော ကမ္ဘာ့ဗဟိုလွှာ ဂျာနီဘီကော့ဗ် ခွဲဖြာရွေ့လျားခြင်း (ECDO) \cite{1} ဆိုသည့် ထူးခြားသော သီအိုရီတစ်ခုကို တင်ပြခဲ့သည်။ ဤသီအိုရီတွင် ကမ္ဘာမြေကြီး၏ ဝင်ရိုးလှည့်ပတ်မှုတွင် ရုတ်တရက် သဘာဝဘေးအန္တရာယ်ဆိုးကြီးများ ဖြစ်ပေါ်နိုင်သော ပြောင်းလဲမှုများကို ယခင်က ကြုံတွေ့ဖူးကြောင်း၊ ဝင်ရိုးလည်မှု အရှိန်ကြောင့် ပင်လယ်သမုဒ္ဒရာများဆီမှ ရေများသည် ကုန်းမြေများပေါ်သို့ စီးဝင်ကာ ကမ္ဘာတစ်ဝန်း ရေကြီးရေလျှံမှုကြီးများ ဖြစ်ပေါ်ခဲ့ကြောင်း အဆိုပြုထားသည်သာမက အလားတူ ရွေ့လျားမှုမျိုး မကြာမီဖြစ်ပွားနိုင်သည့် ဘူမိရူပ ဖြစ်စဉ် အကြောင်းရင်းကိုလည်း ဒေတာအချက်အလက်များနှင့်တကွ ရှင်းပြချက်ကိုပါ အဆိုပြုခဲ့ပါသည်။ ထိုသို့သော ရေကြီးရေလျှံမည့် သဘာဝဘေးအန္တရာယ်နှင့် သဘာဝဘေးအန္တရာယ်ဆိုးကြီးများ ခန့်မှန်းချက်များသည် အသစ်မဟုတ်သော်လည်း ECDO သီအိုရီသည် ခေတ်သစ်၊ သိပ္ပံနည်းကျ၊ နည်းလမ်းပေါင်းစုံနှင့် ဒေတာအခြေခံ ချဉ်းကပ်မှု ဖြစ်မှုကြောင့် ထူးထူးခြားခြား စိတ်ဝင်စားဖွယ် ကောင်းပါသည်။

ဤသုတေသနစာတမ်းသည် ECDO သီအိုရီနှင့်ပတ်သက်၍ ၆ လတာ လွတ်လပ်စွာ ဆောင်ရွက်ခဲ့သော သုတေသန အနှစ်ချုပ် \cite{2,20} အပိုင်းနှစ်ပိုင်းအနက် ဒုတိယပိုင်းဖြစ်ပြီး သဘာဝဘေးအန္တရာယ်ဆိုးကြီးများ ဖြစ်ပေါ်နိုင်သော “ကမ္ဘာမြေကြီးရွေ့လျားမှု” အရ အကောင်းဆုံး ရှင်းပြထားသည့် သိပ္ပံနည်းကျ၊ သမိုင်းဖြစ်ရပ် မူမမှန်မှုများအပေါ် အထူးအာရုံစိုက် လေ့လာထားပါသည်။

\end{abstract}

%%%%%%%%% BODY TEXT

\section{နိဒါန်း}

ကမ္ဘာ့မျက်နှာပြင် ပြောင်းလဲမှုဆိုင်ရာ ခေတ်သစ် ဘူမိဗေဒပညာရပ်နှင့် သမိုင်းကြောင်းများအရ ဂရင်းကင်ညွန်ချောက်ကြီး ကဲ့သို့သော ကြီးမားသည့် ဘူမိဗေဒ မြေအနေအထားများသည် နှစ်သန်းပေါင်းများစွာ ကြာမြင့်သောကာလအတွင်း ဖြစ်ပေါ်လာကြောင်း \cite{143}၊ သေမင်းတမန်တောင်ကြား (ကယ်လီဖိုးနီးယားပြည်နယ်) တွင် ဆားများရှိနေသည်မှာ ထိုဒေသသည် လွန်ခဲ့သည့် နှစ်သန်းပေါင်း ရာနှင့်ချီသည့်အချိန်က ပင်လယ်အောက်တွင် တည်ရှိခဲ့သောကြောင့်ဖြစ်ကြောင်း \cite{144}၊ လွန်ခဲ့သော မျိုးဆက် ၁၅၀ က ကျွန်ုပ်တို့၏ ဘိုးဘေးများသည် သူတို့၏ဘဝတစ်ခုလုံးတွင် ကြီးမားလှစွာသော အုတ်ဂူများ တည်ဆောက်ခဲ့ကြကြောင်း \cite{29,70}နှင့် “ကျောက်ဖြစ်ရုပ်ကြွင်း” ဆိုသည့်အရာများသည် နှစ်သန်းပေါင်းများစွာ ကြာမြင့်ပြီဖြစ်ကြောင်း \cite{104} အဆိုပြုထားပါသည်။ စိတ်ဝင်စားစရာ အကောင်းဆုံး ဖြစ်ဖွယ်ရှိသော အချက်မှာကား လူသားတို့တည်ရှိခဲ့သည်မှာ နှစ်ပေါင်း ၃၀၀,၀၀၀ ကြာမြင့်ပြီ \cite{145}ဟု ယူဆကြသော်လည်း သမိုင်းမှတ်တမ်းနှင့် လူ့ယဉ်ကျေးမှုတို့သည် နှစ်ပေါင်း ၅,၀၀၀ ခန့်သာ ရှိသေးသည့်အတွက် မျိုးဆက် ၁၅၀ နှင့်ညီမျှသည်ဆိုသောအချက် ဖြစ်ပါသည်။

ကျွန်ုပ်တို့လေ့လာသွားမည်ဖြစ်သော အဆိုပါ ထူးခြားချက်များသည် သဘာဝဘေးအန္တရာယ်ဆိုးကြီးများ ဖြစ်ပေါ်နိုင်သော ဘူမိဗေဒ ပြင်းအားများကြောင့် ဖြစ်ကြောင်း အကောင်းဆုံး ရှင်းပြနိုင်ပါသည်။

\section{ရေခဲထုအောက်၌ နစ်မြုပ်ကာ အသားစိုင်များ အေးခဲထားသည့် ဧရာမဆင်ကြီး}

\begin{figure}[t]
\begin{center}
% \fbox{\rule{0pt}{2in} \rule{0.9\linewidth}{0pt}}
   \includegraphics[width=1\linewidth]{jarkov-mammoth.jpg}
\end{center}
   \caption{နှစ်ပေါင်း ၂၀,၀၀၀ သက်တမ်းရှိ ရေခဲထုအောက်တွင် မပျောက်မပျက် အေးခဲထားသည့် ဆိုက်ဘေးရီးယား ဧရာမဆင်ကြီးဖြစ်သည့် ဂျားကော့ဗ်ဆင်ကြီးတစ်ကောင် \cite{51}.}
\label{fig:1}
\label{fig:onecol}
\end{figure}

အဆိုပါထူးခြားချက် အမျိုးအစားတစ်ခုမှာ အာတိတ်ဒေသများ၌ တွေ့ရလေ့ရှိသည့် ရေခဲထုအောက်တွင် အသားစိုင်များ မပျောက်မပျက် အေးခဲထားသည့် ဧရာမဆင်ကြီးများဖြစ်သည် (ပုံ \ref{fig:1})။ ဆိုက်‌ဘေးရီးယားတွင် ရှာဖွေတွေ့ရှိခဲ့သည့် ကျောက်စရစ်နုန်းမြေထဲ၌ နစ်မြုပ်နေသော ဘရီဆော့ဗ်ကာ ဧရာမဆင်ကြီးသည် မူလအတိုင်း မပျောက်မပျက်ရှိနေသောကြောင့် သေဆုံးပြီးနောက် နှစ်ပေါင်းထောင်နှင့်ချီ၍ ကြာမြင့်ပြီဖြစ်သည့်တိုင် ၎င်း၏အသားမှာ စားသုံး၍ရနိုင်လောက်အောင် ကောင်းမွန်နေဆဲဖြစ်သည်။ ထိုဆင်ကြီး၏ ပါးစပ်နှင့် အစာအိမ်ထဲတွင် အသီးအရွက်စားစရာများလည်း ရှိနေသောကြောင့် ၎င်းမသေဆုံးမီအချိန်လေးမှာပင် အသီးအပွင့်များကို စားသုံးနေစဉ် အလွန်လျင်မြန်လှစွာ အေးခဲသွားရလောက်အောင် မည်သို့များ ကြုံတွေ့ခဲ့လေသနည်းဆိုပြီး သိပ္ပံပညာရှင်များကို ဇဝေဇဝါဖြစ်စေခဲ့ပါသည် \cite{17}။ တင်ပြထားချက်များအရ \textit{"၁၉၀၁ ခုနှစ်တွင် ဘရီဆော့ဗ်ကာ မြစ်အနီး၌ ဧရာမဆင်ကြီးတစ်ကောင်လုံး၏ ရုပ်အလောင်းကို ရှာဖွေတွေ့ရှိခဲ့ရာတွင် ထိုသတ္တဝါကြီးသည် နွေခေါင်ခေါင်၌ အအေးဓာတ်ကြောင့် သေဆုံးသွားခဲ့ပုံရသည့်အတွက် စိတ်ဝင်စားခဲ့ကြခြင်းဖြစ်ကြောင်း သိရှိရပါသည်။  ၎င်း၏ အစာအိမ်ထဲမှ အစာများမှာ ကောင်းစွာ မပျက်မစီးရှိနေရာ မဟာလှေကားပန်းနှင့် တောရိုင်းပဲညွန့်များ ပါဝင်နေပါသည်။ ဆိုလိုသည်မှာ ယင်းတို့ကို ဇူလိုင်လကုန် သို့မဟုတ် ဩဂုတ်လဆန်းခန့်၌ မြိုချစားသုံးခဲ့ခြင်း ဖြစ်ရပါမည်။ ထိုဆင်ကြီးမှာ ရုတ်တရက်ကြီး သေဆုံးခဲ့သည့်အတွက် သူ့မေးရိုးများတွင် မြက်ပင်နှင့် ပန်းပင်များကို ကိုက်ထားဆဲပင် ဖြစ်သည်။ ၎င်းသည် ကြီးမားလှစွာသော အင်အားကြောင့် ၎င်း၏စားကျက်မြေမှ မိုင်ပေါင်းများစွာဝေးသည့်နေရာအထိ လွင့်စင်သွားခဲ့သည်။ တင်ပါးဆုံနှင့် ခြေတစ်ဖက်မှာ ကွဲအက်နေရာ ဧရာမသတ္တဝါကြီးမှာ ဒူးထောက်ကိုင်ပေါက်ခံထားရပြီးနောက် အေးခဲသေဆုံးသွားခြင်းဖြစ်ရာ ထိုကာလမှာ ပုံမှန်အားဖြင့် တစ်နှစ်တာ၏ အပူဆုံးကာလ ဖြစ်နေပါသည်။"} \cite{18}. ထို့ပြင် \textit{"“[ရုရှားသိပ္ပံပညာရှင်များ] က ထိုသတ္တဝါကြီး၏ အစာအိမ်ရှိ အူအတွင်းပိုင်း အပေါ်ယံလွှာများမှာပင်လျှင် ကောင်းစွာ ထိန်းသိမ်းကျန်ရှိသော အမျှင်ဓာတ်ဖွဲ့စည်းပုံများ ရှိနေသည့်အချက်က သဘာဝအလျောက် အံ့မခန်း အင်အားကြီးမားသော ဖြစ်စဉ်တစ်ခုခုကြောင့် သူ့ခန္ဓာကိုယ် အပူချိန် ကွယ်ပျောက်သွားရကြောင်း မှတ်တမ်းတင်ထားပါသည်။ ဆန်ဒါဆင်သည် ဤအချက်တစ်ခုတည်းကို အထူးမှတ်သားကာ ဤကိစ္စကို အမေရိကန်နိုင်ငံ အေးခဲစားအသောက်များ သုတေသနအဖွဲ့အစည်းထံသို့ ယခုလို တင်ပြခဲ့ပါသည်- ဧရာမဆင်ကြီး တစ်ကောင်လုံး၏ အစာအိမ် အူနံရံလွှာများအထိ ခန္ဓာကိုယ်တွင်း အစိတ်အပိုင်းများမှ အစိုဓာတ်သည် အသားအမျှင်ဓာတ်ဖွဲ့စည်းပုံကို ပျက်စီးစေရန် ပုံဆောင်ခဲ ဖြစ်ခွင့်မရလောက်အောင် အင်အားကြီးမားသည့် မည်သည့်အရာက ပြုလုပ်ခဲ့ပါသနည်း။... သီတင်းပတ်အနည်းငယ်အကြာတွင် ထိုအဖွဲ့အစည်းသည် ဆန်ဒါဆင်ကို အဖြေတစ်ခုဖြင့် ယခုလို အကြောင်းပြန်လာပါသည်- ဤသည်မှာ လုံးဝမဖြစ်နိုင်ပါ။ ကျွန်ုပ်တို့၏ သိပ္ပံနှင့် အင်ဂျင်နီယာ အသိပညာ အားလုံးအရဆိုလျှင် ဧရာမဆင်တစ်ကောင်တမျှ ကြီးမားသည့် အသေကောင်ထံမှ အသားထဲ၌ အစိုဓာတ် ပုံဆောင်ခဲကြီးများ မဖြစ်ပေါ်ဘဲ လျင်မြန်စွာ အေးခဲသွားလောက်အောင် ခန္ဓာကိုယ် အပူချိန် ဖယ်ရှားသည့်နည်းလမ်းကို လုံးဝ မသိရှိထားပါ။ ထို့ပြင် သိပ္ပံနှင့် အင်ဂျင်နီယာနည်းပညာများ အားလုံးကို လက်လျှော့ပြီးနောက် သဘာဝတရားကို အားကိုးကာ ထိုကိစ္စကို ဆောင်ရွက်နိုင်သည့် သဘာဝဖြစ်စဉ်တစ်ခုခုကို မသိရှိထားကြောင်း ကောက်ချက်ချခဲ့ပါသည်"} \cite{19}.

\section{ဂရင်းကင်ညွန်ချောက်ကြီး}

ဂရင်းကင်ညွန်ချောက်ကြီးသည် မြောက်အမေရိကတိုက် အနောက်တောင်ဘက်ရှိ မဟာလွင်ပြင်ကြီး၏ အစိတ်အပိုင်းတစ်ခုဖြစ်ကာ သဘာဝဘေးအန္တရာယ်ဆိုးကြီးများ ဖြစ်ပေါ်ခဲ့သော မူလဇစ်မြစ်ဟု ယူဆရသည့် နောက်ထပ် သဘာဝဖြစ်စဉ်တစ်ခု ဖြစ်သည် (ပုံ \ref{fig:2})။ ထိုအချက်ကို စပြောရလျှင် ဂရင်းကင်ညွန်ကို ဖွဲ့စည်းထားသည့် အနည်ထိုင်နေသော သဲကျောက်နှင့် ထုံးကျောက်လွှာများသည် စတုရန်း ကီလိုမီတာ ၂.၄ သန်း အထိ ကြီးမားကျယ်ပြောလှပါသည်။ ပုံ \ref{fig:3} တွင် ကိုကိုနီနိုသဲကျောက်လွှာသည် အမေရိကန်နိုင်ငံ အနောက်ပိုင်းတစ်လျှောက် ကျယ်ပြန့်သည်ကို ဖော်ပြထားပါသည်။ ဤမျှကြီးမားသော အလျားလိုက် တညီတညာတည်းဖြစ်သော အလွှာအားလုံးကို တစ်ချိန်တည်း ပို့ချခဲ့ခြင်းသာ ဖြစ်နိုင်ပါသည်။

\begin{figure}[b]
\begin{center}
% \fbox{\rule{0pt}{2in} \rule{0.9\linewidth}{0pt}}
   \includegraphics[width=1\linewidth]{grand-canyon.jpg}
\end{center}
   \caption{အမေရိကန်ပြည်ထောင်စု၊ အရီဇိုးနားပြည်နယ်ရှိ ဂရင်းကင်ညွန်ချောက်ကြီးA \cite{49}.}
\label{fig:2}
\label{fig:onecol}
\end{figure}

\begin{figure}[t]
\begin{center}
% \fbox{\rule{0pt}{2in} \rule{0.9\linewidth}{0pt}}
   \includegraphics[width=1\linewidth]{coconino.jpg}
\end{center}
   \caption{အမေရိကန်နိုင်ငံ အနောက်ပိုင်းရှိ ကိုကိုနီနို သဲကျောက်လွှာ၏ အရွယ်အစား \cite{21}.}
\label{fig:3}
\label{fig:onecol}
\end{figure}

ဂရင်းကင်ညွန်ချောက်ကြီးကို အနီးကပ်လေ့လာပါက ဤမျှကျယ်ပြန့်သော အနည်လွှာများ ပို့ချခဲ့သည့် တစ်ချိန်တည်းတွင် သိသာသော ပြတ်ရွေ့ပြင်းအားများလည်း ဖြစ်ပွားခဲ့ကြောင်း သိရှိနိုင်ပါသည်။ ဤအချက်ကို နားလည်ရန်အတွက် ချောက်ကြီးအတွင်း အနည်လွှာများ တွန့်ခေါက်ပေါ်ထွက်နေသည့် အချို့နေရာများကို အနီးကပ် လေ့လာကြည့်ရပါမည်။ ကမ္ဘာဦးကျမ်း၏အဖြေများ (Answers in Genesis) အဖွဲ့အစည်းမှ သုတေသီများ \cite{42} သည် ဂရင်းကင်ညွှန်၏ မဟာကျောက်တိုင်ကြီး ကဲ့သို့သော တွန့်ခေါက်မှုအချို့၏ ကျောက်တုံးနမူနာများကို အဏုကြည့်မှန်ပြောင်းဖြင့် ကြည့်ရှုခဲ့ပါသည်။ ထိုတွန့်ခေါက်မှုများသည် အပူနှင့် ဖိအားအောက်တွင် ကာလကြာမြင့်စွာ ဖွဲ့စည်းခဲ့ပါက တွေ့ရှိရမည်ဖြစ်သော လက္ခဏာများ မရှိခြင်းကြောင့် အနည်လွှာများသည် အနည်ထိုင်ပြီး သိပ်မကြာခင် နူးညံ့နေစဉ်မှာပင် ပြတ်ရွေ့ပြင်းအားများကြောင့် တွန့်ခေါက်ခဲ့ကြောင်း ကောက်ချက်ချခဲ့ပါသည် \cite{43}။

\begin{figure*}
\begin{center}
% \fbox{\rule{0pt}{2in} \rule{.9\linewidth}{0pt}}
\includegraphics[width=1\textwidth]{Grand_Staircase-big.jpg}
\end{center}
   \caption{ဂရင်းကင်ညွန်ကို ဖွဲ့စည်းထားသည့် အနည်လွှာများ (ပုံ၏ညာဘက်ခြမ်း) သည် မြောက်ဘက်သို့ ယူတားပြည်နယ် စီဒါအက်ကွဲကြောင်းအထိ တိုက်ရိုက်ကျယ်ပြန့်နေကာ ထိုအလွှာအားလုံး အပေါ်သို့ တွန့်ခေါက်နေခြင်း \cite{50}.}
\label{fig:4}
\end{figure*}

အဝေးမှကြည့်သော် ဂရင်းကင်ညွန်တွင် ဖွဲ့စည်းထားသည့် အလွှာများသည် ချောက်ထဲတွင်သာ တွန့်ခေါက်နေခြင်း မဟုတ်ကြောင်း တွေ့ရှိရပါသည်။ ထိုအလွှာများသည် ကိုင်းဘတ်ကုန်းမြင့်အရှေ့ဘက် တွန့်ခေါက်လွှာ \cite{46} ၏ အရှေ့ဘက်တွင် တွန့်ခေါက်ပြီး ယူတားပြည်နယ် စီဒါအက်ကွဲကြောင်း (ပုံ \ref{fig:4}) ၏ မြောက်ဘက်သို့လည်း တွန့်ခေါက်ပါသည်။ ဤအချက်အရ ထိုအလွှာများသည် တစ်ခုပေါ်တစ်ခု အလွန်လျင်မြန်စွာ ထပ်ပြီး တွန့်ခေါက်ခဲ့ကြောင်း ယူဆရပါသည်။ ရည်ညွှန်းချက်အဖြစ် ဂရင်းကင်ညွန်တွင်ရှိသည့် အလျားလိုက် အလွှာများသည် မီတာ ၁၇၀၀ ခန့် ထူပါသည်။ တစ်မိုင်အထူရှိသည့် အနည်လွှာများကို ပို့ချရန်ရန် လိုအပ်သည့် ဘူမိဗေဒဖြစ်စဉ်သည် အလွန်ကြီးမားလှပေသည်။

ဂရင်းကင်ညွန်၏ အမှန်တယ် ဖွဲ့စည်းမှုသည် ခေတ်သစ် ဘူမိဗေဒတွင် နောက်ထပ် သဘောထားကွဲပြားမှုတစ်ရပ် ဖြစ်နေပါသည်။ ကမ္ဘာ့မျက်နှာပြင် ပြောင်းလဲမှုဆိုင်ရာ ဘူမိဗေဒပညာရပ်အရ ဂရင်းကင်ညွန်သည် ကော်လိုရာဒိုမြစ်က နှစ်သန်းပေါင်းများစွာအတွင်း ပုံဖော်ခဲ့သည်ဟု အဆိုပြုထားပါသည် \cite{47}။ သို့သော် ကမ္ဘာဦးကျမ်း၏အဖြေများ အဖွဲ့အစည်းမှ သုတေသနအဖွဲ့က ဂရင်းကင်ညွန်သည် ရှေးခေတ်ရေကန်ကြီးတစ်ခု၏ ရေသွယ်မြောင်း တိုက်စားမှုကြောင့် ကန်ဘောင်များ ကျိုးပေါက်ရာမှ ချောက်ကြီးကို ပုံဖော်သည့် များလှစွာသော အနယ်များကို ဖယ်ရှားသည့်အတွက် သီတင်းပတ် အနည်းငယ်အတွင်း ဖွဲ့စည်းဖြစ်ပေါ်လာခြင်း ဖြစ်နိုင်သည်ဟု ယူဆထားပါသည်။ ရေကန် အနည်ပို့ချမှုများနှင့် ပင်လယ်ကျောက်ဖြစ်ရုပ်ကြွင်းများအရ ဂရင်းကင်ညွန်၏ အရှေ့ဘက်တွင် မြင့်မားသော ဒေသမှ ရေကန်တစ်ခု ရှိကြောင်း သက်သေအထောက်အထား ရှိပါသည်။ ဂရင်းကင်ညွန်ကို အက်ဖတွန်ချောက်ကြီးနှင့် စိန့်ဟယ်လင်တောင် ကဲ့သို့ အခြားသော ပမာဏကြီးမားသည့် ရေသွယ်မြောင်း တိုက်စားမှု နမူနာများနှင့် နှိုင်းယှဉ်ပါက အလားတူ မြေမျက်နှာသွင်ပြင်ကို တွေ့ရှိရပါသည်။ ထို့ကြောင့် ရေပမာဏ များပြားစွာ စီးဝင်ရာမှတစ်ဆင့် ချောက်ကြီးများကို အလျင်အမြန် ဖန်တီးခဲ့နိုင်ကြောင်း ဖော်ပြနေပါသည် \cite{48}။

ဤမျှကြီးမားသော ကုန်းမြေချပ်ကြီးများအပေါ်သို့ အနည်များပို့ချရန် လိုအပ်သည့် ဘူမိဗေဒ ဖြစ်စဉ်များ၏ အတိုင်းအတာ၊ အနည်ပို့ချပြီး မကြာခင် တစ်ပြိုင်နက် ဖြစ်ပွားသည့် ကြီးမားသော ပြတ်ရွေ့ပြင်းအားများ၊ ဂရင်းကင်ညွန်နှင့် နှိုင်းယှဉ်သော် သေးငယ်လှသည့် ကော်လိုရာဒိုမြစ် စသည့်အချက်များကိုကြည့်လျှင် ဤဖွဲ့စည်းမှုသည် တဖြည်းဖြည်းအချိန်ယူပြီး ဖြစ်ပေါ်ခြင်းမဟုတ်နိုင်လောက်တော့ပေ။

\section{ဒရင်ကူယူ မြေအောက်မြို့တော်}

ပိရမစ်များပြီးလျှင် ရှေးခေတ် အင်ဂျင်နီယာပညာ၏ ကြီးမားသော နမူနာတစ်ခုမှာ တူရကီနိုင်ငံ၊ ကပ်ပါဒိုစီယာတွင် တည်ရှိသည့် ဒရင်ကူယူ မြေအောက်မြို့ကြီး (ပုံ ၅\ref{fig:5}) ဖြစ်သည်။ ၎င်းသည် ထိုဒေသရှိ မြေအောက်ခိုလှုံရာနေရာ ၂၀၀ ကျော်အနက် အကြီးမားဆုံးဖြစ်သည် \cite{54}။ ဤမြေအောက်မြို့ကြီးတွင် လူ ၂၀,၀၀၀ အထိ နေထိုင်နိုင်ပြီး အနက် ၈၅ မီတာအထိ အထပ် ၁၈ ထပ် ရှိပါသည်။ ၎င်း၏သက်တမ်းကို အတိအကျ မသိရသော်လည်း အနည်းဆုံး နှစ်ပေါင်း ၂၈၀၀ ရှိပြီဟု ခန့်မှန်းကြပါသည်။ ဤမြို့ကြီးကို ပျော့ပြောင်းသော မီးတောင်ကျောက်များဖြင့် ထွင်းထုထားခဲ့ပါသည် \cite{52, 53}။

\begin{figure}[b]
\begin{center}
% \fbox{\rule{0pt}{2in} \rule{0.9\linewidth}{0pt}}
   \includegraphics[width=1\linewidth]{derinkuyu.jpeg}
\end{center}
   \caption{ဒရင်ကူယူ မြေအောက်မြို့တော်ပုံ \cite{56}။}
\label{fig:5}
\label{fig:onecol}
\end{figure}

ဒရင်ကူယူ၏ စိတ်ဝင်စားစရာကောင်းသော အကြောင်းရင်းမှာ မည်သို့သော လူ့အသိုင်းအဝိုင်းကများ မည်သည့်အကြောင်းပြချက်ကြောင့် မြို့တစ်မြို့လုံးကို မြေအောက်မှာ တည်ဆောက်ရန် ဆုံးဖြတ်ခဲ့သလဲ ဆိုသည့်အချက်ပင် ဖြစ်သည်။ မြေအောက်တွင် နေထိုင်နိုင်သော နေရာများဖန်တီးရန် အခန်းတိုင်းကို ကျောက်သား၌ ထွင်းထုပေးရပါမည်။ မြေအောက်ဥမင်များ၏ ကြမ်းတမ်းသော ပုံသဏ္ဌာန်နှင့် သွင်ပြင်များအရ ထိုလိုဏ်ခေါင်းများကို ဓာတ်အားသုံးကိရိယာများ မသုံးဘဲ လူကိုယ်တိုင်လုပ်အားဖြင့် ထွင်းထုခဲ့ကြကြောင်း သိသာပါသည်။ ယင်းမှာ မြေပြင်အထက် နေထိုင်စရာနေရာများ တည်ဆောက်ရသည်ထက် များစွာခက်ခဲသည့် ကြီးမားသော အလုပ်များဖြစ်ပါသည်။ တကယ်တော့ စိုက်ပျိုးရေး၊ နေရောင်ခြည်၊ သဘာဝနှင့် စူးစမ်းလေ့လာမှုတို့ကို မြေပြင်ပေါ်၌သာ လုပ်ဆောင်နိုင်သည့် ကာလ၌ မည်သည့်လူသားကများ မြေတွင်းအောင်း ဘဝဖြင့် ပုန်းအောင်းကာ မြေအောက်တွင် အမြဲတမ်း နေထိုင်လိုသလဲ ဆိုသည့်အချက်မှာ မသဲကွဲလှပါ။ ရှေးရိုး "ရာဇဝင်" အရ ဒရင်ယူကူမြို့ကြီးကို ကိုယ်ပိုင်ဘာသာရေး လှုပ်ရှားရန် သီးခြားနေရာတစ်ခု လိုအပ်သော ခရစ်ယာန်များက တည်ထောင်ခဲ့သည်ဟု ဆိုပါသည် \cite{53}။ သို့သော် သာမန်တွေးကြည့်လျှင်ပင် ရန်သူကို ရင်ဆိုင်ရာ၌ အသိသာဆုံးနည်းလမ်းမှာ “တိုက်မလား၊ ပြေးမလား” နည်းလမ်းသာ ဖြစ်ပြီး ကျောက်သားကို ထွင်းထုကာ မြေအောက်မြို့တော်တည်ဆောက်ခြင်း မဟုတ်ကြောင်း ကောက်ချက်ချနိုင်ပါသည်။

မြေအောက်မြို့ကြီး၏ အတိုင်းအတာ၊ အနက်နှင့် စိတ်ကူးကောင်းမှုတို့ ကြည့်လျှင် ၎င်းသည် ခြိမ်းခြောက်ခံရသည့် ကာလများတွင် ကျူးကျော်သူများကို ပိုမိုကောင်းမွန်စွာ တိုက်ခိုက်နိုင်အောင် ယာယီ စစ်ရေးကာကွယ်မှုအတွက် တည်ဆောက်ထားခြင်း မဟုတ်ဘဲ မြေပြင်ပေါ်မှ သေစေနိုင်လောက်အောင် အန္တရာယ်ကြီးများရန်မှ ကာကွယ်နိုင်အောင် ‌ရေရှည်နေထိုင်ရေးအတွက် စီစဉ်ထားခြင်းဖြစ်ကြောင်း သိသာပါသည်။ ဒရင်ကူယူမြို့ကြီးတွင် အခြေခံ အိပ်ခန်းများ၊ မီးဖိုများနှင့် သန့်စင်ခန်းများ ထည့်သွင်းထားသည်သာမက တိရစ္ဆာန်များအတွက် မြင်းဇောင်းများ၊ ရေလှောင်ကန်များ၊ အစားအသောက် သိမ်းဆည်းရာနေရာများ၊ ဝိုင်နှင့် ဆီကျိတ်စက်များ၊ ကျောင်းများ၊ ဘုရားရှိခိုးကျောင်းများ၊ အုတ်ဂူများနှင့် ကြီးမားသည့် လေဝင်လေထွက်တိုင်များလည်း ပါဝင်ပါသည် (ပုံ \ref{fig:6})။ စစ်ရေးခိုလှုံရာ နေရာတစ်ခုတွင် ဝိုင်ကျိတ်စက် ဘာကြောင့်လိုအပ်ပါသနည်း။ ဤမျှရှုပ်ထွေးသည့် ၈၅ မီတာအနက်တူးရန်လည်း ဘာကြောင့်လိုအပ်ပါသနည်း။

ဒရင်ကူယူကို တည်ဆောက်ရခြင်းအတွက် ယုတ္တိအရှိဆုံး အကြောင်းပြချက်မှာ ကမ္ဘာ့မျက်နှာပြင်ပေါ်ရှိ သဘာဝဘေးအန္တရာယ်ဆိုးကြီးများ ဖြစ်ပေါ်နိုင်သော ဘူမိရူပ ပြင်းအားများရန်မှ ကာကွယ်ရန် ရေရှည် ပြည့်စုံလုံလောက်သော နေထိုင်စရာ နေရာတစ်ခု ကြိုတင်ပြင်ဆင်ရန် အပူတပြင်း လိုအပ်ချက်ကြောင့်သာ ဖြစ်ရပါမည်။

\begin{figure}[t]
\begin{center}
% \fbox{\rule{0pt}{2in} \rule{0.9\linewidth}{0pt}}
   \includegraphics[width=1\linewidth]{derinkuyu-air.jpg}
\end{center}
   \caption{ဒရင်ကူယူရှိ မြေအောက် လေဝင်လေထွက်တွင်းတစ်တွင်း \cite{53}။}
\label{fig:6}
\label{fig:onecol}
\end{figure}

% \section{Additional Anomalies Best Explained By An Earth Flip}

% Before wrapping up, we will mention some additional scientific anomalies that, once viewed in the context of cataclysmic geophysical forces, are well explained.

\section{ဇီဝဒြပ်ထု စုစည်းလာမှုများ}

အနည်လွှာများထဲတွင် ကျောက်ဖြစ်ရုပ်ကြွင်း ဖြစ်နေသည်ကို မကြာခဏ တွေ့ရလေ့ရှိသည့် တိရစ္ဆာန်နှင့် အပင်အမျိုးမျိုး၏ ဇီဝဒြပ်ထု အရောအနှောများသည် နောက်ထပ်ပ‌ဟေဠိဆန်သော ထူးခြားမှုတစ်ခုဖြစ်ပါသည်။ “ကျောက်ဖြစ်ရုပ်ကြွင်းများကို လေ့လာမှု” စာအုပ်တွင် ဘုန်းတော်ကြီး ဝီလီယံဘက္ကလင်းက အတူတကွ တွေ့ရှိစရာအကြောင်းမရှိသည့် တိရစ္ဆာန်အမျိုးစားများစွာသည် ဗြိတိန်နှင့် ဥရောပတိုက်တစ်ခွင်ရှိ အနည်လွှာများထဲ၌ နစ်မြုပ်နေကြောင်း အသေးစိတ်တွေ့ရှိချက်များကို ဖော်ပြခဲ့ပါသည် \cite{58}။ အဆိုပါ တိရစ္ဆာန် အကြွင်းအကျန် အစုအဝေးများကို နော်ဝေနိုင်ငံ ဗဲလ်ဒရွိုင်းကျွန်းရှိ ဂျွန်ဂယ်လာရန်ဂူထဲတွင်လည်း တွေ့ရှိခဲ့ပါသည်။ ဤဂူထဲတွင် နို့တိုက်သတ္တဝါ၊ ငှက်နှင့် ငါးများ၏ အရိုး ၇,၀၀၀ ကျော်သည် အနည်လွှာ အထပ်ထပ်ထဲတွင် ရောနှောနေသည်ကို တွေ့ရှိခဲ့ပါသည် \cite{59}။ နောက်ထပ် ဥပမာတစ်ခုမှာ အီတလီနိုင်ငံမှ တိရစ္ဆာန်ကြီးများ၏ဂူဆိုသည့် ဆန်ကိုင်ရို ဂူဖြစ်သည်။ ဤဂူထဲတွင် နို့တိုက်သတ္တဝါ၊ အရိုးများ တန်နှင့်ချီ၍ တွေ့ရှိခဲ့ပါသည်။ အများစုမှာ ရေမြင်းအရိုးများဖြစ်သည်။ ယင်းတို့သည် အလွန်လတ်လတ်ဆတ်ဆတ် အခြေအနေ၌ ရှိနေသေးသည့်အတွက် အဆင်တန်ဆာများအဖြစ် ဖြတ်တောက်ကာ မီးခဲပြာထုတ်လုပ်ရန်အတွင် တင်ပို့ခဲ့ပါသည်။ တိရစ္ဆာန်မျိုးစုံ၏ အရိုးများသည် အတူတကွရောနှောပြီး ပျက်စီးကြေမွကာ အစိတ်စိတ်အမွှာမွှာ ပြန့်ကျဲနေသည်ဟု သိရှိရပါသည် \cite{60,61}။ အီဂျစ်နိုင်ငံရှိ မန်းဒက်စ်ရှေးဟောင်းမြို့တွင် တိရစ္ဆာန်မျိုးစိတ်များစွာ၏ အရိုးများသည် မှန်အဖြစ်ပြောင်းလဲသွားသော ရွှံ့နှင့် ရောနှောနေသည်ကို တွေ့ရှိထားပါသည် \cite{57}။ အဆိုပါတွေ့ရှိမှုများသည် ပဟေဠိဆန်သည်ဟု ထင်ရသော်လည်း ရေကြီးရေလျှံမှု များစွာဖြစ်ပြီးနောက် တိရစ္ဆာန်အသေကောင် အရောအနှောများကို အနည်လွှာများထဲတွင် ပို့ချရာမှ တိရစ္ဆာန်များကို ဂူများထဲတွင် အရှင်လတ်လတ် ပို့ချမြှုပ်နှံခြင်းဖြစ်ခဲ့သည်ဟု အလွယ်တကူ ရှင်းပြ၍ရပါသည်။ အီဂျစ်နိုင်ငံရှိ မှန်အဖြစ်ပြောင်းလဲသွားသော ဇီဝဒြပ်ထုများနှင့် ပတ်သက်၍ ရေကြီးရေလျှံပြီးနောက် ကမ္ဘာ့ဗဟိုချက်အလွှာ ရွေ့လျားရာမှ လျှပ်စစ်ဓာတ် အကြီးအကျယ် ထုတ်လွှတ်မှုများ ရှိခဲ့ပါသည်။ ပုံ \ref{fig:7} တွင် အလက်စကာ ဇီဝဒြပ်ထု ‘အညစ်အကြေးများ' ပုံမှန်အားဖြင့် ပေါ်ထွက်မှုကို ပုံဖော်ထားပါသည် \cite{56}။

\begin{figure}[t]
\begin{center}
% \fbox{\rule{0pt}{2in} \rule{0.9\linewidth}{0pt}}
   \includegraphics[width=1\linewidth]{muck-crop.jpeg}
\end{center}
   \caption{သစ်ပင်ကြီးများ၊ အပင်ငယ်များနှင့် တိရစ္ဆာန်များ၏ အစိတ်အပိုင်းများကို အေးခဲနေသော နုန်းနှင့် ရေခဲပြင်ထဲ၌ ရောထွေးပြန့်ကျဲကာ ဖွဲ့စည်းထားသည့် အလက်စကာ 'အညစ်အကြေးများ’ \cite{146}။}
\label{fig:7}
\label{fig:onecol}
\end{figure}

\section{ရှေးခေတ် ဗုံးခိုကျင်းများ}

ကျွန်ုပ်တို့၏ ဘိုးဘေးများ ချန်ထားရစ်ခဲ့သော အင်ဂျင်နီယာနည်းပညာမြင့်မားသည့် ရှေးခေတ်အဆောက်အအုံများစွာထဲတွင် လူရိုးစုများကို တွေ့ရှိရပါသည်။ ထိုအဆောက်အအုံများကို များသောအားဖြင့် ကြီးကျယ်ခမ်းနားသည့် အုတ်ဂူများအဖြစ် ယူဆကြလေ့ရှိသော်လည်း အနီးကပ် လေ့လာကြည့်ပါက ယင်းတို့သည် တကယ်တော့ ရှေးခေတ် ဗုံးခိုကျင်းများ ဖြစ်ခဲ့ပါသည်။

\begin{figure}[b]
\begin{center}
% \fbox{\rule{0pt}{2in} \rule{0.9\linewidth}{0pt}}
   \includegraphics[width=1\linewidth]{ww19.jpg}
\end{center}
   \caption{အိုင်ယာလန်နိုင်ငံရှိ နယူးဂရန့်ဂူ - ဝင်ပေါက်ရှိ ခရီးသွားများကိုကြည့်လျှင် အတိုင်းအတာကို သိနိုင်ပါသည်။}
\label{fig:8}
\label{fig:onecol}
\end{figure}

ထင်ရှားသော နမူနာတစ်ခုမှာ နယူးဂရန့်ဂူ (ပုံ \ref{fig:8}) ဖြစ်ပြီး ၎င်းသည် ဝင်ပေါက်ပါသော အုတ်ဂူများအပါအဝင် ရှေးခေတ်အဆောက်အအုံများ အစုအဝေးဖြစ်သည့် ဘရူနာဘွန်ညဲ အုတ်ဂူအစုအဝေးတွင် အဓိကအဆောက်အအုံ ဖြစ်သည်။ ထိုအုတ်ဂူများထဲတွင် မြေကြီး သို့မဟုတ် ကျောက်တုံးများဖြင့် ဖုံးအုပ်ထားသည့် မြှုပ်နှံသောအခန်းတစ်ခု သို့မဟုတ် တစ်ခုထက်ပို၍ ပါဝင်ကာ ကျောက်တုံးကြီးများဖြင့် ပြုလုပ်ထားသည့် ခပ်ကျဉ်းကျဉ်း ဝင်ပေါက်လမ်းကြောင်းတစ်ခုရှိသည် \cite{70}။ ၎င်းသည် တည်ဆောက်မှုစတင်သောအချိန်၌ အသက်ရှင်လျှက်ပင် မရှိတော့သော လူအနည်းငယ်၏ အလောင်းမြှုပ်နှံရန်အတွက် မျိုးဆက်များစွာ တည်ဆောက်ထားရသော ရှုပ်ထွေးသည့် ကာကွယ်ရေး အဆောက်အအုံကြီးတစ်ခု၏ ကြီးမားသော အင်ဂျင်နီယာပညာ နမူနာတစ်ခုဖြစ်ပါသည်။ ၁၆၉၉ ခုနှစ်တွင် ဒေသခံမြေရှင်တစ်ဦးက ၎င်းကို ပြန်လည်ရှာဖွေတွေ့ရှိချိန်၌ မြေကြီးအောက်တွင် မြုပ်နေပါသည်။

အဆောက်အအုံကို သာမန်ကြည့်မိလျှင် ယင်းကိုတည်ဆောက်ရာ၌ ကြီးစွာသော အားထုတ်မှု ပြုလုပ်ခဲ့ရသည်ကို တွေ့ရှိနိုင်ပါသည်။ နယူးဂရန့်ဂူတွင် တန်ချိန် ၂၀၀,၀၀၀ ခန့်ရှိသော ပစ္စည်း များပါဝင်သည်။ အတွင်းဘက်တွင် \textit{“...အခန်းပါဝင်သော လမ်းကြောင်းတစ်ခုရှိပြီး ဂူ၏ အရှေ့တောင်ဘက်မှ ဝင်ပေါက်ကတစ်ဆင့် ဝင်ရောက်နိုင်ဖွယ် ရှိပါသည်။ ထိုလမ်းကြောင်းသည် ၁၉ မီတာ (၆၀ ပေ) ရှည်ကာ တစ်နည်းအားဖြင့် အဆောက်အအုံ၏ အလယ်ဗဟိုထဲအထိ သုံးပုံတစ်ပုံခန့် ရှိပါသည်။ လမ်းအဆုံးတွင် အလယ်ဗဟို၌ ကျောက်လွှာများဖြင့်ထပ်ပြီး မြင့်မားသော အမိုးခုံး တည်ဆောက်ထားသည့် အခန်းကြီးတစ်ခန်းရှိပြီး ဘေးဘက်တွင် အခန်းငယ်သုံးခန်းရှိပါသည်... ဤလမ်း၏နံရံများကို အော်သိုစတက်ဟုခေါ်သည့် ကျောက်ပြားကြီးများဖြင့် ဖွဲ့စည်းထားသည်။ အနောက်ဘက်တွင် ကျောက်ပြား နှစ်ဆယ့်နှစ်ချပ်နှင့် အရှေ့ဘက်တွင် နှစ်ဆယ့်တစ်ချပ် ရှိပါသည်။ ပျမ်းမျှ ၁.၅ မီတာ အမြင့်ရှိပါသည်"} \cite{70}။ ရေလုံသော အသေးစိတ် အင်ဂျင်နီယာလုပ်ငန်းများလည်း ပါဝင်ပါသည်။ ဥပမာအားဖြင့် အမိုးပေါ်တွင် \textit{“အမိုးအက်ကွဲကြောင်းများသည် မီးဖုတ်ထားသောမြေကြီးနှင့် ပင်လယ်သဲများ ရောစပ်၍ ရေမဝင်အောင် မံထားပါသည်။ ဤအရောအစပ်ပစ္စည်းမှ ဂူ၏ဖွဲ့စည်းပုံအတွက် ရယူထားသည့် ရေဒီယိုကာဘွန် ရက်စွဲနှစ်ခုသည် ခရစ်တော်မပေါ်မီ နှစ်ပေါင်း ၂၅၀၀ ကို ဗဟိုပြုဖော်ပြပါသည်"} \cite{71}။ ထို့ပြင် အတွင်းခန်းသို့ မြေမျက်နှာပြင် မြင့်တက်သွားသည်မှာလည်း ယခုလို အလားတူရည်ရွယ်ချက်အတွက် ဖြစ်နိုင်ပါသည်- \textit{“ဂူ၏ လမ်းနှင့် အခန်း၏ ကြမ်းပြင်သည် ထိုအထိမ်းအမှတ်အဆောက်အအုံ တည်ရှိသည့် တောင်စောင်း မြေမျက်နှာပြင် မြင့်တက်မှုအတိုင်း လိုက်သွားသောကြောင့် အခန်း၏ ဝင်ပေါက်နှင့် အခန်းအတွင်းဘက်အကြားတွင် ကြမ်းပြင် ၂ မီတာနီးပါး အမြင့်ကွာခြားပါသည်”} \cite{71}။

\begin{figure}[b]
\begin{center}
% \fbox{\rule{0pt}{2in} \rule{0.9\linewidth}{0pt}}
   \includegraphics[width=1\linewidth]{dolmen.jpg}
\end{center}
   \caption{ဒေါ်မင်ဒီဆိုတို၊ စပိန် \cite{53}။}
\label{fig:9}
\label{fig:onecol}
\end{figure}

အတွင်းဘက်တွင် ရုပ်အလောင်းများ မတွေ့ရခြင်းသည်လည်း စိတ်ဝင်စားစရာဖြစ်သည်။ တူးဖော်မှုများအရ မီးလောင်ထားသော အရိုးအပိုင်းအစများနှင့် မီးမလောင်သော အရိုးအပိုင်းအစများကို တွေ့ရှိခဲ့ရာ လူအနည်းငယ်ကိုသာ ကိုယ်စားပြုပြီး လမ်းကြောင်းထဲတွင် ပြန့်ကျဲနေပါသည်။ နယူးဂရန့်ကို ဆောက်လုပ်ရာတွင် အတွင်းဘက်ရှိပစ္စည်းများမှ ကာဗွန်နေ့စွဲများအရ အနည်းဆုံးမျိုးဆက်ပေါင်းများစွာ အချိန်ယူခဲ့မည်ဟု ခန့်မှန်းရပါသည်။ ရှေးခေတ်လူ့အသိုင်းအဝိုင်း တစ်ခုသည် သေဆုံးသူအနည်းငယ်၏ အရိုးအပိုင်းအစများကို ဂူထဲဝင်သည့်လမ်းပေါ်တွင် ဖြန့်ကျဲထားလောက်ရုံအတွက်မျှနှင့် ကြီးမားသော အင်ဂျင်နီယာနည်းပညာ မြင့်မားသည့် အုတ်ဂူကြီးတစ်ခု တည်ဆောက်ရန် အလွန်ကြီးလေးသော အားထုတ်မှု ဘာကြောင့်ပြုလုပ်ခဲ့ပါသနည်း။ ကမ္ဘာကြီးတွင် ပြန်ဖြစ်လေ့ရှိသော သဘာဝဘေးအန္တရာယ်ကြီးများကို ကာကွယ်ရန် လူနေအိမ်များအဖြစ် ရေလုံအောင် ဂရုတစိုက်ဆောက်လုပ်ခဲ့သည့် အဆိုပါ ရှေးခေတ် မဟာအဆောက်အအုံကြီးများကို ဆောက်လုပ်ခဲ့ကြသည်ဟု ယူဆလျှင် ပို၍ယုတ္တိရှိပါသည်။

စပိန်နိုင်ငံ တောင်ပိုင်းရှိ ဟူဝဲလ်ဗာမြို့တွင် အလားတူနမူနာတစ်ခုမှာ ဒေါ်မင်ဒီဆိုတို (ပုံ \ref{fig:9}) ဖြစ်ပြီး ထိုဒေသရှိ အလားတူနေရာ ၂၀၀ ခန့်အနက်တစ်ခု ဖြစ်ပါသည် \cite{72,32}။ ၎င်းသည် ဧရာမကျောက်တုံးကြီးများ အသုံးပြုပြီး အင်ဂျင်နီယာနည်းပညာ မြင့်မားစွာ အချောသတ်ဆောက်လုပ်ထားသည့် အဆောက်အအုံတစ်ခုဖြစ်ပြီး အချင်း ၇၅ မီတာရှိပါသည်။ တူးဖော်သောအခါ ရုပ်အလောင်း ရှစ်ခုသာ တွေ့ရှိပြီး ယင်းတို့အားလုံးကို သန္ဓေသားပုံစံဖြင့် မြှုပ်နှံထားသည်ဟု သိရပါသည်။

\section{မှတ်သားဖွယ်ရာ ထူးခြားမှုအကြောင်း ဖော်ပြချက်များ}

ဤအပိုင်းတွင် အနည်းငယ်ပို၍ မှတ်သားဖွယ်ကောင်းသော ထူးခြားမှုများကို အကျဉ်းချုပ် ဖော်ပြထားပြီး ယင်းတို့အားလုံးကို ECDO ကဲ့သို့ သဘာဝဘေးအန္တရာယ်ဖြစ်စဉ်အရ ကောင်းစွာ ရှင်းပြထားပါသည်။

\subsection{ဇီဝဗေဒ ထူးခြားမှုများ}

\begin{figure}[b]
\begin{center}
% \fbox{\rule{0pt}{2in} \rule{0.9\linewidth}{0pt}}
   \includegraphics[width=1\linewidth]{bottleneck.jpg}
\end{center}
   \caption{လွန်ခဲ့သောနှစ်ပေါင်း ၆,၀၀၀ ခန့်က အမျိုးသား ၉၅\% လျော့ကျမှုကို ကိုယ်စားပြုသည့် မျိုးဗီဇ လျော့ကျမှု \cite{62}။}
\label{fig:10}
\label{fig:onecol}
\end{figure}

မှတ်သားဖွယ်ရာ ဇီဝဗေဒ ထူးခြားမှု အချို့သည် မျိုးဗီဇ လျော့ကျမှုနှင့် ကုန်းတွင်း ဝေလငါး ကျောက်ဖြစ်ရုပ်ကြွင်းများ ဖြစ်သည်။ ဇန်နှင့်အဖွဲ့ (၂၀၁၈) က ခေတ်သစ်လူသားများထံမှ Y ခရိုမိုဇုန်းအတွဲ ၁၂၅ ခုကို စံပုံစံဖြင့် လေ့လာကာ မျိုးဗီဇ တူညီမှုများ၊ မြုံနေမှုများကို အခြေခံကာ လွန်ခဲ့သောနှစ်ပေါင်း ၅,၀၀၀ မီ ၇,၀၀၀ ခန့်က အမျိုးသားဦးရေ ၉၅\% လျော့ကျမှုကို ဖော်ထုတ်ခဲ့ပါသည် (ပုံ \ref{fig:10}) \cite{62}။ ဝေလငါး ကျောက်ဖြစ်ရုပ်ကြွင်းများကို ဆွီဒင်ဘော့၊ မစ်ရှီဂန်၊ ဗားမောင့်၊ ကနေဒါ၊ ချီလီနှင့် အီဂျစ်ရှိ ပင်လယ်ရေမျက်နှာပြင်အထက် မီတာပေါင်း ရာနှင့်ချီ၍မြင့်သောနေရာတွင် တွေ့ရှိခဲ့ပါသည် \cite{63,64,65,66}။ ထိုဝေလငါးများကို- အလုံးစုံကောင်းစွာ တည်ရှိနေခြင်း၊ ရေခဲမြစ် အနည်ပို့ချမှုများအထက်ရှိ နွံထဲတွင် လဲနေခြင်း သို့မဟုတ် အနည်ပို့ချမှုများထဲတွင် မြုပ်နေခြင်းစသည့် အနေအထားမျိုးစုံဖြင့် တွေ့ရပါသည်။ အဆိုပါနေရာများရှိ နမူနာ အရေအတွက်သည် အနည်းငယ်မှ တစ်ရာကျော်အထိရှိပါသည်။ ဝေလငါးများသည် ပင်လယ်နက်ထဲမှ သတ္တဝါများဖြစ်ပြီး ကမ်းခြေအနီးသို့ ရောက်လာလေ့မရှိပါ။ ထို ဝေလငါးများသည် ဤမျှအမြင့်ရှိသောနေရာသို့၊ တစ်ခါတစ်ရံ အလွန်ဝေးကွာသည့် ကုန်းတွင်းပိုင်းသို့ မည်သို့မည်ပုံ ရောက်ရှိလာခဲ့ပါသနည်း။

ကမ္ဘာမြေပေါ်တွင် အစုလိုက်အပြုံလိုက် မျိုးသုဉ်းပျောက်ကွယ်မှုများ ယခင်က ဖြစ်ပွားခဲ့ကြပြီး ထိုအထဲမှ အကောင်းဆုံးလေ့လာခဲ့သည်များမှာ ဖာနီရိုဇွိုက်အဖြစ်အပျက် "အကြီးဆုံး ငါးခု" ဟုခေါ်သည့်- အော်ဒိုဗီရှန်ခေတ် နှောင်းပိုင်း (LOME)၊ ဒီနိုဗီယန်ခေတ် နှောင်းပိုင်း (LDME)၊ ပါမီယန်ခေတ် ကုန်ဆုံးချိန် (EPME)၊ ထရိုင်ယက်စစ်ခေတ် ကုန်ဆုံးချိန် (ETME) နှင့် ခရီတေးရှပ်စ်ခေတ် ကုန်ဆုံးချိန် (ECME) မျိုးသုဉ်းပျောက်ကွယ်မှုများ ဖြစ်သည် \cite{88,89}။ စိတ်ဝင်စားစရာကောင်းသည့်အချက်မှာ အဆိုပါ မျိုးသုဉ်းပျောက်ကွယ်မှုများစွာကို ပါမီယန်ခေတ်နှင့် ဒီနိုဗီယန်ခေတ်စသည့် ဂရင်းကင်ညွန်၏ မြေလွှာအများအပြားနှင့် သမိုင်းဆိုင်ရာကာလ တူညီသောအချိန်တွင် ဖြစ်ပွားကြောင်း သတ်မှတ်ထားခြင်းဖြစ်ပါသည်။

\subsection{ရုပ်ပိုင်းဆိုင်ရာ ထူးခြားချက်များ}

\begin{figure}[b]
\begin{center}
% \fbox{\rule{0pt}{2in} \rule{0.9\linewidth}{0pt}}
   \includegraphics[width=1\linewidth]{columbia.jpg}
\end{center}
   \caption{ဝါရှင်တန်ပြည်နယ်၊ ကိုလံဘီယာ ရေခဲမြစ်ရေကန်ရှိ အလွန်ကြီးမားသော ရေစီးကြောင်းလှိုင်းပုံစံများ။ \cite{80}.}
\label{fig:11}
\label{fig:onecol}
\end{figure}

ဂရင်းကင်ညွန်အပြင် သဘာဝဘေးအန္တရာယ်ဆိုးကြီးများ ဖြစ်ပေါ်နိုင်သော ပြင်းအားများကြောင့် ပေါ်ပေါက်လာပုံရသည့် အခြားသော မြေပြင်ပုံစံများစွာ ရှိပါသည်။ ကြီးမားသော ကုန်းမြေပေါ်သို့ ရေစီးဆင်းမှု အထောက်အထားများကို ကမ္ဘာတစ်ဝန်းရှိ ကြီးမားသော ရေစီးကြောင်း လှိုင်းပုံစံများကိုကြည့်၍ တွေ့ရှိနိုင်ပါသည်။ အဆိုပါ ဥပမာတစ်ခုမှာ ပစိတ်ဖိတ်သမုဒ္ဒရာ အနောက်မြောက်ပိုင်းရှိ ချန်နယ်စခပ်လန်းဆိုသည့် ရေစီးကြောင်းအရာထင်ခဲ့သောဒေသ ဖြစ်သည်။ ဤနေရာတွင် အနည်ထိုင်ပို့ချမှု မြေပြင်ပုံစံများကိုသာမက ပုံမှန်မဟုတ်သော ကျောက်တုံးကြီးများကိုလည်း တွေ့ရပါသည်။ ထို့ပြင် ကြီးမားလှသော ရေစီးကြောင်းများကြောင့် ဖြစ်ပေါ်လာသည့် လှိုင်းပုံစံအစဉ်တန်း တစ်ရာကျော်ကိုလည်း တွေ့ရပါသည် \cite{78,79} ။ ၎င်းတို့သည် စမ်းချောင်းများ၏ သဲသောင်ပြင်၌ ဖြစ်ပေါ်သည့် လှိုင်းပုံစံများ၏ အကြီးစားပုံစံများ ဖြစ်ပါသည်။ ထိုလှိုင်းပုံစံများကို တစ်ကမ္ဘာလုံး၌ ပြင်သစ်၊ အာဂျင်တီးနား၊ ရုရှားနှင့် မြောက်အမေရိကတို့တွင် တွေ့နိုင်ပါသည် \cite{81} ။ ပုံ \ref{fig:11} တွင် အမေရိကန်ပြည်ထောင်စု၊ ဝါရှင်တန်ပြည်နယ်ရှိ လှိုင်းပုံစံအချို့ကို ဖော်ပြထားပါသည် \cite{80}။

\begin{figure}[b]
\begin{center}
% \fbox{\rule{0pt}{2in} \rule{0.9\linewidth}{0pt}}
   \includegraphics[width=1\linewidth]{zhangjiajie.jpg}
\end{center}
   \caption{တရုတ်နိုင်ငံတောင်ပိုင်း ကျန်းကျားကျယ် အမျိုးသားသစ်တောထဲမှ အလွန်ကြီးမားသော ကျောက်တိုင်ကြီးများ။}
\label{fig:12}
\label{fig:onecol}
\end{figure}

\begin{figure}[b]
\begin{center}
% \fbox{\rule{0pt}{2in} \rule{0.9\linewidth}{0pt}}
   \includegraphics[width=1\linewidth]{hoy.jpg}
\end{center}
   \caption{စကော့တလန်၊ ဟွိုက်ကျွန်းပေါ်ရှိ ပင်လယ်ကမ်းစပ်ကျောက်တိုင်ကြီး \cite{83}။}
\label{fig:13}
\label{fig:onecol}
\end{figure}

ကုန်းတွင်း တိုက်စားမှုကြောင့် ဖြစ်ပေါ်သောပုံစံများကို ECDO နှင့်တူသော ကမ္ဘာ့ဝင်ရိုး ချော်ထွက်မှုဖြင့် ကောင်းစွာ ရှင်းပြနိုင်ပါသည်။ တရုတ်နိုင်ငံတောင်ပိုင်းသည် ကျောက်သားများကို ရေတိုက်စားမှုကြောင့် ဖြစ်ပေါ်လာသည့် ကြီးမားသော ကျောက်သားတိုက်စားမှု မြေပြင်ပုံစံကြီးများ၏ နမူနာတစ်ခုဖြစ်သည် \cite{82}။ ဤမြေပြင်ပုံစံကြီးများတွင် မျှော်စင် တာဝါတိုင်ကျောက်သား၊ အမိုးချွန် ကျောက်သား၊ ကန်တော့ချွန် ကျောက်သား၊ သဘာဝတံတားများ၊ လျှိုများ၊ ကြီးမားသော ဂူအဖွဲ့အစည်းများနှင့် တွင်းများ ပါဝင်ပါသည်။ ယင်းတို့အနက် အထင်ရှားဆုံးတစ်ခုမှာ ကျန်းကျားကျယ် အမျိုးသားသစ်တောဖြစ်ပြီး ကြီးမားသော သလင်းကျောက် သဲကျောက်တိုင်ကြီးများ ပါဝင်ပါသည် (ပုံ \ref{fig:12}) \cite{84}။ ဤကျောက်တိုင်များသည် ပျမ်းမျှ အမြင့် ၁,၀၀၀ မီတာကျော်ရှိပြီး ကျောက်တိုင်စုစုပေါင်း ၃,၁၀၀ ကျော်ရှိပါသည်။ ၎င်းတို့အနက် ၁,၀၀၀ ကျော်သည် ၁၂၀ မီတာကျော် မြင့်မားကာ ၄၅ ခုသည် ၃၀၀ မီတာကျော်ပါသည် \cite{85}။ ဤကျောက်တိုင်များသည် ပင်လယ်ရေ တိုက်စားသည့် ကျောက်တိုင်များနှင့် ဆင်တူပါသည် (ပုံ \ref{fig:13})။ ယင်းတို့မှာ ပင်လယ်လှိုင်းများကြောင့် အနီးဝန်းကျင်မှ အရာဝတ္ထုများ ပြိုလဲရာမှ ဖြစ်ပေါ်လာသည့် ကမ်းရိုးတန်း ကျောက်တိုင်ကြီးများ ဖြစ်သည်။ အလားတူ တိုက်စားသော မြေပြင်ပုံစံများကို တူရကီနိုင်ငံ၊ အူဂူရှိ ကန်တော့ချွန်ပုံကျောက်တုံးများနှင့် စပိန်နိုင်ငံ ဆောဒက်အန်ခန်တာဒါတို့တွင်လည်း တွေ့နိုင်ပြီး ယင်းတို့နှစ်ခုစလုံးမှာ ပင်လယ်ရေမျက်နှာပြင်အထက် မီတာ ၁,၀၀၀ ကျော် မြင့်ပါသည်။ ထိုနေရာအားလုံးတွင် ယင်းတို့အနီးအနား၌ ဆားနှင့် ပင်လယ်ရေသတ္တဝါ ကျောက်ဖြစ်ရုပ်ကြွင်းများ ပေါင်းစပ်ထားမှုအချို့ကို တွေ့ရသည့်အတွက် ယခင်က ပင်လယ်ရေ ဝင်ရောက်ခဲ့ခြင်းဟု ယူဆရပါသည် \cite{15,86,87}။ ရေကြီးခဲ့သောဇာတ်လမ်းများ \cite{3} တွင် ပင်လယ်ရေသည် ၁,၀၀၀ မီတာကျော် မြင့်တက်ခဲ့သည်ဟု ဖော်ပြထားပြီး ယင်းကို ပင်လယ်ရေမျက်နှာပြင်အထက် ကီလိုမီတာများစွာရှိသည့် အင်းဒီးနှင့် ဟိမဝန္တာတောင်တန်းများမှ ဆားငန်ရေနှင့် ကြီးမားသော ဆားလွင်ပြင်များ တည်ရှိမှုဖြင့် အတည်ပြုနိုင်ပါသည်။ ဥပမာအားဖြင့် ဘိုလီးဗီးယားနိုင်ငံရှိ အူယူနီ ဆားလွင်ပြင်သည် ပင်လယ်ရေမျက်နှာပြင်အထက် ၃၆၅၃ မီတာရှိပါသည် \cite{94}။

\subsection{လျင်မြန်သော ရာသီဥတု ပြောင်းလဲမှုဖြစ်ရပ်များ}

ခေတ်သစ် သိပ္ပံစာပေများတွင် ကမ္ဘာကြီး၏ မကြာသေးမီက သမိုင်းကြောင်းတွင် အလွန်လျင်မြန်သည့် ကမ္ဘာလုံးဆိုင်ရာ ရာသီဥတုအပြောင်းအလဲ ဖြစ်ရပ်များရှိခဲ့သည်ကို ဖော်ပြထားပါသည်။ ထင်ရှားသော ဥပမာနှစ်ခုမှာ နှစ်ပေါင်း ၄,၂၀၀ နှင့် နှစ်ပေါင်း ၈,၂၀၀ ဖြစ်ရပ်များ ဖြစ်ပြီး ထိုနှစ်ခုစလုံးသည် လူဦးရေ လျော့နည်းခြင်းနှင့် လူမှုရေးအခြေချမှုဆိုင်ရာ ရပ်တန့်ခြင်းများနှင့် တိုက်ဆိုင်နေပါသည်။ ဤအဖြစ်အပျက်များကို အနည်လွှာများနှင့် ရေခဲပြင် ဗဟိုချက်များ၊ ကျောက်ဖြစ်ရုပ်ကြွင်း  သန္တာကျောက်များ၊ O18 အိုင်ဆိုတုပ် တန်ဖိုးများ၊ ဝတ်မှုန်နှင့် ကျောက်အနည်ထိုင်မှု မှတ်တမ်းများနှင့် ပင်လယ်ရေမျက်နှာပြင် အချက်အလက်များတွင် ထူးခြားချက်များအဖြစ် သိမ်းဆည်းထားပါသည်။ ယူဆရသည့် ရာသီဥတုအပြောင်းအလဲများထဲတွင် ကမ္ဘာလုံးဆိုင်ရာ အပူချိန် လျင်မြန်စွာ လျော့ကျသွားခြင်း၊ မိုးခေါင်ရေရှားခြင်း၊ အတ္တလန္တိတ် တောင်ပိုင်း နောက်ပြန်ရေစီးကြောင်းများ ရပ်တန့်ခြင်းနှင့် ရေခဲပြင်များ တိုးပွားလာခြင်းတို့ ပါဝင်ပါသည် \cite{90,91,92}။ အထူးသဖြင့် နှစ်ပေါင်း ၈,၂၀၀ ဖြစ်ရပ်မှာ ခရစ်တော်မပေါ်မီ နှစ်ပေါင်း ၆၄၀၀ ဝန်းကျင်၌ ပင်လယ်နက်တွင် သိသာသော ဆားငန်ရေ တိုးဝင်မှု ဖြစ်ပွားခဲ့နိုင်ခြင်းနှင့် တိုက်ဆိုင်နေပါသည် \cite{93}။

\subsection{ရှေးဟောင်းသုတေသနဆိုင်ရာ ထူးခြားမှုများ}

ရှေးဟောင်းမြို့တော်အချို့ရှိ ရှေးဟောင်းသုတေသန အထောက်အထားများအရ အတိတ်ကာလ သဘာဝဘေးအန္တရာယ်ဆိုးကြီးများ ဖြစ်ပေါ်ခဲ့သော ဖြစ်ရပ်များကို မှတ်တမ်းတင်ထားသည့် မြှုပ်နှံမှုနှင့် ပျက်စီးမှုများပါဝင်သည့် အလွှာအထပ်ထပ်ကို တွေ့ရှိရပါသည်။ ယနေ့ခေတ် ပါလက်စတိုင်းတွင် တည်ရှိသည့် ဂျယ်ရီခိုရှေးဟောင်းမြို့တော်မှာ ထိုသို့သောမြို့တစ်မြို့ဖြစ်သည်။ ဤမြို့တွင် ကျောက်သားအဆောက်အအုံများ ပြိုကျမှုနှင့် အကြီးအကျယ် မီးလောင်မှုတို့ပါဝင်သည့် အပျက်အစီးအလွှာများစွာ ပါဝင်ပါသည် \cite{96,97}။ ထိုအလွှာများတွင် မှတ်တမ်းတင်ထားသောကာလသည် ခရစ်တော်မပေါ်မီ နှစ်ပေါင်း ၉,၀၀၀ မှ ၂,၀၀၀ ခန့်ရှိပါသည်။ အထူးမှတ်သားဖွယ် အချက်မှာ ခရစ်တော်မပေါ်မီ နှစ်ပေါင်း ၇၄၀၀ ဝန်းကျင်၌ ထိုမြို့မှမျှော်စင်သည် ပြတ်တောက်ခဲ့ပြီး အနည်လွှာများထဲ၌ နစ်မြုပ်နေခြင်း ဖြစ်သည် (ပုံ \ref{fig:14}) \cite{95}။ ချာတယ်ဟူယွတ် \cite{99}၊ ဂရာမာလိုတီ \cite{98} နှင့် ခရိကျွန်းပေါ်ရှိ နော့ဆို၏ မီနိုအန်နန်းတော် \cite{100,101} တို့သည်လည်း အပျက်အစီး အထောက်အထားများ တစ်ခါတစ်ရံ ပါဝင်သည့် အလွှာအထပ်ထပ်ရှိသော ရှေးဟောင်းသုတေသနဆိုင်ရာ နမူနာပြစရာနေရာများ ဖြစ်ပါသည်။

\begin{figure}[t]
\begin{center}
% \fbox{\rule{0pt}{2in} \rule{0.9\linewidth}{0pt}}
   \includegraphics[width=1\linewidth]{jericho.jpg}
\end{center}
   \caption{ခရစ်တော်မပေါ်မီ နှစ်ပေါင်း ၇၄၀၀ ဝန်းကျင် နစ်မြုပ်ခဲ့သော ဂျယ်ရီခို မျှော်စင်ကို ရှေးဟောင်းသုတေသနဆိုင်ရာ ပြန်လည်တည်ဆောက်ထားမှု \cite{95}။}
\label{fig:14}
\label{fig:onecol}
\end{figure}

လူ့အဖွဲ့အစည်း ယဉ်ကျေးမှုကို သဘာဝဘေးအန္တရာယ်ဆိုးကြီးများဖြင့် ကပ်ဘေးဆိုက်ပြီး ဖျက်စီးခဲ့သည့် နောက်ထပ် အ‌ထောက်အထား တစ်ခုမှာ အိုင်ဒါဟိုပြည်နယ်ရှိ မီးတောင်ချော်များ၏ ၁၀၀ မီတာခန့်အနက်ထဲ၌ တွေ့ရှိရသည့် ရွှံ့ရုပ်တစ်ခုဖြစ်သော နန်ပါရုပ်တု ဖြစ်ပါသည် \cite{102,103}။ ထိုရုပ်တုကို တွေ့ရှိသည့် မီးတောင်ချော်များသည် တာရှရီ နှောင်းပိုင်း သို့မဟုတ် ကွာတာနရီ အစောပိုင်းကာလအတွင်း နစ်မြုပ်ခဲ့မည်ဟု ခန့်မှန်းထားရာ သက်တမ်း နှစ်ပေါင်း ၂ သန်းကျော်ရှိကြောင်း ယူဆရပါသည်။ သို့သော် ထိုဒေသရှိ မီးတောင်ချော်ရည်စီးဆင်းမှုမှာ အတော်လေး လတ်ဆတ်ကြောင်း တွေ့ရှိရသည်။ ဤတွေ့ရှိချက်များအရ လူ့အဖွဲ့အစည်း ယဉ်ကျေးမှုပေါ်တွင် သဘာဝဘေးအန္တရာယ်ဆိုးကြီးများ ဖြစ်ပေါ်ခဲ့သော အချက်များသာမက ယနေ့ခေတ် သက်တမ်းခန့်မှန်းမှုများကိုလည်း မေးစရာ ဖြစ်စေပါသည်။

\section{ခေတ်ပေါ် ရက်စွဲခန့်မှန်းနည်းများအကြောင်း}

ခေတ်ပေါ် ရက်စွဲခန့်မှန်းချက်များသည် ရုပ်ဝတ္ထုအမျိုးမျိုးကို သက်တမ်းနှစ်ပေါင်း သန်းနှင့်ချီ၍ သို့မဟုတ် သန်းပေါင်း ရာနှင့်ချီ၍ပင် အလွန်ရှည်ကြာသောကာလအထိ ခန့်မှန်းကြရာ သံသယဖြစ်စရာ အရေးကြီးသော အကြောင်းပြချက် ရှိနေပါသည်။

ပုံမှန်အယူအဆအရ ကျောက်မီးသွေး၊ ဆီနှင့် သဘာဝဓာတ်ငွေ့စသည့် "ကျောက်ဖြစ်ရုပ်ကြွင်းများ" သည် နှစ်သန်းပေါင်း ရာနှင့်ချီ၍ သက်တမ်းရှိသည်ဟု ဖော်ပြကြပါသည် \cite{104}။ သို့သော် မက္ကဆီကို ပင်လယ်ကွေ့ရှိ ဆီများကို လက်တွေ့ကာဗွန် သက်တမ်းခန့်မှန်းချက်၌ ထိုဆီကို နှစ်ပေါင်း ၁၃,၀၀၀ ခန့် ဖြစ်ကြောင်း တွေ့ရှိခဲ့ပါသည် \cite{105}။ ကာဗွန်-၁၄ ၏ သက်တမ်းထက်ဝက်ယိုယွင်းသည့် အချိန် (၅,၇၃၀ နှစ်) မှာ တိုတောင်းသည့်အတွက် နှစ်ပေါင်းသောင်းဂဏန်း အနည်းငယ်အတွင်း လုံးဝပျက်စီးသွားကြောင်း ယူဆရပါသည်။ သို့သော် ထိုထက်အဆထောင်နှင့်ချီ၍ သက်တမ်းရှိသည်ဟု ယူဆရသော ကျောက်မီးသွေးနှင့် ကျောက်ဖြစ်ရုပ်ကြွင်းများထဲတွင် ယင်းကိုတွေ့ရှိထားရပါသည် \cite{106}။ တကယ်တော့ ကျောက်မီးသွေးအတုကို ဓာတ်ခွဲခန်းထဲတွင် အဓိကအားဖြင့် မြင့်မားသောအပူချိန်ကဲ့သို့ ထိန်းချုပ်ထားသော အခြေအနေများအောက်၌ ၂ လမှ ၈ လအတွင်းမှာပင် ထုတ်လုပ်နိုင်ပါသည် \cite{107}။

ကာဗွန် သက်တမ်းခန့်မှန်းနည်းအပြင် ရေဒီယိုအိုင်ဆိုတုပ် သက်တမ်းခန့်မှန်းနည်းများသည်လည်း မှန်ကန်မှု ရှိချင်မှ ရှိပါမည်။ ကမ္ဘာဦးကျမ်း၏အဖြေများ အဖွဲ့အစည်းမှ သုတေသနအဖွဲ့သည် ထိုနည်းလမ်းများဖြင့် ရရှိလာသည့် သက်တမ်းရက်စွဲများ မကိုက်ညီမှုများ တွေ့ရှိကာ ယင်းတို့၏ တိကျမှုအပေါ် မေးခွန်းထုတ်စရာ ဖြစ်နေပါသည် \cite{108}။ ဒိုင်နိုဆော ရုပ်ကြွင်းများထဲတွင်တွေ့ရသည့် သွေးဆဲလ်များ၊ သွေးကြောများနှင့် အရေပြားဆဲလ်များပါဝင်သော ကြွက်သားများသည် သက်တမ်း နှစ်ပေါင်းသန်းတစ်ရာရှိကြောင်း ယူဆရပါသည် \cite{109,110}။ ဤကဲ့သို့လေ့လာချက်များအရ ကမ္ဘာ့၏ ဘူမိဗေဒ အချိန်အပိုင်းအခြားနှင့် ကျောက်တုံးနှင့် ကျောက်ဖြစ်ရုပ်ကြွင်း လောင်စာများကဲ့သို့ ရုပ်ဝတ္ထုများ၏ ပုံမှန်လက်ခံထားသော သက်တမ်းများသည် ပမာဏများပြားလှစွာ မှားယွင်းနေနိုင်သည့်အချက်မှာ ဖြစ်နိုင်ပါသည်။

\section{နိဂုံးချုပ်}

ဤစာတမ်းတွင် သဘာဝဘေးအန္တရာယ်ကြီးများကြောင့် ဖြစ်ပေါ်ခဲ့သည်ဟု ယူဆကာ ECDO ကမ္ဘာမြေကြီးရွေ့လျားမှုအရ အကောင်းဆုံး ရှင်းပြထားသော အထင်ရှားဆုံး ထူးခြားချက်များကို ဖော်ပြခဲ့ပါသည်။ ဤကဲ့သို့ စုစည်းဖော်ပြမှုများသည် စုံလင်သော်လည်း မပြည့်စုံသေးပါ။ နောက်ထပ် ထူးခြားချက်များကို စုစည်းလျက်ရှိပြီး ကျွန်ုပ်၏ GitHub သုတေသန စုစည်းထားသည့်စာရင်းတွင် လူတိုင်းဖတ်ရှုနိုင်ပါသည် \cite{2}။

\section{ဝန်ခံချက်}

ECDO သီအိုရီ၏ မူရင်းစာတမ်းရေးသားသူဖြစ်သော ကျင့်ဝတ် ဝေဖန်သူအား သူ၏ စိတ်ဝင်စားဖွယ် ကွဲပြားခြားနားသော စာတမ်းနှင့် ယင်းကို တစ်ကမ္ဘာလုံးနှင့် မျှဝေခဲ့ခြင်းအတွက် ကျေးဇူးတင်ပါသည်။ သူ၏ အပိုင်းသုံးပိုင်း စာတမ်း \cite{1} သည် အပူထုတ်လွှတ်သော ကမ္ဘာ့ဗဟိုလွှာ ဇန်နီဘီကော့ဗ် ခွဲဖြာရွေ့လျားခြင်း (ECDO) သီအိုရီအတွက် ခိုင်မာသောသုတေသနတစ်ခုအဖြစ် ဆက်လက်တည်ရှိနေပြီး ယခုကျွန်ုပ် အကျဉ်းချုပ် တင်ပြထားသည်ထက် ပို၍စုံလင်သော အချက်အလက်များ ပါဝင်ပါသည်။

ထို့ပြင် ယနေ့ကျွန်ုပ်တို့အတွက် ထောက်ပံ့ပေးခဲ့သော ယခုသုတေသနကို ဆောင်ရွက်နိုင်အောင် သုတေသနနှင့် လေ့လာမှုများစွာ ဆောင်ရွက်ခဲ့သလို လူသားထုအပေါ် အသိဉာဏ်ပွင့်လင်းစေရန် ကြိုးစားအားထုတ်ခဲ့သူများဖြစ်သည့် ပညာရှင်ကြီးများကိုလည်း မပျက်မကွက် ကျေးဇူးတင်ပါသည်။

\clearpage
\twocolumn

{\small
\bibliographystyle{ieee}
\bibliography{egbib}
}

\end{document}

\documentclass[10pt,twocolumn,letterpaper]{article}

% My own stuff
\usepackage{booktabs}
% \usepackage{caption}
% \captionsetup[table]{skip=8pt}   % Only affects tables
\usepackage{stfloats}  % Add this to the preamble
\usepackage{xeCJK}  % Supports Simplified & Traditional Chinese
\setCJKmainfont{IPAMincho} 
\usepackage{cvpr}
\usepackage{times}
\usepackage{epsfig}
\usepackage{graphicx}
\usepackage{amsmath}
\usepackage{amssymb}

% Include other packages here, before hyperref.

% If you comment hyperref and then uncomment it, you should delete
% egpaper.aux before re-running latex.  (Or just hit 'q' on the first latex
% run, let it finish, and you should be clear).
\usepackage[breaklinks=true,bookmarks=false]{hyperref}

\cvprfinalcopy % *** Uncomment this line for the final submission

\def\cvprPaperID{****} % *** Enter the CVPR Paper ID here
\def\httilde{\mbox{\tt\raisebox{-.5ex}{\symbol{126}}}}

% Pages are numbered in submission mode, and unnumbered in camera-ready
%\ifcvprfinal\pagestyle{empty}\fi
\setcounter{page}{1}
\begin{document}

%%%%%%%%% TITLE
\title{ECDOデータ駆動プライマーパート2/2:科学的および歴史的異常の調査、最もよく説明されたECDO「地球フリップ」}

\author{ジュンホ\\
ウェブサイト: \href{https://sovrynn.github.io}{sovrynn.github.io}\\
ECDO研究レポ: \href{https://github.com/sovrynn/ecdo}{github.com/sovrynn/ecdo}\\
{\tt\small junhobtc@proton.me}
}

\maketitle
%\thispagestyle{empty}

%%%%%%%%% ABSTRACT
\begin{abstract}
2024年5月、オンラインで匿名の著者「The Ethical Skeptic」\cite{0}が、Exothermic Core-Mantle Decoupling Dzhanibekov Oscillation (ECDO) \cite{1}と呼ばれる画期的な理論を投稿しました。この理論は、地球が以前に回転軸の急激な劇的な変化を経験し、回転慣性により海洋が大陸に溢れ出すことで大規模な世界的洪水を引き起こしただけでなく、もう一つのそのようなフリップが差し迫っている可能性を示唆するデータと共に説明的な原因となる地球物理学的プロセスも提案しています。そのような大災害の洪水と終末論的な予測は新しいものではありませんが、ECDO理論は科学的、現代的、多分野にわたるデータベースのアプローチにより、非常に魅力的です。

この研究論文は、ECDO理論に関する6か月間の独立した研究\cite{2,20}の要約の2部構成の第2部を構成し、特に劇的なECDO「地球フリップ」で最もよく説明される科学的および歴史的異常に焦点を当てています。

\end{abstract}

%%%%%%%%% BODY TEXT

\section{紹介}

現代の均一主体の地質学と歴史は、グランドキャニオンのような主要な地質学的風景が数百万年にわたり形成されたと主張しています\cite{143}。Death Valley(カリフォルニア)の塩は何億年も前に海の底にあったため存在すると主張している\cite{144}。150世代前の祖先が一生をかけて巨大な墓を作り続けたと\cite{29,70}。「化石燃料」と呼ばれるものは何億年も前のものであると\cite{104}。最も興味深いのは、人類が30万年前に存在していたと信じられている一方で、記録された歴史と文明はわずか5,000年前、つまり150世代ほど前からしか存在していないということです\cite{145}。

これらの異常は、私たちが見るように、劇的な地質学的力によって最もよく説明されます。

\section{泥に埋まった瞬間冷凍マンモス}

\begin{figure}[t]
\begin{center}
   \includegraphics[width=1\linewidth]{jarkov-mammoth.jpg}
\end{center}
   \caption{ヤーコフマンモス、20,000年前の完全に保存されたシベリアのマンモス。凍結した泥の中で発見された\cite{51}。}
\label{fig:1}
\label{fig:onecol}
\end{figure}

このような異常のカテゴリーの1つは、泥に埋もれた状態で完全に保存された凍結されたマンモスであり、通常は北極地域で発見されます(図\ref{fig:1})。シベリアで発見されたベレゾフカマンモスは、泥の礫に埋もれて発見され、その肉は死後数千年経ってもまだ食べられるほど完全に保存されていました。また、それは口と胃に植物性の食物を持っており、科学者たちはその死の直前に開花植物を食べていたとすれば、どのようにしてこれほど迅速に凍結されたのかを不思議に思っています \cite{17}。報告によれば、\textit{"1901年にベレゾフカ川近くで完全なマンモスの死骸が発見され、この動物が真夏に寒さで死んだように見えたことがセンセーションを巻き起こしました。彼の胃の内容物はよく保存されており、キンポウゲや開花中の野生の豆が含まれていました。これは、それらが7月末か8月初めに飲み込まれたことを意味しました。その生き物は突然死したため、草と花を口にくわえたままでした。それは明らかに強大な力によって捕らえられ、放牧地から数マイル運ばれたのです。骨盤と1本の脚が折れており、この巨大な動物は膝をつかんで、その後、通常はその年の最も暑い時期に凍死したのです"} \cite{18}。さらに、\textit{"[ロシアの科学者たちは]その動物の胃の最も内側の裏地でさえ完全に保存された繊維構造を持っていたと記録しており、自然界のある超驚異的な過程によって体温が取り除かれたことを示しています。サンダーソンはこの1点に特に注目し、この問題をアメリカ冷凍食品協会に持ち込んだのです:全身をマイナス温度にするためには何が必要か、体の最も内側の部分でさえ、特に胃の裏地が、肉の繊維構造を破壊するほど大きな結晶を形成する時間さえないほど迅速に凍らせるには何が必要か?... 数週間後、協会はサンダーソンに答えを戻しました:それは完全に不可能です。私たちの科学知識と工学知識をすべて駆使しても、マンモスのように大きな死体から体温を速やかに取り除いて、大きな結晶が形成されないように凍らせる方法は絶対に知られていません。さらに、科学的および工学的技術を尽くした後、自然に目を向け、これを達成できる自然の過程は知られていないと結論付けました"} \cite{19}。

\section{グランドキャニオン}

北アメリカ南西部のグレートベースの一部であるグランドキャニオンは、壊滅的な起源を示唆するもう一つの自然現象です(図\ref{fig:2})。まず、グランドキャニオンを構成する堆積砂岩と石灰岩の層は、最大2.4百万平方キロメートルに及ぶ広大なエリアに広がっています \cite{21}。図\ref{fig:3}は、米国西部のココニノ砂岩層の広がりを示しています。このような巨大で水平な均一の材料の層は、一度に一斉に堆積したものである可能性が高いです。

\begin{figure}[b]
\begin{center}
   \includegraphics[width=1\linewidth]{grand-canyon.jpg}
\end{center}
   \caption{アメリカ、アリゾナ州のグランドキャニオン \cite{49}。}
\label{fig:2}
\label{fig:onecol}
\end{figure}

\begin{figure}[t]
\begin{center}
   \includegraphics[width=1\linewidth]{coconino.jpg}
\end{center}
   \caption{米国西部におけるココニノ砂岩層の規模 \cite{21}。}
\label{fig:3}
\label{fig:onecol}
\end{figure}

グランドキャニオンを詳しく見ると、これらの広範な堆積層の堆積は、重大なテクトニックな力と同時に起こったこともわかります。これを理解するために、堆積層が折れて露出しているキャニオンの特定の領域を詳しく見てみる必要があります。Answers in Genesis の研究者は、これらの折れ目のいくつか、例えば Monument Fold からの岩石サンプルを顕微鏡で観察し、折れ目が長い時間の枠組みの中で熱と圧力の下で形成された場合に存在するはずの特徴の欠如に基づき、堆積層がまだ柔らかかった、すなわち堆積直後にテクトニックな力によって折りたたまれたと結論づけました \cite{42}

\begin{figure*}
\begin{center}
\includegraphics[width=1\textwidth]{Grand_Staircase-big.jpg}
\end{center}
   \caption{グランドキャニオンを構成する堆積層(写真の右側)は、直接北にユタ州シーダーブレイクス(写真の左側)まで広がっており、そこですべてが上に曲がっています \cite{50}。}
\label{fig:4}
\end{figure*}

視野を広げると、グランドキャニオンを構成する層がキャニオン内だけでなく、東カイバブモノクラインで東に、またユタ州シーダーブレイクスで北に折り畳まれていることがわかります(図\ref{fig:4})。これは、これらの層が一度に重ねられた後に一緒に折り畳まれた可能性を示唆しています。参考までに、グランドキャニオンの水平層の厚さはおよそ1700メートルです。堆積層を1マイルの厚さに形成するために必要な地質学的な過程の規模は巨大です。

グランドキャニオンの実際の形成は、現代地質学における別の論争の的です。均質主義地質学は、グランドキャニオンがカラ川によって何百万年もの間かかって削られたと提案しています \cite{47}。しかし、アンサーズ・イン・ジェネシスの研究チームは、古代の湖がその境界を破った際の放水路侵食によって、数週間でグランドキャニオンが形成された可能性が高いと考えています。この侵食は、キャニオンを削りながら大量の堆積物を取り除きました。グランドキャニオンの東部に湖の堆積物と海洋化石に証拠があります。Afton CanyonやMount St. Helensのような放水路侵食の他の大規模な例とグランドキャニオンを比較すると、類似の地形が現れ、大量の水流によって大きなキャニオンが急速に形成され得ることが示されています \cite{48}。

このように広大な土地に堆積物を堆積させるために必要な地質学的過程の規模、堆積層が堆積した直後に発生した大規模な地殻変動力の同時性、そしてカラ川のグランドキャニオンに比しての小ささを考え合わせると、その形成には何も緩やかなことはなかった可能性が高いです。

\section{デリンクユ地下都市}

ピラミッドの他に、古代工学の素晴らしい例としてトルコ、カッパドキアに位置するデリンクユ地下都市(図\ref{fig:5})があります。この地下都市は、この地域の200以上の地下シェルターの中で最大であると言われています \cite{54}。この地下都市は2万人の人々を収容したと推定されており、18階にわたり、深さ85メートルに達します。その年齢は不明ですが、少なくとも2800年前のものであると推定されています。都市は柔らかい火山岩から彫り出されました \cite{52, 53}。

\begin{figure}[b]
\begin{center}
   \includegraphics[width=1\linewidth]{derinkuyu.jpeg}
\end{center}
   \caption{デリンクユ地下都市の図 \cite{56}。}
\label{fig:5}
\label{fig:onecol}
\end{figure}
```
Derinkuyuが興味深い理由は、なぜあるコミュニティが地下に都市全体を建設することを決めたのか明らかではないからです。地下に生活空間を作るためには、すべての空洞を岩から彫り出さなければなりません。地下トンネルの粗い形状や質感を見ると、これらが手作業で彫られたことが明らかであり、電動工具ではなかったため、地上にシェルターを建設するよりも何桁も難しかったことでしょう。実際、農業、日光、自然、探検が地上でのみ利用可能な時に、なぜ誰もが地上生活の限界の中で地下に永住したいと思うのか理解しがたいのです。従来の「歴史」では、Derinkuyuは宗教を静かに実践する場所を必要としたキリスト教徒によって作られたと提案されています\cite{53}。しかし、常識的には、「敵に対処する最も直接的な方法は『戦うか逃げるか』であり、『岩を彫って地下都市を作ることではない』」と結論付けられるでしょう。

地下都市の規模、深さ、デザインの思慮深さを見ると、これは困難な時に侵略者と戦うための一時的な軍事防衛構造として設計されたのではなく、むしろ地表での致命的な力から守るための長期的なシェルターとして設計されたことが明らかです。Derinkuyuには基本的な寝室、台所、浴室だけでなく、動物の厩舎、水槽、食料保存庫、ワインとオイルのプレス、学校、礼拝堂、墓、巨大な換気シャフト(図\ref{fig:6})がありました。なぜ軍事シェルターがワインのプレスを必要とし、85メートルもの深さに複雑に掘られなければならなかったのでしょうか?

Derinkuyuの創造に対する最も説得力のある説明は、地表での壊滅的な地球物理的力に対して守るために長期的で自給自足可能なシェルターを準備する緊急の必要性があったということです。

\begin{figure}[t]
\begin{center}
   \includegraphics[width=1\linewidth]{derinkuyu-air.jpg}
\end{center}
   \caption{Derinkuyuにおける深い換気井戸\cite{53}。}
\label{fig:6}
\label{fig:onecol}
\end{figure}

\section{バイオマスの集積}

\begin{figure}[b]
\begin{center}
   \includegraphics[width=1\linewidth]{muck-crop.jpeg}
\end{center}
   \caption{凍ったシルトと氷の中に、木、植物、動物の断片が混沌と分散しているアラスカの「マック」\cite{146}。}
\label{fig:7}
\label{fig:onecol}
\end{figure}

動物や植物のさまざまな種類のバイオマス混合物が、しばしば堆積層に化石化されて発見されることがあるのも、もう一つの謎の異常です。「Reliquoæ Diluvianæ」では、Rev. William Bucklandが、イギリスやヨーロッパに散在する、理由もなく一緒に見つかる多数の動物種を詳述しています\cite{58}。そのような動物の遺骸の混合物は、ノルウェーのValdroy島のSkjonghelleren洞窟でも見つかりました。この洞窟では、哺乳類、鳥類、魚類の骨が7000本以上、複数の堆積層にわたって混合して見つかりました\cite{59}。別の例として、イタリアのSan Ciro、「ジャイアントの洞窟」があります。この洞窟では、およそ何トンもの哺乳類の骨(主にカバ)が見つかり、あまりにも新鮮で装飾品に加工され、ランプブラックの製造のために出荷されました。それらの動物の骨は、報告によると、混ざり合い、砕かれ、粉々にされ、断片として散乱していました\cite{60,61}。古代メンデス、エジプトでは、さまざまな動物の骨の混合物が、ガラス質(ガラス状)の粘土と混ざって見つかりました\cite{57}。このような発見は不可解に見えるかもしれませんが、大洪水によって死んだ動物が堆積層に混ざり込まれ、洞窟に投げ込まれたり、生き埋めにされたりして堆積したという説明が容易につきます。また、エジプトのガラス化したバイオマスの場合、洪水後の核マントル変位からの大規模な電気放電によるものです。図\ref{fig:7}は、アラスカのバイオマス「マック」の典型的な露出を描いています\cite{56}。


\section{古代のバンカー}
私たちの祖先は、多くの高度に設計された古代構造物を残しました。その中には人間の遺骨が発見されたものもあります。これらは通常、手の込んだ墓と解釈されますが、よく調べてみると、実際には古代の防空壕だった可能性があります。

\begin{figure}[b]
\begin{center}
   \includegraphics[width=1\linewidth]{ww19.jpg}
\end{center}
   \caption{ニューグレンジ、アイルランド - 入口にいる訪問者をスケールとして参照。}
\label{fig:8}
\label{fig:onecol}
\end{figure}

優れた例の一つがニューグレンジ(図\ref{fig:8})です。この構造はBrú na Bóinne複合体の主要モニュメントで、通路墓と呼ばれる古代構造物の集まりです。これらの墓は土や石で覆われた1つ以上の埋葬室で構成されており、大きな石で作られた狭い通路が設けられています \cite{70}。これは、一握りの人々を埋葬するために、何世代にもわたって建てられた複雑で保護された構造の広範な工学の一例であり、墓の建設が始まったときには亡くなっていなかった人々のために建てられたとされています。1699年に地元の地主によって再発見された際には、土に埋まっていました。

構造をざっと見るだけでも、建設に費やされた莫大な努力がうかがえます - ニューグレンジは約20万トンの材料で構成されています。その内部には、\textit{“...記念碑の南東側の入口からアクセスできる通路があり、通路は19メートル(60フィート)、または構造の中心までの道のりの約3分の1を伸びている。通路の終わりには、高い片持ちドーム型の屋根を持つ大きな中央室から三つの小さな部屋がある...この通路の壁は"オルソスタット"と呼ばれる大きな石板で構成されており、西側に22枚、東側に21枚ある。平均の高さは1.5メートルです" } \cite{70}。 防水のための複雑な工学的詳細もあります。例えば屋根については、\textit{“屋根の隙間は焼土と海砂の混合物でコーキングされ、防水性を持たせ、その混合物から2500 BCEを中心とする2つの放射性炭素年代が得られました" } \cite{71}。さらに、内部室への高低差が同様の目的のために実施された可能性もあります:\textit{“記念碑が建てられた丘の地面の上昇に沿って、通路と墓室の床が続いているため、入口と室内の床面の間にはほぼ2メートルの高低差がある”} \cite{71}。

\begin{figure}[b]
\begin{center}
   \includegraphics[width=1\linewidth]{dolmen.jpg}
\end{center}
   \caption{ドルメン・デ・ソト、スペイン \cite{53}。}
\label{fig:9}
\label{fig:onecol}
\end{figure}

内部に人間の遺骨がほとんどないことも興味深い点です。発掘調査では、一握りの人々を代表する焼かれた骨や焼かれていない骨片が通路に散らばっていることが明らかになりました。ニューグレンジの建設は、内部の材料の炭素年代に基づいて少なくとも数世代かかったと推定されています。なぜ古代の共同体が、数人の亡くなった人の骨片をその通路に撒くだけのために、大規模で高度に設計された墓を建設するのにそれほどの努力を費やすのでしょうか?むしろ、これらの古代で注意深く防水された巨大構造物は、むしろ地球の再発災害の際に人々を保護するために建てられた人間のシェルターだったと考える方がはるかにもっともらしいです。

スペイン南部のウエルバでは、ドルメン・デ・ソト(図\ref{fig:9})という類似の例があり、この地域には約200の同様の遺跡があります \cite{72,32} 。それは巨大石を使用して建てられた流線型の高度に設計された構造で、直径は75メートルです。報告によれば、発掘時に発見されたのはわずか8体で、すべてが胎児の姿勢で埋葬されていました。

\section{注目すべき異常の言及}

このセクションでは、ECDOのようなカタクリズムによって十分に説明可能な、さらに注目すべき異常のいくつかを簡単に言及します。

\subsection{生物学的異常}
\begin{figure}[b]
\begin{center}
   \includegraphics[width=1\linewidth]{bottleneck.jpg}
\end{center}
   \caption{約6,000年前に男性の95\%が淘汰された遺伝的ボトルネック \cite{62}。}
\label{fig:10}
\label{fig:onecol}
\end{figure}

いくつかの注目すべき生物学的異常は、遺伝的ボトルネックと内陸のクジラ化石である。Zengら(2018)は現代の人間から採取した125のY染色体配列をモデル化し、DNAの類似性と変異に基づいて、約5,000から7,000年前に男性の人口が95\%減少したボトルネックを特定した(図 \ref{fig:10})\cite{62}。クジラ化石は海抜数百メートルの高さで発見されており、スウェーデンボルグ、ミシガン、バーモント、カナダ、チリ、エジプトで見つかっている \cite{63,64,65,66}。これらのクジラは完璧に保存された状態であったり、氷河堆積物の上に横たわる泥炭地に存在したり、堆積物に埋もれていた。これらのサイトにおける標本の数は数個から百を超えるものまで様々である。クジラは深海生物であり、通常は浜辺近くに行くことはほとんどない。どのようにしてこれらのクジラがこのような高地に、しばしば内陸の極端な距離にたどり着くことができたのだろうか?

地球の過去には多数の大量絶滅が発生しており、最も徹底的に研究されているのは「ビッグファイブ」と呼ばれる顕生代の出来事である:後期オルドビス紀(LOME)、後期デボン紀(LDME)、ペルム紀末(EPME)、三畳紀末(ETME)、および白亜紀末(ECME)の大量絶滅 \cite{88,89}。不思議なことに、これらの絶滅のいくつかは、グランドキャニオンの層の多く、すなわちペルム紀とデボン紀の層と同じ歴史的期間に発生したと分類されている。

\subsection{物理的異常}

\begin{figure}[b]
\begin{center}
   \includegraphics[width=1\linewidth]{columbia.jpg}
\end{center}
   \caption{ワシントン州氷河湖コロンビアの巨大な現在のさざ波 \cite{80}。}
\label{fig:11}
\label{fig:onecol}
\end{figure}

グランドキャニオン以外にも、破滅的な力によって形成された可能性のある多くの景観がある。巨大な大陸水流の痕跡は、世界中の巨大な現在のさざ波に見られる。この例の一つが北西部のチャンネルド・スキャブランズである。ここでは、堆積的な堆積物の景観や不規則な岩石だけでなく、メガ電流流から形成された100以上の巨大なさざ波の連続も見られる \cite{78,79}。これらは川床の砂に形成されるさざ波の大規模なバージョンである。フランス、アルゼンチン、ロシア、北アメリカなど世界中で見つけることができる \cite{81}。図 \ref{fig:11} は、アメリカ合衆国ワシントン州にあるこれらのさざ波の一部を描写している \cite{80}。

\begin{figure}[b]
\begin{center}
% \fbox{\rule{0pt}{2in} \rule{0.9\linewidth}{0pt}}
   \includegraphics[width=1\linewidth]{zhangjiajie.jpg}
\end{center}
   \caption{中国南部、張家界国家森林公園の巨大な石柱。}
\label{fig:12}
\label{fig:onecol}
\end{figure}

\begin{figure}[b]
\begin{center}
   \includegraphics[width=1\linewidth]{hoy.jpg}
\end{center}
   \caption{ホイの老人海峡の海柱、スコットランド \cite{83}。}
\label{fig:13}
\label{fig:onecol}
\end{figure}

内陸浸食構造もECOD風の地球反転でよく説明されます。中国南部は水浸食によって形成された大規模なカルスト地形の素晴らしい例です \cite{82}。これらの地形には、塔カルスト、尖塔カルスト、円錐カルスト、天然橋、峡谷、大規模な洞窟システム、シンクホールが含まれます。最も注目すべきものの一つは、張家界国家森林公園で、巨大な石英砂岩の柱が含まれています(図\ref{fig:12})\cite{84}。これらの柱は平均標高1000メートル以上にそびえ、合計で3100を超えます。そのうち1000以上が120メートル以上に達し、45が300メートルを超えます\cite{85}。これらの柱は、海洋波による周囲の崩壊で形成された沿岸岩柱(図\ref{fig:13})に似ています。同様の浸食地形は、トルコのウルギュップの岩石円錐や、スペインのエンカンタダ市など、いずれも海面から1000メートル以上に位置する場所でも見られます。これらの場所すべてに、塩や海洋の化石が近接して見られ、過去の海進を示唆しています\cite{15,86,87}。もちろん、洪水の話\cite{3}は、海が1000メートルをはるかに超えて上昇したことを述べており、これはアンデスやヒマラヤの数キロメートル以上の標高における海水と巨大な塩原の存在によって確認されます。例えば、ボリビアのウユニ塩湖は海抜3653メートルに達します\cite{94}。

\subsection{急速な気候変動イベント}

現代の科学文献は、地球の最近の歴史における急速な地球規模の気候変動イベントの存在を認識しています。2つの注目すべき例は、4.2キロイヤーと8.2キロイヤーのイベントで、いずれも大規模な地理的地域での人口減少と社会的定着の混乱と一致しています。これらのイベントは、堆積物や氷床コア、化石サンゴ、O18同位体値、花粉や鍾乳洞記録、海面データとして異常として保存されています。推測される気候変動には、全球気温の急落、乾燥化、大西洋子午面循環の混乱、氷河の進行が含まれます\cite{90,91,92}。特に8.2キロイヤーのイベントは、紀元前6400年頃の黒海の潜在的な劇的な塩水洪水と同時期です\cite{93}。

\subsection{考古学的異常}

いくつかの古代都市の考古学的証拠は、埋葬と破壊を含む多層構造を示し、過去の大激変イベントの記録を作成しています。古代都市エリコはその一例で、現代のパレスチナに位置しています。そこには、石の構造物の崩壊と激しい火災とともに複数の破壊層が含まれています\cite{96,97}。その層に記録された年代は約紀元前9000年から紀元前2000年に及びます。特に注目すべきはその塔で、紀元前7400年頃に剪断され堆積物に埋没しているように見えます(図\ref{fig:14})\cite{95}。カタル・ホユック\cite{99}、グラマロテ\cite{98}、クレタ島のミノア宮殿クノッソス\cite{100,101}は、いずれも破壊の証拠を含む多層構造を持つ考古学的遺跡の類似例です。

\begin{figure}[b]
\begin{center}
   \includegraphics[width=1\linewidth]{jericho.jpg}
\end{center}
   \caption{紀元前7400年頃のエリコの塔の埋葬に関する考古学的再構築\cite{95}。}
\label{fig:14}
\label{fig:onecol}
\end{figure}

人類文明を破壊する大災害のもう一つの証拠として、ナンパ・イメージがあります。これは、アイダホ州で約100メートルの溶岩の下で発見された粘土人形です\cite{102,103}。人形が発見された溶岩流は、後期第三紀または初期第四紀に堆積したと推定され、約200万年前とされています。しかし、その地域の溶岩流は比較的新しいように見えます。このような発見は、文明を破壊する主要な大災害を示すだけでなく、現代の年代学の信憑性に疑問を投げかけます。

\section{現代の年代測定法に関して}

現代の年代表に懐疑的になる十分な理由がある。これらの年代表は、さまざまな物理的物質に対して何百万年、さらには数億年という非常に長い年代を割り当てている。

一般的な説では、石炭、石油、天然ガスといったいわゆる「化石燃料」は数億年前のものであるとされている \cite{104}。しかし、メキシコ湾の石油の実際の炭素年代測定では、その石油は約13,000年前のものであることがわかった \cite{105}。炭素14の半減期は非常に短く(5,730年)、数十万年経てば完全に崩壊するはずである。しかし、それは石炭や化石、推定ではその1,000倍の古さがあるとされている物質の中に発見されている \cite{106}。実際、人工石炭は、主に高温という制御された条件下で、わずか2〜8ヶ月で研究所で生産されている \cite{107}。

炭素年代測定以外の放射性同位体年代測定法も正確ではない場合がある。ジェネシス研究グループの解答では、これらの方法から得られる年代に矛盾が見つかり、その真実性が問われている \cite{108}。血球や血管、コラーゲンを含む軟組織が、数億年前のものであると推定されている恐竜の遺骸の中に発見されている \cite{109,110}。私たちの知るところによれば、地球の地質時代や岩石、化石燃料といった物理的物質の通常受け入れられている年代は、桁違いに間違っている可能性がある。

\section{結論}

この論文では、壊滅的起源を示唆し、ECDO地球反転で最もよく説明される最も魅力的な異常を取り上げた。提示されたコレクションは多様であるが、まだ不完全であり、さらなる異常がまとめられ、私の研究GitHubリポジトリで公開されている \cite{2}。

\section{謝辞}

ECDOの仮説を最初に形作ったEthical Skeptic氏に感謝する。彼の洞察に満ちた画期的な論文を完成させ、世界と共有してくれた。彼の三部作の論文 \cite{1} は、発熱核-マントル脱結合ジャニベコフ振動(ECDO)理論の権威ある作品であり、ここで短く要約したよりもはるかに多くの情報を含んでいる。

そしてもちろん、私たちが立っている巨人たちの肩にも感謝する。この仕事を可能にし、人類に光をもたらすためにすべての研究と調査を行ってくれた人々に。

{\small
\bibliographystyle{ieee}
\bibliography{egbib}
}

\end{document}

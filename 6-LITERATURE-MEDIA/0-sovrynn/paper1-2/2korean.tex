\documentclass[10pt,twocolumn,letterpaper]{article}

% My own stuff
\usepackage{booktabs}
% \usepackage{caption}
% \captionsetup[table]{skip=8pt}   % Only affects tables
\usepackage{stfloats}  % Add this to the preamble
\usepackage{kotex}

\usepackage{cvpr}
\usepackage{times}
\usepackage{epsfig}
\usepackage{graphicx}
\usepackage{amsmath}
\usepackage{amssymb}

% Include other packages here, before hyperref.

% If you comment hyperref and then uncomment it, you should delete
% egpaper.aux before re-running latex.  (Or just hit 'q' on the first latex
% run, let it finish, and you should be clear).
\usepackage[breaklinks=true,bookmarks=false]{hyperref}

\cvprfinalcopy % *** Uncomment this line for the final submission

\def\cvprPaperID{****} % *** Enter the CVPR Paper ID here
\def\httilde{\mbox{\tt\raisebox{-.5ex}{\symbol{126}}}}

% Pages are numbered in submission mode, and unnumbered in camera-ready
%\ifcvprfinal\pagestyle{empty}\fi
\setcounter{page}{1}
\begin{document}

%%%%%%%%% TITLE
\title{데이터 기반 ECDO 입문서 2/2: ECDO "지구 뒤집힘"으로 가장 잘 설명되는 과학적 및 역사적 이상현상에 대한 조사}

\author{준호\\
2025년 2월 출간\\
웹사이트: \href{https://sovrynn.github.io}{sovrynn.github.io}\\
ECDO 연구 저장소: \href{https://github.com/sovrynn/ecdo}{github.com/sovrynn/ecdo}\\
{\tt\small junhobtc@proton.me}
}

\maketitle
%\thispagestyle{empty}

%%%%%%%%% ABSTRACT
\begin{abstract}
2024년 5월에 "The Ethical Skeptic"이라는 가명의 온라인 저자는 발영ㄹ성 핵-맨틀 분리 드자니코브 진동(the Exothermic Core-Mantle Decoupling Dzhanibekov Oscillation, 이하 ECDO)라는 혁신적인 이론을 발표했습니다 \cite{0} \cite{1}. 이 이론은 지구가 이전에 회전 축의 갑작스럽고 파멸적인 변화를 겪었다고 주장할 뿐만 아니라 이는 지구의 회전 관성으로 인해 대양이 대륙 위로 넘치게 하면서 전 세계적으로 거대한  홍수를 일으켰으며, 또 다른 그러한 변화가 임박했을 수도 있다는 것을 제시하는 자료와 함께 그 원인이 되는 일련의 지구물리학적 과정을 설명을 담고 있다. 이러한 재앙적 홍수 및 종말 예측들이 새로운 것은 아니지만, ECDO 이론은 과학적,현대적이며 여러 전문 분야에 걸친 데이터 기반의 접근 방식으로 인해 독특한 설득력이 있다. 
이 연구 논문은 ECDO 이론에 대한 6개월 간의 독립적 연구 \cite{2,20}의 두 번째 요약 부분이며, 특히 재앙적 ECDO "지구 뒤집기"로 가장 잘 설명되는 과학적 그리고 역사적 이상현상들에 그 초점을 맞추고 있다.

\end{abstract}

%%%%%%%%% BODY TEXT

\section{서론}

현대의 동일과정설에 기반한 지질학과 역사는 그랜드 캐년과 같은 주요 지질학적 풍경이 수백만 년에 걸쳐 형성되었고\cite{143}; 데스밸리(캘리포니아)에 소금이 존재하는 이유는 수억 년 전 그것이 대양 아래에 있었기 때문이며, \cite{144}; 150세대 전의 우리의 조상들이 그들의 일생을 거대한 무덤을 짓는 데 보냈다고 주장한다 \cite{29,70}. 또한 그들은 소위 "화석 연료"가 수억 년 전 만들어진 것이라고 이야기한다. \cite{104}. 아마도 가장 흥미로운 것은 인류가 출현한지 30만년이 되었다고 믿어지지만\cite{145}, 정작 기록된 역사와 문명은 약 5,000년, 즉 150세대 정도 전으로만 거슬러 올라간다는 것이다.

우리가 다음에서 보듯이, 이러한 이상현상은  대홍수적 지질학적 힘에 의해 가장 잘 설명된다.

\section{진흙에 묻힌 순간냉동된  매머드}

\begin{figure}[t]
\begin{center}
   \includegraphics[width=1\linewidth]{jarkov-mammoth.jpg}
\end{center}
   \caption{얼어붙은 진흙에서 발견된 20,000년 전의 완벽하게 보존된 시베리아 매머드 \cite{51}.}
\label{fig:1}
\label{fig:onecol}
\end{figure}

이러한 이상현상들 중 하나는 주로 북극 지역에서 발견되는 진흙에 묻혀 완벽하게 보존되어 있는 순간냉동된 매머드들이다. (그림 \ref{fig:1}). 실트질 자갈에 묻힌 채 시베리아에서 발견된 베레조프카 매머드는 죽은 지 수천 년이 지난 후에도 고기가 여전히 먹을 수 있을 정도로 완벽하게 보존되었다. 또한   메머드의 입과 위에는 식물성 먹이가 있었으며,이는 매머드가 죽기 직전에 꽃이 피어 있는 식물들을 뜯어먹고 있었다면, 어떻게 그렇게 빠르게 냉동될 수 있었는지에 대해 과학자들을 당황하게 만들었다. \cite{17}. 보고에 따르면, \textit{"1901년 베레조프카강 근처에서 완벽한 매머드의 사체가 발견되면서 큰 소동을 일으켰다. 이 동물은 한여름에 추위로  죽은 것처럼 보였기 때문이다. 매머드의 위 속의 내용물은 잘 보존되어 있었는데, 미나리아재비와 꽃이 핀 야생 콩들을 포함하고 있었다. 이는 풀들이 약 7월 말이나 8월 초쯤에 삼켜졌어야 했음을 의미한다. 그 생물체는 너무나 갑작스럽게 죽어, 그 입에는 여전히 풀과 꽃을 가득 담고 있었다. 메머드는 분명 엄청난 힘에 사로잡혀 방목지로부터 수 마일이나 떨어진 곳에 던져졌다. 골반과 한쪽 다리는 부러졌으며, 거대한 동물은 무릎을 꿇게 되었고, 일년 중  가장 더운 시기에 얼어 죽게 되었다"} \cite{18}. 또한, \textit{"[러시아 과학자들은] 심지어 동물의 위 내부의 가장 안쪽 층이 완벽하게 보존된 섬유 구조를 가지고 있었다고 기록했다. 이는 그 동물의 체온이 사실상  어떤 아주 비정상적인  과정에 의해 제거되었음을 나타낸다. 샌더슨은 이 점에 주목하면서 문제를 미냉동식품연구소(the American Frozen Food Institue, AFFI)로 가져갔다.: 매머드와 같이 큰 동물이 몸의 가장 안쪽의 습기,  심지어는 위장 내벽의 습기 조차도  고기의 섬유질 구조를 파괴하기에 충분한 크기의 크리스탈을 형성할 시간 조차 없이, 매머드 전체를 얼어붙게 할 수 있는 것은 과연 무엇인가?... 몇 주 후 연구소는 샌더슨에게 이렇게 답해왔다.: 그것은 절대 불가능합니다. 우리의 모든 과학적 및 공학적 기술을 다 동원했지만, 매머드만큼 큰 사체에서 수분 결정이 육의 섬유 구조를 파괴하지 않게 냉동하는 방법은 전혀 알려져 있지 않습니다. 게다가, 과학적 및 엔지니어링 기술을 소진한 후, 그들은 자연에서도 그 업적을 이룰 수 있는 기로 알려지지 않음을 결론지었습니다"} \cite{19}.

\section{그랜드 캐년}

북미 남서부의 그레이트 분지의 일부인 그랜드 캐년은 또 다른 대재앙적 기원을 제시하는 또 하나의 자연 현상이다.(그림 \ref{fig:2}). 우선, 그랜드 캐년을 구성하는 퇴적 사암 및 석회암 층은 최대 240만 평방킬로미터에 걸쳐
치는 광대한 지역에 펼쳐져 있 다 \cite{21}. 그림 \ref{fig:3}은 코코니노 사암 층이 미국 서부 전역에 걸쳐 있는 광활한 모습을 보여준다. 이와 같이 균일한 물질의 거대한 수평층은 오직 한꺼번에  모두 놓였을 때만 형성될 수 있었을 것이다.

\begin{figure}[b]
\begin{center}
   \includegraphics[width=1\linewidth]{grand-canyon.jpg}
\end{center}
   \caption{미국 애리조나 주의 그랜드 캐년 \cite{49}.}
\label{fig:2}
\label{fig:onecol}
\end{figure}

\begin{figure}[t]
\begin{center}
   \includegraphics[width=1\linewidth]{coconino.jpg}
\end{center}
   \caption{미국 서부에 있는 코코니노 사암 층의 크기 \cite{21}.}
\label{fig:3}
\label{fig:onecol}
\end{figure}

그랜드 캐년을 자세히 보면, 우리는 이 광대한 퇴적층들의 축적 또한  상당한 지각 변동의 움직임과 동시에 발생했음도 알 수 있다. 이것을 이해하려면, 우리는 퇴적층이 접히고 노출된 캐년의 특정 지역을 자세히 살펴보야야 한다.  Answers in Genesis의 연구원들은 \cite{42} 모뉴먼트 폴드(the Monument Fold)와 같은 몇몇 접힘지역에서 채취한 암석 샘플을 현미경으로 조사하였으며, 접힘이 열과 압력 하에서 오랜 세월에 걸쳐 형성되었다면 존재했어야 할 특징이 결핍된 것을 기반으로 하여, 퇴적층이 아직 부드럽고, 즉 퇴적 후 얼마 지나지 않아 일어난 지각 변동 운동에 의해 접혔다고 결론 내렸다 \cite{43}.

\begin{figure*}
\begin{center}
\includegraphics[width=1\textwidth]{Grand_Staircase-big.jpg}
\end{center}
   \caption{ 유타주의 세다 브레이크(그림의 왼쪽)까지 북쪽으로 확장되며, 그곳에서 모두 위로 휘어진 그랜드 캐년의 퇴적층(그림 오른쪽) \cite{50}.}
\label{fig:4}
\end{figure*}

거시적으로 보면, 그랜드 캐니언을 형성하는 지층이 단순히 캐니언 내부에서만 접힌 것이 아님을 알 수 있다. 지층은 동카이밥 단층(the East Kaibab Monocline)에서 동쪽으로 접혀 있고 \cite{46}, 또한 유타의 시더 브레이크(Cedar Breaks) 북쪽으로도 접혀 있다. (그림 \ref{fig:4}). 이는 이러한 지층이 급속도로 연속적으로 쌓인 후 함께 접혔을 가능성을 시사한다. 참고로, 그랜드 캐니언의 수평 지층은 두께가 약 1700미터에 이른다. 1마일 두께의 퇴적층을 형성시키기 위한 지각운동의 규모는 엄청나다.

그랜드 캐니언의 실제 형성과정은 현대 지질학에서 또 다른 논쟁거리이다. 동일과정 지질학은 그랜드 캐니언이 수백만 년 동안 콜로라도 강에 의해 깎였다고 주장한다 \cite{47}. 그러나 Answer's in Genesis 연구팀은 그랜드 캐니언이 고대 호수의 경계가 파괴되면서 넘쳐는 흐르던 물의 침식으로 인해 계곡을 깎아내면서 대량의 퇴적물을 제거하면서 수 주일 내에 형성되었을 가능성이 높다고 믿는다. 그랜드 캐니언 동쪽의 고지대 호수의 퇴적물과 해양 화석에서 증거를 찾을 수 있다. 애프턴 캐니언과 세인트 헬렌스 산과 같은 다른 대규모 방류수의 흐름에 의한 침식 사례와 그랜드 캐니언을 비교하면 비슷한 지형이 드러나며, 대규모 유수로 인해 큰 계곡이 빠르게 형성될 수 있음을 보여준다 \cite{48}.

이처럼 광대한 감싸는 듯한 띠 모양의 육지 위에 퇴적물을 쌓기 위한 지질학적 과정의 규모를 생각하면, 퇴적층이 쌓인 직후 발생한 대규모 지각변동 운동의 동시 발생과 거대한 그랜드 캐니언에 비해 아주 작은 콜로라도 강의 크기를 감안할 때, 그 형성 과정에 점진적인 과정은 전혀 없었다라고 볼 수 있다.
\section{데린쿠유 지하도시}

\begin{figure}[b]
\begin{center}
% \fbox{\rule{0pt}{2in} \rule{0.9\linewidth}{0pt}}
   \includegraphics[width=1\linewidth]{derinkuyu.jpeg}
\end{center}
   \caption{데린쿠유 지하 도시의 도면 \cite{56}.}
\label{fig:5}
\label{fig:onecol}
\end{figure}

\begin{figure}[t]
\begin{center}
   \includegraphics[width=1\linewidth]{derinkuyu-air.jpg}
\end{center}
   \caption{데린쿠유의 깊은 환기구 \cite{53}.}
\label{fig:6}
\label{fig:onecol}
\end{figure}


피라미드들 외 고대 공학의 훌륭한 예로는 터키 카파도키아의 데린쿠유 지하도시가 있다. (그림 \ref{fig:5}). 이곳은 그 지역에 있는 200개 이상의 지하 대피 시설들 중 가장 큰 곳이다 \cite{54}. 이 지하도시는 최대 20,000명의 사람들이 거주했을 것으로 추정되며, 18층으로 구성되어 있으며, 깊이는 85미터에 이릅니다. 그 연령은 확실치 않지만, 최소 2800년 이상 되었을 것으로 추정된다. 이 도시는 부드러운 화산암을 깎아서 만들어졌다 \cite{52, 53}.

데린쿠유가 흥미로운 이유는 왜 어느 공동체가 도시 전체를 지하에 건설하려 했는지 명확하지 않기 때문이다. 지하에 생활공간을 만들기 위해서는 모든 공간이 암석을 깍아서 만들어져야만 한다. 지하 터널들의 거친 형태들과 결들은 이런 것이 전동 공구 대신 수작업으로 깎였다는 것을 분명하게 보여주고 있으며, 이는 지상에 대피소를 세우는 것보다 수백, 수천 배`더 어려운 일이었을 것이다. 사실, 농업, 햇빛, 자연 그리고 탐험은 오직 지상에서만  가능한데 왜 인간이 한정된 이 세상에서의 생애 동안, 지하에서 평생을 살고 싶어햇을까 하는 의문이 든다. 전통적인 "사학"은 데린쿠유가 신앙생활을 할 격리된 곳이 필요했던 기독교인들에 의해 만들어졌다고 그 이유를 제안한다. \cite{53}. 그러나 상식적으로 생각해보면 적과 맞서는 가장 직접적인 방법은  싸우거나 도망가는 것이지, 바위를 깎아 지하 도시를 만드는 것이 아니라고 결론지을 수 있다.

지하 도시의 규모, 깊이, 그리고 설계의 세심함은 이곳이 압박을 받는 시기에
 침략자와 더 잘 싸우기 위한 일시적 군사적 방어시설로 설계된 것이 아니라, 지상에서의 치명적인 힘에 대비하기 위한 장기적인 대피소로 설계되었음이 분명하다.. 데린쿠유에는 기본적인 침실, 주방, 욕실뿐만 아니라 동물을 위한 외양간, 물탱크, 식량 저장고, 와인 및 오일 압착기, 학교, 예배당, 무덤, 그리고 거대한 환기 통로가 마련되어 있다 (그림 \ref{fig:6}). 군사 대피소에 와인 압착기가 필요한 이유는 무엇이며, 왜 그렇게 복잡한 구조로 85미터 깊이까지 파야한단 말인가?

데린쿠유를 만든 가장 그럴듯한 설명은 지표면의 재앙적인 지구 물리학적 힘으로부터 보호하기 위한 장기적으로 지속가능하고 자급자족할 수 있는 대피소를 준비할 긴급하고도 절박한 필요가 있었을 것이라는 점이다.

\section{바이오매스 퇴적물}
 
종종 퇴적층에서 화석화되어 발견되는 다양한 동물과 식물의 바이오매스 혼합물은 또 다른 수수께끼의 이상현상이다. "Reliquoæ Diluvianæ"에서, 윌리엄 버클랜드 목사는 함께 발견된 이유를 설명하기 어려운 수많은 종들의 동물군의 발견에 대해 자세히 다룬다. 그 동물군들은 홍적층(홍수로 인한 퇴적층)에 묻혀져 영국과 유럽 전역에 흩어져 있다. \cite{58}. 이와 같은 동물 유해의 혼합물은 노르웨이 발드로이 섬의 스쾅힐렌 동굴에서도 발견되었다. 이 동굴에서는 7,000개 이상의 포유류, 조류, 어류의 뼈가 여러 퇴적층에서 혼합된 상태로 발견되었다 \cite{59}. 또 다른 예로는 이탈리아에 있는 산 치로,즉 "거인의 동굴"이 있다. 이 동굴에서는 대부분이 하마인 수 톤의 포유류 뼈가 너무 신선한 상태로 발견되어 장신구로 잘려져 검정 램프 제조를 위해 보내졌다. 서로 다른 동물의 뼈가 함께 섞이고, 부서지고, 산산조각 나서 흩어져 있었다고 전해진다 \cite{60,61}. 이집트의 고대 멘데스에서는 자기화된(유리같은) 점토와 혼합된 여러 종들의 동물 뼈가 발견되었다 \cite{57}. 이러한 발견은 수수께끼로 보일 수 있지만, 이는 죽은 동물들의 사체 혼합물을 퇴적층에 버리고, 동물을  동굴에 내려 놓거나 산채로 묻는 거대한 홍수로 쉽게 설명될 수 있다. 이집트의 자기화된 바이오메스의 경우는,  대홍수 후핵과 맨틀의 분리로부터 발생한 대규모 전기 방전으로 쉽게 설명될 수 있다. 그림 \ref{fig:7}은 전형적인 알래스카의 바이오매스 '머크'의 노출을 보여준다 \cite{56}.

\begin{figure}[b]
\begin{center}
   \includegraphics[width=1\linewidth]{muck-crop.jpeg}
\end{center}
   \caption{ 냉동 실트와 얼음 속에 나무, 식물 및 동물 조각이 어지럽게 흩어져 있는 알래스카의 '머크' \cite{146}.}
\label{fig:7}
\label{fig:onecol}
\end{figure}

\section{고대 벙커}

우리의 조상은 인간 유해가 발견된 매우 공학적으로 만들어진 고대 건축물들을 남겼다. 이들은 대개 정교한 무덤으로 설명되지만, 자세히 살펴보면 사실 고대 벙커였을 가능성을 보여준다.

\begin{figure}[b]
\begin{center}
   \includegraphics[width=1\linewidth]{ww19.jpg}
\end{center}
   \caption{아일랜드의 뉴그레인지 - 입구에서 보는 방문자들의 크기를 보세요.}
\label{fig:8}
\label{fig:onecol}
\end{figure}

그 훌륭한 예 중 하나는 소위 통로 무덤이라 불리는 무덤을 포함한 고대 건축물의 단지이다. 브루 나 보인 단지의 주요 기념물인 뉴그레인지이다 (그림 \ref{fig:8}). . 이 무덤들은 흙이나 돌로 덮힌 하나 또는 그 이상의 매장실과  큰 돌로 만든 좁은 출입 통로가 있다.\cite{70}. 이는 여러 세대에 걸쳐 지어진 복잡한 보호 구조물의 광대한 토목 공사의 예이며, 무덤의 건설이 시작될 때 이미 죽어있던 소수의 사람들을 매장하기 위한 것이라고 여겨진다. 1699년 지역의 토지 소유주에 의해 재발견되었을 때, 뉴그레인지는 땅에 묻혀 있었다.

간단히 구조를 살펴보면 건설에 엄청난 노력이 기울여졌음을 알 수 있다. 뉴그레인지(Newgrange)는 약 200,000톤의 재료로 구성되어 있다. 그 내부에는 \textit{“...기념물의 남동쪽에 있는 입구를 통해 접근할 수 있는 방이 있는 통로가 있다. 통로는 약 19미터 (60피트)가량 또는 구조물의 중앙까지 이르는 길의 약 3분의 1가량 이어져 있고, 통로의 끝에는 높게 받쳐진 둥근 아치형 천장으로 된 큰 중앙 방을 벗어나면 세 개의 작은 방들이 있다... 이 통로의 벽은 정교한 건축 기법에 의해 만들어졌으며 방수 기능을 갖추고 있다... 벽은 정면벽이라 불리는 매우 큰 돌판을 맞추어 만들어졌다. 서쪽에 스물두 개, 동쪽에 스물한 개의 돌판이 있으며 평균 높이는 1½미터이다”} \cite{70}. 또한 정교한 방수 공학적 세부사항도 있다. 예를 들어, 지붕에는 \textit{“지붕의 간극은 방수 처리를 위해 소각된 흙과 바다모래의 혼합물로 메워졌으며, 이 혼합물의 방사성탄소 연대 측정 결과, 무덤 건축물은 기원전 2500년 경의 것으로 밝혀졌다. ”} \cite{71}. 또한, 유사한 목적을 위해 내실로 가는 상승 구간이 만들어졌을 수 있다: \textit{“기념물이 세워진 언덕의 경사면에 따라 무덤의 통로와 방의 바닥이 연결되므로, 입구와 방 내부의 바닥 사이에는 거의 2미터의 높이 차이가 있다”} \cite{71}.

\begin{figure}[t]
\begin{center}
   \includegraphics[width=1\linewidth]{dolmen.jpg}
\end{center}
   \caption{스페인 돌멘 드 소토(Dolmen de Soto) \cite{53}.}`
\label{fig:9}
\label{fig:onecol}
\end{figure}

무덤 건축물 내부에 인간의 유해가 남아있지 않다는 것 또한 이상한 점이다. 발굴에서 단지 소수의 사람들 것으로 보이는 소각된 내지는 소각되지 않은 뼛조각들이 통로에 흩어져 있는 것이 드러났다. 무덤 내부 물질의 탄소연대 측정 자료에 미루어 뉴그레인지를 건설하는 데는 최소 몇 세대가 걸렸을 것으로 추정된다. 왜 고대 공동체가 단지 그 통로에 몇몇 사체의 뼈 조각들을 흐트려 놓기위해  그토록 많은 노력을 기울여 거대한고도 공학적으로 정교하게 설계된 무덤을 건설하겠는가? 이렇게 정교하게 방수 처리된 고대의 거석 구조물은 인류가 지구의 반복적인 대홍수 동안 자신을 보호하기 위해 지어진 쉼터였다는 것이 오히려 훨씬 더 그럴듯하다.

남부 스페인 후엘바(Huelva)에는 유사한 예로써 약 200여 개의 유적지가 있는데, 그 중 하나가 돌멘 드 소토(Dolmen de Soto)(그림 \ref{fig:9})이다. \cite{72,32}. 이 구조물은 거석을 사용하여 유선형으로 지어지고 매우 공학적으로 설계된 구조물로, 그 직경이 75미터이다. 발굴당시 태아 자세로 묻힌 8구의 시체만이 발견되었다고 전해진다.

\section{주목할 만한 이상 현상의 언급}

여기서는 ECDO와 같은 대재앙으로서  잘 설명되는 몇 가지 주요 이상 현상들을 추가로 간략히 언급하겠다.

\subsection{생물학적 이상 현상}

\begin{figure}[t]
\begin{center}
   \includegraphics[width=1\linewidth]{bottleneck.jpg}
\end{center}
   \caption{약 6,000년 전 남성의 95\% 급감 현상을 보여주는 유전적 병목현상 \cite{62}.}
\label{fig:10}
\label{fig:onecol}
\end{figure}

주목할 만한 생물학적 이상 현상으로 유전적 병목현상들과 내륙 고래 화석들이 있다. Zeng et al.(2018)은 현대 인류의 125개의 Y 염색체 서열을 모델링하였고, DNA의 유사성과 돌연변이를 기반으로 약 5,000년에서 7,000년 전  남성 인구의 95\% 감소 병목현상을 확인하였다 (그림 \ref{fig:10}) \cite{62}. 고래 화석은 스웨덴버그, 미시간, 버몬트, 캐나다, 칠레, 이집트에서 해발 발견되었다 \cite{63,64,65,66}. 이 고래들은 완벽하게 보존되거나  빙하 퇴적물 위의 습지대 혹은  퇴적물에 묻혀 있는 등의 다양한 형태로 발견되었다. 이 유적지들의 표본 수는 몇몇 개로부터 100개 이상에 이른다. 고래는 심해 생물로, 해안 근처로 잘 오지 않는다. 이런 고래들이 어떻게 그처럼 높은 고도에, 그리고 많은 경우에 극도로 육지 깊숙이 도달하였을까?

과거 지구에는 수 없이 많은 대량 멸종이 발생했으며, 그 중 가장 철저히 연구된 것은 "빅 파이브"라는 현생대의 대멸종 사건들이다: 후기 오르도비스기(LOME), 후기 데본기(LDME), 페름기말기(EPME), 트라이아스기말기(ETME), 그리고 백악기말기(ECME) 대멸종 \cite{88,89}. 놀랍게도, 이러한 멸종 사건 중 몇몇은 그랜드 캐니언의 많은 지층들, 즉 페름기와 데본기의  지층들과 같은 역사적 시기에 발생했다고 분류된다.

\subsection{물리적 이상 현상들}

\begin{figure}[b]
\begin{center}
   \includegraphics[width=1\linewidth]{columbia.jpg}
\end{center}
   \caption{워싱턴 주 글래이셜 레이크 컬럼비아의 거대한 현재 물살 계단 \cite{80}.}
\label{fig:11}
\label{fig:onecol}
\end{figure}

그랜드 캐년 외에도 대재앙으로 인한 가공할 만한 힘에 의해 형성되을 듯한 많은 지형들이 있다. 대규모의 대륙 간 물 흐름의 증거는 전 세계에 있는 해류의 파도모양의 무늬에서 찾아볼 수 있다. 그러한 예 중 하나가 태평양 북서부의 채널드 스캡랜드(Channeled Scablands) 이다. 여기에는 퇴적층 지형과 괴상한  큰 바위들 뿐만 아니라  거대한 해류에 의해 물결무늬가 100개 이상 이어져 있는 것을 볼 수 있다. 이들은 하천의 모래바닥에 형성된 물격무늬의 대형 버젼으로, \cite{78,79}.  프랑스, 아르헨티나, 러시아, 북아메리카 등 전 세계에서 찾아 수 있다 \cite{81}. 그림 \ref{fig:11}은 미국 워싱턴 주에 있는 이러한 모래 언덕 중 일부를 보여준다\cite{80}.

\begin{figure}[t]
\begin{center}
% \fbox{\rule{0pt}{2in} \rule{0.9\linewidth}{0pt}}
   \includegraphics[width=1\linewidth]{zhangjiajie.jpg}
\end{center}
   \caption{중국 남부의  장가계 국유림 속 거대한 돌기둥들.}
\label{fig:12}
\label{fig:onecol}
\end{figure}

\begin{figure}[t]
\begin{center}
   \includegraphics[width=1\linewidth]{hoy.jpg}
\end{center}
   \caption{스코틀랜드의 올드 맨 오브 호이(Old Man of Hoy) 바다 기둥 \cite{83}.}
\label{fig:13}
\label{fig:onecol}
\end{figure}

내륙 침식 구조 또한 ECDO와 같은 지구 뒤집기에 의해 잘 설명될 수 있다. 남중국은 물의 침식작용에 의해 형성된 거대한 카르스트 지형의 좋은 예이다 \cite{82}. 이러한 지형에는 타워 카르스트, 첨탑 카르스트, 원뿔 카르스트, 천연, 협곡, 대규모 동굴 시스템 및 싱크홀 등이 포함된다. 그중에서도 가장 눈에 띄는 것 중 하나는 장가계 국유림으로, 이곳에는 거대한 석영 사암 기들둥이 있다 (그림 \ref{fig:12}) \cite{84}. 이 기둥들은 평균 고도는 해발 1,000미터 이상이며, 그 수는 3,100개가 넘는다. 그 중 1,000개 이상이 120미터 이상 높이 솟아 있고, 45개는 300미터 이상에 달한다 \cite{85}. 이러한 기둥들은 이는 해양의 파도로 인해 주변 물질이 붕괴되어 형성된 바닷가의 바위 기둥인 해안 침식 기둥(그림 \ref{fig:13})을 닮았다. 발 1,000미터를 넘는 유사한 침식 지형들이 터키의 위르괴프 바위뿔이나 스페인의 시우다드 엔칸타다 같은 곳에서도 찾을 수 있다. 이 모든 장소들은 인근에 소금 및 해양 화석을 포함한 조합을 가지고 있으며 이는 과거 바닷물의 유입을 시사한다 \cite{15,86,87}. 물론 홍수 이야기는 \cite{3} 해양이 1,000미터 이상까지 올랐다는 것을 언급했으며, 이는 해수면에서 수 킬로미터 위에 있는 안데스와 히말라야에 있는 염수와 거대한 소금 평원들의 존재로서 확인된다. 예를 들어 볼리비아의 우유니 소금 평야는 해발 3653미터에 이른다 \cite{94}.

\subsection{급격한 기후 변화 사건}

현대 과학 문헌은 지구의 최근 역사에서 빠른 글로벌 기후 변화의 존재를 인정하고 있다. 두 가지 주목할 만한 예는 4200년 전 사건과 8200년 전 사건으로, 둘 다 인구 감소와 광범위한지역에서의 사회적 정착지의 붕괴와 일치한다. 이러한 사건들은 퇴적물과 빙하 중심부, 화석화된 산호,ㅠO18 동위원소 값, 꽃가루 및 동굴생성물 기록, 해수면 데이터 에서의 이상 현상으로 보존되어 있다. 앞서 언급된 기후 변화에는 전 세계 기온의 급격한 하락, 건조화, 대서양 남쪽 해류 뒤집힘의 교란현상, 그리고 빙하의 발달이 포함된다 \cite{90,91,92}. 특히 8200년 전 사건은 기원전 6400년경 흑해의 잠재적인 극적 염수 홍수와 동시에 발생했다 \cite{93}.

\subsection{고고학적 이상 현상}

\begin{figure}[t]
\begin{center}
   \includegraphics[width=1\linewidth]{jericho.jpg}
\end{center}
   \caption{기원전 7400년경 여리고(Jericho) 탑의 매몰을 고고학적으로 재구성한 것 \cite{95}.}
\label{fig:14}
\label{fig:onecol}
\end{figure}

몇몇 고대 도시의 고고학적 증거는 과거 대재앙적 사건들의 기록들을 만드는 매몰 및 파괴를 포함하는 여러 층을 보여준다. 오늘날의  팔레스타인에 위치한 고대 도시 여리고가 그런 도시 중의 하나이다. 그것은 돌 구조물의 붕괴와 강렬한 화재로 인한 다수의 파괴층들을 갖고 있다 \cite{96,97}. 그 층들에 기록된 연대는 약 기원전 9000년 경부터 기원천 2000년경으로 추정된다. 특히 주목할만한 것은 기원전 7400년경 잘려나가 퇴적층에 묻힌 것으로 보이는 탑이다 (Figure \ref{fig:14}) \cite{95}. 차탈 휘크(Chatal Huyuk) \cite{99}, 그라마로테(Gramalote) \cite{98}, 그리고 크레타섬에 있는 미노스문영의 크노소스 궁전등 \cite{100,101}은 종종 파괴의 증거들을 담고 있는 다수의 층들을 갖고 있는 유사한 고고학적 유적지들의 예이다. 

인류문명의 대재앙적 파괴의 또 다른 증거는 아이다호의 용암 약 100미터 아래에서 발견된 점토 인형인 남파 이미지(the Nampa Image)이다\cite{102,103}. 그 아래에서 작은 조각상이 발견된 용암류는 제3기의 말기나 혹은 제4기의 초기 것으로 추정되며, 약 200만 년 가량 된 것으로 여겨진다. 그러나 그 지역의 용암류는 비교적 최근에 만들어진 것으로 보인다. 이러한 발견은 문명을 파괴하는 주요 대격변을 가리킬 뿐만 아니라 현대의 연대기에 의문을 일으킨다.

\section{현대 연대 측정법에 관하여}

다양한 물질에  수백만 년, 혹은 심지어 수억 년 등과 같은 극도로 긴 연대를  부여하는 현대 연대학에 회의적일 만한 상당한 이유가 있다.전통적인 서사는 석탄, 석유, 그리고 천연가스와 같은 소위 "화석 연료"가 수억 년 된 것이라고 주장한다 \cite{104}. 그러나 멕시코 만의 석유에 대한 실제 탄소 연대 측정 결과, 그 석유의 연대가 약 13,000년으로 판명되었다 \cite{105}. 탄소-14의 반감기는 너무 짧아서 (5,730년) 몇십만 년 후에는 완전히 붕괴되어야 한다. 그러나 이는 수천 배나 더 오래된 것으로 여겨지는석탄과 화석에서 발견되었다 \cite{106}. 실제로 인공 석탄은 실험실의 통제된 조건, 즉 주로 높은 열에서 단 2-8개월만에 생산되었다 \cite{107}.

탄소 연대 측정법 외에  방사성 동위원소 연대 측정법도 정확하지 않을 수 있다. Answers in Genesis 연구팀은 그러한 방법으로부터 얻은 날짜에 일관성이 없음을 발견하고 그 정확성에 의문을 가졌다\cite{108}. 심지어 수천만 년 전 것으로 여겨지는 공룡 유해에서 혈액 세포, 혈관, 콜라겐이 포함된 연조직이 발견되었다 \cite{109,110}. 우리가 아는 바에 따르면, 지구의 지질학적 시간의 척도와 암석 및 화석 연료와 같은 물질들에 대해 일반적으로 인정되는 연대가 몇 자릿 수나 잘못된 것일 수 있다.

\section{결론}

이 논문에서, 나는 대재해의 기원을 제시하고 ECDO 지구 뒤집힘에 의해 최고로 잘 설명되는 가장 강력한 비정상적 특이현상을 다루었다. 다양하지만, 수집되어 제시된 일련의 자료들은 다양하지만 불불완전하다.  더 많은 이상 현상들이 수집되어 있으며 내 연구 GitHub 저장소에서 공개적으로 제공된다 \cite{2}.

\section{감사의 말씀}

 ECDO 논문의 원저자인 Ethical Skeptic에게 통찰력있고 획기적인 이론을 완성한 것에 대해 감사드린다. 그의 3부작 논문 \cite{1}은 발열성 핵-맨틀 분리 자니베코프 진동 (ECDO) 이론의 권위 있는 작품으로 남아 있으며, 이 짧은 요약본보다 훨씬 더 많은 정보를 담고 있다.

그리고 이 작업을 가능하게 만든 모든 연구와 조사를 수행하고 인류에 빛을 가져오려고 노력한 위대한 이들에게 감사드린다.

{\small
\renewcommand{\refname}{참고문헌}
\bibliographystyle{ieee}
\bibliography{egbib}
}

\end{document}

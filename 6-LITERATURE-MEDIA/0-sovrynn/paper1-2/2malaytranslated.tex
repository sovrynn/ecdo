\documentclass[10pt,twocolumn,letterpaper]{article}

% Barang-barang sendiri
\usepackage{booktabs}
% \usepackage{caption}
% \captionsetup[table]{skip=8pt}   % Hanya memberi kesan kepada jadual
\usepackage{stfloats}  % Tambah ini ke preamble

\usepackage{cvpr}
\usepackage{times}
\usepackage{epsfig}
\usepackage{graphicx}
\usepackage{amsmath}
\usepackage{amssymb}

% Sertakan pakej-pakej lain di sini, sebelum hyperref.

% Jika anda komen hyperref dan kemudian nyahkomen ia, anda harus padamkan
% egpaper.aux sebelum jalankan latex semula.  (Atau hanya tekan 'q' pada jalankan latex pertama,
% biar ia habis, dan anda patut jelas).

\usepackage[breaklinks=true,bookmarks=false]{hyperref}

\cvprfinalcopy % *** Nyahkomen baris ini untuk penyerahan akhir

\def\cvprPaperID{****} % *** Masukkan ID Kertas CVPR di sini
\def\httilde{\mbox{\tt\raisebox{-.5ex}{\symbol{126}}}}

\makeatletter
\def\abstract{%
  \centerline{\large\bf Abstrak}% <-- your new label
  \vspace*{12pt}%
  \it%
}
\makeatother

\renewcommand{\tablename}{Jadual}  % now \caption will say “Tabla 1” instead of “Table 1”
\renewcommand{\figurename}{Rajah}   % or whatever you like instead of "Hình"

% Muka surat bernombor dalam mod penyerahan, dan tidak bernombor dalam versi siap cetak
%\ifcvprfinal\pagestyle{empty}\fi
\setcounter{page}{1}
\begin{document}

%%%%%%%%% TAJUK
\title{Primer Berpacukan Data ECDO Bahagian 2/2: Penyelidikan Terhadap Anomali Saintifik dan Sejarah yang Paling Baik Diterangkan oleh “Pembalikan Bumi” ECDO}

\author{Junho\\
Diterbitkan Februari 2025\\
Laman Web (Muat turun kertas di sini): \href{https://sovrynn.github.io}{sovrynn.github.io}\\
Repo Penyelidikan ECDO: \href{https://github.com/sovrynn/ecdo}{github.com/sovrynn/ecdo}\\
{\tt\small junhobtc@proton.me}
}

\maketitle
%\thispagestyle{empty}

\begin{abstract}
Pada Mei 2024, seorang penulis dalam talian tanpa nama yang dikenali sebagai “The Ethical Skeptic” \cite{0} telah menyiarkan satu teori perintis yang dinamakan sebagai Hayunan Dzhanibekov Penyahgandingan Teras Mantel Eksoterma (Exothermic Core-Mantle Decoupling
Dzhanibekov Oscillation - ECDO) \cite{1}. Teori ini bukan sahaja mencadangkan bahawa Bumi pernah mengalami perubahan mendadak dan bencana pada paksi putarannya yang menyebabkan banjir besar di seluruh dunia apabila lautan tumpah ke daratan akibat inersia putaran, tetapi turut mencadangkan satu proses geofizik penyebab berserta data yang menunjukkan satu lagi fenomena pembalikan seperti ini mungkin akan berlaku tidak lama lagi. Walaupun ramalan bencana banjir dan hari kiamat seperti ini bukanlah sesuatu yang baharu, teori ECDO amat menarik kerana pendekatannya yang bersifat saintifik, moden, multidisiplin, dan berasaskan data.

Kertas kajian ini merupakan bahagian kedua daripada dua ringkasan padat hasil 6 bulan penyelidikan bebas \cite{2,20} mengenai teori ECDO yang memberi tumpuan khusus kepada anomali saintifik dan sejarah yang paling baik diterangkan oleh fenomena bencana "pembalikan Bumi" ECDO.

\end{abstract}

\section{Pengenalan}

Geologi dan sejarah moden beraliran uniformitarian mendakwa bahawa landskap geologi utama seperti Grand Canyon terbentuk selama berjuta-juta tahun \cite{143}; bahawa garam yang terdapat di Death Valley (California) wujud kerana kawasan itu dahulunya berada di bawah lautan ratusan juta tahun lalu \cite{144}; bahawa nenek moyang kita 150 generasi yang lalu menghabiskan seluruh hidup mereka membina makam-makam gergasi \cite{29,70}; dan bahawa "bahan api fosil" berusia ratusan juta tahun \cite{104}. Perkara yang mungkin paling menarik adalah, manusia dipercayai telah wujud selama 300,000 tahun \cite{145}, namun sejarah bertulis dan peradaban hanya bermula kira-kira 5,000 tahun lalu – iaitu bersamaan dengan 150 generasi manusia.

Anomali-anomali tersebut, seperti yang akan kita lihat, diterangkan dengan paling baik oleh kuasa geologi yang bersifat bencana.

\section{Gajah Mammoth Beku Segera yang Terkambus Dalam Lumpur}
\begin{figure}[b]
\begin{center}
% \fbox{\rule{0pt}{2in} \rule{0.9\linewidth}{0pt}}
   \includegraphics[width=1\linewidth]{jarkov-mammoth.jpg}
\end{center}
   \caption{Mamut Jarkov, seekor mamut Siberia berusia 20,000 tahun dengan awet sempurna yang dijumpai dalam lumpur beku \cite{51}.}
\label{fig:1}
\label{fig:onecol}
\end{figure}

Salah satu kategori anomali tersebut ialah mamut dengan awet sempurna dan beku secara tiba-tiba yang tertanam di dalam lumpur, lazim ditemui di kawasan Artik (Rajah \ref{fig:1}). Mamut Beresovka yang ditemui di Siberia, terkubur dalam kerikil berlumpur dengan awet yang begitu sempurna sehingga dagingnya masih boleh dimakan ribuan tahun selepas kematiannya. Ia juga mempunyai makanan berasaskan tumbuhan di dalam mulut dan perutnya, menyebabkan para saintis tertanya-tanya cara ia menjadi beku dengan begitu cepat sekiranya ia sedang meragut tumbuhan berbunga sebelum kematiannya \cite{17}. Dilaporkan, \textit{"Pada tahun 1901, satu sensasi telah berlaku akibat penemuan bangkai mamut lengkap berhampiran sungai Berezovka kerana haiwan ini nampaknya telah mati akibat kesejukan pada pertengahan musim panas. Kandungan perutnya diawetkan dengan baik dan termasuk buttercup serta kacang liar berbunga: ini bermakna makanan itu mesti telah ditelan sekitar penghujung bulan Julai atau awal Ogos. Makhluk itu mati dengan sangat tiba-tiba sehingga masih terdapat rumput dan bunga di mulutnya. Ia jelas telah disambar oleh kuasa yang sangat dahsyat dan dicampakkan jauh beberapa batu dari tempat ia meragut. Pelvis dan satu kakinya telah patah-haiwan besar itu telah terhempap sehingga ke lututnya dan kemudian mati beku pada waktu yang biasanya merupakan masa paling panas pada tahun tersebut"} \cite{18}. Selain itu, \textit{"[Pakar sains Rusia] telah mencatatkan bahkan lapisan paling dalam perut haiwan itu mempunyai struktur berserat yang terpelihara sempurna, menunjukkan haba badan haiwan itu telah dikeluarkan oleh suatu proses semula jadi yang sangat luar biasa. Sanderson, dengan mengambil perhatian khas tentang perkara ini, membawa masalah ini kepada Institut Makanan Beku Amerika: Apakah yang diperlukan untuk membekukan seekor mamut sepenuhnya sehinggakan kandungan lembapan di bahagian dalam tubuhnya, lapisan dalam perutnya, tidak mempunyai masa yang cukup untuk membentuk kristal besar yang boleh memusnahkan struktur berserat dagingnya?... Beberapa minggu kemudian, Institut menghubungi Sanderson dengan jawapan: Ia benar-benar mustahil. Dengan segala ilmu saintifik dan kejuruteraan yang ada, tiada cara yang diketahui dapat mengeluarkan haba badan daripada bangkai sebesar mamut dengan cukup cepat untuk membekukannya tanpa pembentukan kristal lembapan besar pada daging. Tambahan pula, setelah meneliti semua kaedah sains dan kejuruteraan, mereka melihat kepada alam dan menyimpulkan bahawa tiada proses semula jadi yang diketahui boleh melaksanakan perkara ini"} \cite{19}.

\section{Grand Canyon}

Grand Canyon, sebahagian daripada Great Basin di barat daya Amerika Utara, ialah satu lagi fenomena semula jadi yang menunjukkan asal-usul katastrofik (Rajah \ref{fig:2}). Sebagai permulaan, lapisan batu pasir dan batu kapur sedimen yang membentuk Grand Canyon meliputi kawasan yang sangat luas sehingga 2.4 juta km$^2$ \cite{21}. Rajah \ref{fig:3} menunjukkan keluasan lapisan Coconino Sandstone merentasi barat Amerika Syarikat. Lapisan mendatar bahan seragam sebesar ini hanya boleh dihamparkan secara serentak.

\begin{figure}[t]
\begin{center}
% \fbox{\rule{0pt}{2in} \rule{0.9\linewidth}{0pt}}
   \includegraphics[width=1\linewidth]{grand-canyon.jpg}

\end{center}
   \caption{Grand Canyon di Arizona, Amerika Syarikat \cite{49}.}
\label{fig:2}
\label{fig:onecol}
\end{figure}

\begin{figure}[t]
\begin{center}
% \fbox{\rule{0pt}{2in} \rule{0.9\linewidth}{0pt}}
   \includegraphics[width=1\linewidth]{coconino.jpg}
\end{center}
   \caption{Saiz lapisan Coconino Sandstone di barat Amerika Syarikat \cite{21}.}
\label{fig:3}
\label{fig:onecol}
\end{figure}

Pemerhatian yang lebih dekat pada Grand Canyon memberitahu kami bahawa pemendapan lapisan-lapisan sedimen yang luas ini juga berlaku serentak dengan kuasa tektonik yang besar. Untuk memahami perkara ini, kita perlu melihat dengan teliti pada kawasan-kawasan tertentu di ngarai tempat lapisan sedimen terlipat dan terdedah. Pengkaji dari Answers in Genesis \cite{42} telah melihat secara mikroskopik pada sampel batuan daripada beberapa lipatan ini, seperti Monument Fold, dan berdasarkan ketiadaan ciri-ciri yang sepatutnya muncul jika lipatan terbentuk dalam jangka masa panjang di bawah haba dan tekanan, mereka menyimpulkan bahawa lapisan sedimen tersebut telah terlipat oleh kuasa tektonik semasa ia masih lembut, iaitu, tidak lama selepas pemendapannya \cite{43}.

\begin{figure*}
\begin{center}
% \fbox{\rule{0pt}{2in} \rule{.9\linewidth}{0pt}}
\includegraphics[width=1\textwidth]{Grand_Staircase-big.jpg}
\end{center}
   \caption{Lapisan sedimen yang membentuk Grand Canyon (sebelah kanan gambar) sehingga ke utara Cedar Breaks, Utah (sebelah kiri gambar), tempat semuanya melengkung ke atas \cite{50}.}
\label{fig:4}
\end{figure*}

Apabila kami besarkan pandangan, kami dapati lapisan-lapisan yang membentuk Grand Canyon bukan sahaja telah dilipat di dalam ngarai. Lapisan-lapisan ini telah dilipat ke arah timur di East Kaibab Monocline \cite{46}, tetapi juga ke arah utara di Cedar Breaks, Utah (Rajah \ref{fig:4}). Ini mencadangkan bahawa semua lapisan ini mungkin telah dilipat bersama-sama selepas ia bertindih antara satu sama lain dalam tempoh yang singkat. Sebagai rujukan, lapisan mendatar Grand Canyon mempunyai ketebalan kira-kira 1700 meter. Skala proses geologi yang diperlukan untuk membentuk lapisan sedimen setebal hampir satu batu adalah sangat besar.

Pembentukan sebenar Grand Canyon pula merupakan satu lagi isu yang didebatkan dalam geologi moden. Geologi uniformitarian mencadangkan Grand Canyon telah diukir oleh Sungai Colorado selama berjuta-juta tahun \cite{47}. Walau bagaimanapun, pasukan pengkaji Answers in Genesis percaya Grand Canyon kemungkinan besar terbentuk dalam masa beberapa minggu sahaja akibat hakisan limpahan dari sebuah tasik purba yang memecahkan sempadannya, lalu menanggalkan sejumlah besar sedimen semasa mengukir ngarai tersebut. Terdapat bukti kewujudan sebuah tasik beraltitud tinggi di sebelah timur Grand Canyon dalam bentuk mendapan tasik dan fosil marin. Membandingkan Grand Canyon dengan contoh-contoh hakisan limpahan berskala besar yang lain, seperti Afton Canyon dan Mount St. Helens, menunjukkan topografi yang serupa dan membuktikan ngarai besar boleh terbentuk dengan cepat melalui aliran air yang sangat banyak \cite{48}.

Memandangkan skala proses geologi yang diperlukan untuk membentuk lapisan sedimen di kawasan yang begitu luas, serta berlakunya daya tektonik yang sangat besar tidak lama selepas lapisan-lapisan sedimen terbentuk, dan juga saiz Sungai Colorado yang sangat kecil berbanding skala Grand Canyon, ia kelihatan bahawa tiada apa-apa yang berlaku secara beransur-ansur dalam pembentukannya.

\section{Bandar Bawah Tanah Derinkuyu}

Selain piramid, satu contoh kejuruteraan purba yang hebat ialah bandar bawah tanah Derinkuyu (Rajah \ref{fig:5}) yang terletak di Cappadocia, Turki. Ia adalah yang terbesar di antara lebih 200 tempat perlindungan bawah tanah di kawasan tersebut \cite{54}. Bandar bawah tanah ini dianggarkan mampu menampung sehingga 20,000 orang dan terdiri daripada 18 tingkat, dengan kedalaman sehingga 85 meter. Walaupun usia sebenarnya tidak diketahui dengan pasti, ia dianggarkan berusia sekurang-kurangnya 2800 tahun. Bandar ini diukir daripada batu gunung berapi yang lembut \cite{52, 53}.

\begin{figure}[b]
\begin{center}
% \fbox{\rule{0pt}{2in} \rule{0.9\linewidth}{0pt}}

   \includegraphics[width=1\linewidth]{derinkuyu.jpeg}
\end{center}
   \caption{Rajah bandar bawah tanah Derinkuyu \cite{56}.}
\label{fig:5}
\label{fig:onecol}
\end{figure}

Sebab Derinkuyu menarik adalah kerana tiada sebab yang jelas untuk mana-mana komuniti memutuskan pembinaan seluruh bandar di bawah tanah. Untuk mencipta ruang kediaman di bawah tanah, setiap rongga mesti dipahat daripada batu. Bentuk tidak sekata dan tekstur terowong bawah tanah ini jelas menunjukkan ia dipahat menggunakan tenaga manusia, bukannya peralatan moden yang akan jauh lebih sukar berbanding membina perlindungan di atas tanah. Malah, tiada sebab yang jelas untuk mana-mana manusia ingin tinggal secara kekal di bawah tanah sepanjang hidup mereka, sedangkan pertanian, cahaya matahari, alam semula jadi, dan penerokaan hanya boleh didapati di atas tanah. "Sejarah" konvensional mencadangkan bahawa Derinkuyu dicipta oleh penganut Kristian yang memerlukan tempat terpencil untuk mengamalkan agama mereka \cite{53}. Namun, logik akal menyimpulkan cara yang paling mudah untuk berurusan dengan musuh ialah "lawan atau lari", bukannya "memahat sebuah bandar bawah tanah daripada batu".

Skala, kedalaman, dan ketelitian reka bentuk bandar bawah tanah ini jelas menunjukkan ia bukan direka sebagai struktur pertahanan ketenteraan sementara untuk melawan penceroboh semasa kecemasan, tetapi sebaliknya, sebagai perlindungan jangka panjang daripada ancaman maut di permukaan. Derinkuyu bukan sahaja dilengkapi dengan bilik tidur asas, dapur, dan bilik mandi, tetapi juga kandang haiwan, tangki air, stor makanan, alat pemerah anggur dan minyak, sekolah, gereja, makam, dan lubang pengudaraan besar (Rajah \ref{fig:6}). Mengapa perlindungan ketenteraan memerlukan alat pemerah anggur dan perlu digali sedalam 85 meter dengan begitu kompleks?

Penjelasan yang paling munasabah untuk penciptaan Derinkuyu ialah wujudnya keperluan mendesak untuk menyediakan perlindungan jangka panjang dan mandiri untuk melindungi daripada kuasa geofizik bencana di permukaan Bumi.

\begin{figure}[t]
\begin{center}
% \fbox{\rule{0pt}{2in} \rule{0.9\linewidth}{0pt}}
   \includegraphics[width=1\linewidth]{derinkuyu-air.jpg}
\end{center}
   \caption{Perigi pengudaraan yang dalam di Derinkuyu \cite{53}.}
\label{fig:6}

\label{fig:onecol}
\end{figure}

% \section{Anomali Tambahan Yang Dijelaskan Dengan Baik Oleh Pembalikan Bumi}

% Sebelum mengakhiri, kami akan menyebutkan beberapa anomali saintifik tambahan yang apabila dilihat dalam konteks kuasa geofizik yang dahsyat, dapat dijelaskan dengan baik.

\section{Pengumpulan Biojisim}

Campuran biojisim daripada pelbagai jenis haiwan dan tumbuhan yang sering ditemui dalam bentuk fosil dalam lapisan sedimen, merupakan satu lagi anomali yang membingungkan. Dalam "Reliquoæ Diluvianæ", Rev. William Buckland memperincikan penemuan pelbagai spesies fauna yang tidak mempunyai sebab yang dapat dijelaskan untuk ditemui bersama, bertaburan di seluruh Britain dan Eropah yang tertanam dalam lapisan 'diluvium' sedimen \cite{58}. Campuran sebegini turut ditemui di Gua Skjonghelleren di pulau Valdroy, Norway. Dalam gua ini, lebih daripada 7,000 tulang mamalia, burung, dan ikan ditemui bercampur dalam beberapa lapisan sedimen \cite{59}. Satu lagi contoh ialah San Ciro, "Gua Gergasi", di Itali. Dalam gua ini, beberapa tan tulang mamalia, kebanyakannya badak air, ditemui dalam keadaan yang sangat segar sehingga ia dipotong untuk dijadikan perhiasan dan dihantar untuk pembuatan jelaga lampu. Tulang-tulang daripada pelbagai haiwan itu dilaporkan bercampur, patah, pecah, dan bertaburan dalam bentuk serpihan \cite{60,61}. Di Mendès Purba, Mesir, campuran pelbagai spesies tulang haiwan ditemui bercampur dengan tanah liat tervitrifikasi (seperti kaca) \cite{57}. Penemuan sebegini mungkin tampak membingungkan, tetapi boleh dijelaskan dengan mudah oleh banjir besar yang menempatkan campuran haiwan mati dalam lapisan sedimen, menghanyutkan haiwan atau menanamnya hidup-hidup di dalam gua, dan dalam kes biojisim tervitrifikasi di Mesir, pelepasan elektrik besar-besaran pascabanjir akibat anjakan teras-mantel. Rajah \ref{fig:7} menggambarkan satu pendedahan tipikal 'muk' biojisim Alaska \cite{56}.

\begin{figure}[t]
\begin{center}
% \fbox{\rule{0pt}{2in} \rule{0.9\linewidth}{0pt}}
   \includegraphics[width=1\linewidth]{muck-crop.jpeg}
\end{center}
   \caption{'Muk' Alaska, terdiri daripada serpihan pokok, tumbuhan dan haiwan yang bertaburan dalam kelodak beku dan ais \cite{146}.}
\label{fig:7}
\label{fig:onecol}
\end{figure}
\section{Bunker Purba}

Nenek moyang kita telah meninggalkan banyak struktur purba yang direka bentuk dengan sangat teliti di mana sisa manusia telah ditemui. Ini biasanya ditafsirkan sebagai makam berhias, tetapi pemerhatian yang lebih teliti mencadangkan ini sebenarnya mungkin satu bunker purba.

\begin{figure}[b]
\begin{center}
% \fbox{\rule{0pt}{2in} \rule{0.9\linewidth}{0pt}}
   \includegraphics[width=1\linewidth]{ww19.jpg}
\end{center}
   \caption{Newgrange, Ireland - lihat pelawat di pintu masuk untuk skalanya.}
\label{fig:8}
\label{fig:onecol}
\end{figure}

Satu contoh yang sangat baik ialah Newgrange (Rajah \ref{fig:8}), monumen utama dalam kompleks Brú na Bóinne, satu koleksi struktur purba termasuk struktur yang dipanggil makam laluan. Makam-makam ini terdiri daripada satu atau lebih ruang pengebumian yang dilitupi tanah atau batu dan mempunyai laluan masuk yang sempit dan diperbuat daripada batu-batu besar \cite{70}. Ia adalah contoh kejuruteraan meluas bagi struktur perlindungan yang kompleks, dibina merentasi beberapa generasi yang kononnya untuk mengebumikan sebilangan kecil orang yang sebenarnya tidak lagi hidup apabila pembinaan makam bermula. Apabila ia ditemui semula oleh seorang pemilik tanah tempatan pada tahun 1699, ia telah ditimbus dengan tanah.

Pemerhatian ringkas pada struktur ini menunjukkan usaha besar yang dicurahkan untuk membinanya - Newgrange terdiri daripada kira-kira 200,000 tan bahan. Di dalamnya, \textit{“...terdapat laluan berbilik yang boleh diakses melalui pintu masuk di sebelah tenggara monumen. Laluan ini memanjang sepanjang 19 meter (60 kaki), atau kira-kira satu pertiga ke bahagian tengah struktur. Di hujung laluan terdapat tiga bilik kecil yang bercantum dengan satu bilik tengah yang lebih besar dengan bumbung berkubah corbel yang tinggi… Dinding laluan ini terdiri daripada kepingan batu besar yang dipanggil ortostat, dua puluh dua di bahagian barat dan dua puluh satu di bahagian timur. Purata ketinggiannya adalah 1½ meter”} \cite{70}. Terdapat juga perincian kejuruteraan kalis air yang rumit. Sebagai contoh, di bumbung, \textit{“Retakan bumbung telah disumbat dengan campuran tanah bakar dan pasir laut untuk menjadikannya kalis air dan daripada campuran ini, dua radiokarbon bertarikh sekitar 2500 SM telah diperoleh untuk struktur makam"} \cite{71}. Selain itu, tangga aras menuju ke ruang dalam mungkin telah digunakan untuk tujuan yang sama: \textit{“Oleh kerana lantai laluan dan bilik makam mengikut ketinggian tanah bukit yang menjadi tapak binaan monumen ini, terdapat perbezaan hampir 2 meter dari aras lantai antara pintu masuk dan bahagian dalam bilik”} \cite{71}.

\begin{figure}[t]

\begin{center}
% \fbox{\rule{0pt}{2in} \rule{0.9\linewidth}{0pt}}
   \includegraphics[width=1\linewidth]{dolmen.jpg}
\end{center}
   \caption{Dolmen de Soto, Sepanyol \cite{53}.}
\label{fig:9}
\label{fig:onecol}
\end{figure}

Ketiadaan tinggalan manusia di dalamnya juga merupakan satu perkara yang aneh. Ekskavasi (pengorekan) telah menemui serpihan tulang yang terbakar dan tidak terbakar yang mewakili beberapa manusia, berselerakan di sepanjang laluan. Pembinaan Newgrange dianggarkan telah mengambil masa sekurang-kurangnya beberapa generasi berdasarkan tarikh karbon bahan di dalamnya. Mengapa komuniti purba sanggup bersusah payah membina makam besar yang tereka bentuk hanya untuk menyelerakkan serpihan tulang beberapa manusia mati di laluan masuknya? Ia lebih munasabah untuk struktur megalitik purba yang dibina dengan teliti dan kalis air ini sebenarnya dibina sebagai tempat perlindungan manusia untuk melindungi orang ramai semasa bencana besar di Bumi.

Di Huelva, selatan Sepanyol, contoh yang serupa ialah Dolmen de Soto (Rajah \ref{fig:9}), salah satu daripada kira-kira 200 tapak seumpama itu di kawasan tersebut \cite{72,32}. Ia adalah struktur yang ringkas dan sangat tereka bentuk menggunakan batu-batu megalitik dan mempunyai diameter berukuran 75 meter. Dilaporkan bahawa hanya lapan mayat telah ditemui semasa ekskavasi, semuanya dikebumikan dalam posisi janin.

\section{Sebutan Anomali yang Penting}

Dalam bahagian ini, saya secara ringkas menyebut beberapa lagi anomali penting yang semuanya dapat dijelaskan dengan baik oleh bencana seumpama ECDO.

\subsection{Anomali Biologi}

\begin{figure}[b]

\begin{center}
% \fbox{\rule{0pt}{2in} \rule{0.9\linewidth}{0pt}}
   \includegraphics[width=1\linewidth]{bottleneck.jpg}
\end{center}
   \caption{Kejadian penyempitan genetik yang menggambarkan pengurangan sekitar 95\% lelaki kira-kira 6,000 tahun yang lalu \cite{62}.}
\label{fig:10}
\label{fig:onecol}
\end{figure}

Beberapa anomali biologi yang ketara ialah kejadian penyempitan genetik dan fosil paus di kawasan pedalaman. Zeng et al. (2018) memodelkan 125 urutan kromosom Y daripada manusia moden, dan berdasarkan persamaan serta mutasi dalam DNA, mereka mengenal pasti kejadian penyempitan populasi dengan pengurangan 95\% dalam populasi lelaki sekitar 5,000 hingga 7,000 tahun yang lalu (Rajah \ref{fig:10}) \cite{62}. Fosil paus telah ditemui ratusan meter di atas paras laut di Swedenborg, Michigan, Vermont, Kanada, Chile, dan Mesir \cite{63,64,65,66}. Paus-paus ini ditemui dalam keadaan berbeza: terpelihara dengan sempurna, dalam paya yang terletak di atas mendapan glasier, atau tertanam dalam sedimen. Bilangan spesimen di tapak-tapak ini berbeza dari beberapa hingga lebih seratus. Paus ialah hidupan laut dalam dan jarang sekali menghampiri pantai. Bagaimanakah paus-paus ini boleh berakhir di altitud sedemikian tinggi yang selalunya sangat jauh ke pedalaman?

Pelbagai kepupusan besar-besaran telah berlaku dalam sejarah lalu Bumi, kepupusan yang paling dikaji ialah peristiwa "Lima Utama" Zaman Fanerozoik: kepupusan Ordovician Lewat (LOME), Devonian Lewat (LDME), Tamat Permian (EPME), Tamat Triassic (ETME) dan Tamat Kapur (ECME) \cite{88,89}. Menariknya, beberapa kepupusan ini diklasifikasikan berlaku dalam tempoh sejarah yang sama seperti kebanyakan lapisan Grand Canyon, iaitu lapisan Permian dan Devonian.

\subsection{Anomali Fizikal}

\begin{figure}[b]
\begin{center}
% \fbox{\rule{0pt}{2in} \rule{0.9\linewidth}{0pt}}
   \includegraphics[width=1\linewidth]{columbia.jpg}
\end{center}
   \caption{Gegaran arus besar di Tasik Glasier Columbia, negeri Washington \cite{80}.}
\label{fig:11}
\label{fig:onecol}
\end{figure}

Terdapat banyak landskap selain Grand Canyon yang berkemungkinan besar terbentuk melalui daya kataklismik. Bukti aliran air benua berskala besar boleh ditemui dalam bentuk gelombang arus gergasi di seluruh dunia. Salah satu contohnya ialah Channeled Scablands di Pacific Northwest. Di sini, kami bukan sahaja melihat landskap endapan sedimen dan bongkah batu ganjil, tetapi juga lebih daripada seratus turutan gelombang besar yang terbentuk daripada aliran arus mega \cite{78,79}. Ini merupakan versi gelombang berskala besar yang terbentuk di dasar pasir sungai. Ciri-ciri ini boleh ditemui di serata dunia seperti di Perancis, Argentina, Rusia, dan Amerika Utara \cite{81}. Rajah \ref{fig:11} menunjukkan sebahagian daripada gelombang ini di negeri Washington di Amerika Syarikat \cite{80}.

\begin{figure}[b]
\begin{center}
% \fbox{\rule{0pt}{2in} \rule{0.9\linewidth}{0pt}}
   \includegraphics[width=1\linewidth]{zhangjiajie.jpg}
\end{center}
   \caption{Tiang batu besar di Hutan Negara Zhangjiajie, selatan China.}
\label{fig:12}
\label{fig:onecol}
\end{figure}

\begin{figure}[b]
\begin{center}
% \fbox{\rule{0pt}{2in} \rule{0.9\linewidth}{0pt}}
   \includegraphics[width=1\linewidth]{hoy.jpg}
\end{center}
   \caption{Tiang laut Old Man of Hoy, Scotland \cite{83}.}
\label{fig:13}
\label{fig:onecol}
\end{figure}

Struktur hakisan daratan juga dapat dijelaskan dengan baik oleh kejadian pembalikan Bumi ala ECDO. Selatan China merupakan contoh hebat bagi landskap kars yang luas, terbentuk melalui hakisan air \cite{82}. Landskap ini merangkumi kars menara, kars puncak, kars kon, jambatan semula jadi, gaung, sistem gua besar, dan lubang benam. Salah satu yang paling menonjol ialah Taman Negara Hutan Zhangjiajie yang mengandungi tiang pasir kuarza besar (Rajah \ref{fig:12}) \cite{84}. Tiang-tiang ini berdiri pada ketinggian purata melebihi 1,000 meter dengan bilangan melebihi 3,100. Lebih daripada 1,000 tiang ini menjulang setinggi lebih 120 meter, dan 45 daripadanya melebihi 300 meter \cite{85}. Tiang-tiang ini menyerupai tiang hakisan laut (Rajah \ref{fig:13}), iaitu tiang batu di pesisir pantai yang terbentuk hasil runtuhan bahan sekeliling akibat ombak laut. Landskap hakisan serupa boleh ditemui di kon batu Urgup, Turki, serta Ciudad Encantada, Sepanyol, kedua-duanya terletak lebih 1,000 meter dari aras laut. Semua lokasi ini mempunyai beberapa gabungan fosil garam dan marin lautan yang berdekatan, menandakan pencerobohan laut pada masa lalu \cite{15,86,87}. Sudah tentu, kisah banjir \cite{3} menyebut yang lautan naik jauh lebih tinggi daripada 1,000 meter, dan ini disahkan oleh kewujudan air masin serta lapangan garam besar di Andes dan Himalaya yang beberapa kilometer di atas aras laut. Sebagai contoh, dataran garam Uyuni di Bolivia mencapai 3653 meter di atas aras laut \cite{94}.

\subsection{Peristiwa Perubahan Iklim Mendadak}

Literatur sains moden mengakui kewujudan peristiwa perubahan iklim global yang mendadak dalam sejarah Bumi yang terkini. Dua contoh utama ialah peristiwa 4.2 ribu tahun dan 8.2 ribu tahun, kedua-duanya bertepatan dengan pengurangan populasi dan gangguan penempatan masyarakat di kawasan geografi yang luas. Peristiwa-peristiwa ini direkodkan sebagai anomali dalam teras sedimen dan ais, fosil karang, nilai isotop O18, rekod debunga dan speleotem, serta data aras laut. Perubahan iklim yang dirumuskan termasuk penurunan suhu global yang mendadak, pengeringan, gangguan arus songsang meridional Atlantik, dan kemajuan glasier \cite{90,91,92}. Peristiwa 8.2 ribu tahun khususnya berlaku serentak dengan kemungkinan berlaku banjir air masin dramatik di Laut Hitam sekitar 6400 SM \cite{93}.

\begin{figure}[b]
\begin{center}
% \fbox{\rule{0pt}{2in} \rule{0.9\linewidth}{0pt}}

\includegraphics[width=1\linewidth]{jericho.jpg}
\end{center}
   \caption{Pembinaan semula arkeologi pengebumian Menara Jericho sekitar 7400 SM \cite{95}.}
\label{fig:14}
\label{fig:onecol}
\end{figure}

\subsection{Anomali Arkeologi}

Bukti arkeologi beberapa bandar purba menunjukkan berbilang lapisan yang melibatkan pengebumian dan kemusnahan, mewujudkan rekod peristiwa malapetaka masa lampau. Bandar purba Jericho adalah salah satu daripadanya, terletak di Palestin moden. Ia mengandungi beberapa lapisan kemusnahan, dengan runtuhan struktur batu dan kebakaran yang kuat \cite{96,97}. Kronologi yang direkodkan dalam lapisan-lapisannya bertarikh dari sekitar 9000 SM hingga 2000 SM. Paling menarik ialah menaranya yang kelihatan telah dipotong dan tertimbus dalam sedimen sekitar 7400 SM (Rajah \ref{fig:14}) \cite{95}. Catal Huyuk \cite{99}, Gramalote \cite{98}, dan istana Minoan Knossos di Crete \cite{100,101} adalah contoh-contoh tapak arkeologi yang mempunyai lapisan berbilang yang selalunya mengandungi bukti kemusnahan.

Satu lagi bukti bencana besar yang mengganggu tamadun manusia ialah Gambar Nampa, sebuah patung tanah liat yang ditemui di bawah kira-kira 100 meter lava di Idaho \cite{102,103}. Aliran lava di tempat patung itu ditemui dianggarkan telah mendap semasa zaman Tertier Akhir atau awal zaman Kuarterner yang didakwa berusia 2 juta tahun. Namun begitu, aliran lava di kawasan tersebut kelihatan agak baharu. Penemuan sebegini bukan sahaja menunjukkan bencana besar yang memusnahkan tamadun, malah turut menimbulkan persoalan terhadap kronologi penentuan tarikh moden.

\section{Mengenai Kaedah Penentuan Tarikh Moden}

Terdapat sebab yang kukuh untuk meragui kronologi moden yang memberikan usia berjuta-juta, malah sehingga ratusan juta tahun kepada pelbagai bahan fizikal.

Naratif konvensional menyatakan bahawa "bahan api fosil" seperti arang batu, minyak, dan gas asli berusia ratusan juta tahun \cite{104}. Namun, penentuan tarikh karbon sebenar terhadap minyak di Teluk Mexico mendapati minyak itu berusia kira-kira 13,000 tahun \cite{105}. Karbon-14 mempunyai separuh hayat yang begitu singkat (5,730 tahun) sehingga ia sepatutnya reput sepenuhnya selepas beberapa ratus ribu tahun. Namun, ia telah ditemui dalam arang batu dan fosil yang dikatakan beribu kali ganda lebih tua \cite{106}. Malah, arang batu tiruan telah dihasilkan di makmal di bawah keadaan terkawal, terutamanya suhu tinggi, hanya dalam masa 2-8 bulan \cite{107}.

Kaedah penentuan tarikh radioisotop selain daripada penentuan tarikh karbon juga mungkin tidak tepat. Kumpulan pengkaji Answers in Genesis telah menemui percanggahan dalam tarikh yang diperolehi melalui kaedah sebegini yang menyebabkan kesahihannya diragui \cite{108}. Tisu lembut yang mengandungi sel darah, salur darah, dan kolagen juga telah ditemui dalam tinggalan dinosaur yang kononnya berusia seratus juta tahun \cite{109,110}. Berdasarkan yang kita tahu, terdapat kemungkinan yang usia masa yang diterima secara konvensional bagi geologi Bumi dan bahan-bahan fizikal seperti batuan dan bahan api fosil adalah salah beribu kali ganda.

\section{Kesimpulan}

Dalam kertas ini, saya telah mengupas anomali-anomali paling meyakinkan yang menunjukkan asal-usul bencana dan diterangkan dengan paling baik oleh pembalikan Bumi ECDO. Walaupun pelbagai, koleksi yang dibentangkan ini adalah tidak lengkap - lebih banyak anomali telah dihimpunkan dan boleh diakses secara terbuka di repositori GitHub penyelidikan saya \cite{2}.
\section{Penghargaan}

Terima kasih kepada Ethical Skeptic, penulis asal tesis ECDO kerana telah menyiapkan tesisnya yang bernas dan inovatif serta berkongsi kepada dunia. Tesis tiga bahagian \cite{1} miliknya kekal sebagai karya autoritatif untuk teori Hayunan Dzhanibekov Penyahgandingan Teras Mantel Eksoterma, dan mengandungi lebih banyak maklumat tentang topik ini daripada yang saya ringkaskan secara singkat di sini.

Dan sudah tentu, terima kasih kepada para gergasi yang telah menjadi asas kepada pencapaian kami; mereka yang telah menjalankan segala penyelidikan dan siasatan yang membolehkan hasil kerja ini tercapai serta berusaha membawa pencerahan kepada kemanusiaan.

\clearpage
\twocolumn

{\small
\renewcommand{\refname}{Rujukan}
\bibliographystyle{ieee}
\bibliography{egbib}
}

\end{document}
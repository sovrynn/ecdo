\documentclass[10pt,twocolumn,letterpaper]{article}

% मेरो आफ्नै सामग्री
\usepackage{booktabs}
% \usepackage{caption}
% \captionsetup[table]{skip=8pt}   % यो केवल तालिकाहरूमा असर गर्छ
\usepackage{stfloats}  % यसलाई प्रिम्बलमा थप्नुहोस्

\usepackage{fontspec}
\usepackage{ucharclasses}

%–– define your two fonts ––
\newfontfamily\latinfont{Latin Modern Roman}              % for all non-Devanagari text
% \newfontfamily\nepalifont[Script=Devanagari]{Noto Serif Devanagari}    % for all Nepali text; change “Phobikha” to your font’s exact name
\newfontfamily\nepalifont[Script=Devanagari]{Noto Sans Devanagari}    % for 
% \newfontfamily\nepalifont{Noto Sans Devanagari}    % for 
% \newfontfamily\nepalifont{NewCM08Devanagari}    % for 

%–– ucharclasses auto-detects Unicode blocks ––
\setDefaultTransitions{\latinfont}{}                      
\setTransitionsForDevanagari{\nepalifont}{\latinfont}   % switch to Nepali font in Devanagari, then back
\setTransitionTo{DevanagariDanDa}{\nepalifont}

\usepackage{cvpr}
\usepackage{times}
\usepackage{epsfig}
\usepackage{graphicx}
\usepackage{amsmath}
\usepackage{amssymb}

% यहाँ अन्य प्याकेजहरू समावेश गर्नुहोस्, hyperref भन्दा पहिले।

% यदि तपाईंले hyperref लाई टिप्पणी गर्नुभयो र त्यसपछि uncomment गर्नुभयो भने,
% egpaper.aux लाई latex पुन: चलाउनु भन्दा पहिला मेटाउनु पर्छ।  (वा पहिलो latex रनमा 'q' थिच्नुहोस्,
% पूरा हुन दिनुहोस्, र तपाईँ सुरक्षित हुनुहुनेछ)।
\usepackage[breaklinks=true,bookmarks=false]{hyperref}

\cvprfinalcopy % *** अन्तिम बुझाउनेका लागि यो रेखा अनकमेन्ट गर्नुहोस्

\def\cvprPaperID{****} % *** यहाँ CVPR पेपर आईडी प्रविष्ट गर्नुहोस्
\def\httilde{\mbox{\tt\raisebox{-.5ex}{\symbol{126}}}}

% सबमिशन मोडमा पृष्ठहरू नम्बर गरिन्छ, र क्यामेरा-रेडीमा नम्बर नगरिएका हुन्छन्
%\ifcvprfinal\pagestyle{empty}\fi
\setcounter{page}{1}
\begin{document}

%%%%%%%%% TITLE
\title{ECDO डेटा-आधारित प्रारम्भिक पुस्तक भाग २/२: वैज्ञानिक र ऐतिहासिक विसंगतिहरूको अनुसन्धान, जुन ECDO "पृथ्वीको उल्टोफेर" ले सबैभन्दा राम्रोसँग व्याख्या गर्छ}

\author{जुनहो\\
प्रकाशित फेब्रुअरी २०२५\\
वेबसाइट (कागजातहरू यहाँ डाउनलोड गर्नुहोस्): \href{https://sovrynn.github.io}{sovrynn.github.io}\\
ECDO अनुसन्धान रिपो: \href{https://github.com/sovrynn/ecdo}{github.com/sovrynn/ecdo}\\
{\tt\small junhobtc@proton.me}
}

\maketitle
%\thispagestyle{empty}

\begin{abstract}
मे २०२४ मा, "द इथिकल स्केप्टिक" उपनाम द्वारा चिनिने एक अनलाइन लेखकले उत्सर्जनात्मक कोर-म्यान्टल छुट्टिने जानिबेकोभ कम्पन (Exothermic Core-Mantle Decoupling Dzhanibekov Oscillation) (ECDO) नामक एउटा नयाँ मार्ग खोल्ने सिद्धान्त प्रकाशित गरे \cite{0}। यस सिद्धान्तले पृथ्वीले अघिल्लो समयमा ध्रुवीय अक्षमा अचानक विनाशकारी विस्थापन भोगीसकेको र, जसले समुद्रहरूलाई जड़त्वाघूर्णका कारण महादेशहरूमा बगाई विश्वव्यापी बाढी निम्त्याएको मात्र प्रस्ताव गर्दैन, तर स्पष्ट कारणगत भूभौतिकीय प्रक्रिया समेत प्रस्ताव गर्छ र अर्को यस्तै उल्टोफेर (फ्लिप) नजिकिँदै गरेको हुन सक्ने डाटा समेत प्रस्तुत गर्छ \cite{1}। यस्ता महाविपत्ति र प्रलयका पूर्वानुमानहरू नयाँ होइनन्, तर ECDO सिद्धान्त वैज्ञानिक, आधुनिक, बहुविषयगत र डाटा-आधारित दृष्टिकोणका कारण अद्वितीय रूपमा आकर्षक छ।

यो अनुसन्धानपत्र ECDO सिद्धान्तमा ६ महिनासम्म गरिएको स्वतन्त्र अनुसन्धानको दुइ-भागको संक्षिप्त सारांशको दोस्रो भाग हो \cite{2,20}, जसले प्रमुख रूपमा वैज्ञानिक र ऐतिहासिक विसंगतिहरूलाई केन्द्रीत गरेर तिनीहरूलाई एक विनाशकारी ECDO "पृथ्वीको उल्टोफेर" बाटै सबैभन्दा राम्रो व्याख्या गर्न सकिन्छ भन्ने धारणा प्रस्तुत गर्छ।

\end{abstract}

\section{परिचय}

आधुनिक समान रूपले क्रियाशील भूविज्ञान र इतिहासले ठूला भू-परिदृश्यहरू जस्तै ग्राण्ड क्यान्यन लाखौं वर्ष लागेर बनेको दाबी गर्छ \cite{143}; डेथ भ्याली (क्यालिफोर्निया) मा नुन पाइन्छ किनभने त्यो क्षेत्र सयौं लाखौं वर्षअघि समुद्रमुनि थियो \cite{144}; हाम्रा १५० पुस्ताअघि भएका पुर्खाहरूले आफ्नो पूरा जीवन विशाल समाधिस्थल बनाउँदैं बिताए \cite{29,70}; र तथाकथित "फोसिल इन्धनहरू" सयौं लाखौं वर्ष पुराना छन् \cite{104}। सम्भवत: सबैभन्दा रोमाञ्चक कुरा के छ भने मानिसहरू ३००,००० वर्ष पुराना रहेको विश्वास गरिन्छ \cite{145}, तर लेखिएको इतिहास र सभ्यता केवल झन्डै ५,००० वर्ष मात्रै पुरानो छ — जुन मानवीय १५० पुस्ताको बराबरी हो।

यस्ता विसंगतिहरू, जुन् हामी देख्नेछौं, प्रलयकारी भूवैज्ञानिक बलहरूद्वारा सबैभन्दा राम्रोसँग व्याख्या गर्न सकिन्छ।

\section{माटोमा गाडिएका र तिव्रगतिमा जमेका म्यामोथहरू}

\begin{figure}[t]
\begin{center}
% \fbox{\rule{0pt}{2in} \rule{0.9\linewidth}{0pt}}
   \includegraphics[width=1\linewidth]{jarkov-mammoth.jpg}
\end{center}
   \caption{जार्कोव म्यामोथ, जमेको हिलोमा निक्कै राम्रोसँग संरक्षित भएको अवस्थामा फेला परेको २०,००० वर्ष पुरानो साइबेरियन म्यामोथ\cite{51}.}
\label{fig:1}
\label{fig:onecol}
\end{figure}
यस्तै प्रकारको विसंगति भनेको आर्कटिक क्षेत्रहरूमा सामान्यतया पाइने हिलोमा गाडिएका पूर्ण रूपमा संरक्षित तिव्रगतिमा जमेका म्यामथहरू हुन्। (Figure \ref{fig:1}). साइबेरियामा फेला परेको बेरेसोभ्का म्यामथ, जुन बालुवा मिसिएको गिट्टीमा गाडिएको थियो, यति राम्रोसँग संरक्षित भएको थियो कि यसको मासु मृत्यु भएको हजारौं वर्ष पछि पनि खान योग्य थियो। यसको मुख र पेटमा वनस्पतिजन्य आहार पनि थियो, जसले गर्दा वैज्ञानिकहरू अलमल्लमा परे कि यदि यो ठिक मृत्यु भन्दा अगाडी फूल फुल्ने बिरुवाहरू चरिरहेको थियो भने यो कसरी यति छिटो जमेको हुन सक्छ? \cite{17}. बताइन्छ कि \textit{"१९०१ मा बेजेरोभका नदी नजिक पूर्णरुपमा संरक्षित म्यामोथको एक शव फेला परेको समाचारले सनसनी फैलायो, किनभने यो जनावर जाडोका कारण गर्मी यामको मध्यमा मरेको देखिन्थ्यो। यसको पेटको सामग्री अक्षुण्ण अवस्थामा थियो र त्यसमा बटरकप र फुल्ने जङ्गली बियाहरु समावेश थिए: यसले जनाउँछ कि यी करिब जुलाईको अन्त्य वा अगस्टको सुरुवातमा निलिएको हुनुपर्छ  प्राणी यति आकस्मिक ढंगमा मरेको थियो कि यसको मुखभरी अझै बोक्रा र फूलहरूको थुप्रो थियो। यो स्पष्ट छ कि यसलाई एक विशाल शक्तिले उठाएर यसको चारो मैदानबाट केही किलोमीटर टाढा फालिएको  थियो। कूल्हो हड्डी र एक खुट्टा भाँचिएको थियो—त्यो विशाल जनावर घुँडामा ढलेको थियो र त्यसपछि जमेर मरेको थियो, जतिखेर सामान्यतया वर्षको सबभन्दा तातो समय हुन्छ"} \cite{18}. थप रूपमा, \textit{"[रूसी वैज्ञानिकहरू] ले रेकर्ड गरेका छन् कि जनावरको पेटको सबैभन्दा भित्री तह समेत राम्ररी सुरक्षित फाइबरयुक्त बनावटमा रहेछ, जसले जनाउँछ कि यसको शरीरको ताप कुनै असाधारण प्राकृतिक प्रक्रियाले हटाइएको रहेछ। स्यान्डरसनले यो कुरा विशेष रूपमा ध्यानमा राख्दै, समस्यालाई अमेरिकन फ्रोजन फूड्स इन्स्टिच्युटसम्म (अमेरिकाको जमेका खाना सम्बन्धी संस्था) लगे: पूरै म्यामोथलाई त्यसरी जमाउन के चाहिन्छ ताकि शरीरको सबैभन्दा भित्री भागसम्म—पेटको भित्री तहमा समेत—पानीको मात्रा यति छिटो जमे कि मांसपेशीको फाइबर संरचना नष्ट हुने ठूला क्रिस्टलहरू बन्न पाएन?... केही हप्ता पछि संस्थाले स्यान्डरसनलाई जवाफ दियो: यो असम्भव छ। हाम्रो सबै वैज्ञानिक र इन्जिनियरिङ ज्ञान हुँदाहुँदै पनि, म्यामथ जत्तिकै ठूलो शवबाट शरीरको ताप यति छिटो हटाएर मासुमा ठूला  क्रिस्टलहरू नबनाईकन जमाउने  कुनै ज्ञात तरिका छैन। अझ, वैज्ञानिक तथा इन्जिनियरिङ प्रविधिहरूको सबै अनुसंधान पछि, उनीहरूले प्रकृतितिर हेरे र निष्कर्ष निकाले कि प्रकृतिमा यस्तो प्रक्रिया नै छैन, जसले यो कार्य गर्न सक्छ"} \cite{19}.

\section{ग्रान्ड क्यान्यन}

ग्रान्ड क्यान्यन, उत्तर अमेरिकाको दक्षिणपश्चिमी भागको ग्रेट बेसिनको एक भाग, अर्को प्राकृतिक घटना हो जुन विनाशकारी उत्पत्तिको संकेत दिन्छ (Figure \ref{fig:2})। सुरुमा, ग्रान्ड क्यान्यन बनाउने अवसादी बालुवा ढुंगा र चुनापत्थर तहहरू दस लाख वर्ग किलोमिटर सम्म फैलिएको विशाल क्षेत्र ओगट्छन् \cite{21}। Figure \ref{fig:3} ले कोकोनिनो बालुवा ढुंगा तह संयुक्त राज्य अमेरिकाको पश्चिमी भागमा फैलिएको देखाउँछ। यस्ता विशाल तेर्सो समान तहहरू एकैचोटी थुप्रिएका हुनुपर्छ।

\begin{figure}[b]
\begin{center}
% \fbox{\rule{0pt}{2in} \rule{0.9\linewidth}{0pt}}
   \includegraphics[width=1\linewidth]{grand-canyon.jpg}

\end{center}
   \caption{ग्राण्ड क्यान्यन, एरिजोना, अमेरिका \cite{49}.}
\label{fig:2}
\label{fig:onecol}
\end{figure}

\begin{figure}[t]
\begin{center}
% \fbox{\rule{0pt}{2in} \rule{0.9\linewidth}{0pt}}
   \includegraphics[width=1\linewidth]{coconino.jpg}
\end{center}
   \caption{पश्चिम संयुक्त राज्य अमेरिकामा कोकोनिनो बालुवा ढुंगा तहको आकार \cite{21}।}
\label{fig:3}
\label{fig:onecol}
\end{figure}

ग्राण्ड क्यान्यनलाई नियालेर हेर्दा हामीलाई थाहा हुन्छ कि यी विशाल अवसादी तहहरूको निक्षेपण महत्वपूर्ण भूगर्भीय शक्तिहरूसँगै भएको थियो। यसलाई बुझ्नको लागि, हामीले क्यान्यनका ती क्षेत्रहरूलाई ध्यानपूर्वक हेर्नुपर्छ जहाँ अवसादी तहहरू मोडिएका छन् र बाहिर देखिएका छन्। एन्सर्स इन जेनेसिस (धार्मिक विश्वासीहरूका लागि विज्ञान र धर्मसम्बन्धी तर्कहरू प्रस्तुत गर्ने संस्था) का अनुसन्धानकर्ता \cite{42} ले यस्ता मोडहरूमध्ये केही, जस्तै मोनुमेन्ट फोल्ड, बाट ढुंगाका नमूनाहरूलाई सूक्ष्म रूपमा अध्ययन गरे र यदि मोडहरू लामो समयसम्म ताप र दवाबमा बनेका हुन् भने देखिनुपर्ने विशेषताहरू नभएको आधारमा निष्कर्ष निकाले कि अवसादी तहहरू निक्षेपण पछिको छोटो समयमै, अझै नरम हुँदा, भूगर्भीय शक्तिहरूले मोडिएका हुन् \cite{43}।

\begin{figure*}
\begin{center}
```latex
% \fbox{\rule{0pt}{2in} \rule{.9\linewidth}{0pt}}
\includegraphics[width=1\textwidth]{Grand_Staircase-big.jpg}
\end{center}
   \caption{ग्रान्ड क्यान्यन बनाउने अवसादी तहहरू (तस्वीरको दायाँ भाग) उत्तरमा सीडर ब्रेक्स, उटाह (तस्वीरको बायाँ भाग) सम्म फैलिएको छ, जहाँ तिनीहरू सबै माथि उठेका छन् \cite{50}।}
\label{fig:4}
\end{figure*}

समग्रमा हेर्दा, हामी पाउँछौं कि ग्रान्ड क्यान्यन बनाउने तहहरू केवल क्यान्यन भित्रै मात्र मोडिएका छैनन्। ती तहहरू पूर्वतर्फ ईस्ट काइबाब मोनोक्लाइनमा मोडिएका छन् \cite{46}, तर उत्तरतर्फ पनि सीडर ब्रेक्स, उटाह (चित्र \ref{fig:4}) मा मोडिएका छन्। यसले संकेत गर्दछ कि ती सबै तहहरू एकअर्काको माथि द्रुत गतिमा राखिएपछि एकसाथ जोडिएका हुन सक्छन् र भित्रै मोडिएको हुन सक्छन्। सन्दर्भको लागि, ग्रान्ड क्यान्यनका तेर्सो तहहरू करिब १७०० मिटर बाक्लो छन्। करिब एक किलोमीटर बाक्लो अवसादी तह राख्न आवश्यक भूवैज्ञानिक प्रक्रियाको स्तर अत्यन्त ठूलो छ।

ग्रान्ड क्यान्यनको वास्तविक निर्माण हालको भूविज्ञानमा अर्को विवादको विषय हो। समान रूपले क्रियाशील भूविज्ञान अनुसार ग्रान्ड क्यान्यन करोडौं वर्षमा कोलोराडो नदीले काटेको हो \cite{47}। तर, एन्सर्स इन जेनेसिस अनुसन्धान टोलीको विश्वास छ कि ग्रान्ड क्यान्यन प्राचीन तालको किनाराहरू भत्किदा भएको स्पिलवे (वा पानी निकास मार्ग) कटावका कारण केबल केही हप्तामै बनेको हुन सक्छ, जसले ठूलै मात्रामा अवसाद हटाउँदै क्यान्यन कोरेको थियो। ग्रान्ड क्यान्यनको पूर्वमा रहेका ताल अवसाद जम्मा भएकै तहहरू र समुद्री जीवाश्महरूमा उच्च-उचाइका तालका प्रमाणहरू छन्। ग्रान्ड क्यान्यनलाई अन्य ठूला स्पिलवे कटावका उदाहरणहरू, जस्तै अफ्टन क्यान्यन र माउन्ट सेंट हेलेन्ससँग तुलना गर्दा, समान स्थलाकृति देखिन्छ, र देखाउँछ कि ठूलो मात्रामा बग्ने पानीद्वारा ठूला क्यान्यनहरू छिट्टै बनाइन सक्छन् \cite{48}।
```
यति विशाल जमिनमा अवसाद बिछ्याउन आवश्यक पर्ने भूगर्भीय प्रक्रियाहरूको मात्रा, अवसादी तहहरू बिछ्याएको लगत्तै हुने विशाल टेक्टोनिक बलहरूको सहवर्तीता र ग्रान्ड क्यान्यनको विशाल आकारको तुलनामा कोलोराडो नदीको सानो आकारलाई विचार गर्दा, यसको गठनमा क्रमिक रूपमा केही नभएको जस्तो देखिन्छ।

\section{डेरिन्कुयु भूमिगत सहर}

पिरामिड बाहेक, प्राचीन ईन्जिनियरिङ्को एक उत्कृष्ट उदाहरण हो डेरिन्कुयु भूमिगत सहर (चित्र \ref{fig:5}), जुन क्यापाडोसिया, टर्कीमा अवस्थित छ। यो सो क्षेत्रमा रहेका २०० भन्दा बढी भूमिगत आश्रय स्थलहरुमध्ये सबैभन्दा ठूलो हो \cite{54}। यो भूमिगत शहरमा लगभग २०,००० मानिस बस्न सक्थे भन्ने अनुमान छ र १८ तल्ला फैलिएको छ, जसको गहिराइ ८५ मिटर छ। यसको उमेर निश्चित नभए पनि, कम्तीमा २८०० वर्ष पुरानो भएको अनुमान छ। यो सहर मुलायम ज्वालामुखी ढुंगाबाट काटेर बनाइएको हो \cite{52, 53}।

\begin{figure}[b]
\begin{center}
% \fbox{\rule{0pt}{2in} \rule{0.9\linewidth}{0pt}}

\includegraphics[width=1\linewidth]{derinkuyu.jpeg}
\end{center}
   \caption{डेरीनकुयु भूमिगत सहरको योजनाचित्र \cite{56}।}
\label{fig:5}
\label{fig:onecol}
\end{figure}

डेरिन्कुयु रोचक हुनुको कारण के हो भने कुनैपनि समुदायले पूरै सहरलाई भूमिगत रूपमा निर्माण गर्ने निर्णय किन लियो भन्ने स्पष्ट छैन। भूमिगत आवास क्षेत्र बनाउनको लागि, हरेक गुफा ढुंगाबाट काट्नु पर्छ। भूमिगत सुरुङहरूको असजिलो आकार र बनावटले देखाउँछ कि यीनीहरू हातैले कोटिएको हो, यान्त्रिक  औजारहरूको प्रयोग बिना, जुन सतहमा आवास निर्माण गर्नभन्दा कैयौं गुणा बढी गाह्रो थियो। वास्तवमा, यो स्पष्ट छैन कि कुनै पनि मानिसले आफ्नो पार्थिव जीवन स्थायी रूपमा भूमिगत बस्न  चाहन्थ्यो, जब कृषि, सूर्यको प्रकाश, प्रकृति र अन्वेषण जमिन माथि मात्र उपलब्ध छन।। पारंपरिक "इतिहास"ले भन्छ कि डेरिन्कुयु क्रिश्चियनहरूले बनाएका थिए जसलाई आफ्ना धर्मको अभ्यास गर्न छुट्टै ठाउँ चाहिएको थियो \cite{53}। तर सामान्य सोचले त भन्छ कि शत्रुहरूको सामना गर्ने सबैभन्दा सोझो उपाय भनेको "लड्नु वा भाग्नु" हो, "ढुंगाबाट पूरै भूमिगत सहर बनाउनु" होइन।

भूमिगत शहरको ढाँचाको स्केल, गहिराइ र विचारशीलताले यो स्पष्ट पार्छ कि यो दबाबको समयमा आक्रमणकारीहरूसँग राम्रोसँग लड्न अस्थायी सैन्य रक्षात्मक संरचनाको रूपमा डिजाइन गरिएको थिएन, बरु, सतहमा घातक शक्तिहरू विरुद्ध सुरक्षाको लागि दीर्घकालीन आश्रयको रूपमा डिजाइन गरिएको थियो। डेरिन्कुयुमा केवल आधारभूत शयनकक्ष, भान्सा, र बाथरूम मात्र होइन, जनावरका लागि खोर, पानीका ट्यांकी, खाद्यान्न भण्डारण, रक्सी र तेलका प्रेसहरू, विद्यालय, चैपलहरू, समाधिहरू, र विशाल हावासञ्चारका इनार (Figure \ref{fig:6}) समेत थिए। सैनिक आश्रयका लागि किन रक्सी प्रेस आवश्यक पर्थ्यो र किन ८५ मिटर गहिरो र यति जटिलताका साथ खन्नु पर्ने?
डेरिन्कुयुको सिर्जनाको सबैभन्दा सम्भाव्य स्पष्टीकरण भनेको पृथ्वीको सतहमा हुने विनाशकारी भूभौतिकीय शक्तिहरूबाट जोगाउन दीर्घकालीन, आत्मनिर्भर आश्रय तयार गर्ने तत्काल आवश्यकता थियो।

\begin{figure}[t]
\begin{center}
% \fbox{\rule{0pt}{2in} \rule{0.9\linewidth}{0pt}}
   \includegraphics[width=1\linewidth]{derinkuyu-air.jpg}
\end{center}
   \caption{डेरिन्कुयुमा गहिरो हावासञ्चारका इनार \cite{53}.}
\label{fig:6}

\label{fig:onecol}
\end{figure}

% \section{ पृथ्वी पल्टिदा राम्रोसँग बुझिने थप असामान्यताहरू }

% अन्त्य गर्नु अघि, हामी केही थप वैज्ञानिक असामान्यताहरू उल्लेख गर्नेछौं, जसलाई विनाशकारी भूगर्भीय शक्तिहरूको सन्दर्भमा हेर्दा राम्ररी व्याख्या गर्न सकिन्छ।

\section{बायोमास संचयहरू}

विभिन्न किसिमका जनावर र वनस्पतिहरूको जैव पदार्थ मिश्रण, प्रायः अवसादी तहहरूमा जीवाश्मको रूपमा पाइने, अर्को अनौठो असामान्यता हो। "Reliquoæ Diluvianæ"(प्रलयकालीन अवशेषहरू) मा, रेभरेन्ड विलियम बकल्यान्डले ब्रिटेन र युरोपभरि विस्तारित भएको विभिन्न प्रजातिका जनावरहरूको जीवाश्म विवरण दिएका छन्, जसलाई एउटै स्थानमा फेला पार्नुको कुनै स्पष्ट कारण थिएन, तिनीहरू अवसादी 'diluvium' (बाढीजन्य निक्षेप) तहमा गाडिएका थिए \cite{58}। यस्तै जीवाश्म मिश्रण नर्वेको वाल्द्रोय टापुको स्कजोंघेललेरेन गुफामा पनि फेला पर्‍यो। यस गुफामा, ७,००० भन्दा बढी कंकालहरू स्तनधारी, चराहरू र माछाहरूका, धेरै अवसादी तहहरूमा मिसिएका थिए \cite{59}। अर्को उदाहरण इटालीको "San Ciro", "Cave of the Giants"(दानवहरूको गुफा), हो। यस गुफामा, धेरै टन स्तनधारी हड्डीहरू, प्रायः जलगैंडा, यति ताजा अवस्थामा फेला परेका थिए कि तिनीहरूलाई गहनामा काटेर ल्याम्प ब्ल्याकको निर्माणको लागि पठाइएको थियो।। विभिन्न जनावरको हड्डी मिसिएका मात्र थिएनन्, बरु टुक्रा, फुटेका र छरिएका रूपमा फेला परे \cite{60,61}। प्राचीन मेंडेस, इजिप्टमा, विभिन्न प्रजातिका जनावरका हड्डीहरू काँचो (glassy) माटोसँग मिसिएको अवस्थामा भेटियो \cite{57}। यस्ता फेला परेका कुरा अनौठो लाग्न सक्छ, तर विशाल बाढीले मरेका जनावरहरूको मिश्रण  अवसादी तहमा राखेको, जनावरहरूलाई गुफाहरूमा जिउँदै जम्मा गरेको वा गाडेको सन्दर्भमा सजिलै बुझ्न सकिन्छ, र इजिप्टमा भेटिएको काँचो बायोमासको सन्दर्भमा, बाढी पछि मूल-म्यान्टल विस्थापनबाट आएको विशाल विद्युतीय डिस्चार्जले निर्माण भएको बुझिन्छ। चित्र \ref{fig:7} मा अलास्काली बायोमास 'मक' को साधारण नमुना देखाइएको छ \cite{56}।
\begin{figure}[t]
\begin{center}
% \fbox{\rule{0pt}{2in} \rule{0.9\linewidth}{0pt}}
   \includegraphics[width=1\linewidth]{muck-crop.jpeg}
\end{center}
   \caption{अलास्काली ‘मक’, जसमध्ये बोटबिरुवा, वनस्पति र जनावरका टुक्रा हरू एकआपसमा अस्तव्यस्त रूपले वितरित जमेको बालुवा र बरफमा छन् \cite{146}.}
\label{fig:7}
\label{fig:onecol}
\end{figure}
\section{प्राचीन बंकरहरू}

हाम्रा पूर्वजहरूले धेरै उच्च इन्जिनियरिङ्ग गरिएका प्राचीन संरचनाहरू छाडेका छन् जहाँ मानव अवशेषहरू फेला परेका छन्। यी प्रायः सज्जित समाधिहरूको रूपमा व्याख्या गरिन्छ, तर नजिकबाट हेर्दा यी वास्तवमा प्राचीन बंकरहरू हुन सक्ने सुझाव दिन्छ।

\begin{figure}[b]
\begin{center}
% \fbox{\rule{0pt}{2in} \rule{0.9\linewidth}{0pt}}
   \includegraphics[width=1\linewidth]{ww19.jpg}
\end{center}
   \caption{न्युग्रेन्ज, आयरल्यान्ड - स्केलको लागि प्रवेशद्वारमा रहेका भ्रमणकर्ताहरू देख्नुहोस्।}
\label{fig:8}
\label{fig:onecol}
\end{figure}

न्युग्रेन्ज (चित्र \ref{fig:8}) एक उत्कृष्ट उदाहरण हो, जुन ब्रु ना बोइन्ने परिसरको मुख्य स्मारक हो, जुन प्राचीन संरचनाहरूको संग्रह हो जसमा तथाकथित मार्गीय समाधिहरू सामेल छन्। यी समाधिहरू एक वा धेरै शवगृहहरूमा बनाइएका छन् जुन माटो वा ढुङ्गाले ढाकिएको हुन्छ र ठूलो ढुङ्गाबाट बनेको साँघुरो प्रवेशमार्ग हुन्छ \cite{70}। यो एक जटिल संरचनाको व्यापक इञ्जिनियरिङको उदाहरण हो, जुन धेरै पुस्ताहरूमा निर्माण गरिएको थियो, भनिन्छ, थोरै मानिसहरूलाई गाड्नका लागि, जो समाधिको निर्माण सुरु हुँदा जीवित पनि थिएनन्। जब सन् १६९९ मा एक स्थानीय जमिनधनीद्वारा पुनः पत्ता लाग्यो, यसलाई माटोले ढाकेको थियो।

संरचनामा सरसर्ती हेर्दा नै यसलाई बनाउन कति धेरै मेहनत गरिएको छ भन्ने थाहा हुन्छ - न्युग्रेन्ज करिब २००,००० टन सामग्रीबाट बनेको छ। यसको भित्र, \textit{“...एक कक्षसहितको पैदलमार्ग छ, जसमा स्मारकको दक्षिणपूर्वी छेउको प्रवेशद्वारबाट जान सकिन्छ। प्रवेशमार्ग १९ मिटर (६० फिट) लामो छ, जुन संरचनाको केन्द्रीय भागतर्फ लगभग एक तिहाइ पुग्छ। प्रवेशमार्गको अन्त्यमा एक ठूलो केन्द्रीय कक्षबाट तीनवटा साना कक्षहरू छन्, जसमा माथिल्लो भागमा मुकुटाकार संरचना गरिएको छाना छ… यस प्रवेशमार्गका भित्ताहरू विशाल ढुङ्गाका पट्टिहरू (orthostats) बाट बनेका छन्, जसमा पश्चिमी छेउमा २२ वटा र पूर्वी छेउमा २१ वटा छन्। तिनीहरूको औसत उचाइ १½ मिटर छ”} \cite{70}। यहाँ पानी प्रतिरोधी बनाउने इञ्जिनियरिङसम्बन्धी जटिल विवरणहरू पनि भेटिन्छन्। उदाहरणका लागि, छानामा \textit{“छानाका फाटहरू जलरोधक बनाउनको लागि जलेको माटो र समुन्द्रको बालुवाको मिश्रणले छोपिएको थियो र यस मिश्रणबाट समाधिको संरचनाका लागि २५०० ई.पू. वरिपरिका दुई रेडियोकार्बन मितिहरू प्राप्त गरियो"} \cite{71}। थप रूपमा, भित्री कक्षमा जाने उचाइ वृद्धि समान उद्देश्यका लागि लागू गरिएको हुन सक्छ: \textit{“समाधिको बाटो र कक्षको भुइँ पहाडको जमिनको उचाइसँगै मिल्न आएको हुनाले, जहाँ स्मारक निर्माण गरिएको छ, त्यहाँ प्रवेशद्वार र कोठाको भित्री भाग बीचको भुइँको तहमा लगभग २ मिटरको भिन्नता छ”} \cite{71}।

\begin{figure}[b]
\begin{center}
% \fbox{\rule{0pt}{2in} \rule{0.9\linewidth}{0pt}}
   \includegraphics[width=1\linewidth]{dolmen.jpg}
\end{center}
   \caption{डोल्मेन दे सोटो, स्पेन \cite{53}.}
\label{fig:9}
\label{fig:onecol}
\end{figure}

भित्र मानव अवशेषहरूको अभाव पनि एक जिज्ञासु पक्ष हो। उत्खनन गर्दा बाटोभरि छरिएका जलेका र नजलेका हड्डीका टुक्राहरू फेला परे जसले सानो संख्याको मानिसको प्रतिनिधित्व गर्छ। भित्रका सामग्रीहरूको कार्बन मितिका आधारमा न्यूग्रेन्जको निर्माण कम्तिमा केही पुस्ताहरू लागेको अनुमान गरिएको छ। किन एउटा प्राचीन समुदायले यत्रो प्रयास गरेर विशाल, अत्यन्तै इन्जिनियरिङ्ग गरिएको स्थल निर्माण ग¥यो भने पनि केवल केही मृतकका हड्डीका टुक्राहरू मार्गमा छरिन थियो? यो भन्दा धेरै विश्वसनीय कुरा के हो भने यी प्राचीन र ध्यानपूर्वक जलरोधक बनाइएका मेगालिथिक संरचनाहरू मानिसहरूलाई पृथ्वीका बारम्बार हुने प्रलयहरूबाट जोगाउन आश्रय का लागि निर्माण गरिएका थिए।
दक्षिण स्पेनको हुवेल्भा शहरमा, यसकाे एक समान उदाहरण डाेल्मन दे साेटाे (Figure \ref{fig:9}) हाे, जुन यहाँकाे करिब २०० मध्येकाे एउटा स्थल हाे \cite{72,32}। याे सरल र व्यवस्थित, अत्यन्त प्रविधियुक्त संरचना मेगालिथिक ढुङ्गाहरू प्रयोग गरेर निर्माण गरिएकाे छ र यसको व्यास ७५ मिटर छ। खननका क्रममा केवल आठ वटा शवहरू पाइएकाे बताइन्छ, सबैलाई गर्भस्थ अावस्थामा गाडिएकाे थियो।

\section{विशेष उल्लेखनीय अनियमितताहरू}

यस खण्डमा, म केही थप उल्लेखनीय अनियमितताहरू संक्षिप्त रूपमा उल्लेख गर्छु, जुन सबैलाई ECDO-जस्ता प्रलयद्वारा राम्ररी व्याख्या गर्न सकिन्छ।

\subsection{जैविक अनियमितताहरू}

\begin{figure}[b]
\begin{center}
% \fbox{\rule{0pt}{2in} \rule{0.9\linewidth}{0pt}}
   \includegraphics[width=1\linewidth]{bottleneck.jpg}
\end{center}
   \caption{६,००० वर्षअघि करिब ९५\% पुरुषहरूको जनसंख्यामा देखिएको आनुवंशिक सङ्कीर्णताको प्रतिनिधित्व गर्ने एक बोटलनेक \cite{62}।}
\label{fig:10}
\label{fig:onecol}
\end{figure}

केही उल्लेखनीय जैविक असमानताहरूमा आनुवंशिक सङ्कीर्णता र भूपरिवेशी ह्वेल जीवाश्महरू छन्। झेङ्ग आदिले (२०१८) आधुनिक मानिसको १२५ वाई-क्रोमोसोम अनुक्रमहरूको मोडेलिङ्ग गरे, र डिएनएको समानता तथा उत्परिवर्तनहरूका आधारमा, करिब ५,००० देखि ७,००० वर्षअगाडि पुरुष जनसंख्यामा ९५\% ले घटेको बोटलनेक पत्ता लगाए (चित्र \ref{fig:10}) \cite{62}। ह्वेलका जीवाश्महरू समुद्री सतहभन्दा सयौँ मिटर माथि, स्वीडेनबर्ग, मिशिगन, भर्मन्ट, क्यानडा, चिली, र इजिप्टमा फेला परेका छन् \cite{63,64,65,66}। यी ह्वेलहरू फरक-फरक अवस्थामा भेटिएका थिए: पूर्ण रूपमा संरक्षित, दलदलमा हिमनदीको भण्डारभन्दा माथि पसेका, वा बालुवामा गाडिएका अवस्थामा। यी स्थानहरूमा नमुनाहरूको सङ्ख्या केहीदेखि एक सयभन्दा बढी सम्म छ। ह्वेलहरू गहिरो समुद्री जीव हुन् र प्रायः किनारामा आउँदैनन्। यी ह्वेलहरू कसरी यति उच्च स्थानहरूमा, प्रायः अत्यधिक टाढा भूपरिवेशमा, पुग्न पुगे?
पृथ्वीको विगतमा धेरै पटक सामूहिक विलुप्तिहरू  घटेका छन्, जसमा सबैभन्दा विस्तृत रूपमा अध्ययन गरिएका "ठूला पाँच" फ्यानेरोजोइक घटनाहरू हुन्: पछिल्लो ओर्डोभिसियन (LOME), पछिल्लो डेवोनियन (LDME), अन्त-परमियन (EPME), अन्त-ट्राइसिक (ETME) र अन्त-क्रिटेशियस (ECME)(यी सबै भूवैज्ञानिक इतिहासका पाँचवटा प्रमुख विलुप्ती घटनाहरू हुन्।) सामूहिक विलुप्तिहरू \cite{88,89}। रोचक रूपमा, यी विलुप्तिहरूमध्ये धेरै त्यही ऐतिहासिक अवधिमा वर्गीकृत गरिएका छन् जसमा ग्राण्ड क्यान्यनका धेरै तहहरू पर्दछन्, नामसहित परमियन र डेवोनियन तहहरू।

\subsection{भौतिक अनियमितताहरू}

\begin{figure}[b]
\begin{center}
% \fbox{\rule{0pt}{2in} \rule{0.9\linewidth}{0pt}}
   \includegraphics[width=1\linewidth]{columbia.jpg}
\end{center}
   \caption{ग्लेशियल लेक कोलम्बिया, वाशिंगटन राज्यमा अत्यधिक ठूलो धाराको तरंगहरू \cite{80}.}
\label{fig:11}
\label{fig:onecol}
\end{figure}

ग्रान्ड क्यान्यन बाहेक धेरै भू–दृश्यहरू छन् जुन सम्भवतः विनाशकारी शक्तिहरूले बनेका हुन्। विश्वभरका विशाल धाराका लहरहरूमा विशाल महादेशीय पानी प्रवाहको प्रमाण पाउन सकिन्छ।। यस्तो एउटा उदाहरण प्यासिफिक नर्थवेस्टमा रहेको च्यानल्ड स्क्याबल्याण्ड्स हो। यहाँ, हामी तलछट निक्षेप परिदृश्य र अनियमित ढुङ्गाहरू मात्र देख्दैनौं, तर विशाल समुद्री प्रवाहबाट बनेको सय भन्दा बढी ठूला तरंगहरूको अनुक्रम पनि देख्छौं \cite{78,79}। यी खोलाहरूको बालुवाको तहमा बनेका लहरहरूको ठूला-ठूला संस्करणहरू हुन्।। यी फ्रान्स, अर्जेन्टिना, रूस, र उत्तरी अमेरिकाजस्ता संसारभरका स्थानहरूमा पाइन्छन् \cite{81}। तस्वीर \ref{fig:11} मा संयुक्त राज्य अमेरिकाको वाशिंगटन राज्यमा देखिएका केही यस्ता तरंगहरूको चित्रण छ \cite{80}।

\begin{figure}[b]
\begin{center}
% \fbox{\rule{0pt}{2in} \rule{0.9\linewidth}{0pt}}

   \includegraphics[width=1\linewidth]{zhangjiajie.jpg}
\end{center}
   \caption{चिनको दक्षिणी भागमा रहेको झाङ्जीयाजिए नेसनल फरेस्टका ठूला शिला स्तम्भहरू।}
\label{fig:12}
\label{fig:onecol}
\end{figure}

\begin{figure}[b]
\begin{center}
% \fbox{\rule{0pt}{2in} \rule{0.9\linewidth}{0pt}}

   \includegraphics[width=1\linewidth]{hoy.jpg}
\end{center}
   \caption{हॉयको पुरानो मान्छे समुद्री स्तम्भ, स्कटल्यान्ड \cite{83}.}
\label{fig:13}
\label{fig:onecol}
\end{figure}

भू-भागीय कटाव संरचनाहरू पनि ECDO-जस्तो पृथ्वी उल्टोफेरले राम्रोसँग व्याख्या गर्न सकिन्छ। दक्षिणी चीन पानीको कटावबाट बनेको विशाल कार्स्ट परिदृश्यको उत्कृष्ट उदाहरण हो \cite{82}। यी दृश्यहरूमा टावर कार्स्ट, पिनाकल कार्स्ट, कोन कार्स्ट, प्राकृतिक पुलहरू, खोँचहरू, ठूला गुफा प्रणालीहरू, र भासिने खाल्डाहरू समावेश छन्। यीमध्ये सबैभन्दा उल्लेखनीय झाङ्जियाजिए राष्ट्रिय वन हो, जसमा विशाल क्वार्ट्जयुक्त बालुवापत्थर स्तम्भहरू (चित्र \ref{fig:12}) छन् \cite{84}। यी स्तम्भहरू औसत १,००० मिटर भन्दा माथि उचाइमा उभिएका छन्, र तिनीहरूको संख्या ३,१०० भन्दा बढी छ। तीमध्ये १,००० भन्दा बढी स्तम्भहरूको उचाइ १२० मिटरभन्दा माथि छ, र ४५ वटा ३०० मिटरभन्दा अग्ला छन् \cite{85}। यी स्तम्भहरू समुद्री अपरदन स्तम्भहरू (चित्र \ref{fig:13}) जस्तै देखिन्छन्, जुन तटीय चट्टान स्तम्भहरू हुन्, जसका वरिपरिको पदार्थ समुद्री छालले भत्काएर बनाएका हुन्छन्। यस्तै अपरदन दृश्यहरू तुर्कीको उरगुपका चट्टान शिखरहरूमा, साथै स्पेनको सिउडाड इनकान्तादामा पनि भेटिन्छन्, जुन दुबै स्थान समुद्र सतहबाट १,००० मिटरभन्दा माथि छन्। यी सबै स्थानहरूमा नजिकै नै नुन र समुद्री जीवाश्महरू केही न केही रूपमा भेटिन्छन्, जसले विगतमा समुद्री पानी पसेको संकेत गर्छ \cite{15,86,87}। अवश्य पनि, बाढीका कथाहरू \cite{3} मा समुद्र १,००० मिटरभन्दा धेरै माथि गएकाबारे लेखिएको छ, र यो कुरालाई एन्डीज र हिमालयका समुद्र सतहभन्दा धेरै माथि रहेका नुनिलो पानी र विशाल नुन मैदानले प्रमाणित गर्छ। उदाहरणका लागि, बोलिभियाको उयुनी नुन मैदान समुद्र सतहबाट ३६५३ मिटर उचाइमा छ \cite{94}।

\subsection{छिटो जलवायु परिवर्तन घटना}
आधुनिक वैज्ञानिक साहित्यले पृथ्वीको हालैको इतिहासमा तीव्र विश्वव्यापी जलवायु परिवर्तन घटनाहरूको अस्तित्वलाई मान्यता दिन्छ। दुई उल्लेखनीय उदाहरणहरू ४.२ किलोवर्ष र ८.२ किलोवर्ष घटनाहरू हुन्, दुवै घटनाहरु जनसंख्या घटावट र ठूलो भौगोलिक क्षेत्रमा समाजिक बसोबासको अवरोधसँग मेल खान्छन्। यी घटनाहरु अवसाद र बरफको कोर, जीवाश्म कोरल, O18 आइसोटोप मान, परागकण र स्पिलियोथेम अभिलेख, तथा समुद्री सतहका डेटामा विसंगतिहरूको रूपमा संरक्षित भएको पाइन्छ। अनुमान गरिएको मौसम परिवर्तनहरूमा विश्वव्यापी तापक्रमको तिब्र गिरावट, मरुभूमीकरण, एटलान्टिक मेरिडियनल ओभरटर्निङ करेन्टको विघटन, र हिमनदीको विस्तार समावेश छ \cite{90,91,92}। विशेष गरी ८.२ किलोवर्षको घटना ६४०० ई.पू.  तिर कालो सागरमा सम्भावित उल्लेखनीय नुनिलो पानीको बाढीसँग मेल खान्छ \cite{93}।

\subsection{पुरातात्विक अनियमितताहरू}

केही प्राचीन शहरहरूको पुरातात्विक प्रमाणहरूले गाडिने र विनाश हुने बहुपरतीय तहहरू देखाउँछ, जसले विगतका विपत्तिपूर्ण घटनाहरूको अभिलेख सिर्जना गरेको छ। ऐतिहासिक शहर जेरीको यस्तै एक शहर हो, जसको स्थान हालको प्यालेस्टाइनमा छ। यसमा ध्वस्त पत्थरका संरचनाहरू र तीव्र आगोको साथै, धेरै विनाशका तहहरू छन् \cite{96,97}। यसको तहहरूमा रेकर्ड गरिएको कालक्रम लगभग ९००० ई.पू. देखि २००० ई.पू. सम्मको छ। विशेष उल्लेखनीय यसको मिनार हो, जुन झण्डै ७४०० ई.पू. मा तलबाट काटिएको जस्तो देखिन्छ र अवसादमा गाडिएको छ (चित्र \ref{fig:14}) \cite{95}। चाल हात्युक \cite{99}, ग्रामालोट \cite{98}, र क्रेटको मिनोअन दरबार नोसोस \cite{100,101} सबै यस्तै प्रकृतिका पुरातात्विक स्थलहरू हुन् जसमा बहुपरतीय तह पाइन्छन्, प्रायः विनाशको प्रमाणसमेत समावेश छ।

\begin{figure}[t]
\begin{center}
% \fbox{\rule{0pt}{2in} \rule{0.9\linewidth}{0pt}}

\includegraphics[width=1\linewidth]{jericho.jpg}
\end{center}
   \caption{इरीहोको स्तम्भको अन्त्येष्टि प्रक्रिया, करिब ७४०० ई.पू. को पुरातात्त्विक पुनर्निर्माण \cite{95}.}
\label{fig:14}
\label{fig:onecol}
\end{figure}

मानव सभ्यतालाई अवरोध गर्ने ठूला प्रलयहरूको अर्को प्रमाण नम्पा इमेज हो, जुन एक माटोको पुतली हो, इडाहोमा करिब १०० मिटर जति लावाको मुनि फेला परेको थियो \cite{102,103}। पुतली फेला परेका लावा प्रवाहलाई तृतीय युग वा प्रारम्भिक चतुर्थ युगमा जम्मा भएको अनुमान गरिएको थियो, जुन करिब २ मिलियन वर्ष पुरानो भनिन्छ। तर, त्यो क्षेत्रको लावा प्रवाह अपेक्षाकृत ताजै देखिन्छ। यस्ता फेला परेका कुराहरुले न केवल सभ्यता विनाश गर्ने ठूला प्रकोपहरूलाई संकेत गर्छन्, तर आधुनिक मिति निर्धारणका कालक्रमहरूमा पनि प्रश्न उठाउँछन्।

\section{आधुनिक मिति निर्धारण विधिहरूका बारेमा}

आधुनिक कालक्रमहरूमा शंका गर्नुपर्ने महत्त्वपूर्ण कारण छ, जसले विभिन्न भौतिक वस्तुहरूलाई लाखौं, वा सयौं लाखौं वर्षको असाध्यै लामो उमेर निर्धारण गर्छ।

परम्परागत कथनले भन्छ कि कोइला, पेट्रोलियम, र प्राकृतिक ग्यास जस्ता भनिने "जीवाश्म इन्धन" सयौं लाखौं वर्ष पुरानो हुन् \cite{104}। तर, मेक्सिकोको खाडीमा तेलको वास्तविक कार्बन डेटिङले तेलको उमेर लगभग १३,००० वर्ष पाएको छ \cite{105}। कार्बन-१४ को आधा जीवन (५,७३० वर्ष) यति छोटो छ कि केही लाख वर्षपछि पूरै नष्ट भइसकेको हुनुपर्थ्यो। तर, यो कोइला र जीवाश्‍महरूमा भेटिएको छ जुन भनिन्छ कि हजारौं गुणा पुरानो छन् \cite{106}। वास्तवमा, प्रयोगशालामा नियन्त्रित अवस्था (मुख्यतया बढी ताप) मा कृत्रिम कोइला बनाइएको छ, सिर्फ़ २-८ महिनाभित्रै \cite{107}।

कार्बन डेटिङ बाहेक अन्य रेडियोआइसोटोप डेटिङ विधिहरू पनि सटीक नहुन सक्छन्। एन्सर्स इन जेनेसिस अनुसन्धान समूहले यस्ता विधिहरूबाट प्राप्त मितिहरूमा असमानता भेट्टायो जसले तिनीहरूको सत्यता माथि प्रश्न उठाउँछ \cite{108}। रक्त कोशिका, रक्तनली, र कोलाजेन रहेको नरम तन्तु डाइनोसरका अवशेषमा पनि भेटिएको छ जुन भनिन्छ कि सयौं लाख वर्ष पुरानो हुन् \cite{109,110}। हामीलाई थाहा भएको आधारमा, पृथ्वीको भूगोल सम्बन्धी कालक्रम तथा ढुंगा र जीवाश्म इन्धन जस्ता भौतिक वस्तुहरूको परम्परागत उमेर धेरै गुणा फरक पर्न सक्ने सम्भावना छ।

\section{निष्कर्ष}

यस लेखमा, मैले विनाशकारी उत्पत्तिहरू संकेत गर्ने सबैभन्दा प्रभावशाली विसंगतिहरू समेटेको छु जसलाई ECDO पृथ्वी उल्टोफेरले सबैभन्दा राम्रोसँग व्याख्या गर्न सक्छ। यद्यपि प्रस्तुत संग्रह विविध छ, यो अपूर्ण छ - अझ धेरै विसंगतिहरू संकलन गरिएको छ र मेरो अनुसन्धान GitHub रिपोजिटोरीमा सार्वजनिक रूपमा उपलब्ध छन् \cite{2}।
\section{धन्यवाद}

मूल ECDO सिद्धान्तका लेखक इथिकल स्केप्टिक लाई, उहाँको सजीव, नविनतम् थेसिस सम्पन्न गरी संसारसंग साझा गर्नुभएकोमा धन्यवाद। उहाँको त्रि-भागीय थेसिस \cite{1} उत्सर्जनात्मक कोर-म्यान्टल छुट्टिने कम्पन (ECDO) सिद्धान्तको लागि अधिकारिक कृति रहिरहन्छ, र त्यहाँ संक्षेपमा मैले यहाँ संकलन गरेको भन्दा यो विषयमा अझ धेरै जानकारी समेटिएको छ।

र, अवश्य पनि, ती महान् व्यक्तिहरूलाई धन्यवाद जसको काँधमा हामी उभिएका छौं; जसले सम्पूर्ण अनुसन्धान र अन्वेषण गरेका छन्, जसले यो कार्य सम्भव बनाएका छन् र मानवजातिलाई उज्यालो ल्याउन मेहनत गरेका छन्।

\clearpage
\twocolumn
{\small
\bibliographystyle{ieee}
\bibliography{egbib}
}

\end{document}
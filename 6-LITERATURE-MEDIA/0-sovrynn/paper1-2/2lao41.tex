\documentclass[10pt,twocolumn,letterpaper]{article}

% ຂອງຂ້ອຍເອງ
\usepackage{booktabs}
% \usepackage{caption}
% \captionsetup[table]{skip=8pt}   % ເບິ່ງແຕ່ຕາຕະລາງ
\usepackage{stfloats}  % ເພີ່ມນີ້ໄປທີ່ preamble

% \usepackage{fontspec}
\usepackage[english]{babel}

% load Lao via babelprovide, turn on "onchar=ids" for automatic shaping
\babelprovide[import,onchar=ids fonts]{lao}

% main (rm) font for Latin
\babelfont{rm}{Noto Serif}

% Lao text in Noto Serif Lao at 1.2× scale
\babelfont[lao]{rm}{Noto Serif Lao}
\babelfont[lao]{sf}{Noto Serif Lao}

% alternate (sans-serif) font for Latin
\babelfont{alt}{Lato}

% Lao text in Noto Serif Lao for the alt family too
\babelfont[lao]{alt}{Noto Serif Lao}

% GPT: LuaTeX doesn’t have built-in Lao line-breaking rules, but babel can assimilate line breaks to hyphenation if you supply some simple “patterns” for common syllable boundaries. Add this to your preamble:
\babelpatterns[lao]{%
  1ດ 1ມ 1ອ 1ງ 1ກ 1າ 1ັ 1ິ 1ີ 1້ 1ົ 1ູ%
}

\usepackage{cvpr}
\usepackage{times}
\usepackage{epsfig}
\usepackage{graphicx}
\usepackage{amsmath}
\usepackage{amssymb}

% ເພີ່ມ package ອື່ນໆ ຢູ່ນີ້, ກ່ອນ hyperref.

% ຖ້າທ່ານຄວາມຄິດເຫັນ hyperref ແລະຫຼັງຈາກນັ້ນຍົກເລີກ, ທ່ານຄວນລຶບ
% egpaper.aux ກ່ອນທີ່ຈະໃຊ້ latex ອີກຄັ້ງ.  (ຫຼືກໍ່ແຕະ 'q' ໃນການຮັນ latex ຄັ້ງທຳອິດ,
% ໃຫ້ມັນສຳເລັດ, ແລະທ່ານຈະພ້ອມໃຊ້).
\usepackage[breaklinks=true,bookmarks=false]{hyperref}

\makeatletter
\def\cvprsubsection{\@startsection {subsection}{2}{\z@}
    {8pt plus 2pt minus 2pt}{6pt}{\bfseries\normalsize}}
\makeatother

\cvprfinalcopy % *** Uncomment this line for the final submission

\def\cvprPaperID{****} % *** Enter the CVPR Paper ID here
\def\httilde{\mbox{\tt\raisebox{-.5ex}{\symbol{126}}}}

% ໜ້າຖືກນັບໃນໂໝດສົ່ງ, ແລະບໍ່ນັບໃນ camera-ready
%\ifcvprfinal\pagestyle{empty}\fi
\setcounter{page}{1}
\begin{document}

%%%%%%%%% TITLE
\title{ECDO ພາກນຳແບບ Data-Driven ສ່ວນທີ 2/2: ການສືບຫາຜົນປົກກະຕິທາງວິທະຍາສາດ ແລະປະຫວັດສາດ ທີ່ອະທິບາຍໄດ້ດີທີ່ສຸດໂດຍ “Earth Flip” ຂອງ ECDO}

\author{Junho\\
ເຜີຍແຜ່ ກຸມພາ 2025\\
ເວັບໄຊ (ດາວໂຫລດບົດຄົ້ນຄວ້າທີ່ນີ້): \href{https://sovrynn.github.io}{sovrynn.github.io}\\
ຄັງຄົ້ນຄວ້າ ECDO: \href{https://github.com/sovrynn/ecdo}{github.com/sovrynn/ecdo}\\
{\tt\small junhobtc@proton.me}
}

\maketitle
%\thispagestyle{empty}

\begin{abstract}
ໃນເດືອນພຶດສະພາ 2024, ນັກຂຽນອອນໄລນ໌ນາມສົມມຸດວ່າ “The Ethical Skeptic” \cite{0} ໄດ້ເຜີຍແຜ່ທິດສະດີໃໝ່ທີ່ມີອິດທິພົນສູງ ເຊື່ອມໂຍງ Exothermic Core-Mantle Decoupling Dzhanibekov Oscillation (ECDO) \cite{1}. ທິດສະດີນີ້ບໍ່ພຽງແຕ່ເສະຫຼຸບວ່າ ໂລກເຄີຍມີການປ່ຽນແປງແກນຫມຸນຢ່າງກະທັນຫັນ ເຮັດໃຫ້ເກີດນ້ຳຖ້ວມໃຫຍ່ໂລກໂດຍຫມຸນອິນເທີຍຂອງໂລກເຮັດໃຫ້ທະເລອອກໄປທົ່ວພື້ນດິນ, ແຕ່ຍັງອະທິບາຍກະບວນການທາງພູມສາດທີ່ເປັນເຫດຜົນ ແລະມີຂໍ້ມູນຢືນຢັນວ່າ ການກັບດ້ານນີ້ອາດຈະເກີດຂຶ້ນອີກໃນໄວໆນີ້. ເຖິງແມ່ນວ່າການທຳນາຍນ້ຳຖ້ວມໃຫຍ່ຫຼືວັນປິນດົງແບບນີ້ຈະບໍ່ໃໝ່, ແຕ່ທິດສະດີ ECDO ນີ້ມີຈຸດເດັ່ນແບບເຫັນໄດ້ຊັດ ເນື່ອງຈາກການຄົ້ນຄວ້າທາງວິທະຍາສາດ ຫຼາຍສາຂາ ແລະອີງໃສ້ຂໍ້ມູນທີ່ມີຢູ່ໃນປັດຈຸບັນ.

ບົດຄົ້ນຄວ້ານີ້ແມ່ນສ່ວນທີ່ສອງຂອງບົດສະຫຼຸດສັ້ນສອງສ່ວນຈາກການຄົ້ນຄວ້າອິດຕະຫຼາດເອົາເອງ 6 ເດືອນ \cite{2,20} ກ່ຽວກັບທິດສະດີ ECDO, ໂດຍເນັ້ນໄປທີ່ສ່ວນຜິດປົກກະຕິທາງວິທະຍາສາດ ແລະປະຫວັດສາດ ທີ່ອະທິບາຍໄດ້ດີທີ່ສຸດໂດຍອິດທິພົນຂອງ "ການກັບດ້ານໂລກ" ແບບ ECDO.

\end{abstract}

\section{ການແນະນຳ}

ວິຊາພູມສາດແບບສະໝຳເສມໍພັນເປັນສະເລີຍ ແລະປະຫວັດສາດສະໄມ ອ້າງວ່າ ພູມປູນການທີ່ສຳຄັນເຊັ່ນ Grand Canyon ໄດ້ຖືກສ້າງຂຶ້ນນານຫຼາຍລ້ານປີ \cite{143}; ແລະເກືອທີ່ມີໃນ Death Valley (California) ເພາະເຄີຍຢູ່ໃຕ້ທະເລມາຫຼາຍຮ້ອຍລ້ານປີກ່ອນ \cite{144}; ບັນພະບູລຸດຂອງພວກເຮົາ 150 ຊົ່ວໂຄງຈົນຜ່ານດ້ວຍການສ້າງສຸສານຍັກໆຕະຫລອດຊີວິດ \cite{29,70}; ແລະນ້ຳມັນເຊື້ອໄຟໂບຮານທີ່ເອີ້ນກັນມີອາຍຸຫຼາຍຮ້ອຍລ້ານປີ \cite{104}. ບາງຢ່າງທີ່ນ້າສົນໃຈຄື ມະນຸດຖືກເຊື່ອວ່າມີອາຍຸ 300,000 ປີ \cite{145}, ແຕ່ປະຫວັດສາດແລະອາລະຍະທຳທີ່ບັນທຶກໄດ້ມີແມ່ນພຽງ 5,000 ປີ - ຄືກັນກັບຮຸ່ນມະນຸດ 150 ຊົ່ວໂຄງ.

ສິ່ງຜິດປົກກະຕິແບບນີ້, ເຊິ່ງພວກເຮົາຈະເຫັນ, ຖືກອະທິບາຍໄດ້ດີທີ່ສຸດໂດຍອິດທິພົນດ້ານພູມສາດອັນລຸນແຮງ.

\section{ຊ້າງແມມມອດແຊ່ແຂງທີ່ຖືກຝັງໃນດິນຟົນ}

\begin{figure}[t]
\begin{center}
% \fbox{\rule{0pt}{2in} \rule{0.9\linewidth}{0pt}}
   \includegraphics[width=1\linewidth]{jarkov-mammoth.jpg}
\end{center}
   \caption{ມາໂມດ Jarkov, ອາຍຸ 20,000 ປີ ທີ່ຖືກຮັກສາແບບສົມບູນຢ່າງຫນາວແຂງ ໃນດິນຂີ້ເຫື້ອນເຢັນແຂງ ໃນຊິເບີເຣຍ \cite{51}.}
\label{fig:1}
\label{fig:onecol}
\end{figure}

ໝວດໜຶ່ງຂອງຄວາມພິລິດທີ່ພົບໄດ້ກໍແມ່ນ ມາໂມດທີ່ຖືກຮັກສາໄວ້ຢ່າງສົມບູນແລະສົມບູນໃນສະພາບແຂງໃນດິນຂີ້ເຫື້ອນ, ມັກພົບໃນພື້ນທີ່ອາຄຕິກ (ຮູບ \ref{fig:1}). ມາໂມດ Beresovka, ທີ່ຖືກພົບໃນຊິເບີເຣຍ ໃນດິນຫລົງທະຫານດິນຫົວສີ່ນ, ຖືກຮັກສາໄວ້ຢ່າງສົມບູນຈົນເນື້ອຂອງມັນຍັງກິນໄດ້ຫລາຍພັນປີຫລັງຈາກມັນຕາຍ. ມັນຍັງມີອາຫານພືດໃນປາກແລະທ້ອງ ທີ່ເຮັດໃຫ້ນັກວິທະຍາສາດສັບສົນວ່າເຮັດແນວໃດມັນຈຶ່ງຖືກຮັບເຢັນຢ່າງໄວຫາກວ່າມັນກຳລັງກິນພືດອອກດອກກ່ອນຈະຕາຍ \cite{17}. ມີລາຍງານວ່າ, \textit{"ໃນປີ 1901 ໄດ້ເກີດການຮືອຮື ຈາກການພົບຊິ້ນສ່ວນມາໂມດສົມບູນໃກ້ແມ່ນທົ່ວໃນແມ່ນໍ້າ Berezovka, ເພາະສັດນີ້ເຫັນໄດ້ວ່າຕາຍຈາກຄວາມເຢັນໃນກາງລະດູຮ້ອນ. ສິ່ງທີ່ຢູ່ໃນທ້ອງຂອງມັນຖືກຮັກສາໄວ້ແບບດີ ແລະມີທັງດອກບຸດເດືອນ ແລະໝາກເຖົ່າປ່າ: ນີ້ໝາຍຄວາມວ່າມັນຕ້ອງຖືກກິນໃນປາຍເດືອນກໍລະກົດ ຫຼືເຕັ້ນເດືອນສິງຫາ. ສັດນີ້ຕາຍຢ່າງພະຍາບາດຈົນມັນຍັງຄາຢູ່ໃນປາກດ້ວຍຫຍ້າແລະດອກໄມ້. ມັນຖືກຊັກຂຶ້ນໂດຍພະລັງແຮງສຸດອະນຸສົດແລະຖືກໂຍນອອກຈາກພື້ນຫນ້າລົມເປັນລະດັບກິ່ງກິ່ງ. ກະດູກເບັນແລະຂາຂຶ້ນສົດແລະມີການຊຳລະ – ສັດຕົວໃຫຍ່ນີ້ຖືກຊົກເຕະຈົນລົງຄຸກແລະຕາຍເຢັນ, ໃນເວລາທີ່ເປັນຊ່ວງຮ້ອນທີ່ສຸດຂອງປີ"} \cite{18}. ຍັງມີ, \textit{"[ນັກວິທະຍາສາດລັດເຊຍ] ໄດ້ບັນທຶກວ່າ ກະທັ້ງຊັ້ນດ້ານໃນສຸດຂອງທ້ອງມັນກໍຖືກຮັກສາແບບເສ້ນໃຍຫຸ້ມສົມບູນ, ເຮັດໃຫ້ເຫັນວ່າອຸນຫະພູມໃນຕົວຖືກດຶງອອກດ້ວຍການທີ່ນ່າມີວິທີການຢ່າງຫນ້າທຶງໃນທໍາມະຊາດ. Sanderson, ພົບວ່າເລື່ອງນີ້ມີຈຸດຫນຶ່ງໃນໃຈ, ໄດ້ນໍາຫນ້າບັນຫາໄປຫາສະຖາບັນອາຫານແຂງແຫ້ງແຫ້ງສະຫະລັດ: ມັນຕ້ອງເຮັດແນວໃດເພື່ອເຮັດໃຫ້ມາໂມດດັ້ງເກົ່າແຂງຢ່າງພ້ອມກັນຈົນມີນ້ຳໃນສ່ວນທີ່ເຂົ້າເຖິງຢ່າງເຕັມທີ່, ແມ້ແຕ່ຊັ້ນໃນສຸດຂອງທ້ອງຂອງມັນ, ບໍ່ມີເວລາເພີ່ມຕົວເພື່ອໃຫ້ເກີດຫຼັກຜົນໃຫຍ່ພໍທີ່ຈະທໍາລາຍເນື້ອ... ຜ່ານໄປຫຼາຍອາທິດ, ສະຖາບັນໄດ້ຕອບກັບມາຫາ Sanderson ດ້ວຍຄໍາຕອບ: ມັນເປັນໄປບໍ່ໄດ້ເລີຍ. ດ້ວຍຄວາມຮູ້ທາງວິທະຍາສາດ ແລະວິສະວະກຳເຫຼົ່ານັ້ນ, ບໍ່ມີວິທີໃດທີ່ຮູ້ຊັດເພື່ອດຶງຄວາມຮ້ອນອອກຈາກສັດສົກຕົວໃຫຍ່ເຫມືອນມາໂມດຢ່າງໄວຈົນບໍ່ເຮັດໃຫ້ເกີດຫຼັກຜົນໃຫຍ່ໃນເນື້ອ. ຜົນອີກດ້ວຍ, ຫລັງຈາກເລີກທຸກວິທີວິທະຍາສາດແລະວິສະວະກຳ, ເຂົາໄດ້ມອງໄປຫາທໍາມະຊາດແລະສຸດທ້າຍກໍສົບຜົນວ່າບໍ່ມີຂະບວນການໃດໃນທໍາມະຊາດທີ່ສາມາດເຮັດໄດ້ເຊຊ່ນນັ້ນ"} \cite{19}.

\section{ແກ່ງ Grand Canyon}

ແກ່ງ Grand Canyon, ເປັນສ່ວນໜຶ່ງຂອງພື້ນທີ່ Great Basin ທາງພາກໃຕ້ຕະວັນຕົກຂອງອາເມລິກາເຫນືອ, ແມ່ນປະກົດການທຳມະຊາດອື່ນໜຶ່ງທີ່ຊີ້ໃຫ້ເຫັນເຖິງຕົ້ນກຳເນີດທີ່ເຫດການຮ້າຍແຮງ (ຮູບ \ref{fig:2}). ເລີ່ມຕົ້ນ, ຊັ້ນຫິນຊາຍແລະຫິນປູນຕະກະນິດທີ່ປະກອບກັນເປັນແກ່ງ Grand Canyon ແຜ່ກວ້າງຫຼາຍພື້ນທີ່ສູງສຸດເຖິງ 2.4 ລ້ານ km$^2$ \cite{21}. ຮູບ \ref{fig:3} ຫຼັງເຫັນຂອບເຂດຂອງຊັ້ນ Coconino Sandstone ຫຼາຍທາງຕະວັນຕົກຂອງສະຫະລັດ. ຊັ້ນທີ່ມີຂະໜາດໃຫຍ່ ແລະເລີຍກັນເຫມືອນກັນໄດ້ອາດຈະຖືກຖົມພ້ອມກັນໃນເທື່ອດຽວ.

\begin{figure}[b]
\begin{center}
% \fbox{\rule{0pt}{2in} \rule{0.9\linewidth}{0pt}}
   \includegraphics[width=1\linewidth]{grand-canyon.jpg}

\end{center}
   \caption{ແກ່ງ Grand Canyon, ໃນລັດ Arizona, ສະຫະລັດອາເມລິກາ \cite{49}.}
\label{fig:2}
\label{fig:onecol}
\end{figure}

\begin{figure}[t]
\begin{center}
% \fbox{\rule{0pt}{2in} \rule{0.9\linewidth}{0pt}}
   \includegraphics[width=1\linewidth]{coconino.jpg}
\end{center}
   \caption{ຂະໜາດຂອງຊັ້ນຊິ້ນຫີນຊາຍ Coconino ໃນພາກຕາເວັນຕົກຂອງສະຫະລັດອາເມລິກາ \cite{21}.}
\label{fig:3}
\label{fig:onecol}
\end{figure}

ການພິຈາລະນາຢ່າງໃກ້ຊິດທີ່ແກ່ງ Grand Canyon ໄດ້ແຈ້ງໃຫ້ເຫັນວ່າການຕົກທັບຂອງຊັ້ນຕະກະກົນອັນກວ້າງໃຫຍ່ເຫົາຫົນເຫົາເກີດຂຶ້ນຄວບຄູ່ກັບກຳລັງເຕັກໂທນິກສຳຄັນ. ເພື່ອເຂົ້າໃຈເລື່ອງນີ້, ພວກເຮົາຈຳເປັນຕ້ອງພິຈາລະນາຢ່າງໃກ້ຊິດທີ່ບາງບ່ອນໃນແກ່ງ, ທີ່ຊັ້ນຕະກະກົນຖືກພັບແລະຖືກເຜີຍໃຫ້ເຫັນ. ນັກຄົ້ນຄ້ວາຈາກ Answers in Genesis \cite{42} ໄດ້ສຳຫຼວດຕົວຢ່າງຫີນດ້ວຍກັນສ່ອງກຸ້ມຂະໜາດໃນຈາກບາງສ່ວນຂອງພັບເຫຼົ່ານີ້, ເຊັ່ນ Monument Fold, ແລະອີງຕາມການຂາດຫາຍຂອງລັກສະນະທີ່ຄວນຈະມີຖ້າຫາກວ່າພັບເຫຼົ່ານີ້ເກີດຂຶ້ນໃນໄລຍະເວລາຍາວນານທ່າມກາງຄວາມຮ້ອນແລະແຮງດັນ, ກໍເລີຍສະຫຼຸບວ່າຊັ້ນຕະກະກົນໄດ້ຖືກພັບໂດຍກຳລັງເຕັກໂທນິກໃນຂະນະທີ່ມັນຍັງອ່ອນຢູ່, ນັ່ນຫມາຍເຖິງຫຼັງຈາກການຕົກທັບໄມ່ດົນ \cite{43}.

\begin{figure*}
\begin{center}
% \fbox{\rule{0pt}{2in} \rule{.9\linewidth}{0pt}}
\includegraphics[width=1\textwidth]{Grand_Staircase-big.jpg}
\end{center}
   \caption{ຊັ້ນຕະກະນິດທີ່ປະກອບເປັນ Grand Canyon (ດ້ານຂວາຂອງຮູບ) ແຜ່ນຕົງເໜືອໄປຫາ Cedar Breaks, Utah (ດ້ານຊ້າຍຂອງຮູບ), ທີ່ທັງໝົດບີບຂຶ້ນເທິງ \cite{50}.}
\label{fig:4}
\end{figure*}

ເມື່ອຂະຫຍາຍຄອບເຂດອອກ, ພວກເຮົາພົບວ່າຊັ້ນຕະກະນິດທີ່ເຮັດໃຫ້ເກິດ Grand Canyon ບໍ່ໄດ້ຖືກພັບພາຍໃນຫວ່າງຫວ່າ Grand Canyon ເທົ່ານັ້ນ. ຊັ້ນຕະກະນິດໄດ້ຖືກພັບຢູ່ທາງຕາເວັນອອກໃນ East Kaibab Monocline \cite{46}, ແຕ່ກໍຍັງຖືກພັບເທິງໃຕ້ເໜືອທີ່ Cedar Breaks, Utah (ຮູບ \ref{fig:4}). ນີ້ບົງບອກວ່າຊັ້ນຫຼ້ານີ້ອາດຈະຖືກພັບຣວມກັນພາຍຫຼັງຈາກທີ່ມັນຖືກວາງທັບກັນເອົາໄວໆ. ເພື່ອໃຫ້ເຫັນພາບ, ຊັ້ນຕະກະນິດແບບນອນນິງໃນ Grand Canyon ຫນາປະມານ 1700 ແມັດ. ຂະໜາດຂອງການເຄື່ອນໄຫວທາງທາງພູມສາດທີ່ຈຳເປັນໃນການວາງຊັ້ນຕະກະນິດໃຫ້ໜາເຖິງໃກ້ໜຶ່ງແມັດນັ້ນມີຂະໜາດມາຫາສານ.

ການກໍ່ເກີດຈິງໆຂອງ Grand Canyon ແມ່ນເລື່ອງທີ່ຫຼາຍຄົນຍັງບໍ່ແນ່ນອນໃນພູມສາດສະໄຫມ. ທິດສະດີ Uniformitarian ຂອງພູມສາດກ່າວວ່າ Grand Canyon ໄດ້ຖືກສະຫຼັກໂດຍແມ່ນ້ຳ Colorado ຕາມເວລາຫລາຍລ້ານປີ \cite{47}. ແຕ່ຄະນະວິຈັຍຂອງ Answers in Genesis ເຊື່ອວ່າ Grand Canyon ມັກຈະເກີດຂຶ້ນໃນໄລຍະໄມ່ຫຼາຍອາທິດເນື່ອງຈາກການກັດຊອກເທິງກໍລະສົກທີ່ນ້ຳມັກແຕກຫຼົດຂອງອ່າງນ້ຳໂບຮານ, ຊຶ່ງໄດ້ກຳຈັດຊັ້ນຕະກະນິດຈຳນວນມາກໃນຂະນະທີ່ກ່າຍຫວ່າງ Grand Canyon. ມີຫຼັກຐານຂອງອ່າງນ້ຳຄວາມສູງທາງຕາເວັນອອກຂອງ Grand Canyon ໃນຊັ້ນຕະກະນິດຂອງອ່າງນ້ຳແລະຊິ້ນສະເລີຍທະເລ. ການປຽບທຽບ Grand Canyon ກັບຕົວຢ່າງການກັດຊອກຂອງກໍລະສົກທີ່ມີຂະໜາດໃຫຍ່ອື່ນໆ, ເຊັ່ນ Afton Canyon ແລະ Mount St. Helens, ເຮັດໃຫ້ເຫັນລັກສະນະພື້ນທີ່ຄືກັນ, ແລະສະແດງໃຫ້ເຫັນວ່າຫວ່າງຫວ່າງທີ່ໃຫຍ່ສາມາດເກີດຂື້ນໄດ້ຢ່າງວ່ອງໄວ້ໂດຍນ້ຳໄຫຼຈຳນວນຫຼາຍ \cite{48}.

ເມື່ອພິຈາລະນາຂະໜາດຂອງກະບວນການທາງພູມສາດທີ່ຈຳເປັນໃນການວາງຊັ້ນຕະກະນິດໃນພື້ນທີ່ຫຼາຍຂະໜາດນີ້, ການບັງເກີດຂອງພະລັງງານທີ່ໃຫ້ຄວາມຄົນໃຫຍ່ຫຼັງຈາກການວາງຊັ້ນຕະກະນິດ, ແລະຂະໜາດຂອງແມ່ນ້ຳ Colorado ທີ່ໜ້ອຍຫຼືນອຍເທື່ອກັບຂອບເຂດ Grand Canyon, ເຮັດໃຫ້ເຫັນວ່າອາດບໍ່ມີຫຍັງທີ່ເກີດຂຶ້ນຢ່າງຊ້າໆໃນການສ້າງ Grand Canyon.

\section{ເມືອງພາຍໃຕ້ດິນ Derinkuyu}

ນອກຈາກພິຣາມິດ, ຕົວຢ່າງຍິ່ງໃຫຍ່ຂອງວິສະວະກຳໂບຮານແມ່ນເມືອງພາຍໃຕ້ດິນ Derinkuyu (ຮູບ \ref{fig:5}), ທີ່ຕັ້ງຢູ່ Cappadocia, ປະເທດຕູຣະກີ. ມັນແມ່ນເມືອງພາຍໃຕ້ດິນທີ່ໃຫຍ່ທີ່ສຸດໃນບັນດາເມືອງພາຍໃຕ້ດິນກວ່າ 200 ທີ່ຢູ່ໃນເຂດນັ້ນ \cite{54}. ເມືອງນີ້ຄາດວ່າເຄີຍຮອງຮັບຜູ້ອາໄສໄດ້ສູງສຸດຮອດ 20,000 ຄົນ, ແລະມີ 18 ຊັ້ນ, ມີຄວາມລຶກຮອດ 85 ແມັດ. ຖຶງແມ່ນອາຍຸຈະບໍ່ແນ່ນອນ, ແຕ່ຄາດວ່າມີອາຍຸຢ່າງຕໍ່າເຖິງ 2800 ປີ. ເມືອງນີ້ຖືກແກະອອກຈາກຫິນໄຟຄຸລອນທີ່ນຸ່ມ \cite{52, 53}.

\begin{figure}[b]
\begin{center}
% \fbox{\rule{0pt}{2in} \rule{0.9\linewidth}{0pt}}
\includegraphics[width=1\linewidth]{derinkuyu.jpeg}
\end{center}
   \caption{ແຜນຜັງຂອງເມືອງໃຕ້ດິນ Derinkuyu \cite{56}.}
\label{fig:5}
\label{fig:onecol}
\end{figure}

ເຫດຜົນທີ່ Derinkuyu ໜ້າສົນໃຈ ແມ່ນເນື່ອງຈາກມັນບໍ່ແນ່ນອນວ່າ ຊຸມຊົນໃດໜຶ່ງຈະຕັດສິນໃຈສ້າງເມືອງທັງໝົດໃຕ້ດິນ. ເພື່ອສ້າງພື້ນທີ່ຢູ່ອາໃສໃຕ້ດິນ, ທຸກຫ້ອງຕ້ອງໄດ້ຖືກແກະຈາກຫິນ. ຮູບລັກສະນະຫຍາບໆ ແລະ ພື້ນຜິວຂອງອໂມງທາງໃຕ້ດິນ ທຳໃຫ້ເຫັນໄດ້ຊັດເຈນວ່າ ເຊິ່ງຫຍັງໄດ້ຖືກແກະດ້ວຍແຮງງານຄົນ, ບໍ່ແມ່ນດ້ວຍເຄື່ອງມືໄຟຟ້າ, ຊຶ່ງຈະເປັນການຫຍາບກວ່າຫຼາຍເທົ່າໃນການສ້າງສະຫຼຸກເຫນືອພື້ນດິນ. ຢ່າງແທ້ຈິງ, ບໍ່ແນ່ນອນວ່າ ເປັນຫຍັງມະນຸດໃດໜຶ່ງຈຶ່ງຢາກຢູ່ໃຕ້ດິນຖາວອນໃນສະໄໝຊີວິດຂອງຕົນ, ໃນເມື່ອການປຸກພືດ, ແສງອາທິດ, ທຳມະຊາດ, ແລະ ການສຳຫຼວດ ມີຢູ່ຫຼັງພື້ນດິນພວກນັ້ນເທົ່ານັ້ນ. "ປະຫວັດສາດ" ທົ່ວໄປເສັ້ນເສີນວ່າ Derinkuyu ໄດ້ຖືກສ້າງໂດຍຄຣິດສະຕະຊົນທີ່ຕ້ອງການບ່ອນເຮັດພິທີສາສະໜາຢ່າງສົງົບ \cite{53}. ແຕ່ສຳລັບວິນຍານສາມັນສຳຫຼັບເຫດຜົນ ແມ່ນວ່າວິທີທີ່ງ່າຍທີ່ສຸດໃນການຈັດການກັບສັດຕູ ແມ່ນ "ສູ້ ຫລື ຫນີ", ບໍ່ແມ່ນ "ແກະເມືອງໃຕ້ດິນອອກຈາກຫິນ".

ຂະ ໜາ ດ, ຄວາມລຶກ, ແລະ ຄວາມໃສ່ໃຈດີໄຊໃນການອອກແບບເມືອງໃຕ້ດິນ ເຮັດໃຫ້ເຫັນວ່າມັນບໍ່ແມ່ນຖືກອອກແບບເປັນໂຄງສ້າງກອງທະຫານຊົ່ວຄາວເພື່ອສູ້ກັບຜູ້ບຸກຮານໃນຕອນໄພອັນຕະລາຍ, ແຕ່ເປັນບ່ອນເຫຼີງຊົ່ວໂມງຍາວເພື່ອປົກປ້ອງຈາກວິບັດສາດທີ່ຮ້າຍແຮງຢູ່ພື້ນດິນ. Derinkuyu ໄດ້ຕິດຕັ້ງບໍ່ສຳລັບຫ້ອງນອນພຶ້ນຖານ, ເຮືອນຄົນຄົວ, ແລະ ຫ້ອງນ້ຳເທົ່ານັ້ນ, ແຕ່ຍັງມີຄອບຄອງເຮືອນສັດ, ຖັງນ້ຳ, ບ່ອນເກັບອາຫານ, ບ່ອນບີບໄວນ໌ແລະນ້ຳມັນ, ໂຮງຮຽນ, ໂບດສວັດ, ຮຸບຝັງສົບ, ແລະ ຮູແກ່ອາກາດຈຳນວນຫຼາຍ (ຮູບທີ່ \ref{fig:6}). ເປັນຫຍັງບ່ອນຫຼົ່ນແຫ່ງການທະຫານຈະຕ້ອງການເຄື່ອງບີບໄວນ໌ແລະຕ້ອງແກະລຶກເຖິງ 85 ແມັດ ພ້ອມຄວາມຊັບຊ້ອນເຊັ່ນນີ້?

ຄຳອະທິບາຍທີ່ເປັນໄປໄດ້ຫຼາຍທີ່ສຸດສຳລັບການສ້າງ Derinkuyu ແມ່ນ ຄວາມຈຳເປັນຢ່າງເຫຼົ່າແຮງໃນການຕຽມບ່ອນຫຼົ່ນທີ່ຢູ່ອາໄສລະຍະຍາວ ແລະ ພື້ນທີ່ທີ່ສາມາດເລີ້ຍຊີວິດໄດ້ເພື່ອປົກປ້ອງຈາກກຳລັງພິເສດທາງທິບພິບິບຕິທີ່ຢູ່ໃນພື້ນຜິວໂລກ.

\begin{figure}[t]
\begin{center}
% \fbox{\rule{0pt}{2in} \rule{0.9\linewidth}{0pt}}
   \includegraphics[width=1\linewidth]{derinkuyu-air.jpg}
\end{center}
   \caption{ຮູບບໍ່ລົມລືກໃນ Derinkuyu \cite{53}.}
\label{fig:6}

\label{fig:onecol}
\end{figure}

% \section{ຄວາມຜິດປົກກະຕິເພີ່ມເຕີມທີ່ອະທິບາຍໄດ້ດີທີ່ສຸດໂດຍການກັບຄັ້ນໂລກ}

% ກ່ອນສິ້ນສຸດ, ພວກເຮົາຈະກ່າວເຖິງຄວາມຜິດປົກກະຕິທາງວິທະຍາສາດເພີ່ມເຕີມບາງຢ່າງ, ເມື່ອຕະຫຼອງໃນເນື້ອໃນຂອງພະລັງງານພິບັດທາງພູມິກະສາດ, ເຫຼົ່ານັ້ນອະທິບາຍໄດ້ດີ. 

\section{ສິ່ງມີຊີວິດແລະພືດສະສົມ}

ສ່ວນຜະສົມຂອງສິ່ງມີຊີວິດແລະພືດຫຼາກຫຼາຍ, ມັກພົບປະໃນຮູບແບບຟອດຊິວໃນຊັ້ນຕະກະ, ເປັນເຫັນອັນຫນຶ່ງທີ່ພິລຶກໃຈ. ໃນ “Reliquoæ Diluvianæ”, ພະອາຈານ William Buckland ໄດ້ລາຍງານການພົບເຫັນສັດຫຼາກຫຼາຍຊະນິດທີ່ບໍ່ມີເຫດຜົນອະທິບາຍໃດໆທີ່ພົບພາກັນ, ກະຈາຍຢູ່ທົ່ວອັງກິດແລະຢູໂຣບ, ຝັງໜ້າຢູ່ໃນຊັ້ນຄອນກະກະ 'diluvium' \cite{58}. ສ່ວນຜະສົມເຫຼົ່ານີ້ຂອງຊິ້ນສ່ວນສັດກໍພົບໃນຖ້ຳ Skjonghelleren ໃນເກາະ Valdroy ປະເທດນໍເວ. ໃນຖ້ຳນີ້, ພົບກະດູກສັດລາຍການ, ນົກ, ແລະປາ ກວ່າ 7,000 ຊິ້ນ ປະສົມແຖກຢູ່ໃນຊັ້ນຕະກະຫຼາຍຊັ້ນ \cite{59}. ຕົວຢ່າງອື່ນແມ່ນ San Ciro, “ຖ້ຳຂອງຍັກ”, ທີ່ອິຕາລີ. ໃນຖ້ຳນີ້, ມີກະດູກສັດລະຫຼາຍຕົກຄ້າງຮອບຖ້ຳ, ສ່ວນໃຫຍ່ແມ່ນຮິໂປໂປຕາມັດ, ຢູ່ໃນສະພາບສົດຫຼາຍ ຈົນໄດ້ນຳໄປຕັດເປັນເຄື່ອງປະດັບ ແລະສົ່ງອອກເພື່ອຜະລິດ lamp black. ກະດູກຂອງສັດນານາພັນຈົດກະຈາຍ, ເຫຼົ່ານັ້ນຖືກຜະຫຼາຍແບ່ງ, ແຕກ, ແລະກະຈາຍເປັນຊິ້ນໆ \cite{60,61}. ຢູ່ໃນມິນດີສອານິດ ທີ່ປະເທດອີຢິບ, ມີຫຼາຍປະເພດສັດທີ່ພົບປະອຍູ່ໃນດິນເຄົາທີ່ຖືກຮອງໃສ່ຄວາມຮ້ອນຈົນເຫັນເປັນແກ້ວ \cite{57}. ການຄົ້ນພົບເຫຼົ່ານີ້ອາດຈະນ້າສົນໃຈ, ແຕ່ອະທິບາຍໄດ້ງ່າຍດາຍໂດຍນໍ້າຖ້ວມຂະໜາດໃຫຍ່ ທີ່ນຳພາຊິ້ນສັດແລະພືດມາກະພົມປະສົມກັນໃນຊັ້ນຕະກະ, ນຳພາສັດເຂົ້າໄປ ຫຼື ຝັງພວກມັນມີຊີວິດຢູ່ໃນຖ້ຳ, ແລະສໍາລັບການເກີດແກ້ວຫຼັງນ້ຳຖ້ວມໃນອີຢິບ, ແມ່ນມີການປະທຸທາງໄຟຟ້າຂະໜາດໃຫຍ່ຈາກການເຄືອນຕົວແກນໂລກ. ຮູບທີ່ \ref{fig:7} ສະແດງການເປີດເຜີຍປົກກະຕິຂອງ 'muck' ທີ່ອະລາດກະ \cite{56}.

\begin{figure}[t]
\begin{center}
% \fbox{\rule{0pt}{2in} \rule{0.9\linewidth}{0pt}}
   \includegraphics[width=1\linewidth]{muck-crop.jpeg}
\end{center}
   \caption{‘muck’ ຂອງອະລາດກາ, ເປັນສ່ວນຜະສົມທີ່ກະຈາຍຢ່າງສັບສົນຂອງຕົ້ນໄມ້, ພືດ ແລະສັດ ໃນກອນດິນແລະກ້ອນນ້ຳແຂງ \cite{146}.}
\label{fig:7}
\label{fig:onecol}
\end{figure}
\section{ເບີ້ງໂບຮານ}

ບັນພະບຸລຸດຂອງເຮົາໄດ້ປ່ອຍຫຼັງໄວ້ຫຼາຍໂຄງສ້າງໂບຮານທີ່ມີການອອກແບບຢ່າງປິທີທີ່ພົບຮ່າງມະນຸດໃນນັ້ນ. ສ່ວນໃຫຍ່ມັກຈະແປພິຈາລະນາວ່າເປັນຫຼຸມຝັງສົບທີ່ວິຈິດວິໄສ, ແຕ່ເມື່ອເບິ່ງໃກ້ໆກໍ່ພົບວ່າມັນອາດຈະເປັນເບີ້ງໂບຮານເທື່ອກໍ່ໄດ້.

\begin{figure}[b]
\begin{center}
% \fbox{\rule{0pt}{2in} \rule{0.9\linewidth}{0pt}}
   \includegraphics[width=1\linewidth]{ww19.jpg}
\end{center}
   \caption{ນິວແກຣນຈ໌, ໄອຣ໌ແລນ - ເບິ່ງນັກທ່ອງທ່ຽວຢູ່ທາງເຂົ້າເພື່ອເປັນສັດສ່ວນ.}
\label{fig:8}
\label{fig:onecol}
\end{figure}

ຕົວຢ່າງທີ່ດີຫນຶ່ງແມ່ນນິວແກຣນຈ໌ (ຮູບ \ref{fig:8}) ໂຄງການອະນຸສົງວະຫຼັກໃນບຣູນາໂບອິນແບບຄອມເພັກ, ຊຶ່ງເປັນການຮວບຮວມເຮືອນໂບຮານຊຸດໃຫຍ່ທີ່ປະກອບມີຫຼຸມຝັງສົບທີ່ເຂົ້າໄດ້. ຫຼຸມຝັງສົບເຫຼົ່ານີ້ປະກອບມີຫ້ອງຝັງສົບເກີນວ່າຫນຶ່ງຫ້ອງທີ່ຖືກປົກດ້ວຍດິນຫຼືຫີນ ແລະມີທາງເຂົ້າແຄບຕັ້ງທໍາດ້ວຍຫີນກ້ອນໃຫຍ່ \cite{70}. ມັນເປັນຕົວຢ່າງຂອງການອອກແບບຢ່າງຫຼາຍໃນໂຄງສ້າງທີ່ຖືກປົກປ້ອງ ສ້າງຂຶ້ນຜ່ານຫຼາຍຮຸ່ນຊົນເພື່ອຝັງຄົນໄວ້ບໍ່ກີ່ຄົນ, ເຫຼົ່າພວກເຂົາຍັງບໍ່ມີຊີວິດເມື່ອເລີ່ມສ້າງຮຸ່ມຝັງສົບນີ້. ເມື່ອມີການຄົ້ນພົບເມື່ອປີ 1699 ໂດຍເຈົ້າຂອງທີ່ດິນໃນທ້ອງຖິ່ນ, ມັນຖືກຝັງໄວ້ໃນດິນ.

ເມື່ອເບິ່ງພາຍນອກໂຄງສ້າງນີ້ຈະເຫັນຄວາມພະຍາຍາມຢ່າງຫຼາຍທີ່ໃຊ້ໃນການກໍ່ສ້າງ - ນິວແກຣນຈ໌ປະກອບມີປະມານ 200,000 ຕັນຂອງວັດຖຸກໍ່ສ້າງ. ພາຍໃນ, \textit{“...ເປັນທາງເຂົ້າຫ້ອງຢ່າງຫນຶ່ງ ສາມາດເຂົ້າໄດ້ທາງປາກທາງດ້ານທິດຕາເວັນອອກສາຍລົງຂອງໂຄງການ ທາງນັ້ນຍາວ 19 ແມັດ (60 ຟຸດ), ຫຼືປະມານໜຶ່ງໃນສາມເຖິງກາງໂຄງສ້າງ ທ້າຍສຸດທາງມີຫ້ອງນ້ອຍ 3 ຫ້ອງຢູ່ຮອບຫ້ອງກາງທີ່ມີຫຼັງຄາດິນໂຄ້ງສູງ... ພະນັງຂອງທາງນີ້ເຮັດດ້ວຍຫີນກ້ອນໃຫຍ່ທີ່ເອີ້ນວ່າ orthostats, ມີຫຼວງກ່ອນຝັ່ງຕາເວັນຕົກ 22 ກ້ອນ ແລະ ຝັ່ງຕາເວັນອອກ 21 ກ້ອນ ສູງເຉລຍ 1.5 ແມັດ”} \cite{70}. ຍັງມີລາຍລະອຽດການກໍ່ສ້າງກັນນໍ້າຢ່າງປິທີ ຕົວຢ່າງເຊ່ນ, ບ່ອນຫຼັງຄາ \textit{“ຊ່ອງຫວ່າງຂອງຫຼັງຄາໄດ້ຖືກອຸດໃສ່ດ້ວຍພົງດິນເຜົາ ຜົມກັບຊາຍທະເລຕົວເພື່ອປ້ອງກັນນໍ້າໃຫ້ຢູ່ໄດ້ ແລະຈາກສ່ວນຜົມນີ້ໄດ້ມີວັນທີຄາບອນ 2 ຄັ້ງຢູ່ປະມານ 2500 ກ່ອນຄຣິດສັກກະຫັດ”} \cite{71}. ນອກນີ້, ຄວາມສູງພິເສດຂຶ້ນໄປຫາຫ້ອງຂ້າງໃນອາດຖືກອອກແບບເພື່ອຈຸດປະສົງຄ້າຍຄືນີ້: \textit{“ເນື່ອງຈາກພື້ນຂອງທາງເຂົ້າແລະຫ້ອງຝັງສົບຕິດຕາມການສູງຂອງພູເຂົາທີ່ສ້າງອະນຸສະວະນີ້ຢູ່ ມີຄວາມແຕກຕ່າງຄວາມສູງເກືອບ 2 ແມັດລະຫວ່າງທາງເຂົ້າກັບພາຍໃນຫ້ອງ”} \cite{71}.

\begin{figure}[b]
\begin{center}
% \fbox{\rule{0pt}{2in} \rule{0.9\linewidth}{0pt}}
   \includegraphics[width=1\linewidth]{dolmen.jpg}
\end{center}
   \caption{ດອນເມນ ເດ ໂຊໂຕ, ສະເປນ \cite{53}.}
\label{fig:9}
\label{fig:onecol}
\end{figure}

ການບໍ່ມີຊິ້ນສ່ວນຊີວະມະນຸດຢູ່ຂ້າງໃນນັ້ນເປັນເລື່ອງໜ້າສົນໃຈອີກຢ່າງ. ການຂຸດຄົ້ນໄດ້ເຜยວ່າມີຊິ້ນສ່ວນກະດູກທີ່ໄດ້ຖືກເຜົາແລະບໍ່ໄດ້ຖືກເຜົາ ທີ່ສະແດງເຖິງຄົນພຽງບໍ່ກີ່ຄົນ, ກະຈາຍຢູ່ຕາມທາງຜ່ານ. ການກໍ່ສ້າງ Newgrange ໄດ້ຖືກຄາດຄະເນວ່າໃຊ້ເວລາຫຼາຍຊົ່ວຮຸ່ນ ອີງຕາມອາຍຸຄາບອນຂອງວັດຖຸພາຍໃນ. ເພາະຫຍັງຊຸມຊົນເກົ່າຈຶ່ງລົງແຮງກຽມຢ່າງຫນັກໃນການສ້າງຫຼັງທີ່ໃຫຍ່ ແລະມີວິສາວະກຳສູງພ້ອມນໍາເພື່ອພຽງແຕ່ກະຈາຍຊິ້ນສ່ວນກະດູກຂອງຫຼາຍຄົນໃນທາງຜ່ານຂອງມັນ? ເປັນເລື່ອງທີ່ນ່າເຊື່ອຖືຫຼາຍກວ່າ ອາຄານມະໂຫອະນຸລົມປຸກປົງແລະສ້າງຢ່າງລະມັດລະວັງເຫຼົ່ານີ້ ໄດ້ຖືກສ້າງເພື່ອເປັນທີ່ຢູ່ອາໄສຂອງມະນຸດ ສໍາລັບປົກປ້ອງຜູ້ຄົນໃນເວລາທີ່ແມ່ນມີເຫດການຖືກທໍາລາຍຊ້ຳໆຂອງໂລກ.

ໃນ Huelva, ທາງໃຕ້ຂອງສະເປນ, ຕົວຢ່າງຄ້າຍຄືກັນ ແມ່ນ Dolmen de Soto (ຮູບ \ref{fig:9}), ເປັນແຫຼ່ງເກືອບ 200 ແຫ່ງໃນພື້ນທີ່ນັ້ນ \cite{72,32}. ມັນແມ່ນອາຄານທີ່ອອກແບບຢ່າງທັນສະໄໝແລະສ້າງດ້ວຍຫິນມະໂຫອະນຸລົມ ເພື່ອມີສະເໝົາລູບຫົວໂຕ 75 ແມັດ. ມີລາຍງານວ່າ ພົບສົມບຸກຄົນພຽງ 8 ສົມໃນການຂຸດຄົ້ນ, ທຸກສົມຖືກຝັງໃນທ່າເດັກທາຣົດ.

\section{ການກ່າວເຖິງສິ່ງຜິດປົກກະຕິທີ່ໜ້າສົນໃຈ}

ໃນພາກນີ້, ຂ້ອຍຂໍກ່າວລວມເຖິງສິ່ງຜິດປົກກະຕິທີ່ໜ້າສົນໃຈເພີ່ມເຕີມອີກບາງຢ່າງ, ທີ່ທຸກຢ່າງອະທິບາຍໄດ້ດີໂດຍສະເຫຼີມອິງໃສ່ເຫດການທຳລາຍແບບ ECDO.

\subsection{ຜິດປົກກະຕິດ້ານຊີວະະນາ}

\begin{figure}[b]
\begin{center}
% \fbox{\rule{0pt}{2in} \rule{0.9\linewidth}{0pt}}
   \includegraphics[width=1\linewidth]{bottleneck.jpg}
\end{center}
   \caption{ສະຫຼຸບພັນທຸກຳທາງພັນທຸກຳເຊິ່ງເປັນການລົດຈຳນວນພູມິຊາຍ 95\% ປະມານ 6,000 ປີກ່ອນ \cite{62}.}
\label{fig:10}
\label{fig:onecol}
\end{figure}

ການຜິດປົກກະຕິທາງຊີວະະວິທະຍາທີ່ນ່າສົນໃຈມີທັງ bottleneck ທາງພັນທຸກຳ ແລະ ຊິ້ນຫຼັກວາລ໌ທີ່ພົບໃນພື້ນດິນພາຍໃນ. Zeng ແລະຄະນະ (2018) ໄດ້ແບບຈຳລອງລຳດັບ Y-chromosome 125 ຊຸດຈາກມະນຸດສະໄຫມ ແລະອີງຕາມຄວາມຄ້າຍຄືນັ້ນ ແລະການກັບຕົວໃນ DNA, ໄດ້ຄົ້ນພົບວ່າເກີດການຫຼຸດລົງຂອງປະຊາກອນພູມິຊາຍປະມານ 95\% ປະມານ 5,000 ເຖິງ 7,000 ປີກ່ອນ (ຮູບ \ref{fig:10}) \cite{62}. ຊິ້ນຫຼັກວາລ໌ ຖືກພົບກັນຫຼາຍສິບແມັດເທິງລະດັບນ້ຳທະເລ ໃນ Swedenborg, Michigan, Vermont, Canada, Chile, ແລະ Egypt \cite{63,64,65,66}. ວາລ໌ເຫຼົ່ານີ້ຖືກພົບໃນສະພາບຫຼາຍຢ່າງ: ໃນສະພາບທີ່ສົມບູນ ໃນບໍສ້າງຢູ່ເທິງຕະກະບອນຫິມະນາທົດ ຫຼື ຝັງໃນຊັ້ນຕະກະ. ຈຳນວນຕົວຢ່າງໃນພື້ນທີ່ເຫຼົ່ານີ້ມີຕັ້ງແຕ່ບໍ່ກີ່ຕົວເຖິງຫຼາຍກວ່າຮ້ອຍຕົວ.  ວາລ໌ເປັນສັດທະເລລຶກແລະຫາກພົບໃກ້ຝັ່ງທະເລຢ່າງຫາຍາກ. ວາລ໌ເຫຼົ່ານີ້ມາຢູ່ທີ່ສູງໄດ້ຢ່າງໃດ ແລະບໍ່ໄດ້ໃກ້ກັບທະເລໄດ້ຢ່າງໃດ?

ມີການສູນພັນກຸ່ມມາກເກີດຂຶ້ນໃນອະດີດຂອງໂລກ ແລະສິ່ງທີ່ຖືກສຶກສາຫຼາຍຄື "Big Five" ໝາຍເຖິງການສູນພັນໃນຍຸກພະແນໂຣໂຊອິກ: ຍຸກ Late Ordovician (LOME), Late Devonian (LDME), ສິ້ນສຸດ Permian (EPME), ສິ້ນສຸດ Triassic (ETME) ແລະ ສິ້ນສຸດ Cretaceous (ECME) \cite{88,89}. ແບບນ້ຳຫນ້າ, ໃນເຫຼົ່ານີ້ຫຼາຍຢ່າງຖືກຈັດຢູ່ໃນຊ່ວງເວລາດຽວກັນກັບຊັ້ນຕະກະຂອງ Grand Canyon ຄື ຊັ້ນ Permian ແລະ Devonian.

\subsection{ຄວາມຜິດປົກກະຕິທາງກາຍະພາບ}

\begin{figure}[b]
\begin{center}
% \fbox{\rule{0pt}{2in} \rule{0.9\linewidth}{0pt}}
   \includegraphics[width=1\linewidth]{columbia.jpg}
\end{center}
   \caption{ຄື່ນຍັກໃຫຍ່ໃນທ່າບຶງນ້ໍາແຂງ Columbia, ລັດ Washington \cite{80}.}
\label{fig:11}
\label{fig:onecol}
\end{figure}

ຍັງມີພູມສັນຖານອີກຫຼາຍນອກເນືອຈາກ Grand Canyon ທີ່ອາດຈະຖືກສ້າງຂຶ້ນໂດຍອໍານາດທີ່ປະທານຫຼາຍ. ເຫດຫຼັກຖານຂອງການໄຫຼຂອງນ້ໍາທວົນທວາລະບົດອາດສາມາດພົບໄດ້ໃນຄື່ນຍັກໃຫຍ່ທີ່ພົບທົ່ວໂລກ. ຕົວຢ່າງຫນຶ່ງກໍຄື Channeled Scablands ໃນເຂດ Pacific Northwest. ທີ່ນີ້, ບໍ່ພຽງແຕ່ພົບພູມສັນຖານທີ່ເວົ້າດ້ວຍການຕົກທົກປະເພດເຊັ່ນຕະກະກະ, ຫົນຜາເຫຼັ້ມກະດານ, ແຕ່ຍັງພົບລຳດັບຄື່ນຍັກໃຫຍ່ຫຼາຍກວ່າຮ້ອຍສາຍທີ່ເກີດຈາກນ້ໍາໄຫຼຍັກໃຫຍ່ \cite{78,79}. ເຫຼົ່ານີ້ແມ່ນຮູບແບບຂອງຄື່ນຂະໜາດໃຫຍ່ກວ່າທີ່ພົບໃນຊາຍທີ່ລຸ່ມນ້ໍາ. ຄື່ນເຫຼົ່ານີ້ສາມາດພົບໄດ້ທົ່ວໂລກ ໃນປະເທດຝຣັ່ງ, ອາຈີນຕີນາ, ລັດເຊຍ, ແລະ ອາເມຣິກາເໜືອ \cite{81}. ຮູບທີ່ \ref{fig:11} ແສດງໃຫ້ເຫັັນຄື່ນບາງສ່ວນໃນລັດ Washington ປະເທດສະຫະລັດອາເມຣິກາ \cite{80}.

\begin{figure}[b]
\begin{center}
% \fbox{\rule{0pt}{2in} \rule{0.9\linewidth}{0pt}}
   \includegraphics[width=1\linewidth]{zhangjiajie.jpg}
\end{center}
   \caption{ເສົາຫິນຍັກໃຫຍ່ໃນອຸທະຍານແຫ່ງຊາດ Zhangjiajie, ພາກໃຕ້ຂອງປະເທດຈີນ.}
\label{fig:12}
\label{fig:onecol}
\end{figure}

\begin{figure}[b]
\begin{center}
% \fbox{\rule{0pt}{2in} \rule{0.9\linewidth}{0pt}}
\includegraphics[width=1\linewidth]{hoy.jpg}
\end{center}
   \caption{ເສົາຫີນທະເລ Old Man of Hoy, ສະກັອດແລນ \cite{83}.}
\label{fig:13}
\label{fig:onecol}
\end{figure}

ໂຄງສ້າງການກັດກິນພາຍໃນພື້ນດິນກໍ່ຖືກອະທິບາຍໄດ້ແບບດີໂດຍການເຄື່ອນໂລກທີ່ຄ້າຍ ECDO. ພາກໃຕ້ຂອງຈີນເປັນຕົວຢ່າງທີ່ດີຂອງພູມທັດຫິນປູນສາຍດອຍ, ເກີດຈາກການກັດກິນຂອງນໍ້າ \cite{82}. ພູມທັດເຫຼົ່ານີ້ລວມເອົາຫົວເຂົາມອນ, ຫົວເຂົາແຫລມ, ຫົວເຂົາຮູບກະໂລນ, ສະພານທໍາມະຊາດ, ຫວ່າງເຂົາ, ຖ້ຳໃຫຍ່, ແລະ ບໍ່ນໍ້າທີ່ຈົມ. ທີ່ເກີນທີ່ສຸດແມ່ນທີ່ປ່າແຫ່ງຊາດຈັງຈຽເຈຍ, ມີເສົາຊາຍ quartz ຂະໜາດໃຫຍ່ (ຮູບ \ref{fig:12}) \cite{84}. ເສົາເຫຼົ່ານີ້ມີລະດັບສູງກວ່າ 1,000 ແມັດ ແລະ ຈຳນວນຫຼາຍກວ່າ 3,100 ຕົວ. ຫຼາຍກວ່າ 1,000 ຕົວມີຄວາມສູງເທິ່ງ 120 ແມັດ, ແລະ 45 ຕົວສູງກວ່າ 300 ແມັດ \cite{85}. ເສົາເຫຼົ່ານີ້ຄ້າຍກັບເສົາຫີນກັດກິນຂອງທະເລ (ຮູບ \ref{fig:13}) ເຊິ່ງແມ່ນເສົາຫິນຕາມຊາຍຝັ່ງທີ່ເກີດຈາກການພັງລົມຂອງວັດຖຸຮ້ອມຂ້າງໂດຍຄື້ນທະເລ. ພູມທັດການກັດກິນຄ້າຍຄືກັນສາມາດເຫັນໃນກອງຫິນ Urgup ທີ່ຕູລະກີ, ແລະ Ciudad Encantada ທີ່ສະເປນ, ເຊິ່ງທັງສອງສູງກວ່າ 1,000 ແມັດເໝືອທະເລສາບ. ສະຖານທີ່ເຫຼົ່ານີ້ມີເກືອ ແລະ ຊິ້ນສ່ວນຊີວະສາດທະເລຢູ່ໃກ້ຄຽງ, ເປັນຫຼັກຖານຂອງທະເລໃນອະດີດ \cite{15,86,87}. ນອກຈາກນີ້, ເຣື່ອງນິທານນໍ້າຖ້ວມເວົ້າເຖິງທະເລຂຶ້ນໄປສູງກວ່າ 1,000 ແມັດ ແລະ ພິສູດໄດ້ຈາກເກືອ ແລະ ແປ້ນເກືອຂະໜາດໃຫຍ່ໃນ Andes ແລະ Himalayas ທີ່ຢູ່ສູງກວ່າປະລິມານເມັດເໝືອທະເລສາບ. ຢ່າງຕົວຢ່າງ, ແປ້ນເກືອ Uyuni ທີ່ Bolivia ເຖິງ 3653 ແມັດເໝືອລະດັບນໍ້າທະເລ \cite{94}.

\subsection{ເຫດການປ່ຽນແປງສະພາບອາກາດຢ່າງວ່ອງໄວ}

ວິທະຍາສາດສະໄຫມສະເຫນີການມີຢູ່ຂອງເຫດການປ່ຽນແປງສະພາບອາກາດໂລກແບບວ່ອງໄວໃນປະຫວັດສາດເມື່ອໄມ່ດົນມານີ້ຂອງໂລກ. ສອງຕົວຢ່າງທີ່ສຳຄັນແມ່ນເຫດການ 4,200 ປີກ່ອນ ແລະ 8,200 ປີກ່ອນ, ທີ່ເກີນຂຶ້ນຄຽງຄູ່ກັບການຫຼຸດລົງຂອງປະຊາກອນ ແລະ ການອົບພະຍົບທາງສັງຄົມຢ່າງກວ້າງຂວາງ. ເຫດການເຫຼົ່ານີ້ຖືກບັນທຶກເປັນຂໍ້ຜິດປົກກະຕິໃນສະຫນິບພົບຕະກອນດິນ ແລະ ນ້ຳກ້ອນ, ປູນປະກອບຊີວະສາດບາງຊະນິດ, ຄ່າເອິ້ນ isotope O18, ລາຍງານ pollen ແລະ speleothem, ແລະ ຂໍ້ມູນລະດັບນ້ຳທະເລ. ການປ່ຽນແປງອາກາດທີ່ຖືກອະທິບາຍວ່າເປັນ ການຍົກຕົວລົງໂດຍດ່ວນຂອງອຸນຫະພູມໂລກ, ຄວາມແຫ້ແຫ້ງ, ການປະສານຝົນຂອງ Atlantic meridional overturning current, ແລະ ການຂະຫຍາຍຕົວຂອງພາກນ້ຳແຂງ \cite{90,91,92}. ເຫດການອອກສູນປີ 8,200 ແມ່ນກ່ຽວຂ້ອງກັບການນໍ້າເຄັມທ້ວມທະເລດຳຂອງ Black Sea ປະມານປີ 6400 ກ່ອນຄຣິດສັກກະຫຼາດ \cite{93}.

\subsection{ພິພິດທັດທາງບູຮານຄະດີ}

ຫຼັກຖານທາງບູຮານຄະດີຂອງເມືອງເກົ່າບາງແຫ່ງຊີ້ເຫັນຊັ້ນຫຼາຍຊັ້ນທີ່ເກີດຈາກການຝັງແລະການທຳລາຍ, ເຮັດໃຫ້ເກີດບັນທຶກເຫດການພິບັດໃນອະດີດ. ເມືອງເກົ່າ Jericho ແມ່ນຕົວຢ່າງໜຶ່ງ, ຕັ້ງຢູ່ໃນປາ່ຈຸບັນຂອງ Palestine ມີຫຼາຍຊັ້ນການທຳລາຍ ເຊັ່ນອາຄານຫິນພັງແລະໄຟໄໝ້ຮຸນແຮງ \cite{96,97}. ເວລາໃນຊັ້ນເຫຼົ່ານີ້ແຕ່ປະມານ 9000 ປີ ກ່ອນຄຣິດການ ຫາ 2000 ປີ ກ່ອນຄຣິດການ. ທີ່ນ່າສົນໃຈແມ່ນຫໍ Jericho ທີ່ຖືກຕັດຂາດແລະຝັງໃນຕະກອນດິນປະມານ 7400 ປີ ກ່ອນຄຣິດການ (ຮູບ \ref{fig:14}) \cite{95}. Catal Huyuk \cite{99}, Gramalote \cite{98}, ແລະ ຫວັງທຳມະນິຍມ Minoan ທີ່ Knossos ໃນ Crete \cite{100,101} ລ້ວນເປັນຕົວຢ່າງຂອງເຫດການທາງບູຮານຄະດີທີ່ມີຫຼາຍຊັ້ນ ແລະ ມັກຈະພົບຫຼັກຖານການທຳລາຍ.

\begin{figure}[t]
\begin{center}
% \fbox{\rule{0pt}{2in} \rule{0.9\linewidth}{0pt}}
   \includegraphics[width=1\linewidth]{jericho.jpg}
\end{center}
   \caption{ການສ້າງສະຫຼຸບທາງບູຮານຄະດີຂອງການຝັງ Tower of Jericho ປະມານ 7400 ປີ ກ່ອນຄິດສະກົດ \cite{95}.}
\label{fig:14}
\label{fig:onecol}
\end{figure}

ອີກຫຼັກຖານຫນຶ່ງສໍາລັບເຫດການພິບັດທີ່ທໍາລາຍອາລະຍະທຳມະນຸດຄື Nampa Image, ຕຸກຕາດິນທີ່ຖືກພົບຢູ່ລຶກປະມານ 100 ແມັດໃຕ້ຊັ້ນລາວາໃນ Idaho \cite{102,103}. ຊັ້ນລາວາທີ່ຕຸກຕານີ້ຖືກພົບຖືກຄາດຄະເນວ່າຖືກສະສົມໃນຊ່ວງທ້າຍ Tertiary ຫຼື Quaternary, ມີອາຍຸປະມານ 2 ລ້ານປີ. ແຕ່ຢ່າງໃດກໍຕາມ, ຊັ້ນລາວາໃນພື້ນທີ່ນັ້ນກະທັດສົມທີ່ຈະໃໝ່. ການພົບເຫັນເຊັ່ນນີ້ບໍ່ພຽງແຕ່ຊີ້ໃຫ້ເຫັນເຖິງເຫດການເສຍເສັ້ນອາລະຍະທຳ, ແຕ່ຍັງເຮັດໃຫ້ເຮົາຄອຍສົງໃສໃນອາຍຸທີ່ເລື່ອນເວລາຕາມການກຳນົດຂອງຍຸກໃຫມ່.

\section{ກ່ຽວກັບວິທີການລະບຸອາຍຸໃນຍຸກສະໄໝໃຫມ່}

ມີເຫດຜົນສຳຄັນທີ່ຄວນສົງໃສໃນລັດຖະບານກ່ຽວກັບອາຍຸທີ່ຖືກກຳນົດເປັນລ້ານ ຫຼືເຖິງຮ້ອຍລ້ານປີໃຫ້ກັບສານວັດຖຸຕ່າງໆ.

ເກີດເຮັດທີ່ຍອມຮັບກັນຄື່ນວ່າ "ເຊື້ອເພີງຟອສຊິນ" ເຊັ່ນ ຖ່ານຫີນ, ນ້ຳມັນ, ແລະ ກາຊະຕິ ມີອາຍຸເປັນຮ້ອຍລ້ານປີ \cite{104}. ແຕ່ເວລາທີ່ເຮັດການກວດອາຍຸຄາເບີນຂອງນ້ຳມັນໃນກົບຂອງເມັກຊິໂກ, ພົບວ່າມີອາຍຸປະມານ 13,000 ປີ \cite{105}. ອາຍຸຄຶ້ນຊີວິດຂອງ Carbon-14 ແມ່ນສັ້ນຫຼາຍ (5,730 ປີ) ໃຫ້ມັນຄວນຈະຍ່ອຍສະຫຼາຍເສັ້ນໃນບໍ່ກີ່ຮ້ອຍພັນປີ. ແຕ່ມັນກະຖືກພົບໃນຖ່ານຫີນ ແລະ ຊິ້ນສ່ວນຂອງຟອສຊິນທີ່ເຮັດວ່າມີອາຍຸຫຼາຍກວ່ານັ້ນຫຼາຍພັນເທົ່າ \cite{106}. ແທ້ຈິງແລ້ວ, ຖ່ານຫີນທີ່ຖືກຜະລິດແບບປະດິດສາມາດທໍາໃນຫ້ອງປະດິດພາຍໃນເວລາ 2-8 ເດືອນ ໂດຍເທັ່ງໃສ່ອຸນຫະພູມສູງ \cite{107}.

ວິທີການກຳນົດອາຍຸດ້ວຍອິດທິພົນຂອງຮັດສະສະຫຼາດອາດຈະບໍ່ແມ່ນຖືກຕ້ອງເສມໄປ. ກຸ່ມວິຈັຍ Answers in Genesis ໄດ້ພົບຄວາມຂັດແຍ້ງໃນຂໍ້ມູນອາຍຸທີ່ໄດ້ຈາກວິທີນີ້ ເຊິ່ງທໍາໃຫ້ສົງໃສໃນຄວາມເປັນຈິງຂອງມັນ \cite{108}. ມີການພົບຊິ້ນສ່ວນອອກຕິດອ່ອນທີ່ມີເລືອດ, ຫຼອດເລືອດ ແລະ ຄໍາແພບໂມທໍາ ໃນຊານໄດໂນເສົາທີ່ຖືກກ່າວຫາວ່າມີອາຍຸເປັນຮ້ອຍລ້ານປີດ້ວຍ \cite{109,110}. ອີງຕາມສິ່ງທີ່ເຮົາຮູ້, ເປັນໄປໄດ້ວ່າອາຍຸທີ່ເຖິງກຳນົດທັງຫມົດຂອງໄລຍະຫິນສາຍພູມິສາດໂລກ ແລະ ວັດຖຸທາງກາຍພາບເຊັ່ນຫິນ ແລະ ເຊື້ອເພີງອາດຈະບໍ່ຖືກຕ້ອງໂດຍຕ່າງກັນຫຼາຍລະດັບ.

\section{ສະຫຼຸບ}

ໃນເອກະສານນີ້, ຂ້ອຍໄດ້ສະແດງຫຼັກຖານທີ່ນ້າສົນໃຈທີ່ສຸດເຊິ່ງບົດບາດບອກເຖິງຕົ້ນກຳເນີດທີ່ເຫັນໄດ້ຊັດແລະອະທິບາຍໄດ້ດີທີ່ສຸດໂດຍ ECDO Earth flip. ແມ່ນວ່າຫຼາກຫຼາຍ, ການລວບລວມຫຼັກຖານນີ້ຍັງບໍ່ຄົບຖ້ວນ - ຍັງມີຫຼັກຖານອື່ນໃຫມ່ທີ່ຂ້ອຍໄດ້ລວບລວມໄວ້ແລະເປີດໃຫ້ສາທາລະນະໃນຖານຂໍ້ມູນ GitHub ງານຄົ້ນຄວ້າຂອງຂ້ອຍ \cite{2}.
\section{ການຂອບໃຈ}

ຂອບໃຈ Ethical Skeptic ຜູ້ຂຽນແຕ່ງຕົ້ນສະບັບຂອງວິທິສະດີ ECDO ທີ່ໄດ້ສຳເລັດງານວິຈັບອັນປັນຍາ ແລະ ໃຫມ່ສຸດຂອງເຂົາ ແລະ ແບ່ງປັນໃຫ້ກັບໂລກໃບນີ້. ວິທິສະດີທີ່ໃຫຍ່ສາມສ່ວນຂອງເຂົາ \cite{1} ຍັງຄงເປັນຜົນງານທີ່ມີອຳນາດສູງສຸດສຳລັບທິດສີ Exothermic Core-Mantle Decoupling Dzhanibekov Oscillation (ECDO), ແລະມີຂໍ້ມູນອື່ນໆອີກຫຼາຍກ່ຽວກັບຫົວຂໍ້ນີ້ມາກກວ່າທີ່ຂ້ອຍໄດ້ສັງລວມສັ້ນໆໃນນີ້.

ແລະແນ່ນອນ, ຂອບໃຈຕໍ່ບັນດາຍັກສາດເຫົາທີ່ພວກເຮົາຍືນຢູ່ບນບ່າຂອງພວກເຂົາ; ຜູ້ທີ່ໄດ້ເຮັດວິຈັຍ ແລະ ການສືບສວນທີ່ເຮັດໃຫ້ວຽກນີ້ເກີດຂຶ້ນໄດ້ ແລະ ຮ່ວມນຳແສງສວ່າງມາສູ່ມະນຸດສະບັບ.

\clearpage
\twocolumn

{\small
\renewcommand{\refname}{ບັນຊີອ້າງອີງ}
\bibliographystyle{ieee}
\bibliography{egbib}
}

\end{document}
\documentclass[10pt,twocolumn,letterpaper]{article}

% My own stuff
\usepackage{booktabs}
% \usepackage{caption}
% \captionsetup[table]{skip=8pt}   % Only affects tables
\usepackage{stfloats}  % Add this to the preamble
\usepackage{xeCJK}  % Supports Simplified & Traditional Chinese
\usepackage{cvpr}
\usepackage{times}
\usepackage{epsfig}
\usepackage{graphicx}
\usepackage{amsmath}
\usepackage{amssymb}

% Include other packages here, before hyperref.

% If you comment hyperref and then uncomment it, you should delete
% egpaper.aux before re-running latex.  (Or just hit 'q' on the first latex
% run, let it finish, and you should be clear).
\usepackage[breaklinks=true,bookmarks=false]{hyperref}

\cvprfinalcopy % *** Uncomment this line for the final submission

\def\cvprPaperID{****} % *** Enter the CVPR Paper ID here
\def\httilde{\mbox{\tt\raisebox{-.5ex}{\symbol{126}}}}

\renewcommand{\figurename}{图}   % or whatever you like instead of "Hình"

\makeatletter
\def\abstract{%
  \centerline{\large\bf 摘要}% <-- your new label
  \vspace*{12pt}%
  \it%
}
\makeatother

% This makes the font slightly bigger than base (10) and bold in Subsection headings rather than using ptmb
\makeatletter
\def\cvprsubsection{%
  \@startsection{subsection}{2}{\z@}%
    {8pt plus 2pt minus 2pt}{6pt}%
    % {\normalfont\bfseries\selectfont}%
    {\normalfont\bfseries\fontsize{11}{13}\selectfont}%
}
\makeatother

% So this hardcodes the style for the numbers in the section/subsection headings so they're bold
\font\elvbf=ptmb scaled 1100
\font\elvbfs=ptmb scaled 1200
\makeatletter
% Section number: Large + bold
\renewcommand\thesection{%
  {\elvbfs\arabic{section}}%
}

% Subsection number: normalsize + bold + custom punctuation
\renewcommand\thesubsection{%
  {\elvbf
   \arabic{section}.\arabic{subsection}}%
}
\makeatother

% Pages are numbered in submission mode, and unnumbered in camera-ready
%\ifcvprfinal\pagestyle{empty}\fi
\setcounter{page}{1}
\begin{document}

%%%%%%%%% TITLE
\title{ECDO 数据驱动型入门书 (下):调查科学界历史上的异常现象,以 ECDO“地球翻转”理论作出完美解释}

\author{Junho\\
2025年2月发表\\
网站:\href{https://sovrynn.github.io}{sovrynn.github.io}\\
ECDO 研究资源库:\href{https://github.com/sovrynn/ecdo}{github.com/sovrynn/ecdo}\\
{\tt\small junhobtc@proton.me}
}

\maketitle
%\thispagestyle{empty}

%%%%%%%%% ABSTRACT
\begin{abstract}
2024年5月,一位化名为“The Ethical Skeptic”的网络作者 \cite{0} 发布了一项突破性理论,称为放热地核-地幔分离贾尼别科夫振荡 (ECDO) \cite{1}。该理论不仅提出地球曾经历过突如其来的旋转轴灾难性转变,由于惯性导致海洋溢出大陆从而引发全球大洪水,还提出了一种解释性因果地球物理过程,并附有数据表明这种翻转可能即将来临。尽管此类灾难性洪水和世界末日的预测并不新鲜,ECDO 理论由于其科学、现代、多学科和数据支持的研究路径而独具说服力。

这篇研究论文是对 6 个月独立研究 \cite{2,20} 的简要总结的下半部分,专注于那些由灾难性 ECDO“地球翻转”提供最佳解释的科学与历史异常现象。

\end{abstract}

%%%%%%%%% BODY TEXT

\section{引言}

现代均变地质学和历史学声称,像大峡谷这样的主要地质景观是在数百万年间形成的 \cite{143};加利福尼亚死亡谷的盐存在是因为数亿年前这里曾被海洋覆盖 \cite{144};我们的祖先在 150 代人之前花费其一生时间于修建巨型坟墓 \cite{29,70};所谓的“化石燃料”已有数亿年历史 \cite{104}。或许最有趣的是,人类被认为已有 30 万年的历史 \cite{145},而记录的历史和文明则仅追溯到大约 5,000 年前,相当于 150 代人。

正如我们将看到的,这些异常现象最好的解释是灾难性的地质力量。

\section{泥中冰封猛犸象}

\begin{figure}[t]
\begin{center}
   \includegraphics[width=1\linewidth]{jarkov-mammoth.jpg}
\end{center}
   \caption{Jarkov 猛犸象,20,000 年前的西伯利亚猛犸象,于冻土中发现,仍保存完好 \cite{51}。}
\label{fig:1}
\label{fig:onecol}
\end{figure}

其中一类异常现象是埋藏在泥土中,完好保存的冻结猛犸象,通常见于北极地区 (图 \ref{fig:1})。发现于西伯利亚的贝雷佐夫卡猛犸象埋藏在在泥砾中,保存完好,乃至其肉质在死后数千年内仍然可食用。它的口腔和胃中还存有植被类食物,科学家对此困惑:如果它死前正在进食开花植物,那怎么可能会在瞬间冻住 \cite{17}。据报道,\textit{“1901年,贝雷佐夫卡河附近发现了一具完整猛犸象尸体,而这只动物似乎是在仲夏时因寒冷而死;该事引起轰动。其胃中的食物保存完好,包括毛茛和开花的野豆:这表示这些食物一定是于7月底或8月初吞食的。这只动物死得突然,乃至它的下颚中仍然含着一口花和草。显然,它由一种巨大的力卷走,到达草地数里之外。它的骨盆和一条腿骨折,标志这只巨大动物的膝盖受到严重撞击,然后在平常一年中最热的时候冻死”} \cite{18}。此外,\textit{“[俄罗斯科学家] 记录到,甚至连这只动物的胃部最内层都保持了完好的纤维结构,这表明随着自然界中某种非凡的变化它丧失了全部体温。桑德森特别注意到这一点,并向美国冷冻食品研究所提问:什么条件下才能冻结一只完整的猛犸象,使其身体的最内层,乃至胃的内衬水分都没能来得及形成足以破坏肉纤维结构的大水晶?…几周后,该研究所回信给桑德森,给出答案:绝对不可能。凭借我们所有的科学和工程知识,绝对没有已知的方法可以快速移除猛犸象这样的大型尸体的体热,乃至肉体中不会形成大水晶。此外,在穷尽所有科学和工程技术后,他们观察自然,得出结论:自然界中没有已知现象可以完成这种难事”} \cite{19}。

\section{大峡谷}

大峡谷,位于北美西南部的大盆地,是另一个暗示灾难性起源的自然现象 (图 \ref{fig:2})。首先,组成大峡谷的沉积砂岩和石灰岩层跨越了多达240万平方公里的巨大区域 \cite{21}。图 \ref{fig:3} 显示了科科尼诺砂岩层在美国西部的范围。如此大范围的均匀材料水平层只能是一次性沉积而成。

\begin{figure*}[t]
\begin{center}
\includegraphics[width=1\textwidth]{Grand_Staircase-big.jpg}
\end{center}
   \caption{构成大峡谷的沉积层 (图片右侧) 直接向北延伸至犹他州的雪松断崖 (图片左侧),在那里它们全部向上弯曲 \cite{50}。}
\label{fig:4}
\end{figure*}

\begin{figure}[t]
\begin{center}
   \includegraphics[width=1\linewidth]{grand-canyon.jpg}
\end{center}
   \caption{美国亚利桑那州的大峡谷 \cite{49}}
\label{fig:2}
\label{fig:onecol}
\end{figure}

\begin{figure}[t]
\begin{center}
   \includegraphics[width=1\linewidth]{coconino.jpg}
\end{center}
   \caption{美国西部科科尼诺砂岩层的大小 \cite{21}}
\label{fig:3}
\label{fig:onecol}
\end{figure}

仔细观察大峡谷告诉我们,这些广泛的沉积层的沉积也同时伴随着显著的构造力量。为了理解这一点,我们必须仔细研究峡谷中某些沉积层被弯曲和暴露的区域。Genesis 答案的研究人员 \cite{42} 对这些折叠中的岩石样本进行了显微观察,例如纪念碑褶皱,并根据缺乏在长时间跨度的热和压力下应该出现的特征,得出结论认为这些沉积层在仍然柔软时,即沉积后不久,被构造力量折叠 \cite{43}。

放大视野,我们发现构成大峡谷的层不仅在峡谷内部被折叠。层已经在东凯巴布单斜褶皱 \cite{46} 向东折叠,但也在犹他州雪松断崖向北折叠 (图 \ref{fig:4})。这表明这些层可能在很快相继沉积在彼此之上后全部一起被折叠。作为参考,大峡谷的水平层厚度约为 1700 米。要求铺设厚达一英里的沉积层的地质过程的规模是巨大的。

大峡谷的实际形成是现代地质学中的另一个争议点。均变论地质学提出大峡谷是由科罗拉多河在几百万年内雕刻而成的 \cite{47}。然而,Genesis 答案研究小组认为,大峡谷很可能是由于一个古老湖泊突破其边界的溢洪道侵蚀在数周内形成的,这在开凿峡谷时带走了大量的沉积物。大峡谷东侧的湖泊沉积物和海洋化石中存在一个高海拔湖泊的证据。将大峡谷与其他大规模溢洪道侵蚀的例子比较,如阿夫顿峡谷和圣海伦斯火山,揭示了类似的地形,并表明大型峡谷可以通过大量流动的水迅速形成 \cite{48}。

考虑到要求在如此大面积的土地上铺设沉积物的地质过程规模,在沉积层铺设后不久发生的巨大构造力量的并发性,以及科罗拉多河相对于大峡谷的巨大的规模微不足道的事实,似乎其形成过程中可能没有任何渐进的特点。

\section{代林库尤地下城}

除了金字塔之外,古代工程的一个伟大例子是土耳其卡帕多西亚的代林库尤地下城 (图 \ref{fig:5})。它是该地区 200 多个地下庇护所中最大的一座 \cite{54}。据估计,这座地下城市曾经容纳多达 20,000 人,并且延伸到 18 层,深达 85 米。虽然它的年龄尚不确定,但据估计至少有 2800 年的历史。该市是从软火山岩雕刻而成的 \cite{52, 53}。

\begin{figure}[t]
\begin{center}
   \includegraphics[width=1\linewidth]{derinkuyu.jpeg}
\end{center}
   \caption{代林库尤地下城示意图 \cite{56}}
\label{fig:5}
\label{fig:onecol}
\end{figure}

\begin{figure}[t]
\begin{center}
   \includegraphics[width=1\linewidth]{derinkuyu-air.jpg}
\end{center}
   \caption{Derinkuyu 的一个深通风井 \cite{53}。}
\label{fig:6}
\label{fig:onecol}
\end{figure}

Derinkuyu 引人入胜的原因在于,目前尚不清楚为什么有社区会决定建立一个完整的地下城市。为了在地下创造生活空间,每个空洞都必须从岩石中凿出。地下隧道的粗糙形状和质地清楚地表明,这些是用手工而不是电动工具雕刻的,与在地上建造庇护所相比,这将是一个数量级的难度。事实上,不清楚为什么人们在尘世生命的限制期间会想永久生活在地下,因为农业、阳光、大自然和探索只有在地上才可获取。传统“历史”认为 Derinkuyu 是由需要一个隐秘场所来实践他们信仰的基督徒创造的 \cite{53}。但常识认为,处理敌人的最直接方式是“战斗或逃跑”,而不是“从岩石中挖出地下城市”。

地下城的规模、深度和设计考究性表明,其并非设计为为了在紧急情况下方便与入侵者作战的临时军事防御结构,而是为了长期躲避地面上的破坏性力量。Derinkuyu 不仅配备了基本的卧室、厨房和浴室,还有牲畜棚、水箱、粮仓、酒榨、油榨、学校、教堂、坟墓和巨大通风井 (图 \ref{fig:6})。军事庇护所怎么会需要酒榨,又挖掘出 85 米深,并具有如此复杂的结构呢?

Derinkuyu 最合理的解释是,迫切需要准备一个长期的、自我维持的庇护所,以防护地球表面灾难性的地球物理力量。

\section{生物质累积}

\begin{figure}[t]
\begin{center}
   \includegraphics[width=1\linewidth]{muck-crop.jpeg}
\end{center}
   \caption{阿拉斯加的“淤泥”,由混乱分布的树木、植物和动物碎片组成,处于冰冻的淤泥和冰中 \cite{146}。}
\label{fig:7}
\label{fig:onecol}
\end{figure}


由各种动物和植物组成的生物质混合物,通常在沉积层中以化石状态出现,是另一个令人费解的异常现象。在《Reliquoæ Diluvianæ》中,威廉·巴克兰牧师详细描述了在英国和欧洲各地发现的无数无法解释为何会在一起的动物物种,这些动物被埋在沉积性“洪积层”中 \cite{58}。这样的动物遗骸混合在挪威 Valdroy 岛的 Skjonghelleren 洞穴中也有发现。在这个洞穴中,发现了超过 7,000 根哺乳动物、鸟类和鱼类的骨头,混杂在多个沉积层之中 \cite{59}。另一个例子是意大利的“巨人洞”San Ciro。在这个洞穴中,发现了几吨哺乳动物的骨头,主要是河马,这些骨头状态如此新鲜,以至于被切割成饰品,并运出用于制造灯黑。报告称,不同动物的骨头被混在一起,破碎,粉碎,并分散成碎片 \cite{60,61}。在古代 Mendes,埃及,一种混合了各种动物骨头的混合物与玻璃化粘土在一起 \cite{57}。这些发现可能看起来令人费解,但可以通过大规模洪水在沉积层中沉积动物尸体或者将其活埋在洞穴中,以及在埃及玻化生物质的情况下,由于地核-地幔位移后的巨大电放电来轻松解释。图 \ref{fig:7} 描绘了一种典型的阿拉斯加生物质“淤泥”暴露 \cite{56}。

\section{古代地堡}
我们的祖先留下了许多高度工程化的古代建筑,其中发现了人类遗骸。这些通常被解释为精致的坟墓,但仔细观察表明,这些实际上可能是古代地堡。

\begin{figure}[t]
\begin{center}
   \includegraphics[width=1\linewidth]{ww19.jpg}
\end{center}
   \caption{爱尔兰纽格兰奇,参观者位于入口处以示比例}
\label{fig:8}
\label{fig:onecol}
\end{figure}

\begin{figure}[t]
\begin{center}
   \includegraphics[width=1\linewidth]{dolmen.jpg}
\end{center}
   \caption{西班牙多尔门德索托 \cite{53}}
\label{fig:9}
\label{fig:onecol}
\end{figure}

极好的例子是纽格兰奇 (图\ref{fig:8}),Brú na Bóinne 综合体中的主要纪念碑,这是一组古代建筑,包括所谓的通道墓。这些坟墓由一个或多个被土或石覆盖的墓室组成,并有一个由大石块构成的狭窄通道\cite{70}。这是复杂保护结构广泛工程的例子,建造历时多代,据称是为了埋葬少数人,而这些人在墓穴建造时甚至还未出生。当1699年被一位当地地主重新发现时,它被埋在土中。

对结构粗略观察,可以发现地堡建设所投入的巨大努力:纽格兰奇由约 20 万吨材料构成。在其内部,\textit{“……是可以通过纪念碑东南侧入口进入的室内通道。通道长达 19 米 (60英尺),大约是到结构中心距离的三分之一。在通道尽头有三个小房间,各位于一个较大的中央房间外面,房间有高的尖券拱顶……通道的墙壁由立石石板构成,西侧有 22 块,东侧有 21 块。平均高 1½ 米”}\cite{70}。这里还有复杂的防水工程细节。例如,在屋顶上,\textit{“屋顶的缝隙填满烧土和海砂的混合物,用以防水,并且在检测坟墓结构是,我们从这种混合物中获得了公元前2500年左右的两个放射性碳”}\cite{71}。此外,通往内室的高度上升可能出于类似目的实施:\textit{“由于墓室的通道和墓室地板随着纪念碑所在山丘的地面上升,因此入口和墓室内部的地面水平存在几乎2米的差异”}\cite{71}。


内部缺少人类遗骸也是一个奇怪的点。挖掘发现了烧过和未烧过的骨头碎片,代表少数人,散布在通道中。根据内部材料的碳年代,纽格兰奇的建造估计至少花费了几代人的时间。为什么一个古代社区会花费如此大的精力建造一个巨大的、高度工程化的坟墓,仅仅为了把几位死者的骨头碎片分散在其通道中?更合理的解释是,这些古老且经过精心防水的巨石建筑实际上是为了在地球的周期性灾变中保护人类而建造的庇护所。

在西班牙南部的韦尔瓦,有一个类似的例子是多尔门德索托 (图\ref{fig:9}),这是该地区约 200 个此类遗址之一\cite{72,32}。这是一个使用巨石建造的流线型、工程化的结构,直径为 75 米。据报道,挖掘时仅发现了八具遗体,全部以胎儿的姿态埋葬。

\section{重大异常现象案例}

在本节中,我将简要提及一些更重大的异常现象,所有这些都可以通过类似ECOD的灾变得到很好的解释。

\subsection{生物异常}
\begin{figure}[t]
\begin{center}
   \includegraphics[width=1\linewidth]{bottleneck.jpg}
\end{center}
   \caption{约 6000 年前代表 95\% 男性削减的遗传瓶颈 \cite{62}。}
\label{fig:10}
\label{fig:onecol}
\end{figure}

一些显著的生物异常现象是遗传瓶颈和内陆鲸鱼化石。Zeng et al. (2018) 模拟了现代人类的 125 条 Y 染色体序列,并基于 DNA 的相似性和突变,识别出男性人口在约 5000 至 7000 年前出现了 95\% 人口减少的瓶颈 (图 \ref{fig:10}) \cite{62}。鲸鱼化石被发现于距海平面数百米的高处,位于瑞典堡, 密歇根, 佛蒙特, 加拿大, 智利, 和埃及 \cite{63,64,65,66}。这些鲸鱼的保存状态各异:有的完好无损,有的躺在冰川沉积物之上的泥炭中,还有的埋在沉积物中。这些地点的标本数量从几个到超过一百个不等。鲸鱼是深海生物,很少靠近海岸。如何这些鲸鱼会出现在如此高的海拔,有时甚至是在极为内陆的地方?

地球上发生过多次大规模灭绝事件,其中研究最为彻底的是五次"大五"显生宙事件:晚奥陶纪 (LOME)、晚泥盆纪 (LDME)、末二叠纪 (EPME)、末三叠纪 (ETME) 和末白垩纪 (ECME) 大灭绝 \cite{88,89}。有趣的是,其中若干次灭绝事件被分类为与许多大峡谷地层的历史时期相同,即二叠纪和泥盆纪层。

\subsection{物理异常}

\begin{figure}[b]
\begin{center}
   \includegraphics[width=1\linewidth]{columbia.jpg}
\end{center}
   \caption{华盛顿州冰川哥伦比亚湖中的巨大河流波纹 \cite{80}}
\label{fig:11}
\label{fig:onecol}
\end{figure}

除大峡谷外,还有许多奇景可能是由灾难带来的力所形成。世界各地的巨形河流纹路就可以证明大范围大陆流水的存在。例如太平洋西北的沟槽状滩地,这里,我们不仅能看到沉积物景观和不规则巨石,还能看到由巨型水流形成的 100 条排列在一起的纹路 \cite{78,79}。这些就是溪流沙床中那些纹路的大号版本。法国、阿根廷、俄罗斯和北美皆有类似纹路 \cite{81}。图 \ref{fig:11} 展示了部分位于美国华盛顿州的这种纹路 \cite{80}。

\begin{figure}[b]
\begin{center}
% \fbox{\rule{0pt}{2in} \rule{0.9\linewidth}{0pt}}
   \includegraphics[width=1\linewidth]{zhangjiajie.jpg}
\end{center}
   \caption{中国南部张家界国家森林公园中的巨型石柱。}
\label{fig:12}
\label{fig:onecol}
\end{figure}

\begin{figure}[t]
\begin{center}
   \includegraphics[width=1\linewidth]{hoy.jpg}
\end{center}
   \caption{苏格兰的海上石柱Hoy老人 \cite{83}。}
\label{fig:13}
\label{fig:onecol}
\end{figure}

内陆侵蚀结构也可以由 ECDO 之类的地球翻转理论作出良好解释。巨型喀斯特景观通过水侵蚀而成,中国南部就是典型例子 \cite{82}。其中包括塔状喀斯特、尖峰喀斯特、锥形喀斯特、天然桥梁、峡谷、大型洞穴系统和洼地。其中最壮观的是张家界国家森林公园,那里有巨大的石英砂岩柱 (图\ref{fig:12}) \cite{84}。这些石柱平均海拔超过 1000 米,总数超过 3100 根。其中超过 1000 根柱子高于 120 米,有 45 根高达 300 多米\cite{85}。这些柱状结构类似于海蚀柱 (图\ref{fig:13}),它们是由于海浪导致周围材料崩塌而形成的海岸岩柱。类似的侵蚀景观可以在土耳其乌尔居普的岩锥以及西班牙奇达德·恩坎塔达发现,这两者都位于海拔 1000 米以上。这些地点的附近都发现了一些盐和海洋化石,暗示着过去的海水入侵 \cite{15,86,87}。当然,洪水传说 \cite{3} 提到海洋曾经上升超过 1000 米,安第斯山脉和喜马拉雅山脉数公里以上的盐水和大型盐沼可以为证。例如,玻利维亚的乌尤尼盐沼面海拔高达 3653 米\cite{94}。

\subsection{快速气候变化事件}

现代科学文献承认地球最近历史上存在快速全球气候变化事件。两个显著的例子是 4.2 年和 8.2 千年事件,二者都与人口减少和大范围地区的社会定居点中断相吻合。这些事件被保存在沉积物和冰芯的异常中,化石珊瑚、O18 同位素值、花粉和钟乳石记录以及海平面数据中。推测的气候变化包括全球气温的快速下降、干旱化、大西洋经向翻转环流的破坏和冰川推进 \cite{90,91,92}。尤其是 8.2 千年事件,可能与公元前 6400 年左右黑海的戏剧性盐水泛滥同时发生\cite{93}。

\subsection{考古学中的异常现象}

\begin{figure}[t]
\begin{center}
   \includegraphics[width=1\linewidth]{jericho.jpg}
\end{center}
   \caption{耶利哥塔在公元前7400年左右埋葬的考古重建\cite{95}。}
\label{fig:14}
\label{fig:onecol}
\end{figure}

一些古代城市的考古证据显示出多层涉及掩埋和破坏,形成了过去灾难性事件的记录。耶利哥古城就是这样一个城市,位于现今的巴勒斯坦。它包含了多个破坏层,具有石结构的崩塌和剧烈火灾 \cite{96,97}。层中的年代记录大约从公元前 9000 年到公元前 2000 年。特别值得注意的是其塔楼,似乎在公元前 7400 年左右被剪切并埋在沉积物中 (图\ref{fig:14}) \cite{95}。卡塔尔胡尤克 \cite{99}、格拉马洛特 \cite{98} 以及克里特的米诺斯宫殿 \cite{100,101} 都是类似的考古遗址实例,通常包含多层,常含有破坏的证据。

另一个破坏人类文明的重大灾变的证据是南帕图像,这是在爱达荷州约 100 米厚的熔岩下发现的一个粘土娃娃 \cite{102,103}。发现该雕像的熔岩流估计是在晚第三纪或早第四纪期间沉积,假定已有 200 万年历史。然而,该地区的熔岩流看起来相对较新。这样的发现不仅指向了破坏文明的重大灾变,也对现代年代测定的时间表提出了质疑。

\section{关于现代年代测定方法}

有充分的理由对现代年代学持怀疑态度,这些年代学将各种物理材料的年龄归为数百万甚至数亿年。

传统的说法是,所谓的“化石燃料”如煤、石油和天然气已有数亿年的历史 \cite{104}。然而,在墨西哥湾对石油的实际碳测年结果发现该石油的年龄约为 13,000 年 \cite{105}。碳-14 的半衰期非常短 (5,730 年),理论上它应该在几百万年后完全衰变。然而,碳-14 已在据称已有几千倍更长历史的煤炭和化石中被发现 \cite{106}。事实上,在实验室中,在受控条件下 (主要是高热) 在短短 2-8 个月内即可生产出人造煤 \cite{107}。

除碳测年外的放射性同位素测年方法也可能不准确。创世纪答案研究小组发现了这些方法导出的日期的不一致性,这引发了对其真实性的质疑 \cite{108}。甚至在据称已有一亿年历史的恐龙遗骸中发现了含有血细胞、血管和胶原蛋白的软组织 \cite{109,110}。根据我们的了解,地球地质时间尺度和诸如岩石和化石燃料等物质的传统接受的年龄可能是错误的,误差可能达数个数量级。

\section{结论}

在这篇论文中,我介绍了那些引人注目的,预示着大灾难源头的异常现象,并以 ECDO 地球翻转理论做出了最完美的解释。虽然提出的记载繁多,但并不完整。我还整理了更多异常现象,并在我的 GitHub 研究资源库中公开 \cite{2}。

\section{致谢}

感谢 ECDO 理论的原作者 Ethical Skeptic,是他完成了这个深刻而又创新理论,并将其分享给全世界。他的三篇论文 \cite{1} 对于放热性地核-地幔分离詹尼别科夫振荡 (ECDO) 理论来说仍是权威文献,而且相对于我在此简要总结的内容来说甚是详细。

当然,还要感谢那些巨人的肩膀;那些做出所有研究和调查,使这项工作成为可能并努力为人类带来光明的人们。

\clearpage
\twocolumn

{\small
\renewcommand{\refname}{参考文献}
\bibliographystyle{ieee}
\bibliography{egbib}
}

\end{document}

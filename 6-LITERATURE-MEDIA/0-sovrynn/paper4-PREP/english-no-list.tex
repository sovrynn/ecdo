\documentclass[10pt,twocolumn,letterpaper]{article}

% My own stuff
\usepackage{booktabs}
% \usepackage{caption}
% \captionsetup[table]{skip=8pt}   % Only affects tables
\usepackage{stfloats}  % Add this to the preamble
\usepackage{float}

\usepackage{cvpr}
\usepackage{times}
\usepackage{epsfig}
\usepackage{graphicx}
\usepackage{amsmath}
\usepackage{amssymb}

% Include other packages here, before hyperref.

% If you comment hyperref and then uncomment it, you should delete
% egpaper.aux before re-running latex.  (Or just hit 'q' on the first latex
% run, let it finish, and you should be clear).
\usepackage[hyphens]{url}
\usepackage[breaklinks=true,bookmarks=false]{hyperref}

\cvprfinalcopy % *** Uncomment this line for the final submission

\def\cvprPaperID{****} % *** Enter the CVPR Paper ID here
\def\httilde{\mbox{\tt\raisebox{-.5ex}{\symbol{126}}}}

% Pages are numbered in submission mode, and unnumbered in camera-ready
%\ifcvprfinal\pagestyle{empty}\fi
\setcounter{page}{1}
\begin{document}

%%%%%%%%% TITLE
\title{ECDO Paper 4/4: Civilian's Guide to ECDO Survival - Future Steps}

\author{Junho\\
Published July 2025\\
Website (Download papers here): \href{https://sovrynn.github.io}{sovrynn.github.io}\\
ECDO Research Repo: \href{https://github.com/sovrynn/ecdo}{github.com/sovrynn/ecdo}\\
{\tt\small junhobtc@proton.me}
}

\maketitle
%\thispagestyle{empty}

%%%%%%%%% ABSTRACT
\begin{abstract}
In May 2024, a pseudonymous online author by the name of “The Ethical Skeptic” \cite{0} shared a groundbreaking theory called the Exothermic Core-Mantle Decoupling Dzhanibekov Oscillation (ECDO) \cite{1}. This theory suggests that Earth has previously experienced sudden, catastrophic shifts in its rotational axis, triggering massive worldwide floods as the oceans spilled over the continents due to rotational inertia. Additionally, it presents an explanatory geophysical process and data indicating that another such flip may be imminent. While such cataclysmic flood and doomsday predictions are not new, the ECDO theory is uniquely compelling due to its scientific, modern, multidisciplinary, and data-based approach.

This paper is my fourth work \cite{2,3} on this topic, and building on the information discussed in my three previous papers, provides a civilian guide on the best way to prepare for an imminent ECDO cataclysm:
\begin{flushleft}
\begin{enumerate}
    \item What physical dangers and risks exist, and how they might be prepared for.
    \item Geopolitical and modern mass-murder crowd-control weapon(s) risks.
    \item The best locations to survive an ECDO cataclysm.
\end{enumerate}
\end{flushleft}

This paper will assume the reader is familiar with my prior three papers \cite{3}, which provide a comprehensive overview of academic and political parts of the ECDO theory which will not be repeated or cited again in this paper for the sake of brevity.

This paper aims to be a definitive data-driven guide on preparing for a potential imminent ECDO cataclysm based on one year of independent, comprehensive, focused research on the topic.
\end{abstract}

%%%%%%%%% BODY TEXT
\section{Introduction}

\begin{figure}[b]
\begin{center}
   \includegraphics[width=1\linewidth]{noah.jpg}
\end{center}
   \caption{A depiction of Noah's ark during the flood. The ark may be parable, but we may live to see another such "flood" in our lifetimes. \cite{2}}
\label{fig:1}
\label{fig:onecol}
\end{figure}

Objectively, the physical dangers posed by an ECDO cataclysm can be summarized as follows:

\begin{flushleft}
\begin{enumerate}
    \item The flooding of a majority of Earth's continental landmass due to the oceans splashing over the land when the Earth's surface flips.
    \item The lack of protection from a weakened and disabled geomagnetic field.
    \item The tectonic and volcanic effects of a rapid movement of the Earth's surface over inner layers.
    \item High-speed surface winds and flying debris due to a rapid crustal shift through the atmosphere.
    \item Continental geoelectric shocks sparked by a rapid displacement of inner Earth layers.
\end{enumerate}
\end{flushleft}

A basic approach to mitigating these dangers will be discussed.

Additionally, an ECDO cataclysm creates significant geopolitical risks for civilians due to humanity's government-manipulating crab-heap fiat warlocks and witches attempting to take advantage of the situation to weaken/eliminate adversaries and enslave people through fear and military might. Various kinds of new, modern weapons may be used and civilians may find themselves being collateral casualties or even primary targets. This will be addressed in the paper.

Finally, I will include a comprehensive list of "Junho-endorsed safe spots" based on various metrics and data sources.

\section{Physical Dangers}

An ECDO cataclysm involves the decoupling of the crust and mantle from the inner and outer core of the Earth, leading to a rapid departure of the Earth's surface from its normal rotational axis due to rotational physics \cite{1}. In this section, I dive into five categories of physical dangers to human life that would occur during such an event and provide evidence from past events that backs up the existence of each. Recall Table \ref{tab: 1} from the first paper categorizing various catastrophic effects mentioned across 117 ancient cataclysm stories. Some previously omitted categories have been included:

\begin{table}[ht]
\begin{center}
\renewcommand{\arraystretch}{1.2}  % Optional, to increase row spacing
\begin{tabular}{|l|c|c|}
\hline
\textbf{Cataclysm Type} & \textbf{Count} & \textbf{Occurrence \%} \\
\hline\hline
Deluge/flood            & 84 & 71.79 \\
Conflagration/firestorm & 39 & 33.33 \\
Topographical changes   & 29 & 24.79 \\
Thunder/noise/din       & 16 & 13.68 \\
Stellar derangement     & 15 & 12.82 \\
Collapsed sky           & 15 & 12.82 \\
Prolonged darkness      & 14 & 11.97 \\
Lost lands and lakes    & 12 & 10.26 \\
Aerial debris           & 11 & 9.4   \\
Cyclonic winds          & 10 & 8.55  \\
Lightning               & 10 & 8.55  \\
Falling celestial visitor & 9 & 7.7 \\
Axial/rotational changes & 9 & 7.69  \\
Boiling rivers/lakes/oceans & 8 & 6.84 \\
\hline
\end{tabular}
\end{center}
\caption{Occurrences of Catastrophic Effects in Stories \cite{12}}
\label{tab: 1}
\end{table}

\subsection{Oceanic Megaflood}

The first and most obvious danger is the oceans splashing over the continents \cite{3}. This is well documented in an expansive collection of ancient flood testimonies, and the presence of salt and marine fossils on a large majority of Earth's continental surface. The seawater erosion line on the Khafre pyramid, receding ancient coastlines on the Arab peninsula \cite{1} and the Chinese Gun-Yu prolonged flood stories also suggest that Earth may temporarily have different coastlines, and that certain areas may experience prolonged submergence, after an ECDO event.

In particular, the same evidence suggests that the oceans will rise \textit{very high}. Testimonies that the ocean rose up to mountaintop heights of several thousand meters and also flooded the Himalayas should be taken into account here.

Flood stories also suggest that the oceanic megaflood will happen in multiple waves; I recall the Makah Native-American flood story from my first paper \cite{3}: \textit{"The ocean rose high enough to cut off the cape. Then it withdrew, reaching its low ebb four days later, leaving Neah Bay high and dry. Then it rose again to cover all but the mountain tops."} This makes sense when one considers that the ocean will be \textit{"splashing around"} for quite a bit as the Earth's surface spins around before finding a new axis. The oceanic waves (read: mountain-topping tsunamis) caused by the flip will be lapping back and forth between cross-oceanic continental shores.

\begin{figure}[t]
\begin{center}
   \includegraphics[width=1\linewidth]{aquifer.jpg}
\end{center}
   \caption{A map of a deep aquifer (underground lake) in Colorado, USA. Seneca was evidently aware of the existence of these.}
\label{fig:2}
\label{fig:onecol}
\end{figure}

The water may come not only from the oceans but also from under our feet. This is well documented in geysers, aquifers, and the water tables from which wells draw water. It's also theorized that the inside of the Earth may contain multiple times more water than what is on the Earth's surface \cite{4}. Such water may also well up to the surface of the Earth during an ECDO event. Lucius Annaeus Seneca the Younger, a Stoic philosopher of Ancient Rome and a contemporary of Jesus Christ \cite{7}, had this to say on the role of water in Earth's recurring cataclysms: \textit{"Some suppose that in the final catastrophe the earth, too, will be shaken, and through clefts in the ground will uncover sources of fresh rivers which will flow forth from their full source in larger volume... The nearer the part is to the soil that is being liquefied, the more quickly will it be washed off, dissolved, and finally carried away. The rock will everywhere gape in fissures, and the fresh supplies of water will leap down into the gulfs, and unite in forming one great sea. There will be no Adriatic any longer, no strait in the Sicilian Sea, no Charybdis, no Scylla. All the fabulous dangers will be swallowed up in the new sea; the existing Ocean which surrounds the fringes of the earth will come into the centre... A single day will see the burial of all mankind. All that the long forbearance of fortune has produced, all that has been reared to eminence, all that is famous and all that is beautiful, great thrones, great nations all will descend into the one abyss, will be overthrown in one hour. Remember, too, that there are huge lakes hidden deep in the earth, great quantities of sea stored up, and many rivers that glide through the unseen depths. On all sides, therefore, will be causes of deluge; for some waters flow in beneath the earth and others flow round it. Though long restrained they will at last prevail, and will join stream to stream and pool to marsh. The sea will fill up the mouth of every fountain, and will open it out to wider extent. Just as the bowels drain the body in the draught, or as the strength goes off into perspiration, so the earth will dissolve, and though other causes are inactive, it will find within itself a flood in which to sink. All the great forces will thus, I should suppose, combine. Nor will destruction tarry. The harmony is assailed and broken when once the world has relaxed aught of its needed care. At once, from all sides, open and hidden, above and beneath, will rush the influx of waters"}. \cite{8}

In closing, flood stories indicate that the Earth may experience many days of mountain-topping oceanic flooding before settling back to potentially different coastlines for a period of anywhere from 20 to 60 years (China, Aztec stories) \cite{5,6} before things start to change again.

\subsection{Weakened Geomagnetic Field - Fire}

The Earth's geomagnetic field is well established and measured. Paleomagnetic data indicates that the field's dipole moment has been weakening over the last 2000 years \cite{9} (Figure \ref{fig:3}). It's theorized that spinning magma columns inside the Earth generate the geomagnetic field. At the very least, it's clear that Earth is a giant magnet, and that it is breaking down.

\begin{figure}[t]
\begin{center}
   \includegraphics[width=1\linewidth]{paleo.png}
\end{center}
   \caption{Paleomagnetic and archaeomagnetic measurements indicate that Earth's geomagnetic dipole moment has been weakening over the last 2000 years. Measurements get more sparse further back in time. \cite{11}}
\label{fig:3}
\label{fig:onecol}
\end{figure}

Earth's geomagnetic field shields the Earth from cosmic debris, and perhaps more importantly, particles that are emitted from the Sun (Figure \ref{fig:4}). The equatorial region of the Earth's surface, which faces the Sun, is significantly warmer than the poles, demonstrating the immense energy transmitted from the Sun to Earth.

\begin{figure}[t]
\begin{center}
   \includegraphics[width=1\linewidth]{solargeo.jpg}
\end{center}
   \caption{A depiction of the Sun's interaction with Earth's geomagnetic field \cite{10}.}
\label{fig:4}
\label{fig:onecol}
\end{figure}

One has to expect that the geomagnetic field will be at its very weakest when the Earth flips. What might be the consequences of this with regards to the Sun's energetic emissions? Once again, ancient cataclysm stories may tell us the answer. In Table \ref{tab: 1}, after flooding, \textit{conflagrations and firestorms} were the second most commonly reported occurrence, being mentioned in 33\% of all stories. Some of the more notable mentions from these stories include Greek mythology, which has themes of alternating destruction by kataklysmos (water) and ekpyrosis (fire) \cite{13}, and Aztec mythology, which says the third sun ended with a \textit{"rain of fire"} \cite{14}. One particularly gruesome story from Australia goes as follows: \textit{"Wildfire is also recorded as blowing the great rivers out of their beds, as the oral histories of the Indians of California and the traditions in the Near East around the Euphrates record. The Aborigines in southern Australia gave an account of the horrible heat descending from a red-hot sky. The heat was so extreme that people could not endure it, and the men killed their children and wives, and finally themselves"} \cite{15}.

\subsection{Tectonic and Volcanic Effects}

The effects of Earth's tectonic plates during a rapid crustal shift will likely also pose a significant hazard. After conflagration, the next most widely reported occurrence is "topographical changes" (Table \ref{tab: 1}). Here I note one of several myths from the South American Andes region that describe mountain-building occurring in concert with a mountain-topping flood: \textit{"The shepherd and his six children gathered all the food and sheep they could and took them to the top of the very tall mountain Ancasmarca. As the flood water rose, the mountain rose higher, so its top was never submerged, and the mountain later sank with the water"} \cite{17}. It's likely that we will experience significant movements in tectonic plates and their boundaries during an Earth flip - significant subduction and orogeny.

Another factor that comes into play is Earth's equatorial bulge. The Earth is known to form the shape of an \textit{oblate spheroid} due to its plastic layers bulging outward slightly at the equator, where they spin the fastest (Figure \ref{fig:5}). Such an effect is likely not confined to the surface of the Earth, but also exists in the lower layers. If the Earth's outer layers were to shift across the inner layers, not only would we experience immediate and sudden adjustments to the shifting misalignment of the bulges, but soon after, the plastic layers of the Earth would then begin to adjust to form the new oblate spheroid shape dictated by rotational physics around a new surface equator.

\begin{figure*}[t]
\begin{center}
\includegraphics[width=0.9\textwidth]{rotate2.png}
\end{center}
   \caption{A visualization of average wind speeds in a 6 hour, 104 degree rotation \cite{2}.}
   \label{fig:6}
\end{figure*}

\begin{figure}[t]
\begin{center}
   \includegraphics[width=1\linewidth]{bulgecrop.png}
\end{center}
   \caption{A depiction of the Earth's equatorial bulge (with some other extraneous notes) \cite{16}.}
\label{fig:5}
\label{fig:onecol}
\end{figure}

Such processes also have significant ramifications when it comes to volcanism. Multiple mentions are made of hot or even boiling water (Table \ref{tab: 1}) in cataclysm stories. For example, a flood myth from Taiwan says, \textit{"In an earthquake, mountains tumbled down, the earth gaped, and hot subterranean waters gushed out and flooded the whole earth."} \cite{17}. Taiwan lies right on the edge of major plate boundaries in the Asian Pacific ocean, indicating that the waters around it were likely heated from underneath, implying significant volcanic activity occurs as a result of the flip.

Based on these testimonies, we will likely see a convergence of tectonic and volcanic activity worldwide; the scale of which has likely never been measured before in recent human history.

\subsection{High-Speed Surface Winds}

Cyclonic winds and aerial debris are mentioned in many of the cataclysm stories (Table \ref{tab: 1}). This makes sense when you consider that the Earth's surface will be moving through the atmosphere across enormous distances in a short time. A story from Native Americans in California tells how, \textit{"People came into existence and dwelt a long, long time. Then one of them dreamed of a whirlwind, and the others said he had dreamed something bad. After that it blew, and the wind increased. The world was going bad. At noon they all went into an earth lodge. It blew terribly. Trees fell down westward. The one who had dreamed stayed outside and told the others it was raining, the water was coming, the earth will be destroyed. All the other houses were blown away. He came into the earth lodge and leaned against the pole. At last the pole came loose too. The one who dreamed was the last destroyed of all the people"} \cite{17}. Figure \ref{fig:6} depicts average rotation speeds, which vary depending on the location and rotation, in a worst-case scenario of a 104 degree rotation in 6 hours. The speeds are significantly faster than the speed of sound. At the very least, we should expect wind speeds to be as fast as the strongest hurricanes recorded in recent history, which are already fast enough to cause significant damage to manmade structures.

\subsection{Massive Geoelectric Shocks}

There is evidence that the surface of the Earth may also experience massive geoelectric shocks emanating from inside of the Earth. Tiny static electricity from rubbing one's hands on a blanket and giant lightning shocks are caused by the same reason - imbalance of electric charge. What then, might happen, if the giant inner layers of the mantle and inner core, which are already giant magnets, decouple and rotate across each other?

\begin{figure}[t]
\begin{center}
   \includegraphics[width=1\linewidth]{lightning.jpg}
\end{center}
   \caption{A spectacular shot of upward lightning. Yes, we live on a giant spinning ball of electric material. \cite{22}}
\label{fig:7}
\label{fig:onecol}
\end{figure}

There are various kinds of archaeological evidence of melting at extremely high temperatures. One great example is the vitrified forts of Scotland (Figure \ref{fig:8}). Geologist Robert Schoch provides his opinion on them: \textit{"About 60 ancient stone forts in Scotland have vitrified walls, with the stone melted into a glasslike state. Why these forts are vitrified has never been adequately explained by archaeologists... The stone walls and structures have been subject to intense heat, to the point where some of the rocks have melted and fused, forming masses of material that can resemble the slag produced when smelting and refining metals, similar to materials produced during some types of volcanic eruptions... In order to melt and vitrify the rock, incredibly high temperatures are required - on the order of 1,000 degrees centigrade... Schoch suggested that an ancient plasma event occurred and the forts sitting at high elevations could have served as lightning rods, attracting intense electrical currents that resulted in their vitrification. Although most Scottish vitrified forts have been dated to the Iron Age, Schoch believes they could be much older, especially the Mote of Mark which he claims was occupied from 8000 BCE to the 9th century CE... Could at least some of the vitrification be due to plasma events? I believe so, and that some of the forts date as far back as 10,000 years ago or more, just after the end of the last ice age, a time when I believe a major plasma event (or events) occurred on Earth, probably evidenced by the nitrate data"} \cite{18,19}. Such sites also exist in the Middle East \cite{20,21}.

\begin{figure}[t]
\begin{center}
   \includegraphics[width=1\linewidth]{vitrified.jpeg}
\end{center}
   \caption{A vitrified fort in Scotland \cite{51,52}.}
\label{fig:8}
\label{fig:onecol}
\end{figure}

\begin{figure}[t]
\begin{center}
   \includegraphics[width=1\linewidth]{sendai.png}
\end{center}
   \caption{An example of "earthquake light" from a magnitude 7.3 earthquake in Sendai, Japan in 2022 \cite{48}.}
\label{fig:9}
\label{fig:onecol}
\end{figure}

There is in fact research that suggests earthquakes can cause electric currents and atmospheric electromagnetic emissions \cite{49,50}. To quote research by Antonio Lira and Jorge A. Heraud, Peruvian researchers, on this topic: \textit{"Electromagnetic emission in the atmosphere usually occurs in relation with charge acceleration between clouds and the Earth’s surface. Lightning is the best known electromagnetic emission in nature and takes place in thunderstorms. We can also ask the question, if it is possible to have electromagnetic emission -for example, a flash of light- in the atmosphere originating in the Earth´s interior. The answer is affirmative... Electromagnetic emission is a secondary effect which can take place in the atmosphere caused by earthquakes"} \cite{50}. Such electromagnetic emissions, in the form of earthquake lights, are depicted in Figure \ref{fig:9}. One must wonder, then, what the effects of a sudden shift of the outer rotational body of Earth, which would involve never-before-measured scales of tectonic activity, would be in light of this information.

\section{A Basic and Primitive Approach to Preparation}

Essentially, in order to survive ECDO, one has to prepare for all the previously discussed risks happening at the same time. Here are some basic guidelines:

\begin{flushleft}
\begin{enumerate}
    \item To avoid the flooding, one should stay at a location of considerable elevation \textit{relative to surrounding terrain} so that water will have plenty of room to flow around the location without submerging it. Based on perusing flood stories, I recommend an in-practice prominence (explained below) of no less than 2000 meters. Such locations that are also significantly inland may be slightly better than coastal ones, and allow you to get away with lower prominence.
    \item To avoid conflagrations, winds, and debris, one should build a shelter underground, preferrably in bedrock to avoid soil liquefaction. Alternatively, a cave may work.
    \item To avoid being a casualty of significant tectonic changes or earthquakes, one should avoid close proximity to significant plate boundaries.
    \item I would personally avoid digging my shelter into a volcano.
    \item To avoid being a casualty of electric discharges which will occur at higher frequencies in high-prominence mountain peaks, one should construct a Faraday cage (Figure \ref{fig:10}) around or into their underground shelter to redirect electric currents.
\end{enumerate}
\end{flushleft}

\begin{figure}[t]
\begin{center}
   \includegraphics[width=1\linewidth]{faraday.jpg}
\end{center}
   \caption{A Faraday Cage protects the contents of the cage from external electrical shocks \cite{54}.}
\label{fig:10}
\label{fig:onecol}
\end{figure}

There are many stories of people surviving by grabbing onto floating objects in the water, riding boats, and staying in trees, but I don't recommend such exciting activities. When selecting a location, I recommend using Google Earth Pro and scanning the topography to look for areas that have high prominence (click the scroll wheel for easy navigation). Additionally, you can evaluate the geological layers of the area to make sure you don't have significant sedimentation or missing geological layers.

\subsection{Prominence - Best Metric For Avoiding Water}

Prominence is defined as the elevation of a summit relative to the highest point to which one must descend before reascending to a higher summit, depicted in Figure X. In order to understand why I recommend prominence and not absolute elevation as the primary metric for flood safety, you have to understand that the Earth is extremely flat. If we imagine the Earth's ocean as a puddle, then if the puddle is about 6 feet deep at its deepest, the highest you can go to get away from this water is about 4-5 feet (Earth's surface). More importantly, the puddle would be about \textbf{18,000 feet long and wide}. With these ratios, there is no absolute height that the water cannot get to if it needs to. This is evidenced by extensive salt deposits in the Andes and Himalayan mountain ranges, the two largest areas of high-elevation land in the world.

\begin{figure}[t]
\begin{center}
   \includegraphics[width=1\linewidth]{prominence.jpeg}
\end{center}
   \caption{A visual example of prominence ("p") \cite{53}.}
\label{fig:11}
\label{fig:onecol}
\end{figure}

What is more important than absolute elevation is having relative elevation above the surrounding land so that any water that flows through the area will have plenty of room to go around the higher location. High prominence peaks on the coast will be safer than high elevation, low prominence valleys in the Himalayas. In closing, I will provide some final notes:

\begin{flushleft}
\begin{enumerate}
    \item Prominence is an approximate metric, as it has a very specific definition. You could have a high-prominence plateau that is not reflected in normal prominence values since prominence is specifically for mountain peaks. The precise academic prominence value is an underestimation of the in-practice prominence.
    \item For land that is surrounded by water, you should generally incorporate the depth of the surrounding water body into the prominence.
    \item Prominence is just a metric - water will interact with each topography differently. The most authoritative prediction would thus come from a high-fidelity, multi-material fluid simulation that models water flow for each area of the world. The computational power required for such a fluid simulation, however, is extensive.
    \item An alternative may be generating a custom topographical map based on relative elevation, similar to prominence.
\end{enumerate}
\end{flushleft}

\subsection{Basic Post-Flip Geophysical Timeline}

When the flip occurs, you will probably want to stay in your shelter for at least a week straight until the outer rotational body of the Earth finds a new rotational axis. Afterwards, the majority of the continental landmass will be wiped clean, buried in sediment, or submerged by water. Pretty much everything humanity has built will be gone. There will likely be a lot of water lingering around on surface basins. I expect the "sea level" to be higher than it is today, on average, due to the displacement of the Pacific Ocean away from the equatorial bulge. The weather will be disrupted for decades, and the temperature may be significantly different from before depending on the new proximity to the equator i.e. if you end up at the poles, you might freeze to death. This period is expected to last multiple decades. Eventually, as the exothermic core process that led to the decoupling reverses into an endothermic process, the Earth may gradually return to its previous state, cooling down significantly as it does.

At the end of the paper, I provide a comprehensive list of likely-safe mountaintops based mainly on flood myths and mountainpeak prominence.

\section{Geopolitical and Mass-Murder Weapon Risks}

In addition to geophysical risks, humanity must contend with geopolitical and modern mass-murder crowd-control weapon(s) that may be used by certain individual(s) or group(s) in an attempt to eliminate adversaries, keep populations afraid or weakened, and hold on to "power". The truth is, particularly in the Western world, we have little transparency about the motives and intents behind the unelected warlocks and witches manipulating the masses via control of governments. In particular, recent policy decisions, such as the deployment of mass-murder bioweapons disguised as "beneficial and well-intentioned medical treatment" in the form of mRNA vaccines, are quite worrisome, and should leave us concerned regarding what will be sent our way in the near future. The sobering reality is that simply focusing on saving one's own skin may be futile in the face of a real, "hands-off" war, at least when it comes to remaining a sovereign individual. As this topic is very broad and also widely discussed outside of the context of ECDO, I will attempt to walk a fine line of being accurate, comprehensive, and brief in addressing potential military weapons risks while staying relevant to ECDO.

\subsection{"World War III"}

The whistleblower testimony of a British Senior Mason of the contents of a 2005 meeting of Senior Masons in London, published by Project Camelot, was discussed in my third paper. It was discussed that senior Western individuals planned to instigate a limited nuclear exchange targeting Iran and China, and then unleash biological weapons. Afterwards, Project Camelot claims, \textit{"This is intended to be just the beginning. After this, a full nuclear exchange would be triggered: the "real" war, with widespread destruction and loss of life. Our source tells us that the planned population reduction through these combined means is 50\%. \textbf{He heard this figure stated in the meeting.}"} \cite{23}. The motives of course, were to eliminate adversaries, with China being the primary target, so that after the geophysical cataclysm, there would be no contenders in terms of having the most military might. Today, we have already seen significant conflict in Ukraine and the Middle East, and tactical nukes have likely already been used in combat. In addition, biological weapons have already been deployed across the world in the form of mRNA vaccines, resulting in injury, sterilization, and death of significant amounts of people. Although the world has obviously changed in the 20 years since the meeting, one has to wonder whether the main targets and overarching plan discussed in the meeting remain intact, including the planned \textbf{50\% population reduction through hands-off nuclear war}. I can assure you, if you are reading this, you probably won't be the beneficiary of a \textbf{50\% population reduction} achieved through \textbf{nuclear war}.

\subsection{Recent Tactical Nuke Usage in Ukraine and the Middle East}

It has come to my attention that while humanity in the "developed world" was swimming in financial degeneracy and 21st century silicon distractions, a multi-decadal and still ongoing nuclear war was occurring on the other side of the world. In particular, it seems that in the population of West Asia, this is widely known among the general population due to being primary victims of a majority of these nuclear weapons. Cirnosad, a pseudonymous geopolitical analyst and nuclear expert primarily operating on Twitter/X, has analyzed many of these nuclear explosions and explained how they can be identified \cite{24}. Nuclear explosions have several unique characteristics that can be used to identify them - a rapidly vertical rising smoke plume caused by a fluid dynamics phenomenon known as a Rayleigh-Taylor instability, a bright flash, shockwave, plasma surrounding the top of the smoke cloud, a shotgun sound, and unique seismic and sound signatures. In addition, radioactivity samples taken from the impact site and unique burn signatures on unfortunate victims can provide additional confirmation. While I have identified a dozen explosions from the last decade shared on his account and verified that they matched the unique signatures of a nuclear weapon, other sources indicate that the usage of nuclear weapons extends even earlier than a decade, going back all the way till 2003 in West Asia, and potentially even earlier. Overall, it is likely that dozens of nuclear weapons have been detonated in mass-murder rituals over the last several decades in NATO-occupied Russian-Ukrainian territory, Russia, and in multiple countries in West Asia.

It's important to note that most of these nukes have been so-called "tactical" nukes. Nuclear technology seems to have been refined significantly since first murder usage in Japan at the end of WWII. Nukes have become more efficient, producing less radiation, and smaller versions have been created that enable precise limited destruction and can be used with plausible deniability \cite{29}. Cirnosad estimates the largest nuke used in recent murder to be the Tver explosion in Russia, with a yield of approximately 11 kT \cite{24} (Figure \ref{fig:12}). For reference, the Hiroshima and Nagasaki bombs had yields of 15 and 21 kT \cite{30}.

In West Asia in particular, nations have accused Israel and the United States of using nuclear weapons. There have been many cases of birth defects \cite{26,27} and unique burn signatures on victims which would be caused by nuclear weapons \cite{28}. Heightened radiation levels, depicted below, have also been measured in various explosion sites. In any case, attacks have already been publicly acknowledged on Russia's Zaporizhzhia nuclear plant \cite{60} and Iran's nuclear facilities \cite{61}, representing clear and dangerous nuclear escalations.

The full list I've found so far, which is likely very incomplete, indicates the following locations may have been struck by nuclear weapons at the specified dates over the last several decades \cite{24}:

\begin{flushleft}
\begin{enumerate}
    \item Fallujah, Iraq, 2003
    \item Lebanon, 2006
    \item Gaza, 2008
    \item Homs, Syria, 2013
    \item Yemen, 2014/2015
    \item Beirut port, Lebanon, 2020
    \item Engels Airbase, Russia, 2022
    \item Tartous, Syria, 2024
    \item Bryansk, Russia, 2024
    \item Kharkov, NATO-occupied Ukraine, 2024
    \item Toropets, Tver, Russia, 2024
    \item Dummar, Syria, 2024
    \item Zhytomyr, NATO-occupied Ukraine, 2025
\end{enumerate}
\end{flushleft}

I believe such data constitutes substantial proof that the timeline discussed in the "Anglo-Saxon Mission" has been progressing full-steam ahead since its 2005 status was first shared in 2010 - a premeditated nuclear war commenced for the purposes of securing power in a post-cataclysmic world - and that we may not have yet seen the worst of it.

\begin{figure*}[t]
\begin{center}
\includegraphics[width=1\textwidth]{tver.png}
\end{center}
   \caption{A recording of the 2024 Tver explosion next to a 11kT nuclear test (Operation Plumbbob) from 1957, showing a rapidly rising mushroom cloud (Rayleigh-Taylor instability) caused by a nuclear explosion \cite{24}.}
   \label{fig:12}
\end{figure*}

\begin{figure*}[t]
\begin{center}
\includegraphics[width=1\textwidth]{toropets.png}
\end{center}
   \caption{Satellite image analysis of an explosion in Toropets, a Russian military site. Evident is an airburst explosion that knocked down trees without setting fire to everything, clear evidence of an aerial nuclear explosion \cite{24}.}
   \label{fig:13}
\end{figure*}

\begin{figure*}[t]
\begin{center}
\includegraphics[width=1\textwidth]{nukes.png}
\end{center}
   \caption{Chart showing elevated radiation levels in West Asia across multiple explosion sites - by Chris Busby, a PhD research director in Latvia \cite{32,33}}
   \label{fig:14}
\end{figure*}

\begin{figure*}[t]
\begin{center}
\includegraphics[width=1\textwidth]{tartous.jpeg}
\end{center}
   \caption{Analysis of radiation levels in Tartous, Syria after the 2024 explosion by Swiss PhD physicist Hans-Benjamin Braun: \textit{"Recently reported radiation dose rate near Tartus/Syria ground-zero (GZ) after nuclear attack on Dec 15, ’24, exceeds that of Nagasaki and Hiroshima GZs by more than 100\% and 50\% respectively, effectively dwarfing the latter two in comparison"} \cite{34}.}
   \label{fig:15}
\end{figure*}

\begin{figure*}[t]
\begin{center}
\includegraphics[width=1\textwidth]{sound.jpeg}
\end{center}
   \caption{Analysis of unique nuclear explosion sound signature in Tartous, Syria: \textit{"And yet it was a nuke in Tartus: Seismogram together with one used to identify a N-Korean test.  Both are in clear contrast to the (“delayed”) explosion in Khmelnytskyi (’23) proving the nuclear nature of the >0.3kt expl in Tartus, thus firmly invalidating the ammo depot hypothesis"} \cite{34}.}
   \label{fig:16}
\end{figure*}

\clearpage
\twocolumn

\begin{figure*}[t]
\begin{center}
\includegraphics[width=1\textwidth]{japan.png}
\end{center}
   \caption{Analysis of 21 million Covid-19 vaccination records in Japan shows an increase in mortality spiking at 90-120 days post-vaccination \cite{40,41}.}
   \label{fig:17}
\end{figure*}

\subsection{Bioweapons}

Another important topic in the "Anglo-Saxon Mission" was the covert use of "biological weapons". In late 2019, the world had its first apparent Covid-19 case in Wuhan, China. In early 2020, a "pandemic" was declared, and roll-out of mRNA vaccines began quickly in late 2020, with large portions of the world population lining up to get a foreign, purportedly untested and quickly-developed substance injected into their bloodstream because of the "pandemic emergency" \cite{31}. As of late, it has become clear that the vaccines, far from being a quickly-developed substitute that would protect its recipients from an unknown "coronavirus", may have actually been biological weapons intended to quietly injure, sterilize, and kill recipients. As these testimonies are quite expansive now, I will simply cite a few of them below.

Professor Masanori Fukishima, Japan's most senior oncologist, professor emeritus at Kyoto University, has publicly spoken negatively about the Covid mRNA vaccines multiple times \cite{35,36}: \textit{"Until now, vaccines were proteins, and when injected, they didn't go elsewhere so easily. But the moment they were put into nanoparticles, they enter the bloodstream and go everywhere... So, it goes to the brain, to the tips of your nails, from the top of your head to your toes - it goes everywhere. The government misrepresented this, saying it stays there and produces antibodies. No way. That's a joke. If it goes to your head, it enters your brain too, and it goes into cells all over the place. And there, it starts producing antigens. The antigen is the spike protein. The spike protein, so kindly, is extremely toxic. Very toxic. Mitochondria, which produce energy in cells - the energy production factories in cells - it destroys them"} \cite{37,38}. Mike Yeadon, a former Pfizer vice president, has also said the same thing: \textit{"The purported vaccines against this alleged illness Covid-19 were, in my view, deliberately designed intentionally to injure, kill and reduce fertility"} \cite{39}. An analysis of 21 million Covid-19 vaccination records in Japan, led by Professor Yasufumi Murakami and Masako Ganaha, showed a peak in deaths 90-120 days post vaccination, with the time till death shortening when additional vaccine doses were received \cite{40,41}.

As this is a well-discussed topic, I will close by saying that there may still be more bioweapons in the arsenal.

\begin{figure*}[t]
\begin{center}
\includegraphics[width=1\textwidth]{tes1.jpeg}
\end{center}
   \caption{Analysis by the Ethical Skeptic showing a trendline increase in non-Covid natural cause mortality in the US post-vaccination \cite{42}.}
   \label{fig:18}
\end{figure*}

\begin{figure*}[t]
\begin{center}
\includegraphics[width=1\textwidth]{tes2.png}
\end{center}
   \caption{Analysis by the Ethical Skeptic showing a trendline increase in malignant neoplasms in the US post-vaccination \cite{42}.}
   \label{fig:19}
\end{figure*}

\clearpage
\twocolumn

\begin{figure*}[t]
\begin{center}
\includegraphics[width=1\textwidth]{farmcrop.PNG}
\end{center}
   \caption{A map of Earth's land used for agriculture, by USGS \cite{45}.}
   \label{fig:20}
\end{figure*}

\subsection{False Messiahs}

While mind-controlling, lying to, deceiving, poisoning, and sedating the public clearly aid in the task, ultimately, rule without blatant displays of force is based on consent of the governed. In an attempt to maintain public consent, in addition to false flags, the warlocks and witches manipulating the masses may attempt to deploy a "false messiah" trojan horse to steer the public narrative towards "false opposition" that they control, in the same way that political theatrics consumers are presented with a false dichotomy choice between the "Republicans" and "Democrats" which in reality, are all part of the uniparty "band of brothers".

\subsection{Additional Secret Weapons and Dangers}

This last section briefly mentions relatively unsubstantiated claims and rumors that I will mention merely for the sake of comprehensiveness. I have previously come across a claim that the American military has on standby biological agents that can be quickly deployed and sprayed onto major population centers to neutralize or liquidate large amounts of the population. Eric Hecker, a former Raytheon employee who worked as a firefighter and plumber in Antarctica, claims that there are massive "Directed Energy Weapons" (DEWs) constructed in Antarctica that can cause earthquakes by transmitting energy towards a specific location \cite{43,44}. There are also many rumors flying around about UFO craft powered by unknown propulsion methods, and that such technology may be possessed and hidden in secret military programs. The Pentagon has supposedly funded programs to investigate such phenomenon \cite{59}. Additionally, rumors about "alien/reptilian intelligence" being behind certain Western governments exist \cite{58,55}. Having to contend with such modern mass-destruction or crowd-control weapons and unknown dangers is a legitimate possibility.

\section{Potential Future Timelines}

My thoughts on this topic center around whether or not everyone can be saved given everything that has been discussed in prior sections. While the situation may seem grim, and indeed it is, I undoubtedly believe that everyone indeed can be saved. But in order to do so, humanity must look away from the screens and up at the world around them, resolve to die noble deaths, and fight together to create a better world for those who will inherit in the future. Half-hearted efforts \textbf{will} fail. Humanity must decide whether they are truly sheep or not. I conclude that there are two basic paths forward for humanity in light of an impending geocataclysm, and I discuss them in this section.

\subsection{Can Everyone Be Saved?}

The current world population is estimated to be over 8 billion people \cite{47}. The population is sustained by about 38 percent of Earth's surface that is used for agriculture \cite{46} (Figure \ref{fig:20}) and almost all of it is in low lying areas that will be wiped out by an oceanic displacement. 

The short of it is, it will be very difficult, and so long as the people who control nuclear arsenals are planning to drop them over the world and kill billions of people, and then sterilize and kill anyone who can be duped into letting strangers inject foreign substances into their bodies, and humanity is doped up and sedated on distractions and poisons, you can be sure it won't happen.

\subsection{Signals of Escalation in the Timeline}

Once again I direct your attention to the timeline discussed in the "Anglo-Saxon Mission". In 2005, the plan was to create a limited conflict and unleash biological weapons, and then switch to a totalitarian military martial-law footing in Western nations. The next step would be a hands-off nuclear exchange to eliminate adversaries before the geophysical cataclysm. The Senior Mason who gave the 2010 testimony estimated that there could be about 20 years left until the cataclysm. Currently, it is 2025, which means there are 5 years left till that 2030 prediction. The world has already been through a discreet nuclear and significant missile exchange (Ukraine, West Asia) as well as the deployment of mRNA biological weapons, indicating that the timeline has been progressing through its basic steps. Should we see a shift to blatant martial law footing in Western nations, and the commencement of a hands-off nuclear exchange, then we will know we are right on the brink of the geophysical cataclysm.

% If this is not enough cause for concern, Craig Stone, a fellow ECDO-focused researcher \cite{56}, has identified an anomaly in JPL Earth rotational data which began approximately two weeks ago on July 4 2025. This data is updated daily, and the recent anomaly may be the first sign of a step-change since the rapid acceleration of the north pole wander in approximately 1975-1990.

% \begin{figure*}[t]
% \begin{center}
% \includegraphics[width=1\textwidth]{farmcrop.PNG}
% \end{center}
%    \caption{A map of Earth's land used for agriculture, by USGS \cite{45}.}
%    \label{fig:20}
% \end{figure*}

\subsection{Two Paths Forward}

Unless humanity accepts its mistaken path, disarms, unites in peace, and puts all its effort and resources towards constructing facilities for housing people safely in all safe spots worldwide, the answer is no, the majority of people will die and there will be a global hunger games scenario. Massive mountaintop arks for humanity are even possible, but obviously the question is access to resources, and perhaps more importantly, dealing with potential saboteurs. To be very clear, so long as there is a group of warlocks and witches directing an arsenal of weapons of mass destruction and mind-controlled soldiers that they may not be able to maintain through a global cataclysm, we have a very dangerous situation on our hands. No amount of personal preparation may be sufficient. The first step to getting out of the situation is directing everyone's eyes on this topic. What comes after is very open ended.

\section{Conclusion}

This fourth paper concludes, conclusively, all that I have to say on this topic after one year of ECDO-focused research. If the narrative I have constructed is anywhere near accurate, unless things change rapidly, the world will shortly be heading into an apocalypse. I don't expect to write any new papers on this topic anytime soon unless we see a significant progression in the cataclysm timeline. Thank you for your attention to this matter!

\section{Acknowledgements}

Thanks to the Ethical Skeptic \cite{0} for sharing his groundbreaking ECDO thesis with the world. Thanks to Craig Stone (@nobulart) \cite{56} for sharing an expansive amount of knowledge on the topic, a substantial amount of which has been included across my four papers. Special thanks to all individuals whose work I have included and cited in my papers, including but not limited to Catherine Austin Fitts \cite{55}, Richard Sauder, Cirnosad \cite{24}, Hans-Benjamin Braun \cite{57}, and Chris Busby.

\section{Junho-Endorsed "ECDO Safe Spots"}

The following list of safe spots is based on two main metrics: flood myths specifically stating that the flood did not submerge certain areas, and mountain peak prominence lists. I have included the two lists separately. Both lists should be considered highly incomplete but should be an excellent starting point for selecting a safe spot.

For the flood myths, I only included myths that provide a specific name or specifically mentioned that unnamed mountain peaks were safe. Myths that specify floating survivors have been removed, even if they landed on a mountaintop. I encourage you to read through the flood myths yourself. Ultimately, most unique locations with flood myths probably had survivors, but it must be considered that survivors from one location may have migrated to different local regions afterwards.

For the prominence list, I have filtered the mountain peak prominence list so that it only includes peaks of at least 2000m prominence that are not a volcano, which in the vast majority of cases, will be safe. Note that in most cases, the academic prominence value (which is very specific in its definition) will be an underestimate (potentially very huge underestimate) of the "in-practice" prominence, which is determined by how well the specific shape of the local topography allows water to flow through. Thus there are many safe peaks missing from this list. Again, you will be able to get away with less prominence in very inland areas that don't have to deal with as much water.

I also include a shortlist of general areas I would choose to stay in for each continent.

\subsection{General Safe Areas}

\begin{flushleft}
\begin{enumerate}
    \item \textbf{North America:} Rockies, West Coast peaks
    \item \textbf{Mesoamerica:} Mexican highlands
    \item \textbf{South America:} Andes, Guiana Highlands, Serra do Mar peaks
    \item \textbf{Europe:} Alps, Pyrenees
    \item \textbf{Africa:} East African Rift Mountains, South African mountains
    \item \textbf{West Asia:} Mountain peaks
    \item \textbf{Asia:} Himalayas, Southeast Asian archipelago peaks
    \item \textbf{Australia:} Southeast mountainous area
    \item \textbf{New Zealand:} Mountain peaks
\end{enumerate}
\end{flushleft}

\clearpage
\twocolumn

\section{Flood Myth Safe Spots}

\begin{figure}[H]
\begin{center}
   \includegraphics[width=1\linewidth]{flood.jpg}
\end{center}
   \caption{The following safe spot locations are a subset of, and extracted from, the above mapped flood myths \cite{2,17}.}
\label{fig:21}
\label{fig:onecol}
\end{figure}

For each myth, the general region, along with the specified safe location, has been included. Links to the stories have also been embedded. Many of the stories contain names which seem to be antiquated.

\subsection{Asia Flood Myth Safe Spots}

\begin{flushleft}
\begin{enumerate}
\item Zhuang (China), Asia: \href{http://www.talkorigins.org/faqs/flood-myths.html#Zhuang}{Mount Bachi}
\end{enumerate}
\end{flushleft}

\subsection{South America Flood Myth Safe Spots}

\subsection{Mesoamerica Flood Myth Safe Spots}

\subsection{Australia Flood Myth Safe Spots}

\subsection{Europe Flood Myth Safe Spots}

\subsection{North America Flood Myth Safe Spots}

\subsection{West Asia Flood Myth Safe Spots}

\subsection{Pacific Islands Flood Myth Safe Spots}

\clearpage
\twocolumn

\section{2000m Prominence Peak List}

% \begin{figure}[H]
% \begin{center}
%    \includegraphics[width=1\linewidth]{peaks.png}
% \end{center}
%    \caption{A map of the following 365 high-prominence non-volcanic peaks \cite{62}.}
% \label{fig:22}
% \label{fig:onecol}
% \end{figure}

\begin{figure*}[t]
\begin{center}
\includegraphics[width=0.7\textwidth]{peaks.png}
\end{center}
   \caption{A map of the following 365 high-prominence non-volcanic peaks \cite{62}.}
   \label{fig:22}
\end{figure*}

The name of the mountain, along with elevation and prominence values (in that order) have been included.

\subsection{Asia 2000m Prominence Peak List}
\begin{flushleft}
\begin{enumerate}
    \item Chomolhari, Bhutan, China, 7050 m, 2065 m
    \item Kangkar Pünzum, Bhutan, China, 7570 m, 2995 m
    \item Altun Shan, China, 5830 m, 2528 m
    \item Bairiga, China, 6882 m, 2444 m
    \item Barkol Shan, China, 4320 m, 2104 m
    \item Bogda Shan, China, 5445 m, 4122 m
    \item Chakragil, China, 6727 m, 2901 m
    \item Dashennongjia, China, 3100 m, 2270 m
    \item Daxue Shan, China, 3500 m, 2041 m
    \item Ge/nyen, China, 6204 m, 2000 m
    \item Gurla Mandhata, China, 7694 m, 2788 m
    \item Gyalha Peri, China, 7294 m, 2942 m
    \item Helan Shan HP, China, 3540 m, 2098 m
    \item Heyuan Feng, China, 5289 m, 2618 m
    \item Huatou Jian, China, 4750 m, 2223 m
    \item Hung-Wang Shan HP, China, 4330 m, 2470 m
    \item Jiuding Shan, China, 4969 m, 2808 m
    \item Kangze/gyai, China, 5808 m, 2231 m
    \item Kawagebo, China, 6740 m, 2232 m
    \item Kawarani, China, 5992 m, 2018 m
    \item Kertau, China, 3282 m, 2365 m
    \item Kongur Shan, China, 7649 m, 3585 m
    \item Lamo-She, China, 6070 m, 2093 m
    \item Minya Konka, China, 7556 m, 3642 m
    \item Muztagh Ata, China, 7509 m, 2698 m
    \item Namcha Barwa I, China, 7782 m, 4106 m
    \item Norin Kang, China, 7206 m, 2160 m
    \item Nyainqêntanglha Feng, China, 7162 m, 2239 m
    \item Sepu Kangri, China, 6956 m, 2213 m
    \item Shisha Pangma, China, 8027 m, 2897 m
    \item Sulamutag Feng, China, 6245 m, 2028 m
    \item Taibai Shan, China, 3750 m, 2232 m
    \item Tekilik Shan, China, 5470 m, 2337 m
    \item Tomurty, China, 4886 m, 3243 m
    \item Xuebao Ding, China, 5588 m, 2057 m
    \item Xuelian Feng, China, 6627 m, 3068 m
    \item Yaomei Feng, China, 6250 m, 2571 m
    \item Yulong Xueshan HP, China, 5596 m, 3202 m
    \item Kangtö, China, India, 7060 m, 2195 m
    \item Pauhunri, China, India, 7128 m, 2035 m
    \item Gora Alagordy, China, Kazakhstan, 4622 m, 2480 m
    \item Sauyr Zhotasy, China, Kazakhstan, 3840 m, 3252 m
    \item Jengish Chokusu, China, Kyrgyzstan, 7439 m, 4148 m
    \item Dunheger, China, Mongolia, 3315 m, 2075 m
    \item Nayramadlin Orgil, China, Mongolia, 4374 m, 2342 m
    \item Cho Oyu, China, Nepal, 8188 m, 2340 m
    \item Makalu, China, Nepal, 8485 m, 2378 m
    \item Mt. Everest, China, Nepal, 8848 m, 8848 m
    \item Yangra Kangri, China, Nepal, 7422 m, 2352 m
    \item Gasherbrum I, China, Pakistan, 8080 m, 2155 m
    \item K2, China, Pakistan, 8611 m, 4017 m
\end{enumerate}
\end{flushleft}

\subsection{South America 2000m Prominence Peak List}

\subsection{Mesoamerica 2000m Prominence Peak List}

\subsection{Africa 2000m Prominence Peak List}

\subsection{Australia 2000m Prominence Peak List}

\subsection{Europe 2000m Prominence Peak List}

\subsection{North America 2000m Prominence Peak List}

\subsection{West Asia 2000m Prominence Peak List}

\subsection{Pacific Islands 2000m Prominence Peak List}

\subsection{Antarctica 2000m Prominence Peak List}


\clearpage
\twocolumn

{\small
\bibliographystyle{ieee}
\bibliography{egbib}
}

\end{document}

\documentclass[10pt,twocolumn,letterpaper]{article}

% 自定义内容
\usepackage{booktabs}
% \usepackage{caption}
% \captionsetup[table]{skip=8pt}   % 仅作用于表格
\usepackage{stfloats}  % 将此加入导言区
\usepackage{float}
\usepackage{xeCJK}  % Supports Simplified & Traditional Chinese

\usepackage{cvpr}
\usepackage{times}
\usepackage{epsfig}
\usepackage{graphicx}
\usepackage{amsmath}
\usepackage{amssymb}
% 在hyperref之前包含其他包。

% 如果注释掉hyperref后又取消注释,应在重新运行latex前删除egpaper.aux文件。(或者在第一次运行latex时直接按'q'键)
% 运行,让它完成,你应该就清楚了)。
\usepackage[hyphens]{url}
\usepackage[breaklinks=true,bookmarks=false]{hyperref}

\cvprfinalcopy % *** 取消此行注释以用于最终提交
\def\cvprPaperID{****} % *** 在此处输入CVPR论文ID
\def\httilde{\mbox{\tt\raisebox{-.5ex}{\symbol{126}}}}

% 提交模式下页码编号,最终版中不编号
%如果cvprfinal则设置页面样式为空\fi
\setcounter{page}{1}
\begin{document}

%%%%%%%%% 标题
\title{ECDO 论文 4/4:公民 ECDO 生存指南 - 未来步骤}

\author{Junho\\
发布于 2025年7月\\
网站(此处下载论文):\href{https://sovrynn.github.io}{sovrynn.github.io}\\
ECDO 研究仓库: \href{https://github.com/sovrynn/ecdo}{github.com/sovrynn/ecdo}\\
{\tt\small junhobtc@proton.me}
}

\maketitle
%\thispagestyle{empty}

%%%%%%%%% 摘要
\begin{abstract}
2024年5月,一位化名为"伦理怀疑者"的网络作者\cite{0}提出了名为"放热性地核-地幔解耦贾尼别科夫振荡"(ECDO)\cite{1}的开创性理论。该理论认为地球曾经历过自转轴的突然灾难性偏移,由于旋转惯性导致海水漫过大陆,从而引发全球性大洪水。此外,该理论还提出了解释性地球物理过程及相关数据,表明类似翻转可能即将再次发生。虽然此类灾难性洪水与世界末日的预言并不新鲜,但ECDO理论因其科学、现代、跨学科及数据驱动的研究方法而显得尤为引人注目。

这篇论文是我关于该主题的第四部作品\cite{2,3},基于前三篇论文中讨论的信息,为平民提供了应对即将到来的ECDO灾难的最佳准备指南:
\begin{flushleft}
\begin{enumerate}
    \item 存在哪些物理危险与风险,以及如何进行应对准备。
    \item 地缘政治与现代大规模杀伤性人群控制武器的风险。
    \item 在ECDO大灾难中最佳的生存地点。
\end{enumerate}
\end{flushleft}
本文假设读者已熟悉本人之前的三篇论文\cite{3},这些论文全面阐述了ECDO理论的学术与政治部分。为求简洁,本文将不再重复或引用这些内容。

本文旨在基于一年来独立、全面且专注的研究,为可能迫近的ECDO灾难提供一份权威的数据驱动型应对指南。
\end{abstract}
%%%%%%%%% 正文部分
\section{引言}

\begin{figure}[b]
\begin{center}
   \includegraphics[width=1\linewidth]{noah.jpg}
\end{center}
   \caption{诺亚方舟在大洪水中的描绘。方舟或许只是寓言,但我们有生之年可能亲眼目睹另一场类似的"洪水"。 \cite{2}}
\label{fig:1}
\label{fig:onecol}
\end{figure}

客观而言,ECDO灾难造成的物理危害可概括如下:

\begin{flushleft}
\begin{enumerate}
    \item 地球表面翻转导致海洋涌上陆地,淹没大部分大陆板块。
    \item 地磁场减弱失效导致保护缺失。
    \item 地球表层相对内部圈层快速移动引发的构造与火山效应。
    \item 地壳在大气层中快速位移产生的高速地表风及飞溅碎屑。
    \item 由地球内部层快速位移引发的大陆地电冲击。
\end{enumerate}
\end{flushleft}

将讨论缓解这些危险的基本方法。
此外,ECDO大灾变会因人类社会中那些操纵政府的蟹堆法币术士与女巫而给平民带来重大地缘政治风险——他们企图利用局势削弱/消灭对手,并通过恐惧与军事力量奴役民众。各类新型现代武器可能被投入使用,平民或将沦为附带伤亡甚至成为主要攻击目标。这一问题将在论文中予以探讨。

最后,我将根据多项指标和数据来源,整理一份详尽的"俊昊认证安全地点"清单。
\section{物理危险}

ECDO大灾变涉及地球地壳与地幔同内外核的分离,导致地球表面因旋转物理学原理迅速偏离其正常自转轴\cite{1}。本节将深入探讨此类事件中可能危及人类生命的五类物理危险,并提供历史事件证据支持每类危险的存在。回顾首篇论文中表\ref{tab: 1}对117个古代灾变故事所述各类灾难性效应的分类,现补充部分先前遗漏的类别:

\begin{table}[ht]
\begin{center}
\renewcommand{\arraystretch}{1.2}  % 可选,用于增加行间距
\begin{tabular}{|l|c|c|}
\hline
\textbf{灾变类型} & \textbf{数量} & \textbf{发生概率\%} \\
\hline\hline
大洪水/洪水            & 84 & 71.79 \\
大火/火灾风暴          & 39 & 33.33 \\
地形变化               & 29 & 24.79 \\
雷鸣/噪音/喧嚣         & 16 & 13.68 \\
恒星错位     & 15 & 12.82 \\
天穹坍缩           & 15 & 12.82 \\
长夜无尽          & 14 & 11.97 \\
失落的陆地与湖泊    & 12 & 10.26 \\
空中残骸           & 11 & 9.4   \\
旋风                  & 10 & 8.55  \\
闪电                  & 10 & 8.55  \\
天降陨石              & 9 & 7.7 \\
地轴/自转变化         & 9 & 7.69  \\
沸腾的河流/湖泊/海洋  & 8 & 6.84 \\
\hline
\end{tabular}
\end{center}
\caption{故事中灾难性事件的出现次数 \cite{12}}
\label{tab: 1}
\end{table}

\subsection{海洋大洪水}

首要且最显而易见的危险是海水漫过大陆\cite{3}。这一点在大量古代洪水记载中有着充分记录,同时地球上绝大多数大陆表面存在的盐分和海洋化石也佐证了这点。哈夫拉金字塔上的海水侵蚀线、阿拉伯半岛后退的古海岸线\cite{1},以及中国鲧禹治水的长期洪水传说都表明,在ECDO事件后,地球可能会暂时形成不同的海岸线,某些区域或将经历长期淹没。
特别是,同样的证据表明,海洋将上升至\textit{极高}的位置。关于海水曾涨至数千米高的山顶甚至淹没喜马拉雅山脉的记载,在此应予以考虑。

洪水传说还暗示,这场海洋大洪水将以多波次形式发生;我在首篇论文中引用的马卡族美洲原住民洪水故事\cite{3}提到:\textit{"海水先是涨到切断海岬的高度,四天后退至最低点,使尼阿湾完全裸露。而后再次上涨,仅余山巅未被淹没。"} 若考虑到地球表面在找到新自转轴前会持续旋转,导致海水处于\textit{"剧烈晃荡"}状态,这一描述就合乎逻辑了。地轴翻转引发的海洋巨浪(实为漫过山顶的海啸)将在跨洋大陆海岸线之间来回激荡。
\begin{figure}[t]
\begin{center}
   \includegraphics[width=1\linewidth]{aquifer.jpg}
\end{center}
   \caption{美国科罗拉多州一处深层含水层(地下湖)的地图。显然,塞内卡早已意识到这些含水层的存在。}
\label{fig:2}
\label{fig:onecol}
\end{figure}

水源不仅可能来自海洋,更可能来自我们脚下。这一现象在间歇泉、含水层以及井水汲取的地下水位中已得到充分证实。有理论认为,地球内部蕴藏的水量可能是地表水量的数倍之多 \cite{4}。在ECDO事件期间,这些地下水很可能涌出地表。古罗马斯多葛学派哲学家、与耶稣基督同时代的卢修斯·安内乌斯·塞内卡 \cite{7} 曾如此描述水在地球周期性灾变中的作用:\textit{"有人认为在终极灾难来临时,大地亦将震颤,通过地壳裂隙暴露出新河流的源头,这些源自充沛水源的河流将以更浩大的水量奔涌而出...越接近液化土壤的区域,地表物质被冲刷、溶解乃至最终卷走的速度就越快。岩层将处处龟裂,新生的水流将跃入深渊,汇聚成一片汪洋。亚得里亚海将不复存在,西西里海峡将消失无踪,卡律布迪斯漩涡与斯库拉海妖的传说都将被新生的海洋吞噬;环绕地球边缘的现有大洋将向中心倒灌...全人类的葬身之日将在一日之内降临。命运长久忍耐所造就的一切,所有巍峨的文明丰碑,一切显赫与美好,伟大的王座与强盛的国度,都将坠入无底深渊,在一小时内倾覆消亡。切记,地底深处还潜伏着巨大的湖泊,储存着浩瀚的海水,以及无数暗河在不可见的深渊中流淌。因此洪水之因无处不在——既有在地壳下奔流的水脉,也有环绕地核的暗涌。这些被长期禁锢的水体终将突破束缚,溪流相汇成河,沼泽连片为泽。海水将灌入每处泉眼,并将其扩张为更宽广的出口。正如内脏通过引流排空体液,或如体力随汗液流失,大地也将自我溶解,即便没有其他诱因,它也会在自身内部找到沉没的洪流。据我推测,所有这些伟力终将联合发难。毁灭不会延迟片刻。当世界稍懈其必要的维系,和谐便被打破遭受侵袭。霎时间,来自四面八方的水流——无论显隐、上下、明暗——都将奔涌而至"}。 \cite{8}
最后,洪水故事表明,地球可能会经历多日的漫山遍野的海洋洪水,然后回归到可能不同的海岸线,持续约20至60年(中国、阿兹特克)\cite{5,6},之后情况将再次开始变化。

\subsection{减弱的磁场——火}
地球的地磁场已被充分确立并测量。古地磁数据表明,在过去的2000年里,磁场的偶极矩一直在减弱 \cite{9}(图 \ref{fig:3})。理论认为,地球内部旋转的岩浆柱产生了地磁场。至少可以明确的是,地球是一个巨大的磁体,而且它正在逐渐瓦解。

\begin{figure}[t]
\begin{center}
   \includegraphics[width=1\linewidth]{paleo.png}
\end{center}
   \caption{古地磁与考古磁学测量表明,地球的磁偶极矩在过去2000年间持续减弱。时间越久远,测量数据越稀疏。 \cite{11}}
\label{fig:3}
\label{fig:onecol}
\end{figure}
地球的地磁场保护地球免受宇宙碎片的侵袭,或许更重要的是,它还能阻挡来自太阳的粒子(图\ref{fig:4})。地球表面朝向太阳的赤道区域明显比两极温暖,这展示了太阳向地球传递的巨大能量。

\begin{figure}[t]
\begin{center}
   \includegraphics[width=1\linewidth]{solargeo.jpg}
\end{center}
   \caption{太阳与地球磁场相互作用示意图 \cite{10}.}
\label{fig:4}
\label{fig:onecol}
\end{figure}

人们必须预料到,当地球磁场翻转时,地磁场将处于最弱状态。这对于太阳高能辐射可能产生什么后果?古老的灾难传说或许再次为我们提供了答案。在表\ref{tab: 1}中,洪水之后,\textit{大火与火风暴}是第二常见的记载现象,在33\%的传说中被提及。其中一些著名的记载包括:希腊神话中关于kataklysmos(水)与ekpyrosis(火)交替毁灭的主题\cite{13},以及阿兹特克神话记载第三个太阳纪终结于\textit{"火雨"}\cite{14}。澳大利亚原住民的一个骇人传说这样描述:\textit{"野火将大河吹离河床,正如加利福尼亚印第安人口述历史与幼发拉底河流域近东地区传统所记载的那样。澳大利亚南部的原住民讲述了炽红天空降下可怕热浪的经过。酷热难当,人们无法忍受,男人们先杀死自己的妻儿,最后结束了自己的生命"}\cite{15}。

\subsection{构造与火山效应}
在地壳快速位移期间,地球板块运动的影响很可能构成重大威胁。根据表\ref{tab: 1}显示,"地形变化"是火灾之后被最广泛记载的现象。此处我援引南美洲安第斯山区关于山体抬升与灭世洪水同步发生的传说为例:\textit{"牧羊人与他的六个孩子收集了所有能找到的食物和羊群,将它们带至安卡马卡高峰之巅。当洪水上涨时,山峰也随之升高,因此峰顶始终未被淹没,而后山体随洪水一同沉降"}\cite{17}。当地球发生翻转时,我们很可能会经历板块及其边界的剧烈运动——包括显著的俯冲作用和造山运动。

另一个关键因素是地球的赤道隆起。众所周知,由于塑性层在自转最快的赤道区域略微外凸(图\ref{fig:5}),地球形成了\textit{扁球体}形态。这种效应很可能不仅存在于地球表面,也存在于深层结构中。若地球外层相对于内层发生位移,我们不仅会立即经历隆起带错位导致的剧烈调整,随后地球的塑性层还将根据绕新赤道的旋转物理学规律,逐步调整形成新的扁球体形态。
\begin{figure*}[t]
\begin{center}
\includegraphics[width=0.9\textwidth]{rotate2.png}
\end{center}
   \caption{6小时104度旋转过程中平均风速的可视化展示 \cite{2}.}
   \label{fig:6}
\end{figure*}

\begin{figure}[t]
\begin{center}
   \includegraphics[width=1\linewidth]{bulgecrop.png}
\end{center}
   \caption{地球赤道隆起部分的示意图(附有其他一些无关注释)\cite{16}。}
\label{fig:5}
\label{fig:onecol}
\end{figure}

此类过程对火山活动同样具有深远影响。大灾难故事中多次提及热水甚至沸水(表\ref{tab: 1})。例如,台湾的一则洪水神话记载:"地震时山岳崩塌,大地裂开,滚烫的地下水喷涌而出,淹没了整个大地。" \cite{17}。台湾正位于亚洲太平洋主要板块边界,表明其周边水域很可能受到来自地底的热力作用,这意味着地轴翻转将引发剧烈的火山活动。

根据这些记载,我们很可能会看到全球范围内构造运动与火山活动的集中爆发;其规模在近代人类历史上很可能前所未见。
\subsection{高速地表风}

在许多灾难故事中都提到了旋风与空中碎片(表\ref{tab: 1})。考虑到地球表面将在短时间内穿越大气层移动极远距离,这种现象就很好理解。加利福尼亚原住民的一个故事讲述道:\textit{"人类诞生并居住了很久很久。后来其中一人梦见了旋风,其他人说这是个凶兆。随后狂风骤起,愈演愈烈。世界开始崩坏。正午时分,他们都躲进了地屋。狂风肆虐,树木向西倾倒。那个做梦者留在外面,告诉其他人正在下雨,洪水将至,大地即将毁灭。所有其他房屋都被卷走。他进入地屋倚着支柱。最终支柱也松动了。这个做梦者是所有人中最后一个消亡的"} \cite{17}。图\ref{fig:6}描绘了最坏情况下(6小时内旋转104度)的平均旋转速度,该速度因地理位置和旋转情况而异。这些速度远超音速。至少我们可以预期,风速将堪比近代史上记录的最强飓风——这种风速已足以对人造建筑造成重大破坏。
\subsection{大规模地电冲击}

有证据表明,地球表面可能也会经历来自地球内部的大规模地电冲击。从毯子上摩擦双手产生的微小静电与巨大的闪电冲击,都是由电荷失衡引起的相同原因造成的。那么,如果已经具有巨大磁性的地幔和内核这些巨大内部层解耦并相互旋转,可能会发生什么?

\begin{figure}[t]
\begin{center}
   \includegraphics[width=1\linewidth]{lightning.jpg}
\end{center}
   \caption{一幅壮观的向上闪电抓拍图。没错,我们居住在一个巨大的带电旋转球体上。 \cite{22}}
\label{fig:7}
\label{fig:onecol}
\end{figure}

关于极高温度熔化的考古证据种类繁多。一个绝佳例证是苏格兰的玻璃化堡垒(图\ref{fig:8})。地质学家罗伯特·舒赫对此提出了见解:\textit{"苏格兰约有60座古代石质堡垒的墙体呈现玻璃化状态,岩石熔融成类似玻璃的形态。考古学家始终未能合理解释这些堡垒玻璃化的成因...石墙和建筑结构曾遭受剧烈高温,部分岩石甚至熔解黏结,形成类似金属冶炼炉渣的物质团块,与某些火山喷发产物相似...要使岩石熔融玻璃化,需要约1000摄氏度的极高温...舒赫推测远古等离子体事件发生时,这些高海拔堡垒可能成为避雷针,吸引强电流导致玻璃化。虽然多数苏格兰玻璃化堡垒定年为铁器时代,但舒赫认为其历史更为悠久,特别是马克土墩遗址,他宣称该遗址从公元前8000年延续使用至公元9世纪...是否至少部分玻璃化现象源于等离子体事件?我认为确实如此,某些堡垒可能追溯至一万年前甚至更早,即末次冰期结束之际,当时地球很可能发生了重大等离子体事件(或系列事件),硝酸盐数据或许就是证据"} \cite{18,19}。类似遗址在中东地区也有分布 \cite{20,21}。
\begin{figure}[t]
\begin{center}
   \includegraphics[width=1\linewidth]{vitrified.jpeg}
\end{center}
   \caption{苏格兰一座玻璃化的堡垒 \cite{51,52}.}
\label{fig:8}
\label{fig:onecol}
\end{figure}

\begin{figure}[t]
\begin{center}
   \includegraphics[width=1\linewidth]{sendai.png}
\end{center}
   \caption{2022年日本仙台7.3级地震中出现的"地震光"现象示例 \cite{48}.}
\label{fig:9}
\label{fig:onecol}
\end{figure}

现有研究确实表明地震可能引发电流和大气电磁辐射现象\cite{49,50}。引用秘鲁研究者安东尼奥·利拉和豪尔赫·A·埃劳德的相关论述:\textit{"大气中的电磁辐射通常与云层和地表之间的电荷加速有关。闪电是自然界最广为人知的电磁辐射现象,发生于雷暴天气中。我们不禁要问:是否存在源自地球内部的大气电磁辐射——例如闪光现象?答案是肯定的...电磁辐射是地震可能引发的大气次级效应"} \cite{50}。图\ref{fig:9}所呈现的地震光现象正是此类电磁辐射的表现形式。考虑到这些发现,当地球外部旋转体发生突然位移——这种位移将涉及史无前例的构造活动规模——其可能产生的效应值得深入思考。
\section{一种基础而原始的准备方法}

本质上,为了在ECDO中生存下来,必须为之前讨论过的所有风险同时发生做好准备。以下是一些基本指导原则:

\begin{flushleft}
\begin{enumerate}
    \item 为避免洪水侵袭,应选择一处相对于周边地形具有相当高度的地点,以便水流有充足空间绕过该处而不致淹没。根据对洪水案例的研究,建议实际突出高度(下文将解释)不低于2000米。此类地点若同时深入内陆,可能略优于沿海区域,并允许适当降低突出高度要求。
    \item 为防范火灾、狂风和飞溅碎屑,应在地下建造庇护所,最好选择基岩层以避免土壤液化。洞穴亦可作为备选方案。
    \item 为避免重大地质变动或地震造成的伤害,应远离重要板块边界地带。
    \item 就个人而言,我不会将庇护所修建在火山内部。
    \item 为避免成为高耸山峰上高频放电现象的受害者,应在地下避难所周围或内部建造法拉第笼(图 \ref{fig:10})以引导电流。
\end{enumerate}
\end{flushleft}

\begin{figure}[t]
\begin{center}
   \includegraphics[width=1\linewidth]{faraday.jpg}
\end{center}
   \caption{法拉第笼能保护笼内物体免受外部电击的干扰 \cite{54}.}
\label{fig:10}
\label{fig:onecol}
\end{figure}

关于人们抓住水中漂浮物、乘船或待在树上求生的故事有很多,但我并不推荐这类惊险活动。选址时,我建议使用Google Earth Pro扫描地形,寻找具有显著高差的区域(点击滚轮可轻松导航)。此外,你还可以评估该地区的地质层,确保没有严重的沉积或缺失的地质层。
\subsection{突出度——避免水患的最佳指标}

突出度定义为某座山峰相对于必须下降到的最低点的高度差,该最低点是再次攀登至更高峰前必须经过的最低处,如图X所示。为了理解为何我推荐将突出度而非绝对海拔作为防洪安全的首要指标,必须认识到地球表面极其平坦。若将地球上的海洋想象为一处水洼,假设其最深处约6英尺深,那么你能逃离这片水域的最高点仅约4-5英尺(即地球表面)。更重要的是,这个水洼的直径将达\textbf{18,000英尺}。按照这种比例关系,只要水量足够,任何绝对高度都无法阻挡水的侵袭。安第斯山脉和喜马拉雅山脉——这两个世界上海拔最高的陆地区域存在的大规模盐沉积,正是这一事实的有力证据。

\begin{figure}[t]
\begin{center}
   \includegraphics[width=1\linewidth]{prominence.jpeg}
\end{center}
   \caption{突显度("p")的视觉示例 \cite{53}.}
\label{fig:11}
\label{fig:onecol}
\end{figure}

比绝对海拔更重要的是拥有相对于周围地形的相对高度,这样流经该区域的任何水流都会有充足的空间绕过较高位置。沿海地区的高突出山峰将比喜马拉雅山脉中高海拔但低突出的山谷更安全。最后,我将提供一些补充说明:
\begin{flushleft}
\begin{enumerate}
    \item 突出度是一个近似指标,因其定义非常特定。可能存在高突出度的高原,但常规突出度数值无法体现这一点,因为突出度专指山峰。精确的学术突出度数值在实践中会被低估。
    \item 对于被水域包围的陆地,通常应将周边水体的深度纳入突出度计算。
    \item 突出度仅是一个指标——水体与每种地形的相互作用方式各异。因此最具权威性的预测需通过高保真度、多材料流体模拟来实现,该模拟需对全球各区域的水流进行建模。然而,此类流体模拟所需的计算资源极其庞大。
    \item 另一种选择可能是根据相对高程生成自定义地形图,类似于突出度。
\end{enumerate}
\end{flushleft}

\subsection{翻转后基础地球物理时间线}
当翻转发生时,你可能需要在避难所中连续躲避至少一周,直到地球外部旋转体找到新的自转轴。之后,大部分大陆地块将被夷为平地、被沉积物掩埋或被水淹没。人类建造的几乎所有东西都将不复存在。地表盆地很可能会滞留大量水体。由于太平洋水体从赤道隆起带向外位移,我预计"海平面"平均将比现在更高。天气系统将紊乱数十年,根据新位置与赤道的距离差异,温度可能与之前截然不同——比如若最终处于极地区域,你可能会被冻死。这个阶段预计将持续数十年。最终,随着导致地核解耦的放热过程逆转为吸热过程,地球可能逐渐恢复原状,在此过程中会显著降温。

在论文结尾处,我主要根据洪水传说和山峰突出度,提供了可能安全的山顶综合列表。
\section{地缘政治与大规模杀伤性武器的威胁}

除地球物理风险外,人类还须应对某些个人或团体可能使用的地缘政治手段及现代大规模杀伤性人群控制武器——这些手段旨在消灭敌对势力、使民众持续处于恐惧或虚弱状态,从而维系"权力"。事实上,尤其在西方世界,我们对于通过操控政府来愚弄民众的非民选术士们的真实动机几乎一无所知。近期某些政策决策尤其令人忧心,例如以"善意医疗措施"为伪装部署的mRNA疫苗这类大规模杀伤性生物武器,这不得不让我们担忧不久的将来还会面临怎样的威胁。清醒的现实是:当真正的"非接触式战争"来临时,若想保持独立个体的主权,仅关注自保可能是徒劳的。由于该议题涉及面广且在ECDO范畴外已有广泛讨论,我将力求在保持与ECDO相关性的前提下,以准确、全面而简洁的方式探讨潜在军事武器风险。

\subsection{"第三次世界大战"}
英国共济会高级成员关于2005年伦敦高层会议的举报证词,由"卡米洛特计划"公布,已在我第三篇论文中讨论。该证词披露西方高层人士曾计划发起针对伊朗和中国的有限核打击,继而释放生物武器。"卡米洛特计划"宣称,\textit{"这仅仅是开端。此后将引发全面核交火:即'真正'的战争,造成大规模破坏与人员伤亡。消息源告知我们,通过这种组合手段计划实现50\%的人口削减。\textbf{他亲耳听闻会议中明确提及这一数字。}"} \cite{23}。其动机显然是为铲除对手——中国作为首要目标,以确保地球物理灾难后不再存在军事力量的竞争者。如今我们已目睹乌克兰和中东的重大冲突,战术核武器很可能已投入实战。此外,以mRNA疫苗形式部署的生物武器已遍布全球,导致大量人群受伤、绝育及死亡。尽管距该会议已过去20年,世界格局明显改变,但人们不得不质疑会议讨论的主要目标与总体计划——包括通过\textbf{无接触核战争实现50\%人口削减}——是否仍在推进。我可以断言,正在阅读本文的你,大概率不会成为通过\textbf{核战争达成50\%人口削减}的受益者。

\subsection{乌克兰与中东近期战术核武使用情况}
我注意到,当"发达世界"的人类沉溺于金融堕落和21世纪硅基娱乐时,世界的另一端正进行着一场持续数十载且仍在持续的核战争。特别是在西亚地区,由于当地民众成为大多数核武器的直接受害者,这一事实已在普通民众中广为人知。化名地缘政治分析师兼核专家Cirnosad(主要在Twitter/X平台活动)分析过多次核爆炸事件,并阐述了如何辨识这些爆炸\cite{24}。核爆炸具有若干独特特征可供识别——由瑞利-泰勒不稳定性这种流体动力学现象引发的快速垂直上升烟柱、强烈闪光、冲击波、烟云顶部环绕的等离子体、霰弹枪式爆响,以及独特的地震与声波特征。此外,从爆炸现场采集的放射性样本及遇难者身上特殊的灼伤痕迹也能提供佐证。虽然我已核验出他账号分享的过去十年间十二起爆炸事件符合核武器独特特征,但其他消息源显示核武器的使用甚至可追溯到更早的2003年西亚地区,或许还更早。总体而言,过去几十年间,在北约占领的俄乌领土、俄罗斯及西亚多国境内,很可能已有数十枚核武器被用于大规模屠杀仪式。

需要特别指出的是,这些核弹多为所谓"战术"核武器。自二战末期首次在日本用于屠杀以来,核技术显然已取得显著改进。核弹效率更高、辐射更少,还研制出可实现精准有限破坏的小型版本,便于实施后抵赖\cite{29}。Cirnosad评估近期屠杀中当量最大的核弹是俄罗斯特维尔爆炸事件,约1.1万吨TNT当量\cite{24}(图\ref{fig:12})。作为参照,广岛和长崎原子弹的当量分别为1.5万和2.1万吨\cite{30}。

尤其在西亚地区,多国指控以色列和美国使用核武器。当地已出现大量先天缺陷案例\cite{26,27}以及核武器造成的特殊灼伤痕迹\cite{28}。下文所示的多处爆炸地点也检测到辐射水平升高。无论如何,俄罗斯扎波罗热核电站\cite{60}和伊朗核设施\cite{61}遭袭事件已被公开承认,这标志着危险而明确的核升级。
到目前为止我所发现的完整清单(很可能非常不完整)显示,过去几十年中以下地点可能在指定日期遭受过核武器打击 \cite{24}:

\begin{flushleft}
\begin{enumerate}
    \item 伊拉克费卢杰,2003年
    \item 黎巴嫩,2006年
    \item 加沙地带,2008年
    \item 叙利亚霍姆斯,2013年
    \item 也门,2014/2015年
    \item 黎巴嫩贝鲁特港,2020年
    \item 俄罗斯恩格斯空军基地,2022年
    \item 叙利亚塔尔图斯,2024年
    \item 俄罗斯布良斯克,2024年
    \item 北约占领下的乌克兰哈尔科夫,2024年
    \item 俄罗斯,特维尔州,托罗佩茨,2024年
    \item 叙利亚,杜马尔,2024年
    \item 北约占领下的乌克兰,日托米尔,2025年
\end{enumerate}
\end{flushleft}
我认为这些数据充分证明,《盎格鲁-撒克逊计划》中讨论的时间表自2010年首次披露其2005年进展状况以来,一直在全速推进——这是一场为在后灾难世界中夺取权力而蓄意发动的核战争——而我们可能尚未目睹最糟的情况。

\begin{figure*}[t]
\begin{center}
\includegraphics[width=1\textwidth]{tver.png}
\end{center}
   \caption{2024年特维尔爆炸事件的录像与1957年11千吨当量核试验(铅锤行动)的对比画面,展示了核爆引发的快速升腾蘑菇云(瑞利-泰勒不稳定性)\cite{24}。}
   \label{fig:12}
\end{figure*}
\begin{figure*}[t]
\begin{center}
\includegraphics[width=1\textwidth]{toropets.png}
\end{center}
   \caption{对俄罗斯军事基地托罗佩茨爆炸事件的卫星图像分析。明显可见一次空中爆炸,冲击波击倒了树木但未引燃一切,这是空中核爆炸的确凿证据 \cite{24}。}
   \label{fig:13}
\end{figure*}

\begin{figure*}[t]
\begin{center}
\includegraphics[width=1\textwidth]{nukes.png}
\end{center}
   \caption{图表显示西亚多个爆炸地点辐射水平升高——拉脱维亚博士研究主任克里斯·巴斯比绘制 \cite{32,33}}
   \label{fig:14}
\end{figure*}

\begin{figure*}[t]
\begin{center}
\includegraphics[width=1\textwidth]{tartous.jpeg}
\end{center}
   \caption{瑞士物理学博士汉斯-本杰明·布劳恩对2024年爆炸后叙利亚塔尔图斯辐射水平的分析:\textit{"据最新报告,2024年12月15日核爆后叙利亚塔尔图斯爆心投影点的辐射剂量率,分别比长崎和广岛爆心高出100\%和50\%以上,使后两者相形见绌"} \cite{34}。}
   \label{fig:15}
\end{figure*}

\begin{figure*}[t]
\begin{center}
\includegraphics[width=1\textwidth]{sound.jpeg}
\end{center}
   \caption{叙利亚塔尔图斯独特核爆声学特征分析:\textit{"然而塔尔图斯确实发生了核爆:与朝鲜核试验识别所用的地震图对比。两者均与赫梅利尼茨基(2023年)'延迟'爆炸形成鲜明对比,确证了塔尔图斯>0.3千吨当量爆炸的核性质,从而彻底否定了弹药库爆炸假说"} \cite{34}.}
   \label{fig:16}
\end{figure*}

\clearpage
\twocolumn

\begin{figure*}[t]
\begin{center}
\includegraphics[width=1\textwidth]{japan.png}
\end{center}
   \caption{对日本2100万份新冠疫苗接种记录的分析显示,接种后90-120天死亡率出现激增 \cite{40,41}。}
   \label{fig:17}
\end{figure*}

\subsection{生物武器}
《盎格鲁-撒克逊使命》中另一个重要议题是"生物武器"的隐秘使用。2019年末,中国武汉出现全球首例新冠肺炎确诊病例。2020年初,世界卫生组织宣布"大流行"状态,mRNA疫苗在2020年底迅速投入接种,全球大量民众因"疫情紧急状态"排队接受这种据称未经充分测试且快速研发的外来物质注射\cite{31}。时至今日,越来越明显的证据表明,这些疫苗绝非保护接种者免受未知"冠状病毒"侵害的应急产物,而很可能是旨在悄无声息地伤害、绝育乃至杀害接种者的生物武器。鉴于相关证词已呈海量之势,现仅摘录数例于下。

日本最资深肿瘤学家、京都大学名誉教授福岛正伸多次公开批评新冠mRNA疫苗\cite{35,36}:\textit{"以往疫苗都是蛋白质,注射后不易扩散至其他部位。但纳米颗粒包裹的疫苗进入血液后就会遍布全身...直达大脑、指甲末端,从头顶到脚趾无所不至。政府谎称其会停留在注射部位产生抗体,简直荒谬绝伦。当疫苗进入大脑,就会渗透全身细胞并持续产生抗原——即具有极强毒性的刺突蛋白。这种蛋白会摧毁细胞能量工厂线粒体"\cite{37,38}}。辉瑞前副总裁迈克·耶登同样指出:\textit{"在我看来,所谓新冠疫苗从设计之初就是蓄意制造伤害、致死和降低生育能力的工具"\cite{39}}。由村康文教授和我那霸正子主导的日本2100万份疫苗接种记录分析显示,接种后90-120天出现死亡高峰,且追加接种剂量会缩短死亡间隔期\cite{40,41}。
由于这是一个被广泛讨论的话题,我最后只想说,武器库中可能还存在更多生物武器。

\begin{figure*}[t]
\begin{center}
\includegraphics[width=1\textwidth]{tes1.jpeg}
\end{center}
   \caption{伦理怀疑论者分析显示,美国接种疫苗后非新冠自然原因死亡率呈趋势性上升 \cite{42}。}
   \label{fig:18}
\end{figure*}

\begin{figure*}[t]
\begin{center}
\includegraphics[width=1\textwidth]{tes2.png}
\end{center}
   \caption{伦理怀疑论者分析显示,美国接种疫苗后恶性肿瘤呈现趋势线上升 \cite{42}。}
   \label{fig:19}
\end{figure*}

\clearpage
\twocolumn

\begin{figure*}[t]
\begin{center}
\includegraphics[width=1\textwidth]{farmcrop.PNG}
\end{center}
\caption{美国地质调查局(USGS)绘制的全球农业用地分布图 \cite{45}。}
\label{fig:20}
\end{figure*}

\subsection{伪救世主}
在实施精神控制、欺骗公众、下毒和镇静等手段显然有助于达成目标的同时,归根结底,不依赖赤裸武力的统治建立在被统治者的默许之上。为了维持这种公众默许,除了制造假旗事件外,操纵群众的术士们可能会试图推出"伪救世主"特洛伊木马,将舆论导向他们掌控的"伪反对派"——就像政治戏剧的消费者们被给予"共和党"与"民主党"的虚假二元选择,而实际上二者同属一体两面的"兄弟同盟党"。

\subsection{其他秘密武器与危险}
最后这一部分简要提及一些相对未经证实的说法和传闻,仅为求全责备而录之。我曾见过一种说法,称美军备有可快速部署的生物制剂,能喷洒至主要人口中心以消灭大量人口。曾在南极担任消防员和管道工的雷神公司前雇员埃里克·赫克声称,南极建有巨型"定向能武器"(DEWs),可通过向特定地点传输能量引发地震\cite{43,44}。此外还有诸多关于采用未知推进方式的UFO飞行器的传言,称此类技术可能被秘密军事项目掌握并隐藏。据传五角大楼曾资助相关研究项目\cite{59}。另有传言称某些西方政府背后存在"外星/爬虫人智能体"\cite{58,55}。人类确实可能面临此类现代大规模杀伤性武器、人群控制武器及未知威胁的挑战。

\section{未来可能的时间线}

基于前文所述,我的思考核心在于:在讨论过所有这些情况后,是否还能拯救所有人。尽管形势看似严峻——事实也确实如此——但我坚信每个人都有获救的可能。但要做到这一点,人类必须将视线从屏幕移开,审视周遭世界,决心进行崇高抗争,携手为未来继承者创造更美好的世界。三心二意的努力\textbf{必将}失败。人类必须决定自己是否真是待宰羔羊。我的结论是:面对即将到来的地理大灾变,人类有两条基本前进道路,本节将对此展开讨论。
\subsection{每个人都能得救吗?}

目前世界人口估计超过80亿人\cite{47}。维持这一人口规模需要占用地球表面约38%的农业用地\cite{46}(图\ref{fig:20}),而几乎所有农业用地都位于低洼地区,将在海洋位移中被完全摧毁。
简而言之,这将非常困难。只要掌控核武库的人还计划向世界投掷核弹、杀死数十亿人,并欺骗人们让陌生人将外来物质注入体内以实施绝育与杀戮,而人类仍沉溺于各种分心事物与毒品的麻痹之中,你就可以确信这一切不会改变。

\subsection{时间线上的升级信号}

请再次关注《盎格鲁-撒克逊使命》中讨论的时间线。2005年的计划是制造有限冲突并释放生物武器,随后在西方国家实施极权主义的军事戒严。下一步则是在地球物理大灾变前,通过代理人核战争消灭对手。2010年作证的那位共济会高阶成员估计,距离灾变可能还有约20年时间。现在是2025年,意味着距离2030年的预测只剩5年。世界已经历了隐蔽的核打击与大规模导弹交锋(乌克兰、西亚地区)以及mRNA生物武器的部署,这表明时间线正按基本步骤推进。如果我们看到西方国家转向公开戒严状态,并启动代理人核战争,那么我们就知道地球物理大灾变已迫在眉睫。

\subsection{两条前进之路}

除非人类承认其错误道路、解除武装、和平团结,并将全部精力和资源投入于在全球所有安全地点建造保障人类安全的居住设施,否则答案是否定的——大多数人将会死亡,全球将陷入饥饿游戏的局面。为人类建造大规模山顶方舟甚至也是可能的,但问题显然在于资源获取,或许更重要的是应对潜在的破坏者。必须明确指出,只要还存在一群操纵大规模杀伤性武器和心智控制士兵的术士与女巫(这些士兵在全球灾难中可能失控),我们就面临着极其危险的局面。任何个人准备都可能无济于事。摆脱困境的第一步是让所有人的目光聚焦于此。后续发展则充满无限可能。
\section{结论}

作为第四篇论文,本文最终总结了我经过一年专注于ECDO研究后关于该主题的全部见解。倘若我所构建的叙述与事实相去不远,除非形势急剧变化,世界很快将走向一场浩劫。除非大灾变的时间线出现重大推进,我预计短期内不会再就此主题撰写新论文。感谢各位对此议题的关注!

\section{致谢}
感谢伦理怀疑论者\cite{0}向世界分享其开创性的ECDO理论。感谢克雷格·斯通(@nobulart)\cite{56}就该主题分享了大量知识,其中相当部分内容已被纳入我的四篇论文中。特别致谢所有被我论文引用参考的学者,包括但不限于凯瑟琳·奥斯汀·菲茨\cite{55}、理查德·桑德尔、Cirnosad\cite{24}、汉斯-本杰明·布劳恩\cite{57}以及克里斯·巴斯比。

\section{俊昊推荐的"ECDO安全区域"}
以下安全地点列表基于两个主要指标:明确指出洪水未淹没某些区域的洪水神话,以及山峰突出度列表。我将这两个列表分开列出。尽管这两个列表都极不完整,但它们应能作为选择安全地点的绝佳起点。

关于洪水神话,我只收录了那些提供具体名称或明确提及未命名山峰安全的神话。即使幸存者最终降落在山顶,那些描述漂浮幸存者的神话已被排除。我建议您亲自阅读这些洪水神话。归根结底,大多数拥有独特洪水神话的地点很可能都有幸存者,但必须考虑到幸存者之后可能已迁移至不同地区。

关于突出度列表,我已对山峰突出度列表进行筛选,仅包含突出度至少2000米且非火山的山峰,这些地点在绝大多数情况下都是安全的。需要注意的是,学术定义的突出度数值(其定义非常具体)往往会低估(可能是严重低估)"实际"突出度——后者取决于当地地形特征对水流通过的影响。因此本列表遗漏了许多安全山峰。同样地,在内陆地区由于无需应对大量积水,对突出度的要求可以适当放宽。
我还列出了每个大陆上我会选择停留的总体安全区域清单。

\subsection{总体安全区域}
\begin{flushleft}
\begin{enumerate}
    \item \textbf{北美洲:} 落基山脉,西海岸群峰
    \item \textbf{中美洲:} 墨西哥高地
    \item \textbf{南美洲:} 安第斯山脉,圭亚那高地,马尔山脉群峰
    \item \textbf{欧洲:} 阿尔卑斯山脉,比利牛斯山脉
    \item \textbf{非洲:} 东非大裂谷山脉,南非山脉
    \item \textbf{西亚:} 山峰群
    \item \textbf{亚洲:} 喜马拉雅山脉,东南亚群岛峰峦
    \item \textbf{澳大利亚:} 东南部山区
    \item \textbf{新西兰:} 山峰
\end{enumerate}
\end{flushleft}

\clearpage
\twocolumn

\section{洪水神话安全点}

\begin{figure}[H]
\begin{center}
   \includegraphics[width=1\linewidth]{flood.jpg}
\end{center}
   \caption{以下安全点位置是从上述洪水神话地图中提取的子集 \cite{2,17}。}
\label{fig:21}
\label{fig:onecol}
\end{figure}

针对每个神话,均标注了大致地域及指定的安全地点。故事链接亦已嵌入。许多故事中出现的名称似乎已显陈旧。
\subsection{亚洲洪水神话中的安全地点}

\begin{flushleft}
\begin{enumerate}
\item 壮族(中国),亚洲:\href{http://www.talkorigins.org/faqs/flood-myths.html#Zhuang}{巴奇山}
\end{enumerate}
\end{flushleft}

\subsection{南美洲洪水神话安全地带}
\subsection{中美洲洪水神话安全地}

\subsection{澳大利亚洪水神话安全地}

\subsection{欧洲洪水神话安全地}
\subsection{北美洪水神话安全地带}

\subsection{西亚洪水神话安全地带}
\subsection{太平洋岛屿洪水神话中的安全地点}

\clearpage
\twocolumn
\section{2000米突出山峰列表}

% \begin{figure}[H]
% \begin{center}
%    \includegraphics[width=1\linewidth]{peaks.png}
% \end{center}
%    \caption{以下365座高突出度非火山山峰分布图 \cite{62}。}
% \label{fig:22}
% \label{fig:onecol}
% \end{figure}
\begin{figure*}[t]
\begin{center}
\includegraphics[width=0.7\textwidth]{peaks.png}
\end{center}
   \caption{一幅包含以下365座高突出度非火山山峰的地图 \cite{62}。}
   \label{fig:22}
\end{figure*}

图中标注了山峰名称及其海拔高度、突出度数值(按此顺序排列)。
\subsection{亚洲2000米突出山峰列表}
\begin{flushleft}
\begin{enumerate}
    \item 卓木拉日峰,不丹、中国,7050米,2065米
    \item 康格普山,不丹、中国,7570 米,2995 米
    \item 阿尔金山,中国,5830 米,2528 米
    \item 白里嘎峰,中国,6882 米,2444 米
    \item 巴里坤山,中国,4320 米,2104 米
    \item 博格达山,中国,5445 米,4122 米
    \item 中国,公格尔山,6727 米,2901 米
    \item 中国,大神农架,3100 米,2270 米
    \item 中国,大雪山,3500 米,2041 米
    \item 中国,格聂山,6204 米,2000 米
    \item 中国,纳木那尼峰,7694 米,2788 米
    \item 嘉黎佩里峰,中国,7294 米,2942 米
    \item 贺兰山最高点,中国,3540 米,2098 米
    \item 河源峰,中国,5289 米,2618 米
    \item 华头尖,中国,4750 米,2223 米
    \item 黄王山最高点,中国,4330 米,2470 米
    \item 九顶山,中国,4969 米,2808 米
    \item 岗则/结雅,中国,5808 米,2231 米
    \item 卡瓦格博,中国,6740 米,2232 米
    \item 卡瓦拉尼,中国,5992 米,2018 米
    \item 克塔乌,中国,3282 米,2365 米
    \item 公格尔山,中国,7649 米,3585 米
    \item 喇嘛峰,中国,6070 米,2093 米
    \item 贡嘎山,中国,7556 米,3642 米
    \item 慕士塔格峰,中国,7509 米,2698 米
    \item 南迦巴瓦峰 I,中国,7782 米,4106 米
    \item 诺日康峰,中国,7206 米,2160 米
    \item 念青唐古拉峰,中国,7162 米,2239 米
    \item 色浦岗日峰,中国,6956 米,2213 米
    \item 希夏邦马峰,中国,8027 米,2897 米
    \item 苏拉马塔格峰,中国,6245 米,2028 米
    \item 太白山,中国,海拔3750米,相对高差2232米
    \item 特克力克山,中国,海拔5470米,相对高差2337米
    \item 托木尔峰,中国,海拔4886米,相对高差3243米
    \item 雪宝顶,中国,海拔5588米,相对高差2057米
    \item 雪莲峰,中国,海拔6627米,相对高差3068米
    \item 药莓峰,中国,6250米,相对高差2571米
    \item 玉龙雪山最高点,中国,5596米,相对高差3202米
    \item 康格山,中国、印度,7060米,相对高差2195米
    \item 泡洪里峰,中国、印度,7128米,相对高差2035米
    \item 阿拉戈尔迪山,中国、哈萨克斯坦,4622米,相对高差2480米
    \item 索尤尔峰,中国、哈萨克斯坦,3840 米,3252 米
    \item 托木尔峰,中国、吉尔吉斯斯坦,7439 米,4148 米
    \item 冬德山,中国、蒙古国,3315 米,2075 米
    \item 友谊峰,中国、蒙古国,4374 米,2342 米
    \item 卓奥友峰,中国、尼泊尔,8188 米,2340 米
    \item 马卡鲁峰,中国、尼泊尔,8485 米,2378 米
    \item 珠穆朗玛峰,中国、尼泊尔,8848 米,8848 米
    \item 央然康日峰,中国、尼泊尔,7422 米,2352 米
    \item 加舒尔布鲁木I峰,中国、巴基斯坦,8080 米,2155 米
    \item 乔戈里峰(K2),中国、巴基斯坦,8611 米,4017 米
\end{enumerate}
\end{flushleft}

\subsection{南美洲2000米突出高峰列表}
\subsection{中美洲2000米突出山峰列表}

\subsection{非洲2000米突出山峰列表}

\subsection{澳大利亚2000米突出山峰列表}
\subsection{欧洲2000米突出山峰列表}

\subsection{北美2000米突出山峰列表}
\subsection{西亚2000米突出山峰列表}

\subsection{太平洋岛屿2000米突出山峰列表}

\subsection{南极洲2000米突出山峰列表}

\clearpage
\twocolumn

{\small
\bibliographystyle{ieee}
\bibliography{egbib}
}

\end{document}
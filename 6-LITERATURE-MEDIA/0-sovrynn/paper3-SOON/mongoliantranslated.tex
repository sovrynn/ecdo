\documentclass[10pt,twocolumn,letterpaper]{article}

\usepackage{booktabs}
% \usepackage{caption}
% \captionsetup[table]{skip=8pt}   % Зөвхөн хүснэгтэд нөлөөлнө
\usepackage{stfloats}  % Үүнийг өмнөх хэсэгт нэмнэ үү
\usepackage{float}

\usepackage{newunicodechar}
\usepackage{graphicx}

% Define a half-width em-dash
\newcommand{\shortemdash}{\scalebox{0.5}[1]{\textemdash}}

% Map U+2015 to this shorter dash
\newunicodechar{─}{\shortemdash}

\usepackage{fontspec}
\usepackage{ucharclasses}

%–– your fonts ––
\newfontfamily\latinfont{Latin Modern Roman}                % default (Latin)
\newfontfamily\mongscriptfont[Script=Mongolian]{Noto Serif}  
\newfontfamily\cyrillicfont[Script=Cyrillic]{Noto Serif}      % or Noto Serif, etc.

%–– detect and switch ––
\setDefaultTransitions{\latinfont}{}                        
\setTransitionsFor{Mongolian}{\mongscriptfont}{\latinfont} 
\setTransitionsFor{Cyrillic}{\cyrillicfont}{\latinfont}

\usepackage{cvpr}
\usepackage{times}
\usepackage{epsfig}
\usepackage{graphicx}
\usepackage{amsmath}
\usepackage{amssymb}

\usepackage[breaklinks=true,bookmarks=false]{hyperref}

\cvprfinalcopy % *** Энэ мөрийг эцсийн хувилбарт тайлбарлаарай

\def\cvprPaperID{****} % *** CVPR өгүүллийн ID-г энд оруулна уу
\def\httilde{\mbox{\tt\raisebox{-.5ex}{\symbol{126}}}}

\renewcommand{\refname}{Эшлэлүүд}
\renewcommand{\tablename}{Хүснэгт}
\renewcommand{\figurename}{Зураг}   % or whatever you like instead of "Hình"

\makeatletter
\def\abstract{%
  \centerline{\large\bf Хураангуй}% <-- your new label
  \vspace*{12pt}%
  \it%
}
\makeatother

% This makes the font slightly bigger than base (10) and bold in Subsection headings rather than using ptmb
\makeatletter
\def\cvprsubsection{%
  \@startsection{subsection}{2}{\z@}%
    {8pt plus 2pt minus 2pt}{6pt}%
    % {\normalfont\bfseries\selectfont}%
    {\normalfont\bfseries\fontsize{11}{13}\selectfont}%
}
\makeatother

% So this hardcodes the style for the numbers in the section/subsection headings so they're bold
\font\elvbf=ptmb scaled 1100
\font\elvbfs=ptmb scaled 1200
\makeatletter
% Section number: Large + bold
\renewcommand\thesection{%
  {\elvbfs\arabic{section}}%
}

% Subsection number: normalsize + bold + custom punctuation
\renewcommand\thesubsection{%
  {\elvbf
   \arabic{section}.\arabic{subsection}}%
}
\makeatother

% Хуудасны дугаарлалтыг илгээх хэлбэртээр дугаарлаж, хэвлэхэд дугааргүй болгодог
%\ifcvprfinal\pagestyle{empty}\fi
\setcounter{page}{1}
\begin{document}

\title{ECDO судалгааны ажил 3: Орчин үеийн барууны засаглалын төлөөлөгчид ойрын ирээдүйн геофизикийн их гамшигт бэлтгэж буй нотолгоо}

\author{Жунхо\\
Зарлагдсан огноо: 2025 оны 6 сар\\
Вэбсайт (Өгүүллүүдийг эндээс татаж авна уу): \href{https://sovrynn.github.io}{sovrynn.github.io}\\
ECDO судалгааны сан: \href{https://github.com/sovrynn/ecdo}{github.com/sovrynn/ecdo}\\
{\tt\small junhobtc@proton.me}
}

\maketitle
%\thispagestyle{empty}

\begin{abstract}

\begin{flushleft}
2024 оны тавдугаар сард “Ёс зүйт скептик” \cite{0} хэмээх нууц нэртэй онлайн зохиогч “Дулаан ялгаруулах цөм-мантийн салалтын джаанибековын хэлбэлзэл” (ECDO)\cite{1} хэмээх хувьсгалт онолыг нийтэлсэн. Энэхүү онолд дэлхий өмнө нь эргэлтийн тэнхлэгийн огцом, сүйрлийн шилжилтэд орж, эргэлтийн инерцийн улмаас далай тэнгисүүд эх газрын дээгүүр даван цутгаж, аварга их үер дэгдэж байсныг дэвшүүлээд зогсохгүй, мөн энэхүү эргэлт дахин ойртсоор байгааг илтгэх баримтат мэдээлэл болон түүнийг үүсгэх геофизикийн тайлбарлагдсан шалтгаант процессыг санал болгож буй. Ийм гамшгийн үер болон дэлхийн сүйрлийн таамаглал шинэ зүйл биш ч, шинжлэх ухааны суурьт, орчин үеийн, олон салбарт тулгуурласан, өгөгдөлд суурилсан арга барилтай ECDO онол нь өөрийн гэсэн онцлогтой бөгөөд үнэмшил төрүүлэхүйц юм.

Энэхүү өгүүлэл нь уг сэдвээр бичсэн миний гурав дахь судалгааны ажил \cite{2,3} бөгөөд орчин цагийн улс төрийг онцолсон болно:
\begin{enumerate}
    \item Барууны эрх мэдэлтнүүд геофизикийн сүйрэл ойртож байгаа гэж үздэг бөгөөд энэхүү үйл явдлыг улс төр, цэргийн давуу тал болгон ашиглахаар төлөвлөж буй талаар илчилсэн мэдүүлэг.
    \item Уг үйл явдалд бэлтгэх зорилгоор өргөн хүрээтэй, газар болон далай доор хоргодох байруудыг барьсан гэх нотолгоо.
    \item Эдгээр баазуудыг санхүүжүүлэхийн тулд барууны санхүүгийн бүтцээс их хэмжээний мөнгө гадагш урсаж байгаа талаарх баримтууд.
\end{enumerate}
\end{flushleft}
Энэхүү судалгаа нь барууны засаглалын эрх мэдэлтнүүд ойрын ирээдүйд тохиох геофизикийн гамшигт бэлтгэж буй өргөн хүрээтэй бэлтгэл ажлуудыг баримтжуулсан болно.
\end{abstract}

\section{Чөлөөт өрлөгийн нууц холбоо ба "Англо-саксон ажиллагаа"}

2010 оны нэгдүгээр сард, шүгэл үлээгчдийн мэдүүлгийг цуглуулдаг чөлөөт хэвлэл мэдээллийн байгууллага болох Камелот төсөл нь 2005 оны зургаадугаар сард Лондон хотын Ахмад өрлөгчдийн уулзалтад биечлэн оролцсон мэдээлэгчтэй \cite{4,6} ярилцсан юм. Уулзалтын үеэр хэлэлцсэн сэдвүүд нь цэргийн болон улс төрийн төлөвлөгөөнүүд бөгөөд удахгүй тохиох \textbf{"геофизикийн үзэгдэл"}, өөрөөр хэлбэл дэлхийн хэмжээний байгалийн гамшгийн хүрээнд төвлөрч байв.

\begin{figure}[b]
\begin{center}
\includegraphics[width=1\linewidth]{freemason.jpg}
\end{center}
   \caption{Британийн чөлөөт өрлөгийнхөн өөрсдийн байгалийн төлөвт: чимээгүйхэн цөмийн бөмбөг хаяж, дэлхийг эзлэх талаар хуйвалдаж байна - Лондонгийн Ээрлс танхимд, 1992 он \cite{5}.}
\label{fig:1}
\label{fig:onecol}
\end{figure}

\begin{figure*}[t]
\begin{center}
\includegraphics[width=1\textwidth]{british.jpg}
\end{center}
   \caption{1937 онд Британийн эзэнт гүрэн, Английн эртний уугуул ардын хүч чадлын сүр жавхлантай илэрхийлэл \cite{14}.}
   \label{fig:2}
\end{figure*}

Энэ мэдээлэгчийн хэлснээр, уг уулзалтад оролцсон 25-30 хүн \textit{"... бүгд англичууд байсан бөгөөд тэдний зарим нь Их Британид олон нийтийн танил болсон, нэр хүндтэй хүмүүс байлаа... бага зэрэг язгууртны аяс илэрч байсан ба зарим нь ч нэлээд язгууртны гаралтай хүмүүс харагдсан. Энэ уулзалтад миний танихаар нэг хүн бол өндөр албан тушаалтай улс төрч. Хоёр нь цагдаагийн албаны тушаал өндөртэй хүмүүс ба нэг цэргийн хүн байсан. Хоёул улс даяар танигдсан бөгөөд аль аль нь одоогийн засгийн газарт зөвлөх байр суурьтай хүмүүс юм — яг одоогийн байдлаар"} \cite{4}. Мэдээлэгч хэлэхдээ, тэр энэ уулзалтад очсон нь,\ \textit{"Үнэхээр санаандгүй тохиолдол! Энгийн гурван сарын уулзалт гэж бодсон... Би өөрөө энэ уулзалтад явж очсон ч гэсэн, миний хүлээж байсан зүйл огт биш байсан. Миний албан тушаалаас болоод намайг урьсан гэж боддог... , мөн тэд намайг өөрсөдтэйгөө адил гэж санасан байх."} \cite{4}.

Уулзалтаар хэлэлцсэн үйл явдлын үндсэн цагийн хуваарь (2005 онд) дараах байдалтай байна:

\begin{flushleft}

\begin{enumerate}
    \item Ираныг эсвэл Хятадыг тактикийн цөмийн зэвсэг ашиглуулахад өдөөж, хязгаарлагдмал цөмийн солилцоонд хүргэн, дараа нь гал зогсоох тогтоол гаргах.
    \item Хятад руу биологийн зэвсэг тараах, 1970-аад оноос гол бай нь болсон гэж мэдээлэгдсэн.
    \item Айдас болон үймээнээс үүдэн хянах бүрэн дарангуйллын цэрэгчилсэн засгийн газруудыг гаргаж ирэх.
\end{enumerate}
\end{flushleft}

Гэвч хамгийн чухал зүйл нь эдгээр үйл явдлуудын дараа юу болохыг хүлээж байгаа явдал юм: \textit{"Бид энэ дайныг дэгдээх болно, үүний дараа нь... дэлхий дээр геофизикийн нэгэн үзэгдэл болох ба энэ нь хүн бүрд нөлөөлнө"} \cite{4}. Мэдээлэгч хэлэхдээ энэ геофизикийн үзэгдлийн үед, \textit{\textbf{"Дэлхийн царцдас ойролцоогоор 30 градус, 1700-2000 миль урагшаа шилжих бөгөөд энэ нь асар их өөрчлөлтийг үүсгэж, нөлөө нь маш удаан хугацаанд үргэлжлэх болно"}} \cite{4}.

Энэ бүх нууц төлөвлөгөөний шалтгаан нь мэдээж эрх мэдэл юм. Мэдээлэгч тайлбарлахдаа, \textit{"Тухайн үед бид бүгд цөмийн болон биологийн дайныг даван гарсан байх болно. Энэ тохиолдолд дэлхийн хүн ам эрс буурна. Энэ геофизикийн гамшиг тохиолдох үед үлдсэн хүмүүсийн тоо дахиад ойролцоогоор тэн хагас болж багасах байх. Үүнийг давж үлдсэн хүмүүс дэлхий болон үлдсэн хүн амыг дараачийн эринд удирдах хүмүүсийг тодорхойлно. Тэгэхээр бид гамшгийн дараах эрнийг ярьж байна. Хэн удирдах вэ? Хэн хяналтыг гартаа авах вэ? Гол нь энэ л юм. Иймээс тэд эдгээр үйл явдлыг тодорхой хугацааны дотор болгохоор үхэн хатан зүтгэж байгаа. Замбараагүй байдал дэгдэхээс өмнө зайлшгүй бүтэц нь бэлэн байх ёстой, ингэснээр ирээдүйд болох гамшгаас амьд мултарч, маргааш нь хоёр хөл дээрээ зогсоод, эрх мэдлээ хадгалан, өмнөх хүчээ тогтоон барих боломжтой болно"} \cite{4}. Ярилцлагын үеэр энэ төлөвлөгөөний нэр болох "Англо-саксон ажиллагаа" буюу Английн эртний уугуул иргэдийн ажиллагаа гэж нэрлэгддэг тухай мөн ярилцсан: \textit{[Сэтгүүлч]: "...Үүний нэрний цаад шалтгаан нь үндсэндээ төлөвлөгөө бол Хятадуудыг устгаад, гамшгийн дараа, дэлхий дахин сэргээгдэх үед Английн эртний уугуул ард түмэн уг шинэ дэлхийг дахин байгуулах, өвлөн эзэмших байр сууринд үлдэх, ба үүнийгээ эргэн тойронд ямар ч амьд амьтан үлдээлгүйгээр хийх явдал уу. Зөв үү?" [Мэдээлэгч]: "Зөв эсэхийг сайн мэдэхгүй ч, тантай санал нэг байна. Адаглаад ХХ зуунаас хойш, бүр XIX, XVIII зуунд ч дэлхийн түүх болон том бодлогууд голчлон баруунаас болон дэлхийн умард бүсээс явагдаж иржээ"} \cite{4}.

Энэхүү геофизикийн үзэгдэл цаг хугацааны хувьд яг хэзээ болох талаар мэдээлэгч маань өөрийн тааврыг дэвшүүлсэн нь: \textit{"...өөрт төрсөн зөнгөөрөө таамаглан хэлбэл, одоо тэд өөрсдийгөө бэлтгэж байгаа ... Миний бодлоор тэд хэзээ болохыг нь таг сайн мэдэж байгаа болов уу... \textbf{Миний таамгаар, намайг амьд байгаа үед болно гэхээр, ер нь 20 жилийн дотор тохиох нь гарцаагүй}... бид одоо энэ геофизикийн үзэгдэл болох гэж байгаа үе рүү аль хэдийн орчихсон байна. Хамгийн сүүлд энэхүү үзэгдэл нь 11,500 жилийн өмнө болсон бөгөөд, энэ нь ойролцоогоор 11,500 жил тутамд, циклийн дагуу тохиодог. Одоо ер нь дахиж тохиох цаг нь болчихсон... Тэд ч болно гэдгийг ойлгож байгаа. Мөн болох нь тодорхой гэсэн баттай мэдлэг бас байгаа... Давтан хэлэхэд, тэд мэдэхгүй байна гэдэг нь байж боломгүй зүйл — хамгийн шилдэг ухаантнууд тэдний төлөө энэ ажиллагаан дээр ажиллаж байгаа шүү дээ"} \cite{4}.

Энэ нь талархахуйц хүчтэй мэдүүлэг юм. Ярилцлагын үеэр тэрээр мөн дэлхийн нэг болон хоёрдугаар дайн нь зохиомол дайн байсан гэж итгэдэг тухайгаа хэлсэн ба энэхүү Английн эртний уугуул ардын ажиллагаа нь бараг л баттайгаар олон үе дамжин явагдаж байгаа гэв. Энэхүү ярилцлага нь 2010 онд болсон ба өдгөө 15 жил өнгөрчээ. Мэдээлэгчийн хэлсэнчлэн геофизикийн үзэгдэл болох 20 жилийн таамагт нь ердөө 5 жил үлдсэн байна. 

\subsection{Гамшгийн тухай Друидын нууцлаг барууны мэдлэг}

Давтан тохиох усан галавын талаар барууны мэдлэг маш сайн хадгалагдсан бөгөөд энэ нь зөвхөн Чөлөөт өрлөгүүдээр хязгаарлагдахгүй. Хамгийн багадаа 2400 жилийн түүхтэй Кельтийн эртний соёлт Друидууд \cite{7} дэлхий дээр давтан болдог гамшигт явдлуудын тухай мэдлэгийг дамжуулсан байдаг. Друид урсгалын хамгийн сүүлийнх хэмээн нэрлэгдсэн Бен МакБрэди, "Сүүлчийн Друид" нэртэй, 1992 оны баримтат кинонд Друидуудын мэдлэгийн тухай дараах мэдээллийг хуваалцсан: \textit{"Уламжлалын дагуу би сүүлчийн гишүүн нь байж болох баримт нь сүүлчийн их галав юүлэлт буюу дэлхийг хамарсан сүйрлийн дараа үүссэн юм. Аймшигтай хүчтэй цахилгаан шуурга, солирын сүүлэнд өртөх, эсвэл солирын борооны дунд орогносноор бидний мэдэх соёл иргэншил бүрэн сүйрсэн гэдэг ... Бүх мэдлэгийг энэхүү бүлгийн хүрээнд төвлөрүүлдэг байсан ч, тэд ялангуяа одон орон судлалд анхаарлаа хандуулж байсан ба энэ нь, маш олон ноцтой сүйрэл амссаных. Бүрэн хэмжээний одон орон судлалын мэдлэгтэй байснаар уг сүйрлүүд болж болзошгүй нөхцөл байдлыг урьдчилан таамаглах, өөрсдийгөө хамгаалах арга хэмжээ авах боломжтой гэж үздэг байжээ. Ирландын агуйны нүсэр том байгууламжуудыг ажиглавал 'ногоон оршуулга' гэж нэрлэдэг тэр газрууд үнэндээ маш бүдүүн хүлэмж маягийн бөмбөгнөөс хамгаалах барилга байгууламжууд гэдгийг олж харна. Эдгээр нь ямар ч түрлэгийн түвшнээс өндөр цэг дээр байрладаг бөгөөд солирын борооноос ч бас хамгаалалт болох юм"} \cite{8,9}.

% Мөн Чөлөөт өрлөгүүд өөрсдөө шууд Друидуудын гаралтай гэж үздэг \cite{10}.
\section{Өрнөдийнхөн орчин үеийн энэхүү сүйрэлд бэлдэж буй нотолгоо}

Үндсэндээ барууны эрх баригч хүчнүүд дэлхий нийтийн геофизикийн сүйрэл ойртоод байна гэж үзэж байгаа бол ийм үйл явдлаас өөрсдийгөө хамгаалахын тулд маш их бэлтгэл ажлууд хийгдэхийг таамаглаж болно. Үнэхээр ч олон барууны орнуудад гүнд суурилагдсан нууц хонгилын баазуудын өргөн сүлжээ олон нийтэд ил байгаа нотолгоо бий. Ийм байгууламжууд нь цөмийн дайны үед оршин суугчдаа хамгаалж чадахаас гадна төрөл бүрийн байгалийн гамшгаас хамгаалах зориулалттай. Камелот төслөөс \cite{4,6} мэдээлэл өгсөн Их Британийн ахмад чөлөөт өрлөгийн мэдүүлгээр судлахад эдгээр нь зүгээр нэг боломжит хувилбарууд бус, харин урьдчилан боловсруулсан төлөвлөгөө мэт санагдахуйц юм. Мөн эдгээр хонгилын байгууламжуудыг барих, хүний нөөцөөр хангах, засвар үйлчилгээ хийхэд асар их мөнгө шаардагдах бөгөөд энэ нь АНУ-ын засгийн газарт 18 жилийн туршид алга болсон хэдэн арван их наяд ам.доллартой (дараагийн хэсэгт авч үзнэ) \cite{11,12,13} тэнцэж байна. Үхлийн хэмжээнд хүргэх эрсдэлтэй үйл явдалд бэлтгэх өөр жишээнүүдэд тариан үр болон мэдлэгийн архивын төслүүд багтдаг.

\subsection{Америкийн газар доорх ба далай доорх баазууд}

Миний олсон газар доорх баазуудыг ил тод судалсан хамгийн өргөн судалгаа нь Ричард Саудерын хийсэн ажил бөгөөд тэрээр АНУ-ын бие даасан судлаачийнхаа хувьд газар доорх баазын тухай хэд хэдэн ном хэвлүүлсэн байна \cite{22}. Саудерын ажилд засгийн газрын баримт бичиг, төлөвлөгөөг цуглуулж архивлах, түүхэн болон өнөөгийн мэдээ, технологиудыг нягтлан шалгах, эх сурвалжуудтай харилцах, дотор ажиллаж буй хүмүүсийн мэдүүлгийг цуглуулах зэрэг багтдаг. Саудерын судалгааны дүнд АНУ, түүний газар нутагт болон ойр орчимд орших гүнзгий газар доорх ба далай доорх баазуудын том сүлжээ (зураг \ref{fig:4}) байгаа нь илэрч, гүн нь хамгийн багадаа 4.8 км хүрч болзошгүй бөгөөд газар доорх вакум хоолой бүхий өндөр хурдны соронзон хөвүүрийн галт тэрэгнүүдээр холбоотой байх магадлалтай. Эдгээр баазуудыг \textit{"төсөв өндөртэй, олон улсын, салбар хоорондын, мөнгө угаах далд тоглоом"}-оор санхүүжүүлдэг бөгөөд энэ нь Америкийн Нэгдсэн Улс гэдэг нэртэй компанийг эзэмшдэг хүмүүстэй ижил бүлэглэлийн бүрэн хяналтад байдаг \cite{22}. Кэтрин Остин Фиттс (түүний ажил дараагийн хэсэгт дурдагдана) болон түүний хамтрагч нэгэн судлаачийн хамт эдгээр баазуудын тоо хэмжээг тодорхойлох судалгаа явуулахад Америкт 170 газар доорх болон далай доорх бааз байгааг тооцоолжээ \cite{16,20}.

\begin{figure}[b]
\begin{center}
% \fbox{\rule{0pt}{2in} \rule{0.9\linewidth}{0pt}}
   \includegraphics[width=1\linewidth]{penta.jpg}
\end{center}
   \caption{Цагаан ордон ба Пентагоны доор үнэндээ юу байдаг вэ? Магадгүй, газар доорх гүн хонгилын сүлжээ байж болзошгүй (Зураг: \cite{31}).}
\label{fig:3}
\label{fig:onecol}
\end{figure}
\begin{figure*}[t]
\begin{center}
% \fbox{\rule{0pt}{2in} \rule{.9\linewidth}{0pt}}
\includegraphics[width=0.9\textwidth]{basescrop.png}
\end{center}
   \caption{Саудерын судалгаагаар газар доорх болон далайн доорх баазууд, мөн далай доорх шумбагч онгоцны хонгилоор эх газарт хүрэх замууд хамгийн магадлалтай байгаа нарийн байршлуудыг харуулсан газрын зураг. Саудер \textit{"итгэлтэйгээр [эдгээрээс] \textbf{илүү олон} байгууламж бий"} гэж хэлсэн байна \cite{22}.}
   \label{fig:4}
\end{figure*}

Энд Саудерын эх сурвалжуудаас эдгээр баазуудын заримын цар хүрээг тодорхойлсон гэрчийн мэдүүлгийн хэсгүүдийг багтаав:

\begin{flushleft}
\begin{enumerate}
    \item Кэмп Дэвид, Мэрилэнд: \textit{"Эх сурвалж минь надад Кэмп Дэвидын газар доорх хэсгүүд маш өргөн цар хүрээтэй, нарийн төвөгтэй бөгөөд маш олон километрийн нууц хонгилуудтай учраас яг энэ байгууламжийн бүтэн зургийг хэн нэг хүн толгойдоо бүрэн хадгалж чадах эсэх нь эргэлзээтэй гэж хэлсэн"} \cite{22}.
    \item Цагаан ордон, Вашингтон ДС: \textit{"Манай нэг найзыг 1960-аад онд Линдон Б. Жонсон ерөнхийлөгчийн үед энэ байгууламж руу авч орсон. Цагаан ордонд нэгэн цахилгаан шатаар доош шууд буусан гэдэг. Тэр цахилгаан шат 17 давхар доош явсан болов уу гэж таамаглаж байгаа. Газар доор хаалга онгойход түүнийг алсад тасарсан мэт санагдах, урт хонгилоор дагуулан явуулсан. Тэр хонгилоос өөр хаалга, хонгилууд салж байв"} \cite{22}. Зураг~\ref{fig:3}-д дүрсэлсэн.
    \item Форт Мид, Мэрилэнд - 1970-аад онд санамсаргүйгээр "подвал" руу орсон нэгэн эх сурвалжаас: \textit{"Намайг хаалга онгойлгоход доошоо буух шат харагдсан. Би дөхөж очоод хашлага дагуу доош харахад доошоо хэдэн давхар бэ гэдгийг тоолоогүй ч, ойролцоогоор 15-20 давхар юм шиг санагдсан... нэг давхар уруудахад нэгэн хаалга байсан ба үүгээр толгойгоо цухуйлган зүүн, баруун тийш харвал хоёр чиглэлд бараа сураггүй урсах хонгил харагдсан. Энэ нь газар дээрх барилга болон машины зогсоолын хамрах хүрээнээс хамаагүй том. Эсрэг талын ханан дагуу 9-12 метрийн зайтай хаалганууд байсан... Би бусад давхруудыг үзэхээр ахин нэг давхрыг буухад мөн л ижил бүтэцтэй байсан... ахиад нэг давхар уруудахад эхний 2 давхартай яг ижил зураглалтай байв"} \cite{22}.
\end{enumerate}
\end{flushleft}

\begin{figure}[t]
\begin{center}
% \fbox{\rule{0pt}{2in} \rule{0.9\linewidth}{0pt}}
   \includegraphics[width=1\linewidth]{undersea.jpg}
\end{center}
   \caption{Усан доорх баазын зураглал, Вальтер Кёршнерийн бүтээл. Тэрээр 1960-аад онд АНУ-ын тэнгисийн цэргийн Хятадын нуур, Калифорни Зэвсгийн төвийн Тэнгисийн цэргийн Рок-сайт усан доорх баазын багт зурагчаар ажиллаж байв. Саудерын нэгэн эх сурвалж Хятадын нуурт нэг миль буюу 1.6 км гүнд газар доорх бааз бий гэдгийг илчилсэн байдаг \cite{22,23}.}
\label{fig:5}
\label{fig:onecol}
\end{figure}

Саудер мөн 3,218 цаг км хурдтай явдаг газар доорх соронзон хөвөлтийн галт тэрэгнүүд, далайн ёроолд баригдсан баазууд (Зураг \ref{fig:5}), мөн хуурай газар руу чиглэсэн усан доорх шумбагч онгоцны хонгилуудын талаар мэдүүлэг авсан. Мексикийн булан дахь усан доорх баазын нэгэн мэдүүлгийн тухайд Саудер хэлэхдээ, \textit{"Усан дор болон газар доорх баазууд ном хэвлэгдсэний хагас жилийн дараа надтай нэг хүн холбогдож, өөрөө ер бусын усан доорх төслийн талаар мэдээлэл бий гэж хэлсэн... тэрээр уг төсөл Мексикийн булангийн ёроолд байдаг ба Парсонс компани гүйцэтгэгчээр нь ажилласан гэдгийг тодорхойлсон. Цаашлаад, Парсонс компани далайн ёроолоос 850 метр гүнд ажиллах зориулалттай тусгай тоног төхөөрөмж худалдан авсан талаар ярив... Тэр тоног төхөөрөмж нь суурилагдах газарт хүмүүс оршин суудаг болов уу гэмээр онцолмоор байсан"} \cite{22}.

\begin{figure}[t]
\begin{center}
% \fbox{\rule{0pt}{2in} \rule{0.9\linewidth}{0pt}}
   \includegraphics[width=1\linewidth]{sub.jpg}
\end{center}
   \caption{Усан доорх шумбагч онгоцны хонгилын зураг, Вальтер Кёршнерийн бүтээл \cite{22,23}.}
\label{fig:6}
\label{fig:onecol}
\end{figure}
\begin{figure}[t]
\begin{center}
% \fbox{\rule{0pt}{2in} \rule{0.9\linewidth}{0pt}}
   \includegraphics[width=1\linewidth]{iran.jpeg}
\end{center}
   \caption{Ираны албан ёсны видеоноос авсан нь, тэд газар доорх "пуужингийн хот"-оо харуулж байна \cite{39,40}.}
\label{fig:12}
\label{fig:onecol}
\end{figure}

Хэрвээ үнэхээр бидний хөл доор нүсэр том, нууц, тив дамнасан, гадаргуугаас олон километрийн гүнд ухагдсан 170 гаруй  сүлжээ, газар доорх болон далайн ёроолын бааз, тэдгээрийг холбосон дуунаас хурдан вакум хоолойн соронзон хөвөлттэй галт тэрэг, бидний хөдөлмөрийн үр шимээр санхүүжүүлэгдсэн байгаа бол өнөөдөр хүн төрөлхтний ихэнх нь гүн бөгөөд жаргалтай мунхаглалд байгаа бөгөөд зөвхөн өөрсдийн доор юу байгааг мэдэхгүй төдийгүй, ойрын ирээдүйд өөрсдийг нь юу хүлээж байгааг ч мэдэлгүй, улстөрч "арчилж дэмжигчдийн" нь хоосон, хоорондоо хуйвалдан бэлдсэн мэдэгдлүүдэд автсаар байна.

Нэмэлт тайлбар - Ойрх дорнодод өрнөж буй зөрчилдөөний үед томоохон газар доорх хонгилын сүлжээнүүдийн оршин байгааг эргэлзээгүйгээр илчилсэн (Газа хэсгийн газар доор ухсан ХАМАС-ын хонгилууд \cite{38}, мөн Ираны газрын доорх "пуужингийн хот" (Зураг \ref{fig:12}) \cite{39,40}). Эдгээр нь ийнхүү бүтцийг барих боломж болон бодитойгоор оршин байгаа эсэхэд эргэлзэх зай үлдээлгүйгээр баталж байгаа юм. Мөн бусад илүү өндөр санхүүжилттэй улсууд ямар байгууламжууд барьсан байж болох талаар бидэнд бодол төрүүлэх учиртай.

\subsection{Нэмэлт нөөц хонгил, гамшгийн бэлэн байдлын нотолгоо}

\begin{figure}[t]
\begin{center}
% \fbox{\rule{0pt}{2in} \rule{0.9\linewidth}{0pt}}
\includegraphics[width=1\linewidth]{tyrol.jpg}
\end{center}
\caption{Өмнөд Тирол дахь хоргодох байр, Швейцар. Европын Альпын нуруу дамжсан Швейцар нь уулын хоргодох байруудаа ухаалгаар нууж чаддагаараа алдартай \cite{32}.}
\label{fig:7}
\label{fig:onecol}
\end{figure}

\begin{figure}[t]
\begin{center}
% \fbox{\rule{0pt}{2in} \rule{0.9\linewidth}{0pt}}
   \includegraphics[width=1\linewidth]{svalbard.jpg}
\end{center}
   \caption{Норвегт байрлах Свалбардын Дэлхийн Үрийн Сан нь нэг саяас илүү үр хадгалж байна \cite{24}. Ямар нөхцөлд уг асар нөөцийг ашиглах шаардлагатай болох талаар гайхахгүй байхын аргагүй.}
\label{fig:8}
\label{fig:onecol}
\end{figure}

АНУ-ын газар доорх хааны баазуудаас гадна дэлхий даяар гамшгийн бэлтгэлтэй холбоотой олон нэмэлт ул мөр бий. Норвег, Швейцар, Швед, Финлянд нь сайн жишээ юм:

\begin{flushleft}
\begin{enumerate}
    \item Камелот төсөл нь нэгэн Норвег улс төрчийн холбогдох мэдүүлгийг хуваалцсан \cite{25,26}, түүнийг хэн болохыг нь баримтжуулсан ч, нэрийг нь нууц хэвээр хадгалсан байна. Тэр Норвегт 18 өргөн цар хүрээ бүхий газар доорх бааз байгаа, мөн Норвеги (Израил болон "өөр олон улс орнууд"-ын хамт) эдгээр баазуудыг ямар нэгэн байгалийн гамшигт бэлтгэхээр барьж байгаа гэж мэдэгдсэн. Ричард Саудер мөн Норвегт нэгэн ухсан уулын дотор баригдсан асар том газрын доорх бааз дотор орж үзсэн хүнээс мэдүүлэг авсан \cite{22}.
    \item Швейцар улс Альпын өндөрлөгт олон цөмийн хоргодох байрыг барьсан гэдгээрээ алдартай (Зураг \ref{fig:7}). Нийт 370,000 гаруй баазтай гэх бөгөөд энэ нь тус улсын бүх оршин суугчийг хоргодох байраар хангахад хангалттай юм \cite{27}.
    \item Швед болон Финлянд нь томоохон хот бүрийн оршин суугчдыг хоргодох байраар хангахад хүрэлцэхүйц нөөцтэй \cite{27}. 
\end{enumerate}
\end{flushleft}

Силикон хөндийн бизнесийн тэрбумтнууд ч бас үүний талаар мэдэж байгаа бололтой. Энэ тал дээр, \textit{"ЛинкедИн-ий хамтран бүтээгч бөгөөд нэрт хөрөнгө оруулагч Рейд Хоффман, энэ жил Нью Йоркер сэтгүүлд хэлэхдээ Силикон хөндийн тэрбумтнуудын 50\%-иас илүү нь ямар нэг "гамшгийн даатгал", жишээлбэл газрын доорх хоргодох байр худалдаж авсан хэмээн тооцоолж байгаагаа дурдсан... Форбес-ийн зохиолч Жим Добсон-ы хэлснээр олон тэрбумтан хувийн онгоцтой бөгөөд "ямар ч үед явахад бэлэн" байдаг. Тэд мөн мотоцикл, зэвсэг, цахилгаан үүсгүүр ч эзэмшдэг"} \cite{28}.

Мөн хүн төрөлхтний гол нөөцийг мөхлийн гамшгийн үед хадгалахаар зорилготой Дэлхийн мэдлэгийн санг Арч Мишион Сан удирдаж байгаа \cite{29}, мөн Свалбард Дэлхийн үрийн сан \cite{30} зэрэг томоохон архивын төслүүд бий.
\begin{figure*}[t]
\begin{center}
% \fbox{\rule{0pt}{2in} \rule{.9\linewidth}{0pt}}
\includegraphics[width=0.9\textwidth]{govcrop2.png}
\end{center}
   \caption{АНУ-ын засгийн газрын орлого, зарлага, болон нууц газар доорх баазын зарлага 1998-2023 оны хооронд \cite{19}.}
   \label{fig:9}
\end{figure*}
\section{Асар том газар доорх баазуудыг ардчилсан байдлаар санхүүжүүлэх механизм}

Тив дамнасан 170 гаруй газар доорх болон далай доорх асар том баазын сүлжээг хэрхэн санхүүжүүлэхийн зэрэгцээ өрний боолуудыг хэрхэн харанхуйд байлгах вэ? Эдгээр төслүүдэд ямар хэмжээний мөнгө ордог, мөн хаанаас ирдгийг бидэнд харуулж чадах нэг цаасны мөр байна. 2017 онд Америкийн хөрөнгө оруулалтын банкир, Бушийн засаг захиргааны хуучин албан тушаалтан Кэтрин Остин Фитс болон Мичиганы Мужийн Их Сургуулийн эдийн засагч Марк Скидмор нар АНУ-ын засгийн газар 1998-2015 санхүүгийн онуудад зөвшөөрөлгүйгээр 21 их наяд ам.долларын зарцуулалт хийсэн болохыг олж тогтоожээ \cite{11,12,13}.

Тэдний илтгэлд дурдсанаар, \textit{"2016 оны аравдугаар сарын 7-нд Ройтерс агентлагаас Скэт Палтроу (2016) нийтлэл нийтэлсэн бөгөөд түүнд 2015 санхүүгийн онд Армийн зүгээс \$6.5 их наяд ам.долларын нотолгоогүй нягтлан бодох бүртгэлийн тохируулга хийж, \"өрийн дэвтрүүдээ тэнцвэртэй харагдуулах хуурмаг сэтгэгдэл төрүүлсэн” гэжээ. Тухайн онд Армийн ерөнхий сангийн төсөв нь \$122 тэрбум байсан нь цочирдом нээлт болов... Батлан хамгаалах яам олон жилийн өмнө 2001 оны есдүгээр сарын 10-нд Төрийн нарийн бичгийн дарга Доналд Рамсфелд Конгрессын хурлын үеэр (C-SPAN, 2014) Батлан хамгаалах яам \$2.3 их наяд ам.долларын гүйлгээний баримтыг алдсан гэж хэлснээр нягтлан бодох бүртгэлийн асуудлаараа томоохон хэвлэлийн гарчиг болсон... Энэ мэдэгдэл тэр өдөр мэдээний гарчиг болсон ч маргааш нь 9/11-ийн гамшиг дэлхийн анхаарлыг татсанаар мартагдсан... Профессор Марк Скидмор Армийн нотлогдоогүй \$6.5 их наяд ам.долларын гүйлгээний талаарх мэдээг сонсоод хатагтай Фитстэй холбогдож, 2017 оны хавартаа хамтран ажиллаж, Орон сууц, хот байгуулалтын яам болон Батлан хамгаалах яаманд баталгаажуулагдаагүй, их хэмжээний бусад ижил төстэй засгийн газрын тайланд дүн шинжилгээ хийхээр тохиролцсон. Дараагийн зургаан сарын турш Скидмор, Фитс болон цөөн оюутны баг хамтран албан ёсны засгийн газрын баримтыг цуглуулж, 1998-2016 оны хугацаанд нийт \$21 их наяд ам.долларын баримтжуулагдаагүй гүйлгээг илрүүлсэн"} \cite{12}.

1998-2015 оны 18 жилийн хугацаанд нийтэд зарласан АНУ-ын засгийн газрын орлого 40.8 их наяд ам.доллар байсан \cite{15}, үүнээс харвал АНУ-ын засгийн газрын орлогын талаас илүү хэсэг нь газар доорх баазуудад нууцаар зарцуулагдсан ба үүн дээр олон нийтэд зарлагдсан зарлага нэмэгдэж орж ирэх юм. Мөн онцолмоор зүйл нь, энэ нууц зарлагад олон жилийн турш үргэлжилсэн төсвийн алдагдалтай зэрэгцэн явагдаж ирсэн бөгөөд өнөөдрийг хүртэл үргэлжилсээр байгаа, мөн 1998 оноос өмнө ч байсан байж болох учир нийт дүн нь 21 их наядаас ч хавьгүй их байна. 2016-2023 оны зарлагын хувьтай адилтган тооцвол 1998 оноос хойш нийт 36.6 их наяд ам.доллар зарцуулсан гэсэн үг.

2021 онд Марк Скидмор энэ судалгаагаа шинэчилж, Блүүмберг 2017-19 санхүүгийн оны хугацаанд Пентагон бүртгэлдээ итгэмээргүй 94.7 их наяд ам.долларын нягтлан бодох бүртгэлийн тохируулга хийсэн тухай зарласантай холбогдуулан нийтэлсэн байна \cite{17,18}. Хэрвээ 1913 онд Холбооны Нөөцийн тогтолцоо байгуулагдсанаас хойш зуу гаруй жилийн турш төв банкны тогтолцоогоор АНУ-ын долларыг хуурамчаар бүтээгээд байгааг харгалзан үзвэл, нийтийн долларын тооцоо бол цэвэр хоёрдмол иртэй, утгагүй зүйл гэдэг тодорхой байна. АНУ-ын мөнгө болон засгийн газар нь зүгээр л нөөцийн хуваарилалтын тогтолцоо болж хувирсан бөгөөд үүний хаад эзэд дуртай мөнгөө чимээгүй хусаж (эсвэл бүр цутгаж) чаддаг болжээ.

\section{Йовын үр сад: Барууны сүүдрийн хаадын үнэн дүр төрх}

Тэгэхээр, хэн үнэндээ энэ тоглолтын ард байна вэ? Бид үүнийг яг мэдэх боломжгүй, учир нь хөрөнгийг гартаа авсан барууны хаад өөрсдийгөө сүүдэрт нууж байдаг. Янз бүрийн онол таамаг дэвшүүлэхдээ олны танил хүмүүсээс эхлээд харь гаригийн амьтад хүртэл багтаадаг ч, миний хамгийн сайн хариултыг "Амаллула" гэх нэрээр алдаршсан нэгэн нэрээ нууцалсан блогчин хүний амьдралын бүтээлээс олж болно. Түүний бүтээл нь 20 гаруй зохиолч, 50 "орлуулшгүй" баримтыг хамарсан өргөн хүрээний судалгаа бөгөөд эртний ба орчин цагийн түүх, нууцлаг бэлгэ тэмдэг, барууны улс төрийн сэдвүүдийг хамарсан юм \cite{33,34}. Би түүний ажлыг удахгүй болох геофизикийн сүйрлийн талаарх "зөгнөлт" гэж итгэлтэйгээр хэлж чадна - энэ нь миний судалгаанаас \textit{илт давуу} өргөн хүрээтэй.

Амаллула нь барууны улс төрийн гурван бүлгийг тодорхойлж, эдгээрийг нийлүүлэн "Йовын үр сад" гэж нэрлэсэн бөгөөд тэд дэлхийн "сүүлийн мөч" - давтан тохиолддог галав юүлэх аюулын тухай мэдлэгтэй гэж үзжээ. Тэрээр эдгээр гурван бүлэг хамтдаа өнөөгийн барууны улсуудыг удирддаг гэж итгэдэг ч, гарал үүсэл, түүхэн үнэн дүр төрх, өнгөрсөн маргаан зөрөлдөөн, үнэт зүйл, үйлдлийн ялгаа зэргийг үндэслэн гурван өөр бүлэгт хуваасан байна.

Гурван бүлгийг ерөнхийд нь дараах байдлаар ангилж болно:

\begin{flushleft}
\begin{enumerate}
\item \textbf{Банкчид}: Эртний Ромын элитүүд, хожим нь Америк дахь Тамплийн баатрууд болон Хойд нутгийн Чөлөөт өрлөгүүд болон хувирсан.
\item \textbf{Оюун ухаантнууд}: Розенкройцчууд болон Өмнөд Америкийн чөлөөт өрлөгүүд.
\item \textbf{Езуитүүд ба Хар Пап}: Йовын үр сад буюу Ромын Католик сүм дэх Зевсийн үр удмын бүлэглэл. 
\end{enumerate}
\end{flushleft}

Өнөө үед эдгээр гурван бүлэглэл нийлж Европын Иллюминатууд, Чөлөөт өрлөгүүд, мөн Тагнуулын төв газар (CIA)-г бүрдүүлдэг. Амаллуллагийн бичсэнээр, \textit{"Яг одоо, төгсгөлийн мөчид, Йовын үр удам нь АНУ-ын одоогийн ерөнхийлөгч хүртэл мэдэх ёсгүй, тусгай зөвшөөрлийн ард сайнаар нуугдаж байгаа. Өөрөөр хэлбэл, тэд өөрсдийгөө олон нийтийн хяналтаас төгс нуудаг болсон. \textbf{Йовын үр удам зөвхөн Америкийн Нэгдсэн Улсын цэргийн болон засгийн эрхийг төдийгүй, цаасан мөнгөний хүч, томоохон корпорацууд, өөрсдийн бүтээсэн бүгд найрамдах засаглалаар (улс төрчид амархан авлигад өртдөг, түүнд орж удирдах боломжтой гэдгийг мэдэж байсан учраас) барууны ертөнцийг тэр чигээр нь хянадаг}"} \cite{33,34}.

\begin{figure}[t]
\begin{center}
% \fbox{\rule{0pt}{2in} \rule{0.9\linewidth}{0pt}}
   \includegraphics[width=1\linewidth]{illuminati.jpg}
\end{center}
   \caption{Йовегийн үр удам гэж яг хэн бэ? (Зураг: \cite{35})}
\label{fig:10}
\label{fig:onecol}
\end{figure}

\begin{figure}[t]
\begin{center}
% \fbox{\rule{0pt}{2in} \rule{0.9\linewidth}{0pt}}
   \includegraphics[width=1\linewidth]{pike.jpg}
\end{center}
   \caption{Алдарт Пайк Пийкийн батолит, улаанаар тодруулсан, АНУ-ын өрнөд нутгийн байгаль орчин хамт байна \cite{36}. Америкийн Нэгдсэн Улс үнэхээр энэ газрыг хянахын тулд үүссэн байж болох уу?}
\label{fig:11}
\label{fig:onecol}
\end{figure}

Амаллуллагийн хэлснээр, эдгээр хүмүүс шашныг басамжилж, дэлхийн томоохон шашнуудын ариун судруудыг өөрсдийн ашиг сонирхлын үүднээс өөрчилдөг бөгөөд бэлгэ тэмдгийг ихээр ашигладаг. Мөн тэд дайснууддаа өршөөлгүй ханддаг: \textit{"\textbf{2600 гаруй жилийн хугацаанд тэд дэлхийн төгсгөлийн тухай онцгой мэдлэгтэй байсан хэнийг ч бай системтэйгээр устгаж ирсэн. Энэ нь зөвхөн друидууд, еврей каббалистууд, эртний египетчүүд, арабчууд, энэтхэгийн мэргэдийг хэлээд зогсохгүй, Өмнөд Америкийн сунасан гавалтнууд болон Төв Америкийн Майя шүтээний лам нарыг ч хамруулж хэлж байгаа юм. Тэд Хойд Америкт цэцэглэн хөгжиж байсан хүн амыг ч мөн устган үүнийг Дэлхийн төгсгөлийн орон болгон хадгалсан нь маргаангүй нотолгоотой. Америкийн “уугуул иргэд”-ыг устгасан нь зүгээр л үлдэгдэл цэвэрлэх ажиллагаа байсан.}"} \cite{33,34}.

Амаллулла мөн "Америкийн Нэгдсэн Улс" төслийг бүхэлд нь "Пайкс Пик Батолит"-ийг хянах зорилгоор хэрэгжүүлсэн гэж үзсэн бөгөөд энэхүү боржин уулын нуруу нь Газрын ховор геофизикийн гамшгаас хамгаалах маш сайн байрлалтай юм (Зураг \ref{fig:11}). Амаллуллагийн хэлснээр, \textit{"АНУ-ын Иргэний дайн болсны өмнө, үеэр болон хойно санхүүчид ба сэтгэгчид Америкийн Нэгдсэн Улсыг хянах гэж бус, харин дэлхийн хамгийн өвөрмөц боржин батолит болох Пайкс Пик батолитийг хянахын төлөө тэмцсэн... Ийм өндөрт, далайн эргээс ийм хол байрладаг өөр ямар ч боржин батолит дэлхий дээр байхгүй. Энэ бол газар дэлхийн цахилгаан хөдөлгөөний гажилтад тэсвэртэй үлдэхэд хамгийн тохиромжтой байрлал"} \cite{33,34}. Амаллуллагийн судалгааны дагуу өнөөдөр энэ бүс нутгийн доор болон эргэн тойронд өргөн хэмжээний газар доорх хонгилын систем баригдсан байна \cite{36}.

\section{Дүгнэлт}

Энэ өгүүлэлд би өрнөдийн элитүүд хэдэн мянган жилийн турш дэлхийд давтан тохиох галав юүлэлтийн тухай мэдлэгээ болгоомжтой хадгалж ирсэн, тун удахгүй дахин тохиолдох гэж итгэдэг, энэ мэт үйл явдалд бэлдэж газар доор өргөн далайцтай хоргодох байгууламж барьсан, мөн ийм нөхцөлийг улс төрийн болон цэргийн хувьд ашиглан дэлхий ертөнцийн хяналтыг авахаар төлөвлөж буйг гэрчлэх тал талын мэдүүлгийг дэлгэрэнгүй тайлбарласан. Энэ бүхнийг хэрхэн санхүүжүүлсэн тухай Америк дахь баримтуудыг мөн дурдсан бөгөөд, нөхцөл байдлыг удирдаж буй яг аль овгийн удам болох талаар хамгийн бодитой онолыг иш татлаа. Илүү ихийг мэдэхийг хүсэж байгаа хүмүүст, би дурдаагүй үлдээсэн олон нэмэлт мэдээлэл нь эшлэл хэсгээс олдох юм.

Ойртож буй геофизикийн үйл явдлын хамгийн хүчтэй хэмжигдэхүйц нотолгоо нь дэлхийн гео-соронзон орон зай хурдацтай өөрчлөгдөж байгаа явдал юм. Энэ нь зөвхөн соронзон умард туйл түргэн шилжиж байгаа (Зураг \ref{fig:13}) болон Өмнөд Атлантын гео-соронзон гажилт өсөж байгаа төдийгүй, сүүлийн 400 жилийн турш ерөнхийдөө гео-соронзон орон зайн суларч, гажиж буйг хэмжихэд илэрхий байгаагаар нотлогдоно \cite{3}. Ийм шинжлэх ухааны өгөгдлийг миний анхны хоёр ECDO өгүүлэлд нарийвчлан хэлэлцсэн бөгөөд тэдгээрийг миний вебсайтаас унших боломжтой \cite{3}.

\begin{figure}[t]
\begin{center}
% \fbox{\rule{0pt}{2in} \rule{0.9\linewidth}{0pt}}
   \includegraphics[width=1\linewidth]{npw.jpg}
\end{center}
   \caption{1590 оноос 2025 он хүртэлх гео-соронзон хойд туйлын байрлалыг 5 жилийн давтамжтайгаар харуулсан байна \cite{41}. Түүний хөдөлгөөн 1975 оноос эрчимтэй хурдасжээ.}
\label{fig:13}
\label{fig:onecol}
\end{figure}

Эцэст нь би танд сургагч Амаллуулагийн энэхүү эшлэлийг үлдээмээр байна. Тэрээр \textit{"\textbf{бүх юм ганц зүйл}"} хэмээн ийнхүү тайлбарласан: \textit{"Энд би таны төсөөллийг туйлын заагт хүртэл нь түлхэхээс өөр аргагүйд хүрлээ. Та одоо амьдарч буй, бага наснаасаа мэддэг байсан хорвоог мартан орхих хэрэгтэй. Ардаа орхи. Энэ бол Матрикс кинонд дүрслэгддэг шиг, хүмүүсийг сүүлийн мөч хүртэл унтаа байлгахыг зорьсон, бүрэн зохиомол бодит байдал юм. Заримдаа би өөрийгөө киноны зохиол бичиж байгаасай гэж хүсдэг. Гэвч энэ вэбсайт дээр танд хуваалцаж буй зүйл бол бодит үнэн. “Бүх юм ганц зүйл” гэдгийг ойлгоход хагас арван жил зарцуулсан бөгөөд үүнийг би сүйрлийн нэгдмэл үзлийн уриа болгон сонгосон. Үүнийг тайлбарлахад тун хэцүү. Түр зуур Матрикс киноны жишээгээр бодъё. Энэ нь маш сайн зүйрлэл болж чадна. Харин надад хамгийн хэцүү нь хэлэх гэж буй маань дэгсдүүлэг биш гэдгийг ойлгуулах явдал. Яг одоогоор Матрикстай зүйрлэх нь хэлэх гэж буй бодит байдалд хамгийн ойр сонголт байна. \textbf{Таны амьдрал дахь бүхий л зүйлс, бичигдсэн түүх, түгээмэл, нийтэд зөвшөөрөгдсөн шинжлэх ухаан болон академи, улс төр, шашин гээд бүгд нэг юман дээр л, газар дэлхийн хөрс шилжилт эсвэл тэнхлэгийн хазайлттай холбоотой юм.} Та үүнийг одоо харахгүй байна. Мөн энэ бодит байдлаас гэнэт сэрнэ гэж байхгүй. Үүнд цаг хугацаа шаардагдана. Гэхдээ замын төгсгөлд та бүхий л амьдралаа Матрикс гэх компьютерын загварчилсан бодит байдалд амьдарч байсныгаа ойлгох болно гэдгийг би амлаж байна"} \cite{33,34}.

Та бүхэнд амжилт хүсье.

\section{Талархал}

Мэдлэгээ нийтийн хүртээл болгон хуваалцсан хэн бүхэнд баярлалаа. Тангүйгээр энэ ажил боломжгүй байх байсан ба хүн төрөлхтөн харанхуйд үлдэх байлаа. Таны сонголт үүрд мөнхөд цэцэглэнэ. Бид бүгд танд өртэй бөгөөд миний зүгээс хязгааргүй талархаж байна.

\clearpage
\twocolumn

{\small
\bibliographystyle{ieee}
\bibliography{egbib}
}

\end{document}
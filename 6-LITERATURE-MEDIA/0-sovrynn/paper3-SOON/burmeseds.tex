\documentclass[10pt,twocolumn,letterpaper]{article}

\usepackage{booktabs}
% \usepackage{caption}
% \captionsetup[table]{skip=8pt}   % ဇယားများကိုသာ သက်ရောက်မှုရှိသည်
\usepackage{stfloats}  % ဤအား preamble တွင် ထည့်ပါ
\usepackage{float}

% \usepackage{fontspec}
\usepackage[english]{babel}

% load Lao via babelprovide, turn on "onchar=ids" for automatic shaping
\babelprovide[import,onchar=ids fonts]{burmese}

% main (rm) font for Latin
\babelfont{rm}{Noto Serif}

% Lao text in Noto Serif Lao at 1.2× scale
\babelfont[burmese]{rm}{Padauk}
\babelfont[burmese]{sf}{Padauk}

% alternate (sans-serif) font for Latin
\babelfont{alt}{Lato}

% Lao text in Noto Serif Lao for the alt family too
\babelfont[lao]{alt}{Padauk}

% GPT: LuaTeX doesn’t have built-in Lao line-breaking rules, but babel can assimilate line breaks to hyphenation if you supply some simple “patterns” for common syllable boundaries. Add this to your preamble:
\babelpatterns[burmese]{%
  1က 1ခ 1ဂ 1ဃ 1င 1စ 1ဆ 1ဇ 1ဈ 1ဉ 1ည 1ဋ 1ဌ 1ဍ 1ဎ 1ဏ
  1တ 1ထ 1ဒ 1ဓ 1န 1ပ 1ဖ 1ဗ 1ဘ 1မ 1ယ 1ရ 1လ 1ဝ 1သ 1ဟ
  1ဢ 1ါ 1ာ 1ိ 1ီ 1ု 1ူ 1ေ 1ဲ 1ဲ
  1ျ 1ြ 1ွ 1ှ
  1္ 1့ 1း 1ံ 1်%
}

\usepackage{cvpr}
\usepackage{times}
\usepackage{epsfig}
\usepackage{graphicx}
\usepackage{amsmath}
\usepackage{amssymb}
\usepackage[breaklinks=true,bookmarks=false]{hyperref}

\makeatletter
\def\cvprsubsection{\@startsection {subsection}{2}{\z@}
    {8pt plus 2pt minus 2pt}{6pt}{\bfseries\normalsize}}
\makeatother

\cvprfinalcopy % *** အဆုံးသတ်တင်ပြရန်အတွက် ဤစာကြောင်းကို မဖြည့်ပါနှင့်

\def\cvprPaperID{****} % *** CVPR စာတမ်းအတွက် ID ကို ဤနေရာတွင် ထည့်ပါ
\def\httilde{\mbox{\tt\raisebox{-.5ex}{\symbol{126}}}}

% စာမျက်နှာများကို တင်သွင်းသည့်ပုံစံတွင် နံပါတ်တပ်ပြီး ကင်မရာပြီးသားတွင် နံပါတ်မတပ်ပါ
%\ifcvprfinal\pagestyle{empty}\fi
\setcounter{page}{1}
\begin{document}

\title{ECDO စာတမ်း ၃: အနောက်တိုင်း အုပ်စိုးသူများ၏ မကြာမီဖြစ်ပေါ်လာမည့် ဘူမိရူပ ကပ်ဘေးအတွက် ပြင်ဆင်မှုများအပေါ် အထောက်အထား}

\author{Junho\\
ဇွန် ၂၀၂၅ တွင် ထုတ်ဝေခဲ့သည်\\
ဝက်ဘ်ဆိုက် (စာတမ်းများ ဒေါင်းလုတ်ရယူရန်): \href{https://sovrynn.github.io}{sovrynn.github.io}\\
ECDO သုတေသန ရီပိုစီတိုရီ: \href{https://github.com/sovrynn/ecdo}{github.com/sovrynn/ecdo}\\
{\tt\small junhobtc@proton.me}
}
\maketitle
%\thispagestyle{empty}

\begin{abstract}
In May 2024, a pseudonymous online author known as “The Ethical Skeptic” \cite{0} shared a groundbreaking theory called the Exothermic Core-Mantle Decoupling Dzhanibekov Oscillation (ECDO) \cite{1}. This theory suggests that Earth has previously experienced sudden, catastrophic shifts in its rotational axis, triggering massive worldwide floods as the oceans spilled over the continents due to rotational inertia. Additionally, it presents an explanatory geophysical process and data indicating that another such flip may be imminent. While such cataclysmic flood and doomsday predictions are not new, the ECDO theory is uniquely compelling due to its scientific, modern, multidisciplinary, and data-based approach.

This paper is my third work \cite{2,3} on this topic, and focuses on present-day political aspects of this theory:
\begin{flushleft}
\begin{enumerate}
    \item အနောက်နိုင်ငံများက ကမ္ဘာလုံးဆိုင်ရာ ဘေးအန္တရာယ်ကြီးတစ်ခု ဖြစ်တော့မည်ဟု ယုံကြည်နေပြီး ၎င်းဖြစ်စဉ်မှ နိုင်ငံရေးနှင့် စစ်ရေးအကျိုးကျေးဇူးများရယူရန် အစီအစဉ်ရှိကြောင်း ဖော်ထုတ်ပြောဆိုသူများ၏ ထွက်ဆိုချက်များ။
    \item ထိုဖြစ်ရပ်အတွက် ပြင်ဆင်မှုအဖြစ် အနောက်နိုင်ငံများက တည်ဆောက်ထားသော မြေအောက်နှင့် ပင်လယ်အောက်စခန်းများ၏ အထောက်အထားများ။
    \item ဤစခန်းများအတွက် ငွေကြေးထောက်ပံ့ရန် အနောက်နိုင်ငံများ၏ ငွေကြေးစနစ်များမှ အလုံးအရင်းနှင့် ငွေများ စီးဆင်းနေမှုဆိုင်ရာ အထောက်အထားများ။
\end{enumerate}
\end{flushleft}

ဤစာတမ်းသည် အနောက်တိုင်း အုပ်စိုးသူ အာဏာပိုင်များက ၎င်းတို့ ချဉ်းကပ်လာနေသည်ဟု ယုံကြည်နေသော ဘူမိရူပ ကပ်ဘေးကြီးအတွက် ပြုလုပ်နေသည့် ကျယ်ပြန့်သော ပြင်ဆင်မှုများကို မှတ်တမ်းတင်ထားသည်။
\end{abstract}

\section{ဖရီးမဆင်အသင်းနှင့် "အင်္ဂလို-ဆက်ဆွန် အစီအစဉ်"}

၂၀၁၀ ခုနှစ် ဇန်နဝါရီလတွင်၊ whistleblower များ၏ သက်သေခံချက်များကို စုစည်းထားသည့် အခြားမီဒီယာနှင့် သတင်းစာနှီးနှောဖလှယ်မှုအဖွဲ့ Project Camelot သည် \cite{4,6} ၂၀၀၅ ခုနှစ် ဇွန်လတွင် လန်ဒန်မြို့ရှိ Senior Masons အစည်းအဝေးတစ်ခုတွင် ရုပ်ပိုင်းဆိုင်ရာတက်ရောက်ခဲ့သူတစ်ဦးနှင့် တွေ့ဆုံမေးမြန်းခဲ့သည်။ အဆိုပါအစည်းအဝေးတွင် ဆွေးနွေးခဲ့သည့်အကြောင်းအရာများမှာ ကမ္ဘာလုံးဆိုင်ရာ သဘာဝဘေးအန္တရာယ်တစ်ခုဖြစ်သည့် \textbf{"ဘူမိရူပဖြစ်စဉ်"} နောက်ခံတွင် ဗဟိုပြုထားသော စစ်ရေးနှင့် နိုင်ငံရေးအစီအစဉ်များဖြစ်သည်။

\begin{figure}[b]
\begin{center}
\includegraphics[width=1\linewidth]{freemason.jpg}
\end{center}
   \caption{ဗြိတိသျှ ဖရီးမဲဆင်များ ၎င်းတို့၏ သဘာဝအတိုင်း နျူကလီးယားဗုံးများ ချပစ်ကာ ကမ္ဘာ့အာဏာသိမ်းရန် တိတ်တဆိတ် အကြံအစည်များ ရက်စက်နေပုံ - ၁၉၉၂ ခုနှစ်တွင် လန်ဒန်မြို့ရှိ Earls Court တွင် \cite{5}။}
\label{fig:1}
\label{fig:onecol}
\end{figure}

\begin{figure*}[t]
\begin{center}
\includegraphics[width=1\textwidth]{british.jpg}
\end{center}
   \caption{၁၉၃၇ ခုနှစ်က ဗြိတိသျှအင်ပါယာ၊ အင်္ဂလို-ဆက်ဆွန်တို့၏ အံ့မခန်းဖွယ်ရာ စွမ်းအားပြပုံ \cite{14}.}
   \label{fig:2}
\end{figure*}
ဒီအတွင်းလူရဲ့ အဆိုအရ၊ အစည်းအဝေးမှာ ပါဝင်ခဲ့သူ ၂၅-၃၀ ဦးဟာ \textit{"...အားလုံးဗြိတိသျှနိုင်ငံသားတွေဖြစ်ပြီး၊ အချို့ကတော့ ယူကေမှာရှိတဲ့ လူတွေ အလွယ်တကူ မှတ်မိကြမယ့် အထင်ကရ ပုဂ္ဂိုလ်တွေပါ... အထက်တန်းစားလွှာ အချို့လည်း ပါဝင်ပြီး၊ အချို့ကတော့ အထက်တန်းစား မိသားစုတွေကနေ ဆင်းသက်လာသူတွေပါ။ အဲဒီအစည်းအဝေးမှာ ကျွန်တော် ဖော်ထုတ်နိုင်ခဲ့တဲ့ တစ်ဦးကတော့ အကြီးတန်း နိုင်ငံရေးသမားတစ်ဦးပါ။ အခြား နှစ်ဦးကတော့ ရဲတပ်ဖွဲ့က အကြီးတန်း ပုဂ္ဂိုလ်တွေဖြစ်ပြီး၊ တစ်ဦးကတော့ စစ်တပ်ကပါ။ သူတို့အားလုံးဟာ နိုင်ငံတစ်ဝှမ်း လူသိများပြီး လက်ရှိအစိုးရကို အကြံပေးနေတဲ့ အဓိက ပုဂ္ဂိုလ်တွေပါ"}\cite{4}။ အတွင်းလူက သူအစည်းအဝေးကို တက်ရောက်ခဲ့တဲ့အကြောင်း၊ \textit{"တကယ်တော့ အလွန်အမင်း မတော်တဆပါပဲ! သာမန် သုံးလတစ်ကြိမ် အစည်းအဝေးလို့ ထင်ခဲ့တာ... ဒီအစည်းအဝေးကို သွားတော့ ကျွန်တော် မျှော်လင့်ထားတဲ့ အစည်းအဝေးမဟုတ်ဘူး။ ကျွန်တော်ကို ဖိတ်ကြားခဲ့တာက... ကျွန်တော်ရဲ့ ရာထူးနဲ့ သူတို့လိုပဲ သူတို့ထဲက တစ်ယောက်လို့ ယုံကြည်ခဲ့လို့ပါ"}\cite{4}။

အစည်းအဝေးမှာ ဆွေးနွေးခဲ့တဲ့ အချိန်ဇယား (၂၀၀၅ ခုနှစ်က) က အောက်ပါအတိုင်းဖြစ်ပါတယ်:

\begin{flushleft}
\begin{enumerate}
    \item အီရန် သို့မဟုတ် တရုတ်ကို တိုက်ရိုက်နျူကလီးယားလက်နက် အသုံးပြုရန် လှုံ့ဆော်ကာ ကန့်သတ်ထားသော နျူကလီးယား ပဋိပက္ခ ဖြစ်ပွားစေပြီး စစ်ရပ်စဲရေး ထူထောင်ခြင်း။
    \item တရုတ်အပေါ် ဇီဝလက်နက်များ အသုံးပြုခြင်း၊ "၁၉၇၀ ပြည့်လွန်နှစ်များမှစ၍" အဓိက ပစ်မှတ်အဖြစ် သတ်မှတ်ထားသည်ဟု သိရသည်။
    \item ရလဒ်အဖြစ် ဖြစ်ပေါ်လာသော အကြောက်တရားနှင့် ရောထွေးမှုများကို အကြောင်းပြကာ စစ်အာဏာရှင်စနစ် အစိုးရများ ပေါ်ပေါက်လာစေခြင်း။
\end{enumerate}
\end{flushleft}

ဒါပေမယ့် အရေးအကြီးဆုံးကတော့ ဒီဖြစ်ရပ်တွေနောက်မှာ ဘာတွေဖြစ်လာမယ်ဆိုတာပါပဲ- \textit{"ဒါကြောင့် ကျွန်တော်တို့ဟာ ဒီစစ်ပွဲထဲကို ဝင်ရောက်သွားမယ်၊ ပြီးရင်... ကမ္ဘာပေါ်မှာ လူတိုင်းကို သက်ရောက်မှုရှိမယ့် ဘူမိရုပ်ပိုင်းဆိုင်ရာ ဖြစ်ရပ်တစ်ခု ဖြစ်ပျက်လာမယ်"} \cite{4}။ အတွင်းလူက ယုံကြည်တာက ဒီဘူမိရုပ်ပိုင်းဆိုင်ရာ ဖြစ်ရပ်ဖြစ်နေစဉ်မှာ၊ \textit{\textbf{"ကမ္ဘာ့အပေါ်ယံလွှာဟာ ဒီဂရီ ၃၀ လောက်၊ မိုင် ၁၇၀၀ ကနေ ၂၀၀၀ လောက် တောင်ဘက်ကို ရွေ့လျားသွားမယ်၊ ဒါဟာ ကြီးမားတဲ့ အပြောင်းအလဲတစ်ခုကို ဖြစ်စေမယ်၊ ဒါရဲ့အကျိုးဆက်တွေဟာ အချိန်အကြာကြီး ကြာမြင့်မယ်"}} \cite{4}။

ဒီလျှို့ဝှက်စီစဉ်မှုတွေရဲ့ အကြောင်းရင်းကတော့ အာဏာပါပဲ။ အတွင်းလူက ရှင်းပြတာက၊ \textit{"အဲဒီအချိန်မှာတော့ ကျွန်တော်တို့အားလုံးဟာ နျူကလီးယားနဲ့ ဇီဝစစ်ပွဲတွေကို ဖြတ်သန်းပြီးသွားပြီ။ ဒါဖြစ်လာရင် ကမ္ဘာ့လူဦးရေဟာ အလွန်အမင်း လျော့နည်းသွားမယ်။ ဒီဘူမိရုပ်ပိုင်းဆိုင်ရာ ဖြစ်ရပ် ဖြစ်ပျက်တဲ့အခါ ကျန်ရှိနေသူတွေဟာ ထပ်ပြီး တဝက်လောက် လျော့သွားနိုင်တယ်။ ဒါကိုကျော်ဖြတ်နိုင်သူတွေဟာ နောက်ခေတ်သစ်ကို ကမ္ဘာနဲ့ လူဦးရေကို ဦးဆောင်သွားမယ့်သူတွေ ဖြစ်လာမယ်။ ဒါကြောင့် ကျွန်တော်တို့ပြောနေတာက ကပ်ဘေးကြီးနောက်ပိုင်း ခေတ်အကြောင်းပါ။ ဘယ်သူတွေက အာဏာရှိမလဲ? ဘယ်သူတွေက ထိန်းချုပ်မှုရှိမလဲ? ဒါကြောင့် ဒါဟာအားလုံးက ဒီအကြောင်းပါပဲ။ ဒါကြောင့်လည်း သူတို့ဟာ ဒီအရာတွေကို သတ်မှတ်ထားတဲ့ အချိန်အတွင်းမှာ ဖြစ်ပျက်ဖို့ အရမ်းစိတ်အားထက်သန်နေကြတာပါ... ဖရိုဖရဲဖြစ်မှု မဖြစ်ခင်မှာ တည်ငြိမ်တဲ့ အဆောက်အဦတစ်ခု ရှိနေဖို့လိုတယ် - ဒါက ကျရောက်လာမယ့်အရာကို ရှင်သန်ကျော်လွှားနိုင်မယ်ဆိုတဲ့ သေချာမှုမျိုးနဲ့ - ဒါကြောင့် နောက်တစ်နေ့မှာ ခြေနှစ်ဖက်စလုံးနဲ့ ရပ်တည်နိုင်ပြီး အာဏာကို ဆက်လက်ထိန်းသိမ်းထားနိုင်ပြီး အရင်က ခံစားခဲ့ရတဲ့ အာဏာကို ရရှိထားနိုင်မယ်"} \cite{4}။ တွေ့ဆုံမေးမြန်းချက်မှာ၊ "Anglo-Saxon Mission" လို့ခေါ်တဲ့ ဒီအစီအစဉ်ရဲ့ အမည်ကိုလည်း ဆွေးနွေးထားပါတယ်- \textit{[အင်တာဗျူးသူ]: "...ဒါကို The Anglo-Saxon Mission လို့ခေါ်ရတဲ့ အကြောင်းရင်းက တရုတ်တွေကို သုတ်သင်ဖို့ အစီအစဉ်ဖြစ်လို့ပါ၊ ဒါကြောင့် ကပ်ဘေးကြီးနောက်ပိုင်းနဲ့ အရာရာတွေ ပြန်လည်တည်ဆောက်တဲ့အခါမှာ Anglo-Saxon တွေသာ နေရာယူပြီး ကမ္ဘာသစ်ကို အမွေဆက်ခံနိုင်မယ်၊ အခြားသူတွေ မရှိတော့ဘူး။ ဒါမှန်လား?" [အတွင်းလူ]: "ဒါမှန်မမှန်တော့ ကျွန်တော် အတိအကျမသိဘူး၊ ဒါပေမယ့် သင့်နဲ့သဘောတူတယ်။ အနည်းဆုံး ၂၀ ရာစုတစ်လျှောက်လုံးနဲ့ ၁၉ နဲ့ ၁၈ ရာစုတွေကတည်းက ဒီကမ္ဘာ့သမိုင်းကို အနောက်တိုင်းနဲ့ ကမ္ဘာ့မြောက်ပိုင်းဒေသက ဦးဆောင်ခဲ့တာပါ"} \cite{4}။
မြေပြင်လှိုင်းဖြစ်စဉ်၏အတိအကျအချိန်ကာလနှင့်ပတ်သက်၍ အတွင်းလူကဤသို့ခန့်မှန်းထားသည်- \textit{"...ဒါကတော့ အလွန်အမှတ်မထင်သိရှိနိုင်တဲ့ခံစားချက်တစ်ခုပါ...သူတို့အခုအချိန်မှာ စနစ်တကျပြင်ဆင်ဖို့လိုနေပြီလို့ခံစားရတယ်...ဒီဖြစ်စဉ်ဘယ်အချိန်မှာဖြစ်မယ်ဆိုတာ သူတို့တော်တော်သိကြပြီလို့ထင်တယ်... \textbf{�ျွန်တော့်ဘဝတစ်လျှောက်မှာ ဖြစ်လာမယ်လို့ အလွန်ပြင်းထန်တဲ့ခံစားချက်ရှိနေတယ်၊ နှစ်၂၀အတွင်းမှာပေါ့}...နှစ်ပေါင်း ၁၁,၅၀၀ လောက်ကြာပြီးနောက် နောက်တစ်ကြိမ်ထပ်ဖြစ်တော့မယ့်အချိန်ကာလကို ကျွန်တော်တို့ရောက်နေပြီ...ဒီဖြစ်စဉ်ဟာ နှစ်ပေါင်း ၁၁,၅၀၀ လောက်တစ်ကြိမ် သံသရာလည်နေကျဖြစ်ပြီး အခုထပ်ဖြစ်တော့မယ့်အချိန်ရောက်နေပြီ...သူတို့ဒါဖြစ်လာမယ်ဆိုတာကို နားလည်ထားကြတယ်။ ဒါဖြစ်လာမယ်ဆိုတဲ့အသိပညာကို သေချာပေါက်ရရှိထားကြတယ်...ဒါကလည်း သူတို့မသိဘူးဆိုရင် မယုံနိုင်စရာကောင်းတဲ့အရာတစ်ခုပါပဲ။ ဆိုလိုတာက ကမ္ဘာ့အတော်ဆုံးဦးနှောက်တွေက ဒီကိစ္စအတွက် သူတို့အတွက်အလုပ်လုပ်နေကြတယ်"} \cite{4}။

ဤသည်မှာ ကျွန်ုပ်တို့အတွက် အလွန်ကျေးဇူးတင်ရမည့် အစွမ်းထက်သော သက်သေခံချက်တစ်ရပ်ဖြစ်သည်။ အင်တာဗျူးတွင် စာရေးသူသည် ပထမနှင့်ဒုတိယကမ္ဘာစစ်များသည် လူလုပ်စစ်ပွဲများဖြစ်သည်ဟူသောယုံကြည်ချက်နှင့် Anglo-Saxon Mission သည် မျိုးဆက်ပေါင်းများစွာကတည်းက စတင်ခဲ့သည်ဟူသောအချက်ကိုလည်း ဆွေးနွေးထားသည်။ ယခုအခါ ၂၀၁၀ ခုနှစ်ကပြုလုပ်ခဲ့သော အင်တာဗျူးမှစ၍ ၁၅ နှစ်ကျော်လွန်ခဲ့ပြီဖြစ်သည်။ အတွင်းလူဖော်ပြခဲ့သော မြေပြင်လှိုင်းဖြစ်စဉ်အတွက် ၂၀ နှစ်အတွင်းဖြစ်လာမည်ဟူသော ခန့်မှန်းချက်မှာ နောက်ထပ် ၅ နှစ်သာလိုတော့သည်။
\subsection{ဒရူးအစ်ဒ်တို့၏ ကပ်ဘေးဆိုင်ရာ အနောက်တိုင်းဆန်သော လျှို့ဝှက်အသိပညာ}

ကပ်ဘေးများ ပြန်လည်ဖြစ်ပွားခြင်းဆိုင်ရာ အနောက်တိုင်းအသိပညာကို ဖရီးမေဆင်များသာမက အခြားသူများကလည်း ကောင်းစွာ ထိန်းသိမ်းထားကြသည်။ ဒရူးအစ်ဒ်များသည် လွန်ခဲ့သော နှစ်ပေါင်း ၂၄၀၀ ကျော်ကတည်းက တည်ရှိခဲ့သည့် ဆဲလ်တစ်ယဉ်ကျေးမှု၏ မှတ်တမ်းတင်ထားသော ရှေးဟောင်းယဉ်ကျေးမှုတစ်ခုဖြစ်ပြီး \cite{7}၊ ကမ္ဘာမြေပေါ်တွင် ပြန်လည်ဖြစ်ပွားသော ကပ်ဘေးဆိုင်ရာ အသိပညာကို လက်ဆင့်ကမ်းခဲ့ကြသည်။ နောက်ဆုံးသိရှိရသည့် ဒရူးအစ်ဒ်မှာ ဘန်မက်ဘရေဒီဖြစ်သည်ဟု ယုံကြည်ရသည်။ "The Last Druid" ဟူသော ၁၉၉၂ ခုနှစ်က ရုပ်ရှင်တစ်ခုတွင် သူသည် ဒရူးအစ်ဒ်များ၏ အသိပညာအကြောင်း အချက်အလက်များကို မျှဝေခဲ့သည်- \textit{"ကျွန်ုပ်သည် ရိုးရာအရ နောက်ဆုံးအဖွဲ့ဝင်ဖြစ်နိုင်သည့် အစဉ်အလာသည် နောက်ဆုံးကြီးမားသော ကပ်ဘေး သို့မဟုတ် ကမ္ဘာလုံးဆိုင်ရာ ဘေးအန္တရာယ်ဖြစ်ပြီးနောက် ပေါ်ပေါက်လာခြင်းဖြစ်သည်။ ယခုအခါတွင် ကြီးမားပြင်းထန်သော လျှပ်စီးမုန်တိုင်းများ၊ ဥက္ကာခဲများ၏ အမြီးများတွင် ဖမ်းမိခြင်း သို့မဟုတ် ဥက္ကာပျံများ၏ မိုးရွာခြင်းတို့ကြောင့် ကမ္ဘာမြေပေါ်တွင် ကြီးမားသော သက်ရောက်မှုများဖြစ်ပေါ်ကာ ကျွန်ုပ်တို့သိထားသည့် လူ့အဖွဲ့အစည်းသည် အလုံးစုံပျက်စီးခဲ့သည်... အသိပညာအားလုံးသည် အစဉ်အလာ၏ လွှမ်းခြုံမှုအောက်တွင်ရှိသော်လည်း၊ သူတို့သည် နက္ခတ္တဗေဒကို အထူးဂရုပြုခဲ့ကြသည်။ အဘယ်ကြောင့်ဆိုသော် သူတို့သည် အလွန်အရေးပါသော ဘေးဒုက္ခများစွာကို ကြုံတွေ့ခဲ့ရသည်။ နက္ခတ္တဗေဒဆိုင်ရာ အပြည့်အဝသိရှိမှုသည် ဤဘေးဒုက္ခများ ဖြစ်ပွားနိုင်ချေရှိသော အခြေအနေများကို ကြိုတင်ခန့်မှန်းနိုင်ပြီး သူတို့ကိုယ်သူတို့ ကာကွယ်ရန် အချို့သော လုပ်ဆောင်မှုများကို ဆောင်ရွက်နိုင်မည်ဟု ယူဆခဲ့ကြသည်။ အိုင်ယာလန်ရှိ ကြီးမားသော ကျောက်တုံးကျောက်ဆောင်အစုအဝေးများကို ကြည့်ပါက၊ လမ်းသင်္ချိုင်းများဟု ဖော်ပြထားသော အရာများသည် အမှန်တကယ်တွင် အလွန်ရှေးကျသော ဗုံးခိုကျင်းများဖြစ်သည်ကို တွေ့ရမည်။ ထိုအရာများသည် ဒီရေလှိုင်းများ၏ အဆင့်ထက် အလွန်မြင့်မားပြီး ဥက္ကာပျံမိုးရွာခြင်းမှလည်း ကာကွယ်ပေးသည်"} \cite{8,9}။

% ဖရီးမေဆင်အဖွဲ့သည် အမှန်တကယ်တွင် ဒရူးအစ်ဒ်များမှ ဆင်းသက်လာသည်ဟုလည်း ယုံကြည်ရသည် \cite{10}။
\section{လက်ရှိအနောက်တိုင်း ကပ်ဘေးကြီး ပြင်ဆင်မှုများ၏ အထောက်အထား}

အနောက်တိုင်းအာဏာရှင်များသည် ကမ္ဘာလုံးဆိုင်ရာ ဘူမိရုပ်ပိုင်းဆိုင်ရာ ကပ်ဘေးကြီးတစ်ခု မကြာမီဖြစ်ပွားတော့မည်ဟု ယုံကြည်နေသည့်အတွက်၊ ထိုသို့သောဖြစ်ရပ်မှ သူတို့ကိုယ်သူတို့ ကာကွယ်ရန် ကြီးမားသောပြင်ဆင်မှုများ ဖြစ်ပေါ်နေမည်ဟု မျှော်လင့်ရပါသည်။ ထို့ပြင်၊ အများပြည်သူသိရှိသည့် အထောက်အထားများအရ အနောက်နိုင်ငံများစွာတွင် နက်ရှိုင်းသောမြေအောက်စခန်းများ၏ ကျယ်ပြန့်သောကွန်ရက်များ တည်ဆောက်ထားသည်ကို တွေ့ရပါသည်။ ထိုသို့သောအဆောက်အအုံများသည် နျူကလီးယားစစ်အတွင်း နေထိုင်သူများကို ကာကွယ်ပေးနိုင်သည်သာမက၊ သဘာဝဘေးအန္တရာယ်အမျိုးမျိုးမှလည်း အကာအကွယ်ပေးနိုင်ပါသည်။ Project Camelot \cite{4,6} မှ ဗြိတိသျှ အကြီးတန်း Freemason ၏ သက်သေခံချက်အရ၊ ဤဖြစ်နိုင်ခြေများသည် အလားအလာများသာမက၊ ကြိုတင်စီစဉ်ထားသော အစီအစဉ်များဖြစ်သည်ဟု ထင်ရှားပါသည်။ ထို့ပြင်၊ ဤစခန်းများကို တည်ဆောက်ရန်၊ ဝန်ထမ်းများထားရှိရန်နှင့် ထိန်းသိမ်းရန် လိုအပ်မည့် ငွေကြေးပမာဏသည်လည်း အလွန်များပြားပြီး၊ အမေရိကန်အစိုးရမှ နှစ် ၁၈ အတွင်း ပျောက်ဆုံးနေသော ထရီလီယံဒေါ်လာများစွာ (နောက်အပိုင်းတွင် ဖော်ပြထားသည်) \cite{11,12,13} ကဲ့သို့သော ပမာဏများနှင့် ကိုက်ညီပါသည်။ မျိုးသုဉ်းခြင်းအဆင့်ဖြစ်ရပ်အတွက် အခြားပြင်ဆင်မှုဥပမာများတွင် မျိုးစေ့နှင့် အသိပညာသိုလှောင်ရုံများကဲ့သို့သော အမျိုးမျိုးသော မော်ကွန်းစာကြည့်တိုက်စီမံကိန်းများ ပါဝင်ပါသည်။
\subsection{အမေရိကန်မြေအောက်နှင့် ရေအောက်စခန်းများ}

ကျွန်ုပ်တွေ့ရှိသမျှ မြေအောက်စခန်းများဆိုင်ရာ အများပြည်သူသုတေသနအကျယ်ဆုံးမှာ Richard Sauder ဆိုသည့် အမေရိကန်လွတ်လပ်သော သုတေသီတစ်ဦး၏ လုပ်ဆောင်ချက်ဖြစ်သည်။ ၎င်းသည် နက်ရှိုင်းသော မြေအောက်စခန်းများအကြောင်း စာအုပ်များစွာထုတ်ဝေခဲ့သည် \cite{22}။ Sauder ၏လုပ်ငန်းတွင် အစိုးရစာရွက်စာတမ်းများနှင့် အစီအစဉ်များကို မော်ကွန်းတင်ခြင်း၊ သမိုင်းနှင့် လက်ရှိသတင်းများနှင့် နည်းပညာများကို လေ့လာခြင်း၊ အရင်းအမြစ်များမွေးမြူခြင်းနှင့် အတွင်းလူများ၏ပြောဆိုချက်များကို စုစည်းခြင်းတို့ပါဝင်သည်။ Sauder �ုနှစ်၏သုတေသနအရ အမေရိကန်နှင့်၎င်း၏နယ်မြေများတစ်ဝိုက်တွင် မိုင် ၃ မိုင်အထိ နက်ရှိုင်းနိုင်သည့် မြေအောက်နှင့် ရေအောက်စခန်းများ၏ ကွန်ရက်ကြီးတစ်ခုရှိကြောင်း၊ မြေအောက်လေဟာနယ်ပြွန်များဖြင့် အမြန်နှုန်းမြင့် သံလိုက်မြှင့်တင်ရထားများဖြင့် ချိတ်ဆက်ထားနိုင်ကြောင်း (ပုံ \ref{fig:4}) ဖော်ပြထားသည်။ ဤစခန်းများကို အမေရိကန်ပြည်ထောင်စုကုမ္ပဏီကို ပိုင်ဆိုင်သူများမှ စီမံသော \textit{"အဆင့်မြင့်ဘဏ္ဍာရေး၊ အပြည်ပြည်ဆိုင်ရာ၊ အေဂျင်စီချင်း၊ ငွေကြေးဆေးကြောသည့် အခွံကစားနည်း"} ဖြင့် တိတ်တဆိတ်ငွေကြေးထောက်ပံ့ထားသည် \cite{22}။ Catherine Austin Fitts (၎င်း၏လုပ်ဆောင်ချက်ကို နောက်အပိုင်းတွင် ဖော်ပြထားသည်) နှင့် ၎င်း၏လက်တွဲလုပ်ဆောင်သူတစ်ဦးတို့မှ ဤစခန်းများ၏အတိုင်းအတာကို လုပ်ဆောင်သော နောက်ထပ်လုပ်ဆောင်ချက်တစ်ခုတွင် အမေရိကန်မြေအောက်နှင့် ရေအောက်စခန်း ၁၇၀ ခန့်ရှိကြောင်း ခန့်မှန်းချက်ထုတ်ခဲ့သည် \cite{16,20}။

\begin{figure}[b]
\begin{center}
% \fbox{\rule{0pt}{2in} \rule{0.9\linewidth}{0pt}}
   \includegraphics[width=1\linewidth]{penta.jpg}
\end{center}
   \caption{အိမ်ဖြူတော်နှင့် ပင်တဂွန်အဆောက်အဦတို့၏ အောက်တွင် အမှန်တကယ် ဘာတွေရှိနေသနည်း။ ထင်ရှားသည်မှာ မြေအောက်လှိုဏ်ခေါင်းများ၏ နက်ရှိုင်းသော ကွန်ရက်တစ်ခုဖြစ်သည် (ဓာတ်ပုံ: \cite{31})။}
\label{fig:3}
\label{fig:onecol}
\end{figure}
\begin{figure*}[t]
\begin{center}
% \fbox{\rule{0pt}{2in} \rule{.9\linewidth}{0pt}}
\includegraphics[width=0.9\textwidth]{basescrop.png}
\end{center}
\caption{ဆော်ဒါ၏ သုတေသနမှ ဖော်ထုတ်ထားသည့် မြေအောက်နှင့် ရေအောက်အခြေစိုက်စခန်းများ အမှန်တကယ်တည်ရှိသည့် တည်နေရာများကို ဖော်ပြထားသော မြေပုံ။ ထို့ပြင် ကုန်းတွင်းပိုင်းသို့ ဦးတည်သော ရေငုပ်သင်္ဘောဥမင်လှိုဏ်ခေါင်းများလည်း ပါဝင်သည်။ ဆော်ဒါက \textit{"ဤစခန်းများထက် \textbf{များစွာပို၍} အခြေစိုက်စခန်းများ ရှိမည်ဟု သေချာစွာယုံကြည်သည်"} \cite{22}။}
\label{fig:4}
\end{figure*}

ဤသည်မှာ ဆော်ဒါ၏ အရင်းအမြစ်များမှ ဤအခြေစိုက်စခန်းများ၏ အတိုင်းအတာကို ရှင်းပြထားသော သက်သေခံအထောက်အထား အပိုင်းအစများဖြစ်သည်-
\begin{flushleft}
\begin{enumerate}
    \item ကမ်ပ် ဒေးဗစ်၊ မယ်ရီလန်း: \textit{"ကျွန်တော့်ရဲ့ အရင်းအမြစ်က ကမ်ပ် ဒေးဗစ်ရဲ့ မြေအောက်အပိုင်းတွေဟာ အလွန်ကျယ်ပြန့်ပြီး ရှုပ်ထွေးကာ၊ လျှို့ဝှက်တူးမြောင်းများ မိုင်ပေါင်းများစွာရှိပြီး မည်သူတစ်ဦးတစ်ယောက်ကမှ ဒီအဆောက်အအုံရဲ့ ပြည့်စုံတဲ့မြေပုံကို သူ့စိတ်ထဲမှာ ပိုင်ဆိုင်ထားနိုင်မယ်လို့ သံသယရှိစရာပါပဲ"} \cite{22}။
    \item အိမ်ဖြူတော်၊ ဝါရှင်တန်ဒီစီ: \textit{"ကျွန်မရဲ့ ရင်းနှီးတဲ့မိတ်ဆွေတစ်ဦးကို ၁၉၆၀ ပြည့်လွန်နှစ်များက လင်ဒန် ဘီ ဂျွန်ဆန် အစိုးရလက်ထက်မှာ ဒီအဆောက်အအုံအတွင်းကို ခေါ်ဆောင်သွားခဲ့ပါတယ်။ သူမဟာ အိမ်ဖြူတော်ထဲက ဓာတ်လှေကားတစ်စီးထဲဝင်ပြီး တည့်တည့်အောက်ကို ဆင်းသွားခဲ့ပါတယ်။ ဓာတ်လှေကားဟာ အဆင့် ၁၇ အထိ ဆင်းသွားတယ်လို့ သူမယုံကြည်ပါတယ်။ တံခါးဖွင့်လိုက်တဲ့အခါ မြေအောက်မှာ ဝေးဝေးအထိ ပျောက်ကွယ်သွားသလို မြင်ရတဲ့ လမ်းကြောင်းတစ်ခုဆီကို ခေါ်ဆောင်သွားခံရပါတယ်။ အခြားတံခါးများနှင့် လမ်းကြောင်းများက ထိုလမ်းကြောင်းမှ ခွဲထွက်သွားပါတယ်"} \cite{22}။ ပုံ \ref{fig:3} တွင် ဖော်ပြထားပါသည်။
    \item Fort Meade, Maryland - ၁၉၇၀ ခုနှစ်များက "အောက်ထပ်"သို့ မတော်တဆ ရောက်သွားသူတစ်ဦး၏ ပြောပြချက်: \textit{"ကျွန်တော်တံခါးကိုဖွင့်လိုက်တော့ အောက်ကိုဆင်းတဲ့လှေကားထစ်တွေ့ရတယ်။ လှေကားစင်ရဲ့အနားသို့သွားပြီး အောက်ကိုကြည့်လိုက်တယ်။ အောက်ထပ်ဘယ်နှစ်ထပ်ရှိမှန်းမရေတွက်ခဲ့ပေမယ့် ၁၅-၂၀ ထပ်လောက်ရှိမယ်လို့ခံစားရတယ်... ကျွန်တော်တစ်ထပ်ဆင်းပြီး တံခါးတစ်ချပ်တွေ့တယ်... တံခါးကိုဖွင့်ပြီး ခေါင်းထောင်ကြည့်လိုက်တော့ နှစ်ဖက်စလုံးမှာ အဆုံးမရှိသလိုရှည်လျားနေတဲ့ တွန်းလှည်းလမ်းတစ်ခုကိုမြင်ရတယ်။ ဒါဟာမြေပြင်အဆင့်ရှိ အဆောက်အဦနဲ့ကားပါကင်ဧရိယာထက်ကျော်လွန်တဲ့နေရာဖြစ်နေတာသေချာတယ်... တံခါးတွေကတော့ ဆန့်ကျင်ဘက်နံရံတွေမှာ ၃၀-၄၀ ပေအကွာလောက်နဲ့ စီထားတယ်... နောက်ထပ်အထပ်နည်းနည်းဆင်းကြည့်ဖို့ဆုံးဖြတ်ပြီး နောက်တစ်ထပ်ဆင်းတယ်... ပထမအထပ်နှစ်ထပ်လိုပဲတွေ့ရတယ်... နောက်ထပ်တစ်ထပ်ဆင်းပြီး ကြည့်လိုက်တော့ ပထမနှစ်ထပ်နဲ့အတူတူပဲ"} \cite{22}။
\end{enumerate}
\end{flushleft}

\begin{figure}[t]
\begin{center}
% \fbox{\rule{0pt}{2in} \rule{0.9\linewidth}{0pt}}
   \includegraphics[width=1\linewidth]{undersea.jpg}
\end{center}
   \caption{ရေအောက်စခန်းတစ်ခု၏ သရုပ်ဖော်ပုံ၊ Walter Koerschner မှရေးဆွဲသည်။ ၁၉၆၀ ခုနှစ်များတွင် US Navy ၏ China Lake၊ California လက်နက်စင်တာရှိ Rock-Site ရေအောက်စခန်းအဖွဲ့အတွက် သရုပ်ဖော်ပန်းချီဆရာတစ်ဦးဖြစ်သည်။ Sauder ၏ အရင်းအမြစ်များအရ China Lake တွင် တစ်မိုင်အနက်ရှိ မြေအောက်စခန်းတစ်ခုရှိကြောင်း ဖော်ပြထားသည် \cite{22,23}။}
\label{fig:5}
\label{fig:onecol}
\end{figure}

Sauder သည် တစ်နာရီလျှင် အမြန်နှုန်း ၂၀၀၀ မိုင်အထိ ရောက်ရှိနိုင်သော မြေအောက်သံလိုက်မောင်းနှင်သည့်ရထားများ၊ သမုဒ္ဒရာကြမ်းပြင်အောက်တွင် တည်ဆောက်ထားသော အခြေစိုက်စခန်းများ (Figure \ref{fig:5})၊ ကုန်းတွင်းပိုင်းသို့ ဦးတည်သော ရေငုပ်သင်္ဘောဥမင်လိုဏ်ခေါင်းများအကြောင်း သက်သေခံချက်များကိုလည်း ရရှိခဲ့သည်။ မက္ကဆီကိုပင်လယ်ကွေ့ရှိ ရေအောက်အခြေစိုက်စခန်းတစ်ခုနှင့်ပတ်သက်သည့် သက်သေခံချက်တစ်ခုအကြောင်း Sauder က \textit{"Underwater and Underground Bases စာအုပ်ထုတ်ဝေပြီးနောက် တစ်နှစ်ခွဲအကြာတွင် ထူးခြားသောရေအောက်စီမံကိန်းတစ်ခုနှင့်ပတ်သက်သော အသိပညာရှိသူတစ်ဦးက ကျွန်ုပ်ထံဆက်သွယ်ခဲ့သည်... သူက ထိုစီမံကိန်းသည် မက္ကဆီကိုပင်လယ်ကွေ့၏ ပင်လယ်ကြမ်းပြင်အောက်တွင်ရှိပြီး Parsons ကုမ္ပဏီက တာဝန်ယူဆောင်ရွက်နေကြောင်း ဖော်ပြခဲ့သည်။ သူက ပင်လယ်ကြမ်းပြင်အောက် ၂၈၀၀ ပေအနက်တွင် လုပ်ငန်းလည်ပတ်ရန်အတွက် Parsons က အထူးစက်ပစ္စည်းအချို့ကို ဝယ်ယူခဲ့ကြောင်း ဆက်လက်ပြောကြားခဲ့သည်... ထိုစက်ပစ္စည်းများသည် ထင်ရှားသောအထူးသဖြင့် ထိုနေရာများတွင် လူသားများ အသက်ရှင်နေထိုင်ကြောင်း ရှင်းရှင်းလင်းလင်း ဖော်ညွှန်းနေသည်"} \cite{22} ဟု ဆိုသည်။
\begin{figure}[t]
\begin{center}
% \fbox{\rule{0pt}{2in} \rule{0.9\linewidth}{0pt}}
   \includegraphics[width=1\linewidth]{sub.jpg}
\end{center}
   \caption{ဝေါ်လ်တာ ကိုးရှ်နာ မှ ရေအောက် ရဟတ်ယာဉ် ဥမင်လိုဏ်ခေါင်း ၏ သရုပ်ဖော်ပုံ \cite{22,23}။}
\label{fig:6}
\label{fig:onecol}
\end{figure}
\begin{figure}[t]
\begin{center}
% \fbox{\rule{0pt}{2in} \rule{0.9\linewidth}{0pt}}
   \includegraphics[width=1\linewidth]{iran.jpeg}
\end{center}
   \caption{အီရန်ရဲ့ တရားဝင် ဗီဒီယိုမှ ဖြတ်ညှပ်တစ်ခု - သူတို့ရဲ့ မြေအောက် "ဒုံးကျည်မြို့" ကို ပြသထားသည် \cite{39,40}.}
\label{fig:12}
\label{fig:onecol}
\end{figure}
အကယ်၍ ကျွန်ုပ်တို့ခြေရင်းအောက်တွင် မိုင်ပေါင်းများစွာ အနက်အထိ တူးဖော်ထားသည့် မြေအောက်နှင့် ပင်လယ်အောက်စခန်း ၁၇၀ ကျော်ပါဝင်သော ကြီးမားသည့် လျှို့ဝှက်ကွန်ရက်တစ်ခု တကယ်ရှိပါက၊ ဟိုက်ပါဆောနစ် လေဟာနယ်မဂ္ဂနက် ရထားများဖြင့် ချိတ်ဆက်ထားပါက၊ ကျွန်ုပ်တို့၏ အလုပ်အကိုင်များ၏ အသီးအပွင့်များဖြင့် ငွေကြေးထောက်ပံ့ထားပါက၊ ယနေ့ လူသားအများစုသည် အဆုံးစွန်သော ပျော်ရွှင်ဖွယ် မသိမှုအခြေအနေတွင် ရှိနေမည်ဖြစ်သည်။ သူတို့ခြေရင်းအောက်တွင် ရှိသည်ကိုသာမက အနာဂတ်အနီးတွင် ရှိလာမည့်အရာကိုပါ မသိဘဲ၊ ၎င်းတို့၏နိုင်ငံရေးသမားများ၏ အချည်းနှီးသော ညှိနှိုင်းထားသည့် ထုတ်ပြန်ချက်များကို ယုံကြည်နေကြမည်ဖြစ်သည်။

အပိုမှတ်စု - ကြီးမားသော မြေအောက်ဥမင်လိုဏ်ခေါင်းကွန်ရက်များ တည်ရှိမှုကို အရှေ့အလယ်ပိုင်းရှိ လက်ရှိတိုက်ပွဲများ (ဂါဇာကမ်းမြှောင်အောက်ရှိ ဟာမတ်စ်ဥမင်များ \cite{38}၊ နှင့် အီရန်၏ မြေအောက် "ဒုံးကျည်မြို့" (ပုံ \ref{fig:12}) \cite{39,40}) တွင် သံသယဖြစ်စရာမရှိဘဲ ဖော်ထုတ်ပြသခဲ့သည်။ ထိုအရာများသည် ထိုကဲ့သို့သော အဆောက်အဦများ တည်ဆောက်နိုင်မှုနှင့် တကယ်တည်ရှိမှုတို့ကို သံသယဖြစ်စရာမရှိစေသင့်ပါ။ ၎င်းတို့သည် အခြားသော အရင်းအနှီးပိုမိုကောင်းမွန်သည့် နိုင်ငံများက တူညီသောကာလအတွင်း တည်ဆောက်ထားနိုင်သည့် အဆောက်အဦများကိုလည်း စဉ်းစားစေသင့်သည်။
\subsection{အပိုဘန်ကာနှင့် ကပ်ဘေးအတွက် အဆင့်ဆင့်ပြင်ဆင်မှု အထောက်အထားများ}

\begin{figure}[t]
\begin{center}
% \fbox{\rule{0pt}{2in} \rule{0.9\linewidth}{0pt}}
\includegraphics[width=1\linewidth]{tyrol.jpg}
\end{center}
   \caption{တောင်တိုင်ရွိုလ်တွင် တည်ဆောက်ထားသော ဘန်ကာများ။ ဥရောပ အဲလ်ပ် တောင်တန်းများကို ဖြတ်သန်းတည်ရှိသည့် ဆွစ်ဇာလန်နိုင်ငံသည် ၎င်း၏ တောင်ဘန်ကာများကို ကျွမ်းကျင်စွာ ဖုံးကွယ်ထားသည့် အတွက် ကျော်ကြားသည် \cite{32}။}
\label{fig:7}
\label{fig:onecol}
\end{figure}

\begin{figure}[t]
\begin{center}
% \fbox{\rule{0pt}{2in} \rule{0.9\linewidth}{0pt}}
   \includegraphics[width=1\linewidth]{svalbard.jpg}
\end{center}
   \caption{နော်ဝေနိုင်ငံရှိ Svalbard Global Seed Vault တွင် မျိုးစေ့တစ်သန်းကျော်သိုလှောင်ထားသည် \cite{24}။ မည်သည့်ကပ်ဘေးကြီးက ၎င်း၏အသုံးပြုမှုကို လိုအပ်လာမည်ကို စဉ်းစားစရာဖြစ်သည်။}
\label{fig:8}
\label{fig:onecol}
\end{figure}

ကမ္ဘာတစ်ဝှမ်းတွင် အမေရိကန် ဘုရင်များ၏ မြေအောက်စခန်းများအပြင် ကပ်ဘေးကြီးအတွက် အခြားပြင်ဆင်မှုများစွာရှိသည်။ နော်ဝေ၊ ဆွစ်ဇာလန်၊ ဆွီဒင်နှင့် ဖင်လန်နိုင်ငံတို့သည် ကောင်းသော ဥပမာများဖြစ်သည်။

\begin{flushleft}
\begin{enumerate}
    \item Project Camelot မှ နော်ဝေနိုင်ငံသား နိုင်ငံရေးသမားတစ်ဦး၏ သက်သေခံချက်ကို မျှဝေခဲ့ပြီး \cite{25,26}၊ ၎င်းတို့သည် သူ၏အထောက်အထားကို အတည်ပြုခဲ့သော်လည်း လျှို့ဝှက်ထားခဲ့သည်။ သူကပြောကြားချက်အရ နော်ဝေတွင် မြေအောက်အခြေစိုက်စခန်းကြီး ၁၈ ခုရှိပြီး၊ နော်ဝေ (အစ္စရေးနှင့် "အခြားနိုင်ငံများစွာ" နှင့်အတူ) သည် ဤအခြေစိုက်စခန်းများကို သဘာဝဘေးအန္တရာယ်တစ်မျိုးမျိုးအတွက် ပြင်ဆင်ရန် တည်ဆောက်နေသည်ဟု ဆိုသည်။ Richard Sauder သည်လည်း နော်ဝေရှိ တောင်တစ်ခုအတွင်း၌ တည်ဆောက်ထားသော ကြီးမားသည့် မြေအောက်အခြေစိုက်စခန်းတစ်ခုအတွင်း ရောက်ရှိခဲ့ဖူးသူတစ်ဦး၏ သက်သေခံချက်ကို ရရှိခဲ့သည် \cite{22}။
    \item ဆွစ်ဇာလန်နိုင်ငံတွင် အဲလ်ပ်စ်တောင်တန်းများ၏ အမြင့်ပိုင်းတွင် တည်ဆောက်ထားသော နျူကလီးယား ခိုလှုံရာများ အများအပြားရှိသည်မှာ လူသိများသည် (ပုံ \ref{fig:7})။ ဤခိုလှုံရာများ၏ အရေအတွက်မှာ ၃၇၀,၀၀၀ ကျော်အထိ ရှိပြီး နိုင်ငံသားတိုင်းအတွက် လုံလောက်သည့် ပမာဏဖြစ်သည် \cite{27}။
    \item ဆွီဒင်နှင့် ဖင်လန်နိုင်ငံတို့တွင် မြို့ကြီးတိုင်းရှိ နေထိုင်သူများအတွက် ခိုလှုံနိုင်သည့် ခိုလှုံရာများ လုံလောက်စွာ ရှိသည် \cite{27}။
\end{enumerate}
\end{flushleft}

ဆီလီကွန်တောင်ကြားရှိ စီးပွားရေးလုပ်ငန်းရှင်ကြီးများသည်လည်း ဤအကြောင်းကို သိရှိထားကြသည်ဟု ဆိုနိုင်ပါသည်။ \textit{"LinkedIn ၏ ပူးတွဲတည်ထောင်သူနှင့် ထင်ရှားသော ရင်းနှီးမြှုပ်နှံသူ Reid Hoffman က The New Yorker သို့ ယခုနှစ်အစောပိုင်းက ပြောကြားချက်အရ ဆီလီကွန်တောင်ကြား ဘီလီယံနာများ၏ ၅၀\% ကျော်သည် မြေအောက်ခိုစရာနေရာကဲ့သို့သော "ကမ္ဘာပျက်ရံပြီ" အာမခံမျိုး ဝယ်ယူထားကြသည်ဟု ခန့်မှန်းခဲ့သည်... Forbes ဆောင်းပါးရှင် Jim Dobson ၏ အဆိုအရ ဘီလီယံနာများစွာတွင် "တဒင်္ဂအတွင်း ထွက်ခွာနိုင်ရန် အဆင်သင့်ဖြစ်နေသော" သီးသန့်လေယာဉ်များ ရှိကြသည်။ ၎င်းတို့တွင် မော်တော်ဆိုင်ကယ်များ၊ လက်နက်များနှင့် လျှပ်စစ်ဓာတ်အားပေးစက်များလည်း ပိုင်ဆိုင်ကြသည်"} \cite{28}။

ထို့အပြင် ကမ္ဘာ့အသိပညာဘဏ်တိုက် (Global Knowledge Vault) နှင့် Svalbard Global Seed Vault \cite{30} အပါအဝင် အကြီးစား မွေးစားထိန်းသိမ်းရေးစီမံကိန်းများစွာရှိပြီး လူသားမျိုးနွယ်၏ အရေးကြီးသော အရင်းအမြစ်များကို မျိုးသုဉ်းခြင်းအဆင့် ဘေးအန္တရာယ်များမှ ကာကွယ်ရန် ပြင်ဆင်နေဟန် တူပါသည်။
\begin{figure*}[t]
\begin{center}
% \fbox{\rule{0pt}{2in} \rule{.9\linewidth}{0pt}}
\includegraphics[width=0.9\textwidth]{govcrop2.png}
\end{center}
   \caption{၁၉၉၈ ခုနှစ်မှ ၂၀၂၃ ခုနှစ်အထိ အမေရိကန်အစိုးရ၏ အမြတ်ငွေ၊ သုံးစွဲမှုနှင့် လျှို့ဝှက်မြေအောက်စခန်းသုံးစွဲမှုများ \cite{19}.}
   \label{fig:9}
\end{figure*}
\section{ကြီးမားသောမြေအောက်စခန်းများအတွက် ဒီမိုကရေစီငွေကြေးထောက်ပံ့မှုစနစ်များ}

ဒါဆို ကမ္ဘာလုံးဆိုင်ရာမြေအောက်နှင့်ရေအောက်စခန်း ၁၇၀ ကျော်ပါဝင်သော ကွန်ရက်ကြီးများကို ကြွေးမြီကျွန်များမသိစေဘဲ ဘယ်လိုငွေကြေးထောက်ပံ့ကြသလဲ။ ဤစီမံကိန်းများအတွက်သုံးစွဲသည့်ငွေပမာဏနှင့် အရင်းအမြစ်များကို နမူနာပြရန် စာရွက်စာတမ်းတစ်ခုရှိသည်။ ၂၀၁၇ ခုနှစ်တွင် အမေရိကန်ရင်းနှီးမြှုပ်နှံမှုဘဏ်လုပ်ငန်းလုပ်ကိုင်သူနှင့် ဘုရှ်အုပ်ချုပ်ရေးလက်ထက်က အစိုးရအရာရှိဟောင်း Catherine Austin Fitts နှင့် Michigan State University မှ စီးပွားရေးပညာရှင် Mark Skidmore တို့သည် ၁၉၉၈-၂၀၁၅ ဘဏ္ဍာရေးနှစ်များအတွင်း အမေရိကန်အစိုးရ၌ ခွင့်ပြုချက်မရှိသော ဒေါ်လာ ၂၁ ထရီလီယံသုံးစွဲမှုကို တွေ့ရှိခဲ့သည် \cite{11,12,13}။

သူတို့၏ အစီရင်ခံစာအရ \textit{"၂၀၁၆ ခုနှစ် အောက်တိုဘာ ၇ ရက်တွင် Reuters သတင်းဌာနက Scot Paltrow (2016) ၏ ဆောင်းပါးတစ်ပုဒ်ထုတ်ဝေခဲ့ပြီး ၎င်းတွင် ၂၀၁၅ ဘဏ္ဍာရေးနှစ်အတွင်း စစ်တပ်သည် သူ၏စာရင်းဇယားများညီမျှသည်ဟူသော အယောင်ပြမှုဖန်တီးရန် ဒေါ်လာ ၆.၅ ထရီလီယံအား အတည်ပြုချက်မရှိသော စာရင်းကိုင်ညှိချက်များပြုလုပ်ခဲ့ကြောင်း ဖော်ပြထားသည်။ ထိုနှစ်အတွက် စစ်တပ်၏အထွေထွေရန်ပုံငွေဘဏ္ဍာမှာ ဒေါ်လာ ၁၂၂ ဘီလီယံသာရှိခဲ့သည်ကို ထောက်လျှင် ဤဖော်ပြချက်မှာ အံ့သြဖွယ်ကောင်းလှပေသည်... DOD ၏စာရင်းကိုင်ပြဿနာများအတွက် မီဒီယာအကြီးစားသတင်းခေါင်းစဉ်များဖြစ်ခဲ့သည်မှာ နှစ်များစွာကတည်းကဖြစ်ပြီး ၂၀၀၁ ခုနှစ် စက်တင်ဘာ ၁၀ ရက်တွင် ကာကွယ်ရေးဝန်ကြီး Donald Rumsfeld က လွှတ်တော်နားထောင်မှုတစ်ခုတွင် (C-SPAN, 2014) DOD သည် ဒေါ်လာ ၂.၃ ထရီလီယံတန်ဖိုးရှိ ငွေလွှဲအမှတ်တရများကို ခြေရာခံမိခြင်းမရှိကြောင်း ထုတ်ဖော်ပြောဆိုခဲ့သည်... ဤအသိအမှတ်ပြုချက်သည် ထိုနေ့တွင်သတင်းခေါင်းစဉ်များဖြစ်ခဲ့သော်လည်း တစ်ရက်အကြာတွင် ၉/၁၁ အဖြစ်အပျက်၏ဝမ်းနည်းဖွယ်အဖြစ်အပျက်က ကမ္ဘာ့အာရုံစိုက်မှုကိုဆွဲယူသွားခဲ့သဖြင့် မေ့လျော့ခံခဲ့ရသည်... Professor Mark Skidmore သည် စစ်တပ်၏ဒေါ်လာ ၆.၅ ထရီလီယံတန်ဖိုးရှိ အတည်ပြု၍မရသောငွေလွှဲမှုများအကြောင်းသိရှိသောအခါ သူသည် Ms. Fitts ထံဆက်သွယ်ခဲ့ပြီး ၂၀၁၇ ခုနှစ်နွေဦးတွင် HUD နှင့် DOD အတွင်းရှိ ပုံမှန်မဟုတ်သောကြီးမားသည့်အတည်ပြု၍မရသောငွေလွှဲမှုများကိုဖော်ပြသည့် အလားတူအစိုးရအစီရင်ခံစာများကိုရှာဖွေရန် သဘောတူညီခဲ့ကြသည်။ နောက်လခြောက်လအတွင်း Skidmore၊ Fitts နှင့် ဘွဲ့လွန်ကျောင်းသားအနည်းငယ်ပါဝင်သည့်အဖွဲ့သည် ၁၉၉၈-၂၀၁၆ ကာလအတွင်း စာရွက်စာတမ်းမထောက်ပြနိုင်သော စုစုပေါင်းဒေါ်လာ ၂၁ ထရီလီယံတန်ဖိုးရှိ ငွေလွှဲမှုများကိုဖော်ပြသည့် တရားဝင်အစိုးရစာရွက်စာတမ်းများကို စုဆောင်းခဲ့ကြသည်"} \cite{12}။
1998 မှ 2015 အထိ 18 နှစ်တာအတွင်း အမေရိကန်အစိုးရ၏ အများသိထုတ်ပြန်ထားသော ဘဏ္ဍာရေးလက်ခံမှုများမှာ 40.8 ထရီလီယံသာ ရှိခဲ့ပြီး \cite{15}၊ ဆိုလိုသည်မှာ အမေရိကန်အစိုးရ၏ အများသိအသုံးစရိတ်အပြင် မြေအောက်စခန်းများအတွက် လျှို့ဝှက်အသုံးစရိတ်အဖြစ် ဘဏ္ဍာရေးလက်ခံမှု၏ တစ်ဝက်ကျော်ကို သုံးစွဲခဲ့ခြင်းဖြစ်သည်။ ထို့အပြင် ဤလျှို့ဝှက်အသုံးစရိတ်မှာ နှစ်ရှည်လများ ဘဏ္ဍာရေးလိုငွေပြမှုအပေါ်တွင် ထပ်မံဖြစ်ပေါ်ခဲ့ပြီး၊ 1998 မတိုင်မီကတည်းက ရှိနှင့်ပြီးဖြစ်ကာ ယနေ့အထိ ဆက်လက်ရှိနေသည်ဟု ယူဆရသဖြင့် ဤစခန်းများအတွက် သုံးစွဲခဲ့သော စုစုပေါင်းငွေပမာဏမှာ 21 ထရီလီယံဒေါ်လာထက် များစွာပိုမိုမြင့်မားသည်ဟု ဆိုနိုင်သည်။ 2016-2023 ကာလအတွက် လျှို့ဝှက်အသုံးစရိတ်၏ တူညီသောအချိုးကို အသုံးပြုပါက 1998 မှစတင်၍ စုစုပေါင်း 36.6 ထရီလီယံဒေါ်လာ သုံးစွဲခဲ့သည်ဟု တွက်ချက်နိုင်သည်။

2021 ခုနှစ်တွင် Mark Skidmore မှ Bloomberg ၏ ကြေညာချက်တစ်ရပ်နှင့်ပတ်သက်၍ ဤသုတေသနကို အပ်ဒိတ်လုပ်ခဲ့ပြီး၊ 2017-19 ဘဏ္ဍာရေးနှစ်များအတွင်း Pentagon မှ 94.7 ထရီလီယံဒေါ်လာ အံ့ဖွယ်စာရင်းကိုက်ညှိမှုများ ပြုလုပ်ခဲ့ကြောင်း ဖော်ပြခဲ့သည် \cite{17,18}။ 1913 ခုနှစ်တွင် Federal Reserve စတင်တည်ထောင်ခဲ့သည့်အချိန်မှစ၍ ရာစုနှစ်တစ်ခုကျော် အမေရိကန်ဒေါ်လာများအား ဗဟိုဘဏ်စနစ်မှတစ်ဆင့် အတုပြုလုပ်မှုကို ထည့်သွင်းစဉ်းစားပါက \cite{37}၊ အမေရိကန်ဒေါ်လာနှင့်အစိုးရ၏ အများသိစာရင်းများမှာ လုံးဝအဓိပ္ပာယ်မဲ့သော လှည့်ဖြားမှုများသာဖြစ်ပြီး၊ အမေရိကန်ငွေကြေးနှင့်အစိုးရသည် ၎င်းတို့၏ ဘုရင်များပိုင်ဆိုင်သော အရင်းအမြစ်ခွဲဝေမှုစနစ်များသာဖြစ်ကာ၊ ၎င်းတို့လိုသလောက် တိတ်တဆိတ် ထုတ်ယူသုံးစွဲနိုင်သည်ဟု ရှင်းလင်းစွာမြင်နိုင်သည်။
\section{ဂျိုးဗ်၏ သားစဉ်မြေးဆက်: အနောက်တိုင်း ဘုရင်များ၏ လျှို့ဝှက်အထောက်အထားများ}

ဒါဆို တကယ်တမ်း ဘယ်သူတွေက လွှမ်းမိုးထားကြတာလဲ? ကျွန်ုပ်တို့ အတိအကျ မသိနိုင်ပါ။ အကြောင်းမှာ အနောက်တိုင်း မြို့တော်ရှိ ဘုရင်များသည် သူတို့ကိုယ်သူတို့ အရိပ်ထဲတွင် ဝှက်ထားကြသောကြောင့်ဖြစ်သည်။ အများသိသူများမှ ကမ္ဘာမဟုတ်သော သတ္တဝါများအထိ အမျိုးမျိုးသော သီအိုရီများ ရှိသော်လည်း၊ ကျွန်ုပ်ရရှိသော အကောင်းဆုံးအဖြေမှာ "Amallulla" ဟူသော ကလောင်အမည်ဖြင့် နာမည်မဖော်လိုသော ဘလော့ဂါတစ်ဦး၏ လုပ်ဆောင်ချက်များတွင် တွေ့ရှိရသည်။ သူ၏လုပ်ဆောင်ချက်များတွင် ရှေးဟောင်းနှင့် ခေတ်သစ်သမိုင်း၊ လျှို့ဝှက်သင်္ကေတများ၊ နှင့် အနောက်တိုင်းနိုင်ငံရေးဆိုင်ရာ အကြောင်းအရာများကို ဖော်ပြထားသော စာရေးဆရာ ၂၀ ကျော်နှင့် "အစားထိုး၍မရသော" စာရွက်စာတမ်း ၅၀ ကျော်တို့၏ ကျယ်ပြန့်သော ပေါင်းစပ်မှုများ ပါဝင်သည် \cite{33,34}။ ကျွန်ုပ်သည် သူ၏လုပ်ဆောင်ချက်ကို မလွဲမသွေ ဖြစ်လာမည့် ဘူမိရူပ ကပ်ဘေးကြီးနှင့်ပတ်သက်၍ "ဗျာဒိတ်ဆိုင်ရာ" ဟုသာ ဖော်ပြနိုင်ပါသည် - ၎င်းသည် ကျွန်ုပ်၏လုပ်ဆောင်ချက်ထက် \textit{သိသိသာသာ} ပိုမိုပြည့်စုံပါသည်။

Amallulla သည် အနောက်တိုင်း နိုင်ငံရေးအုပ်စုသုံးစုကို ဖော်ထုတ်ခဲ့ပြီး၊ ထိုအုပ်စုသုံးစုကို စုပေါင်း၍ "ဂျိုးဗ်၏ သားစဉ်မြေးဆက်" ဟု ခေါ်ဆိုခဲ့သည်။ ထိုအုပ်စုများသည် "နောက်ဆုံးသော ကာလ" - ကမ္ဘာမြေ၏ ထပ်တလဲလဲဖြစ်ပေါ်နေသော ကပ်ဘေးကြီးများ - အကြောင်းကို သိရှိထားကြသည်။ သူက ယနေ့ခေတ် အနောက်တိုင်းနိုင်ငံများကို ထိန်းချုပ်ထားသော်လည်း၊ သူတို့၏မူလအစ၊ သမိုင်းဝင်အထောက်အထားများ၊ ဖြစ်နိုင်ခြေရှိသော ယခင် အချင်းချင်း သဘောထားကွဲလွဲမှုများ၊ နှင့် သူတို့၏တန်ဖိုးစနစ်များနှင့် လုပ်ဆောင်ချက်များတွင် မြင်တွေ့ရသော ကွာခြားမှုများအပေါ် အခြေခံ၍ အုပ်စုသုံးစုအဖြစ် ခွဲခြားထားခဲ့သည်။
အုပ်စုသုံးစုကို အောက်ပါအတိုင်း အကြမ်းဖျဉ်း အမျိုးအစားခွဲနိုင်ပါသည်။

\begin{flushleft}
\begin{enumerate}
    \item \textbf{ဘဏ်လုပ်ငန်းရှင်များ}: ရှေးဟောင်းရောမအထက်တန်းစားများ၊ သူတို့သည် အမေရိကရှိ Knights Templar နှင့် Northern Jurisdiction of Freemasons အဖြစ်ပြောင်းလဲသွားသည်။
    \item \textbf{စဉ်းစားတွေးခေါ်သူများ}: Rosicrucians နှင့် တောင်ပိုင်းအမေရိကန် Freemasons များ။
    \item \textbf{Jesuits နှင့် Black Pope}: ရောမကတ်သလစ်ဘုရားကျောင်းအတွင်းရှိ Jove ၏သားစဉ်မြေးဆက်များ၏အုပ်စု။
\end{enumerate}
\end{flushleft}
ယနေ့အချိန်တွင် ဤအုပ်စုသုံးစုသည် ဥရောပအိုင်လူမိနာတီ၊ ဖရီးမေဆင်များနှင့် CIA တို့ပါဝင်နေသည်။ Amallulla ဖော်ပြခဲ့သည့်အတိုင်း \textit{"ယခုအချိန်၊ နောက်ဆုံးကာလတွင် Jove ၏သားစဉ်မြေးဆက်များသည် အမေရိကန်ပြည်ထောင်စု၏ လက်ရှိသမ္မတအထိပင် သိရန်မလိုအပ်သော လုံခြုံရေးအဆင့်များနောက်တွင် ကောင်းစွာဖုံးကွယ်ထားသည်။ တစ်နည်းဆိုရသော် ၎င်းတို့သည် ပြည်သူလူထု၏ စစ်ဆေးမှုမှ ကိုယ်ကိုယ်ကို ဖုံးကွယ်သည့်အနုပညာကို အကောင်းဆုံးကျွမ်းကျင်စွာ တတ်မြောက်ထားကြသည်။ \textbf{Jove ၏သားစဉ်မြေးဆက်များသည် အမေရိကန်ပြည်ထောင်စု၏ စစ်တပ်နှင့် အစိုးရကိုသာမက fiat ငွေကြေး၊ အဓိကကော်ပိုရေးရှင်းများနှင့် ၎င်းတို့တီထွင်ခဲ့သော သမ္မတနိုင်ငံစနစ် (နိုင်ငံရေးသမားများ အလွယ်တကူဖျက်ဆီးခံရမည်ကို သိထားပြီး ထို့ကြောင့် ထိန်းချုပ်နိုင်မည်) အားဖြင့် အနောက်ကမ္ဘာတစ်ခုလုံးကို ထိန်းချုပ်ထားသည်}"} \cite{33,34}။

\begin{figure}[t]
\begin{center}
% \fbox{\rule{0pt}{2in} \rule{0.9\linewidth}{0pt}}
   \includegraphics[width=1\linewidth]{illuminati.jpg}
\end{center}
   \caption{ဂျူးဗီတာဘုရား၏ သားစဉ်မြေးဆက်များဆိုသည်မှာ မည်သူများနည်း။ (ပုံ: \cite{35})}
\label{fig:10}
\label{fig:onecol}
\end{figure}

\begin{figure}[t]
\begin{center}
% \fbox{\rule{0pt}{2in} \rule{0.9\linewidth}{0pt}}
   \includegraphics[width=1\linewidth]{pike.jpg}
\end{center}
   \caption{ထင်ရှားကျော်ကြားသော Pike Peak batholith အား အနီရောင်ဖြင့် မှတ်သားထားပြီး အမေရိကန်ပြည်ထောင်စု၏ အနောက်ပိုင်းမြေမျက်နှာသွင်ပြင်နှင့်အတူ \cite{36}။ အမေရိကန်ပြည်ထောင်စုကို ဤနေရာကို ထိန်းချုပ်ရန် ရည်ရွယ်ချက်ဖြင့် တကယ်ပင် ဖွဲ့စည်းခဲ့သလား။}
\label{fig:11}
\label{fig:onecol}
\end{figure}

အမာလူလာ၏ အဆိုအရ ဤလူများသည် ဘာသာရေးကို မထီမဲ့မြင်ပြုကြပြီး ကမ္ဘာ့အဓိကဘာသာကြီးများ၏ ပြဋ္ဌာန်းကျမ်းစာများကို သူတို့အကျိုးအတွက် အသုံးချကြသည်။ ထို့အပြင် သူတို့သည် သင်္ကေတပညာကို အလေးအနက်အသုံးပြုကြသည်။ ရန်သူများနှင့်ပတ်သက်လာလျှင် သူတို့သည် မည်သည့်ကရုဏာမျှမရှိပေ - \textit{"\textbf{နှစ်ပေါင်း ၂,၆၀၀ ကျော်ကာလအတွင်း သူတို့သည် နောက်ဆုံးကာလနှင့်သက်ဆိုင်သော အထူးအသိပညာရှိသူများကို စနစ်တကျ ဖယ်ရှားခဲ့ကြသည်။ ဤတွင် ကျွန်ုပ်ဆိုလိုသည်မှာ ဒရူးအစ်များ၊ ဂျူးကဗျာလီများ၊ ရှေးအီဂျစ်လူမျိုးများ၊ အာရပ်များနှင့် အိန္ဒိယဆန်းကြယ်သူများသာမက တောင်အမေရိကရှိ ရှည်လျားသောဦးခေါင်းခွံများနှင့် ဗဟိုအမေရိကရှိ မာယာဘုန်းကြီးများလည်း ပါဝင်သည်။ နောက်ဆုံးကာလ၏မြေအဖြစ် ဤဒေသကို ထိန်းသိမ်းရန် မြောက်အမေရိကတွင် တစ်ချိန်က စည်ပင်ဖွံ့ဖြိုးခဲ့သော လူဦးရေတစ်ရပ်လုံးကို သူတို့ဖျက်ဆီးခဲ့ကြောင်း အထောက်အထားများမှာ အလွန်ပင်များပြားလှသည်။ အမေရိကန် "အိန္ဒိယနွယ်ဖွား" များ၏ လူမျိုးတုံးသတ်ဖြတ်မှုမှာ အကြွင်းအကျန်များကို ရှင်းလင်းခြင်းသာဖြစ်သည်}"} \cite{33,34}။
အမာလူလည်း "ပစ်ခ်တောင်တန်း" ကို ထိန်းချုပ်ရန် ရည်ရွယ်ချက်ဖြင့် "အမေရိကန်ပြည်ထောင်စု" စီမံကိန်းတစ်ခုလုံးကို ဆောင်ရွက်ခဲ့သည်ဟု ယုံကြည်ခဲ့သည်။ ဤသည်မှာ ရော့ကီးတောင်တန်းများရှိ ကျောက်စိမ်းတောင်တန်းတစ်ခုဖြစ်ပြီး ဘူမိရုပ်ပိုင်းဆိုင်ရာ ဘေးအန္တရာယ်များမှ ကောင်းမွန်စွာ ကာကွယ်ပေးနိုင်သည် (ပုံ \ref{fig:11})။ အမာလူလ၏ အဆိုအရ၊ \textit{"ကျွန်ုပ်တို့ စဉ်းစားသည့် ပြည်တွင်းစစ်မတိုင်မီ၊ အတွင်း နှင့် ပြီးနောက်၊ ဘဏ်လုပ်ငန်းရှင်များနှင့် အတွေးအခေါ်ရှင်များသည် အမေရိကန်ပြည်ထောင်စုကို ထိန်းချုပ်ရန် မဟုတ်ဘဲ၊ ကမ္ဘာတစ်ဝှမ်းရှိ အထူးခြားဆုံး ကျောက်စိမ်းတောင်တန်းများထဲမှ တစ်ခုဖြစ်သည့် ပစ်ခ်တောင်တန်းကို ထိန်းချုပ်ရန် ရုန်းကန်ခဲ့ကြသည်... ကမ္ဘာပေါ်တွင် အခြားမည်သည့်နေရာတွင်မှ ဤမြင့်မားသောအမြင့်နှင့် သမုဒ္ဒရာကမ်းခြေမှ ဤမျှဝေးကွာသော ကျောက်စိမ်းတောင်တန်း မရှိပါ။ ၎င်းသည် ကမ္ဘာ့အပေါ်ယံလွှာ ရွေ့လျားမှုမှ ရှင်သန်ရန် အကောင်းဆုံးနေရာဖြစ်သည်"} \cite{33,34}။ အမာလူလ၏ သုတေသနအရ ယနေ့ခေတ်တွင် ဤဒေသအောက်တွင် ကျယ်ပြန့်သော မြေအောက်ဥမင်လိုဏ်ခေါင်းစနစ်တစ်ခု တည်ဆောက်ထားသည်ကို ဖော်ပြခဲ့သည် \cite{36}။

\section{နိဂုံး}

ဤစာတမ်းတွင် အနောက်တိုင်းအထက်တန်းစားများသည် ကမ္ဘာ့ဘေးအန္တရာယ်များ ထပ်ကျော့နေမှုအကြောင်း အသိပညာကို နှစ်ပေါင်းထောင်ချီ၍ ဂရုတစိုက် ထိန်းသိမ်းထားပြီး၊ အခြားတစ်ခုမှာ မကြာမီဖြစ်ပေါ်လာမည်ဟု ယုံကြည်ကာ ထိုသို့သောဖြစ်ရပ်အတွက် ပြင်ဆင်ရန် ကျယ်ပြန့်သော မြေအောက်ခိုလှုံရာနေရာများ တည်ဆောက်ထားပြီး၊ ကမ္ဘာ့အာဏာရရှိရန် နိုင်ငံရေးနှင့် စစ်ရေးအရ အခွင့်အရေးယူရန် အစီအစဉ်ရှိကြောင်း အကြံပြုထားသော သက်သေအမျိုးမျိုးကို ကျွန်ုပ်အသေးစိတ် ဖော်ပြခဲ့ပါသည်။ အမေရိကန်တွင် ဤအရာကို ငွေကြေးထောက်ပံ့ပုံနှင့် ပတ်သက်၍ အရိပ်အမြွက်များကိုလည်း ဖော်ပြခဲ့ပြီး၊ ဤဇာတ်လမ်းကို ဦးဆောင်နေသည့် အတိအကျသော သွေးနွယ်စုများနှင့် ပတ်သက်သည့် အလှမ်းမဝေးဆုံး သီအိုရီကို ကိုးကားဖော်ပြခဲ့ပါသည်။ ပိုမိုသိရှိလိုသူများအတွက် ကိုးကားချက်များကို တူးဖော်ရှာဖွေခြင်းဖြင့် ရှာဖွေနိုင်သည့် အချက်အလက်များစွာ ကျန်ရှိနေသေးသည်။
အနီးစပ်ဆုံး တိုင်းတာနိုင်သော ဘူမိရူပန္တရအဖြစ်အပျက်တစ်ခု ဖြစ်ပေါ်လာမည့်အချက်ကို ညွှန်ပြနေသည့် အချက်မှာ ကမ္ဘာ့သံလိုက်စက်ကွင်း အလျင်အမြန် ရွေ့လျားနေခြင်းပင် ဖြစ်သည်။ ဤအချက်ကို သံလိုက်မြောက်ဝင်ရိုးစွန်း၏ အရှိန်မြင့်ရွေ့လျားမှု (ပုံ \ref{fig:13}) နှင့် တောင်အတ္တလန္တိတ် သံလိုက်စက်ကွင်း ချို့ယွင်းမှု ကြီးထွားလာခြင်းတို့အပြင် လွန်ခဲ့သော နှစ် ၄၀၀ အတွင်း သံလိုက်စက်ကွင်း အားနည်းလာမှုနှင့် ပုံပျက်လာမှုတို့မှ တိုင်းတာနိုင်သည် \cite{3}။ ထိုသိပ္ပံနည်းကျ အချက်အလက်များကို ကျွန်ုပ်၏ ပထမဆုံး ECDO စာတမ်း နှစ်စောင်တွင် အကျယ်တဝင့် ဆွေးနွေးထားပြီး ကျွန်ုပ်၏ ဝက်ဘ်ဆိုက်တွင် ရယူနိုင်ပါသည် \cite{3}။

\begin{figure}[t]
\begin{center}
% \fbox{\rule{0pt}{2in} \rule{0.9\linewidth}{0pt}}
   \includegraphics[width=1\linewidth]{npw.jpg}
\end{center}
   \caption{၁၅၉၀ မှ ၂၀၂၅ အထိ သံလိုက်မြောက်ဝင်ရိုးစွန်း၏ တည်နေရာကို ၅ နှစ်တစ်ကြိမ် ပြသထားခြင်း \cite{41}။ ၁၉၇၅ ခုနှစ်မှစ၍ ၎င်း၏ရွေ့လျားမှုသည် အလျင်အမြန် တိုးမြှင့်လာခဲ့သည်။}
\label{fig:13}
\label{fig:onecol}
\end{figure}

အဆုံးသတ်အနေနဲ့ ကျွန်တော်ဟာ Amallulla ရဲ့ အနာဂတ်ပြောဟောချက်တစ်ခုကို မှတ်သားစရာအဖြစ် ချန်ထားခဲ့မယ်။ \textit{"\textbf{�ရာအားလုံးဟာ တစ်ခုတည်းသောအရာဖြစ်တယ်}"} ဆိုတဲ့ အဆိုအမိန့်ကို ရှင်းပြထားတာပါ။ \textit{"ဒီနေရာမှာ သင့်စိတ်ကူးကို အကန့်အသတ်မရှိ တွန်းပို့ဖို့ လိုအပ်ပါတယ်။ သင်အခုနေထိုင်နေတဲ့ ကလေးဘဝကတည်းက သိခဲ့တဲ့ ကမ္ဘာကြီးကို မေ့ပစ်လိုက်ပါ။ ဒါကို နောက်ချန်ခဲ့လိုက်ပါ။ ဒါဟာ Matrix ရုပ်ရှင်ဇာတ်လမ်းထဲက ပြသထားတဲ့အတိုင်း သင်နိုးထလာတဲ့အချိန်အထိ သင့်ကိုအိပ်ပျော်စေဖို့ ဖန်တီးထားတဲ့ အပြည့်အဝ လုပ်ကြံဖန်တီးထားတဲ့ အမှန်တရားတစ်ခုပါပဲ။ တစ်ခါတစ်လေ ကျွန်တော်ဟာ ရုပ်ရှင်ဇာတ်ညွှန်းတစ်ခုရေးနေမိတယ်လို့ ခံစားမိတယ်၊ ဒါပေမယ့် ဒီဝက်ဘ်ဆိုက်မှာ ကျွန်တော်မျှဝေနေတဲ့အရာက အမှန်တရားပါ။ 'အရာအားလုံးဟာ တစ်ခုတည်းသောအရာဖြစ်တယ်' ဆိုတဲ့အချက်ကို နားလည်ဖို့ ငါးနှစ်ကျော်ကြာခဲ့တယ်၊ ဒါကို An Apocalyptic Synthesis ရဲ့ ဆောင်ပုဒ်အဖြစ် ချက်ချင်းရွေးချယ်လိုက်တယ်။ ဒါဟာ ရှင်းပြဖို့ခက်ခဲတဲ့ အယူအဆတစ်ခုပါ။ အခုအချိန်မှာတော့ Matrix ရုပ်ရှင်ဇာတ်လမ်းကို ဥပမာအဖြစ် စဉ်းစားကြည့်ရအောင်။ ဒါဟာ ကောင်းတဲ့ ဥပမာတစ်ခုပါပဲ။ ကျွန်တော်ရှင်းပြဖို့ခက်ခဲတဲ့အရာက ကျွန်တော်ပြောမယ့်အရာဟာ ချဲ့ကားပြောတာမဟုတ်ဘူးဆိုတဲ့အချက်ပါပဲ။ အခုအချိန်မှာတော့ Matrix ရုပ်ရှင်ဥပမာဟာ ကျွန်တော်ပြောမယ့်အရာရဲ့ ကြမ်းတမ်းတဲ့အမှန်တရားကို သဘောပေါက်နိုင်ဖို့ အနီးစပ်ဆုံးဥပမာပါပဲ။ \textbf{သင့်ဘဝထဲက အရာအားလုံး၊ မှတ်တမ်းတင်ထားတဲ့ သမိုင်းတစ်လျှောက်၊ ပင်မသိပ္ပံနဲ့ ပညာရေးလောက၊ နိုင်ငံရေး၊ ဘာသာရေး၊ အရာအားလုံးဟာ တစ်နည်းနည်းနဲ့ ကမ္ဘာ့အပေါ်ယံလွှာ ရွေ့လျားမှု (သို့) ဝင်ရိုးစွန်းယိမ်းမှုနဲ့ သက်ဆိုင်နေပါတယ်။} သင်အခုမြင်နေရတာမဟုတ်သေးပါဘူး။ မကောင်းတဲ့အိပ်မက်တစ်ခုကနေ နိုးထလာသလိုလည်း ဒီအမှန်တရားကို သတိမထားမိသေးပါဘူး။ အချိန်ယူရပါမယ်။ ဒါပေမယ့် ကျွန်တော်အာမခံပါတယ်၊ ဒီလမ်းရဲ့အဆုံးမှာတော့ သင်ဟာ Matrix ကွန်ပျူတာဖန်တီးထားတဲ့ အမှန်တရားထဲမှာ တစ်သက်လုံးနေထိုင်ခဲ့တယ်ဆိုတဲ့ အသိအမှတ်ပြုမှုကို ရရှိမှာပါ"} \cite{33,34}။
အားလုံးကိုကံကောင်းပါစေ။

\section{ကျေးဇူးတင်လွှာ}

အများပြည်သူအတွက် အသိပညာများကို မျှဝေပေးခဲ့ကြသော လူတိုင်းကို ကျေးဇူးတင်ပါသည်။ သင်တို့မရှိပါက ဤလုပ်ငန်းများ ဖြစ်မြောက်လာမည် မဟုတ်သလို လူသားမျိုးနွယ်သည်လည်း အမှောင်ထဲတွင် ရှိနေဦးမည်ဖြစ်သည်။ သင်တို့၏ ရွေးချယ်မှုများသည် ထာဝရကာလပတ်လုံး ဖူးပွင့်နေမည်ဖြစ်သည်။ ကျွန်ုပ်တို့သည် သင်တို့အား အရာရာတိုင်းကို ကျေးဇူးတင်ရှိပြီး ကျွန်ုပ်သည် အဆုံးမဲ့ ကျေးဇူးတင်နေမည်ဖြစ်သည်။

\clearpage
\twocolumn

{\small
\renewcommand{\refname}{ကိုးကားချက်များ}
\bibliographystyle{ieee}
\bibliography{egbib}
}
\end{document}
\documentclass[10pt,twocolumn,letterpaper]{article}

% 나만의 설정
\usepackage{booktabs}
% \usepackage{caption}
% \captionsetup[table]{skip=8pt}   % 테이블에만 영향
\usepackage{stfloats}  % 프리앰블에 추가하세요
\usepackage{float}
% \usepackage{kotex}

%— load ko.TeX’s XeTeX-based Korean support with Hangul captions —
\usepackage[hangul]{kotex}   % ↪️ loads xetexko with the [hangul] option :contentReference[oaicite:0]{index=0}

%— declare your main Hangul font (real Bold weight) —
\setmainhangulfont[
  Language=Korean,           % enable Korean-specific shaping/ot features
  UprightFont = *-Regular,   % e.g. NotoSerifCJKkr-Regular.otf
  BoldFont    = *-Bold       % e.g. NotoSerifCJKkr-Bold.otf
]{Noto Serif CJK KR}         % make sure you’ve installed this family :contentReference[oaicite:1]{index=1}


\usepackage{cvpr}
\usepackage{times}
\usepackage{epsfig}
\usepackage{graphicx}
\usepackage{amsmath}
\usepackage{amssymb}

% 필요하다면 여기에 다른 패키지를 포함하세요, hyperref 전에.

% hyperref를 주석 처리했다가 다시 해제하면,
% latex을 다시 실행하기 전에 egpaper.aux를 삭제해야 합니다.
% (아니면 latex을 실행할 때 첫 번째로 'q'를 누르세요.)

% run, let it finish, and you should be clear).
\usepackage[breaklinks=true,bookmarks=false]{hyperref}

\cvprfinalcopy % *** Uncomment this line for the final submission

\def\cvprPaperID{****} % *** Enter the CVPR Paper ID here
\def\httilde{\mbox{\tt\raisebox{-.5ex}{\symbol{126}}}}

\renewcommand{\figurename}{그림}   % or whatever you like instead of "Hình"

\makeatletter
\def\abstract{%
  \centerline{\large\bf 개요}% <-- your new label
  \vspace*{12pt}%
  \it%
}
\makeatother

% This makes the font slightly bigger than base (10) and bold in Subsection headings rather than using ptmb
\makeatletter
\def\cvprsubsection{%
  \@startsection{subsection}{2}{\z@}%
    {8pt plus 2pt minus 2pt}{6pt}%
    % {\normalfont\bfseries\selectfont}%
    {\normalfont\bfseries\fontsize{11}{13}\selectfont}%
}
\makeatother

% So this hardcodes the style for the numbers in the section/subsection headings so they're bold
\font\elvbf=ptmb scaled 1100
\font\elvbfs=ptmb scaled 1200
\makeatletter
% Section number: Large + bold
\renewcommand\thesection{%
  {\elvbfs\arabic{section}}%
}

% Subsection number: normalsize + bold + custom punctuation
\renewcommand\thesubsection{%
  {\elvbf
   \arabic{section}.\arabic{subsection}}%
}
\makeatother

% Pages are numbered in submission mode, and unnumbered in camera-ready
%\ifcvprfinal\pagestyle{empty}\fi
\setcounter{page}{1}
\begin{document}

%%%%%%%%% TITLE
\title{ECDO 논문 3: 지질학적 격변에 대비한 서방 지배 세력의 현시점 준비 정황에 대한 증거}

\author{준호\\
2025년 6월 출간\\
웹사이트(논문 다운로드): \href{https://sovrynn.github.io}{sovrynn.github.io}\\
ECDO 연구 저장소: \href{https://github.com/sovrynn/ecdo}{github.com/sovrynn/ecdo}\\
{\tt\small junhobtc@proton.me}
% For a paper whose authors are all at the same institution,
% omit the following lines up until the closing ``}''.
% Additional authors and addresses can be added with ``\and'',
% just like the second author.
% To save space, use either the email address or home page, not both
% \and
% Author2\\
% Institution2\\
% First line of institution2 address\\
% {\tt\small secondauthor@i2.org}
}

\maketitle
%\thispagestyle{empty}

%%%%%%%%% ABSTRACT
\begin{abstract}
2024년 5월, “윤리적 회의론자(The Ethical Skeptic)”이라는 가명을 사용하는 익명의 온라인 저자가 \textit{발열성 핵-맨틀 분리 장니베코프 진동}(ECDO)이라는 획기적인 이론을 발표했다\cite{0,1}. 이 이론에 따르면, 지구는 과거에 자전축의 급격한 전환을 겪었으며, 이로 인해 관성 작용에 따른 대규모 해수 범람이 발생해 대륙 전역이 물에 잠기는 재난이 발생했다. 또한 이 이론은 향후 유사한 지질학적 반전이 임박했음을 시사하는 지구물리학적 과정과 데이터를 제시하고 있다. 이와 같은 격변적 대홍수나 종말론적 예측은 새로운 개념은 아니지만, ECDO 이론은 과학적이고 현대적이며, 다학제적인 데이터 기반 접근법을 통해 고유한 설득력을 가진다.

본 논문은 해당 주제에 대한 나의 세 번째 연구\cite{2,3}이며 특히 이 이론의 현대 정치적 함의를 중점으로 살펴본다.

\begin{flushleft}
\begin{enumerate}
    \item 서구 열강은 지질학적 격변이 임박했다고 믿고 있으며, 이 사건을 정치·군사적으로 이용하려는 계획을 가지고 있음을 진술한 내부고발자들의 증언.
    \item 이 격변에 대비해 광범위한 지하 및 해저 기지를 구축한 정황.
    \item 이러한 기지를 건설하기 위해 서방의 통화 시스템에서 대규모 자금이 유출되고 있는 증거.
\end{enumerate}
\end{flushleft}

본 논문은 서방 지배 권력들이 임박한 지질학적 격변을 믿고 있으며, 이에 체계적이고 장기적인 대비 작업을 수행하고 있음을 문서화한 것이다.
\end{abstract}

\begin{figure}[b]
\begin{center}
% \fbox{\rule{0pt}{2in} \rule{0.9\linewidth}{0pt}}
   \includegraphics[width=1\linewidth]{freemason.jpg}

\end{center}
   \caption{1992년 런던 얼스코트(Earls Court)에서 촬영된 영국 프리메이슨들의 모습. 핵폭탄을 투하하여 세계를 지배하기 위해 은밀히 모의하고 있다 \cite{5}.}
\label{fig:1}
\label{fig:onecol}
\end{figure}

%%%%%%%%% BODY TEXT
\section{프리메이슨과 "앵글로색슨 미션"(영국 옛 민족의 임무)}

2010년 1월, 내부고발자들의 증언을 수집·보도하는 대안 언론 단체, 프로젝트 카멜롯(Project Camelot)은 \cite{4,6} 2005년 6월 런던 시티에서 열린 프리메이슨 고위급 회의에 실제 참석했던 내부 관계자를 인터뷰하였다. 해당 회의에서는 \textbf{"지질학적 사건"}, 즉 전 지구적 자연 재해를 전제로 한 군사 및 정치 전략이 논의되었다


\begin{figure*}[b]
\begin{center}
% \fbox{\rule{0pt}{2in} \rule{.9\linewidth}{0pt}}
\includegraphics[width=1\textwidth]{british.jpg}
\end{center}
   \caption{앵글로색슨 세력의 위세를 보여주는 1937년 당시의 대영제국 지도 \cite{14}.}
   \label{fig:2}
\end{figure*}

내부자의 증언에 따르면, 회의에 참석한 인원은 약 25~30명 정도였으며, \textit{"모두 영국인이었고, 그중 일부는 영국에서 누구나 알아볼 수 있을 만큼 유명한 인물들이었다. 귀족 출신도 있었고, 몇몇은 상당히 전통 있는 가문 출신이었다. 회의에서 내가 확인한 인물 중 한 명은 고위 정치인이었고, 두 명은 경찰 고위 간부였으며, 또 한 명은 군 출신 인물이었다. 이들 모두 전국적으로 잘 알려져 있으며, 현 정부의 핵심 자문 역할을 수행하고 있는 인물들이다. 지금 이 시점에서도 말이다"} \cite{4}라고 밝혔다. 이 내부자는 해당 회의에 참석한 경위에 대해\ \textit{"순전히 우연이었다. 원래는 분기마다 열리는 평범한 회의인 줄 알고 갔는데, 전혀 예상치 못한 회의였다. 아마 내가 초대된 이유는, 내가 맡고 있던 직책 때문이고 또한 나도 그들과 같은 일원이라고 그들이 생각했기 때문이다."} 라고 말했다\cite{4}.

회의(2005년에 진행)에서 논의된 사건의 전개 일정은 다음과 같다.

\begin{flushleft}
\begin{enumerate}
    \item 이란이나 중국을 자극하여 전술핵무기를 사용하게 만들고 제한적인 핵 교전을 유도한 뒤 휴전 협정을 체결함.
    \item 중국을 주요 대상으로 한 생물학 무기의 투입. 보도에 따르면 이는 “70년대부터 계획된 것”으로 알려짐.
    \item 공포와 혼란을 명분으로 한 전체주의적 군사 정권의 출범.
\end{enumerate}
\end{flushleft}

그러나 가장 중요한 것은 이 사건들 이후에 예상되는 일이다. \textit{"결국 전쟁으로 향하게 될 것이며 그 후에는 전 세계인들에게 영향을 미칠 지질학적 사건이 발생할 것이다."} \cite{4}. 내부자의 주장에 따르면, 이 지질학적 사건이 발생하면 \textit{\textbf{"지구의 지각이 약 30도, 즉 남쪽으로 약 1700~2000마일 가량 이동할 것이며 이로 인해 대격변이 일어나게 된다. 그 여파는 장기적으로 지속될 것이다."}} \cite{4}.

이 모든 비밀 계획의 이유는 물론 권력이다. 내부자는 다음과 같이 설명한다. \textit{"그 시점이 되면 우리는 모두 핵전쟁과 생물학전쟁을 겪은 상태일 것이다. 만약 이 일이 실제로 벌어진다면, 지구 인구는 극적으로 감소할 것이다. 그리고 지질학적 사건이 발생하면 생존 인구는 또다시 절반으로 줄어들 가능성이 크다. 재난 이후에 살아남는 자가 다음 시대에 세계와 그 생존 인류를 이끌 자를 결정하게 될 것이다. 결국 우리는 대격변 이후의 시대를 이야기하고 있는 것이다. 누가 지배하게 될 것인가? 누가 통제권을 쥐게 될 것인가? 바로 그게 핵심이다. 그래서 그들은 이 모든 일이 일정한 시간 안에 벌어지기를 그토록 절박하게 바라고 있는 것이다. [혼란]이 닥치기 전에 반드시 어떠한 체계가 미리 구축되어 있어야 하고, 그 체계는 다가올 사태 속에서도 살아남을 수 있을 것이라는 어느 정도의 확신이 있어야 한다. 그래야 재난 이후에도 안정적으로 권력을 유지하고, 이전에 누려왔던 권력을 계속 유지할 수 있기 때문이다."} \cite{4}. 인터뷰에서는 이 계획의 명칭인 "앵글로색슨 미션"(영국 옛 민족의 임무)에 대해서도 논의된다. \textit{[진행자]: "이 계획이 ‘앵글로색슨 미션’이라고 불리는 이유는 결국 중국인을 제거하려는 것이고, 대격변 이후 세상이 재건될 때, 오직 앵글로색슨 계열만이 새로운 세상을 물려받아 이를 재건할 위치에 있도록 하려는 것이 맞는가?" [내부자]: "그게 사실인지는 나도 정확히는 모르지만, 당신의 말에 동의는 한다. 적어도 20세기 동안, 그리고 19세기, 18세기까지 거슬러 올라가면, 이 세계의 역사는 주로 서방과 지구 북반구에서 주도되어 왔다는 것은 분명하다."} \cite{4}.

향후 있을 이 지질학적 사건의 정확한 발생 시점에 대해, 내부자는 다음과 같은 추정을 제시한다. \textit{"지금이 바로 준비를 마쳐야 할 시점이라는 느낌이 든다. 매우 직관적인 감각이긴 하지만... 그들은 그 일이 언제 일어날지에 대해 상당히 명확한 판단을 가지고 있는 것 같다. \textbf{내가 살아 있는 동안, 대략 20년 안에 그 일이 벌어질 것이라는 강한 예감이 든다}. 우리는 지금 그 지질학적 사건이 일어나려는 시기에 접어든 상태다. 마지막으로 그런 사건이 일어난 것이 약 1만 1,500년 전이고, 이 사건은 약 1만 1,500년 주기로 반복된다. 지금이 그 주기가 도래한 시점이다. 그들은 이 사건이 일어날 것을 알고 있다. 그것이 반드시 일어날 것이라는 확신에 가까운 지식을 가지고 있다. 다시 말하지만, 이건 이와 같은 사건이다. 그들이 모르고 있다면 오히려 말이 안 된다. 결국, 세상에서 가장 뛰어난 두뇌들이 그들을 위해 이 문제를 하려고 할 것이다"} \cite{4}.

이는 우리가 매우 감사히 여겨야 할 강력한 증언이다. 인터뷰에서 저자는 제1차 세계대전과 제2차 세계대전이 인위적으로 조작된 전쟁이었다는 자신의 신념과, 앵글로색슨 미션이 수세대 전부터 이어져온 계획임이 거의 확실하다는 점도 언급하고 있다. 이 인터뷰는 지금으로부터 15년이 지난 2010년에 진행되었다. 내부자가 언급한 지질학적 사건의 예측 시한인 20년까지 이제 5년밖에 남지 않았다.

\subsection{지질 격변에 대한 드루이드의 서구 밀교 지식}

지속적으로 반복되는 지질 격변에 대한 서구의 지식은 프리메이슨뿐만 아니라 다양한 집단에 의해 은밀히 전승되어 왔다. 드루이드는 최소 2400년 전부터 시작된 고대 켈트 문화로, 기록이 많이 남아있다. \cite{7}. 이들은 지구에서 주기적으로 발생하는 격변에 대한 지식을 후대에 전승해왔다. 마지막으로 알려진 드루이드는 벤 맥브래디(Ben McBrady)이다. 그는 1992년 다큐멘터리”마지막 드루이드(The Last Druid)”에서 드루이드의 지식에 대해 다음과 같이 말했다. \textit{"내가 전통적으로 마지막일지도 모르는 드루이드의 계보는 세계를 뒤흔든 지난 대격변에서부터 시작되었다. 거대한 전기 폭풍, 혜성의 잔류 궤도, 유성우 등으로 발생한 엄청난 충격으로 인해 우리가 알고 있는 문명은 완전히 파괴되었다. 모든 지식은 드루이드 계보의 손으로 전해졌는데, 드루이드 계보는 특히 천문학에 대한 관심이 높았다. 이는 수많은 재난을 직접 겪은 결과였으며, 천문학을 완전한 이해하면 이들 재난이 발생할 가능성이 있는 시점을 예측하고 그에 따라 스스로를 보호할 수 있는 방법을 마련할 수 있다고 여겨졌기 때문이다. 아일랜드의 거대한 거석 유적지를 살펴보면, 일반적으로 개석식 통로 무덤으로 알려진 구조물이 사실은 매우 원시적인 형태의 방공호였다는 것을 알 수 있다. 이들은 해일이 닿을 수 있는 높이보다 훨씬 높으며 운석도 막을 수 있다."} \cite{8,9}.
\section{현대 서방권의 지질 격변 대비 정황}

서방의 지배 권력들이 전 지구적 지질 격변이 임박했다고 믿고 있다면, 그러한 사태로부터 스스로를 보호하기 위해 상당한 수준의 대비를 진행하고 있을 것으로 예상할 수 있다. 실제로, 이미 공개된 정보 속에서 다수의 서방 국가들이 대규모 지하 기지 네트워크를 설치했음을 보여주는 증거가 있다. 이러한 시설들은 핵전쟁 시에 주민들을 보호하는 용도로 활용될 수 있을 뿐 아니라, 다양한 유형의 자연 재해까지 막을 수 있다. 프로젝트 카멜롯에서 증언한 영국 프리메이슨 고위 인사의 발언을 고려하면 \cite{4,6}, 이러한 시나리오들은 단순한 가능성이 아니라 사전에 계획된 행동으로 보인다. 주목할 점은 이와 같은 지하 기지를 건설하고, 인력과 시설을 유지·관리하기 위해서는 막대한 자금이 필요하다는 사실이다. 이는 미국 정부에서 지난 18년간 사라진 수십조 달러에 달하는 자금 규모와 정확히 일치한다(다음 절에서 다룰 예정)\cite{11,12,13}. 멸종 수준의 재난에 대비한 또 다른 준비 사례로는 종자 저장소나 지식 보관소와 같은 다양한 아카이브 프로젝트가 있다.

\subsection{미국의 지하 및 해저 기지}

지하 기지에 대한 가장 광범위한 공개 조사를 수행한 인물은 리차드 소더(Richard Sauder)로, 그는 지하 심층 기지에 관한 여러 권의 저서를 출간한 미국의 독립 연구자이다 \cite{22}. 소더의 연구는 정부 문서와 설계 계획의 아카이빙, 과거 및 현재 뉴스 기사와 기술 자료의 분석, 소식통의 확보, 내부 고발자의 증언 수집 등으로 구성되어 있다. 그의 연구에 따르면, 미국 본토와 해외 영토 전역에 걸쳐 깊이 약 4.8km 이상에 이르는 방대한 지하 및 해저 기지 네트워크가 존재하는데 일부는 지하 진공관식 자기부상 열차로 연결되어 있을 가능성도 있다(그림 \ref{fig:4}). 이러한 기지는 \textit{"고위 금융권, 국제기관, 정부기관 간의 자금 세탁용 페이퍼 컴퍼니 체계"}를 통해 비밀리에 자금이 지원되고 있으며, 이는 미합중국이라는 회사를 실질적으로 소유하고 있는 세력이 운영하고 있다고 한다 \cite{22}. 이 기지의 규모와 수에 대한 후속 조사는 캐서린 오스틴 피츠(Catherine Austin Fitts)와 그녀의 공동 연구자에 의해 수행되었으며, 그들은 미국 내 지하 및 해저 기지가 약 170개에 달한다는 추정치를 제시하였다 \cite{16,20}.

\begin{figure*}[t]
\begin{center}
% \fbox{\rule{0pt}{2in} \rule{.9\linewidth}{0pt}}
\includegraphics[width=1\textwidth]{baseskor.png}
\end{center}
   \caption{소더의 연구에 따라 지하 및 해저 기지가 존재하는 것으로 확인된 주요 위치를 보여주는 지도. 내륙으로 연결되는 해저 잠수함 터널의 위치도 함께 표시되어 있다. 소더는 \textit{"\textbf{이외} 훨씬 더 많은 시설이 존재한다고 확신한다"}라고 말한다 \cite{22}.}
   \label{fig:4}
\end{figure*}

\begin{figure}[t]
\begin{center}
% \fbox{\rule{0pt}{2in} \rule{0.9\linewidth}{0pt}}
   \includegraphics[width=1\linewidth]{penta.jpg}
\end{center}
   \caption{백악관과 펜타곤 지하에는 무엇이 존재할까? 깊은 지하에 터널 네트워크가 존재하는 것이 분명하다(사진: \cite{31}).}
\label{fig:3}
\label{fig:onecol}
\end{figure}

다음은 소더가 수집한 내부자의 증언 일부로, 기지의 구체적인 규모와 범위를 알 수 있다.

\begin{flushleft}
\begin{enumerate}
    \item 캠프 데이비드, 메릴랜드: \textit{"소식통에 따르면, 캠프 데이비드의 지하 시설은 매우 방대하고 정교하며, 비밀 터널의 길이가 너무 길어서 누구도 그 전체 구조를 파악할 수 없다."} \cite{22}.
    \item 백악관, 워싱턴 DC: \textit{"가까운 친구 한 명은 1960년대 린든 B. 존슨 행정부 시절, 이 시설의 내부로 내려간 적이 있다. 그녀는 백악관 내의 엘리베이터에 탑승했는데 바로 아래로 안내되었다. 그녀는 엘리베이터가 약 17층 아래로 내려갔을 것이라고 생각한다. 지하에서 문이 열리자 그녀는 매우 길게 뻗어 있는 복도로 안내받았으며 복도는 멀리 시야의 끝까지 사라지는 듯한 느낌이었다. 복도 양옆에는 다른 문들과 통로들이 이어져 있었다."} \cite{22}. 그림 \ref{fig:3} 참고.
    \item 메릴랜드 주 포트 미드(1970년대에 우연히 “지하실”을 발견한 한 증언자의 말): \textit{"문을 열었을 때, 아래로 내려가는 계단이 있었다. 난 난간 사이로 아래를 내려다보았다. 정확한 층수는 세지 않았지만, 대략 15~20층 정도 되는 느낌이었다. 한 층을 내려가니 또 다른 문이 있었다. 문을 열고 고개를 내밀어 좌우를 살펴보니, 양쪽으로 끝이 보이지 않는 터널이 있었다. 분명 지상의 건물과 주차장 범위를 훨씬 넘어선 규모였다. 맞은편 벽에는 약 9~12m 간격으로 문들이 있었다. 몇 층을 더 탐색해 보려고 한 층을 더 내려갔는데... 첫 두 층과 동일한 구조였다. 또 한 층 내려가 봤을 때도 같은 모습이 반복되고 있었다."} \cite{22}.
\end{enumerate}
\end{flushleft}

\begin{figure}[t]
\begin{center}
% \fbox{\rule{0pt}{2in} \rule{0.9\linewidth}{0pt}}
   \includegraphics[width=1\linewidth]{undersea.jpg}
\end{center}
   \caption{월터 커슈너(Walter Koerschner)가 그린 해저 기지 상상도. 월터 커슈너는 1960년대 캘리포니아 차이나 레이크(China Lake) 해군 무기 센터 소속 록-사이트(Rock-Site) 해저 기지 팀에서 일했던 삽화가이다. 소더의 한 정보원에 따르면, 차이나 레이크 지하 약 1.6km 깊이에 기지가 존재한다고 한다 \cite{22,23}.}
\label{fig:5}
\label{fig:onecol}
\end{figure}

소더는 또한 시속 약 3,200km의 속도를 내는 지하 자기부상열차, 해저 기지(그림 \ref{fig:5}), 그리고 내륙으로 연결되는 해저 잠수터널에 대한 증언들을 추가로 입수했다. 멕시코만 해저 기지와 관련된 한 증언에 대해 소더는 이렇게 전했다. \textit{"『해저 및 지하 기지(Underwater and Underground Bases)』 출간 후 약 반년 뒤, 한 남성이 연락해 특이한 해저 프로젝트에 대한 정보를 가지고 있다고 말했다. 그는 해당 프로젝트가 멕시코만 해저 아래에 위치하며, 구체적으로 파슨스(Parsons) 사가 계약자라고 언급했다. 더 나아가 파슨스 사가 해저 약 850m 깊이에서 운영되도록 설계된 특수 장비를 구입했다고 덧붙였다. 이 장비는 매우 특수해서, 해당 장소에 인간이 상주할 것을 전제로 한 설계임이 분명했다."} \cite{22}.

\begin{figure}[t]
\begin{center}
% \fbox{\rule{0pt}{2in} \rule{0.9\linewidth}{0pt}}
   \includegraphics[width=1\linewidth]{sub.jpg}
\end{center}
   \caption{월터 커슈너(Walter Koerschner)가 그린 해저 잠수터널의 상상도 \cite{22,23}.}
\label{fig:6}
\label{fig:onecol}
\end{figure}

\begin{figure}[t]
\begin{center}
% \fbox{\rule{0pt}{2in} \rule{0.9\linewidth}{0pt}}
   \includegraphics[width=1\linewidth]{iran.jpeg}
\end{center}
   \caption{지하 "미사일 기지"를 촬영한 이란 정부의 공식 동영상 중 일부\cite{39,40}.}
\label{fig:12}
\label{fig:onecol}
\end{figure}

만약 정말로 지하 및 해저 기지 170여 개가 대륙을 가로질러 지표면 수 km 아래에 건설되어 있고, 극초음속 진공 튜브 자기부상 열차로 연결되어 있으며, 시민들의 노동의 결실로 자금이 조달되고 있다면, 오늘날 인류 대다수는 치명적이면서도 순수한 무지 상태에 빠져 있을 것이다. 그들은 자신들의 발아래 무엇이 있는지뿐만 아니라 가까운 미래에 닥칠 것조차 모른 채, 정치인들의 조율된 공허한 발언을 맹목적으로 받아들이고 있다.

추가 설명 - 중동 지역의 현재 분쟁(가자 지구 아래의 하마스 터널\cite{38} 및 이란의 지하 "미사일 도시"(그림 \ref{fig:12}) \cite{39,40})을 통해 대규모 지하 터널 네트워크의 존재가 의심의 여지 없이 확인되었다. 이 같은 사례들은 그러한 구조물을 건설하는 가능성과 실제 존재 여부가 명백히 사실임을 보여준다. 또한 이를 보면 같은 기간 훨씬 더 많은 자본을 보유한 다른 국가에서는 어떤 구조물을 건설했을지 궁금해진다.

\subsection{추가 벙커 및 대재난 대비의 증거}

미국의 지하 "로열 기지" 외에도 노르웨이, 스위스, 스웨덴, 핀란드 등에서 대규모 재난 대비 시설이 확인되고 있다.

\begin{flushleft}
\begin{enumerate}
    \item 프로젝트 카멜롯은 노르웨이 정치인의 증언\cite{25,26}을 공유했는데, 그의 신원은 확인되었지만 비공개로 유지했다. 그는 노르웨이가 18개의 광대한 지하 기지를 보유하고 있으며 노르웨이(이스라엘 등 "다른 많은 국가들"과 함께) 어떤 종류의 자연 재해에 대비해 이러한 기지들을 건설하고 있다고 주장했다. 리차드 소더 또한 노르웨이에서 산 속을 파내어 건설된 거대한 지하 기지 안에 들어가 본 남자의 증언을 받았다 \cite{22}.
    \item 스위스는 알프스 고지대에 수많은 핵 벙커를 보유한 것으로 잘 알려져 있다(그림 \ref{fig:7}). 그 수는 놀랍게도 37만 개가 넘어 모든 주민을 수용할 수 있을 정도다\cite{27}.
    \item 스웨덴과 핀란드는 모든 주요 도시 주민들을 수용할 수 있을 만큼의 벙커를 보유하고 있다\cite{27}.
\end{enumerate}
\end{flushleft}

\begin{figure}[t]
\begin{center}
% \fbox{\rule{0pt}{2in} \rule{0.9\linewidth}{0pt}}
   \includegraphics[width=1\linewidth]{tyrol.jpg}
\end{center}
   \caption{스위스 남부 티롤(South Tyrol) 지역의 벙커 시설. 유럽 알프스 산맥에 위치한 스위스는 산악 벙커를 교묘하게 위장하는 기술로 유명하다\cite{32}.}
\label{fig:7}
\label{fig:onecol}
\end{figure}

\begin{figure}[t]
\begin{center}
% \fbox{\rule{0pt}{2in} \rule{0.9\linewidth}{0pt}}
   \includegraphics[width=1\linewidth]{svalbard.jpg}
\end{center}
   \caption{노르웨이에 위치한 스발바르 세계 종자 저장고. 100만 개 이상의 종자를 보관하고 있다\cite{24}. 도대체 어떤 종류의 대재앙이 이 시설을 필요로 할지 의문이 들 수밖에 없다.}
\label{fig:8}
\label{fig:onecol}
\end{figure}

실리콘밸리의 기업 거물들도 이 사실을 잘 알고 있는 것으로 보인다. 보도에 따르면, \textit{"링크드인 공동창업자이자 저명한 투자자인 리드 호프만은 올해 초 뉴요커와의 인터뷰에서 실리콘밸리 억만장자 중 절반 이상이 지하 벙커 같은 “종말 대비 보험”을 구입했다고 추정했다". 포브스 기고가 짐 돕슨에 따르면, 수많은 억만장자들이 “즉시 이륙할 수 있는” 개인 전용기를 보유하고 있다. 또한 그들은 오토바이, 무기류, 발전기도 소지하고 있다"}고 전했다 \cite{28}.

또한 아크 미션 재단(Arch Mission Foundation)이 운영하는 글로벌 지식 볼트(Global Knowledge Vault)\cite{29}와 스발바르 세계 종자 저장고(Svalbard Global Seed Vault) \cite{30} 같이 다양한 대규모 기록 보관 프로젝트들이 존재한다. 이들 시설은 인류의 핵심 자산을 멸종 수준의 재난 상황에 대비해 보존하기 위해 마련된 것으로 보인다.

\section{대규모 지하기지 자금 조달 메커니즘}

\begin{figure*}[t]
\begin{center}
% \fbox{\rule{0pt}{2in} \rule{.9\linewidth}{0pt}}
\includegraphics[width=0.9\textwidth]{govkor.png}
\end{center}
   \caption{1998년부터 2023년까지 미국 정부의 세입, 지출, 그리고 비밀 지하 기지 지출 현황 \cite{19}.}
   \label{fig:9}
\end{figure*}

어떻게 170여 개가 넘는 대륙간 지하·해저 기지 네트워크가 일반 대중의 눈을 피해 자금이 조달되는 것일까? 이러한 프로젝트에 투입되는 자금 규모와 출처를 추정할 수 있는 단서가 하나 있다. 2017년, 조지 W. 부시 행정부에서 공직을 역임한 미국 투자 은행가 캐서린 오스틴 피츠(Catherine Austin Fitts)와 미시간 주립대 경제학자 마크 스키드모어(Mark Skidmore)는 1998~2015 회계연도 동안 미국 정부에서 21조 달러의 미승인 지출이 발생했음을 발견했다\cite{11,12,13}.

보고서에 따르면, \textit{"2016년 10월 7일 로이터 통신은 스콧 팔트로우의 기사(2016)를 게재했는데, 이 기사는 2015 회계연도에 육군이 6조 5천억 달러의 무근거 회계 조정을 통해 “장부를 균형 있게 보이게 조작했다”라고 보도했다. 당해 육군 일반회계 예산이 1220억 달러였음을 감안할 때 이는 충격적인 폭로였다. 국방부는 이보다 수년 전인 2001년 9월 10일, 당시 도널드 럼즈펠드 국방장관이 의회 청문회(C-SPAN, 2014)에서 국방부가 2조 3천억 달러 상당의 거래 내역을 추적하지 못했다고 진술하며 회계 문제로 주요 언론 헤드라인을 장식한 바 있다. 이 인정은 당일 뉴스 헤드라인을 장식했지만, 하루 뒤인 9·11 테러 사태가 전 세계의 관심을 사로잡으며 잊혀졌다. 마크 스키드모어 교수가 6조 5천억 달러에 달하는 육군의 검증할 수 없는 거래 사실을 알게 되자, 그는 피츠 여사에게 연락해 2017년 봄 HUD와 국방부 내에서 유사하게 규모가 과도하게 큰 검증 불가능한 거래를 보여주는 정부 보고서를 함께 찾아내기로 합의했다. 이후 6개월 동안 스키드모어, 피츠 그리고 소규모 대학원생 팀은 1998-2016년 동안 총 21조 달러에 이르는 무자료 거래가 확인된 정부 공식 문서들을 수집했다"} \cite{12}.

1998년부터 2015년까지 동일한 18년 기간 동안, 공식적으로 인정된 미 정부의 세입은 고작 40.8조 달러에 불과했다. 이는 미국 정부 세입의 절반 이상이 공식 지출 외에도 비밀리에 지하 기지 건설에 투입되었음을 시사한다\cite{15}. 특히 주목할 점은 이러한 비밀 예산이 장기적 재정 적자 상황 속에서도 지출되었으며, 1998년 이전부터 존재해왔을 뿐만 아니라 현재까지도 지속되고 있을 가능성이 높다는 것이다. 이는 지하 기지 건설에 투입된 총 자금 규모가 21조 달러를 훨씬 상회할 것임을 의미한다. 동일한 비밀 예산 비율을 2016-2023년 기간에 적용할 경우, 1998년 이후 총 투입 금액은 36.6조 달러에 이를 것으로 추정된다.

2021년 마크 스키드모어는 블룸버그의 발표와 관련한 후속 연구를 발표했는데, 여기에는 2017-19회계연도 동안 미 국방부가 어마어마한 94.7조 달러의 회계 조정을 기록했다는 내용이 포함됐다 \cite{17,18}. 1913년 연방준비제도 창설 이후 1세기 이상 지속되어온 중앙은행 시스템을 통한 달러의 위조 발행을 고려하면\cite{37}, 모든 공식 회계 자료가 완전히 허위이며, 미국 통화와 정부는 단순히 소수의 특권층이 원하는 만큼 자원을 착복할 수 있는, 심지어는 마음대로 쓸 수 있는 배분 시스템에 불과하다는 사실이 명백해진다.

\section{제우스의 후예: 그림자 서구 왕들의 정체}
그렇다면 이 쇼를 진행하는 자는 도대체 누구일까? 확실히 알 수 없다. 서구 자본의 왕들은 스스로를 그림자 뒤에 숨기기 때문이다. 공인부터 외계인에 이르기까지 다양한 이론이 존재하지만 가장 설득력 있는 답은 "아말룰라(Amallulla)"라는 필명으로 활동하는 한 익명 블로거가 평생에 걸쳐 연구한 작업에서 찾을 수 있다. 그의 작업은 고대 및 현대 역사, 오컬트 상징주의, 서양 정치를 다루는 20명이 넘는 저자와 50개의 "대체 불가능한" 문서들을 종합한 것이었다 \cite{33,34}. 그의 작업은 다가올 지구물리학적 대재앙에 대해 "예언적"이라고밖에 표현할 수밖에 없다. 필자의 작업보다 \textit{훨씬} 포괄적이기 때문이다.

아말룰라는 서양의 정치 파벌로 세 파벌을 확인했으며, 이들을 통틀어 "제우스의 후예(Progeny of Jove)"라고 불렀다. 이들은 지구에 주기적으로 찾아오는 대격변인 "종말의 때"에 대한 지식을 가지고 있다고 여겨진다. 그는 이 세 파벌이 오늘날 서방 국가들을 공동으로 통제하고 있지만, 서로 다른 기원과 역사적 정체성, 과거의 갈등 가능성, 그리고 가치 체계와 행동에서 드러나는 차이점에 따라 세 그룹으로 나뉜다고 보았다.

이 세 파벌은 대략 다음과 같이 분류할 수 있다.

\begin{flushleft}
\begin{enumerate}
    \item \textbf{은행가들}: 고대 로마 엘리트 출신. 이후 템플러 기사단(Knights Templar)과 미국의 북부 관할 프리메이슨으로 변모한 세력이다.
    \item \textbf{사상가들}: 로지크루시안(Rosicrucians)과 미국 남부 프리메이슨으로 구성된 집단이다.
    \item \textbf{예수회와 검은 교황}: 로마 가톨릭 교회 내부에 자리잡은 제우스의 후예 파벌이다.
\end{enumerate}
\end{flushleft}

현재 이 세 파벌은 유럽의 일루미나티, 프리메이슨, CIA로 통합되어 있다. 아말룰라의 설명에 따르면, \textit{"지금 이 종말의 시점에서 제우스의 후예는 심지어 현직 미국 대통령조차 배제하는 “알필요성(NTK, need-to-know)” 보안 등급 뒤에 완벽하게 숨겨져 있다. 즉, 그들은 대중의 감시로부터 자신들을 숨기는 기술을 완성했다. \textbf{제우스의 후예는 미국의 군사와 정부를 통제할 뿐만 아니라, 법정 통화의 권력, 주요 기업, 그리고 그들이 고안한 공화정 형태의 정부(정치인들이 쉽게 타락하고 따라서 통제될 것임을 알고 설계한)를 통해 서구 세계 전체를 지배한다.}"} \cite{33,34}.

\begin{figure}[t]
\begin{center}
% \fbox{\rule{0pt}{2in} \rule{0.9\linewidth}{0pt}}
   \includegraphics[width=1\linewidth]{illuminati.jpg}

\end{center}
   \caption{과연 제우스의 후예들은 누구인가?(이미지: \cite{35})}
\label{fig:10}
\label{fig:onecol}
\end{figure}

\begin{figure}[t]
\begin{center}
% \fbox{\rule{0pt}{2in} \rule{0.9\linewidth}{0pt}}
   \includegraphics[width=1\linewidth]{pike.jpg}
\end{center}
   \caption{붉은색으로 강조 표시된 그 유명한 파이크스 피크 암체와 미국 서부의 지형\cite{36}. 미국은 정말로 이 지역을 통제하기 위해 설립된 것일까?}
\label{fig:11}
\label{fig:onecol}
\end{figure}

아말룰라에 따르면, 이 세력은 종교를 경멸하며 세계 주요 종교의 성서를 자신들에게 유리하게 조작하고 상징주의를 적극 활용한다. 또한 그들은 적에게는 무자비하다. \textit{"\textbf{2,600년이 넘는 세월 동안 그들은 종말의 때에 대한 특정 지식을 가진 자들을 체계적으로 제거해왔다. 여기에는 드루이드, 유대인 카발리스트, 고대 이집트인, 아랍인, 인도의 신비주의자들뿐만 아니라 남아메리카의 길쭉한 두개골 종족과 중앙아메리카의 마야 사제들도 포함된다. 그리고 그들이 종말의 땅으로 지정하기 위해 북아메리카에서 한때 번성했던 인구를 말살했다는 증거는 압도적이다. 미국 “인디언”에 대한 대량 학살은 사소한 작업에 불과했다.}"}\cite{33,34}.

아말룰라는 또한 ”미합중국” 전체 프로젝트가 로키 산맥에 위치한 화강암 산맥인 ”파이크스 피크 암체”를 통제하기 위해 시작되었으며, 이곳이 지구물리학적 재앙에서 뛰어난 보호를 제공한다고 믿었다(그림 \ref{fig:11}). 아말룰라에 따르면, \textit{"우리가 남북전쟁이라고 생각하는 그 사건 전후로, 은행가들과 사상가들은 미합중국 자체의 통 제보다는 파이크스 피크 암체를 차지하기 위해 싸워 왔다. 이곳은 세계에서 가장 독특한 화강암 암체 중 하나다. 이처럼 해발 고도가 매우 높고 바다로부터 멀리 떨어진 곳에 있는 화강암 암체는 세계 어디에도 없다. 지각 변동에도 생존할 수 있는 최적의 위치인 것이다"} \cite{33,34}. 아말룰라의 연구에 따르면, 오늘날 이 일대에는 광범위한 지하 터널 시스템이 건설되어 있다\cite{36}.
\section{결론}

본 논문에서는 수천 년 동안 지구에서 주기적으로 일어나는 대격변에 대한 지식을 서구 엘리트들이 비밀리에 보존해왔으며, 새로운 대격변이 임박했다고 믿는 다양한 증언들을 상세히 제시했다. 이들은 방대한 지하 대피소를 건설해 이러한 사태에 대비하는 한편, 정치·군사적으로 이를 활용해 세계 지배를 달성하려는 계획을 세우고 있는 것으로 보인다. 미국에서 이 프로젝트가 어떻게 자금을 조달했는지에 대한 단서와, 이러한 음모를 주도하는 정확한 혈통에 관한 가장 그럴듯한 이론도 함께 살펴보았다. 더 깊이 알고 싶은 경우, 참고문헌을 조사하면 추가 정보를 찾을 수 있다.

지구물리학적 사건이 임박했음을 나타내는 가장 설득력 있는 측정 가능 데이터는 지구의 급변하는 지자기장이다. 이는 북자기극의 가속화된 이동(그림 \ref{fig:13})과 남대서양 지자기 이상 현상의 확대뿐만 아니라 지난 400년간 지자기장의 전반적인 약화와 왜곡이 가속화되고 있는 현상에서도 확인할 수 있다 \cite{3}. 이러한 과학적 데이터는 내 첫 두 편의 ECDO 논문에서 자세히 논의되었으며 필자의 웹사이트에서 확인할 수 있다 \cite{3}.


\begin{figure}[t]
\begin{center}
% \fbox{\rule{0pt}{2in} \rule{0.9\linewidth}{0pt}}
\includegraphics[width=1\linewidth]{npw.jpg}
\end{center}
\caption{1590년부터 2025년까지 5년 단위로 표시된 지자기 북극의 위치 \cite{41}. 그 이동은 1975년부터 급격히 가속화되기 시작했다.}
\label{fig:13}
\label{fig:onecol}
\end{figure}


끝으로 예언자 아말룰라의 발언으로 글을 마무리하고자 한다. 그는 \textit{"\textbf{모든 것은 하나다}"}를 이렇게 설명했다. \textit{"여기서 나는 당신의 상상력을 최대한 한계까지 밀어붙여야 할 필요가 있다. 어린 시절부터 알고 있던 현재 세계를 잊어야 한다. 지금 살고 있는 세계는 제쳐두라. 그것은 영화 매트릭스에서 묘사된 것과 다름없는 완전히 조작된 현실이며 최후의 순간까지 진실을 감추기 위한 것이다. 때로는 내가 영화 대본을 쓰고 있기를 바랄 때도 있지만 이 웹사이트에서 당신과 공유하는 것은 현실이다. 나는 "모든 것은 하나다"를 깨닫는 데 5년 이상이 걸렸으며, 이를 An Apocalyptic Synthesis의 모토로 삼았다. 이는 전달하기 어려운 개념이다. 지금은 매트릭스 영화의 비유로 생각해보자. 좋은 비유다. 전달하기 어려운 점은 지금 말하려는 것이 과장이 아니라는 사실이다. 현재로서는 매트릭스 영화의 비유가 내가 말하려는 냉혹한 현실을 당신이 이해할 수 있도록 하는 가장 적절한 방법이다. \textbf{기록된 역사 전체, 주류 과학과 학계, 정치, 종교 등 당신의 인생 속 모든 것이 어떤 식으로든 다가올 지각 변동이나 축 기울기와 관련되어 있다.} 당신은 단지 지금 그것을 보지 못할 뿐이다. 마치 악몽에서 깨어나듯 이 현실을 단번에 깨달을 수도 없다. 시간이 걸린다. 하지만 약속하건대 이 길의 끝에는 평생 매트릭스 컴퓨터 시뮬레이션과 같은 세계에서 살아왔음을 깨닫는 순간이 있을 것이다"} \cite{33,34}.

모두에게 행운이 있기를 바란다. 

\section{감사의 말씀}

대중에 지식을 기여하기로 결정한 모든 분들께 감사드린다. 여러분이 없었다면 이 연구는 불가능했을 것이며 인류는 여전히 진실을 모른 채 어둠 속에 머물렀을 것이다. 여러분의 선택은 영원히 빛을 발할 것이다. 우리는 여러분께 모든 것을 빚졌으며, 이에 무한한 감사를 드린다.

\clearpage
\twocolumn

{\small
\renewcommand{\refname}{참고문헌}
\bibliographystyle{ieee}
\bibliography{egbib}
}

\end{document}
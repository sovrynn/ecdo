\documentclass[10pt,twocolumn,letterpaper]{article}

\usepackage{booktabs}
% \usepackage{caption}
% \captionsetup[table]{skip=8pt}   % ມີຜົນກະທົບພຽງແຕ່ຕາຕະລາງ
\usepackage{stfloats}  % ເພີ່ມນີ້ໃສ່ໃນພາກນຳ
\usepackage{float}

% % \usepackage{fontspec}
% \usepackage[english]{babel}

% % load Lao via babelprovide, turn on "onchar=ids" for automatic shaping
% \babelprovide[import,onchar=ids fonts]{lao}

% % main (rm) font for Latin
% \babelfont{rm}{Noto Serif}

% % Lao text in Noto Serif Lao at 1.2× scale
% \babelfont[lao]{rm}{Noto Serif Lao}
% \babelfont[lao]{sf}{Noto Serif Lao}

% % alternate (sans-serif) font for Latin
% \babelfont{alt}{Lato}

% % Lao text in Noto Serif Lao for the alt family too
% \babelfont[lao]{alt}{Noto Serif Lao}

% % GPT: LuaTeX doesn’t have built-in Lao line-breaking rules, but babel can assimilate line breaks to hyphenation if you supply some simple “patterns” for common syllable boundaries. Add this to your preamble:
% \babelpatterns[lao]{%
%   1ດ 1ມ 1ອ 1ງ 1ກ 1າ 1ັ 1ິ 1ີ 1້ 1ົ 1ູ%
% }

%–– line-breaks for Lao instead of Thai ––
\XeTeXlinebreaklocale "lo"
\XeTeXlinebreakskip = 0pt plus 1pt

\usepackage{fontspec}
\usepackage{ucharclasses}

%–– define your two fonts ––
\newfontfamily\latinfont{Latin Modern Roman}         % for all non-Lao (e.g. Latin) text
\newfontfamily\laofont[Script=Lao]{Noto Serif Lao}   % for all Lao text
% \newfontfamily\laofont[Script=Lao]{FreeSerif}   % for all Lao text

%–– ucharclasses auto-detects Unicode blocks ––
\setDefaultTransitions{\latinfont}{}                   % outside Lao → Latin
\setTransitionsFor{Lao}{\laofont}{\latinfont}          % Lao block → Lao font, then back

\usepackage{cvpr}
\usepackage{times}
\usepackage{epsfig}
\usepackage{graphicx}
\usepackage{amsmath}
\usepackage{amssymb}
\usepackage[breaklinks=true,bookmarks=false]{hyperref}

% \makeatletter
% \def\cvprsubsection{\@startsection {subsection}{2}{\z@}
%     {8pt plus 2pt minus 2pt}{6pt}{\bfseries\normalsize}}
% \makeatother

\cvprfinalcopy % *** ຍົກເລີກການເປັນຄຳເຫັນຂອງບັນທຶກນີ້ສຳລັບການສົ່ງສຸດທ້າຍ

\def\cvprPaperID{****} % *** ໃສ່ CVPR Paper ID ທີ່ນີ້
\def\httilde{\mbox{\tt\raisebox{-.5ex}{\symbol{126}}}}

% 1) Choose your desired fixed leading:
\renewcommand\baselinestretch{1.2}  % or 1.3, 1.1…  adjust to taste

% 2) Force TeX to *always* use \baselineskip, never fall back to \lineskip:
\makeatletter
  \setlength\lineskiplimit{-\maxdimen} % always allow baselineskip
  \setlength\lineskip{0pt}             % no extra glue ever
\makeatother

% \renewcommand{\tablename}{ตาราง}
% \renewcommand{\figurename}{รูปที่}   % or whatever you like instead of "Hình"
% \renewcommand{\refname}{เอกสารอ้างอิง}

\makeatletter
\def\abstract{%
  \centerline{\large\bf Abstract}% <-- your new label
  \vspace*{12pt}%
  \it%
}
\makeatother

% This makes the font slightly bigger than base (10) and bold in Subsection headings rather than using ptmb
\makeatletter
\def\cvprsubsection{%
  \@startsection{subsection}{2}{\z@}%
    {8pt plus 2pt minus 2pt}{6pt}%
    % {\normalfont\bfseries\selectfont}%
    {\normalfont\bfseries\fontsize{11}{13}\selectfont}%
}
\makeatother

% So this hardcodes the style for the numbers in the section/subsection headings so they're bold
\font\elvbf=ptmb scaled 1100
\font\elvbfs=ptmb scaled 1200
\makeatletter
% Section number: Large + bold
\renewcommand\thesection{%
  {\elvbfs\arabic{section}}%
}

% Subsection number: normalsize + bold + custom punctuation
\renewcommand\thesubsection{%
  {\elvbf
   \arabic{section}.\arabic{subsection}}%
}
\makeatother

% ຫນ້າຖືກຈໍານວນໃນໂໝດການສົ່ງ, ແລະບໍ່ມີເລກໃນການພ້ອມກ້ອງ
%\ifcvprfinal\pagestyle{empty}\fi
\setcounter{page}{1}
\begin{document}

\title{ECDO ເອກະສານ 3: ຫຼັກຖານຂອງການກະກຽມຂອງອຳນາດການປົກຄອງຕາເວັນຕົກໃນປະຈຸບັນສຳລັບໄພພິບັດທາງທໍລະນີສາດທີ່ຈະເກີດຂຶ້ນໃນໄວໆນີ້}

\author{Junho\\
ພິມເຜີຍເມື່ອ ມິຖຸນາ 2025\\
ເວັບໄຊທ໌ (ດາວໂຫລດບົດຄວາມທີ່ນີ້): \href{https://sovrynn.github.io}{sovrynn.github.io}\\
ຫໍສະໝຸດຄົ້ນຄວ້າ ECDO: \href{https://github.com/sovrynn/ecdo}{github.com/sovrynn/ecdo}\\
{\tt\small junhobtc@proton.me}
}
\maketitle
%\thispagestyle{empty}

\begin{abstract}
ໃນເດືອນພຶດສະພາ 2024, ນັກຂຽນອອນລາຍທີ່ໃຊ້ຊື່ປົກປິດວ່າ "The Ethical Skeptic" \cite{0} ໄດ້ແບ່ງປັນທິດສະດີທີ່ໜ້າສົນໃຈທີ່ຊື່ວ່າ Exothermic Core-Mantle Decoupling Dzhanibekov Oscillation (ECDO) \cite{1}. ທິດສະດີນີ້ແນະນຳວ່າໂລກໄດ້ປະສົບກັບການປ່ຽນແປງຢ່າງຮຸນແຮງໃນແກນການຫມຸນຂອງມັນມາແລ້ວ, ເຮັດໃຫ້ເກີດນໍ້າຖ້ວມທົ່ວໂລກເນື່ອງຈາກນໍ້າທະເລຖືກກະທົບຈາກແຮງເຄື່ອນທີ່ຂອງການຫມຸນ. ນອກຈາກນີ້, ມັນຍັງສະເໜີຂະບວນການທາງພູມສາດແລະຂໍ້ມູນທີ່ຊີ້ບອກວ່າການປ່ຽນແປງແບບນີ້ອາດຈະເກີດຂຶ້ນອີກໃນໄວໆນີ້. ເຖິງແມ່ນວ່າການຄາດເດົາກ່ຽວກັບນໍ້າຖ້ວມແລະວັນໂລກສິ້ນໂລກແບບນີ້ຈະບໍ່ແມ່ນເລື່ອງໃໝ່, ແຕ່ທິດສະດີ ECDO ກໍ່ເປັນທີ່ສົນໃຈເນື່ອງຈາກວິທີການທາງວິທະຍາສາດ, ທັນສະໄໝ, ຫຼາຍສາຂາວິຊາ ແລະ ອີງໃສ່ຂໍ້ມູນ.

ບົດຄວາມນີ້ແມ່ນຜົນງານທີ່ສາມຂອງຂ້າພະເຈົ້າ \cite{2,3} ກ່ຽວກັບຫົວຂໍ້ນີ້, ແລະ ຈຸດສຸມໃສ່ດ້ານການເມືອງປະຈຸບັນຂອງທິດສະດີນີ້:
\begin{flushleft}
\begin{enumerate}
    \item ການຢັ້ງຢືນຈາກຜູ້ເປີດເຜີຍຄວາມຈິງທີ່ວ່າອຳນາດຕາເວັນຕົກເຊື່ອວ່າຈະເກີດໄພພິບັດທາງທໍລະນີສາດໃນໄວໆນີ້ ແລະ ມີແຜນທີ່ຈະໃຊ້ປະໂຫຍດທາງດ້ານການເມືອງ ແລະ ທາງທະຫານຈາກເຫດການນີ້.
    \item ຫຼັກຖານກ່ຽວກັບຖານທີ່ຕັ້ງຢູ່ເບື້ອງລຸ່ມດິນ ແລະ ໃຕ້ນ້ຳຂອງຕາເວັນຕົກທີ່ຖືກສ້າງຂຶ້ນເພື່ອກຽມພ້ອມຮັບເຫດການດັ່ງກ່າວ.
    \item ຫຼັກຖານກ່ຽວກັບການເບີກເງິນຈຳນວນມະຫາສານຈາກລະບົບເງິນຕາຂອງຕາເວັນຕົກເພື່ອໃຊ້ທຶນຖານເຫຼົ່ານີ້.
\end{enumerate}
\end{flushleft}
ເອກະສານນີ້ບັນທຶກການກະກຽມຢ່າງກວ້າງຂວາງຂອງອຳນາດການປົກຄອງຕາເວັນຕົກທີ່ກຳລັງດຳເນີນການເພື່ອກຽມພ້ອມສຳລັບໄພພິບັດທາງທໍລະນີສາດທີ່ພວກເຂົາເຊື່ອວ່າຈະເກີດຂຶ້ນໃນໄວໆນີ້.
\end{abstract}

\section{ອົງການຟຣີແມສັນແລະ "ພາລະກິດອັງໂກ-ແຊັກຊັນ"}
ໃນເດືອນມັງກອນ 2010, ໂຄງການ Camelot, ອົງການສື່ມວນຊົນແລະວາລະສານທາງເລືອກທີ່ລວບລວມຄຳເຫັນຂອງຜູ້ເປີດເຜີຍຄວາມຈິງ, ໄດ້ສໍາພາດ \cite{4,6} ຜູ້ທີ່ມີສ່ວນພາຍໃນທີ່ເຂົ້າຮ່ວມກອງປະຊຸມຂອງຜູ້ໃຫຍ່ Masons ໃນນະຄອນລອນດອນໃນເດືອນມິຖຸນາ 2005. ຫົວຂໍ້ທີ່ຖືກສົນທະນາໃນກອງປະຊຸມແມ່ນແຜນການທາງກອງທັບແລະການເມືອງທີ່ກ່ຽວຂ້ອງກັບພາຍຫຼັງຂອງ \textbf{"ເຫດການທາງທໍລະນີສາດ"}, ຄືວ່າ, ໄພພິບັດທໍາມະຊາດທົ່ວໂລກ.

\begin{figure}[b]
\begin{center}
   \includegraphics[width=1\linewidth]{freemason.jpg}
\end{center}
   \caption{ສະມາຊິກອິດສະລະພາບອັງກິດຢູ່ໃນສະພາບທຳມະຊາດ, ກຳລັງວາງແຜນຢ່າງງຽບໆເພື່ອຈະລົງລູກລະເບີດນິວເຄຼຍ ແລະ ຍຶດຄອງໂລກ - ທີ່ Earls Court ໃນນະຄອນລອນດອນ, ປີ 1992 \cite{5}.}
\label{fig:1}
\label{fig:onecol}
\end{figure}

\begin{figure*}[t]
\begin{center}
\includegraphics[width=1\textwidth]{british.jpg}
\end{center}
   \caption{ຈັກກະພົບອັງກິດໃນປີ 1937, ການສະແດງອຳນາດອັນເຂັ້ມແຂງຂອງຊາວແອງໂກ-ແຊັກຊັນ \cite{14}.}
   \label{fig:2}
\end{figure*}
ຕາມຂໍ້ມູນຈາກພາຍໃນ, ຜູ້ທີ່ເຂົ້າຮ່ວມກອງປະຊຸມ 25-30 ຄົນນັ້ນແມ່ນ \textit{"...ທັງໝົດເປັນຊາວອັງກິດ, ແລະ ບາງຄົນກໍ່ເປັນບຸກຄົນທີ່ມີຊື່ສຽງຫຼາຍທີ່ປະຊາຊົນໃນສະຫະລາຊະອານາຈັກຈະຮູ້ຈັກທັນທີ... ມີທາງດ້ານຊັ້ນສູງຢູ່ໃນນັ້ນ, ແລະ ບາງຄົນກໍ່ມາຈາກຄອບຄົວຊັ້ນສູງ. ມີຫນຶ່ງຄົນທີ່ຂ້າພະເຈົ້າຮູ້ຈັກໃນກອງປະຊຸມນັ້ນເປັນນັກການເມືອງລະດັບສູງ. ອີກສອງຄົນເປັນບຸກຄົນສຳຄັນຈາກຕຳຫຼວດ, ແລະ ອີກຫນຶ່ງຄົນຈາກກອງທັບ. ທັງສອງຄົນເປັນທີ່ຮູ້ຈັກໃນລະດັບຊາດ ແລະ ທັງສອງກໍ່ເປັນບຸກຄົນສຳຄັນໃນການໃຫ້ຄຳປຶກສາແກ່ລັດຖະບານປະຈຸບັນ"} \cite{4}. ພາຍໃນກ່າວວ່າຕົນເຂົ້າຮ່ວມກອງປະຊຸມ, \textit{"ໂດຍບັງເອີນ! ຂ້າພະເຈົ້າຄິດວ່າມັນເປັນກອງປະຊຸມປົກກະຕິທຸກໆສາມເດືອນ... ຂ້າພະເຈົ້າໄດ້ເຂົ້າຮ່ວມກອງປະຊຸມນີ້ ແລະ ມັນບໍ່ແມ່ນກອງປະຊຸມທີ່ຂ້າພະເຈົ້າຄາດຫວັງ. ຂ້າພະເຈົ້າເຊື່ອວ່າຕົນໄດ້ຮັບເຊີນ... ເນື່ອງຈາກຕຳແຫນ່ງທີ່ຂ້າພະເຈົ້າດຳລົງຢູ່ ແລະ ເນື່ອງຈາກພວກເຂົາເຊື່ອວ່າ, ຄືກັນກັບຕົນເອງ, ຂ້າພະເຈົ້າເປັນໜຶ່ງໃນພວກເຂົາ."} \cite{4}.

ເສັ້ນເວລາພື້ນຖານຂອງເຫດການທີ່ໄດ້ຖືກສົນທະນາໃນກອງປະຊຸມ (ໃນປີ 2005) ແມ່ນດັ່ງຕໍ່ໄປນີ້:

\begin{flushleft}
\begin{enumerate}
    \item ການກະຕຸ້ນອີຣານ ຫຼື ຈີນໃຫ້ໃຊ້ອາວຸດນິວເຄລຍຍຸດທະສາດ ແລະ ນຳໄປສູ່ການປະທະກັນນິວເຄລຍທີ່ຈຳກັດ, ແລ້ວຈຶ່ງສ້າງການຢຸດຍິງ.
    \item ການປ່ອຍອາວຸດຊີວະພາບໃສ່ຈີນ, ທີ່ຖືກລາຍງານວ່າເປັນເປົ້າໝາຍຫຼັກ "ນັບຕັ້ງແຕ່ຊຸມປີ 70".
    \item ການນຳສະເໜີລະບອບລັດຖະບານທະຫານທີ່ເປັນສິ່ງທີ່ຖືກຕ້ອງຕາມກົດໝາຍ ຍ້ອນຄວາມຢ້ານກົວ ແລະ ຄວາມວຸ້ນວາຍທີ່ເກີດຂຶ້ນ.
\end{enumerate}
\end{flushleft}

ແຕ່ສິ່ງທີ່ສຳຄັນທີ່ສຸດແມ່ນສິ່ງທີ່ຄາດວ່າຈະເກີດຂຶ້ນຫຼັງຈາກເຫດການເຫຼົ່ານີ້: \textit{"ດັ່ງນັ້ນພວກເຮົາຈະເຂົ້າສູ່ສົງຄາມນີ້, ແລ້ວຫຼັງຈາກນັ້ນ... ຈະມີເຫດການທາງພູມສາດເກີດຂຶ້ນໃນໂລກທີ່ຈະສົ່ງຜົນກະທົບໃຫ້ທຸກຄົນ"} \cite{4}. ຜູ້ຮູ້ພາຍໃນເຊື່ອວ່າໃນໄລຍະເຫດການທາງພູມສາດນີ້, \textit{\textbf{"ເສັ້ນຂອບໂລກຈະເລື່ອນປະມານ 30 ອົງສາ, ປະມານ 1700 ຫາ 2000 ໄມລ໌ໄປທາງໃຕ້, ແລະຈະກໍ່ໃຫ້ເກີດການປ່ຽນແປງຢ່າງໃຫຍ່ຫຼວງ, ຜົນກະທົບຂອງມັນຈະຍືນຍາຍຕໍ່ໄປອີກດົນນານ"}} \cite{4}.

ເຫດຜົນຂອງການວາງແຜນລັບເຫຼົ່ານີ້ແມ່ນ, ແນ່ນອນ, ອຳນາດ. ຜູ້ຮູ້ພາຍໃນອະທິບາຍວ່າ, \textit{"ໃນເວລານັ້ນ, ພວກເຮົາທຸກຄົນຈະຜ່ານສົງຄາມນິວເຄຼຍແລະຊີວະພາບ. ປະຊາກອນໂລກ, ຖ້າສິ່ງນີ້ເກີດຂຶ້ນ, ຈະຫຼຸດລົງຢ່າງຮຸນແຮງ. ເມື່ອເຫດການທາງພູມສາດນີ້ເກີດຂຶ້ນ, ຜູ້ທີ່ເຫຼືອຢູ່ອາດຈະຫຼຸດລົງເຄິ່ງໜຶ່ງອີກ. ແລະຜູ້ທີ່ລອດຊີວິດຈະກຳນົດວ່າໃຜຈະນຳໂລກແລະປະຊາກອນທີ່ເຫຼືອເຂົ້າສູ່ຍຸກຕໍ່ໄປ. ດັ່ງນັ້ນພວກເຮົາກຳລັງເວົ້າເຖິງຍຸກຫຼັງໄພພິບັດ. ໃຜຈະເປັນຜູ້ຄວບຄຸມ? ໃຜຈະເປັນຜູ້ມີອຳນາດ? ດັ່ງນັ້ນມັນກ່ຽວຂ້ອງກັບສິ່ງນັ້ນ. ແລະນັ້ນແມ່ນເຫດຜົນທີ່ພວກເຂົາຢາກໃຫ້ສິ່ງເຫຼົ່ານີ້ເກີດຂຶ້ນໃນໄລຍະເວລາທີ່ກຳນົດໄວ້... ຕ້ອງມີໂຄງສ້າງທີ່ພ້ອມກ່ອນທີ່ [ຄວາມວຸ້ນວາຍ] ຈະເກີດຂຶ້ນ ແລະ ມີຄວາມແນ່ນອນວ່າມັນຈະລອດຈາກສິ່ງທີ່ຈະເກີດຂຶ້ນ - ເພື່ອວ່າມັນຈະສາມາດຢືນຢູ່ໄດ້ໃນມື້ຕໍ່ມາ, ແລະ ຮັກສາອຳນາດທີ່ມັນເຄີຍມີມາໄດ້"} \cite{4}. ໃນລະຫວ່າງການສຳພາດ, ຊື່ຂອງແຜນການນີ້, "ພາລະກິດອັງກິດ-ແຊັກຊັນ", ກໍ່ໄດ້ຖືກສົນທະນາ: \textit{[ຜູ້ສຳພາດ]: "...ເຫດຜົນທີ່ມັນຖືກເອີ້ນວ່າ ພາລະກິດອັງກິດ-ແຊັກຊັນ ແມ່ນເພາະວ່າແຜນການນີ້ມີຈຸດປະສົງເພື່ອລົບລ້າງຊາວຈີນ ເພື່ອວ່າຫຼັງຈາກໄພພິບັດ ແລະ ເມື່ອສິ່ງຕ່າງໆຖືກກໍ່ສ້າງຂຶ້ນມາໃໝ່, ມັນຈະເປັນຊາວອັງກິດ-ແຊັກຊັນທີ່ຈະມີຕຳແໜ່ງໃນການກໍ່ສ້າງ ແລະ ສືບທອດໂລກໃໝ່, ໂດຍບໍ່ມີໃຜອື່ນເຫຼືອຢູ່. ຖືກຕ້ອງບໍ?" [ຜູ້ຮູ້ພາຍໃນ]: "ບໍ່ວ່າຈະຖືກຕ້ອງຫຼືບໍ່ ຂ້ອຍບໍ່ຮູ້ແນ່ນອນ, ແຕ່ຂ້ອຍຄົງເຫັນດີນຳເຈົ້າ. ຕະຫຼອດສັດຕະວັດທີ 20 ຢ່າງໜ້ອຍ, ແລະ ແມ້ແຕ່ກ່ອນໜ້ານັ້ນເຖິງສັດຕະວັດທີ 19 ແລະ 18, ປະຫວັດສາດຂອງໂລກນີ້ໄດ້ຖືກຄວບຄຸມໂດຍສ່ວນໃຫຍ່ໂດຍຕາເວັນຕົກ ແລະ ພາກພື້ນເໜືອຂອງໂລກ"} \cite{4}.
Regarding the exact timeframe of the expected geophysical event, the insider offers his best guess: \textit{"...ຄວາມຮູ້ສຶກ, ແລະມັນເປັນສິ່ງທີ່ອີງໃສ່ສະຕິພາຍໃນຫຼາຍ, ແມ່ນວ່າພວກເຂົາຕ້ອງກຽມພ້ອມໃນຕອນນີ້... ຂ້ອຍຄິດວ່າພວກເຂົາມີຄວາມຄິດທີ່ດີກ່ຽວກັບເວລາທີ່ມັນຈະເກີດຂຶ້ນ... \textbf{ຂ້ອຍມີຄວາມຮູ້ສຶກຢ່າງເຂັ້ມແຂງວ່າມັນຈະເກີດຂຶ້ນໃນຊີວິດຂອງຂ້ອຍ, ປະມານ 20 ປີຂ້າງໜ້າ}... ພວກເຮົາໄດ້ເຂົ້າສູ່ຊ່ວງເວລານັ້ນແລ້ວທີ່ກໍລະນີທາງພູມສາດນີ້ກຳລັງຈະເກີດຂຶ້ນ, ເມື່ອພວກເຮົາພິຈາລະນາເຖິງໄລຍະເວລາທີ່ຜ່ານມາແຕ່ຄັ້ງສຸດທ້າຍທີ່ເກີດຂຶ້ນປະມານ 11,500 ປີກ່ອນ, ແລະມັນເກີດຂຶ້ນທຸກໆ 11,500 ປີ, ເປັນວົງຈອນ. ດຽວນີ້ມັນກຳລັງຈະເກີດຂຶ້ນອີກ... ພວກເຂົາເຂົ້າໃຈວ່າມັນຈະເກີດຂຶ້ນ. ພວກເຂົາມີຄວາມແນ່ໃຈໃນຄວາມຮູ້ວ່າມັນຈະເກີດຂຶ້ນ... ອີກເທື່ອໜຶ່ງ, ມັນເປັນເລື່ອງໜຶ່ງ - ມັນຈະເປັນເລື່ອງທີ່ບໍ່ສາມາດຈິນຕະນາການໄດ້ຖ້າພວກເຂົາບໍ່ຮູ້. ຂ້ອຍຫມາຍຄວາມວ່າ, ສະໝອງທີ່ດີທີ່ສຸດໃນໂລກຈະກຳລັງເຮັດວຽກໃຫ້ພວກເຂົາກ່ຽວກັບເລື່ອງນີ້"} \cite{4}.

This is a powerful testimony for which we should be very grateful. In the interview, the author also discusses his belief that WWI and WWII were manufactured wars, and that the Anglo-Saxon Mission almost certainly dates back many, many generations. It has now been 15 years since the interview, which occurred in 2010. There are five years remaining until the insider's stated 20-year timeframe prediction for the geophysical event reaches its end.
\subsection{ຄວາມຮູ້ອັນລຶກລັບຂອງພວກ Druidic ກ່ຽວກັບໄພພິບັດ}

ຄວາມຮູ້ຕາເວັນຕົກກ່ຽວກັບໄພພິບັດທີ່ເກີດຂຶ້ນຊ້ຳໆ ໄດ້ຖືກຮັກສາໄວ້ຢ່າງດີ, ແລະ ບໍ່ແມ່ນພຽງແຕ່ໂດຍ Freemasons. ພວກ Druids, ເຊິ່ງເປັນວັດທະນະທຳ Celtic ແຕ່ດຶກດຳບັນທີ່ມີຫຼັກຖານເຫັນຊັດເກືອບ 2400 ປີ \cite{7}, ໄດ້ສືບທອດຄວາມຮູ້ກ່ຽວກັບໄພພິບັດຊ້ຳໆຂອງໂລກ. Druid ຄົນສຸດທ້າຍທີ່ຮູ້ຈັກກັນແມ່ນ Ben McBrady. ໃນສາລະຄະດີ "The Last Druid" ປີ 1992, ລາວໄດ້ແບ່ງປັນຂໍ້ມູນກ່ຽວກັບຄວາມຮູ້ຂອງພວກ Druids: \textit{"ອົງການທີ່ຂ້າພະເຈົ້າອາດຈະເປັນສະມາຊິກສຸດທ້າຍຕາມປະເພນີ, ໄດ້ກໍ່ຕັ້ງຂຶ້ນຫຼັງຈາກໄພພິບັດຄັ້ງໃຫຍ່ຫຼືໄພພິບັດຄັ້ງສຸດທ້າຍທີ່ສົ່ງຜົນກະທົບຕໍ່ໂລກ. ດ້ວຍຜົນກະທົບອັນໃຫຍ່ຫຼວງແລະຫນ້າຢ້ານຕໍ່ໂລກຈາກພາຍຸໄຟຟ້າອັນຮຸນແຮງ, ຖືກຈັບໄວ້ໃນຫາງຂອງດາວຕົກ ຫຼື ຝົນດາວຕົກ, ອາລະຍະທຳດັ່ງທີ່ພວກເຮົາຮູ້ຈັກຖືກທຳລາຍຢ່າງສິ້ນເຊີງ... ຄວາມຮູ້ທັງໝົດຢູ່ໃນຂອບເຂດຂອງອົງການ, ແຕ່ພວກເຂົາເຈົ້າສົນໃຈເປັນພິເສດກ່ຽວກັບດາລາສາດເພາະວ່າພວກເຂົາເຈົ້າໄດ້ປະສົບກັບໄພພິບັດທີ່ສຳຄັນຫຼາຍຄັ້ງ. ມັນໄດ້ຖືກຄິດວ່າຄວາມຮູ້ຢ່າງເຕັມທີ່ກ່ຽວກັບດາລາສາດຈະຊ່ວຍໃຫ້ພວກເຂົາເຈົ້າສາມາດຄາດເດົາເງື່ອນໄຂເມື່ອໄພພິບັດເຫຼົ່ານີ້ອາດຈະເກີດຂຶ້ນແລະດຳເນີນການບາງຢ່າງເພື່ອປ້ອງກັນຕົວເອງ. ຖ້າທ່ານເບິ່ງສະລອຍເຫຼັກອັນໃຫຍ່ຫຼວງໃນປະເທດໄອແລນ, ທ່ານຈະເຫັນວ່າສິ່ງທີ່ຖືກອະທິບາຍວ່າເປັນຫຼວງຜ່ານແມ່ນຕົວຈິງແລ້ວເປັນທີ່ປິດລັງກະສັນຂອງປະເພດດັ້ງເດີມ. ພວກມັນຢູ່ເທິງລະດັບນ້ຳຖ້ວມໃດໆ ແລະ ຍັງໃຫ້ການປ້ອງກັນຈາກຝົນດາວຕົກ"} \cite{8,9}.

% ຍັງເຊື່ອກັນວ່າ Freemasonry ເອງກໍ່ມີຕົ້ນກຳເນີດຈາກພວກ Druids \cite{10}.
\section{ຫຼັກຖານຂອງການກຽມພ້ອມສຳລັບໄພພິບັດທາງທຳມະຊາດຂອງຕາເວັນຕົກໃນປະຈຸບັນ}

ໃນເມື່ອອຳນາດການປົກຄອງຂອງຕາເວັນຕົກເຊື່ອວ່າໄພພິບັດທາງທຳມະຊາດທີ່ຮ້າຍແຮງກຳລັງຈະເກີດຂຶ້ນໃນໄວໆນີ້, ເຮົາຄາດຫວັງວ່າຈະມີການກຽມພ້ອມຢ່າງຫຼວງຫຼາຍເພື່ອປົກປ້ອງຕົນເອງຈາກເຫດການດັ່ງກ່າວ. ແລະແທ້ຈິງແລ້ວ, ມີຫຼັກຖານໃນຂອບເຂດສາທາລະນະທີ່ສະແດງໃຫ້ເຫັນວ່າມີເຄືອຂ່າຍຂອງຖານລັກສະນະເລິກໃຕ້ດິນທີ່ກວ້າງຂວາງໃນຫຼາຍປະເທດຕາເວັນຕົກ. ເຖິງແມ່ນວ່າສິ່ງກໍ່ສ້າງເຫຼົ່ານີ້ຈະສາມາດປົກປ້ອງຜູ້ທີ່ຢູ່ອາໄສໃນສົງຄາມນິວເຄລຍ, ແຕ່ມັນຍັງສາມາດໃຊ້ເປັນການປ້ອງກັນຈາກໄພພິບັດທາງທຳມະຊາດປະເພດຕ່າງໆ. ຈາກຄຳຢັ້ງຢືນຂອງອາຈານເສດຖີອັງກິດຈາກໂຄງການ Camelot \cite{4,6}, ມັນເບິ່ງຄືວ່າສະຖານະການເຫຼົ່ານີ້ບໍ່ແມ່ນເປັນໄປໄດ້, ແຕ່ເປັນແຜນການທີ່ຖືກກຳນົດໄວ້ລ່ວງໜ້າ. ນອກຈາກນີ້, ຍັງມີຈຳນວນເງິນມະຫາສານທີ່ຈຳເປັນໃນການກໍ່ສ້າງ, ຈັດຕັ້ງບຸກຄະລາກອນ, ແລະຮັກສາຖານເຫຼົ່ານີ້, ເຊິ່ງສອດຄ່ອງກັບຈຳນວນເງິນຫຼາຍສິບລ້ານລ້ານໂດລາທີ່ຫາຍໄປຈາກລັດຖະບານສະຫະລັດເປັນເວລາ 18 ປີ (ຈະກ່າວເຖິງໃນພາກຕໍ່ໄປ) \cite{11,12,13}. ຕົວຢ່າງອື່ນໆຂອງການກຽມພ້ອມສຳລັບເຫດການລະດັບສູນພັນລວມເຖິງໂຄງການສະຖາບັນຕ່າງໆເຊັ່ນ: ຄັງເກັບມັດທະຍົມແລະຄວາມຮູ້.
\subsection{ຖານທີ່ຢູ່ເບື້ອງລຸ່ມດິນ ແລະ ເບື້ອງລຸ່ມນ້ຳຂອງອາເມຣິກາ}

ການສຶກສາຄົ້ນຄວ້າສາທາລະນະທີ່ກວ້າງຂວາງທີ່ສຸດກ່ຽວກັບຖານທີ່ຢູ່ເບື້ອງລຸ່ມດິນທີ່ຂ້າພະເຈົ້າໄດ້ພົບເຫັນແມ່ນມາຈາກ Richard Sauder, ນັກຄົ້ນຄວ້າອິດສະຫຼະຊາວອາເມຣິກັນຜູ້ທີ່ໄດ້ພິມປຶ້ມຫຼາຍຫົວກ່ຽວກັບຖານທີ່ຢູ່ເບື້ອງລຸ່ມດິນລຶ່ມ \cite{22}. ຜົນງານຂອງ Sauder ປະກອບດ້ວຍການບັນທຶກເອກະສານ ແລະ ແຜນການຂອງລັດຖະບານ, ການສຶກສາເຫດການ ແລະ ເຕັກໂນໂລຊີທັງໃນອະດີດ ແລະ ປະຈຸບັນ, ການສ້າງແຫຼ່ງຂໍ້ມູນ, ແລະ ການລວບລວມຄຳກ່າວອ້າງຈາກຜູ້ຮູ້ພາຍໃນ. ການຄົ້ນຄວ້າຂອງ Sauder ໄດ້ເປີດເຜີຍວ່າມີເຄືອຂ່າຍຖານທີ່ຢູ່ເບື້ອງລຸ່ມດິນ ແລະ ເບື້ອງລຸ່ມນ້ຳຂະຫນາດໃຫຍ່ໃນ ແລະ ອ້ອມຮອບອາເມຣິກາ ແລະ ດິນແດນຂອງມັນ (ຮູບ \ref{fig:4}), ມີຄວາມລຶ່ມເຖິງ 3 ໄມລ໌, ແລະ ອາດຈະເຊື່ອມຕໍ່ກັນດ້ວຍລົດໄຟໄມໂຄລີຟສູງທີ່ໃຊ້ທໍ່ສູນຍາກາດ. ຖານເຫຼົ່ານີ້ໄດ້ຮັບທຶນຢ່າງລັບລືຜ່ານການ \textit{"ການເງິນລະດັບສູງ, ລະຫວ່າງປະເທດ, ລະຫວ່າງອົງການ, ເກມລ້າງເງິນ"} ທີ່ດຳເນີນໂດຍກຸ່ມຄົນດຽວກັນກັບຜູ້ທີ່ເປັນເຈົ້າຂອງບໍລິສັດສະຫະລັດອາເມຣິກາ \cite{22}. ຜົນງານຕໍ່ເນື່ອງກ່ຽວກັບຂອບເຂດຂອງຖານເຫຼົ່ານີ້ໂດຍ Catherine Austin Fitts (ຜົນງານຂອງນາງຈະຖືກກວມເອົາໃນພາກຕໍ່ໄປ) ແລະ ຜູ້ຮ່ວມງານຂອງນາງໄດ້ຄາດຄະເນວ່າມີຖານທີ່ຢູ່ເບື້ອງລຸ່ມດິນ ແລະ ເບື້ອງລຸ່ມນ້ຳຂອງອາເມຣິກາຈຳນວນ 170 ແຫ່ງ \cite{16,20}.

\begin{figure}[b]
\begin{center}
% \fbox{\rule{0pt}{2in} \rule{0.9\linewidth}{0pt}}
   \includegraphics[width=1\linewidth]{penta.jpg}
\end{center}
   \caption{ສິ່ງທີ່ຕົວຈິງແລ້ວຢູ່ໃຕ້ພະລາຊະວັງຂາວ ແລະ ຫໍປະຊຸມ Pentagon? ຫຼັກຖານສະແດງໃຫ້ເຫັນວ່າ, ມີເຄືອຂ່າຍອຸໂມງໃຕ້ດິນທີ່ເລິກ (ຮູບພາບ: \cite{31}).}
\label{fig:3}
\label{fig:onecol}
\end{figure}

\begin{figure*}[t]
\begin{center}
% \fbox{\rule{0pt}{2in} \rule{.9\linewidth}{0pt}}
\includegraphics[width=0.9\textwidth]{basescrop.png}
\end{center}
\caption{ແຜນທີ່ສະແດງໃຫ້ເຫັນສະຖານທີ່ທີ່ແນ່ນອນທີ່ຜົນງານຄົ້ນຄວ້າຂອງ Sauder ໄດ້ເປີດເຜີຍວ່າມີຖານທະຫານໃຕ້ດິນ ແລະ ໃຕ້ທະເລ, ພ້ອມທັງອ່າງເຮືອດຳນ້ຳທີ່ມີທາງຜ່ານເຂົ້າໄປຍັງພາຍໃນທະເລ. Sauder ໄດ້ກ່າວວ່າ \textit{"ແນ່ໃຈວ່າຍັງມີສະຖານທີ່ອື່ນໆອີກຫຼາຍແຫ່ງທີ່ຍັງບໍ່ໄດ້ລະບຸ"} \cite{22}.}
\label{fig:4}
\end{figure*}

ນີ້ແມ່ນບາງຕົວຢ່າງຈາກແຫຼ່ງຂໍ້ມູນຂອງ Sauder ທີ່ອະທິບາຍລາຍລະອຽດຂອງຖານທະຫານເຫຼົ່ານີ້:
\begin{flushleft}
\begin{enumerate}
    \item ຄໍາແຄມປ໌ ເດວິດ, ແມຣີແລນ: \textit{"ແຫຼ່ງຂໍ້ມູນຂອງຂ້າພະເຈົ້າໄດ້ແຈ້ງໃຫ້ຂ້າພະເຈົ້າຮູ້ວ່າສ່ວນໃຕ້ດິນຂອງຄໍາແຄມປ໌ ເດວິດນັ້ນກວ້າງຂວາງ ແລະ ສະຫຼັບຊັບຊ້ອນຫຼາຍ, ແລະ ມີຂຸມລັບຫຼາຍໄມລ໌, ຈົນບໍ່ແນ່ນອນວ່າຈະມີຜູ້ໃດຜູ້ໜຶ່ງທີ່ຈະມີແຜນທີ່ສົມບູນຂອງສະຖານທີ່ນີ້ໃນຈິດໃຈຂອງຕົນ"} \cite{22}.
    \item ເຮືອນຂາວ, ວໍຊິງຕັນ ດີຊີ: \textit{"ໝູ່ສະໜິດຂອງຂ້າພະເຈົ້າຄົນໜຶ່ງຖືກນຳໄປຍັງສະຖານທີ່ນີ້ໃນຊ່ວງລັດຖະບານລິນດອນ ບີ. ຈອນສັນໃນຊຸມປີ 1960. ນາງໄດ້ເຂົ້າລິບຕຶກໃນເຮືອນຂາວ ແລະ ຖືກນຳພາລົງໄປດ້ວຍທາງຊື່. ນາງເຊື່ອວ່າລິບຕຶກນັ້ນລົງໄປຮອດ 17 ຊັ້ນ. ເມື່ອປະຕູເປີດອອກຢູ່ໃຕ້ດິນ, ນາງຖືກນຳພາໄປຕາມທາງຍ່າງທີ່ເບິ່ງຄືວ່າຈະຫາຍໄປໃນຈຸດທີ່ບໍ່ສາມາດເຫັນໄດ້ໃນທາງໄກ. ມີປະຕູ ແລະ ທາງຍ່າງອື່ນໆ ເປີດອອກຈາກທາງຍ່າງນັ້ນ"} \cite{22}. ສະແດງໃນຮູບ \ref{fig:3}.
    \item ຟອດເມດ, ແມຣີແລນ - ຈາກແຫຼ່ງຂໍ້ມູນຜູ້ໜຶ່ງທີ່ໄດ້ເຂົ້າໄປໃນ "ຫ້ອງໃຕ້ດິນ" ໂດຍບັງເອີນໃນຊຸມປີ 1970: \textit{"ຂ້ອຍເປີດປະຕູ ແລະ ເຫັນບັນໄດທີ່ນຳໄປລຸ່ມ. ຂ້ອຍເດີນໄປທີ່ຂອບແລະເບິ່ງລົງໄປລະຫວ່າງລະບົບກັນ. ຂ້ອຍບໍ່ໄດ້ນັບຈຳນວນຊັ້ນທີ່ຢູ່ລຸ່ມ, ແຕ່ຂ້ອຍຮູ້ສຶກວ່າມັນປະມານ 15-20 ຊັ້ນ... ຂ້ອຍເດີນລົງໄປອີກຊັ້ນໜຶ່ງ ແລະ ເຫັນປະຕູ... ຂ້ອຍເປີດປະຕູແລະເບິ່ງເຂົ້າໄປ ເຫັນທາງຍ່ອຍທີ່ຍາວໄປຈົນເບິ່ງບໍ່ເຫັນທັງສອງທາງ. ມັນແນ່ນອນວ່າຍາວກວ່າເນື້ອທີ່ຂອງຕຶກ ແລະ ບ່ອນຈອດລົດທີ່ຢູ່ພື້ນດິນ. ມີປະຕູຕາມຝາເຊິ່ງຫ່າງກັນປະມານ 30-40 ຟຸດ... ຂ້ອຍຕັດສິນໃຈລົງໄປອີກສອງສາມຊັ້ນ ຈຶ່ງເດີນລົງໄປອີກຊັ້ນໜຶ່ງ... ແລະ ເຫັນຮູບແບບດຽວກັນ... ຂ້ອຍລົງໄປອີກຊັ້ນໜຶ່ງ ແລະ ເບິ່ງເຂົ້າໄປ ເຫັນສິ່ງດຽວກັນກັບສອງຊັ້ນທຳອິດ"} \cite{22}.
\end{enumerate}
\end{flushleft}

\begin{figure}[t]
\begin{center}
% \fbox{\rule{0pt}{2in} \rule{0.9\linewidth}{0pt}}
   \includegraphics[width=1\linewidth]{undersea.jpg}
\end{center}
   \caption{ພາບສະແດງຖານຖານໃຕ້ທະເລ, ໂດຍ Walter Koerschner. ລາວເປັນນັກແຕ້ມຮູບໃຫ້ກັບທີມງານຖານຖານໃຕ້ທະເລ Rock-Site ຂອງກອງທັບເຮືອສະຫະລັດ ທີ່ສູນອາວຸດ China Lake, ລັດ California ໃນຊຸມປີ 1960. ແຫຼ່ງຂໍ້ມູນຂອງ Sauder ເປີດເຜີຍວ່າມີຖານຖານໃຕ້ດິນທີ່ລຶກລັບລົງໄປເຖິງ 1 ໄມລ໌ ຢູ່ China Lake \cite{22,23}.}
\label{fig:5}
\label{fig:onecol}
\end{figure}

Sauder ຍັງໄດ້ຮັບຄຳເຫັນກ່ຽວກັບລົດໄຟແມ່ເຫຼັກລອຍໄດ້ເຊິ່ງສາມາດບັນທຸກຄວາມໄວໄດ້ເຖິງ 2,000 ໄມຕໍ່ຊົ່ວໂວ, ຖານທີ່ຖືກສ້າງໄວ້ເບື້ອງລຸ່ມຂອງຊັ້ນດິນທະເລ (ຮູບ \ref{fig:5}), ແລະ ອຸໂມງຍ່ານເຮືອດຳນໍ້າທີ່ນໍາໄປສູ່ພື້ນດິນ. ກ່ຽວກັບຄຳເຫັນຫນຶ່ງກ່ຽວກັບຖານດຳນໍ້າໃນອ່າວເມັກຊິໂກ, Sauder ເວົ້າວ່າ, \textit{"ປະມານເຄິ່ງປີຫລັງຈາກການພິມຈໍາໜ່າຍຂອງ Underwater and Underground Bases, ຂ້າພະເຈົ້າໄດ້ຮັບການຕິດຕໍ່ຈາກຜູ້ຊາຍຜູ້ຫນຶ່ງທີ່ເວົ້າວ່າລາວມີຄວາມຮູ້ກ່ຽວກັບໂຄງການດຳນໍ້າທີ່ຜິດປົກກະຕິ... ລາວລະບຸວ່າໂຄງການນີ້ຢູ່ເບື້ອງລຸ່ມຂອງຊັ້ນດິນທະເລຂອງອ່າວເມັກຊິໂກ, ແລະ ວ່າ Parsons ແມ່ນຜູ້ຮັບເຫັນ້ວຍ. ລາວເວົ້າຕໍ່ໄປວ່າ Parsons ໄດ້ຊື້ອຸປະກອນພິເສດບາງຢ່າງທີ່ມີຈຸດປະສົງໃຫ້ເຮັດວຽກໄດ້ເບື້ອງລຸ່ມຊັ້ນດິນທະເລ 2,800 ຟຸດ... ອຸປະກອນນີ້ມີຄວາມເປັນເອກະລັກພຽງພໍທີ່ຈະສະແດງໃຫ້ເຫັນວ່າມີມະນຸດທີ່ມີຊີວິດຢູ່ໃນສະຖານທີ່ທີ່ມັນຖືກຕິດຕັ້ງ"} \cite{22}.
\begin{figure}[t]
\begin{center}
% \fbox{\rule{0pt}{2in} \rule{0.9\linewidth}{0pt}}
   \includegraphics[width=1\linewidth]{sub.jpg}
\end{center}
   \caption{ການສະແດງຮູບພາບຂອງອຸໂມງຍ່ານໍ້າຢູ່ໃຕ້ນໍ້າ, ໂດຍ Walter Koerschner \cite{22,23}.}
\label{fig:6}
\label{fig:onecol}
\end{figure}
\begin{figure}[t]
\begin{center}
% \fbox{\rule{0pt}{2in} \rule{0.9\linewidth}{0pt}}
   \includegraphics[width=1\linewidth]{iran.jpeg}
\end{center}
   \caption{ຄລິບຈາກວິດີໂອທາງການຂອງອີຣານ ທີ່ສະແດງໃຫ້ເຫັນ "ເມືອງຈູດລູກສອນ" ໃຕ້ດິນຂອງພວກເຂົາ \cite{39,40}.}
\label{fig:12}
\label{fig:onecol}
\end{figure}
ຖ້າຫາກວ່າມີເຄືອຂ່າຍລັບລັບຂອງຖານທີ່ຢູ່ເທິງດິນ ແລະ ໃຕ້ທະເລ ຫຼາຍກວ່າ 170 ແຫ່ງ ທີ່ຂຸດລົງໄປເຖິງຄວາມເລິກຫຼາຍໄມ່ ເຊິ່ງເຊື່ອມຕໍ່ກັນດ້ວຍລົດໄຟໄຟຟ້າແມ່ເຫຼັກໄຟຟ້າໃນທໍ່ສູນຍາກາດ ແລະ ໄດ້ຮັບທຶນຈາກຜົນງານຂອງພວກເຮົາ, ມວນມະນຸດໃນປະຈຸບັນຈະຢູ່ໃນສະພາບຂອງຄວາມເປັນຕາຍ ແລະ ຄວາມເບີກບານໂດຍບໍ່ຮູ້ຕົວ, ບໍ່ພຽງແຕ່ບໍ່ຮູ້ວ່າມີຫຍັງຢູ່ໃຕ້ຕີນຂອງພວກເຂົາ ແຕ່ຍັງບໍ່ຮູ້ວ່າຫຍັງຈະເກີດຂຶ້ນໃນອະນາຄົດອັນໃກ້ນີ້, ໃນຂະນະທີ່ພວກເຂົາກຳລັງດື່ມດື່ມຄຳເວົ້າທີ່ເປັນຮູບການ ແລະ ປະສານງານຈາກນັກການເມືອງທີ່ຄວບຄຸມພວກເຂົາ.

ຫມາຍເຫດເພີ່ມເຕີມ - ການມີຢູ່ຂອງເຄືອຂ່າຍອ່າງໃຕ້ດິນຂະໜາດໃຫຍ່ໄດ້ຖືກເປີດເຜີຍໂດຍບໍ່ມີຂໍ້ສົງໄສໃນຄວາມຂັດແຍ້ງທີ່ຍັງຄົງຢູ່ໃນຕາເວັນອອກກາງ (ອ່າງຂອງ Hamas ໃຕ້ແຜ່ນດິນ Gaza \cite{38}, ແລະ "ເມືອງຈູດຕະຫຼດ" ໃຕ້ດິນຂອງອີຣ່ານ (ຮູບ \ref{fig:12}) \cite{39,40}). ເຫຼົ່ານີ້ຄວນຈະບໍ່ເຮັດໃຫ້ເກີດຄວາມສົງໃສໃນທາງເລືອກທີ່ຈະກໍ່ສ້າງ ແລະ ການມີຢູ່ຕົວຈິງຂອງໂຄງສ້າງດັ່ງກ່າວ. ພວກມັນຍັງຄວນຈະເຮັດໃຫ້ພວກເຮົາສົງໄສວ່າປະເທດອື່ນໆທີ່ມີທຶນຮອນດີກວ່າຫຼາຍອາດຈະໄດ້ກໍ່ສ້າງໂຄງສ້າງໃດແດ່ໃນໄລຍະເວລາດຽວກັນ.
\subsection{ຫຼັກຖານເພີ່ມເຕີມກ່ຽວກັບການກຽມພ້ອມບັນເທີງ ແລະ ໄພພິບັດ}

\begin{figure}[t]
\begin{center}
% \fbox{\rule{0pt}{2in} \rule{0.9\linewidth}{0pt}}
   \includegraphics[width=1\linewidth]{tyrol.jpg}
\end{center}
   \caption{ບັງເກີໃນພາກໃຕ້ຂອງໄທໂຣນ, ສະວິດເຊີແລນ. ສະວິດເຊີແລນ, ທີ່ຕັ້ງຢູ່ໃນສາຍພູແອລບໃນເອີຣົບ, ເປັນທີ່ຮູ້ຈັກດ້ວຍການປອມບັງເກີໃນພູເຂົາຢ່າງສະຫຼາດ \cite{32}.}
\label{fig:7}
\label{fig:onecol}
\end{figure}

\begin{figure}[t]
\begin{center}
% \fbox{\rule{0pt}{2in} \rule{0.9\linewidth}{0pt}}
   \includegraphics[width=1\linewidth]{svalbard.jpg}
\end{center}
   \caption{ຫໍສາມັນແກ່ນພັນໂລກ Svalbard ໃນປະເທດນໍເວ, ມີແກ່ນພັນຫຼາຍກວ່າໜຶ່ງລ້ານແກ່ນ \cite{24}. ເຮົາຕ້ອງສົງໄສວ່າເຫດການໃດຈະຈຳເປັນຕ້ອງໃຊ້ມັນ.}
\label{fig:8}
\label{fig:onecol}
\end{figure}

ມີຫຼັກຖານອື່ນໆອີກຫຼາຍຢ່າງທີ່ສະແດງເຖິງການກຽມພ້ອມສຳລັບໄພພິບັດທົ່ວໂລກ ນອກເໜືອຈາກຖານລາຊະວົງອາເມຣິກັນໃຕ້ດິນ. ປະເທດນໍເວ, ສະວິດເຊີແລນ, ສະວີເດນ, ແລະ ຟິນແລນ ແມ່ນຕົວຢ່າງທີ່ດີ:

\begin{flushleft}
\begin{enumerate}
    \item ໂຄງການ Camelot ໄດ້ແບ່ງປັນຄຳເຫັນທີ່ກ່ຽວຂ້ອງຈາກນັກການເມືອງນໍເວຍຜູ້ໜຶ່ງ \cite{25,26}, ເຊິ່ງພວກເຂົາໄດ້ຢັ້ງຢືນຕົວຕົນແຕ່ບໍ່ເປີດເຜີຍ. ລາວກ່າວຫາວ່ານໍເວຍມີຖານກອງພັນໃຕ້ດິນ 18 ແຫ່ງ, ແລະ ວ່ານໍເວຍ (ພ້ອມກັບອິສຣາແອລແລະ "ຫຼາຍປະເທດອື່ນໆ") ກຳລັງກໍ່ສ້າງຖານເຫຼົ່ານີ້ເພື່ອກຽມພ້ອມຮັບມືກັບໄພພິບັດທາງທຳມະຊາດບາງຢ່າງ. Richard Sauder ກໍ່ໄດ້ຮັບຄຳເຫັນຈາກຜູ້ຊາຍຜູ້ໜຶ່ງທີ່ເຄີຍເຂົ້າໄປໃນຖານກອງພັນໃຕ້ດິນຂະໜາດໃຫຍ່ທີ່ຖືກສ້າງຂຶ້ນໃນພູທີ່ຖືກຂຸດເປັນຊ່ອງໃນນໍເວຍ \cite{22}.
    \item ສະວິດເຊີແລນເປັນທີ່ຮູ້ຈັກດີວ່າມີບ່ອນຫຼົບພັກນິວເຄລຍຫຼາຍແຫ່ງຖືກສ້າງຢູ່ເທິງສູງຂອງສາຍພູ Alps (ຮູບ \ref{fig:7}). ຈຳນວນເຫຼົ່ານີ້ມີຫຼາຍກວ່າ 370,000 ແຫ່ງ - ພຽງພໍທີ່ຈະສາມາດປ້ອງກັນປະຊາຊົນທຸກຄົນ \cite{27}.
    \item ສະວີເດັນແລະຟິນແລນມີບ່ອນຫຼົບພັກພຽງພໍທີ່ຈະປ້ອງກັນປະຊາຊົນໃນທຸກໆເມືອງໃຫຍ່ \cite{27}. 
\end{enumerate}
\end{flushleft}

ຜູ້ຄ້າອາຍຸດຕະກອນໃນ Silicon Valley ກໍ່ມີຄວາມຮູ້ເລື່ອງນີ້ເຊັ່ນກັນ. ຕາມການລາຍງານ, \textit{"Reid Hoffman, ຜູ້ຮ່ວມກໍ່ຕັ້ງ LinkedIn ແລະ ເປັນນັກລົງທຶນຊື່ດັງ, ໄດ້ບອກ The New Yorker ໃນຕົ້ນປີນີ້ວ່າ ລາວປະເມີນວ່າເກືອບ 50\% ຂອງພວກຕະກອນພັນລ້ານ Silicon Valley ໄດ້ຊື້ "ປະກັນໄພວິນາສະຫວັນ" ບາງລະດັບ, ເຊັ່ນ: ຫ້ອງປ້ອງກັນພາຍໃຕ້ດິນ... ຕາມຄຳເວົ້າຂອງ Jim Dobson, ຜູ້ປະກອບສ່ວນໃຫ້ Forbes, ພວກຕະກອນພັນລ້ານຫຼາຍຄົນມີ "ເຮືອບິນສ່ວນຕົວພ້ອມທີ່ຈະເດີນທາງໃນຊ່ວງເວລາສັ້ນໆ" ນອກຈາກນີ້ຍັງມີລົດຈັກ, ອາວຸດ, ແລະເຄື່ອງປັ່ນໄຟ"} \cite{28}.

ນອກຈາກນີ້ຍັງມີໂຄງການສະສົມໃຫຍ່ໆຫຼາຍຢ່າງ ເຊັ່ນ Global Knowledge Vault, ທີ່ດຳເນີນການໂດຍມູນນິທິ Arch Mission Foundation, \cite{29} ແລະ Svalbard Global Seed Vault \cite{30} ເຊິ່ງເບິ່ງຄືວ່າກຳລັງກຽມພ້ອມໃນການອະນຸລັກຊັບສິນທີ່ສຳຄັນຂອງມະນຸດຊາດໃນກໍລະນີທີ່ເກີດໄພພິບັດລະດັບການສູນພັນ.
\begin{figure*}[t]
\begin{center}
% \fbox{\rule{0pt}{2in} \rule{.9\linewidth}{0pt}}
\includegraphics[width=0.9\textwidth]{govcrop2.png}
\end{center}
   \caption{ລາຍຮັບ, ການໃຊ້ຈ່າຍ, ແລະ ການໃຊ້ຈ່າຍຂອງຖານລັບຢູ່ໃຕ້ດິນຂອງລັດຖະບານສະຫະລັດ ຈາກປີ 1998 ຫາ 2023 \cite{19}.}
   \label{fig:9}
\end{figure*}
\section{ກົນໄກການສະໜອງທຶນແບບປະຊາທິປະໄຕສຳລັບຖານກອງກຳລັງໃຕ້ດິນຂະໜາດໃຫຍ່}

ດັ່ງນັ້ນ ເຄືອຂ່າຍຂະໜາດໃຫຍ່ທີ່ຂະຫຍາຍຂ້າມທະວີບ ທີ່ມີຖານກອງກຳລັງໃຕ້ດິນ ແລະ ໃຕ້ທະເລ ຫຼາຍກວ່າ 170 ແຫ່ງ ໄດ້ຮັບທຶນຢ່າງໃດ ໃນຂະນະທີ່ປິດບັງຂໍ້ມູນຈາກປະຊາຊົນຜູ້ຖືກເປັນເອກະຊົນ? ມີຫຼັກຖານເອກະສານຢູ່ປະການໜຶ່ງທີ່ສາມາດໃຫ້ເຮົາເຫັນພາບລວມຂອງຈຳນວນເງິນທີ່ໄປສູ່ໂຄງການເຫຼົ່ານີ້ ແລະ ມາຈາກແຫຼ່ງໃດ. ໃນປີ 2017, Catherine Austin Fitts, ນັກທະນາຄານການລົງທຶນຊາວອາເມຣິກັນ ແລະ ອະດີດພະນັກງານລັດຖະບານສະໄໝປະທານາທິບໍດີ Bush, ແລະ Mark Skidmore, ນັກເສດຖະສາດຈາກມະຫາວິທະຍາໄລ Michigan State, ໄດ້ຄົ້ນພົບການໃຊ້ຈ່າຍທີ່ບໍ່ໄດ້ຮັບອະນຸຍາດຈຳນວນ 21 ພັນຕື້ໂດລາສະຫະລັດ ໃນລັດຖະບານສະຫະລັດ ໃນລະຫວ່າງປີການເງິນ 1998-2015 \cite{11,12,13}.

ຕາມບົດລາຍງານຂອງພວກເຂົາ, \textit{"ໃນວັນທີ 7 ຕຸລາ 2016 ນະວະສານ Reuters ໄດ້ພິມບົດຄວາມໂດຍ Scot Paltrow (2016), ທີ່ລາຍງານວ່າໃນປີການເງິນ 2015 ກອງທັບສະຫະລັດໄດ້ເຮັດການປັບບັນຊີທີ່ບໍ່ມີຫຼັກຖານສະໜັບສະໜູນຈຳນວນ 6.5 ພັນຕື້ໂດລາ "ເພື່ອສ້າງຠອກວ່າບັນຊີຂອງຕົນມີຄວາມສົມດຸນ." ໃນເມື່ອພິຈາລະນາວ່າໃບງົບປະມານທົ່ວໄປຂອງກອງທັບໃນປີນັ້ນແມ່ນ 122 ຕື້ໂດລາ, ນີ້ແມ່ນການເປີດເຜີຍທີ່ໜ້າຕົກໃຈ... ກະຊວງປ້ອງກັນປະເທດ (DOD) ໄດ້ເປັນຂ່າວຫົວຂໍ້ໃຫຍ່ໃນສື່ມວນຊົນຫຼາຍປີກ່ອນໜ້ານີ້ ໃນວັນທີ 10 ກັນຍາ 2001 ເມື່ອລັດຖະມົນຕີກະຊວງປ້ອງກັນປະເທດ Donald Rumsfeld ໄດ້ກ່າວໃນການສົນທະນາກັບລັດຖະສະພາ (C-SPAN, 2014) ວ່າ DOD ໄດ້ສູນເສຍຕິດຕາມການເຄື່ອນໄຫວທາງດ້ານການເງິນຈຳນວນ 2.3 ພັນຕື້ໂດລາ... ການຍອມຮັບນີ້ໄດ້ເປັນຂ່າວໃນມື້ນັ້ນ, ແຕ່ຖືກລືມໃນມື້ຕໍ່ມາເມື່ອເຫດການໂສກເສົ້າ 9/11 ໄດ້ດຶງດູດຄວາມສົນໃຈຂອງທົ່ວໂລກ... ເມື່ອອາຈານ Mark Skidmore ໄດ້ຮູ້ກ່ຽວກັບການເຄື່ອນໄຫວທາງການເງິນທີ່ບໍ່ສາມາດຢັ້ງຢືນໄດ້ຈຳນວນ 6.5 ພັນຕື້ໂດລາຂອງກອງທັບ, ລາວໄດ້ຕິດຕໍ່ກັບ Ms. Fitts ແລະ ພວກເຂົາໄດ້ຕົກລົງກັນໃນລະດູບານປີ 2017 ທີ່ຈະຮ່ວມມືກັນຄົ້ນຫາບົດລາຍງານອື່ນໆຂອງລັດຖະບານທີ່ຊີ້ບອກເຖິງການເຄື່ອນໄຫວທາງການເງິນທີ່ບໍ່ສາມາດຢັ້ງຢືນໄດ້ຈຳນວນຫຼາຍຢູ່ໃນ HUD ແລະ DOD. ໃນໄລຍະເວລາ 6 ເດືອນຕໍ່ມາ, Skidmore, Fitts ແລະ ທີມນັກສຶກສາປະລິນຍາໂທຈຳນວນໜຶ່ງໄດ້ລວບລວມເອກະສານລັດຖະບານທາງການທີ່ສະແດງໃຫ້ເຫັນການເຄື່ອນໄຫວທາງການເງິນທີ່ບໍ່ມີຫຼັກຖານສະໜັບສະໜູນທັງໝົດ 21 ພັນຕື້ໂດລາ ໃນລະຫວ່າງໄລຍະ 1998-2016"} \cite{12}.
ໃນໄລຍະເວລາ 18 ປີດຽວກັນຈາກປີ 1998-2015, ລາຍຮັບຂອງລັດຖະບານສະຫະລັດທີ່ຮັບຮູ້ຢ່າງເປີດເຜີຍແມ່ນມີພຽງ 40.8 ລ້ານລ້ານໂດລາ \cite{15}, ຊຶ່ງສະແດງໃຫ້ເຫັນວ່າມີມູນຄ່າຫຼາຍກວ່າເຄິ່ງໜຶ່ງຂອງລາຍຮັບລັດຖະບານສະຫະລັດຖືກໃຊ້ຈ່າຍຢ່າງລັບໆໃນຖານທີ່ຢູ່ໃຕ້ດິນນອກຈາກການໃຊ້ຈ່າຍຂອງລັດຖະບານສະຫະລັດທີ່ຮັບຮູ້ຢ່າງເປີດເຜີຍ. ສິ່ງທີ່ສັງເກດໄດ້ອີກຢ່າງໜຶ່ງແມ່ນວ່າການໃຊ້ຈ່າຍລັບໆນີ້ເກີດຂຶ້ນເທິງສະພາບການຂາດດຸນງົບປະມານທີ່ດຳເນີນມາດົນນານ, ແລະຄາດວ່າບໍ່ພຽງແຕ່ຍັງດຳເນີນຕໍ່ມາຈົນເຖິງມື້ນີ້ເທົ່ານັ້ນ ແຕ່ຍັງເຄີຍມີມາກ່ອນປີ 1998 ອີກດ້ວຍ, ຊຶ່ງສະແດງໃຫ້ເຫັນວ່າມູນຄ່າລວມທີ່ໃຊ້ຈ່າຍໃນຖານເຫຼົ່ານີ້ແມ່ນຫຼາຍກວ່າ 21 ລ້ານລ້ານໂດລາ. ຖ້າໃຊ້ອັດຕາສ່ວນການໃຊ້ຈ່າຍລັບໆດຽວກັນນີ້ໃນໄລຍະປີ 2016-2023 ຈະໄດ້ມູນຄ່າລວມ 36.6 ລ້ານລ້ານໂດລາສະຫະລັດທີ່ໃຊ້ຈ່າຍນັບຕັ້ງແຕ່ປີ 1998 ເປັນຕົ້ນມາ.

ໃນປີ 2021, Mark Skidmore ໄດ້ຈັດພິມການປັບປຸງຜົນການຄົ້ນຄວ້ານີ້ກ່ຽວກັບການປະກາດຂອງ Bloomberg ວ່າໃນລະຫວ່າງປີການເງິນ 2017-19, ກະຊວງປ້ອງກັນປະເທດສະຫະລັດໄດ້ບັນທຶກການປັບຕົວເລກບັນຊີທີ່ໜ້າເຕັ້ນສະຫຼາດຂຶ້ນຫາ 94.7 ລ້ານລ້ານໂດລາ \cite{17,18}. ຖ້າພິຈາລະນາເຖິງການປອມແປງເງິນໂດລາສະຫະລັດຜ່ານລະບົບທະນາຄານກາງທີ່ເກີດຂຶ້ນມາເປັນເວລາຫຼາຍກວ່າໜຶ່ງສັດຕະວັດນັບຕັ້ງແຕ່ການສ້າງຕັ້ງທະນາຄານກາງສະຫະລັດ (Federal Reserve) ໃນປີ 1913 \cite{37}, ຈະເຫັນໄດ້ຊັດວ່າການບັນຊີເງິນໂດລາສາທາລະນະທັງໝົດແມ່ນພຽງຄຳເວົ້າທີ່ບໍ່ມີຄວາມຫມາຍ, ແລະວ່າເງິນສະກຸນໂດລາສະຫະລັດແລະລັດຖະບານສະຫະລັດແມ່ນພຽງລະບົບການຈັດສັນຊັບພະຍາກອນທີ່ຜູ້ຄອບຄອງທີ່ເປັນລາຊະວົງສາມາດຖອດຖອນ (ຫຼືຈະເວົ້າໃຫ້ຖືກກວ່າວ່າ ສັກລົງ) ຫຼາຍຫຼືໜ້ອຍຕາມທີ່ຕ້ອງການ.
\section{ລູກຫຼານຂອງ Jove: ຕົວຕົນຂອງກະສັດຕາເວັນຕົກທີ່ຢູ່ໃນຮົ່ມ}

ແຕ່ວ່າ, ໃຜແມ່ນຜູ້ທີ່ກຳລັງຄວບຄຸງການສະແດງຕົວຈິງ? ພວກເຮົາບໍ່ສາມາດຮູ້ໄດ້ແນ່ນອນ, ເພາະວ່າກະສັດຕາເວັນຕົກຂອງທຶນຮອນຮັກສາຕົວເອງໄວ້ໃນຮົ່ມ. ໃນຂະນະທີ່ມີທິດສະດີທຸກຊະນິດ, ຕັ້ງແຕ່ບຸກຄົນສາທາລະນະຈົນເຖິງສິ່ງມີຊີວິດນອກໂລກ, ຄຳຕອບທີ່ດີທີ່ສຸດທີ່ຂ້າພະເຈົ້າມີຕໍ່ຄຳຖາມນີ້ຢູ່ໃນຜົນງານຊີວິດຂອງນັກຂຽນບລັອກທີ່ບໍ່ມີຊື່ຜູ້ໜຶ່ງທີ່ໃຊ້ຊື່ປອມ "Amallulla". ຜົນງານຂອງລາວເປັນການສັງສັນທີ່ກວ້າງຂວາງຂອງນັກຂຽນຫຼາຍກວ່າ 20 ຄົນ ແລະ ເອກະສານ "ທີ່ບໍ່ສາມາດທົດແທນໄດ້" 50 ຊິ້ນທີ່ຄຸມຫົວຂໍ້ຕ່າງໆກ່ຽວກັບປະຫວັດສາດບູຮານ ແລະ ສະໄໝໃໝ່, ສັນຍາລັກອຳມຸດ, ແລະ ການເມືອງຕາເວັນຕົກ \cite{33,34}. ຂ້າພະເຈົ້າສາມາດອະທິບາຍຜົນງານຂອງລາວໄດ້ວ່າເປັນ "ຄຳພະຍາກອນ" ກ່ຽວກັບໄພພິບັດທາງພູມສາດທີ່ຈະເກີດຂຶ້ນ - ມັນເປັນການຮວບຮວມທີ່ລະອຽດຫຼາຍກວ່າຂອງຂ້າພະເຈົ້າ.

Amallulla ໄດ້ລະບຸສາມກຸ່ມການເມືອງຕາເວັນຕົກ, ເຊິ່ງລາວເອີ້ນຮ່ວມກັນວ່າ "ລູກຫຼານຂອງ Jove", ຜູ້ທີ່ມີຄວາມຮູ້ກ່ຽວກັບ "ວັນສຸດທ້າຍ" - ໄພພິບັດທີ່ເກີດຂຶ້ນຊ້ຳໆຂອງໂລກ. ລາວເຊື່ອວ່າສາມກຸ່ມນີ້ຮ່ວມກັນຄວບຄຸງຊາດຕາເວັນຕົກໃນປະຈຸບັນ, ແຕ່ແບ່ງພວກເຂົາເປັນສາມກຸ່ມຕ່າງກັນຕາມຕົ້ນກຳເນີດ ແລະ ຕົວຕົນທາງປະຫວັດສາດ, ຄວາມຂັດແຍ້ງໃນອະດີດທີ່ເປັນໄປໄດ້, ແລະ ຄວາມແຕກຕ່າງທີ່ສັງເກດເຫັນໃນລະບົບຄຸນຄ່າ ແລະ ການກະທຳຂອງພວກເຂົາ.
The three factions can be vaguely categorized as follows:

\begin{flushleft}
\begin{enumerate}
\item ກຸ່ມທີ່ສາມສາມາດຈັດປະເພດໄດ້ຄ້າຍຄືດັ່ງນີ້:
    \item \textbf{ພວກທະນາຍານ}: ຊັ້ນນຳໂບຮານຂອງໂຣມັນ ທີ່ກາຍເປັນອົງຄະນະກຳມະການເມືອງທີ່ມີອຳນາດ ແລະ ອົງການຟຣີແມຊັນເຫນືອຂອງອາເມລິກາ.
    \item \textbf{ພວກນັກຄິດ}: ອົງການໂຣຊີຄຣູເຊຍນ ແລະ ອົງການຟຣີແມຊັນໃຕ້ຂອງອາເມລິກາ.
    \item \textbf{ພວກເຍຊູອິດ ແລະ ສັກດາສີດຳ}: ກຸ່ມຜູ້ສືບທອດຈາກໂຈວໃນສາດສະໜາຈັກກາໂຕິກໂຣມັນ.
\end{enumerate}
\end{flushleft}
ມື້ນີ້, ສາມຝ່າຍນີ້ຮ່ວມກັນປະກອບເປັນອິລລູມິນາຕີເອີຣົບ, ຟຣີແມຊັນ, ແລະ ຊີໄອເອ. ດັ່ງທີ່ອາມາລູລາໄດ້ອະທິບາຍ, \textit{"ດຽວນີ້, ໃນຍຸກສຸດທ້າຍ, ລູກຫຼານຂອງໂຢວເຊື້ອສາຍຖືກປິດບັງໄວ້ຢ່າງດີຫຼັງຄວາມລັບທີ່ແມ່ນຕ້ອງຮູ້ຈົນແມ່ນປະທານາທິບໍດີຂອງສະຫະລັດອາເມລິກາກໍບໍ່ມີສິດຮູ້. ເວົ້າອີກຢ່າງໜຶ່ງ, ພວກເຂົາໄດ້ພັດທະນາສີລະປະການປິດບັງຕົວເອງຈາກການກວດສອບຂອງສາທາລະນະຊົນ. \textbf{ລູກຫຼານຂອງໂຢວບໍ່ພຽງແຕ່ຄວບຄຸມກອງທັບ ແລະ ລັດຖະບານຂອງສະຫະລັດອາເມລິກາເທົ່ານັ້ນ, ແຕ່ຜ່ານອຳນາດຂອງເງິນຟຽດ, ບໍລິສັດຂະໜາດໃຫຍ່, ແລະ ລະບອບການປົກຄອງແບບສາທາລະນະລັດທີ່ພວກເຂົາປະດິດຂຶ້ນ (ດ້ວຍຄວາມຮູ້ວ່ານັກການເມືອງຈະຖືກຊື້ໄດ້ງ່າຍ ແລະ ດັ່ງນັ້ນຈຶ່ງຖືກຄວບຄຸມ), ພວກເຂົາຄວບຄຸມໂລກຕາເວັນຕົກທັງໝົດ}"} \cite{33,34}.

\begin{figure}[t]
\begin{center}
% \fbox{\rule{0pt}{2in} \rule{0.9\linewidth}{0pt}}
   \includegraphics[width=1\linewidth]{illuminati.jpg}
\end{center}
   \caption{ລູກຫຼານຂອງ Jove ແມ່ນໃຜແທ້? (ຮູບ: \cite{35})}
\label{fig:10}
\label{fig:onecol}
\end{figure}

\begin{figure}[t]
\begin{center}
% \fbox{\rule{0pt}{2in} \rule{0.9\linewidth}{0pt}}
   \includegraphics[width=1\linewidth]{pike.jpg}
\end{center}
   \caption{ຫີນບາທໍ່ລິດ Pike Peak ທີ່ມີຊື່ສຽງ, ເນັ້ນໃສ່ສີແດງ, ພ້ອມກັບພູມສັນຖານຂອງພາກຕາເວັນຕົກຂອງສະຫະລັດອາເມລິກາ \cite{36}. ຈິງໆແລ້ວ ສະຫະລັດອາເມລິກາສາມາດຖືກສ້າງຕັ້ງຂຶ້ນເພື່ອຄວບຄຸມສະຖານທີ່ນີ້ໄດ້ບໍ?}
\label{fig:11}
\label{fig:onecol}
\end{figure}

ຕາມຄຳເຫັນຂອງ Amallulla, ຄົນເຫຼົ່ານີ້ຖືກດູຖູກສາດສະໜາ, ຈັດການປື້ມສັກສິດໃນສາດສະໜາຕົ້ນຕໍຂອງໂລກເພື່ອຜົນປະໂຫຍດຂອງພວກເຂົາ, ແລະ ໃຊ້ສັນຍາລັກຢ່າງຫຼວງຫຼາຍ. ນອກຈາກນີ້, ພວກເຂົາບໍ່ມີຄວາມເມດຕາຕໍ່ສັດຕູຂອງພວກເຂົາ: \textit{"\textbf{ໃນໄລຍະເວລາຫຼາຍກວ່າ 2,600 ປີ, ພວກເຂົາໄດ້ລົບລ້າງຜູ້ອື່ນໆທີ່ມີຄວາມຮູ້ສະເພາະກ່ຽວກັບຍຸກສຸດທ້າຍຢ່າງເປັນລະບົບ. ແລະ ໂດຍສິ່ງນີ້, ຂ້ອຍບໍ່ໄດ້ໝາຍເຖິງພຽງແຕ່ພວກ druids, ນັກ Kabbalist ຢິວ, ຊາວອີຢິບເກົ່າ, ຊາວອາຣັບ, ແລະ ນັກປະພຶດອິນເດຍ, ແຕ່ຍັງລວມເຖິງກະໂຫຼກທີ່ຍາວໃນອາເມລິກາໃຕ້ ແລະ ພວກປະໂຫຍດ Maya ໃນອາເມລິກາກາງອີກດ້ວຍ. ແລະ ຫຼັກຖານທີ່ພວກເຂົາໄດ້ທຳລາຍປະຊາກອນທີ່ເຄີຍຈະເລີນຮຸ່ງເຮືອງໃນອາເມລິກາເໜືອເພື່ອຮັກສາສິ່ງນີ້ເປັນດິນແດນແຫ່ງຍຸກສຸດທ້າຍນັ້ນແມ່ນເກີນຈະເວົ້າ. ການຂ້າລ້າງຊົນເຜົ່າ "ອິນເດຍ" ອາເມລິກັນແມ່ນເປັນພຽງການກຳຈັດສ່ວນເຫຼືອ}"} \cite{33,34}.
ອາມາລູລາ ຍັງເຊື່ອອີກວ່າໂຄງການ "ສະຫະລັດອາເມຣິກາ" ທັງໝົດ ຖືກດຳເນີນໄປເພື່ອຮັບປະກັນການຄວບຄຸມ "ພູຜາຫີນແຮ່ Pikes Peak" ຊຶ່ງເປັນພູຜາຫີນແຮ່ແຮງໃນພູເຂົາ Rocky ທີ່ໃຫ້ການປ້ອງກັນທີ່ດີຈາກໄພພິບັດທາງທໍລະນີສາດ (ຮູບ \ref{fig:11}). ຕາມອາມາລູລາ, \textit{"ກ່ອນ, ໃນໄລຍະ, ແລະຫຼັງຈາກສິ່ງທີ່ພວກເຮົາຄິດວ່າເປັນສົງຄາມກາງເມືອງ, ພວກທະນາຍຄວາມແລະນັກຄິດໄດ້ຕໍ່ສູ້ບໍ່ໄດ້ເພື່ອການຄວບຄຸມສະຫະລັດອາເມຣິກາ, ແຕ່ເພື່ອພູຜາຫີນແຮ່ Pikes Peak, ຊຶ່ງເປັນຫນຶ່ງໃນພູຜາຫີນແຮ່ທີ່ເປັນເອກະລັກທີ່ສຸດໃນໂລກ... ບໍ່ມີພູຜາຫີນແຮ່ອື່ນໃດທີ່ຢູ່ສູງຂະຫນາດນີ້ແລະຢູ່ຫ່າງຈາກຝັ່ງທະເລໃນທົ່ວໂລກ. ມັນເປັນສະຖານທີ່ທີ່ເໝາະສົມທີ່ສຸດສຳລັບການລອດຊີວິດຈາກການເຄື່ອນຕົວຂອງແຜ່ນດິນໂລກ"} \cite{33,34}. ການຄົ້ນຄວ້າຂອງອາມາລູລາໄດ້ເປີດເຜີຍວ່າມີລະບົບຂຸມຄອງໃຕ້ດິນທີ່ກວ້າງຂວາງຖືກສ້າງຂຶ້ນໃຕ້ແລະອ້ອມຮອບພື້ນທີ່ນີ້ໃນປະຈຸບັນ \cite{36}.

\section{ສະຫຼຸບ}

ໃນບົດຄວາມນີ້ ຂ້າພະເຈົ້າໄດ້ລະອຽດກ່ຽວກັບບົດສະຫຼຸບຕ່າງໆທີ່ຊີ້ບອກວ່າຊັ້ນນຳຕາເວັບເຊື່ອງໄວ້ຄວາມຮູ້ກ່ຽວກັບໄພພິບັດຂອງໂລກທີ່ເກີດຂຶ້ນຊ້ຳໆມາເປັນເວລາຫຼາຍພັນປີ, ເຊື່ອວ່າອີກຄັ້ງໜຶ່ງຈະເກີດຂຶ້ນໃນໄວໆນີ້, ໄດ້ສ້າງບ່ອນພັກອາໄສໃຕ້ດິນຢ່າງກວ້າງຂວາງເພື່ອກຽມພ້ອມສຳລັບເຫດການດັ່ງກ່າວ, ແລະ ກຳລັງວາງແຜນທີ່ຈະໃຊ້ປະໂຫຍດຈາກເຫດການດັ່ງກ່າວໃນດ້ານການເມືອງແລະການທະຫານເພື່ອບັນລຸການຄອບຄອງໂລກ. ຂ້າພະເຈົ້າໄດ້ກ່າວເຖິງຂໍ້ຄຶດຕ່າງໆກ່ຽວກັບວິທີທີ່ນີ້ໄດ້ຮັບທຶນຮອນໃນອາເມຣິກາ, ພ້ອມທັງອ້າງອີງເຖິງທິດສະດີທີ່ໜ້າເຊື່ອຖືທີ່ສຸດກ່ຽວກັບວົງຕະກູນເລືອດທີ່ກຳລັງດຳເນີນການນີ້. ສຳລັບຜູ້ທີ່ຕ້ອງການຮູ້ເພີ່ມເຕີມ, ຍັງມີຂໍ້ມູນເພີ່ມເຕີມຫຼາຍຢ່າງທີ່ຂ້າພະເຈົ້າໄດ້ລະວັງໄວ້ ສາມາດພົບເຫັນໄດ້ໂດຍການຄົ້ນຄວ້າເພີ່ມເຕີມຈາກບັນດາແຫຼ່ງອ້າງອີງ.
ຈຸດຂໍ້ມູນທີ່ສາມາດວັດແທກໄດ້ທີ່ເຂັ້ມແຂງທີ່ສຸດທີ່ຊີ້ໃຫ້ເຫັນເຖິງເຫດການທາງພູມສາດທີ່ກຳລັງຈະເກີດຂຶ້ນແມ່ນສະໜາມແມ່ເຫຼັກຂອງໂລກທີ່ກຳລັງປ່ຽນແປງຢ່າງໄວວາ. ສິ່ງນີ້ສາມາດວັດແທກໄດ້ບໍ່ພຽງແຕ່ການເຄື່ອນທີ່ທີ່ເລັ່ງຂຶ້ນຂອງຂົ້ວແມ່ເຫຼັກເໜືອ (ຮູບ \ref{fig:13}) ແລະ ການເຕີບໃຫຍ່ຂອງຄວາມຜິດປົກກະຕິຂອງສະໜາມແມ່ເຫຼັກໃນມະຫາສະໝຸດອັດລັງຕິກໃຕ້, ແຕ່ຍັງມີການອ່ອນແອລົງແລະຄວາມບິດເບືອນທົ່ວໄປຂອງສະໜາມແມ່ເຫຼັກທີ່ເລັ່ງຂຶ້ນໃນຊ່ວງ 400 ປີຜ່ານມາ \cite{3}. ຂໍ້ມູນວິທະຍາສາດດັ່ງກ່າວໄດ້ຖືກອະທິບາຍຢ່າງລະອຽດໃນບົດຄວາມ ECDO ສອງເຫຼັ້ມທຳອິດຂອງຂ້າພະເຈົ້າ, ທີ່ສາມາດເຂົ້າເຖິງໄດ້ໃນເວັບໄຊຂອງຂ້າພະເຈົ້າ \cite{3}.

\begin{figure}[t]
\begin{center}
% \fbox{\rule{0pt}{2in} \rule{0.9\linewidth}{0pt}}
   \includegraphics[width=1\linewidth]{npw.jpg}
\end{center}
   \caption{ຕຳແໜ່ງຂອງຂົ້ວແມ່ເຫຼັກເໜືອຈາກປີ 1590 ຫາ 2025, ສະແດງໃນລະດັບ 5 ປີ. ການເຄື່ອນທີ່ຂອງມັນໄດ້ເລີ່ມເຄື່ອນໄວຢ່າງໄວວາໃນປີ 1975 \cite{41}.}
\label{fig:13}
\label{fig:onecol}
\end{figure}

ໃນການສະຫຼຸບ, ຂ້າພະເຈົ້າຈະປະຖິ້ມທ່ານດ້ວຍຄຳເວົ້າຈາກນັກພະຍາກອນ Amallulla ທີ່ອະທິບາຍວ່າ \textit{"\textbf{ທຸກຢ່າງແມ່ນສິ່ງດຽວ}"}: \textit{"ທີ່ນີ້ຂ້າພະເຈົ້າຈຳເປັນຕ້ອງຍູ້ຈິນຕະນາການຂອງທ່ານໄປສູ່ເຂດແດນສຸດທ້າຍ. ທ່ານຕ້ອງລືມໂລກທີ່ທ່ານອາໄສຢູ່ໃນປະຈຸບັນແລະຮູ້ຈັກຕັ້ງແຕ່ເດັກນ້ອຍ. ປະຖິ້ມມັນໄວ້ຂ້າງຫຼັງທ່ານ. ມັນແມ່ນຄວາມເປັນຈິງທີ່ຖືກສ້າງຂຶ້ນທັງໝົດບໍ່ຄ້າຍຄືກັບເລື່ອງໃນຮູບເງົາ Matrix ແລະມີເປົ້າໝາຍເພື່ອຮັກສາທ່ານໃຫ້ຫຼັບນອນຈົນກ່ວາເວລາສຸດທ້າຍ. ບາງຄັ້ງຂ້າພະເຈົ້າປາດຖະໜາວ່າຕົນເອງກຳລັງຂຽນບົດຮູບເງົາ, ແຕ່ສິ່ງທີ່ຂ້າພະເຈົ້າກຳລັງແບ່ງປັນກັບທ່ານໃນເວັບໄຊທ໌ນີ້ແມ່ນຄວາມເປັນຈິງ. ມັນໃຊ້ເວລາເກືອບເຄິ່ງສະຕະວັດຈຶ່ງຈະຮັບຮູ້ວ່າ 'ທຸກຢ່າງແມ່ນສິ່ງດຽວ', ເຊິ່ງຂ້າພະເຈົ້າໄດ້ນຳໃຊ້ເປັນຄຳຂວັນສຳລັບ An Apocalyptic Synthesis. ນີ້ແມ່ນແນວຄິດທີ່ຍາກທີ່ຈະສື່ສານ. ຕອນນີ້, ໃຫ້ພວກເຮົາຄິດໃນແງ່ຂອງຮູບເງົາ Matrix. ມັນເປັນການປຽບທຽບທີ່ດີ. ສິ່ງທີ່ຂ້າພະເຈົ້າພົບວ່າຍາກທີ່ຈະສື່ສານແມ່ນວ່າສິ່ງທີ່ຂ້າພະເຈົ້າກຳລັງຈະເວົ້າຕໍ່ໄປນີ້ບໍ່ແມ່ນການເວົ້າເກີນຈິງ. ຕອນນີ້, ການປຽບທຽບກັບຮູບເງົາ Matrix ແມ່ນໃກ້ທີ່ສຸດທີ່ຂ້າພະເຈົ້າຈະເຮັດໃຫ້ທ່ານເຂົ້າໃຈຄວາມເປັນຈິງອັນຮຸນແຮງຂອງສິ່ງທີ່ຂ້າພະເຈົ້າຈະເວົ້າຕໍ່ໄປ. \textbf{ທຸກຢ່າງໃນຊີວິດຂອງທ່ານ, ລວມທັງປະຫວັດສາດທີ່ບັນທຶກໄວ້, ວິທະຍາສາດແລະວິທະຍາໄລທີ່ເປັນທີ່ຍອມຮັບ, ການເມືອງ, ສາດສະໜາ, ທຸກຢ່າງໃນທາງໃດທາງໜຶ່ງແມ່ນກ່ຽວກັບການເຄື່ອນຕົວຂອງແຜ່ນດິນໂລກຫຼືການເຄື່ອນເສັ້ນແກນ.} ທ່ານພຽງແຕ່ຍັງບໍ່ເຫັນມັນໃນຕອນນີ້. ແລະທ່ານກໍບໍ່ສາມາດຕື່ນຂຶ້ນມາສູ່ຄວາມເປັນຈິງນີ້ໄດ້ຄືກັບຈາກຝັນຮ້າຍ. ມັນໃຊ້ເວລາ. ແຕ່ຂ້າພະເຈົ້າສັນຍາກັບທ່ານວ່າ, ຈຸດຈົບຂອງທາງນີ້ແມ່ນຄວາມຮັບຮູ້ວ່າທ່ານໄດ້ດຳລົງຊີວິດໃນໂລກທີ່ເທົ່າທຽມກັບຄວາມເປັນຈິງທີ່ຖືກຈຳລອງດ້ວຍຄອມພິວເຕີ້ Matrix ຕະຫຼອດຊີວິດຂອງທ່ານ"} \cite{33,34}.
ໂຊກດີໃຫ້ທຸກຄົນ.

\section{ຄຳຂອບໃຈ}

ຂອບໃຈທຸກຄົນທີ່ເລືອກທີ່ຈະປະກອບສ່ວນຄວາມຮູ້ໃຫ້ແກ່ສາທາລະນະ. ຖ້າບໍ່ມີທ່ານ, ຜົນງານນີ້ຈະບໍ່ສາມາດເກີດຂຶ້ນໄດ້ ແລະ ມະນຸດຊາດກໍ່ຈະຍັງຄົງຢູ່ໃນຄວາມມືດມົວ. ທາງເລືອກຂອງທ່ານຈະດອກບານໃນຄວາມອະນັນຕະ. ພວກເຮົາຕິດຫນີ້ທ່ານທຸກສິ່ງທຸກຢ່າງ, ແລະ ຂ້າພະເຈົ້າຮູ້ບຸນຄຸນຢ່າງບໍ່ມີຂອບເຂດ.
\clearpage
\twocolumn

{\small
\renewcommand{\refname}{ບັນທຶກອ້າງອີງ}
\bibliographystyle{ieee}
\bibliography{egbib}
}
\end{document}
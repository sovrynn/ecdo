\documentclass[10pt,twocolumn,letterpaper]{article}

\usepackage{booktabs}
% \usepackage{caption}
% \captionsetup[table]{skip=8pt}   % ဇယားများကိုသာ သက်ရောက်မှုရှိသည်
\usepackage{stfloats}  % ဤအား preamble တွင် ထည့်ပါ
\usepackage{float}

%–– ICU line-breaking for Khmer ––
%–– enable proper line-breaking for Burmese, and 0 space between characters? ––
\XeTeXlinebreaklocale "my"
\XeTeXlinebreakskip = 0pt plus 0pt minus 0pt
%–– you can experiment with some stretch:
% \XeTeXlinebreakskip = 0pt plus 1pt

\usepackage{fontspec}

%–– define your two fonts ––
\newfontfamily\latinfont{Latin Modern Roman}              % for all Latin text
\newfontfamily\burmesefont[Script=Myanmar]{Noto Sans Myanmar}        % for all Burmese text

%–– load ucharclasses to auto-switch when the script block changes ––
\usepackage{ucharclasses}

% default (everything outside Myanmar block) → Latin font
\setDefaultTransitions{\latinfont}{}

% when entering the Myanmar block → switch to Burmese font,
% and when exiting → switch back to Latin
\setTransitionsFor{Myanmar}{\burmesefont}{\latinfont}

% This makes the font slightly bigger than base (10) and bold in Subsection headings rather than using ptmb
\makeatletter
\def\cvprsubsection{%
  \@startsection{subsection}{2}{\z@}%
    {8pt plus 2pt minus 2pt}{6pt}%
    % {\normalfont\bfseries\selectfont}%
    {\normalfont\bfseries\fontsize{11}{13}\selectfont}%
}
\makeatother

% So this hardcodes the style for the numbers in the section/subsection headings so they're bold
\font\elvbf=ptmb scaled 1100
\font\elvbfs=ptmb scaled 1200
\makeatletter
% Section number: Large + bold
\renewcommand\thesection{%
  {\elvbfs\arabic{section}}%
}

% Subsection number: normalsize + bold + custom punctuation
\renewcommand\thesubsection{%
  {\elvbf
   \arabic{section}.\arabic{subsection}}%
}
\makeatother

\usepackage{cvpr}
\usepackage{times}
\usepackage{epsfig}
\usepackage{graphicx}
\usepackage{amsmath}
\usepackage{amssymb}
\usepackage[breaklinks=true,bookmarks=false]{hyperref}

% \makeatletter
% \def\cvprsubsection{\@startsection {subsection}{2}{\z@}
%     {8pt plus 2pt minus 2pt}{6pt}{\bfseries\normalsize}}
% \makeatother

\cvprfinalcopy % *** အဆုံးသတ်တင်ပြရန်အတွက် ဤစာကြောင်းကို မဖြည့်ပါနှင့်

\def\cvprPaperID{****} % *** CVPR စာတမ်းအတွက် ID ကို ဤနေရာတွင် ထည့်ပါ
\def\httilde{\mbox{\tt\raisebox{-.5ex}{\symbol{126}}}}


\renewcommand{\tablename}{ဇယား}
\renewcommand{\figurename}{ပုံ}   % or whatever you like instead of "Hình"
\renewcommand{\refname}{ကိုးကားချက်များ}

\makeatletter
\def\abstract{%
  \centerline{\large\bf အကျဉ်းချုပ်}% <-- your new label
  \vspace*{12pt}%
  \it%
}
\makeatother

% This makes the font slightly bigger than base (10) and bold in Subsection headings rather than using ptmb
\makeatletter
\def\cvprsubsection{%
  \@startsection{subsection}{2}{\z@}%
    {8pt plus 2pt minus 2pt}{6pt}%
    % {\normalfont\bfseries\selectfont}%
    {\normalfont\bfseries\fontsize{11}{13}\selectfont}%
}
\makeatother

% So this hardcodes the style for the numbers in the section/subsection headings so they're bold
\font\elvbf=ptmb scaled 1100
\font\elvbfs=ptmb scaled 1200
\makeatletter
% Section number: Large + bold
\renewcommand\thesection{%
  {\elvbfs\arabic{section}}%
}

% Subsection number: normalsize + bold + custom punctuation
\renewcommand\thesubsection{%
  {\elvbf
   \arabic{section}.\arabic{subsection}}%
}
\makeatother

% စာမျက်နှာများကို တင်သွင်းသည့်ပုံစံတွင် နံပါတ်တပ်ပြီး ကင်မရာပြီးသားတွင် နံပါတ်မတပ်ပါ
%\ifcvprfinal\pagestyle{empty}\fi
\setcounter{page}{1}
\begin{document}

\title{ECDO စာတမ်း ၃- မကြာမီဖြစ်ပေါ်လာနိုင်သည့် ဘူမိရူပဘေးအန္တရာယ်အတွက် ယနေ့ခေတ် အနောက်အင်အားကြီး နိုင်ငံများ၏ ကြိုတင်ပြင်ဆင်မှုအထောက်အထားများ}

\author{Junho\\
၂၀၂၅ ဇွန်လတွင် ထုတ်ပြန်သည်\\
ဝက်ဘ်ဆိုက် (စာတမ်းများကို ဒီမှာဒေါင်းလုတ်ယူပါ)- \href{https://sovrynn.github.io}{sovrynn.github.io}\\
ECDO သုတေသန စုစည်းထားသည့်စာရင်း- \href{https://github.com/sovrynn/ecdo}{github.com/sovrynn/ecdo}\\
{\tt\small junhobtc@proton.me}
}
\maketitle
%\thispagestyle{empty}

\begin{abstract}
၂၀၂၄ ခုနှစ် မေလတွင် “ကျင့်ဝတ် ဝေဖန်သူ” \cite{0} ဟု အမည်ရသော ကလောင်အမည်ဖြင့် အွန်လိုင်းစာရေးသူတစ်ဦးက အပူထုတ်လွှတ်သော ကမ္ဘာ့ဗဟိုလွှာ ဂျာနီဘီကော့ဗ် ခွဲဖြာရွေ့လျားခြင်း (ECDO) \cite{1} ဆိုသည့် ထူးခြားသော သီအိုရီတစ်ခုကို တင်ပြခဲ့သည်။ ဤသီအိုရီတွင် ကမ္ဘာမြေကြီး၏ ဝင်ရိုးလှည့်ပတ်မှုတွင် ရုတ်တရက် သဘာဝဘေးအန္တရာယ်ဆိုးကြီးများ ဖြစ်ပေါ်နိုင်သော ပြောင်းလဲမှုများကို ယခင်က ကြုံတွေ့ဖူးကြောင်း၊ ဝင်ရိုးလည်မှု အရှိန်ကြောင့် ပင်လယ်သမုဒ္ဒရာများဆီမှ ရေများသည် ကုန်းမြေများပေါ်သို့ စီးဝင်ကာ ကမ္ဘာတစ်ဝန်း ရေကြီးရေလျှံမှုကြီးများ ဖြစ်ပေါ်ခဲ့ကြောင်း အဆိုပြုထားသည်သာမက အလားတူ ရွေ့လျားမှုမျိုး မကြာမီဖြစ်ပွားနိုင်သည့် ဘူမိရူပ ဖြစ်စဉ် အကြောင်းရင်းကိုလည်း ဒေတာအချက်အလက်များနှင့်တကွ ရှင်းပြချက်ကိုပါ အဆိုပြုခဲ့ပါသည်။ ထိုသို့သော ရေကြီးရေလျှံမည့် သဘာဝဘေးအန္တရာယ်နှင့် သဘာဝဘေးအန္တရာယ်ဆိုးကြီးများ ဖြစ်ပေါ်နိုင်သည့် ခန့်မှန်းချက်များသည် အသစ်မဟုတ်သော်လည်း ECDO သီအိုရီသည် ခေတ်သစ်၊ သိပ္ပံနည်းကျ၊ နည်းလမ်းပေါင်းစုံနှင့် ဒေတာအခြေခံ ချဉ်းကပ်မှု ဖြစ်သောကြောင့် ထူးထူးခြားခြား စိတ်ဝင်စားဖွယ် ကောင်းပါသည်။

ဤသုတေသနစာတမ်းသည် ဤအကြောင်းအရာနှင့် ပတ်သက်၍ ကျွန်ုပ်၏ တတိယမြောက်စာတမ်း \cite{2,3} ဖြစ်ပြီး ဤသီအိုရီ၏ ယနေ့ခေတ် နိုင်ငံရေးရှုထောင့်များအပေါ် အာရုံစိုက် လေ့လာထားပါသည်-
\begin{flushleft}
\begin{enumerate}
    \item အနောက်အင်အားကြီးနိုင်ငံများက ဘူမိရူပ ဘေးအန္တရာယ်ကြီးတစ်ခု မကြာမီဖြစ်တော့မည်ဟု ယုံကြည်နေပြီး ၎င်းဖြစ်ရပ်အပေါ် နိုင်ငံရေးနှင့် စစ်ရေးအသာစီးရယူရန် အစီအစဉ်ရှိကြောင်း ဖော်ထုတ်ပြောဆိုသူထံမှ ထွက်ဆိုချက်များ။
    \item ထိုဖြစ်ရပ်အတွက် ပြင်ဆင်မှုအဖြစ် အနောက်နိုင်ငံများက ကျယ်ကျယ်ပြန့်ပြန့် တည်ဆောက်ထားသော မြေအောက်နှင့် ပင်လယ်ရေအောက် အခြေစိုက်စခန်းများ၏ အထောက်အထားများ။
    \item ဤအခြေစိုက်စခန်းများကို ငွေကြေးထောက်ပံ့ရန် အနောက်တိုင်း ငွေကြေးစနစ်များမှ ငွေများ အလုံးအရင်းနှင့် စီးဆင်းနေမှုဆိုင်ရာ အထောက်အထားများ။
\end{enumerate}
\end{flushleft}

ဤစာတမ်းသည် အနောက်အင်အားကြီး နိုင်ငံများက မကြာမီဖြစ်ပေါ်လာမည်ဟု ၎င်းတို့ယုံကြည်ထားသည့် ဘူမိရူပ ဘေးအန္တရာယ်အတွက် ကြိုတင်ပြင်ဆင်သည့်အနေနှင့် ကျယ်ကျယ်ပြန့်ပြန့် ပြင်ဆင်ဆောင်ရွက်နေမှုများကို မှတ်တမ်းတင်ထားပါသည်။
\end{abstract}

\section{ဖရီးမေဆင် လျှို့ဝှက်အဖွဲ့အစည်းနှင့် “ရှေးခေတ်အင်္ဂလိပ်လူမျိုးတို့၏ စီမံကိန်း”}

၂၀၁၀ ခုနှစ် ဇန်နဝါရီလတွင် ဖော်ထုတ်ပြောဆိုသူထံမှ ထွက်ဆိုချက်များကို စုစည်းထားသည့် အများနှင့်မတူသော မီဒီယာနှင့် စာနယ်ဇင်းအဖွဲ့တစ်ဖွဲ့ဖြစ်သည့် ကားမလော့စီမံကိန်းသည် ၂၀၀၅ ခုနှစ် ဇွန်လတွင် လန်ဒန်မြို့ရှိ အကြီးတန်း ဖရီးမေဆင်များ အစည်းအဝေးသို့ ကိုယ်တိုင် တက်ရောက်ခဲ့သော အတွင်းသိပုဂ္ဂိုလ်တစ်ဦးအား တွေ့ဆုံမေးမြန်းခဲ့ပါသည် \cite{4,6}။ အဆိုပါအစည်းအဝေးတွင် ဆွေးနွေးခဲ့သည့်အကြောင်းအရာများမှာ ကမ္ဘာလုံးဆိုင်ရာ သဘာဝဘေးအန္တရာယ်တစ်ခုဖြစ်သည့် \textbf{"ဘူမိရူပဖြစ်ရပ်"} တစ်ခု ဖြစ်လာနိုင်သည့်နောက်ကွယ်၌ ဗဟိုပြုထားသော စစ်ရေးနှင့် နိုင်ငံရေးအစီအစဉ်များ ဖြစ်ပါသည်။

\begin{figure}[b]
\begin{center}
\includegraphics[width=1\linewidth]{freemason.jpg}
\end{center}
   \caption{နျူကလီးယားဗုံးအချို့ ကြဲချကာ ကမ္ဘာကြီးကိုသိမ်းယူရန် တိတ်တဆိတ် ကြံစည်နေသည့် ဗြိတိသျှ ဖရီးမေဆင်များကို ၎င်းတို့၏ မူလအခြေအနေဖြင့် ၁၉၉၂ ခုနှစ်တွင် လန်ဒန်မြို့ရှိ အားလ်တရားရုံး၌ တွေ့ရစဉ် \cite{5}။}
\label{fig:1}
\label{fig:onecol}
\end{figure}

\begin{figure*}[t]
\begin{center}
\includegraphics[width=1\textwidth]{british.jpg}
\end{center}
   \caption{ရှေးခေတ်အင်္ဂလိပ်လူမျိုးတို့၏ အံမခန်း အင်အားပြသမှုဖြစ်သည့် ၁၉၃၇ ခုနှစ်က ဗြိတိသျှအင်ပါယာ \cite{14}။}
   \label{fig:2}
\end{figure*}
အဆိုပါ အတွင်းသိပုဂ္ဂိုလ်၏ အဆိုအရ အစည်းအဝေးတွင် ပါဝင်ခဲ့သူ ၂၅-၃၀ ဦးတွင် \textit{"...အားလုံးဗြိတိသျှနိုင်ငံသားများဖြစ်ကာ အချို့မှာ ယူကေ၌ အလွန်လူသိများသူများဖြစ်ကာ မြင်လိုက်သည်နှင့် ချက်ချင်းသိရှိနိုင်ပါသည်... အထက်တန်းလွှာ အချို့လည်း ပါဝင်ပြီး အချို့မှာ အတော်လေး အထက်တန်းကျသည့် မိသားစုများမှ ဆင်းသက်လာသူများဖြစ်သည်။ ထိုအစည်းအဝေးတွင် ကျွန်ုပ်သိရှိဖော်ထုတ်နိုင်သူ တစ်ဦးမှာ အကြီးတန်း နိုင်ငံရေးသမားတစ်ဦးဖြစ်ပါသည်။ အခြား နှစ်ဦးမှာ ရဲတပ်ဖွဲ့မှ အကြီးတန်း ပုဂ္ဂိုလ်များဖြစ်ပြီး တစ်ဦးကတော့ စစ်တပ်ဘက်ကဖြစ်သည်။ သူတို့နှစ်ဦးစလုံးကို တစ်နိုင်ငံလုံး လူသိများပြီး လက်ရှိအစိုးရကို လက်ရှိအချိန်တွင် အကြံပေးနေသည့် အဓိက ပုဂ္ဂိုလ်များဖြစ်ပါသည်"}\cite{4}။ ထိုအတွင်းသိပုဂ္ဂိုလ်က သူအစည်းအဝေးကို တက်ရောက်ခဲ့သည်မှာ \textit{"မတော်တဆသက်သက် ဖြစ်ပါသည်။ ယင်းမှာ သာမန် သုံးလတစ်ကြိမ် အစည်းအဝေးဟု ထင်ခဲ့ပြီး...ဤအစည်းအဝေးသို့ ရောက်သွားသောအခါ ကျွန်ုပ်မျှော်လင့်ထားသည့် အစည်းအဝေးမျိုး မဟုတ်ပါ။ ကျွန်ုပ်အား ဖိတ်ကြားခဲ့သည်မှာ... ကျွန်ုပ်၏ ရာထူးနှင့် ကျွန်ုပ်လည်း သူတို့လိုပဲ သူတို့ထဲက တစ်ယောက်ဟု ထင်သောကြောင့် ဖြစ်ပါသည်။"}\cite{4}။

အစည်းအဝေး၌ ဆွေးနွေးခဲ့သည့် အခြေပြုဖြစ်ရပ် အချိန်ဇယားမှာ (၂၀၀၅ ခုနှစ်က) အောက်ပါအတိုင်းဖြစ်ပါသည်-

\begin{flushleft}
\begin{enumerate}
    \item အီရန်နိုင်ငံ သို့မဟုတ် တရုတ်နိုင်ငံကို နည်းဗျူဟာမြောက် နျူကလီးယားလက်နက် အသုံးပြုအောင် လှုံ့ဆော်ကာ အကန့်အသတ်ဖြင့် နျူကလီးယား အပြန်အလှန်ပစ်ခတ်ပြီးနောက် အပစ်အခတ် ရပ်စဲရေး ထူထောင်ခြင်း။
    \item "၁၉၇၀ ပြည့်လွန်နှစ်များမှစ၍" အဓိက ပစ်မှတ်ဟု ဆိုသည့် တရုတ်နိုင်ငံအပေါ် ဇီဝလက်နက်များ အသုံးပြုခြင်း။
    \item ဖြစ်ပေါ်လာသော အကြောက်တရားနှင့် ပဋိပက္ခများကို အကြောင်းပြကာ စစ်အာဏာရှင်စနစ် အစိုးရများ ပေါ်ပေါက်လာစေခြင်း။
\end{enumerate}
\end{flushleft}

သို့သော် အရေးအကြီးဆုံးအချက်မှာ ထိုဖြစ်ရပ်များအပြီး၌ ဖြစ်ပေါ်လာနိုင်သည့်အရာများ ဖြစ်သည်- \textit{"ထို့ကြောင့် ကျွန်ုပ်တို့သည် ဤစစ်ပွဲထဲသို့ ဝင်ရောက်သွားပြီးနောက်... ကမ္ဘာပေါ်တွင် လူတိုင်းကို သက်ရောက်မှုရှိမည့် ဘူမိရူပဆိုင်ရာ ဖြစ်ရပ်တစ်ခု ပေါ်ပေါက်လာပါမည်"} \cite{4}။ အတွင်းသိပုဂ္ဂိုလ်က ယုံကြည်သည်မှာ ဤဘူမိရူပဆိုင်ရာ ဖြစ်ရပ်အတွင်း \textit{\textbf{"ကမ္ဘာ့အပေါ်ယံအလွှာသည် ၃၀ ဒီဂရီခန့်၊ မိုင်ပေါင်း ၁၇၀၀ မှ ၂၀၀၀ ခန့်အထိ တောင်ဘက်သို့ ရွေ့လျားသွားမည်။ ထိုအခါ ကြီးမားသော ဖြစ်ရပ်ဆိုးကြီး ပေါ်ပေါက်ကာ ယင်း၏အကျိုးဆက်များမှာ အချိန်ကြာမြင့်စွာ တည်ရှိနေပါလိမ့်မည် "}} \cite{4}။

ဤလျှို့ဝှက်အစီအစဉ်များ အားလုံးအတွက် အကြောင်းရင်းမှာ အာဏာပိုင်စိုးရေးဖြစ်သည်မှာ မှန်ပါသည်။ အတွင်းသိပုဂ္ဂိုလ်က ယခုလိုရှင်းပြပါသည်- \textit{"ထိုအချိန်ဆိုလျှင် ကျွန်ုပ်တို့အားလုံး နျူကလီးယားနှင့် ဇီဝစစ်ပွဲကို ဖြတ်သန်းပြီးသွားလောက်ပါပြီ။ ထိုသို့ဖြစ်လျှင် ကမ္ဘာ့လူဦးရေသည် အလွန်အမင်း လျော့နည်းသွားပါမည်။ ဤဘူမိရူပဆိုင်ရာ ဖြစ်ရပ် ဖြစ်လာသောအခါ ကျန်ရှိနေသူများမှာလည်း ထပ်မံ၍ ထက်ဝက်ခန့် လျော့သွားနိုင်ပါသည်။ ယင်းကိုကျော်ဖြတ်နိုင်သူများက ကမ္ဘာကိုချုပ်ကိုင်ကာ ကျန်ရစ်သော လူသားများကို နောက်ခေတ်သစ်ဆီသို့ ဦးဆောင်သွားမည့်သူများ ဖြစ်လာပါမည်။ ထို့ကြောင့် ကျွန်ုပ်တို့သည် သဘာဝဘေးအန္တရာယ်ဆိုးကြီး ဖြစ်ပြီးသည့်နောက်ပိုင်းခေတ်အကြောင်းကို ပြောနေခြင်းဖြစ်ပါသည်။ ဘယ်သူတွေက အာဏာရှိလာမလဲ။ ဘယ်သူတွေက ထိန်းချုပ်မှုရှိမလဲ။ ထို့ကြောင့် ဤအကြောင်းက အဓိကကျလာပါသည်။ ထို့ကြောင့်လည်း သူတို့က ဤကိစ္စများကို သတ်မှတ်ထားသည့် အချိန်တစ်ခုအတွင်းမှာ ဖြစ်ပျက်မည်ကို အလွန်ပင် စိတ်အားထက်သန်နေကြပါသည်... [ပဋိပက္ခ] မဖြစ်ခင် ဖွဲ့စည်းထားမှုတစ်ခု ရှိနေရန် လိုအပ်ပြီး ဖြစ်လာမည့်အရာကို ရှင်သန်ကျော်လွှားနိုင်အောင် အတိုင်းအတာတစ်ခုအထိ သေချာစေရပါမည်။ သို့မှသာ ဖြစ်ပျက်ပြီးနောက်တစ်နေ့တွင် ခြေစုံရပ်တည်နိုင်ပြီး ယခင်က ရရှိထားသည့် အာဏာပိုင်စိုးမှုကို ရရှိနိုင်ပါမည်"} \cite{4}။ တွေ့ဆုံမေးမြန်းစဉ်အတွင်း ဤအစီအစဉ်၏အမည်ဖြစ်သည့် "ရှေးခေတ်အင်္ဂလိပ်လူမျိုးတို့၏ စီမံကိန်း" အကြောင်းကိုလည်း ဤသို့ဆွေးနွေးထားပါသည်- \textit{[အင်တာဗျူးသူ]- "...ယင်းကို ရှေးခေတ်အင်္ဂလိပ်လူမျိုးတို့၏ (အင်္ဂလိုဆက်ဆွန်) စီမံကိန်း ဟု ခေါ်ရသည့် အကြောင်းရင်းက အခြေခံအားဖြင့် တရုတ်လူမျိုးများကို ဖယ်ရှားကာ သဘာဝဘေးအန္တရာယ်ဆိုးကြီး နောက်ပိုင်း အရာရာ ပြန်လည်တည်ဆောက်သောအခါ အခြားသူများမရှိတော့ဘဲ ရှေးခေတ်အင်္ဂလိပ်လူမျိုးများကသာ ကမ္ဘာသစ်ကို ပြန်လည်တည်ဆောက်ပြီး အမွေဆက်ခံနိုင်ပါမည်။ ထိုအချက်က မှန်ပါသလား။" [အတွင်းသိပုဂ္ဂိုလ်]- "ထိုအချက်မှန်၊ မမှန်တော့ အတိအကျမသိသော်လည်း သင်ပြောတာကို သဘောတူပါသည်။ အနည်းဆုံး ၂၀ ရာစုတစ်လျှောက်လုံးနှင့် ၁၉ ရာစု၊ ၁၈ ရာစုကတည်းက ဤကမ္ဘာ့သမိုင်းကို အနောက်တိုင်းနှင့် ကမ္ဘာ့မြောက်ပိုင်းဒေသက ဦးဆောင်လွှမ်းမိုးခဲ့ပါသည်"} \cite{4}။

ဘူမိရူပဆိုင်ရာဖြစ်ရပ် အတိအကျ ဖြစ်ပေါ်နိုင်သည့် အချိန်ကာလနှင့်ပတ်သက်၍ အတွင်းသိပုဂ္ဂိုလ်က ဤသို့ခန့်မှန်းပါသည်- \textit{"...ပင်ကိုသိစိတ်အရ အမှန်ပင် ခံစားမိသည်မှာ သူတို့ယခုကတည်းက စနစ်တကျ ပြင်ဆင်နေကြလောက်ပါပြီ... ဤဖြစ်ရပ် ဘယ်အချိန်မှာဖြစ်မည်ကို သူတို့ကောင်းကောင်း သိပြီးကြပြီဟု ထင်ပါသည်... \textbf{ကျွန်ုပ်ဘဝသက်တမ်းအတွင်း ထိုဖြစ်ရပ် ဖြစ်လာမည်ဟု အလွန်ခိုင်မာသော ခံစားချက်မျိုး ခံစားရပါသည်။ နှစ်ပေါင်း ၂၀ အတွင်းဆိုပါတော့}... ဤဘူမိရူပ ဖြစ်ရပ်မျိုး နောက်ဆုံးဖြစ်ပျက်ခဲ့သည်မှာ လွန်ခဲ့သည့် နှစ်ပေါင်း ၁၁,၅၀၀ ခန့်က ဖြစ်သည့်အတွက် ၎င်းဖြစ်ရပ်မျိုး နှစ်ပေါင်း ၁၁,၅၀၀ ခန့်အကြာတွင် သံသရာလည်ပြီး ဖြစ်တတ်သည်ကို တွက်ကြည့်လျှင် ထိုသို့ဖြစ်လာမည့် ကာလထဲသို့ ကျွန်ုပ်တို့ ယခု ဝင်ရောက်နေပါပြီ။ ယခုထပ်ဖြစ်ရမည့် အချိန်ရောက်နေပါပြီ... ထိုသို့ဖြစ်လာမည်ဟုလည်း သူတို့ နားလည်ထားကြပါသည်။ ထိုသို့ဖြစ်လာမည်ဆိုပြီး ကျိန်းသေသိရှိထားကြပါသည်... သိပြီးသားကိစ္စတစ်ခုဟု ထပ်ပြောလိုပါသည်။ သူတို့မသိဘူး ဆိုလျှင်တော့ ယုံနိုင်ဖွယ်မရှိပါ။ ဆိုလိုသည်မှာ ကမ္ဘာ့အတော်ဆုံးဦးနှောက်များက ဤကိစ္စအတွက် ကြိုးပမ်းနေကြပါလိမ့်မည်"} \cite{4}။

ဤသည်မှာ ကျွန်ုပ်တို့က အလွန်ကျေးဇူးတင်ရမည့် အစွမ်းထက်သော သက်သေခံချက်တစ်ရပ် ဖြစ်ပါသည်။ ထိုအင်တာဗျူးတွင် စာရေးသူသည် ပထမကမ္ဘာစစ်နှင့် ဒုတိယကမ္ဘာစစ်တို့သည် လူဖန်တီးသော စစ်ပွဲများ ဖြစ်သည်ဟူသော သူ့ယုံကြည်ချက်နှင့် ရှေးခေတ်အင်္ဂလိပ်လူမျိုးတို့၏ စီမံကိန်းသည် မျိုးဆက်ပေါင်းများစွာ ကတည်းက အသေအချာ စတင်ခဲ့သည်ဟူသော အချက်ကိုလည်း ဆွေးနွေးထားပါသည်။ ထိုအင်တာဗျူး ပြုလုပ်ခဲ့သည့် ၂၀၁၀ ခုနှစ်ကတည်းကဆိုလျှင် ယခုအခါ ၁၅ နှစ်ကျော်လွန်ခဲ့ပြီဖြစ်သည်။ ထိုအတွင်းသိပုဂ္ဂိုလ် ဖော်ပြခဲ့သော နှစ်ပေါင်း ၂၀ အတွင်း ဘူမိရူပဖြစ်ရပ်ကြီး ဖြစ်လာမည်ဟူသော ခန့်မှန်းချက်မှာ နောက်ထပ် ၅ နှစ်သာလိုပါတော့သည်။
\subsection{သဘာဝ ဘေးအန္တရာယ်ဆိုးကြီးများအကြောင်း  ဒရူးဟိ ရှေးခေတ်ဘာသာရေးဆိုင်ရာ အနောက်တိုင်း လျှို့ဝှက်အသိပညာ}

သဘာဝဘေးအန္တရာယ်ဆိုးကြီးများ ပြန်လည်ဖြစ်ပွားခြင်းအကြောင်း အနောက်တိုင်းအသိပညာကို ဖရီးမေဆင်များသာမက အခြားသူများကလည်း ကောင်းစွာ လျှို့ဝှက်ထားကြသည်။ ဒရူးဟိများသည် လွန်ခဲ့သော နှစ်ပေါင်း ၂၄၀၀ ကျော်က မှတ်တမ်းအခိုင်အမာ တည်ရှိခဲ့သည့် ရှေးခေတ် ဆဲလ်တစ်ယဉ်ကျေးမှု တစ်ခုဖြစ်ပြီး \cite{7} ကမ္ဘာကြီး၏ ပြန်ဖြစ်တတ်သော သဘာဝဘေးအန္တရာယ်ဆိုးကြီးများအကြောင်း အသိပညာကို လက်ဆင့်ကမ်းခဲ့ကြပါသည်။ နောက်ဆုံးသိရှိရသည့် ဒရူးဟိမှာ ဘန်မက်ဘရေဒီဖြစ်သည်ဟု ယူဆရပါသည်။ "နောက်ဆုံး ဒရူးဟိ" ဟူသော ၁၉၉၂ ခုနှစ်က မှတ်တမ်းရုပ်ရှင်တစ်ခုတွင် ဒရူးဟိများအကြောင်းကို သူကယခုလို ပြောခဲ့ပါသည်- \textit{"ဤအဖွဲ့တွင် ကျွန်ုပ်သည် အစဉ်အလာအရ နောက်ဆုံးကျန်ရှိသူ ဖြစ်နိုင်ပါသည်။ ဤအဖွဲ့သည် တစ်ကမ္ဘာလုံးတွင် နောက်ဆုံးအကြိမ် ဖြစ်ခဲ့သည့် သဘာဝဘေးအန္တရာယ်ဆိုးကြီး သို့မဟုတ် ကပ်ဘေးဆိုးကြီး အပြီးတွင် ပေါ်ပေါက်လာခဲ့ပါသည်။ ယခုအခါ ပြင်းထန်သော လျှပ်စစ်မုန်တိုင်းများကြောင့် ကမ္ဘာကြီးအပေါ် ဆိုးရွားပြင်းထန်သော သက်ရောက်မှုများဖြစ်ကာ ဥက္ကာခဲများ၏ သို့မဟုတ် ဥက္ကာပျံ ရွာချခြင်းများကြောင့် ပိတ်မိနေပြီး ကျွန်ုပ်တို့သိထားသည့် လူ့အဖွဲ့အစည်းမှာ အလုံးစုံ ပျက်စီးသွားသည်... အသိပညာအားလုံးသည် အဖွဲ့၏ လွှမ်းခြုံမှုအတွင်းတွင် ရှိသော်လည်း ၎င်းတို့သည် နက္ခတ္တဗေဒနှင့် အထူးသဖြင့် သက်ဆိုင်ပါသည်။ အဘယ်ကြောင့်ဆိုသော် ၎င်းတို့သည် အလွန်ကြီးမားသော ဘေးဒုက္ခများစွာကို ကြုံတွေ့ခဲ့ရသောကြောင့် ဖြစ်ပါသည်။ နက္ခတ္တဗေဒအကြောင်း အပြည့်အဝသိရှိပါက ထိုဘေးဒုက္ခများ ဖြစ်နိုင်ခြေရှိသည့် အခြေအနေများကို ကြိုတင်ခန့်မှန်းနိုင်မည်ဖြစ်သလို သူတို့ကိုယ်သူတို့ ကာကွယ်ရန် တစ်ခုခု လုပ်ဆောင်နိုင်ခဲ့မည်ဟု ယူဆထားကြသည်။ အိုင်ယာလန်နိုင်ငံရှိ ကြီးမားသော ကျောက်တုံးကျောက်ဆောင် အစုအဝေးများကို ကြည့်ပါက ယင်းတို့ကို လမ်းဘေးသင်္ချိုင်းများဟု ဖော်ပြထားသော်လည်း အမှန်တကယ်တွင် အလွန်ရှေးကျသော ဗုံးခိုကျင်းများဖြစ်သည်ကို တွေ့ရပါမည်။ ယင်းတို့သည် ပင်လယ်ဒီရေလှိုင်းများ၏ အမြင့်ထက် များစွာမြင့်မားပြီး ဥက္ကာပျံရွာချခြင်းများမှလည်း ကာကွယ်ပေးပါသည်"} \cite{8,9}။

% ဖရီးမေဆင်အဖွဲ့သည် အမှန်တကယ်တွင် ဒရူးအစ်ဒ်များမှ ဆင်းသက်လာသည်ဟုလည်း ယုံကြည်ရသည် \cite{10}။
\section{သဘာဝဘေးအန္တရာယ်ဆိုးကြီးအတွက် ယနေ့ခေတ် အနောက်တိုင်းမှ ကြိုတင်ပြင်ဆင်မှုများ အထောက်အထား}

အနောက်အင်အားကြီး နိုင်ငံများသည် ကမ္ဘာလုံးဆိုင်ရာ ဘူမိရူပ သဘာဝဘေးအန္တရာယ်ဆိုးကြီးတစ်ခု မကြာမီဖြစ်ပွားနိုင်ဖွယ်ရှိကြောင်း ယုံကြည်နေသည်ဆိုပါက ထိုသို့သောဖြစ်ရပ်မှ သူတို့ကိုယ်သူတို့ ကာကွယ်ရန် ကြီးမားသော ကြိုတင်ပြင်ဆင်မှုများ ပြုလုပ်နေမည်ဟု ခန့်မှန်းရပါသည်။ ထို့ပြင် အများပြည်သူသိရှိသည့် အထောက်အထားများအရ အနောက်နိုင်ငံများစွာတွင် နက်ရှိုင်းသော မြေအောက်စခန်း ကွန်ရက်များ ကျယ်ကျယ်ပြန့်ပြန့် တည်ဆောက်ထားပါသည်။ ထိုသို့သောအဆောက်အအုံများသည် နျူကလီးယားစစ်ပွဲ ဖြစ်ပွားလျှင် နေထိုင်သူများကို ကျိန်းသေကာကွယ်ပေးနိုင်သည်သာမက သဘာဝဘေးအန္တရာယ်အမျိုးမျိုးမှလည်း အကာအကွယ်ပေးနိုင်ပါသည်။ ကားမလော့စီမံကိန်းမှ ဗြိတိသျှအကြီးတန်း ဖရီးမေဆင်၏ သက်သေခံချက်အရ \cite{4,6} ဤဖြစ်နိုင်ခြေများသည် အလားအလာများသာမဟုတ်ဘဲ ကြိုတင်စီစဉ်ထားသော အစီအစဉ်များ ဖြစ်သည်ဟု ထင်ရပါသည်။ ထို့ပြင် ဤစခန်းများကို တည်ဆောက်ရန်၊ ဝန်ထမ်းများထားရှိရန်နှင့် ပြုပြင်ထိန်းသိမ်းရန် လိုအပ်မည့် ငွေကြေးပမာဏသည်လည်း အလွန်များပြားကြောင်း သတိပြုသင့်ပါသည်။ ၁၈ နှစ်ကျော် ကာလအတွင်း အမေရိကန်အစိုးရထံမှ ပျောက်ဆုံးနေသော ဒေါ်လာ ထရီလီယံများစွာနှင့် တိုက်ဆိုင်နေပါသည် (နောက်အပိုင်းတွင် ဖော်ပြထားသည်) \cite{11,12,13}။ မျိုးသုဉ်းကွယ်ပျောက်မည့် အဆင့် ဖြစ်ရပ်တစ်ခုအတွက် အခြားသော ကြိုတင်ပြင်ဆင်မှု ဥပမာများတွင် မျိုးစေ့နှင့် အသိပညာ သိုလှောင်ရုံများကဲ့သို့သော မော်ကွန်းတိုက် စီမံကိန်းအမျိုးမျိုးလည်း ပါဝင်ပါသည်။
\subsection{အမေရိကန်တို့၏ မြေအောက်နှင့် ရေအောက် အခြေစိုက်စခန်းများ}

ကျွန်ုပ်တွေ့ရှိသမျှ မြေအောက် အခြေစိုက်စခန်းများအပေါ် လူသိရှင်ကြား အကျယ်ပြန့်ဆုံး သုတေသန လေ့လာမှုသည် နက်ရှိုင်းသော မြေအောက် အခြေစိုက်စခန်းများအကြောင်း စာအုပ်များစွာထုတ်ဝေခဲ့သူ ရစ်ချတ်ဆောဒါးဆိုသည့် အမေရိကန်လွတ်လပ်သော သုတေသီတစ်ဦး၏ လုပ်ဆောင်ချက်ဖြစ်ပါသည် \cite{22}။ ဆောဒါး၏လုပ်ငန်းတွင် အစိုးရစာရွက်စာတမ်းများနှင့် အစီအစဉ်များကို မော်ကွန်းတင်ထိန်းသိမ်းခြင်း၊ သမိုင်းနှင့် လက်တလော သတင်းဆောင်းပါးများ၊ နည်းပညာများကို လေ့လာခြင်း၊ အရင်းအမြစ်များ မွေးမြူခြင်းနှင့် အတွင်းလူများ၏ ပြောဆိုချက်များကို စုစည်းခြင်းတို့ ပါဝင်ပါသည်။ ဆောဒါး၏ သုတေသနအရ အမေရိကန်နှင့် ၎င်း၏နယ်မြေများတစ်ဝိုက်တွင် အနည်းဆုံး ၃ မိုင်အထိ နက်ရှိုင်းနိုင်သည့် မြေအောက်နှင့် ရေအောက် အခြေစိုက်စခန်းများ၏ ကွန်ရက်ကြီးတစ်ခုရှိကြောင်း (ပုံ \ref{fig:4})၊ မြေအောက်လေပြွန် မြန်နှုန်းမြင့် သံလိုက်ရထားများဖြင့် ချိတ်ဆက်ထားနိုင်ကြောင်း  ဖော်ပြထားပါသည်။ ဤအခြေစိုက်စခန်းများကို အမေရိကန်ပြည်ထောင်စု ကုမ္ပဏီကို ပိုင်ဆိုင်သူများက စီမံဆောင်ရွက်သော \textit{"အဆင့်မြင့်ဘဏ္ဍာရေး၊ အပြည်ပြည်ဆိုင်ရာ၊ အေဂျင်စီအချင်းချင်း၊ ငွေကြေးခဝါချမှု လောင်းကစားနည်း"} ဖြင့် တိတ်တဆိတ်ငွေကြေးထောက်ပံ့ထားပါသည် \cite{22}။ ကက်သရင်း အော်စတင်ဖစ်စ် (၎င်း၏လုပ်ဆောင်ချက်ကို နောက်အပိုင်းတွင် ဖော်ပြထားသည်) နှင့် သူ၏လုပ်ဖော်ကိုင်ဖက် တစ်ဦးက ဤအခြေစိုက်စခန်းများ၏အတိုင်းအတာကို နောက်ဆက်တွဲ လေ့လာမှုတစ်ခု၌ အမေရိကန်မြေအောက်နှင့် ရေအောက် အခြေစိုက်စခန်း ၁၇၀ ခန့်ရှိကြောင်း ခန့်မှန်းထားပါသည် \cite{16,20}။

\begin{figure}[b]
\begin{center}
% \fbox{\rule{0pt}{2in} \rule{0.9\linewidth}{0pt}}
   \includegraphics[width=1\linewidth]{penta.jpg}
\end{center}
   \caption{အိမ်ဖြူတော်နှင့် ပင်တဂွန်စစ်ဌာနချုပ်တို့၏ မြေအောက်တွင် အမှန်တကယ် ဘာတွေရှိနေပါသနည်း။ ထင်ရှားသည်မှာ မြေအောက်ဥမင်လှိုဏ်ခေါင်းများပါဝင်သည့် နက်ရှိုင်းသော ကွန်ရက်တစ်ခုဖြစ်ပါသည် (ဓာတ်ပုံ- \cite{31})။}
\label{fig:3}
\label{fig:onecol}
\end{figure}
\begin{figure*}[t]
\begin{center}
% \fbox{\rule{0pt}{2in} \rule{.9\linewidth}{0pt}}
\includegraphics[width=0.9\textwidth]{basescrop.png}
\end{center}
\caption{ဆောဒါး၏ သုတေသနမှ ဖော်ထုတ်ထားသည့် မြေအောက်နှင့် ရေအောက် အခြေစိုက်စခန်းများနှင့် ကုန်းတွင်းပိုင်းသို့ ဦးတည်သော ရေငုပ်သင်္ဘောဥမင်လှိုဏ်ခေါင်းများ အသေချာဆုံး တည်ရှိနိုင်သည့် တည်နေရာများကို အတိအကျဖော်ပြထားသော မြေပုံ။ ဆောဒါးက \textit{"[အဆိုပါနေရာများ] ထက် \textbf{များစွာပို၍} အခြေစိုက်စခန်းများ ရှိလိမ့်မည်ဟု သေချာစွာယုံကြည်ပါသည်"} \cite{22}။}
\label{fig:4}
\end{figure*}

ဆောဒါး၏ အရင်းအမြစ်များမှ ဤအခြေစိုက်စခန်းများ၏ အတိုင်းအတာကို အသေးစိတ် ရှင်းပြထားသော သက်သေခံအထောက်အထား အချို့မှာ အောက်ပါအတိုင်းဖြစ်ပါသည်-
\begin{flushleft}
\begin{enumerate}
    \item မေရီလင်းပြည်နယ်ရှိ ကမ့်ဒေးဗစ်- \textit{"ကျွန်ုပ်၏ အရင်းအမြစ်က ပြောသည်မှာ ကမ့်ဒေးဗစ်၏ မြေအောက်အပိုင်းများသည် အလွန်ကျယ်ပြန့်ပြီး ရှုပ်ထွေးကာ လျှို့ဝှက်ဥမင်လိုဏ်ခေါင်းများ မိုင်ပေါင်းများစွာရှိသည့်အတွက် မည်သူကများ ဤအဆောက်အအုံ၏ မြေပုံအပြည့်အစုံကို စိတ်ထဲတွင် မှတ်ထားနိုင်မည်နည်းဟု သံသယဖြစ်မိပါသည်"} \cite{22}။
    \item ဝါရှင်တန်ဒီစီရှိ အိမ်ဖြူတော်- \textit{"၁၉၆၀ ပြည့်လွန်နှစ်များက လင်ဒန် ဘီဂျွန်ဆင် အစိုးရလက်ထက်အတွင်း ကျွန်ုပ်၏ ရင်းနှီးသည့် မိတ်ဆွေတစ်ဦးသည် ဤအဆောက်အအုံအတွင်း အောက်ဘက်သို့ ခေါ်ဆောင်သွားခြင်းခံရပါသည်။ သူက အိမ်ဖြူတော်ထဲမှ ဓာတ်လှေကားစီး၍ ဝင်သွားကာ အောက်ဘက်သို့ အစောင့်အရှောက်ဖြင့် ဆင်းသွားခဲ့သည်။ ဓာတ်လှေကားသည် ၁၇ ထပ်အထိ ဆင်းသွားကြောင်း သူကယူဆသည်။ မြေအောက်တွင် တံခါးဖွင့်လိုက်သောအခါ ခပ်ဝေးဝေး၌ ပျောက်ကွယ်သွားသလို မြင်ရသည့် လျှောက်လမ်းတစ်ခုဆီသို့ ခေါ်ဆောင်သွားခြင်း ခံရပါသည်။ အခြားတံခါးများနှင့် လျှောက်လမ်းများက ထိုလျှောက်လမ်း၏ ဘေးဘက်သို့ ခွဲထွက်သွားပါသည်"} \cite{22}။ ပုံ \ref{fig:3} တွင် ဖော်ပြထားပါသည်။
    \item မေရီလင်းပြည်နယ်ရှိ မီဒယ်ခံတပ် - ၁၉၇၀ ပြည့်လွန်နှစ်များတွင် "အောက်ထပ်" ထဲသို့ မတော်တဆ ရောက်သွားသူတစ်ဦး၏ ပြောပြချက်- \textit{"တံခါးကိုဖွင့်လိုက်လျှင် အောက်ကိုဆင်းသည့် လှေကားအခန်းတစ်ခုကို တွေ့ရသည်။ အစွန်းကိုသွားပြီး လက်တန်းများကြားမှ အောက်ကိုကြည့်လိုက်ပါသည်။ အောက်တွင် အထပ်ဘယ်နှစ်ထပ်ရှိမှန်း မရေတွက်ခဲ့သော်လည်း ၁၅-၂၀ ထပ်ခန့် ရှိမည်ဟု ခံစားရပါသည်... အောက်သို့ တစ်ထပ်ဆင်းသောအခါ တံခါးတစ်ချပ်တွေ့ပါသည်... တံခါးကိုဖွင့်ပြီး ခေါင်းပြူကြည့်လိုက်သောအခါ ဘယ်၊ ညာ နှစ်ဖက်စလုံးတွင် အဆုံးမရှိ ရှင်းလင်းအောင်မြင်တွေ့နေရသည် ဥမင်လိုဏ်ခေါင်း တစ်ခုကိုမြင်ရပါသည်။ ၎င်းသည် မြေပြင်ပေါ်က ဖုံးထားသည့် အဆောက်အအုံနှင့် ကားပါကင်ဧရိယာထက် များစွာ ကျော်လွန်သည့်နေရာ ဖြစ်နေသည်မှာ သေချာပါသည်... နံရံနှစ်ဖက်စလုံးတွင် တံခါးများရှိပြီး ပေ ၃၀-၄၀ ခန့် ခြားထားပါသည်... နောက်ထပ် နှစ်ထပ် ဆင်းကြည့်ရန် ဆုံးဖြတ်ပြီး နောက်တစ်ထပ် ဆင်းလျှောက်သွားရာ... အလားတူ အခင်းအကျင်းမျိုး တွေ့ရပါသည်... နောက်တစ်ထပ်ဆင်းပြီး ကြည့်လိုက်သောအခါတွင်လည်း ပထမနှစ်ထပ်နှင့် အတူတူ တွေ့ခဲ့ရပါသည်"} \cite{22}။
\end{enumerate}
\end{flushleft}

\begin{figure}[t]
\begin{center}
% \fbox{\rule{0pt}{2in} \rule{0.9\linewidth}{0pt}}
   \includegraphics[width=1\linewidth]{undersea.jpg}
\end{center}
   \caption{ရေအောက် အခြေစိုက်စခန်းတစ်ခုကို ဝေါ်လ်တာကော့ရှ်နာ၏ သရုပ်ဖော်ပုံ။ သူသည် ၁၉၆၀ ပြည့်လွန်နှစ်များတွင် ကယ်လီဖိုးနီးယားပြည်နယ်၊ ချိုင်းနားလိတ် လက်နက်ထားသိုရာဌာနရှိ အမေရိကန်ရေတပ်၏ ရော့ခ်ဆိုက် ရေအောက် အခြေစိုက်စခန်းအတွက် သရုပ်ဖော်ပန်းချီဆရာတစ်ဦးဖြစ်သည်။ ဆောဒါး၏ အရင်းအမြစ်များအရ ချိုင်းနားလိတ်တွင် တစ်မိုင်အနက်ရှိ မြေအောက် အခြေစိုက်စခန်းတစ်ခုရှိကြောင်း ဖော်ပြထားပါသည် \cite{22,23}။}
\label{fig:5}
\label{fig:onecol}
\end{figure}

ဆောဒါးသည် တစ်နာရီလျှင် အမြန်နှုန်း မိုင် ၂၀၀၀ အထိ ရောက်ရှိနိုင်သော မြေအောက်သံလိုက်ရထားများ၊ သမုဒ္ဒရာကြမ်းပြင်အောက်တွင် တည်ဆောက်ထားသော အခြေစိုက်စခန်းများ (ပုံ \ref{fig:5})၊ ကုန်းတွင်းပိုင်းသို့ ဦးတည်သော ရေငုပ်သင်္ဘော ဥမင်လိုဏ်ခေါင်းများအကြောင်း သက်သေခံချက်များကိုလည်း ရရှိခဲ့ပါသည်။ မက္ကဆီကိုပင်လယ်ကွေ့ရှိ ရေအောက်အခြေစိုက်စခန်းတစ်ခုအကြောင်း သက်သေခံချက်တစ်ခုနှင့် ပတ်သက်၍ ဆောဒါးက \textit{"ရေအောက်နှင့် မြေအောက် အခြေစိုက်စခန်းများ စာအုပ်ထုတ်ဝေပြီးနောက် နှစ်ဝက်အကြာတွင် ထူးခြားသော ရေအောက်စီမံကိန်းတစ်ခုနှင့် ပတ်သက်၍ သိရှိထားသူတစ်ဦးက ကျွန်ုပ်ထံ ဆက်သွယ်ခဲ့သည်... သူက ထိုစီမံကိန်းသည် မက္ကဆီကိုပင်လယ်ကွေ့၏ ပင်လယ်ကြမ်းပြင်အောက်တွင်ရှိပြီး ပါဆန်စ် ကုမ္ပဏီက ကန်ထရိုက်ယူထားကြောင်း ပြောပါသည်။ ပင်လယ်ကြမ်းပြင်အောက် ၂၈၀၀ ပေအနက်တွင် လုပ်ငန်းလည်ပတ်ရန်အတွက် ရည်ရွယ်ကာ ပါဆန်စ်က အထူးစက်ပစ္စည်းအချို့ကို ဝယ်ယူခဲ့ကြောင်း သူက ဆက်လက်ပြောကြားခဲ့သည်... ထိုစက်ပစ္စည်းများသည် ထူးခြားလှသည်ဖြစ်ရာ စက်တပ်ဆင်သော နေရာများတွင် လူသားများ ရှင်သန်နိုင်အောင် ရှင်းရှင်းလင်းလင်း ကြိုတင်သတ်မှတ်ထားပါသည်"} \cite{22} ဟု ဆိုသည်။
\begin{figure}[t]
\begin{center}
% \fbox{\rule{0pt}{2in} \rule{0.9\linewidth}{0pt}}
   \includegraphics[width=1\linewidth]{sub.jpg}
\end{center}
   \caption{ဝေါ်လ်တာကော့ရှ်နာ၏ ရေငုပ်သင်္ဘော ဥမင်လိုဏ်ခေါင်း သရုပ်ဖော်ပုံ \cite{22,23}။}
\label{fig:6}
\label{fig:onecol}
\end{figure}
\begin{figure}[t]
\begin{center}
% \fbox{\rule{0pt}{2in} \rule{0.9\linewidth}{0pt}}
   \includegraphics[width=1\linewidth]{iran.jpeg}
\end{center}
   \caption{အီရန်နိုင်ငံ၏ တရားဝင်ဗီဒီယိုပြကွက်တစ်ခုတွင် ၎င်းတို့၏ မြေအောက် "ဒုံးကျည်မြို့တော်" ကို ပြသထားသည် \cite{39,40}.}
\label{fig:12}
\label{fig:onecol}
\end{figure}
အကယ်၍ တိုက်ကြီးများကို ဖြတ်ကျော်တည်ဆောက်ထားသော လျှို့ဝှက်ကွန်ရက်တစ်ခုသည် ကျွန်ုပ်တို့ခြေထောက်အောက်တွင် မိုင်ပေါင်းများစွာ အနက်အထိ တူးဖော်ထားပြီး မြေအောက်နှင့် ရေအောက် အခြေစိုက်စခန်း ၁၇၀ ကျော်ဖြင့် ကြီးမားကျယ်ပြန့်လှပါက၊ ၎င်းကို အသံထက်မြန်သော လေဟာနယ်နည်းပညာသုံး သံလိုက်ရထားများဖြင့် ချိတ်ဆက်ထားပါက၊ ထိုအတွက် ကျွန်ုပ်တို့၏ အလုပ်အကိုင်များ၏ အသီးအပွင့်များဖြင့် ငွေကြေးထောက်ပံ့ထားပါက ယနေ့ခေတ် လူသားအများစုသည် ၎င်းတို့၏ နိုင်ငံရေးသမားများထံမှ အချည်းနှီးသော ညှိနှိုင်းထားသည့် ထုတ်ပြန်ချက်များကို ယုံကြည်နေရင်း မိမိတို့အောက်တွင် ဘာတွေရှိနေသည်ကို သတိမပြုမိရုံသာမက မကြာခင်အနာဂတ်၌ ဖြစ်လာမည့်အရာကိုပါ မသိရှိဘဲ အဆုံးစွန်သော မသိနားမလည်မှု အခြေအနေတွင် ပျော်ပျော်ကြီး ရှိနေပါတော့မည်။

နောက်ထပ် မှတ်သားရန်အချက်မှာ - ကြီးမားသော မြေအောက် ဥမင်လိုဏ်ခေါင်းကွန်ရက်များ တည်ရှိမှုကို အရှေ့အလယ်ပိုင်းရှိ လက်ရှိတိုက်ပွဲများ (ဂါဇာကမ်းမြှောင်အောက်ရှိ ဟားမတ်စ် ဥမင်လိုဏ်ခေါင်းများ \cite{38} နှင့် အီရန်၏ မြေအောက် "ဒုံးကျည်မြို့တော်" (ပုံ \ref{fig:12}) \cite{39,40}) တို့တွင် သံသယဖြစ်စရာမရှိဘဲ တွေ့ရှိနိုင်ပါသည်။ ထိုသို့တွေ့ရှိမှုကြောင့် အဆိုပါ အဆောက်အအုံများ တည်ဆောက်နိုင်မှု၊ အမှန်တကယ် တည်ရှိမှုတို့နှင့်ပတ်သက်၍ သံသယဖြစ်စရာမရှိတော့ပါ။ ၎င်းတို့သည် အခြားသော အရင်းအနှီးပိုမိုကောင်းမွန်သည့် နိုင်ငံများက ထိုကာလအတွင်းမှာပင် မည်သို့သော အဆောက်အအုံမျိုးကိုများ တည်ဆောက်ထားလေမလဲဆိုပြီး ကျွန်ုပ်တို့အတွက် စဉ်းစားစရာဖြစ်လာပါတော့သည်။
\subsection{နောက်ထပ်ဘန်ကာများနှင့် သဘာဝဘေးအန္တရာယ်ဆိုးကြီးအတွက် ကြိုတင်ပြင်ဆင်ထားသော အထောက်အထားများ}

\begin{figure}[t]
\begin{center}
% \fbox{\rule{0pt}{2in} \rule{0.9\linewidth}{0pt}}
\includegraphics[width=1\linewidth]{tyrol.jpg}
\end{center}
   \caption{ဆွစ်ဇာလန်နိုင်ငံ၊ တာရိုပြည်နယ် တောင်ပိုင်းရှိ ဘန်ကာများ။ ဥရောပ အဲလ်ပ်တောင်တန်းများ ဖြတ်သန်းတည်ရှိသည့် ဆွစ်ဇာလန်နိုင်ငံသည် ၎င်း၏ တောင်တန်းပေါ်မှ ဘန်ကာများကို ကျွမ်းကျင်စွာ ဖုံးကွယ်ထားကြောင်း သိရှိရပါသည် \cite{32}။}
\label{fig:7}
\label{fig:onecol}
\end{figure}

\begin{figure}[t]
\begin{center}
% \fbox{\rule{0pt}{2in} \rule{0.9\linewidth}{0pt}}
   \includegraphics[width=1\linewidth]{svalbard.jpg}
\end{center}
   \caption{နော်ဝေနိုင်ငံ ဆဗဲလ်ဘတ်ကျွန်းပေါ်ရှိ ကမ္ဘာ့မျိုးစေ့သိုလှောင်ရုံတွင် မျိုးစေ့တစ်သန်းကျော် သိုလှောင်ထားသည် \cite{24}။ မည်သို့သော ဘေးအန္တရာယ်ကြီးမျိုးဖြစ်ပွားလျှင် ၎င်းကိုအသုံးပြုရန် လိုအပ်မလဲဆိုသည်မှာ တွေးစရာ ဖြစ်နေပါသည်။}
\label{fig:8}
\label{fig:onecol}
\end{figure}

အမေရိကန် တော်ဝင်မြေအောက် အခြေစိုက်စခန်းများအပြင် ကမ္ဘာတစ်ဝန်းတွင် သဘာဝ ဘေးအန္တရာယ်ဆိုးကြီးအတွက် ကြိုတင်ပြင်ဆင်ထားသော နောက်ထပ် သဲလွန်စများစွာ ရှိပါသည်။ နော်ဝေ၊ ဆွစ်ဇာလန်၊ ဆွီဒင်နှင့် ဖင်လန်နိုင်ငံတို့သည် အကောင်းဆုံးနမူနာများ ဖြစ်ပါသည်။

\begin{flushleft}
\begin{enumerate}
    \item ကားမလော့စီမံကိန်းသည် နော်ဝေနိုင်ငံသား နိုင်ငံရေးသမားတစ်ဦး၏ သက်သေခံချက်ကို မျှဝေခဲ့ပြီး \cite{25,26} ထိုသူ၏ သက်သေခံ အထောက်အထားကို အတည်ပြုထားသော်လည်း လျှို့ဝှက်ထားခဲ့ပါသည်။ ထိုသူက နော်ဝေနိုင်ငံတွင် ကျယ်ပြန့်သော မြေအောက်အခြေစိုက်စခန်း ၁၈ ခုရှိပြီး နော်ဝေ (အစ္စရေးနှင့် "အခြားနိုင်ငံများစွာ" တို့) သည် ဤအခြေစိုက်စခန်းများကို သဘာဝဘေးအန္တရာယ် တစ်မျိုးမျိုးအတွက် ကြိုတင်ပြင်ဆင်ရန် တည်ဆောက်နေသည်ဟု အခိုင်အမာဆိုထားပါသည်။ ရစ်ချတ်ဆောဒါးသည်လည်း နော်ဝေနိုင်ငံရှိ ဖောက်ထွင်းထားသော တောင်တန်းတစ်ခုအတွင်း၌ တည်ဆောက်ထားသော မြေအောက် အခြေစိုက်စခန်းကြီး တစ်ခုအတွင်း ရောက်ရှိခဲ့ဖူးသူတစ်ဦးထံမှ သက်သေခံချက်ကို ရရှိခဲ့ပါသည် \cite{22}။
    \item ဆွစ်ဇာလန်နိုင်ငံတွင် အဲလ်ပ်စ်တောင်တန်းများ၏ အမြင့်ပိုင်း၌ တည်ဆောက်ထားသော နျူကလီးယား ဗုံးခိုကျင်းများ အများအပြားရှိသည်ဟု လူသိများပါသည် (ပုံ \ref{fig:7})။ ဤနေရာများ၏ အရေအတွက်မှာ ၃၇၀,၀၀၀ ကျော်အထိ အံ့အားသင့်ဖွယ်ရှိနေကာ နိုင်ငံသားတစ်ဦးချင်းစီကို ထားနိုင်လောက်သည့် ပမာဏဖြစ်ပါသည် \cite{27}။
    \item ဆွီဒင်နှင့် ဖင်လန်နိုင်ငံတို့တွင် မြို့ကြီးတိုင်းရှိ နေထိုင်သူများအတွက် ခိုလှုံနိုင်သည့် ဗုံးခိုကျင်းများ လုံလောက်စွာ ရှိပါသည် \cite{27}။
\end{enumerate}
\end{flushleft}

ဆီလီကွန်တောင်ကြားရှိ စီးပွားရေးလုပ်ငန်းရှင်ကြီးများသည်လည်း ဤအကြောင်းကို သိရှိထားကြသည်ဟု ဆိုနိုင်ပါသည်။ သိရှိချက်များအရ \textit{"LinkedIn ကုမ္ပဏီ၏ ပူးတွဲတည်ထောင်သူနှင့် ထင်ရှားသော ရင်းနှီးမြှုပ်နှံသူဖြစ်သည့် ရိဒ်ဟော့မန်းက နယူးယော့ကာမဂ္ဂဇင်းသို့ ယခုနှစ်အစောပိုင်းက ပြောကြားခဲ့သည်မှာ ဆီလီကွန်တောင်ကြားရှိ ဘီလီယံနာများ၏ ၅၀\% ကျော်သည် မြေအောက် ဗုံးခိုကျင်းကဲ့သို့သော "ကမ္ဘာပျက် အာမခံ" တစ်မျိုးမျိုးကို ဝယ်ယူထားကြသည်ဟု သူကခန့်မှန်းထားပါသည်... ဖော့စ် မဂ္ဂဇင်း၏ ဆောင်းပါးရှင် ဂျင်ဒေါ့ဆန်၏ အဆိုအရ ဘီလီယံနာများစွာတွင် "တဒင်္ဂအတွင်း ထွက်ခွာနိုင်ရန် အဆင်သင့်ဖြစ်နေသော" ကိုယ်ပိုင်လေယာဉ်များ ရှိကြပါသည်။ ၎င်းတို့သည် မော်တော်ဆိုင်ကယ်များ၊ လက်နက်များနှင့် လျှပ်စစ်ဓာတ်အားပေးစက်များလည်း ပိုင်ဆိုင်ကြပါသည်"} \cite{28}။

ထို့ပြင် အာ့ချ်စီမံကိန်းဖောင်ဒေးရှင်းက စီမံဆောင်ရွက်သည့် ကမ္ဘာ့အသိပညာဘဏ်တိုက်နှင့် ဆဗဲလ်ဘတ် ကမ္ဘာ့မျိုးစေ့သိုလှောင်ရုံ \cite{30} ကဲ့သို့ အကြီးစား စုစည်းထိန်းသိမ်းရေး စီမံကိန်းများစွာလည်း ရှိနေပါသည်။ ယင်းတို့သည် မျိုးသုဉ်းပျောက်ကွယ်မည့်အဆင့် သဘာဝဘေးအန္တရာယ်များ ဖြစ်ပေါ်လာပါက လူသားမျိုးနွယ်၏ အရေးကြီးသော အရင်းအမြစ်များကို ကာကွယ်ရန် ကြိုတင်ပြင်ဆင်နေသည်နှင့် တူပါသည်။
\begin{figure*}[t]
\begin{center}
% \fbox{\rule{0pt}{2in} \rule{.9\linewidth}{0pt}}
\includegraphics[width=0.9\textwidth]{govcrop2.png}
\end{center}
   \caption{၁၉၉၈ ခုနှစ်မှ ၂၀၂၃ ခုနှစ်အထိ အမေရိကန်အစိုးရ၏ မတည်ငွေ၊ သုံးစွဲမှုနှင့် လျှို့ဝှက်မြေအောက် အခြေစိုက်စခန်း သုံးစွဲမှုများ \cite{19}။}
   \label{fig:9}
\end{figure*}
\section{ကြီးမားသော မြေအောက်အခြေစိုက်စခန်းများအတွက် ဒီမိုကရေစီ ငွေကြေးထောက်ပံ့မှုစနစ်များ}

သို့ဆိုလျှင် ကမ္ဘာလုံးဆိုင်ရာ မြေအောက်နှင့်ရေအောက် အခြေစိုက်စခန်း ၁၇၀ ကျော်ပါဝင်သော ကွန်ရက်ကြီးများကို ကြွေးမြီကျွန်များမသိစေဘဲ မည်ကဲ့သို့ ငွေကြေးထောက်ပံ့ကြပါသနည်း။ ဤစီမံကိန်းများအတွက် သုံးစွဲသည့်ငွေပမာဏ အတိုင်းအတာနှင့် ထိုငွေများရရှိသည့် နေရာများကို ကျွန်ုပ်တို့ သိရှိနိုင်သော စာတမ်းတစ်ခု ရှိပါသည်။ ၂၀၁၇ ခုနှစ်တွင် အမေရိကန် ရင်းနှီးမြှုပ်နှံမှု ဘဏ်လုပ်ငန်းရှင်နှင့် ဘုရှ်အစိုးရ လက်ထက်က အစိုးရအရာရှိဟောင်း ကက်သရင်းအော်စတင်ဖစ်စ်နှင့် မစ်ရှီဂန်ပြည်နယ် တက္ကသိုလ်မှ စီးပွားရေးပညာရှင် မာ့ခ်စကစ်မိုးတို့သည် ၁၉၉၈-၂၀၁၅ ဘဏ္ဍာရေးနှစ်များအတွင်း အမေရိကန်အစိုးရ၌ ခွင့်ပြုချက်မရှိသော ဒေါ်လာ ၂၁ ထရီလီယံသုံးစွဲမှုကို တွေ့ရှိခဲ့ပါသည် \cite{11,12,13}။

၎င်းတို့၏ အစီရင်ခံစာအရ \textit{"၂၀၁၆ ခုနှစ်၊ အောက်တိုဘာ ၇ ရက်တွင် ရိုက်တာသတင်းဌာနက စကော့ပေါ်လ်ထရို (၂၀၁၆) ၏ ဆောင်းပါးတစ်ပုဒ်ကို ထုတ်ဝေခဲ့ပါသည်။ ထိုဆောင်းပါးတွင် ၂၀၁၅ ဘဏ္ဍာရေးနှစ်အတွင်း စစ်တပ်သည် “၎င်းတို့၏စာရင်းဇယားများ ကိုက်ညီအောင် အယောင်ပြမှုဖန်တီးရန်” ဒေါ်လာ ၆.၅ ထရီလီယံအား အတည်ပြုချက်မရှိသော စာရင်းအင်း အတိုးအလျှော့များ ပြုလုပ်ခဲ့ကြောင်း ဖော်ပြထားသည်။ ထိုနှစ်အတွက် စစ်တပ်၏ အထွေထွေရန်ပုံငွေ ဘတ်ဂျက်မှာ ဒေါ်လာ ၁၂၂ ဘီလီယံ ရှိခဲ့သည်ကိုထောက်လျှင် ဤဖော်ပြချက်မှာ အံ့သြဖွယ်ကောင်းလှပါသည်... DOD ၏ စာရင်းအင်း ပြဿနာများအတွက် မီဒီယာတွင် အရေးကြီး သတင်းခေါင်းစဉ်များဖြစ်ခဲ့သည်မှာ ယခင်နှစ်ပေါင်းများစွာ ကတည်းကဖြစ်ပြီး ၂၀၀၁ ခုနှစ်၊ စက်တင်ဘာ ၁၀ ရက်တွင် ကာကွယ်ရေးဝန်ကြီး ဒေါ်နယ်ရမ်းစဖဲလ်က လွှတ်တော်ကြားနားမှုတစ်ခုတွင် (C-SPAN, ၂၀၁၄) DOD သည် ဒေါ်လာ ၂.၃ ထရီလီယံတန်ဖိုးရှိ ငွေလွှဲပြောင်းမှုများကို ခြေရာခံမိခြင်းမရှိကြောင်း ထုတ်ဖော်ပြောဆိုခဲ့ပါသည်... ဤအသိအမှတ်ပြုချက်သည် ထိုနေ့တွင် သတင်းခေါင်းစဉ် ဖြစ်ခဲ့သော်လည်း တစ်ရက်အကြာတွင် ၉/၁၁ ဝမ်းနည်းဖွယ် အဖြစ်အပျက်ကို တစ်ကမ္ဘာလုံးက အာရုံစိုက်သွားသဖြင့် မေ့လျော့ခံခဲ့ရသည်... ပါမောက္ခ မာ့ခ်စကစ်မိုးသည် စစ်တပ်၏ ဒေါ်လာ ၆.၅ ထရီလီယံတန်ဖိုးရှိ အတည်ပြု၍မရသော ငွေကြေးလွှဲပြောင်းမှုများအကြောင်းကို သိရှိသောအခါ သူသည် မစ္စဖစ်စ်ထံ ဆက်သွယ်ခဲ့ပြီး ၂၀၁၇ ခုနှစ်၊ နွေဦးရာသီ၌ HUD နှင့် DOD အတွင်းရှိ ပုံမှန်မဟုတ်သော အတည်မပြုနိုင်သည့် ငွေကြေးလွှဲပြောင်းမှုကြီးများကို ဖော်ပြသည့် အလားတူ အခြားသော အစိုးရအစီရင်ခံစာများကို ရှာဖွေရန် သဘောတူညီခဲ့ကြပါသည်။ နောက်ခြောက်လအတွင်း စကစ်မိုး၊ ဖစ်စ်နှင့် ဘွဲ့လွန်ကျောင်းသား အနည်းငယ်ပါဝင်သည့်အဖွဲ့သည် ၁၉၉၈-၂၀၁၆ ကာလအတွင်း မှတ်တမ်းမှတ်ရာ မထောက်ပြနိုင်သော စုစုပေါင်းဒေါ်လာ ၂၁ ထရီလီယံတန်ဖိုးရှိ ငွေကြေးလွှဲပြောင်းမှုများကို ဖော်ပြထားသည့် တရားဝင် အစိုးရ စာရွက်စာတမ်းများကို စုဆောင်းခဲ့ကြပါသည်"} \cite{12}။

၁၉၉၈-၂၀၁၅ အထိကြာမြင့်သည့် ၁၈ နှစ်တာ ကာလအတွင်း အမေရိကန်အစိုးရ၏ အများသိထုတ်ပြန်ထားသော ဘဏ္ဍာရေးအသုံးစရိတ်မှာ ၄၀.၈ ထရီလီယံသာ ရှိခဲ့ပြီး \cite{15} ဆိုလိုသည်မှာ အမေရိကန်အစိုးရ၏ လူသိရှင်ကြား ဝန်ခံထားသော အမေရိကန် အစိုးရ အသုံးစရိတ်အပြင် မြေအောက် အခြေစိုက် စခန်းများအတွက် လျှို့ဝှက်အသုံးစရိတ်အဖြစ် ဘဏ္ဍာရေးအသုံးစရိတ်၏ တစ်ဝက်ကျော်ကို သုံးစွဲခဲ့ခြင်းဖြစ်သည်။ ထို့ပြင် ဤလျှို့ဝှက်အသုံးစရိတ်မှာ နှစ်ရှည်လများ ဘဏ္ဍာရေးလိုငွေပြမှုအပေါ်တွင် ထပ်မံတင်ရှိခဲ့ခြင်းဖြစ်ပြီး ၁၉၉၈ ခုနှစ် မတိုင်မီကတည်းက ရှိနှင့်ပြီးဖြစ်ကာ ယနေ့အထိ ဆက်လက်ရှိနေသည်ဟု ယူဆရသဖြင့် ဤအခြေစိုက် စခန်းများအတွက် သုံးစွဲခဲ့သော စုစုပေါင်းငွေပမာဏမှာ ဒေါ်လာ ၂၁ ထရီလီယံထက် များစွာပိုမိုမြင့်မားသည်ဟု ဆိုနိုင်ပါသည်။ အလားတူ လျှို့ဝှက်အသုံးစရိတ်၏ အချိုးကို ၂၀၁၆-၂၀၂၃ ကာလအတွက် ကြည့်ပါက ၁၉၉၈ ခုနှစ်မှစတင်၍ စုစုပေါင်း ဒေါ်လာ ၃၆.၆ ထရီလီယံ သုံးစွဲခဲ့သည်ဟု တွက်ချက်နိုင်ပါသည်။

၂၀၂၁ ခုနှစ်တွင် မာ့ခ်စကစ်မိုးသည် ဘလွန်းဘတ်ကုမ္ပဏီ၏ ကြေညာချက်တစ်ရပ်နှင့်ပတ်သက်၍ ဤသုတေသနကို အသစ်ထုတ်ပြန်ခဲ့ပြီး ၂၀၁၇-၁၉ ဘဏ္ဍာရေးနှစ်များအတွင်း ပင်တဂွန်သည် ဒေါ်လာ ၉၄.၈ ထရီလီယံ စာရင်းအင်း အတိုးအလျှော့များ အံ့အားသင့်ဖွယ် ပြုလုပ်ခဲ့ကြောင်း ဖော်ပြခဲ့ပါသည် \cite{17,18}။ ၁၉၁၃ ခုနှစ်တွင် အမေရိကန်ဗဟိုဘဏ်ကို စတင်တည်ထောင်ခဲ့သည့် အချိန်မှစ၍ ရာစုနှစ်တစ်ခုကျော် အမေရိကန်ဒေါ်လာများအား ဗဟိုဘဏ်စနစ်မှတစ်ဆင့် အတုပြုလုပ်မှုကို ထည့်သွင်းစဉ်းစားပါက \cite{37} လူသိရှင်ကြား ဒေါ်လာငွေ စာရင်းအင်းများမှာ လုံးဝအဓိပ္ပာယ်မဲ့သော လှည့်ဖြားမှုများသာဖြစ်ပြီး အမေရိကန်ငွေကြေးနှင့် အစိုးရသည် အရင်းအမြစ်ခွဲဝေမှု စနစ်များသာဖြစ်ကာ ၎င်းတို့၏ တော်ဝင်ပိုင်ဆိုင်သူများက ယင်းတို့ကို လိုသလောက် တိတ်တဆိတ် သဲ့ယူခြင်း (သို့မဟုတ် ထုတ်ယူခြင်း) ပြုလုပ်နိုင်သည်မှာ ရှင်းနေပါသည်။
\section{ဂျိုးဗ်၏မျိုးဆက် လျှို့ဝှက်အဖွဲ့အစည်း- အနောက်တိုင်း ဘုရင်များ၏ နောက်ကွယ်မှ အထောက်အထားများ}

သို့ဆိုလျှင် တကယ်တမ်း လွှမ်းမိုးထားသူများက မည်သူနည်း။ ကျွန်ုပ်တို့ အတိအကျ မသိနိုင်ပါ။ အကြောင်းမှာ အနောက်တိုင်း ငွေကြေးဘုရင်များသည် နောက်ကွယ်၌သာ နေကြသောကြောင့်ဖြစ်ပါသည်။ အများသိ ပုဂ္ဂိုလ်များမှသည် ဂြိုဟ်သားများအထိ သီအိုရီအမျိုးမျိုး ကျယ်ပြန့်စွာ ရှိနေသော်လည်း ဤအတွက် ကျွန်ုပ်ရရှိသော အကောင်းဆုံးအဖြေကို "အမ်မာလူလာ" ဆိုသော ကလောင်အမည်ဖြင့် အမည်မဖော်လိုသော ဘလော့ဂါတစ်ဦး၏ တစ်သက်တာ လေ့လာမှု၌ တွေ့ရှိရသည်။ သူ၏လုပ်ဆောင်ချက်များတွင် ရှေးဟောင်းနှင့် ခေတ်သစ်သမိုင်း၊ လျှို့ဝှက်သင်္ကေတများနှင့် အနောက်တိုင်းနိုင်ငံရေးဆိုင်ရာ အကြောင်းအရာများကို ဖော်ပြထားကာ စာရေးဆရာ ၂၀ ကျော်နှင့် "အစားထိုး၍မရသော" စာရွက်စာတမ်း ၅၀ ကျော်တို့ကို ကျယ်ကျယ်ပြန့်ပြန့် ပေါင်းစပ်စုစည်းထားခြင်း ဖြစ်သည် \cite{33,34}။ ကျွန်ုပ်သည် သူ၏လုပ်ဆောင်ချက်ကို ဖြစ်လာမည့် ဘူမိရူပ ကပ်ဘေးကြီးနှင့်ပတ်သက်၍ "ကြိုတင်ဟောကိန်း" ဟုသာ ဖော်ပြနိုင်ပါသည်။ ၎င်းသည် ကျွန်ုပ်၏လုပ်ဆောင်ချက်ထက် \textit{သိသိသာသာ} ပိုမိုပြည့်စုံပါသည်။

အမ်မာလူလာသည် အနောက်တိုင်း နိုင်ငံရေးအုပ်စုသုံးစုကို ဖော်ထုတ်ခဲ့ပြီး ထိုအုပ်စုသုံးစုကို စုပေါင်း၍ "ဂျိုးဗ်၏ မျိုးဆက်" ဟု ခေါ်ဆိုခဲ့သည်။ ၎င်းတို့သည် ကမ္ဘာမြေ၏ ပြန်ဖြစ်တတ်သော ကပ်ဘေးကြီးများဆိုသည့် "နောက်ဆုံးအချိန်" အကြောင်းကို သိရှိထားကြပါသည်။ သူက ထိုအုပ်စုသုံးစုသည် ယနေ့ခေတ် အနောက်တိုင်းနိုင်ငံများကို ထိန်းချုပ်ထားသော်လည်း ၎င်းတို့၏ မတူညီသော မူလအစ၊ သမိုင်းဝင် အထောက်အထားများ၊ ယခင်က သဘောထားကွဲလွဲမှုဖြစ်နိုင်ခြေများနှင့် ၎င်းတို့၏ တန်ဖိုးစနစ်များနှင့် လုပ်ဆောင်ချက်များ၌ မြင်တွေ့ရသော ကွာခြားမှုများအပေါ် အခြေခံ၍ အုပ်စုသုံးစုအဖြစ် ခွဲခြားလိုက်ပါသည်။

ထိုအုပ်စုသုံးစုကို အောက်ပါအတိုင်း အပေါ်ယံ အမျိုးအစားခွဲခြားနိုင်ပါသည်။

\begin{flushleft}
\begin{enumerate}
    \item \textbf{ဘဏ်လုပ်ငန်းရှင်များ}- ရှေးဟောင်းရောမ အထက်တန်းလွှာများဖြစ်ပြီး အမေရိကရှိ ကက်သလစ်အသင်းတော် သူရဲကောင်းများနှင့် မြောက်ပိုင်းအုပ်ချုပ်မှုအဖွဲ့ ဖရီးမေဆင်များအဖြစ် ပြောင်းလဲသွားပါသည်။
    \item \textbf{အတွေးအခေါ်ပညာရှင်များ}- ရိုစီခရူရှန်း ဘာသာရေးလှုပ်ရှားမှုအဖွဲ့နှင့် တောင်ပိုင်းအမေရိကန် ဖရီးမေးဆင်များ။
    \item \textbf{ဂျေးဆုအဖွဲ့အစည်းနှင့် ဂျေးဆုအဖွဲ့အစည်းခေါင်းဆောင်}- ရိုမန်ကတ်သလစ် အသင်းတော်အတွင်းရှိ ဂျိုးဗ်၏မျိုးဆက် အုပ်စု။
\end{enumerate}
\end{flushleft}
ယနေ့အချိန်တွင် ဤအုပ်စုသုံးစု၌ ဥရောပဉာဏ်ကြီးရှင်များ၊ ဖရီးမေဆင်များနှင့် စီအိုင်အေတို့ ပါဝင်နေသည်။ အမ်မာလူလာက ဖော်ပြထားသည်မှာ \textit{"ယခု နောက်ဆုံးအချိန်တွင် ဂျိုးဗ်၏ မျိုးဆက်များကို အမေရိကန်ပြည်ထောင်စု၏ လက်ရှိသမ္မတပင်လျှင် မသိရှိသည့် လိုအပ်သော လုံခြုံရေးအဆင့်များနောက်တွင် ကောင်းစွာဖုံးကွယ်ထားပါသည်။ တစ်နည်းဆိုရသော် ၎င်းတို့သည် ပြည်သူလူထု၏ စိစစ်မေးမြန်းမှု မခံရအောင် မိမိကိုယ်ကို ဖုံးကွယ်ရာတွင် အကောင်းဆုံး ကျွမ်းကျင်ကြပါသည်။ \textbf{ဂျိုးဗ်၏မျိုးဆက်များသည် အမေရိကန်ပြည်ထောင်စု၏ စစ်တပ်နှင့် အစိုးရကိုသာမက အများသုံးငွေကြေး၊ အဓိက ကော်ပိုရေးရှင်းများနှင့် (နိုင်ငံရေးသမားများသည် အလွယ်တကူ အဂတိလိုက်စားကာ ထိုနည်းဖြင့် ထိန်းချုပ်နိုင်မည်ကို သိထားသည့်အတွက်) ၎င်းတို့တီထွင်ဖန်တီးခဲ့သော သမ္မတနိုင်ငံစနစ်အစိုးရတို့ကိုပါ ထိန်းချုပ်ကာ အနောက်ကမ္ဘာတစ်ခုလုံးကို ထိန်းချုပ်ထားပါသည်}"} \cite{33,34}။

\begin{figure}[t]
\begin{center}
% \fbox{\rule{0pt}{2in} \rule{0.9\linewidth}{0pt}}
   \includegraphics[width=1\linewidth]{illuminati.jpg}
\end{center}
   \caption{ဂျိုးဗ်၏မျိုးဆက်များသည် မည်သူများနည်း။ (ပုံ- \cite{35})}
\label{fig:10}
\label{fig:onecol}
\end{figure}

\begin{figure}[t]
\begin{center}
% \fbox{\rule{0pt}{2in} \rule{0.9\linewidth}{0pt}}
   \includegraphics[width=1\linewidth]{pike.jpg}
\end{center}
   \caption{ထင်ရှားကျော်ကြားသော ပိုက်ခ်တောင်ထွတ် ကျောက်အငူကြီးအား အမေရိကန်ပြည်ထောင်စု၏ အနောက်ပိုင်း မြေမျက်နှာသွင်ပြင်နှင့်အတူ အနီရောင်ဖြင့် ဖော်ပြထားသည် \cite{36}။ ဤနေရာကို ထိန်းချုပ်ရန်အတွက် အမေရိကန်ပြည်ထောင်စုအား အမှန်တကယ်ပင် ဖွဲ့စည်းခဲ့လေသလား။}
\label{fig:11}
\label{fig:onecol}
\end{figure}

အမ်မာလူလာ၏ အဆိုအရ ထိုသူတို့သည် ဘာသာရေးကို မထီမဲ့မြင်ပြုကြပြီး ကမ္ဘာပေါ်ရှိ အဓိကဘာသာကြီးများ၏ မြင့်မြတ်သော ကျမ်းစာများကို သူတို့အကျိုးအတွက် အသုံးချကာ သင်္ကေတဝါဒကို အလေးအနက် အသုံးပြုကြပါသည်။ ထို့ပြင် ၎င်းတို့၏ ရန်သူများနှင့်ပတ်သက်လာလျှင်လည်း ကရုဏာ မထားရှိကြပါ- \textit{"\textbf{နှစ်ပေါင်း ၂,၆၀၀ ကျော်ကာလအတွင်း သူတို့သည် နောက်ဆုံးအချိန်နှင့် သက်ဆိုင်သော အထူးအသိပညာရှိသည့် အခြားသူမှန်သမျှကို စနစ်တကျ ဖယ်ရှားခဲ့ကြသည်။ ဤနေရာတွင် ကျွန်ုပ်ဆိုလိုသည်မှာ ဒရူးဟိများ၊ ဂျူးလူမျိုး ကာဘာလာများ၊ ရှေးအီဂျစ်လူမျိုးများ၊ အာရပ်များနှင့် အိန္ဒိယ လျှို့ဝှက်ဆန်းကြယ်သူများသာမက တောင်အမေရိကရှိ ဦးခေါင်းခွံ ရှည်လျားသူများနှင့် ဗဟိုအမေရိကရှိ မာယာဘုန်းတော်ကြီးများလည်း ပါဝင်သည်။ နောက်ဆုံးအချိန် ဖြစ်ပွားမည့်ဒေသအဖြစ် မြောက်အမေရိက ဒေသကို ထိန်းသိမ်းနိုင်ရန် ထိုဒေသတွင် တစ်ချိန်က စည်ပင်ဖွံ့ဖြိုးခဲ့သော လူဦးရေတစ်ရပ်လုံးကို သူတို့အမြစ်ဖြတ်ချေမှုန်းခဲ့သည့် အထောက်အထားများမှာ အလွန်ပင်များပြားပါသည်။ အမေရိကန် "အိန္ဒိယနွယ်ဖွား" များအား လူမျိုးတုံးသတ်ဖြတ်မှုမှာ လက်ကျန်ရှင်းလင်းခြင်းမျှသာ ဖြစ်ပေသည်}"} \cite{33,34}။

အမ်မာလူလာက "ပိုက်ခ်တောင်ထွတ် ကျောက်အငူကြီး" ကို ထိန်းချုပ်ရန် ရည်ရွယ်ချက်ဖြင့် "အမေရိကန်ပြည်ထောင်စု" စီမံကိန်း တစ်ခုလုံးကို ဆောင်ရွက်ခဲ့သည်ဟုလည်း ယုံကြည်ထားပါသည်။ ယင်းသည် ရော့ကီးတောင်တန်းများရှိ နှမ်းဖတ်ကျောက် တောင်တန်းတစ်ခုဖြစ်ပြီး ဘူမိရူပဆိုင်ရာ သဘာဝဘေးအန္တရာယ်များမှ ကောင်းစွာ ကာကွယ်ပေးနိုင်ပါသည် (ပုံ \ref{fig:11})။ အမ်မာလူလာ၏ အဆိုအရ \textit{"ပြည်တွင်းစစ်ဟု ကျွန်ုပ်တို့ထင်နေသည့်အရာ မတိုင်မီကာလ၊ ဖြစ်ပျက်နေသောကလာနှင့် ဖြစ်ပြီးနောက် ကာလတို့တွင် ဘဏ်လုပ်ငန်းရှင်များနှင့် အတွေးအခေါ် ပညာရှင်များသည် အမေရိကန်ပြည်ထောင်စုကို ထိန်းချုပ်ရန် မဟုတ်ဘဲ တစ်ကမ္ဘာလုံး၌ အထူးခြားဆုံး နှမ်းဖတ်ကျောက် တောင်တန်းတစ်ခုဖြစ်သည့် ပိုက်ခ်တောင်ထွတ်ကို ထိန်းချုပ်ရန် တိုက်ခိုက်ခဲ့ကြပါသည်... ကမ္ဘာပေါ်တွင် အခြားမည်သည့်နေရာတွင်မှ ပင်လယ်ရေမျက်နှာပြင်အထက် ဤမြင့်မားပြီး သမုဒ္ဒရာကမ်းခြေမှ ဤမျှဝေးကွာသော နှမ်းဖတ်ကျောက် တောင်တန်း မရှိပါ။ ၎င်းသည် ကမ္ဘာ့အပေါ်ယံလွှာ ရွေ့လျားမှုဖြစ်ပွားပါက ရှင်သန်ကျန်ရစ်ရန် အကောင်းဆုံးနေရာဖြစ်သည်။"} \cite{33,34}။ အ်မာလူလာ၏ သုတေသနအရ ယနေ့ခေတ်တွင် ဤဒေသအောက်၌ ကျယ်ပြန့်သော မြေအောက်ဥမင်လိုဏ်ခေါင်းစနစ်တစ်ခု တည်ဆောက်ထားသည်ကို ဖော်ပြခဲ့ပါသည် \cite{36}။

\section{နိဂုံး}

ဤစာတမ်းတွင် အနောက်တိုင်းအထက်တန်းလွှာများသည် ကမ္ဘာကြီး၌ ပြန်ဖြစ်တတ်သော သဘာဝဘေးအန္တရာယ်ဆိုးကြီးများအကြောင်း အသိပညာကို နှစ်ပေါင်းထောင်ချီ၍ ဂရုတစိုက် လျှို့ဝှက်ထားပြီး ထိုသို့သော အန္တရာယ်တစ်ခုသည် မကြာမီဖြစ်ပေါ်လာမည်ဟု ယုံကြည်ကာ ထိုဖြစ်ရပ်အတွက် ပြင်ဆင်ရန် ကျယ်ပြန့်သော မြေအောက်ခိုလှုံရာနေရာများ တည်ဆောက်ထားသလို ကမ္ဘာကိုထိန်းချုပ်နိုင်အောင် နိုင်ငံရေးနှင့် စစ်ရေးအရ အခွင့်ကောင်းယူရန် အစီအစဉ်ရှိကြောင်း အကြံပြုထားသော သက်သေအမျိုးမျိုးကို အသေးစိတ် ဖော်ပြခဲ့ပါသည်။ အမေရိကန်တွင် ယင်းကို ငွေကြေးထောက်ပံ့ပုံနှင့် ပတ်သက်၍ သဲလွန်စများကို ဖော်ပြကာ ဤဇာတ်လမ်းကို ဦးဆောင်နေသည့် မျိုးဆက် အတိအကျနှင့် ပတ်သက်သည့် အလှမ်းမဝေးလှသော သီအိုရီကိုလည်း ကိုးကားဖော်ပြခဲ့ပါသည်။ ပိုမိုသိရှိလိုသူများအနေနှင့် ကိုးကားချက်များကို စူးစမ်းရှာဖွေခြင်းဖြင့် တွေ့ရှိနိုင်သည့် အချက်အလက်များစွာ ကျန်ရှိနေပါသေးသည်။

ဘူမိရူပဖြစ်ရပ်တစ်ခု ဖြစ်ပေါ်လာနိုင်သည့်အချက်ကို ညွှန်ပြနေသည့် အနီးစပ်ဆုံး တိုင်းတာနိုင်သော အချက်အလက်မှာ ကမ္ဘာ့သံလိုက်စက်ကွင်း အလျင်အမြန် ရွေ့လျားနေခြင်းပင် ဖြစ်သည်။ ဤအချက်ကို သံလိုက်မြောက်ဝင်ရိုးစွန်း၏ လျင်မြန်သောရွေ့လျားမှု (ပုံ \ref{fig:13}) နှင့် တောင်အတ္တလန္တိတ် သံလိုက်စက်ကွင်း ထူးခြားမှု ပိုများလာခြင်းတို့အပြင် လွန်ခဲ့သော နှစ်ပေါင်း ၄၀၀ အတွင်း သံလိုက်စက်ကွင်း အားနည်းလာမှုနှင့် ပုံပျက်လာမှုတို့ဖြင့် တိုင်းတာနိုင်ပါသည် \cite{3}။ ထိုသိပ္ပံနည်းကျ အချက်အလက်များကို ကျွန်ုပ်၏ ပထမဆုံး ECDO စာတမ်းနှစ်စောင်တွင် အကျယ်တဝင့် ဆွေးနွေးထားပြီး ကျွန်ုပ်၏ဝက်ဘ်ဆိုက်တွင် ရယူနိုင်ပါသည် \cite{3}။

\begin{figure}[t]
\begin{center}
% \fbox{\rule{0pt}{2in} \rule{0.9\linewidth}{0pt}}
   \includegraphics[width=1\linewidth]{npw.jpg}
\end{center}
   \caption{၁၅၉၀ မှ ၂၀၂၅ အထိ သံလိုက်မြောက်ဝင်ရိုးစွန်း၏ တည်နေရာကို ၅ နှစ်တစ်ကြိမ်တိုး၍ ပြသထားခြင်း \cite{41}။ ၁၉၇၅ ခုနှစ်မှစ၍ ၎င်း၏ရွေ့လျားမှုသည် အလွန်လျင်မြန်လာပါသည်။}
\label{fig:13}
\label{fig:onecol}
\end{figure}

နိဂုံးချုပ်အားဖြင့် ကျွန်ုပ်သည် ရှေ့ဖြစ်ဟောသော အမ်မာလူလာ၏ အနာဂတ်ပြောဟောချက်တစ်ခုဖြစ်သည့် \textit{"\textbf{အရာအားလုံးသည် တစ်ခုတည်းဖြစ်သည်}"} ဆိုသည့် အဆိုအမိန့်ကို ရှင်းပြကာ ယခုလိုကိုးကားဖော်ပြလိုပါသည်။ \textit{"ဤနေရာတွင် သင့်စိတ်ကူးကို အကန့်အသတ်မရှိ ချဲ့ထွင်ရန် တိုက်တွန်းပါသည်။ သင်ကလေးဘဝကတည်းက ယခုအထိ နေထိုင်နေသည့် ကမ္ဘာကြီးကို မေ့ပစ်လိုက်ပါ။ ချန်ထားခဲ့လိုက်ပါ။ ဤသည်မှာ မက်ထရစ် ရုပ်ရှင်ဇာတ်လမ်းထဲတွင် ပြထားသည့်အတိုင်း မဟုတ်ဘဲ နောက်ဆုံးအချိန်အထိ သင့်ကိုအိပ်ပျော်နေအောင် ပြုလုပ်ထားသော အပြည့်အဝ လုပ်ကြံဖန်တီးထားသည့် အမှန်တရားတစ်ခု ဖြစ်ပါသည်။ တစ်ခါတစ်ရံ ကျွန်ုပ်သည် ရုပ်ရှင်ဇာတ်ညွှန်းတစ်ခုကို ရေးနေသလိုသာ ဖြစ်ပါစေဟု ခံစားမိပါသည်။ သို့သော် ဤဝက်ဘ်ဆိုက်တွင် ကျွန်ုပ်မျှဝေထားသည့်အချက်များသည် အမှန်တကယ် ဖြစ်ပါသည်။ 'အရာအားလုံးသည် တစ်ခုတည်းဖြစ်သည်' ဆိုသည်ကို ကျွန်ုပ်နားလည်ရန် ဆယ်စုနှစ် ထက်ဝက်ကျော် အချိန်ယူခဲ့ရပါသည်။ ယင်းကို ကျွန်ုပ်က ကမ္ဘာပျက်မည့်အယူအဆ၏ ဆောင်ပုဒ်အဖြစ် ချက်ချင်းရွေးချယ်လိုက်ပါသည်။ ယင်းမှာ ရှင်းပြရန်ခက်သည့် အယူအဆတစ်ခုဖြစ်သည်။ ယခုအချိန်မှာတော့ မက်ထရစ် ရုပ်ရှင်ဇာတ်လမ်းကို ဥပမာအဖြစ် စဉ်းစားကြည့်ရအောင်။ ဤသည်မှာ နမူနာကောင်း တစ်ခုလည်းဖြစ်သည်။ ကျွန်ုပ်အဖို့ ရှင်းပြရန်ခက်သည့်အချက်မှာ ကျွန်ုပ်ယခုပြောမည့်အရာသည် ချဲ့ကားပြောခြင်း မဟုတ်သည့်အချက် ဖြစ်ပါသည်။ ယခုအချိန်တွင် မက်ထရစ် ရုပ်ရှင်ဥပမာသည် ကျွန်ုပ်ပြောမည့်အရာ၏ ထင်ရှားသော အမှန်တရားကို သဘောပေါက်စေမည့် အနီးစပ်ဆုံးဥပမာဖြစ်နေပါသည်။ \textbf{မှတ်တမ်းတင်ထားသည့် သမိုင်းကြောင်း၊ ပင်မရေစီးကြောင်း၊ အများလက်ခံသော သိပ္ပံပညာနှင့်  ပညာရေးလောက၊ နိုင်ငံရေး၊ ဘာသာရေး အပါအဝင် သင့်ဘဝထဲက အရာအားလုံးသည် ကမ္ဘာ့အပေါ်ယံလွှာ ရွေ့လျားမှု သို့မဟုတ် ဝင်ရိုးစွန်းတိမ်းစောင်းမှုတို့နှင့် တစ်နည်းနည်းနှင့် သက်ဆိုင်နေပါသည်။} ယင်းကို သင်ယခု မြင်နေရရုံသာ မကပါ။ အိပ်မက်ဆိုးတစ်ခုမှ နိုးထလာသလို ဤအမှန်တရားကို သတိထားမိလာခြင်းလည်း မဟုတ်ပါ။ အချိန်ယူရပါမည်။ သို့သော် ဤခရီးလမ်း၏အဆုံးတွင် သင်သည် ကွန်ပျူတာဖြင့် ဖန်တီးထားသည့် မက်ထရစ် ရုပ်ရှင်ထဲကအတိုင်း သင့်ဘဝတစ်လျှောက်လုံး နေထိုင်ခဲ့ကြောင်း သဘောပေါက်လာလိမ့်မည်ဟု ကျွန်ုပ်အာမခံပါသည်"} \cite{33,34}။

အားလုံးကံကောင်းကြပါစေ။

\section{ဝန်ခံချက်}

အများပြည်သူအတွက် အသိပညာများ မျှဝေပေးခဲ့သူတိုင်းကို ကျေးဇူးတင်ပါသည်။ သင်တို့မရှိပါက ဤစာတမ်း ဖြစ်မြောက်လာမည် မဟုတ်သလို လူသားမျိုးနွယ်သည်လည်း အမှောင်ထဲတွင် ဆက်ရှိနေရပါလိမ့်မည်။ သင်တို့၏ ရွေးချယ်မှုများကြောင့် အကျိုးကျေးဇူးများ ဖြစ်ထွန်းနေပါမည်။ အရာအားလုံးအတွက် သင်တို့အား အနှိုင်းမဲ့ ကျေးဇူးတင်ရှိနေပါမည်။

\clearpage
\twocolumn

{\small
\renewcommand{\refname}{ကိုးကားချက်များ}
\bibliographystyle{ieee}
\bibliography{egbib}
}
\end{document}
\documentclass[10pt,twocolumn,letterpaper]{article}

\usepackage{booktabs}
% \usepackage{caption}
% \captionsetup[table]{skip=8pt}   % មានឥទ្ធិពលតែលើតារាងប៉ុណ្ណោះ
\usepackage{stfloats}  % បន្ថែមចំណុចទាំងនេះ ទៅក្នុងជំពួកដើម
\usepackage{float}

%–– ICU line-breaking for Khmer ––
\XeTeXlinebreaklocale "km"
\XeTeXlinebreakskip = 0pt plus 0pt minus 0pt
% \XeTeXlinebreakskip = 0pt plus 1pt

\usepackage{fontspec}

%–– define your two fonts ––
\newfontfamily\latinfont{Latin Modern Roman}        % for all Latin text
\newfontfamily\khmerfont[Script=Khmer]{Noto Sans Khmer} % for all Khmer text

%–– load ucharclasses to auto‐detect Unicode blocks ––
\usepackage{ucharclasses}

% default (everything outside Khmer) → Latin font
\setDefaultTransitions{\latinfont}{}

% when entering the Khmer Unicode block → switch to Khmer font,
% and when leaving it → switch back to Latin
\setTransitionsFor{Khmer}{\khmerfont}{\latinfont}

% 1) Choose your desired fixed leading:
\renewcommand\baselinestretch{1.2}  % or 1.3, 1.1…  adjust to taste

% 2) Force TeX to *always* use \baselineskip, never fall back to \lineskip:
\makeatletter
  \setlength\lineskiplimit{-\maxdimen} % always allow baselineskip
  \setlength\lineskip{0pt}             % no extra glue ever
\makeatother

\makeatletter
\def\cvprsubsection{%
  \@startsection{subsection}{2}{\z@}%
    {8pt plus 2pt minus 2pt}{6pt}%
    % {\normalfont\bfseries\selectfont}%
    {\normalfont\bfseries\fontsize{11}{13}\selectfont}%
}
\makeatother

% So this hardcodes the style for the numbers in the section/subsection headings so they're bold
\font\elvbf=ptmb scaled 1100
\font\elvbfs=ptmb scaled 1200
\makeatletter
% Section number: Large + bold
\renewcommand\thesection{%
  {\elvbfs\arabic{section}}%
}

% Subsection number: normalsize + bold + custom punctuation
\renewcommand\thesubsection{%
  {\elvbf
   \arabic{section}.\arabic{subsection}}%
}
\makeatother

\usepackage{cvpr}
\usepackage{times}
\usepackage{epsfig}
\usepackage{graphicx}
\usepackage{amsmath}
\usepackage{amssymb}
\usepackage[breaklinks=true,bookmarks=false]{hyperref}

% \makeatletter
% \def\cvprsubsection{\@startsection {subsection}{2}{\z@}
%     {8pt plus 2pt minus 2pt}{6pt}{\bfseries\normalsize}}
% \makeatother

\cvprfinalcopy % *** សូមដោះស្រាយបន្ទាត់នេះ សម្រាប់ការដាក់ស្នើចុងក្រោយ

\def\cvprPaperID{****} % *** សូមបញ្ចូលលេខ ID របស់អត្ថបទ CVPR នៅទីនេះ
\def\httilde{\mbox{\tt\raisebox{-.5ex}{\symbol{126}}}}

\renewcommand{\tablename}{តារាង}
\renewcommand{\figurename}{រូប}   % or whatever you like instead of "Hình"
\renewcommand{\refname}{ឯកសារយោង}

\makeatletter
\def\abstract{%
  \centerline{\large\bf សេចក្តីសង្ខេប}% <-- your new label
  \vspace*{12pt}%
  \it%
}
\makeatother

% ទំព័រត្រូវបានរាប់ក្នុងរបៀបដែលបានដាក់ស្នើ និងមិនបានរាប់ដោយអានយកជាស្ថាពរ
%\ifcvprfinal\pagestyle{empty}\fi
\setcounter{page}{1}
\begin{document}

\title{ឯកសារ ECDO ទី3: ភស្តុតាងនៃការរៀបចំរបស់ប្រទេសមហាអំណាចលោកខាងលិចនាពេលបច្ចុប្បន្ន សម្រាប់គ្រោះមហន្តរាយភូគព្ភសាស្ត្រដែលនឹងកើតឡើងនាពេលឆាប់ៗ}

\author{Junho\\
	បោះពុម្ពផ្សាយ ខែមិថុនា 2025\\
	គេហទំព័រ (ទាញយកឯកសារនៅទីនេះ): \href{https://sovrynn.github.io}{sovrynn.github.io}\\
	របាយការណ៍ស្រាវជ្រាវរបស់ ECDO: \href{https://github.com/sovrynn/ecdo}{github.com/sovrynn/ecdo}\\
{\tt\small junhobtc@proton.me}
}
\maketitle
%\thispagestyle{empty}

\begin{abstract}
	នៅក្នុងខែឧសភាឆ្នាំ2025 អ្នកនិពន្ធអនាមិកអនឡាញម្នាក់ឈ្មោះ "The Ethical Skeptic" \cite{0} បានផ្សព្វផ្សាយទ្រឹស្ដីបំប្លែងមួយដែលមានឈ្មោះថា ការបំបែកលំយោលស្រទាប់ក្រឡាប់ក្តៅក្នុងផែនដី (ECDO) \cite{1}។ ទ្រឹស្ដីនេះមិនត្រឹមតែបានផ្តល់នូវយោបល់ថាផែនដីកាលពីមុនបានបម្លែងអ័ក្សវិលដ៏សាហាវ ធ្វើអោយមានទឹកជំនន់ដ៏ធំទូទាំងពិភពលោក ដែលបណ្តាលអោយមហាសមុទ្រលេចលើទ្វីបនានា។ មិនត្រឹមតែដោយសារនិចលភាពនៃការវិលប៉ុណ្ណោះទេ តែថែមទាំងផ្ដល់ការបកស្រាយនៃដំណើរការភូមិវិទ្យាដែលផ្តោតលើទិន្នន័យហេតុផលដែលបានបង្ហាញថាការផ្លាស់ប្តូរផែនដីអាចនឹងកើតមានម្តងទៀត។ ទោះបីជាការព្យាករណ៍ទឹកជំនន់និងគ្រោះមហន្តរាយមិនមែនជារឿងថ្មីក៏ដោយ ទ្រឹស្ដី ECDO គឺមានភាពទាក់ទាញយ៉ាងខ្លាំងដោយសារតែការយោងទៅលើវិទ្យាសាស្ត្រ ពហុបច្ចេកទេស និងមានមូលដ្ឋានដែលផ្អែកទៅលើទិន្នន័យ។

	អត្ថបទនេះគឺជាស្នាដៃទីបីរបស់ខ្ញុំអំពីប្រធានបទ\cite{2,3}  ដែលផ្តោតទៅលើទិដ្ឋភាពនយោបាយបច្ចុប្បន្ននៃទ្រឹស្តីនេះ៖
\begin{flushleft}
\begin{enumerate}
    \item មានអ្នកបង្ហើបថាប្រទេសមហាអំណាចលោកខាងលិច ជឿថាគ្រោះមហន្ថរាយភូគព្ភសាស្ត្រគឺអាចកើតមានឆាប់ៗ ហើយប្រទេសទាំងនេះមានគម្រោងទាញយកផលប្រយោជន៍ទាំងនយោបាយនិងយោធានៅក្នុងព្រឹត្តិការណ៍ទាំងនេះ។
    \item ភស្តុតាងជាច្រើនដែលបញ្ហាញអំពីមូលដ្ឋានទ័ពក្រោមដីនិងសមុទ្រដែលបានសាងសង់ឡើងដើម្បីត្រៀមសម្រាប់ព្រឹត្តិការណ៍នេះ។
    \item ភស្តុតាងដែលបង្ហាញអំពីការចំណាយដ៏មហិមាសម្រាប់សាងសង់មូលដ្ឋាន។
\end{enumerate}
\end{flushleft}
	អត្ថបទមួយនេះបានបង្ហាញយ៉ាងច្បាស់លាស់អំពីការត្រៀមខ្លួនរបស់ប្រទេសលោកខាងលេចដើម្បីការពារគ្រោះមហន្តរាយភូគព្ភសាស្ត្រត្រដែលត្រូវបានជឿជាក់ថានិងកើតឡើងនាពេលឆាប់ៗ។
\end{abstract}

\section{ហ្រ្វីម៉ាសុនរី និង "បេសកកម្មរបស់ជនជាតិអង់គ្លេសបុរាណ"}
	នៅខែមករា ឆ្នាំ 2010 គម្រោង Camelot ដែលជាអង្គការសារព័ត៌មាននិងកាសែតមួយដែលចងក្រងសក្ខីកម្ម បានសម្ភាសន៍\cite{4,6} ជនបង្កប់ខ្លួនម្នាក់ ដែលមានវត្តមាននៅក្នុងកិច្ចប្រជុំរបស់សមាជិកជាន់ខ្ពស់របស់ហ្វ្រីម៉ាសុនរី នៅក្នុងទីក្រុងឡុងដ៍ ក្នុងខែមិថុនាឆ្នាំ2005។ ប្រធានបទដែលបានពិភាក្សានៅក្នុងកិច្ចប្រជុំនោះគឺអំពីផែនការយោធា និងនយោបាយដែលផ្តោតទៅលើ "ព្រឹត្តិការណ៍ភូគព្ភសាស្ត្រ" ពោលគឺគ្រោះមហន្តរាយធម្មជាតិកម្រិតពិភពលោក។

\begin{figure}[b]
\begin{center}
   \includegraphics[width=1\linewidth]{freemason.jpg}
\end{center}
   \caption{សមាជិកហ្វ្រីម៉ាសុនជនជាតិអង់គ្លេស ​កំពុងរៀបគម្រោងយ៉ាងសង្ងាប់ដើម្បីទម្លាក់គ្រាប់បែកនុយក្លេអ៊ែរនិងគ្រប់គ្រងពិភពលោកទាំងមូល - នៅ Earl's Court ទីក្រុងឡុងដ៍ឆ្នាំ1992\cite{5}.}
\label{fig:1}
\label{fig:onecol}
\end{figure}

\begin{figure*}[t]
\begin{center}
\includegraphics[width=1\textwidth]{british.jpg}
\end{center}
   \caption{ចក្រភពអង់គ្លេសនៅឆ្នាំ1937 ដែលបង្ហាញពីអំណាចដ៏រឹងមាំរបស់ជនជាតិអង់គ្លេសបុរាណ\cite{14}.}
   \label{fig:2}
\end{figure*}
	បើយោងទៅតាមជនបង្កប់ខ្លួន មានមនុស្សប្រហែល25-30នាក់ដែលបានចូលរួមក្នុងកិច្ចប្រជុំនោះគឺ \textit{"...សុទ្ធតែជាជនជាតិអង់គ្លេសទាំងអស់ ហើយសមាជិកមួយចំនួនគឺជាជនល្បីៗដែលប្រជាជនអង់គ្លេសអាចនឹងស្គាល់ភ្លាមៗ... មានពួកអភិជនមួយចំនួនដែលមកពីក្រុមគ្រួសារអភិជនល្បីៗ។ ក្នុងចំណោមអ្នកចូលរួមកិច្ចប្រជុំទាំងអស់នេះ​ ខ្ញុំស្គាល់មនុស្សម្នាក់ដែលជាអ្នកនយោបាយជាន់ខ្ពស់។ មានមនុស្សពីរនាក់ទៀតដែរជាមន្ត្រីប៉ូលីសជាន់ខ្ពស់ និងម្នាក់ទៀតជាមន្ត្រីកងទ័ព។ ជនទាំងពីនាក់នេះសុទ្ធតែជាជនល្បីល្បាញថ្នាក់ជាតិ ហើយមានតួនាទីយ៉ាងសំខាន់ក្នុងជួររដ្ឋាភិបាលនាពេលបច្ចុប្បន្ន"}\cite{4}។ ជនបង្កប់ខ្លួនបានរាយការណ៍ថា គាត់បានចូលរួមកិច្ចប្រជុំមួយនេះ\textit {"ដោយចៃដន្យ! ខ្ញុំគិតថាវាគ្រាន់តែជាកិច្ចប្រជុំធម្មតាដែលធ្វើឡើងរៀងរាល់3ខែម្តង... តាមពិតនៅពេលដែលចូលរួមកិច្ចប្រជុំនេះ វាមិនមែនជាអ្វីដែលខ្ញុំ​បានរំពឹងទុកនោះទេ។ ខ្ញុំជឿជាក់ថាមូលហេតុដែលគេបានអញ្ជើញខ្ញុំ... គឺដោយសារតែតំណែងរបស់ខ្ញុំ និងដោយសារពួកគេគិតថាខ្ញុំគឺជាមនុស្សដែលមានជំនឿដូចគ្នានឹងពួកគេផងដែរ"}\cite{4}។

បញ្ជីព្រឹត្តិការណ៍ដែលជាមូលដ្ឋាននៃការពិភាក្សានៅក្នុងកិច្ចប្រជុំ(នៅឆ្នាំ2005) មានដូចខាងក្រោម៖

\begin{flushleft}
\begin{enumerate}
    \item បង្ខំអុីរ៉ង់ ឬចិនអោយប្រើអាវុធនុយក្លេអ៊ែរយុទ្ធសាស្ត្រដើម្បីវាយប្រហារគ្នាទៅវិញទៅមកមួយចំនួនសិន បន្ទាប់មកបង្កើតសន្ធិសញ្ញាបទឈប់បាញ់។
    \item បាញ់អាវុធជីវសាស្ត្រលើចិន ដែលជាគោលដៅសំខាន់ជាគេ "តាំងពីទសវត្សរ៍ឆ្នាំ70"។
    \item បង្កើតជារដ្ឋាភិបាលយោធាផ្តាច់ការ ដែងជាលទ្ធផលនៃការភ័យខ្លាចនិងភាពច្របូកច្របល់ដែលបានកើតឡើង។
\end{enumerate}
\end{flushleft}

	ប៉ុន្តែអ្វីដែលសំខាន់បំផុតនោះ គឺអ្វីដែលគេរំពឹងថានឹងកើតឡើងបន្ទាប់ពីព្រឹត្តិការណ៍ទាំងនេះ៖ \textit{"ដូច្នេះយើងនឹងលូកដៃទៅក្នុងសង្គ្រាមនេះ បន្ទាប់មក... នឹងមានព្រឹត្តិការណ៍ភូគព្ភសាស្ត្រមួយកើតឡើងនៅលើផែនដី ដែលនឹងប៉ះពាល់ដល់មនុស្សគ្រប់គ្នា"}\cite{4}។ ជនបង្កប់បានជឿជាក់ថា ក្នុងអំឡុងពេលព្រឹត្តិការណ៍ភូគព្ភសាស្ត្រនេះ \textit{\textbf{"សំបកផែនដីនឹងផ្លាស់ប្តូររហូតទៅដល់30ដឺក្រេ ប្រហែល2700ទៅ3200គីឡូម៉ែត្រទៅផ្នែកខាងត្បូង ដែលបណ្តាលអោយមានភាពចលាចលយ៉ាងខ្លាំង ហើយផលប៉ះពាល់របស់វានឹងមានរយៈពេលយូរ" }}\cite{4}។

	មូលហេតុសម្រាប់ការរៀបចំគម្រោងសម្ងាត់ទាំងអស់នេះគឺដើម្បីអំណាច។ ជនបង្កប់បានពន្យល់ថា \textit{"ទម្រាំតែពួកយើងអាចឆ្លងកាត់សង្គ្រាមនុយក្លេអ៊ែរនិងជីវសាស្ត្របាន ចំនួនប្រជាជននៅលើផែនដីនឹងត្រូវបានកាត់បន្ថយយ៉ាងខ្លាំង។ នៅពេលដែលព្រឹត្តិការណ៍ភូគព្ភសាស្ត្រនេះកើតឡើង សូម្បីតែមនុស្សដែលនៅមានជីវិតច្បាស់ជាស្លាប់ពាក់កណ្តាលទៀត។ ហើយអ្នកណាដែលរស់រានមានជីវិតជាអ្នកកំណត់ថា តើអ្នកណានឹងដឹកនាំពួកគេទៅសម័យបន្ទាប់។ ដូច្នេះយើងកំពុងនិយាយអំពីសម័យបន្ទាប់ពីព្រឹត្តិការណ៍គ្រោះមហន្តរាយ ហើយនៅពេលនោះ អ្នកណានឹងគ្រប់គ្រងនិងកាន់កាប់ពួកយើង? និយាយទៅវាទាក់ទងទៅនឹងរឿងនេះទាំងអស់។ នេះជាមូលហេតុដែលពួកគេបានព្យាយាមយ៉ាងខ្លាំង ដើម្បីអោយរឿងទាំងនេះកើតឡើងក្នុងរយៈពេលដែលបានកំណត់... ពួកគេត្រូវការសំណង់ណាមួយដែលអាចជួយអោយពួកគេនៅរស់រានមានជីវិត បន្ទាប់ពីភាពចលាចលទាំងនេះកើតឡើង— គេធ្វើបែបនេះគឺដើម្បីគេចផុតពីគ្រោះមហន្តរាយទាំងនេះ ហើយបន្តកាន់អំណាចយ៉ាងរំភើយ ខ្លាំងជាងសម័យមុនទៅទៀត"}\cite{4}។ ក្នុងអំឡុងពេលដែលសម្ភាសន៍ ឈ្មោះនៃផែនការនេះដែលត្រូវបានហៅថា "បេសកកម្មរបស់ជនជាតិអង់គ្លេសបុរាណ" ក៏ត្រូវបានពិភាក្សាផងដែរ៖ \textit{[អ្នកសម្ភាសន៍]៖ "...មូលហេតុដែលគេហៅវាថា បេសកកម្មរបស់ជនជាតិអង់គ្លេសបុរាណ គឺដោយសារតែផែនការនេះសម្រាប់លុបបំបាត់ជនជាតិចិនទាំងស្រុង និងដើម្បីគ្រប់គ្រងពិភពលោកថ្មី​បន្ទាប់ពីព្រឹត្តិការណ៍គ្រោះមហន្តរាយនេះបានកើតឡើង។  តើរឿងនេះពិតប្រាកដដែរឬទេ?" [ជនបង្កប់]៖ "ខ្ញុំមិនសូវច្បាស់ដែរ ប៉ុន្តែខ្ញុំយល់ស្របជាមួយអ្នក​។ ក្នុងសតវត្សទី20  ហើយសូម្បីតែនៅក្នុងសតវត្សទី19និងទី18ក៏ដោយ ប្រវត្តិសាស្ត្រនៃពិភពលោកនេះត្រូវបានដឹកនាំដោយប្រទេសលោកខាងលិចនិងតំបន់ខាងជើងនៃភពផែនដីនេះ"}\cite{4}។
ទាក់ទងទៅនឹងពេលវេលាច្បាស់លាស់នៃព្រឹត្តិការណ៍ភូគព្ភសាស្ត្រនេះ ជនបង្កប់បានទស្សន៍ទាយថា៖\textit{"...បើតាមអារម្មណ៍ខ្ញុំ ពួកគេកំពុងតែត្រៀមខ្លួន... ហើយខ្ញុំគិតថាពួកគេដឹងអំពីពេលវេលាជាក់លាក់ដែលព្រឹត្តិការណ៍នេះនឹងកើតឡើង... \textbf{ខ្ញុំមានអារម្មណ៍យ៉ាងច្បាស់ថា ​ព្រឹត្តិការណ័នេះនឹងកើតឡើងនៅក្នុងជីវិតរបស់ខ្ញុំ ប្រហែលក្នុងរយៈពេល20ឆ្នាំខាងមុខ}... មានន័យថា វាស្ថិតនៅក្នុងពេលឥឡូវនេះដែលព្រឹត្តិការណ៍ភូគព្ភសាស្ត្រនេះនឹងកើតឡើង។ បើយើងពិចារណាទៅលើរយៈពេលចុងក្រោយដែលព្រឹត្តិការណ៍នេះបានកើតឡើងគឺប្រហែល11500ឆ្នាំកន្លងមកហើយ ជាទូទៅវាតែងតែកើតឡើងរៀងរាល់11500ឆ្នាំម្ដង។ ឥឡូវនេះវាដល់ដំណាក់កាលដែលព្រឹតិ្តការណ៍នេះត្រូវកើតឡើងម្តងទៀតហើយ... ពួកគេទាំងនោះបានដឹងយ៉ាងពិតប្រាកដថាព្រឹត្តិការណ៍នេះនឹងកើតឡើងម្តងទៀត។ ព្រោះមានមនុស្សឆ្លាតវៃជាច្រើនដែលកំពុងស្រាវជ្រាវអំពីព្រឹត្តិការណ៍នេះសម្រាប់ពួកគេ"}\cite{4}។

	នេះគឺជាភស្ថុតាំងដ៏អស្ចារ្យមួយដែលពួកយើងទាំងអស់គ្នាគួរតែដឹងគុណ។ នៅក្នុងកិច្ចសម្ភាសន៍នេះ អ្នកនិពន្ធក៏បានរៀបរាប់អំពីជំនឿរបស់គាត់ថា សង្គ្រាមលោកលើកទីមួយនិងលើកទីពីរគឺជាសង្គ្រាមដែលត្រូវបានគេរៀបចំឡើង ហើយបេសកកម្មរបស់ជនជាតិអង់គ្លេសបុរាណអាចត្រូវបានបង្កើតឡើងជាច្រើនជំនាន់មកហើយ។ ការសម្ភាសន៍នៅឆ្នាំ2010នេះមានរយ:ពេល15ឆ្នាំកន្លងមក។ បើយោងទៅតាមការទស្សន៍ទាយរបស់ជនបង្កប់ យើងនៅមានរយៈពេល5ឆ្នាំទៅមុខទៀត មុននឹងព្រឹត្តិការណ៍ភូគព្ភសាស្ត្រត្រូវបានបញ្ចប់។
\subsection{ចំណេះដឹងអាថ៌កំបាំងនៃជំនឿឌ្រូអីដរបស់លោកខាងលិចអំពីគ្រោះមហន្តរាយ}

	មិនត្រឹមតែរ្វីម៉ាសុននោះទេ សូម្បីតែលោកខាងលេចក៏ដឹងអំពីគ្រោះមហន្តរាយដែលតែងតែកើតឡើងផងដែរ រឿងនេះត្រូវបានគេលាក់យ៉ាងជិត។ អ្នកជំនឿឌ្រូអីដ ដែលជាវប្បធម៌ Celtic មួយតាំងពីបុរាណដែលមានឯកសារចងក្រងយ៉ាងល្អ មានអាយុកាលយ៉ាងហោចណាស់2400ឆ្នាំមកហើយ\cite{7} នៅតែបង្រៀននូវចំណេះដឹងអំពីគ្រោះមហន្តរាយដែលតែងតែកើតឡើងនៅលើផែនដី។ អ្នកជំនឿឌ្រូអីដចុងក្រោយគេបង្អស់ដែលគេស្គាល់គឺលោក Ben Mcbrady។ នៅក្នុងឯកសារ "The Last Druid" ឆ្នាំ1992 គាត់បានចែករំលែកព័ត៌មានអំពីចំណេះដឹងរបស់ពួកឌ្រូអីដ៖ \textit{"បើតាមលំដាប់ ខ្ញុំអាចជាសមាជិកចុងក្រោយគេបង្អស់។ ជំនឿនេះគឺត្រូវបានបង្កើតឡើងបន្ទាប់ពីគ្រោះមហន្តរាយចុងក្រោយគេដែលបានប៉ះពាល់ទូទាំងពិភពលោក។ ឥឡូវនេះដោយមានការប៉ះពាល់ដ៏អាក្រក់​និងគួរអោយភ័យខ្លាចលើភពផែនដីដោយសារព្យុះអគ្គីសនីដ៏កំណាចនិងអាចម៍ផ្កាយ អរិយធម៌របស់មនុស្សកាលពីមុនត្រូវបានបំផ្លាញទាំងស្រុង... ចំណេះដឹងទាំងអស់គឺសេសសល់ដោយសារជំនឿឌ្រូអីដរបស់គេ ប៉ុន្តែពួកយើងគេព្រួយបារម្មណ៍អំពីតារាសាស្ត្រ ពីព្រោះពួកគេបានជួបប្រទះនឹងគ្រោះមហន្តរាយសំខាន់ៗជាច្រើន។ នៅក្នុងជំនឿនេះ​គេគិតថា បើមានចំណេះដឹងពេញលេញអំពីតារាសាស្ត្រ អាចធ្វើអោយពួកគេទស្សន៍ទាយថាគ្រោះមហន្តរាយទាំងនេះអាចកើតឡើងនៅពេលណាបាន និងដើម្បីស្រែងរកវិធីសាស្ត្រដើម្បីការពារខ្លួន។ បើសិនជាអ្នកក្រឡេងទៅមើលសំណង់ថ្មបុរាណដ៏ធំមហិមានៅប្រទេសអៀរឡង់វិញ អ្នកនឹងឃើញថាអ្វីដែលគេហៅថាផ្លូវថ្នូរបុរាណ តាមពិតជាជម្រកការពាររបស់មនុស្សសម័យមុន។ ជម្រករបស់ពួកគេគឺមានកំពស់ខ្ពស់ដើម្បីការពារពីទឹកជំនន់និងអាចម៍ផ្កាយផងដែរ"}\cite{8,9}។

% វាក៏ត្រូវបានគេជឿផងដែរថាហ្វ្រីម៉ាសុនរី(Freemasonry) ខ្លួនឯងពិតជាមកពីពួកឌ្រុយីក(Druids) \cite{10}។
\section{ភស្តុតាងនៃការរៀបចំសម្រាប់គ្រោះមហន្តរាយរបស់លោកខាងលិចនាពេលបច្ចុប្បន្ន}

	ដោយសារតែមហាអំណាចលោកខាងលិចមានជឿថា គ្រោះមហន្តរាយនៃភូគព្ភសាស្ត្រសាកលនឹងកើតឡើងនាពេលឆាប់ៗ យើងអាចរំពឹងថានឹងមានការរៀបចំយ៉ាងសំខាន់ដើម្បីការពារខ្លួនពួកគេសម្រាប់ព្រឹត្តិការណ៍នេះ។  ជាក់ស្តែងមានភស្តុតាងនៅក្នុង domain សាធារណៈអំពីបណ្តាញដ៏ធំទូលាយនៃមូលដ្ឋានក្រោមដីយ៉ាងជ្រៅនៅក្នុងប្រទេសលោកខាងលិចជាច្រើន។ ខណៈដែលការសាងសង់ទីតាំងទាំងនេះពិតជាអាចការពារពួកគេនៅក្នុងសង្គ្រាមនុយក្លេអ៊ែរបានយ៉ាងពិតប្រាកដមែន វាក៏អាចការពារពីគ្រោះមហន្តរាយធម្មជាតិផ្សេងៗគ្នាផងដែរ។ បើផ្អែកទៅលើការបញ្ជាក់របស់សមាជិកជាន់ខ្ពស់ជនជាតិអង់គ្លេសនៃ​ក្រុមហ្វ្រីម៉ាសុនអំពីគម្រោង Camelot\cite{4,6} សេណារីយ៉ូទាំងនេះគឺមិនត្រឹមតែអាចកើតមាននោះទេ​ ប៉ុន្តែជាផែនការដែលបានគ្រោងទុកជាមុន។ អ្វីដែលគួរអោយកត់សម្គាល់ផងដែរនោះ​គឺចំនួនទឹកប្រាក់យ៉ាងមហាសាលដែលត្រូវចំណាយទៅលើការសាងសង់ មើលថែ និងជួសជុលមូលដ្ឋានទាំងនេះ គឺស្រដៀងគ្នាទៅនឹងការបាត់បង់ប្រាក់រាប់រយពាន់លានដុល្លារពីរដ្ឋាភិបាលសហរដ្ឋអាមេរិកនាពេល18ឆ្នាំមកនេះ (ពត៍មាននេះដែលនឹងត្រូវបានគ្របដណ្តប់នៅផ្នែកបន្ទាប់)\cite{11,12,13}។ ឧទាហរណ៍ផ្សេងទៀតអំពីការរៀបចំសម្រាប់ព្រឹត្តិការណ៍វិនាសកម្មមានដូចជាឃ្លាំងក្រោមដីគ្រាប់ពូជ និងចំណេះដឹង។
\subsection{មូលដ្ឋានក្រោមដីនិងក្រោមបាតសមុទ្ររបស់អាមេរិក}

	ការស៊ើបអង្កេតជាសាធារណៈដ៏សុីជម្រៅដែលខ្ញុំបានរកឃើញអំពីមូលដ្ឋានក្រោមដីគឺមានប្រភពពីលោក Richard Sauder។ គាត់ជាអ្នកស្រាវជ្រាវឯករាជ្យជនជាតិអាមេរិកាំងមួយរូប ដែលបានបោះពុម្ពសៀវភៅជាច្រើនក្បាលអំពីមូលដ្ឋាក្រោមដី\cite{22}។ ការងាររបស់លោក Sauder គឺផ្តោតទៅលើការប្រមូលឯកសាររាជរដ្ឋាភិបាលនិងផែនការនានា ដែលផ្តោតទៅលើការសិក្សាអំពីប្រវត្តិសាស្ត្រ រឿងរ៉ាវថ្មីៗប្រចាំថ្ងៃ បច្ចេកវិទ្យា ប្រភពព័ត៌មាន និងការចងក្រងឯកសាររបស់ជនបង្កប់។ តាមការស្រាវជ្រាវរបស់លោក Sauder បានបង្ហាញថាមានបណ្តាញមូលដ្ឋានក្រោមដីនិងក្រោមបាតសមុទ្រធំៗជាច្រើនជុំវិញទឹកដីអាមេរិក (រូបភាពទី \ref{fig:4}) ដែលមានជម្រៅរហូតដល់ទៅជិត4គីឡូម៉ែត្រឯណោះ ហើយមូលដ្ឋានទាំងនេះអាចត្រូវបានតភ្ជាប់ដោយរថភ្លើងម៉ាញេទិចដែលមានល្បឿនលឿនផងដែរ។ មូលដ្ឋានទាំងនេះ​ត្រូវបានសាងសង់យ៉ាងសម្ងាត់ដោយប្រើប្រាស់ \textit{"ហិរញ្ញវត្ថុខ្នាតធំ​ ស្ថាប័នអន្តរជាតិ អន្តរស្ថាប័ន និងតាមល្បិចលាងលុយកខ្វក់"} ដែលគ្រប់គ្រងដោយក្រុមមនុស្សដែលជាម្ចាស់ក្រុមហ៊ុនធំៗនៃសហរដ្ឋអាមេរិក\cite{22}។ លោកស្រី Catherine Austin Fitts បានស្រាវជ្រាវយ៉ាងក្បោះក្បាយទៅលើមូលដ្ឋានទាំងនេះ ហើយជនបង្កប់ម្នាក់ដែលបានសហការជាមួយគាត់បានប៉ានស្មានថាមានមូលដ្ឋានក្រោមដីនិងក្រោមាបាតសមុទ្រយ៉ាងតិច170កន្លែងនៅក្នុងទឹកដីអាម៉េរិច (យើងនឹងបកស្រាយរឿងនេះនៅក្នុងផ្នែកបន្ទាប់)\cite{16,20}។

\begin{figure}[b]
\begin{center}
% \fbox{\rule{0pt}{2in} \rule{0.9\linewidth}{0pt}}
   \includegraphics[width=1\linewidth]{penta.jpg}
\end{center}
   \caption{តើមានអ្វីស្ថិតនៅក្រោមសេតវិមាននិងប៉េនតាហ្គោន? ច្បាស់ណាស់ថានឹងមានបណ្តាញផ្លូវក្រោមដីដែលមានជម្រៅជ្រៅ (រូបភាព៖\cite{31})។}
\label{fig:3}
\label{fig:onecol}
\end{figure}
\begin{figure*}[t]
\begin{center}
% \fbox{\rule{0pt}{2in} \rule{.9\linewidth}{0pt}}
\includegraphics[width=0.9\textwidth]{basescrop.png}
\end{center}
\caption{ផែនទីដែលបង្ហាញទីតាំងយ៉ាងជាក់លាក់នៃការស្រាវជ្រាវរបស់លោកស្រី​ Saunder បង្ហាញអំពីទីតាំងនៃមូលដ្ឋានក្រោមដីនិងក្រោមសមុទ្រ ថែមទាំងមានបណ្តាញសម្រាប់នាវាមុជទឹកថែមទៀត។ លោកស្រី Saunder ជឿជាក់ថា\textit{"នៅមានមូលដ្ឋានច្រើនជាងនេះទៅទៀត"}\cite{22}។}
\label{fig:4}
\end{figure*}

	នេះជាអត្ថបទសក្ខីកម្មមួយចំនួនដែលលោកស្រី Sauder បានស្រាវជ្រាវ​អំពីទំហំដ៏ធំធេងនៃមូលដ្ឋានទាំងនេះ៖
\begin{flushleft}
\begin{enumerate}
    \item ជំរុំ David នៃរដ្ឋ Maryland៖\textit{"ប្រភពព័ត៌មានរបស់ខ្ញុំបានប្រាប់ថា ផ្នែកក្រោមដីនៃជំរុំ David គឺមានទំហំធំធេងនិងស្មុគស្មាញខ្លាំងណាស់ ហើយមានផ្លូវក្រោមដីសម្ងាត់យ៉ាងច្រើនគីឡូម៉ែត្រ យើងសង្ស័យថាគ្មានអ្នកណាម្នាក់អាចស្គាល់ផ្លូវសម្ងាត់ទាំងគ្រប់ជ្រុងជ្រយនោះទេ"}\cite{22}។
    \item សេតវិមាន នៃរដ្ឋ Washington DC៖\textit{"មិត្តភក្តិជិតស្និទ្ធម្នាក់របស់ខ្ញុំបានទៅដល់ទីតាំងនេះនៅពេលដែលលោក Lyndon B. Johnson ជាប្រធានាធិបតីនាទសវត្សរ៍ឆ្នាំ1960។ នៅពេលដែលគាត់បានចូលក្នុងជណ្តើរយន្តនៅសេតវិមាន ភ្លាមៗនោះគាត់ត្រូវបានគេនាំទៅក្រោម ហើយគាត់ជឿថាជណ្តើរយន្តនោះមានជម្រៅរហូតទៅដល់17ជាន់ឯណោះ។ នៅពេលដែលទ្វារជណ្តើរយន្តក្រោមដីបានបើក  គាត់ត្រូវបានគេនាំចេញតាមច្រកមួយដែលហាក់បីដូចជាមានចម្ងាយឆ្ងាយលើសការហួសប្រម៉ាន។ ទ្វារនិងច្រកផ្សេងៗទៀតគឺត្រូវបានបើកចេញតាមច្រកមួយនេះ"}\cite{22}។ បានបង្ហាញក្នុងរូបភាពទី\ref{fig:3}។
   \item បន្ទាយ Meade នៃរដ្ឋ Maryland៖ បើតាមប្រភពមនុស្សម្នាក់ដែលបានចូលទៅក្នុង "បន្ទប់ក្រោមដី" ដោយចៃដន្យក្នុងទសវត្សរ៍ឆ្នាំ1970៖ \textit{"នៅពេលដែលខ្ញុំបានបើកទ្វារខ្ញុំប្រទះឃើញជណ្តើរចុះក្រោមមួយ ខ្ញុំក៏បានដើរទៅជិតហើយសម្លឹងមើលទៅតាមចន្លោះផ្លូវនោះ។ ខ្ញុំមិនបានរាប់ថាវាមានចំនួនប៉ុន្មានជាន់ទៅក្រោមនោះទេ​ ប៉ុន្តែខ្ញុំគិតថាអាចមានរហូតទៅដល់15ទៅ20ជាន់ឯណោះ... ខ្ញុំបានចុះទៅក្រោមមួយជាន់ក៏ប្រទះឃើញទ្វារមួយ... ខ្ញុំបានបើកទ្វារនោះហើយអើតក្បាលមើលឆ្វេងមើលស្តាំ ក៏ប្រទះឃើញផ្លូវក្រោមដីមួយដែលមានចំងាយឡើងដាច់កន្ទុយភ្នែក។ ខ្ញុំច្បាស់ណាស់ថាវាមានចម្ងាយឆ្ងាយជាងអាគារនិងចំណតរថយន្តនៅលើដីទៅទៀត។ ខ្ញុំក៏បានឃើញទ្វារតាមបណ្តោយជញ្ជាំងម្ខាងទៀតផងដែរ ដែលឃ្លាតពីគ្នាប្រហែល9ទៅ12ម៉ែត្រ... ខ្ញុំក៏បានសម្រេចចិត្តដើរចុះទៅក្រោមដើម្បីពិនិត្យមើលពីរបីជាន់ទៀត... អ្វីដែលខ្ញុំឃើញនៅក្នុងជាន់នេះគឺដូចគ្នាទាំងស្រុង ពោលគឺប្លង់ជាន់នេះមិនខុសពីជាន់ខាងលើនោះទេ..ដោយសារតែមិនទាន់អស់ចិត្តខ្ញុំក៏ចុះទៅក្រោមមួយជាន់ទៀត ប៉ុន្តែអ្វីដែលខ្ញុំឃើញគឺមិនខុសពីជាន់ខាងលើទាំងពីរនោះទេ"}\cite{22}។
\end{enumerate}
\end{flushleft}

\begin{figure}[t]
\begin{center}
% \fbox{\rule{0pt}{2in} \rule{0.9\linewidth}{0pt}}
   \includegraphics[width=1\linewidth]{undersea.jpg}
\end{center}
   \caption{រូបគំនូដែលបង្ហាញអំពីមូលដ្ឋានក្រោមសមុទ្រដោយលោក Walter Koerschner។ គាត់ជាអ្នកគូរគំនូម្នាក់សម្រាប់មូលដ្ឋានក្រោមសមុទ្រ Rock-Site របស់កងទ័ពជើងទឹករអាមេរិកនៅ China Lake ដែលជាមជ្ឈមណ្ឌលអាវុធនៃរដ្ឋ California ក្នុងទសវត្សរ៍ឆ្នាំ1960។ បើយោងទៅតាមប្រភពមួយរបស់លោកស្រី Sauder បានបង្ហាញថា មានមូលដ្ឋានក្រោមដីមួយដែលមានជម្រៅជាងមួយគីឡូម៉ែត្រឯណោះនៅ China Lake\cite{22,23}។}
\label{fig:5}
\label{fig:onecol}
\end{figure}

	លោកស្រី​ Saunder ក៏បានទទួលពត៍មានអំពីរថភ្លើងម៉ាញែតិចក្រោមដីដែលមានល្បឿនជាង2000គីឡូម៉ែត្រក្នុងមួយម៉ោងផងដែរ ដែលបាងសាងសង់ក្រោមបាតសមុទ្រ (រូបភាពទី \ref{fig:5}) និងផ្លូវនាវាមុជទឹកក្រោមសមុទ្រដែលអាចធ្វើដំណើរឆ្ពោះទៅលើគោកបាន។ បើយោងទៅតាមពត៍មានមួយទៀតអំពីមូលដ្ឋានក្រោមទឹកនៅឈូងសមុទ្រម៉ិកស៊ិក លោកស្រី Saunder បាននិយាយថា\textit{"បន្ទាប់ពីមានការផ្សព្វផ្សាយអស់រយ:ពេលកន្លះឆ្នាំអំពីមូលដ្ឋានក្រោមទឹកនិងក្រោមដី មានបុរសម្នាក់បានទាក់ទងមកខ្ញុំហើយបកស្រាយថា គាត់បានដឹងអំពីគម្រោងមូលដ្ឋានក្រោមទឹកដ៏សែនចម្លែកមួយនេះ ... គាត់បានបញ្ជាក់ថា គម្រោងនេះស្ថិតនៅក្រោមបាតសមុទ្រនៃឈូងសមុទ្រម៉ិកស៊ិក ហើយក្រុមហ៊ុន Parsons គឺជាអ្នកម៉ៅការ។ គាត់បានបន្តថាក្រុមហ៊ុន Parsons បានទិញឧបករណ៍ពិសេសមួយចំនួន ក្នុងគោលបំណងប្រើប្រាស់នៅក្រោមបាតសមុទ្រនៅក្នុងជម្រៅរហូតទៅដល់850ម៉ែត្រ ... ឧបករណ៍ទាំងនេះមានលក្ខណៈពិសេសដែលយើងអាចសន្និដ្ឋានបានថា នៅពេលដែលសាងសង់រួចមនុស្សនឹងអាចរស់នៅបាន"}\cite{22}។
\begin{figure}[t]
\begin{center}
% \fbox{\rule{0pt}{2in} \rule{0.9\linewidth}{0pt}}
   \includegraphics[width=1\linewidth]{sub.jpg}
\end{center}
   \caption{រូបភាពបង្ហាញអំពីជម្រៅបាតសមុទ្រក្រោមទឹកដោយលោក Walter Koerschner\cite{22,23}។}
\label{fig:6}
\label{fig:onecol}
\end{figure}
\begin{figure}[t]
\begin{center}
% \fbox{\rule{0pt}{2in} \rule{0.9\linewidth}{0pt}}
   \includegraphics[width=1\linewidth]{iran.jpeg}
\end{center}
   \caption{វីដេអូផ្លូវការមួយរបស់ប្រទេសអុីរ៉ង់បង្ហាញអំពី "ទីក្រុងមីស៊ីល" ក្រោមដី\cite{39,40}។}
\label{fig:12}
\label{fig:onecol}
\end{figure}
	បើសិនជាមានបណ្តាញមូលដ្ឋានក្រោមដីនិងក្រោមសមុទ្រឆ្លងទ្វីបចំនួនជាង170ទីតាំងដែលគេបានសាងសង់មានជម្រៅរាប់គីឡូម៉ែត្រទៅក្រោមដី​ ដែលបានតភ្ជាប់ដោយបណ្តាញរដ្ឋភ្លើងម៉ាញេតិចមានល្បឿនលឿនជាងសម្លេងជាងប្រាំដង ហើយបានសាង់សង់ឡើងយ៉ាងប្រយ័ត្នបំផុតមែននោះ មានន័យថាមនុស្សទូទៅនាពេលបច្ចុប្បន្ននេះគឺមិនដឹងអ្វីនោះទេ។ មិនត្រឹមតែមិនដឹងថាមានអ្វីនៅក្រោមពួកគេនោះទេ ថែមទាំងមិនដឹងថានឹងមានរឿងអ្វីកើតឡើងនាពេលខាងមុខ ខណៈដែលពួកគេមានជំជឿទៅអ្នកដឹងនាំទាំងខ្វះការពិចារណា។

	កំណត់សម្គាល់បន្ថែម - បណ្តាញផ្លូវក្រោមដីដ៏ធំដែលត្រូវបានបង្ហាញយ៉ាងច្បាស់នៅក្នុងជម្លោះដែលកំពុងតែផ្ទុះនៅមជ្ឈិមបូព៌ (ផ្លូវក្រោមដីរបស់ពួក Hamas ក្នុង Gaza Strip\cite{38} និង "ទីក្រុងមីស៊ីល" ក្រោមដីរបស់ប្រទេសអុីរ៉ង់ (រូបភាពទី \ref{fig:12}) \cite{39,40})។ នេះជាភស្តុតាងយ៉ាងជាក់លាក់មួយដែលបញ្ជាក់អំពីមូលដ្ឋានក្រោមដីដែលគេបានសាងសង់រួចមកហើយ។ យ៉ាងនេះហើយបានជាយើងនៅមានចម្ងល់ថា តើប្រទេសអ្នកមានបានសាងសង់មូលដ្ឋានបែបណាខ្លះទៅនាពេលបច្ចុប្បន្ននេះ។ 
\subsection{ភស្តុតាងបន្ថែមដែលបញ្ជាក់អំពីការរៀបចំលេណដ្ឋានក្រោមដីដើម្បីការពារគ្រោះមហន្តរាយ}

\begin{figure}[t]
\begin{center}
% \fbox{\rule{0pt}{2in} \rule{0.9\linewidth}{0pt}}
   \includegraphics[width=1\linewidth]{tyrol.jpg}
\end{center}
   \caption{លេណដ្ឋានក្រោមដីនៅតំបន់ Tyrol ខាងត្បូងនៃប្រទេសស្វុីស។ ប្រទេសស្វុីសដែលគ្របដណ្តប់ទៅដោយជួរភ្នំ Apls អឺរ៉ុប ត្រូវបានគេដឹងអំពីការលាក់លានដ្ឋានក្រោមភ្នុំ\cite{32}។}
\label{fig:7}
\label{fig:onecol}
\end{figure}

\begin{figure}[t]
\begin{center}
% \fbox{\rule{0pt}{2in} \rule{0.9\linewidth}{0pt}}
   \includegraphics[width=1\linewidth]{svalbard.jpg}
\end{center}
   \caption{ឃ្លាំងគ្រាប់ពូជក្រោមដីសកលនៅ (Svalbard) ប្រទេសន័រវែស ដែលមានគ្រាប់ពូជជាងមួយលានឯណោះ\cite{24}។ រឿងនេះធ្វើអោយយើងមានចម្ងល់ថា តើនឹងមានគ្រោះមហន្តរាយបែបណាបាងជាគេសាងសង់មូលដ្ឋាននេះ។}​
\label{fig:8}
\label{fig:onecol}
\end{figure}

	មិនត្រឹមតែមូលដ្ឋានក្រោមដីរបស់អាម៉េរិចនោះទេ យើងមានតម្រុយផ្សេងទៀតជាច្រើនដែលបញ្ជាក់អំពីការរៀបចំការពារគ្រោះមហន្តាយជំវិញពិភពលោក ឧទាហរណ៍​ដ៏ល្អគឺនៅប្រទេសន័រវែស ស្វីស ស៊ុយអែត និងហ្វាំងឡង់៖

\begin{flushleft}
\begin{enumerate}
    \item គម្រោង Camelot បានចែករំលែកសក្ខីកម្មមួយពីអ្នកនយោបាយជនជាតិន័រវែសម្នាក់\cite{25,26} ដែលគេបានស្គាល់អត្តសញ្ញាណ ប៉ុន្តែអត្តសញ្ញាណរបស់គាត់ត្រូវគេរក្សាជាការសម្ងាត់។ គាត់បានអះអាងថាប្រទេសន័រវែសមានមូលដ្ឋានក្រោមដីធំៗចំនួន18ទីតាំង ហើយប្រទេសនេះ (មានទាំងប្រទេសអុីស្រាអែល និង "ប្រទេសជាច្រើនផ្សេងទៀត") កំពុងតែសាងសង់មូលដ្ឋានទាំងនេះ ដើម្បីការពារគ្រោះមហន្តរាយធម្មជាតិណាមួយ។ លោក Richard Sauder បានទទួលសក្ខីកម្មមួយពីបុរសម្នាក់ផងដែរ ដែលធ្លាប់ចូលទៅក្នុងមូលដ្ឋានក្រោមដីដ៏ធំមួយនៅក្នុងប្រទេសន័រវែស មូលដ្ឋាននេះត្រូវបានសង់សង់ដោយជីកក្រោមភ្នុំតែម្តង\cite{22}។ 
    \item ប្រទេសស្វីស ត្រូវបានគេស្គាល់ថាមានលេណឋ្ឋានក្រោមដីនុយក្លេអ៊ែរជាច្រើនកន្លែងដែលបានសាងសង់នៅលើជួរភ្នំ Alps\ref{fig:7}។ ចំនួនលេណឋ្ឋានក្រោមដីទាំងនេះគឺលើសពី37មុឺនកន្លែងឯណោះ ដែលគ្រប់គ្រាន់សម្រាប់អោយប្រជាពលរដ្ឋទាំងអស់ជ្រកកោន\cite{27}។
    \item ប្រទេសស៊ុយអែតនិងហ្វាំងឡង់ ក៏មានលេណឋ្ឋានក្រោមដីគ្រប់គ្រាន់សម្រាប់អោយប្រជាពលរដ្ឋនៅក្នុងទីក្រុងធំៗជ្រកកោនផងដែរ\cite{27}។
\end{enumerate}
\end{flushleft}

	អ្នកជំនួញធំៗយនៅ Silicon Valley ក៏ដឹងអំពីលេណឋ្ឋានក្រោមដីទាំងនេះផងដែរ។ តាមរយៈការរាយការណ៍មួយបានបញ្ជាក់ថា \textit{"លោក Reid Hoffman ស្ថាបនិកម្នាក់នៃ LinkedIn និងជាអ្នកវិនិយោគដ៏ល្បីល្បាញម្នាក់បានប្រាប់ទៅកាសែត The New Yorker កាលពីដើមឆ្នាំនេះថា គាត់ប៉ាន់ប្រមាណថាមានមហាសេដ្ឋីជាង50\% នៅ Silicon Valley បានទិញធានារ៉ាប់រង "គ្រោះមហន្តរាយ" ដូចជាលេណឋ្ឋានក្រោមដី... បើយោងទៅតាមលោក Jim Dobson អ្នកការសែត Forbes ម្នាក់បានរៀបរាប់ថា មហាសេដ្ឋីជាច្រើនមានយន្តហោះឯកជន "ដែលត្រៀមខ្លួចជាស្រេចដើម្បីចេញដំណើរភ្លាមៗបើមានរឿងអ្វីមួយកើតឡើង"។ ហើយមហាេសដ្ឋាទាំងនេះក៏មានម៉ូតូ អាវុធ និងម៉ាស៊ីនភ្លើងផងដែរ"}\cite{28}។

	គម្រោងបណ្ណសារធំៗជាច្រើនដូចជាឃ្លាំងចំណេះដឹងសកលដែលគ្រប់គ្រងដោយ Arch Mission Foundation\cite{29} និង Svalbard Global Seed Vault\cite{30} ដែលកំពុងត្រៀមខ្លួនដើម្បីរក្សាទុកទ្រព្យសម្បត្តិសំខាន់ៗរបស់មនុស្សនៅក្នុងករណីមានគ្រោះមហន្តរាយសាបសូន្យណាមួយ។
\begin{figure*}[t]
\begin{center}
% \fbox{\rule{0pt}{2in} \rule{.9\linewidth}{0pt}}
\includegraphics[width=0.9\textwidth]{govcrop2.png}
\end{center}
   \caption{របាយការណ៍ចំណូល ចំណាយ និងការចំណាយទៅលើមូលដ្ឋានក្រោមដីសម្ងាត់របស់រដ្ឋាភិបាលសហរដ្ឋអាមេរិកពីឆ្នាំ1998-2023\cite{19}.}
   \label{fig:9}
\end{figure*}
\section{យន្តការមូលនិធិប្រជាធិបតេយ្យសម្រាប់មូលដ្ឋានក្រោមដីធំៗ}

	តើបណ្តាញឆ្លងទ្វីបដ៏ធំធេងនៃមូលដ្ឋានក្រោមដីនិងក្រោមសមុទ្រជាង170ទីតាំងទាំងនេះ បានប្រាក់សាងសង់តាមរយ:វិធីណាដែរអាចលាក់បាំងពីប្រជាជនបាន? មានឯកសារមួយដែលអាចបញ្ជាក់អំពីប្រភពនិងការចំណាយដ៏ធំមហិមាទៅលើគម្រោងទាំងនេះ។ នៅឆ្នាំ2017 លោកស្រី​ Catherine Austin Fitts ដែលជាអ្នកវិនិយោគជនជាតិអាម៉េរិចកាំងម្នាក់ និងជាអតីតមន្ត្រីសាធារណៈក្នុងអំឡុងប្រធានាធិបតីលោក George W Bush និង Mark Skidmore បានរកឃើញការចំណាយលាក់បាំងមួយដែលមានជំនួនទឹកប្រាក់រហូតទៅដល់21លានលានដុល្លារនាអំឡុងឆ្នាំហិរញ្ញវត្ថុ1998-2015\cite{11,12,13}។

	យោងតាមរបាយការណ៍របស់ពួកគេ \textit{"លោក Scott Paltrow  បានផ្សព្វផ្សាយអត្ថបទមួយនៅថ្ងៃទី7ខែតុលាឆ្នាំ2016 តាមរយ:ការសែត Reuters (2016) ដែលបានរាយការណ៍ថានៅក្នុងឆ្នាំហិរញ្ញវត្ថុ2015 យោធារបស់អាម៉េរិចទទួលបានប្រាក់រហូតទៅដល់6.5លានលានដុល្លារសម្រាប់ចំណាយក្រៅផ្លូវការ "ដោយប្រើវិធីសាស្ត្រថា សៀវភៅគណនេយ្យរបស់យោធាអាម៉េរិចគឺត្រឹមត្រូវហើយ។" បើផ្អែកទៅលើការផ្តល់ថវិកាជាក់ស្តែងរបស់យោធានៅឆ្នាំនោះគឺមានចំនួនត្រឹមតែ112ពាន់លានដុល្លារតែប៉ុណ្ណោះ មានន័យថាវាខុសគ្នាខ្លាំងណាស់... ក្រសួងការពារជាតិត្រូវបានព័ត៌មានជាច្រើនលាតត្រដាងនូវភាពមិនប្រក្រតីនៃរបាយការណ៍គណនេយ្យ ក្នុងថ្ងៃទី10ខែកញ្ញាឆ្នាំ2001  បន្ទាប់ពីរដ្ឋមន្ត្រីការពារជាតិលោក Donald Rumsfeld បានថ្លែងការណ៍អំពីការបាត់បង់ដាននៃការចំណាយប្រាក់ដែលមានចំនួយន2.3លានលានដុល្លារនៅក្នុងការប្រជុំសភា (C-SPAN 2014)... ការទទួលស្គាល់កំហុសនេះបានធ្វើអោយព័ត៍មានជាច្រើនធ្វើការផ្សព្វផ្សាយនៅថ្ងៃនោះ ប៉ុន្តែរឿងនេះត្រូវបានបំភ្លេចវិញដោយសារព្រឹត្តិការណ៍ដ៏អកុសលនៅថ្ងៃទី11ខែកញ្ញា ដែលល្បីល្បាញទូទាំងពិភពលោក... នៅពេលដែលលោកសាស្រ្តាចារ្យ Mark Skimore បានដឹងអំពីភាពមិនប្រក្រតីរនៃការចំណាយរបស់យោធាដែលមានចំនួនទឹកប្រាក់6.5លានលានដុល្លារ គាត់ក៏បានទាក់ទងទៅកាន់លោកស្រី Fitts ហើយពួកគាត់ទាំងពីរនាក់បានសម្រេចចិត្តសហការណ៍គ្នាដើម្បីស្រាវជ្រាវអំពីការចំណាយមិនប្រក្រតីទាំងឡាយរបស់ក្រសួងការពារជាតិ និងក្រសួងលំនៅឋ្ឋាននិងអភិវឌ្ឍន៍ទីក្រុងសហរដ្ឋអាម៉េរិច។  ក្នុងអំឡុងពេល6ខែបន្ទាប់មក លោក Skidmore និងលោកស្រី Fitts ដែលជានិស្សិតពីរនាក់ដែលបានបញ្ចប់ការសិក្សារ បានប្រមូលទ័ន្នន័យរបស់រដ្ឋាភិបាលអាម៉េរិចដែលបង្ហាញថាការចំណាយពីឆ្នាំ1998-2016 មានរហូតទៅដល់21លានលានដុល្លារគឺគ្មានឯកសារត្រឹមត្រូវនោះទេ"}\cite{12}។  បើប្រៀបធៀបទៅនឹងការចំណាយរយ:ពេល18ឆ្នាំ ពីឆ្នាំ1998-2015 ការចំណាយដែលទទួលស្គាល់ជាសាធារណៈរបស់អាម៉េរិចគឹត្រឹមតែ40.8លានលានដុល្លារតែប៉ុណ្ណោះ\cite{15}។ នេះបង្ហាញថាចំនួនប្រាក់ដែលបានចំណាយយ៉ាងសម្ងាត់របស់រដ្ឋាភិបាលដើម្បីសាងសង់មូលឋ្ឋានក្រោមដីគឺមានចំនួនពាក់កណ្តាល។ អ្វីដែលគូរអោយគត់សម្គាល់នោះ ការចំណាយយ៉ាងសម្ងាត់នេះកើតឡើងស្របពេលដែលរដ្ឋាភិបាលអាមេរិចកំពុងខ្វះខាតថវិកា ដែរយើងជឿជាក់ថាកើតឡើងតាំងពីមុនឆ្នាំ1998មកម្លេះរហូតមកដល់នាពេលបច្ចុប្បន្ន។ នេះបញ្ជាក់ថាចំនួនប្រាក់សរុបដែលបានចំណាយលើមូលដ្ឋានទាំងនេះគឺច្រើនជាង21លានលានដុល្លារឯណោះ។ បើយើងប្រៀបធៀបទៅនឹងការចំណាយសម្ងាត់នៅឆ្នាំ1998 ការចំណាយពីឆ្នាំ2016-2023 គឹមានចំនួន36.6លានលានដុល្លារ។

	នៅឆ្នាំ2021 លោក Mark Skidmore បានផ្សព្វផ្សាយឡើងវិញអំពីការស្រាវជ្រាវរបស់គាត់ដែលទាក់ទងនឹងការប្រកាសរបស់កាសែត Bloomberg ដែលបានបញ្ជាក់ថាក្នុងអំឡុងឆ្នាំហិរញ្ញវត្ថុ2017-2019 រដ្ឋាភិបាលអាម៉េរិច បានកែតម្រូវអំពីការចំណាយវាយ ដែលមានចំនួនរហូតទៅដល់98.7លានលានដុល្លារ\cite{17,18}។ បើយើងពិចារណាទៅលើការក្លែងបន្លំក្រដាសប្រាក់ដុល្លារតាមរយៈប្រព័ន្ធធនាគារកណ្តាល ដែលបានកើតឡើងអស់រយៈពេលជាងមួយសតវត្សមកហើយគឹចាប់តាំងពីការបង្កើតធនាគារសហព័ន្ធអាម៉េរិចក្នុងឆ្នាំ1913\cite{37} មានន័យថាគណនេយ្យចំណាយសាធារណៈគឺជាការបោកប្រាស់ទាំងស្រុង តាមពិតទៅរូបិយប័ណ្ណនិងរដ្ឋាភិបាលសហរដ្ឋអាមេរិកគ្រាន់តែជាប្រព័ន្ធចែកចាយនោះទេ អ្នកដែលគ្រប់គ្រងពិតប្រាកដអាចដកលុយចំនួនប៉ុណ្ណាក៏បានដែរ។
\section{ក្រុមពូជវង្សជូវ៖ អត្តសញ្ញាណពិតនៃស្តេចប្រទេសលោកខាងលិច}

	តើអ្នកណាជាអ្នកដឹកនាំពិតប្រាកដ? យើងមិនអាចដឹងបាននោះទេ ពីព្រោះ​ស្តេច​ប្រទេសលោក​ខាង​លិច​តែងតែ​លាក់​ខ្លួន​នៅក្នុង​ស្រមោលជានិច្ច។ ខណៈពេលដែលមានទ្រឹស្ដីផ្សេងៗគ្អាអំពីអត្តសញ្ញាណពិតនៃស្តេចទាំងនេះ អ្នកខ្លះថាគឺជាជនល្បីៗ អ្នកខ្លះទៀតថាស្តេចទាំងនោះគឺជាមនុស្សក្រៅភព។ ចម្លើយល្អបំផុតដែលខ្ញុំមានសម្រាប់សំណួរនេះគឺស្ថិតនៅក្នុងការផ្សព្វផ្សាយរបស់ជនអនាមិកម្នាក់ដែលប្រើឈ្មោះបំភ័ន្ត "Amallulla"។ ការស្រាវជ្រាវរបស់គាត់បានបញ្ជូលនូវអត្ថបទអ្នកនិពន្ធជាង20នាក់និងឯកសារ "ដែលមិនអាចជំនួសបាន" ជាង50 ដែលបកស្រាយទៅលើប្រធានបទនៃប្រវត្តិសាស្ត្របុរាណនិងសម័យទំនើប អបិយជំនៅ និងនយោបាយប្រទេសលោកខាងលិច\cite{33,34}។ ការងាររបស់គាត់គឺដូជជាគ្រួ "ហោរាសាស្រ្ត" ទាក់ទងទៅនឹងគ្រោះមហន្តរាយភូគព្ភសាស្ត្រដែលជិតកើតឡើង - តែអត្ថបទរបស់គាត់​\textit{} បកស្រាយបានល្អជាងខ្ញុំទៅទៀត។

	លោក Amallulla បានកំណត់អត្តសញ្ញាណក្រុមនយោបាយប្រទេសលោកខាងលិចថាមានបីក្រុម ដែលគាត់ហៅថា "ពូជពង្សជូវ" ក្រុមនេះមានចំណេះដឹងអំពី "ចុងបញ្ចប់នៃពិភពលោក" -ឬគ្រោះមហន្តរាយលើផែនដីដែលតែងតែកើតមាន។ គាត់ជឿថាក្រុមទាំងបីនេះកំពុងតែគ្រប់គ្រងប្រទេសលោកខាងលិចនាពេលបច្ចុប្បន្ន ប៉ុន្តែគាត់បានបែងចែកក្រុមនេះជាបីផ្នែកដោយផ្អែកទៅលើសញ្ជាតិ ភាពខុសគ្នានៃអត្តសញ្ញាណប្រវត្តិសាស្ត្រ ការខ្វែងគំនិតគ្នាដែលអាចមាននៃអតីតកាល និងភាពខុសគ្នាន័យជំនឿនិងការអនុវត្ត។
	ក្រុមទាំងបីនេះអាចបែងចែកដូចខាងក្រោម៖
\begin{flushleft}
\begin{enumerate}
    \item \textbf{អ្នកធានាគារ}: អភិជនរ៉ូមាំងបុរាណដែលបានប្រែក្លាយទៅជា Knights Templar និងក្រុមហ្វ្រីម៉ាសុននៅអាម៉េរិច។
    \item \textbf{អ្នកគិតគូរ}: ពួក Rosicrucians និងក្រុមហ្វ្រីម៉ាសុនខាងត្បូងនៃអាម៉េរិច។
    \item \textbf{ពួកយេស៊ូអ៊ីត និងប៉ូបខ្មៅ}: ក្រុមពូជវង្សជូវនៅក្នុងសាសនាកាតូលិករ៉ូមាំង។
\end{enumerate}
\end{flushleft}
	សព្វថ្ងៃក្រុមទាំងបីនេះបង្កើតបានទៅជាក្រុម Illuminati នៅអឺរ៉ុប ក្រុមហ្វ្រីម៉ាសុន និង CIA។ បើតាមការបកស្រាយរបស់ Amallulla \textit{"នាពេលបុច្ចប្បន្ននេះបើមានករណីដែលពិភពលោកត្រូវបានបញ្ជប់  ក្រុមពូជវង្សជូវ នឹងលាក់ខ្លួននៅទីកន្លែងដែលគ្មានអ្នកណាដឹងនោះឡោយ សូម្បីតែប្រធានាធិបតេយ្យអាម៉េរិចដែលកំពុងកាន់តំណែងក៏មិនអាចដឹងអំពីទីតាំងនេះផងដែរ។ មានន័យថាក្រុមទាំងនេះពិតជាពូកែខាងលាក់ខ្លួនមិនអាចអោយអ្នកណាដឹងនោះឡើយ។ \textbf{ក្រុមពូជវង្សជូវមិនត្រឹមតែគ្រប់គ្រងយោធានិងរដ្ឋាភិបាលសហរដ្ឋអាមេរិកប៉ុណ្ណោះទេ ប៉ុន្តែតាមរយៈអំណាចនៃរូបិយប័ណ្ណ ក្រុមហ៊ុនធំៗ និងរដ្ឋាភិបាលសាធារណរដ្ឋដែលពួកគេបានបង្កើត (ដោយដឹងថាអ្នកនយោបាយគឺពុករលួយ ដូច្នេះហើយវាមានការងារស្រួលដើម្បីគ្រប់គ្រង) តាមពិតទៅក្រុមនេះកំពុងតែគ្រប់គ្រងប្រទេសលោកខាងលិចទាំងមូល}"}\cite{33,34}។

\begin{figure}[t]
\begin{center}
% \fbox{\rule{0pt}{2in} \rule{0.9\linewidth}{0pt}}
   \includegraphics[width=1\linewidth]{illuminati.jpg}
\end{center}
   \caption{តើពូជពង្សជូវជាអ្នកណា? (រូបភាពទី៖ \cite{35})}
\label{fig:10}
\label{fig:onecol}
\end{figure}

\begin{figure}[t]
\begin{center}
% \fbox{\rule{0pt}{2in} \rule{0.9\linewidth}{0pt}}
   \includegraphics[width=1\linewidth]{pike.jpg}
\end{center}
   \caption{ភ្នំ Pike Peak Batholith ដ៏ល្បីល្បាញ ដែលត្រូវបានគូសជាពណ៌ក្រហមនៅផ្នែកខាងលិចសហរដ្ឋអាមេរិក\cite{36}។ តើវាជាការពិតនោះទេដែលសហរដ្ឋអាមេរិកពិតជាត្រូវបានបង្កើតឡើងដើម្បីគ្រប់គ្រងទីតាំងនេះ?}
\label{fig:11}
\label{fig:onecol}
\end{figure}

	យោងតាម Amallulla  បានបកស្រាយថាក្រុមមួយនេះជឿទៅលើអបិយជំនឿ ហើយពួកគេស្អប់សាសនានិងបានកែប្រែគម្ពីក្នុងសាសនាធំៗរទូទាំងពិភពលោក ដើម្បីជាផលប្រយោជន៍សម្រាប់ពួកគេ។ លើសពីនេះទៅទៀត ពួកគេគ្មានធម៍មេត្តាចំពោះសត្រូវនោះឡើយ៖ \textit{"\textbf{ក្នុងរយៈពេលជាង2600ឆ្នាំកន្លងមក ពួកគេបានសម្លាប់ជនណាទាំងអស់ដែលមានចំណេះដឹងអំពី "ចុងបញ្ចប់នៃពិភពលោក"។ នៅក្នុងអត្ថន័យនេះ ខ្ញុំមិនត្រឹមតែសំដៅទៅកាន់ពួកឌ្រូអីដ ជ្រិហ្វ-កាប៉ាលីស្ត ជនជាតិអេហ្ស៊ីបបុរាណ ពួកអារ៉ាប់ ពួកជំនឿអាថ៍កំបាំងឥណ្ឌា ពួកលលាដ៍ក្បាលវែង (Elongated Skulls) នៅអាមេរិកខាងត្បូង និងពួកម៉ាយ៉ានៅអាមេរិកកណ្តាលផងដែរ។ ភស្តុតាងដែលបញ្ជាក់ថាពួកគេបានកម្ទីចចោលនូវអរិយធម៌ដ៏រីកចម្រើនរបស់មនុស្សសម័យមុន នៅអាមេរិកខាងជើងគឺមានលើសលប់។ ពួកគេធ្វើនេះដើម្បីរក្សាទីកន្លែងសម្រាប់ព្រឹត្តិការណ៍ "ចុងបញ្ចប់នៃពិភពលោក"។ ការប្រល័យពូជសាសន៍ជនជាតិដើមអាម៉េរិច "ជនជាតិឥណ្ឌា" គ្រាន់តែជាប្រតិបត្តិការចុងក្រោយគេប៉ុណ្ណោះ}"}\cite{33,34}។
លោក Amallulla ក៏ជឿដែរថាគម្រោង "សហរដ្ឋអាមេរិក" ទាំងមូលត្រូវបានបង្កើតឡើងដើម្បីជាការធានាសម្រាប់ការគ្រប់គ្រងភ្នុំ "Pike Peak Batholith" ដែលជាជួរភ្នំថ្មក្រានីតនៅឯ Rocky Mountains ដែលអាចការពារពួកគេបានយ៉ាងល្អបំផុតពីគ្រោះមហន្តរាយភូគព្ភសាស្ត្រ (រូបភាពទី \ref{fig:11})។ យោងតាម Amallulla \textit{"អ្វីដែលយើងបានគិតថាជាសង្គ្រាមស៊ីវិលនៅអាម៉េរិច តាមពិតទៅវាជាសង្គ្រាមរវាងក្រុមអ្នកធនាគារនិងក្រុមអ្នកគិតគូរ ដែរបានប្រយុទ្ធគ្នាគឺដើម្បីគ្រប់គ្រងសហរដ្ឋអាមេរិកនិងគ្រប់គ្រងជួរភ្នុំ Pike Peak Batholith ដែលជាភ្នំថ្មក្រានីតដ៏ប្លែកជាងគេបំផុតនៅលើពិភពលោក... ក្រៅពីភ្នុំមួយនេះ​ គ្មានភ្នុំថ្មក្រានីតផ្សេងទៀតនោះទេដែលមានកំពស់ខ្ពស់ហើយឆ្ងាយពីសមុទ្រ។  វាជាទីតាំងល្អបំផុតដើម្បីការពារជីវិតនាព្រឹត្តិការណ៍ បំលែងនៃសំបកផែនដី"}\cite{33,34}។ ការស្រាវជ្រាវរបស់ Amallulla បានបង្ហាញថាមានប្រព័ន្ធផ្លូវក្រោមដីដ៏ធំទូលាយ ដែលគេបានសាងសង់នៅខាងក្រោមនិងជុំវិញជួរភ្នុំនេះនាពេលបច្ចុប្បន្ន\cite{36}។

\section{សេចក្តីសន្និដ្ឋាន}

	នៅក្នុងអត្ថបទនេះខ្ញុំបានរៀបរាប់យ៉ាងលម្អិតអំពីសក្ខីកម្មផ្សេងៗដែលបញ្ជាក់ថា អភិជនប្រទេសលោកខាងលិចបានរក្សាទុកនូវចំណេះដឹងអំពីគ្រោះមហន្តរាយដែលតែងតែកើតឡើងលើផែនដីអស់រយៈពេលរាប់ពាន់ឆ្នាំមក។ ពួកគេជឿជាក់ថាគ្រោះមហន្តរាយមួយទៀតនឹងកើតឡើងនាពេលឆាប់ៗ ដូចច្នេះហើយបានជាពួកគេបានសាងសង់ជម្រកក្រោមដីដ៏ធំទូលាយដើម្បីការពារខ្លួន។ មិនត្រឹមតែប៉ុណ្ណោះនោះទេ ពួកគេកំពុងតែរៀបចំផែនការដើម្បីទាញយកផលប្រយោជន៍ពីគ្រោះមហន្តរាយនេះ ទាំងនយោបាយនិងយោធាដើម្បីគ្រប់គ្រងពិភពលោកទាំងមូល។ ខ្ញុំបានលើកឡើងអំពីតម្រុយខ្លះៗដែលបានបង្ហញាអំពីរបៀបនៃការផ្តួចផ្តើមនៅសហរដ្ឋអាម៉េរិច ហើយក៏បានយោងទៅតាមទ្រឹស្តីដែលមិនសូវមានការចាប់អារម្មណ៍អំពីខ្សែស្រឡាយដែលកំពុងតែគ្រប់គ្រងនាពេលបច្ចុប្បន្ន។ ចំពោះអ្នកដែលមានចំណាប់អារម្មណ៍បន្ថែម មានព័ត៌មានជាច្រើនទៀតដែលខ្ញុំមិនបានបកស្រាយ លោកអ្នកអាចអានវាបានដោយគ្រាន់តែទៅកាន់ចំណុច "ឯកសារយោង" ខាងក្រោម។
ទិន្នន័យដែលល្អបំផុតសម្រាប់គណនាអំពីព្រឹត្តិការណ៍ភូគព្ភសាស្ត្រ​ដែលនឹងកើតមាននាពេលឆាប់ៗនេះ គឺដែនម៉ាញេទិករបស់ផែនដីដែលកំពុងផ្លាស់ប្តូរយ៉ាងឆាប់រហ័ស។ យើងអាចគណនាវាបានមិនត្រឹមតែតាមរយៈចលនាប៉ូលម៉ាញេទិកខាងជើងនោះទេ (រូបភាពទី \ref{fig:13}) ភាពមិនប្រក្រតីនៃដែនម៉ាញេទិកភូគព្ភសាស្ត្រអាត្លង់ទិកខាងត្បូង និងការចុះខ្សោយនិងការខូចទ្រង់ទ្រាយនៃដែនម៉ាញេទិកក្នុងរយៈពេល400ឆ្នាំចុងក្រោយនេះក៏អាចគណនាបានផងដែរ\cite{3}។ លោកអាចចូលទៅមើលទិន្នន័យវិទ្យាសាស្ត្របែបនេះដែលត្រូវបានពិភាក្សាយ៉ាងហ្មត់ចត់នៅក្នុងអត្ថបទ ECDO ចំនួនពីរដំបូងរបស់ខ្ញុំ តាមរយៈគេហទំព័រ\cite{3}។

\begin{figure}[t]
\begin{center}
% \fbox{\rule{0pt}{2in} \rule{0.9\linewidth}{0pt}}
   \includegraphics[width=1\linewidth]{npw.jpg}
\end{center}
   \caption{ទីតាំងនៃប៉ូលម៉ាញេទិកខាងជើងពីឆ្នាំ1590-2025 ដែលបង្ហាញរៀងរាល់ប្រាំឆ្នាំម្តង។ ចលនារបស់វាបានចាប់ផ្តើមបង្កើនលឿនយ៉ាងឆាប់រហ័សនៅឆ្នាំ1975។}
\label{fig:13}
\label{fig:onecol}
\end{figure}

	ដើម្បីបញ្ជប់សេចក្ដីខ្ញុំនឹងសម្រង់​សម្តីរបស់ Amallulla មួយសម្រាប់អ្នកដែលគាត់បានពន្យល់អំពី \textit{"\textbf{អ្វីៗទាំងអស់គឺមានប្រភពតែមួយ}"}: \textit{"នៅទីនេះខ្ញុំបានព្យាយាមជំរុញនូវការស្រមៃរបស់អ្នក អោយដល់ដែនកំណត់ខ្ពស់បំផុត។ អ្នកត្រូវតែបំភ្លេចអំពីពិភពលោកដែលអ្នកបានស្គាល់តាំងកុមារភាពចោលទៅ។ វាគឺជារឿងដែលគេបង្កើតឡើងដើម្បីបោកប្រាស់យើង ស្រដៀងទៅនឹងអ្វីដែលបានបង្ហាញនៅក្នុងរឿង The Matrix ដែលមានគោលបំណងដើម្បីបោកប្រាស់អ្នករហូតដល់ទីបញ្ចប់។ ពេលខ្លះខ្ញុំប្រាថ្នាថា អ្វីដែលខ្ញុំបានចែករំលែកជាមួយអ្នកគឺជាអត្ថបទសម្រាប់ផលិតសាច់រឿង  ប៉ុន្តែអ្វីៗដែលខ្ញុំបានចែករំលែកគឺជាការពិត។ ទម្រាំតែខ្ញុំដឹង វាពេលវេលាជាងពាក់កណ្តាលទសវត្សរ៍ទៅហើយ «អ្វីៗទាំងអស់គឺមានប្រភពតែមួយ» គឺជាបាវចនាដែលខ្ញុំប្រើប្រាស់សម្រាប់ការប្រមូលផ្តុំទិន្នន័យអំពីគ្រោះមហន្តរាយ។ វាជាគំនិតដែលពិបាកក្នុងការបកស្រាយ តែសម្រាប់ពេលនេះ ចូរយើងទាំងអស់គ្នាពិចារណាទៅលើភាពយន្តមួយដែលមានឈ្មោះថា The Matrix។ រឿងនេះបានបកស្រាយយ៉ាងក្បោះក្បាយ អំពីពិភពលោកនាពេលបច្ចុប្បន្ន។ អ្វីដែលខ្ញុំពិបាកបកស្រាយបំផុតគឺអ្វីដែលខ្ញុំនឹងរាយរាប់នាពេលបន្តិចទៀត ហើយខ្ញុំមិនបានបំផ្លើសវានោះឡើយ។  សម្រាប់ពេលនេះ ការបកស្រាយនៅក្នុងភាពយន្ត The Matrix គឺដូចជាអ្វីដែលខ្ញុំកំពុងព្យាយាមពន្យល់អ្នកទាំងអស់គ្នា។ \textbf{អ្វីៗគ្រប់យ៉ាងក្នុងជីវិតរបស់អ្នក រួមទាំងប្រវត្តិសាស្ត្រទាំងមូល ភាពទូទៅ វិទ្យាសាស្ត្រនិងការសិក្សាស្រាវជ្រាវសំខាន់ៗ  នយោបាយ និងសាសនា គឺទាក់ទងទៅនឹង "បំលែងនៃសំបកផែនដី ឬ លំអៀងនៃអ័ក្ស" ។} នាពេលបច្ចុប្បន្ននេះ យើងមិនទាន់ដឹងអំពីរឿងនេះនោះទេ វាប្រៀបធៀបបានដូចជាសុបិនអាក្រក់មួយដែលយើងមិនអាចគេចផុត។ យើងត្រូវការពេលវេលាបន្តែមដើម្បីយល់អំពីរឿងនេះអោយកាន់តែច្បាស់ ហើយខ្ញុំសន្យាថានឹងពន្យល់អ្វីៗគ្រប់យ៉ាងរហូតអ្នកទាំងអស់គ្នាបានយល់ ថាជីវិតរបស់ពួកយើងនាពេលបបច្ចប្បន្ននេះគឺមិនខុសពីរឿង The Matrix នោះទេ។ មានន័យថា ជីវិតរបស់ពួកយើងគឺស្ថិតនៅក្រោមការគ្រប់គ្រងដោយកំព្យូទ័រពេញមួយជីវិតមកហើយ"}\cite{33,34}។

\section{ការថ្លែងអំណរគុណ}

	សូមអរគុណជាអនេកទៅដល់បុគ្គលទាំងឡាយ​ដែលបានចែករំលែកចំណេះដឹងជាសាធារណៈ។ បើគ្មានអ្នកទាំងអស់គ្នានោះទេ ការស្រាវជ្រាវមួយនេះមិនអាចកើតឡើងបានឡើយ ហើយមនុស្សទូទាំងពីភពលោកនិងស្ថិតនៅក្នុងភាពងងឹតដដែល។ ជម្រើសនូវការចែករំលែករបស់អ្នក នឹងជួយជ្រុំជ្រែងអោយមនុស្សធម្មតាកាន់តែយល់អំពីអត្ថបទដែលបានបកស្រាយកន្លងមក។ សូមអរគុណ។ 
\clearpage
\twocolumn

{\small
\renewcommand{\refname}{ឯកសារយោង}
\bibliographystyle{ieee}
\bibliography{egbib}
}
\end{document}
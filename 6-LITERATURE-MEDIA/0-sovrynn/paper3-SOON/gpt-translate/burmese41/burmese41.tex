\documentclass[10pt,twocolumn,letterpaper]{article}

\usepackage{booktabs}
% \usepackage{caption}
% \captionsetup[table]{skip=8pt}   % ဇယားများအတွက်သာ သက်ရောက်သည်
\usepackage{stfloats}  % ဤကို နိဒါန်းအပိုင်းတွင် ထည့်ပါ
\usepackage{float}

\usepackage{cvpr}
\usepackage{times}
\usepackage{epsfig}
\usepackage{graphicx}
\usepackage{amsmath}
\usepackage{amssymb}

\usepackage[breaklinks=true,bookmarks=false]{hyperref}

\cvprfinalcopy % *** ဤကြောင်းကို နောက်ဆုံးတင်သွင်းမှုအတွက် သတ်မှတ်ချက်ဖြင့်ဖျော်ဖြေပါ

\def\cvprPaperID{****} % *** CVPR စာရွက်အမှတ်စဉ်ကို ဤနေရာတွင် ထည့်သွင်းပါ
\def\httilde{\mbox{\tt\raisebox{-.5ex}{\symbol{126}}}}

% စာမျက်နှာများကို တင်သွင်းမှုအဆင့်တွင် နံပါတ်ထည့်ထားပြီး၊ final version တွင် နံပါတ်မထည့်ပါ။
%\ifcvprfinal\pagestyle{empty}\fi
\setcounter{page}{1}
\begin{document}

\title{ECDO စာတမ်း ၃: ယနေ့ခေတ်တွင် အနောက်တိုင်း အာဏာလှုပ်ရှားသူများက မကြာမီ ဖြစ်ပေါ်လာနိုင်သော ဘူမိဖေါ်ပြမှုကြီးအတွက် ပြင်ဆင်ဆောင်ရွက်နေသည့် သက်သေ}

\author{ဂျွန်ဟို\\

Published June 2025\\
ဝေမျှချိန် ဇွန် ၂၀၂၅\\
Website (Download papers here): \href{https://sovrynn.github.io}{sovrynn.github.io}\\
ဝက်ဘ်ဆိုက် (စာတမ်းများကို ဒေါင်းလုပ်လုပ်ပါ): \href{https://sovrynn.github.io}{sovrynn.github.io}\\
ECDO Research Repo: \href{https://github.com/sovrynn/ecdo}{github.com/sovrynn/ecdo}\\
ECDO သုတေသန သိမ်းဆည်းရာ: \href{https://github.com/sovrynn/ecdo}{github.com/sovrynn/ecdo}\\
{\tt\small junhobtc@proton.me}
}

\maketitle
%\thispagestyle{empty}

\begin{abstract}

In May 2024, a pseudonymous online author known as “The Ethical Skeptic” \cite{0} shared a groundbreaking theory called the Exothermic Core-Mantle Decoupling Dzhanibekov Oscillation (ECDO) \cite{1}. This theory suggests that Earth has previously experienced sudden, catastrophic shifts in its rotational axis, triggering massive worldwide floods as the oceans spilled over the continents due to rotational inertia. Additionally, it presents an explanatory geophysical process and data indicating that another such flip may be imminent. While such cataclysmic flood and doomsday predictions are not new, the ECDO theory is uniquely compelling due to its scientific, modern, multidisciplinary, and data-based approach.

May 2024 တွင်၊ “The Ethical Skeptic” ဟုအမည်မသိ အွန်လိုင်းစာရေးသူတစ်ဦးက Exothermic Core-Mantle Decoupling Dzhanibekov Oscillation (ECDO) \cite{1} ဟုခေါ်သော မျိုးတုပျသော သီအိုရီတစ်ခုကို မျှဝေခဲ့သည် \cite{0}။ ဤသီအိုရီအရ ကမ္ဘာ၌ အတိတ်ကာလများတွင် လှည့်ပတ်ဝင်ရိုးအတွင်း ထပ်မံ ပြင်းထန်လှသော ပျက်စီးမှုများ ဖြစ်ပေါ်ခဲ့ပြီး၊ ပင်လယ်တို့သည် ပျံ့နှံ့ကျူးလွန်ကာ ကမ္ဘာ့မကြီးအနှံ့အပြား ရေပျောက်ဝင်ချိုးဖျက်မှုကြီးများ ဖြစ်ပေါ်စေခဲ့ကြောင်း ဖော်ပြသည်။ ထို့အပြင် သက်ဆိုင်ရာ ဘူမိရောဂါလုပ်ငန်းစဉ်နှင့် အချက်အလက်များမှ နောက်တစ်ကြိမ် ထပ်မံဖြစ်ပွားနိုင်ခြေရှိကြောင်း မျှဝေနိုင်သည်။ ယင်းကဲ့သို့သော ဖျက်ဆီးမှုကြီးများနှင့် နောက်ဆုံးရက်ခန့်မှန်းမှုများသည်အသစ်မဟုတ်သော်လည်း၊ ECDO သီအိုရီသည် သိပ္ပံဆိုင်ရာ၊ ခေတ်မီမှုအရ၊ အတတ်ပညာစုံလင်မှုနှင့် ဒေတာအခြေပြုမှုများကြောင့် ထူးခြားစွာ စိတ်ဝင်စားဖွယ်ရှိသည်။

This paper is my third work \cite{2,3} on this topic, and focuses on present-day political aspects of this theory:
\begin{flushleft}
\begin{enumerate}
    \item ထင်ရှားစွာအသိပေးသူများ၏ သက်သေခံချက်များအရ အနောက်နိုင်ငံများက ဘူမိရောဂါ ဖျက်ဆီးမှုကြီး အလျင်အမြန် လာမည်ဟု ယုံကြည်ကာ ၎င်းဖြစ်ရပ်အပေါ် နိုင်ငံရေးနှင့် စစ်ရေး အားသာမှုရအောင် ကြိုတင်အစီအစဉ်တွေ ချထားသည်။
    \item ဤဖြစ်ရပ်အတွက်ပြင်ဆင်ရန် အနောက်နိုင်ငံများ အောက်မြေပြင်နှင့် သမုဒ္ဒရာအောက်ခံ စခန်းတွေ တော်တော်များများတည်ဆောက်ထားကြောင်း သက်သေရှိသည်။
    \item ဤစခန်းများဖွဲ့စည်းရန်အတွက်  အနောက်ပိုင်းငွေကြေးဖွဲ့စည်းပုံများမှ ကြီးမားသော ငွေကြေးများ ယိုထွင်းယူသည့် သက်သေရှိသည်။
\end{enumerate}
\end{flushleft}
This paper documents the extensive preparations Western ruling powers are carrying out to prepare for a geophysical cataclysm they believe to be imminent.
ဤစာတမ်းသည် အနောက်တိုင်းအုပ်ချုပ်ရေးအာဏာရှိသူများက မကြာမီဖြစ်ပေါ်မည်ဟု ယုံကြည်နေသည့် ဂေယာဖျုခိုက်အခက်အခဲတစ်ရပ်အတွက် ပြင်ဆင်မှုများပြုလုပ်နေမှုကို အကျယ်တဝင့် မှတ်တမ်းတင်ထားပါသည်။
\end{abstract}

\section{Freemasonry and the "Anglo-Saxon Mission"}
\section{Freemasonry နှင့် "Anglo-Saxon Mission"}

In January 2010, Project Camelot, an alternative media and journalism organization which compiles whistleblower testimonies, interviewed \cite{4,6} an insider who was physically present at a meeting of Senior Masons in the City of London in June 2005. The topics discussed at the meeting were military and political plans centered around a backdrop of a coming \textbf{"geophysical event"}, i.e., a global natural disaster.
၂ဝ၁ဝ ခုနှစ်၊ ဇန်နဝါရီလတွင်၊ Project Camelot ဟုခေါ်သော သတင်းစာပေါ်လွင်မီဒီယာနှင့် သတင်းစာ သုတေသနအဖွဲ့သည် သိသိမြင်မြင်သက်သေခံချက်များစုစည်းသောအဖွဲ့ဖြစ်ပြီး၊ ၂၀၀၅ ခုနှစ်၊ ဇွန်လ၌ လန်ဒန်မြို့တွင် Senior Masons များစုဝေးပွဲတွင် တက်ရောက်ခဲ့သူအတွင်းသားတစ်ဦးကို \cite{4,6} မေးမြန်းခဲ့သည်။ ယင်းပွဲတွင် ဆွေးနွေးခဲ့သောအကြောင်းအရာများမှာ လာမည့် \textbf{"ဂေယာဖျုခိုက်ဖြစ်ရပ်"} ကိုအခြေခံထားသည့် စစ်ရေးနှင့် နိုင်ငံရေးအစီအစဉ်များ ဖြစ်သည်။ ယင်းသည် ကမ္ဘာအနှံ့သဘာဝဘေးအန္တရာယ်တစ်ခုဖြစ်သည်။

\begin{figure}[b]
\begin{center}
\includegraphics[width=1\linewidth]{freemason.jpg}
\end{center}
   \caption{အင်္ဂလန် Freemasons များသည် သူတို့၏ သဘာဝအခြေအနေတွင် တိတ်တိတ်လေး နျဉ်းမြှုပ်ဗုံးချပြီး ကမ္ဘာ့အုပ်ချုပ်ရေးအာဏာယူဖို့ အချိန်ပြုလုပ်နေကြသည် - လန်ဒန်မြို့၊ Earls Court တွင်၊ ၁၉၉၂ ခုနှစ် \cite{5}။}
\label{fig:1}
\label{fig:onecol}
\end{figure}

\begin{figure*}[t]
\begin{center}
\includegraphics[width=1\textwidth]{british.jpg}
\end{center}
   \caption{၁၉၃၇ ခုနှစ်အတွင်း ဗြိတိသျှအင်ပါယာ၊ အင်္ဂလို-ဆက်စွန်တို့၏ အင်အားကြီးမှုကို ထင်ဟပ်ပြသသည် \cite{14}။}
   \label{fig:2}
\end{figure*}

ဒီအတွင်းရေးသူ ထံအဆိုအရ၊ အစည်းအဝေးတွင် တက်တုန်းသူ ၂၅-၃၀ ဦးသည် \textit{"...အားလုံးဗြိတိသျှလူမျိုးများဖြစ်ပြီး၊ သူတို့အနက်တချို့မှာ ပြည်ထောင်စုဗြိတိန်နိုင်ငံအတွက် ချက်ချင်းသိရှိရနိုင်တဲ့ နာမည်ကြီးသူတွေပါ။ အနည်းငယ်အထက်တန်းမိသားစုလူမျိုးတွေရှိပြီး၊ တချို့က အထက်တန်းမိသားစုမှ ဆင်းသက်လာသူတွေပါ။ အစည်းအဝေးမှာ တွေ့ခဲ့သူတစ်ဦးက ကြီးကြပ်ရာနိုင်ငံရေးသမားတစ်ဦးပါ။ အခြားနှစ်ဦးက ရဲဘက်အရပ်မှ လုပ်သက်ကြီးသူများဖြစ်ပြီး၊ တစ်ဦးမှာ စစ်အရာရှိတစ်ဦးပါ။ နှစ်ဦးစလုံးက နိုင်ငံတစ်ဝှမ်း အသိအမှတ်ပြုထားသူများဖြစ်ပြီး လက်ရှိအစိုးရကို အကြံပေးမှုတွင် အရေးပါနေသူများဖြစ်ကြသည် — လက်ရှိအချိန်မှာပါ"} \cite{4}။ အတွင်းရေးသူပြောသည်မှာ၊ သူသည် အစည်းအဝေးသို့ တက်ရောက်ခဲ့သည်မှာ\ \textit{"အမှတ်မထင်နဲ့ပါပဲ! သုံးလတစ်ကြိမ်ပြုလုပ်တဲ့ ပုံမှန်အစည်းအဝေးထင်ပါတယ်... ငါ့အနေနဲ့ တက်ရောက်တဲ့အခါ ပုံမှန်ထင်တာမျိုးမဟုတ်ဘူး။ ငါ့ကို ဖိတ်ကြားခဲ့တယ်လို့ ယူဆပါတယ်... ဘာလို့ဆိုတော့ ငါမှာ ကိုယ်တိုင်ပိုင်ရာရာထူးတစ်ခုရှိခဲ့တာနဲ့ သူတို့လိုပဲ သူတို့ထဲက တစ်ယောက်ပါလို့ ယုံကြည်ခဲ့လို့ပဲ"} \cite{4}။

၂၀၀၅ ခုနှစ်တွင် အစည်းအဝေးတွင် ဆွေးနွေးခဲ့သော အခြေခံဖြစ်ရပ်များ၏ အချိန်ဇယားမှာ အောက်ပါအတိုင်းဖြစ်သည် -

\begin{flushleft}
\begin{enumerate}
    \item ဂအားန်ကို သို့မဟုတ် တရုတ်ကို မော်တော်နူးကလီးယားလက်နက်တစ်လက်သုံးဖို့ လှုံ့ဆော်ခြင်း၊ ပြီးနောက် နူးကလီးယားတိုက်ပွဲနယ်သတ်ကို ဖြစ်ပေါ်စေပြီး ထို့နောက် ပဋိပက္ခရပ်စဲမှု သတ်မှတ်ရန်။
    \item တရုတ်နိုင်ငံပေါ် ယေဘုယျကျူးလွန်ရန် ဗြိုလ်ဇီဝလက်နက်များ ထုတ်လွှင့်ခြင်း၊ ဤသို့လုပ်သောအဓိကပစ်မှတ်မှာ မိယ်လူတန်းကနေ “၁၉၇၀ ခုနှစ်များကတည်းက” တရုတ်များ ဖြစ်သည်ဟု သတင်းရသည်။
    \item ဖြစ်ပေါ်လာမည့် ထိတ်လန့်မှုနှင့် မတည်ငြိမ်မှုအပေါ် အခြေခံ၍ စစ်မတော်တစ်စုက အစိုးရအုပ်ချုပ်မှုများကို ခေါင်းထိန်းချပို်။

\end{enumerate}
\end{flushleft}

ဒါပေမယ့် အရေးကြီးဆုံးကတော့ ဤဖြစ်ရပ်များပြီးနောက် ရှုမေတ္တာထားရမည့်အရာဖြစ်သည်- \textit{"ဒါဖြင့် ကျနော်တို့ စစ်ပွဲထဲ ဝင်သွားရမယ်၊ ပြီးရင်... မြေကြီးပေါ်မှာ ဘူမိရုပ်ပိုင်းဆိုင်ရာ ဖြစ်စဉ်တစ်ခု ဆက်ပြီး ဖြစ်လာမယ်၊ အဲ့ဒါကလည်း လူအားလုံးကို သက်ရောက်သွားမယ်"} \cite{4}။ အတွင်းလူက ယင်းဘူမိရုပ်ပိုင်းဆိုင်ရာ ဖြစ်စဉ်အတွင်း \textit{\textbf{"မြေကြီးအလွှာက စုပေါင်းလှည့်ပတ်သွားမယ် ၃၀ ဒီဂရီလောက်၊ တောင်ဘက်ကို ၁၇၀၀ ကနေ ၂၀၀၀ မိုင်ခန့်ရွှေ့သွားမယ်၊ အဲ့ဒီကနေ အထွင်းအထိပ်ကြီးမားသည့် ပြောင်းလဲမှုဖြစ်မယ်၊ ထိုသက်ရောက်မှုက အချိန်အတော်ကြာတာတောင် တည်နေနိုင်မယ်"}} \cite{4}။

ဤလျှိုဝှက်ကြံဆချက်အားလုံး၏အကြောင်းအရင်းကတော့ လွှမ်းမိုးမှုစွမ်းအားပဲ ဖြစ်သည်။ အတွင်းလူက ရှင်းပြသည်- \textit{"အဲဒီအချိန်ဆိုရင် ကျနော်တို့အားလုံး နူးကလီးယားနဲ့ ဗြိုလ်ဇီဝစစ်ပွဲတွေကို တက်ပြီးသား ဖြစ်နေမယ်။ မြေယာအပေါ် လူဦးရေက အကယ်လို့ ဒီလိုဖြစ်လျှင် တော်တော်ကို လျော့နည်းသွားမယ်။ ဘူမိရုပ်ပိုင်းဆိုင်ရာ ဖြစ်စဉ်ဖြစ်တဲ့အချိန်မှာလည်း ကျန်ရစ်တဲ့လူတွေက တစ်ဝက်ထပ်ပြီး လျော့နိုင်ပါတယ်။ ဒါကြောင့် ကြီးကျယ်ထွားသလောက် နောက်ပိုင်းက ဘယ်သူအသက်ရှင်ကျန်နေမလဲ ဆိုတာက နောက်ဦးနဲ့ ကမ္ဘာအသစ်အတွက် လူမျိုးအသစ် ၊ ကျန်ရစ်တဲ့လူဦးရေကို ဦးတည်ညွှန်ကြားသွားမယ့်သူဖြစ်လာမှာပဲ။ ဒါကြောင့် ကျနော်တို့ ပြောပြနေတဲ့အပိုင်းက သဘာဝဘေးအန္တရာယ်ကြီးကြပ်သည့်ပြီးနောက်ကာလ အကြောင်းပါပဲ။ ဘယ်သူရိုးဘယ်သူထိန်းသွားမလဲ? ဘယ်သူအုပ်ချုပ်သွားမှာလဲ? အဲဒါကိုပဲ သက်တမ်းတစ်ခုတည်းသာနောက်ကျွံလိုက်တာ။ ဒါကြောင့်ပဲ သူတို့က ဒီအရာတွေကို သတ်မှတ်ထားတာထဲမှာဖြစ်အောင် အရမ်းမြန်မြန်လုပ်သွားချင်ကြတာ။ ... သေချာတဲ့ ခန့်မှန်းချက်တစ်ခုနဲ့ [အပျက်အစီး] ကို မတိုင်ခင်မှာ ပုံစံတစ်ခုမျိုး တည်ထားနေရမှာပါ... မနက်ဖြန်မှာ ရပ်တန့်နိုင်ဖို့၊ အဲ့ဒီအခါမှာနောက်ထပ် အာဏာရှိထားနိုင်ဖို့၊ ယခင်ကလည်းခံစားလာခဲရတဲ့ အာဏာအတူတူပိုင်နိုင်ဖို့ပါ"} \cite{4}။ မေးမြန်းနေစဉ်မှာ ဒီအစီအစဉ်၏ အမည်ဖြစ်တဲ့ "Anglo-Saxon Mission" ကိုလည်း ဆွေးနွေးထားသည်- \textit{[မေးမြန်းသူ]: "...ဒါကို The Anglo-Saxon Mission လို့ခေါ်တာက အကြောင်းရင်းက တကယ်တော့ တရုတ်လူမျိုးတွေကို ဖျက်စီးဖို့စီစဉ်ထားလို့ပါပဲ၊ ပြီးတော့ သဘာဝဘေးကြီးပြီး သစ်တည်ဆောက်စဉ်မှာ အနောက်တိုင်းလူမျိုးတွေကပဲ အသစ်ကမ္ဘာကို ဆောက်လုပ်နိုင်၊ မြေမြှုပ်နိုင်မှာသာ ဖြစ်သွားမှာပါ၊ တခြားလူမျိုးအရာမရှိတော့ဘူး။ ဒါကို မှန်တယ်လို့ တွေးရင်ရော?" [အတွင်းလူ]: "ဒါမှန်မမှန် ကျနော်မသိဘူး၊ ဒါပေမယ့် ပြောတာကို သဘောတူပါတယ်။ အနည်းဆုံး တိုင်းထွာဝှုတန်းစုနှစ်အတွင်း၊ ဦးစားပေးတိုင်းဒါတောင် ခေတ်ဦးများထဲမှာဖြစ်လေ့ရှိခဲ့သည်။ ဤကမ္ဘာရဲ့ သမိုင်းက အဦးအဦး အနောက်ဘက်၊ မြောက်ဘက်တောင်ဒေသကမှ အဓိကအုပ်ချုပ်မှု ခိုင်မြဲခဲ့တာပါ"} \cite{4}။
Regarding the exact timeframe of the expected geophysical event, the insider offers his best guess: \textit{"...ခံစားမှုတစ်ခုဖြစ်ပြီး၊ ဒါဟာအတော်ပဲအတွေးအမြင်အခြေခံတဲ့ခံစားချက်တစ်ခုပါ၊ သူတို့ဟာအခုအချိန်မှာပဲ ကိုယ့်အလုပ်ကိုလုပ်သင့်ပြီလို့ ထင်ပါတယ်... ဒါတွေဖြစ်မယ့်အချိန်ကို သူတို့ကအတော်လေးသိထားကြောင်း သေချာပါတယ်... \textbf{ကျွန်တော်ကတော့် အရမ်းပြင်းထန်တဲ့ခံစားချက်ရှိပါတယ်။ ဒါဟာကျွန်တော့်အသက်ရှင်တည်နေချိန်အတွင်း ဖြစ်လာမှာပါ၊ ဆိုရင် ၂၀ နှစ်အတွင်း}... ဒီအချိန်မှာတော့ ဒီဘာသာရပ်ပေါ်မှာ ဖြစ်တော့မယ့် အချိန်အပိုင်းအခြားထဲကို ဝင်သွားပြီးပါပြီ၊ վերջինတစ်ကြိမ်က နှစ် ၁၁,၅၀၀ လောက်ကြာခဲ့တဲ့အချိန်နဲ့ နှိုင်းလိုက်ရင်ပါ၊ ဒါဟာနှစ် ၁၁,၅၀၀ လောက်တစ်ကြိမ်ကြတိုင်းဖြစ်ပါတယ်။ အခုတော့ ဆိုတာတကယ် ဖြစ်တော့မယ့်အချိန် ပြန်လာပါပြီ... သူတို့က ဖြစ်တော့မယ်ဆိုတာ သဘောပေါက်နေကြပါတယ်။ သူတို့မှာ ဒါဖြစ်တော့မယ်ဆိုတာအကြောင်းအရာသေချာပါတယ်... တစ်ခါတစ်လေမှာတော့ ငါတို့လည်း တကယ်ပဲ မသိဘူးဆိုတာ မယုံနိုင်တာမျိုးတွေရှိတယ်။ ငါပြောချင်တာကတော့—ကမ္ဘာ့အလွန်တော်သူတွေက သူတို့အတွက် ဒီအကြောင်းအရာမှာ အလုပ်လုပ်နေကြလိမ့်မယ်"} \cite{4}.

This is a powerful testimony for which we should be very grateful. In the interview, the author also discusses his belief that WWI and WWII were manufactured wars, and that the Anglo-Saxon Mission almost certainly dates back many, many generations. It has now been 15 years since the interview, which occurred in 2010. There are five years remaining until the insider's stated 20-year timeframe prediction for the geophysical event reaches its end.

\subsection{Druidic Esoteric Western Knowledge of Cataclysms}

Western knowledge of recurring cataclysms is well kept, and not just by Freemasons. The Druids, a well-documented ancient Celtic culture dating back at least 2400 years \cite{7}, passed on knowledge about the Earth's recurring cataclysms. The last known Druid is believed to be Ben McBrady. In "The Last Druid", a 1992 documentary, he shared information about the knowledge of the Druids: \textit{"ကျွန်တော့်ရဲ့ သာမန်အရ ထင်နိုင်တဲ့အတိုင်းနောက်ဆုံးအဖွဲ့ဝင်ဖြစ်နိုင်တဲ့ အဖွဲ့ဟာ နောက်ဆုံးဘေးအန္တရာယ်ကြီးပြီးနောက်မှာတည်ထောင်ခဲ့တာ ဖြစ်ပါတယ်၊ ဒါပါပဲကမ္ဘာကြီးကို ထိခိုက်စေတဲ့ ဖြစ်စဉ်ကြီးပါ။ အခု ယင်းထဲမှာကြီးမားတဲ့လျှပ်စစ်မုန်တိုင်းတွေဖြစ်ပြီး၊ နဂိုမီးစက်ကြီးတွေ့ တိုက်ကြ့ချိန်တွေနဲ့ သံမဏိမိုးအုပ်ကြီးတွေတက်လာတဲ့အခါမှာ လူ့ယဉ်ကျေးမှုဟာ အပြီးပျက်စီးသွားပါတယ်... အသိပညာအားလုံးဟာ အဖွဲ့ရဲ့အုပ်ချုပ်မှုအောက်မှာပါပါတယ်၊ ဒါပေမယ့် သူတို့ဟာ အထူးသဖြင့် နက္ခတ္တဗေဒကိုစိတ်ပူပါတယ်၊ ဘာလဲဆိုတော့ သူတို့မှာ အရေးကြီးတဲ့ဘေးအန္တရာယ်တွေ အများကြီးကိုတွေ့ကြုံဖူးလို့။ နက္ခတ္တဗေဒအကြောင်းကိုအထူးသိရှိခြင်းက သူတို့အနေနဲ့ ဒီဘေးအန္တရာယ်တွေ ဖြစ်လာနိုင်စရာကာလတွေကို ခန့်မှန်းနိုင်ဖို့နဲ့ ကိုယ့်ကိုယ်ကိုကာကွယ်နိုင်ဖို့ အရေးကြီးတယ်လို့ထင်ကြတယ်။ အိုင်ယာလန်မှာရှိတဲ့ ငယ်ရွယ်ဆုံးခေတ် ကြီးမားတဲ့ကျောက်တုံးအဆောက်အအုံတွေကိုကြည့်မယ်ဆိုရင် လူတွေက အပေါက်လမ်းသုသာန်လို့ ခေါ်ကြပေမယ့် တကယ်တော့ အရမ်းရှေးဟောင်းတဲ့ ဗုံးတိုက်ခိုက်မှုလုံခြုံရေးခန်းေတြပါ။ ဒီအဆောက်အအုံတွေက တိတိကျကျ ရေလှိုင်းအသင့်အတင့် အမြင့်မှာရှိပြီး၊ မိုးရေသေတ္တာကြီးတွေကနေ မွေးကွယ်မှုအကာအကွယ်ရပါတယ်"} \cite{8,9}.

% It is also believed that Freemasonry itself actually originates from the Druids \cite{10}.
\section{လက်ရှိအနောက်တိုင်းကမ္ဘာ မတည့်စီမံကိန်းများအတွက် သက်သေများ}

အုပ်ချုပ်သောအနောက်တိုင်းအာဏာကြီးများက ကမ္ဘာမကျေရာဗေဒဘေးအန္တရာယ်တစ်ခု အသီးသီး နီးစပ်လာသည်ဟု ယုံကြည်နေသလို ထင်မြင်ရပါသည်။ ထိုအခြေအနေတွင် သူတို့ကိုယ်သူတို့ကို ထိန်းသိမ်းကာကွယ်နိုင်ရန် အရေးကြီးသော ပြင်ဆင်မှုများ ပြုလုပ်နေသည်ဟု မျှော်လင့်နိုင်သည်။ ဆိုလိုသည်မှာ ယနေ့ လူထုစိတ်ဝင်စားရာတွင် အနောက်တိုင်းနိုင်ငံအတော်များစွာတွင် မြေအောက်အလွန်နက်သော ကျယ်ပြန့်သည့် စခန်းများ ရှိနေကြောင်း သက်သေအနာဂတ်များ ရှိနေသည်။ ထိုလူနေခြင်းအဆောက်အအုံများသည် nucleaer စစ်ပွဲတစ်ခုတွင် လူနေထိုင်သူများကို ကာကွယ်ပေးနိုင်သော်လည်း၊ သဘာဝဘေးအန္တရာယ် အမျိုးမျိုးမှလည်း ကာကွယ်နိုင်သည်။ Project Camelot မှ ဗြိတိသျှ အဆင့်မြင့် Freemason ၏ မျက်မြင်သက်သေအရ \cite{4,6}၊ ဤဘေးဒဏ်အခြေအနေများသည် ဖြစ်နိုင်သည့်ဖြစ်စဥ်များသာ မဟုတ်ဘဲ၊ ကြိုတင်စီစဉ်ထားသည့်အစီအစဉ်များ ဖြစ်နေသည်ဟု ထင်ရှားသည်။ ထို့အပြင်၊ ထိုစခန်းများကို တည်ဆောက်ရန်၊ ဝန်ထမ်းထားရန် မြှင့်တင်ပြုပြင်စောင့်ရှောက်ရန် များစွာသော သန်းခေါင်ရာချီသောဒေါ်လာများ လိုအပ်မည်ဖြစ်ပြီး၊ ၁၈ နှစ်တိုင်တိုင် အမေရိကန်အစိုးရမှပျောက်ဆုံးနေသော ဒေါ်လာဒ십ချီ ထောင်နှစ်ပေါင်းများစွာပမာဏနှင့် ညီနေသည် (နောက်ပါအခန်းတွင် ဖော်ပြထားသည်) \cite{11,12,13}။ လူမီနတ်ဖြစ်ပျက်မှုအတွက် ပြင်ဆင်မှု၏အခြား ဥပမာများမှာ မျိုးစေ့များနှင့် အသိပညာ အားသိမ်းဆည်းသည့် မတည့်စီမံကိန်းများပါဝင်သည်။

\subsection{အမေရိကန်မြေအောက်နှင့် ပင်လယ်အောက်စခန်းများ}

မြေအောက်စခန်းများနှင့် ပတ်သက်သော ရပ်ရွာလူထု၏ ဆန်းစစ်မှုအကြီးမားဆုံးကို ငါ့မြင် တွေ့မြင်ရသည်မှာ လွတ်လပ်သောအမေရိကန်သုတေသနပညာရှင် Richard Sauder ဖြစ်သည်။ သူသည် မြေအောက်အလွန်နက်သောစခန်းများကို သုံးသပ်ရေးသားထားသည့် စာအုပ်အတော်များစွာ ထုတ်ဝေထားသည် \cite{22}။ Sauder ၏ လုပ်သက်သည် အစိုးရစာရွက်စာတမ်းများနှင့် အစီအစဉ်များကို သိမ်းဆည်းခြင်း၊ သမိုင်းဝင် နှင့် လက်ရှိ သတင်းအကြောင်းအရာများ၊ နည်းပညာများကို တွေ့မြင်သုံးသပ်ခြင်း၊ အရင်းအမြစ်များ တိုးတက်စေခြင်းနှင့် တကယ်ကူးပြောင်းသည့်အကြောင်းအရာများကို စုစည်းသည့်အလုပ်ဖြစ်သည်။ Sauder ၏ သုတေသနသည် အမေရိကန်နှင့် ၎င်း၏နယ်မြေများ အနီးအနားရှိ မြေအောက်အလွန်နက်သည့်စခန်းများနှင့် ပင်လယ်အောက်စခန်းများ၊ (ရုပ်ပုံ \ref{fig:4}) တွင် စုစုပေါင်း အနည်းဆုံး မိုင်သုံးမိုင်အနက်ထိ ရောက်ရှိနိုင်ပြီး မြေအောက်ဗက်ဂျက်ကျော်မြန်နှုန်းမြင့်သံလိုက်ရထားများဖြင့်တွဲဖက်ထားနိုင်မှုအပါအဝင် ဖြစ်နိုင်ကြောင်းဖော်ပြသည်။ ထိုစခန်းများအတွက် ငွေကြေးသည် \textit{"မြင့်မားသောဘဏ္ဍာရေး၊ နိုင်ငံတကာ၊ အစိုးရတပ်ဖွဲ့များ၊ ငွေဖြတ်ထုတ်လှုပ်ရှားမှု shell game"} ဖြင့် လှျားအားဖြင့် မြောက်မြားစွာစီမံခန့်ခွဲထားသည်။ ၎င်းတို့အား အမေရိကန်စီးပွားရေးကုမ္ပဏီပိုင်ဆိုင်သူများက လည်ပတ်ကြသည် \cite{22}။ ထို့နောက် Catherine Austin Fitts (မင်းအနောက်အပေါ်တွင် ဖော်ပြမည့်သူဖြစ်သည်) နှင့် သူ့အဖော်လက်တွဲသူတစ်ဦးက မြေအောက်နှင့် ပင်လယ်အောက်အမေရိကန်စခန်း ၁၇၀ခန့်ရှိနိုင်ကြောင်း ခန့်မှန်းချက်တစ်ခု ထပ်မံပေးခဲ့သည် \cite{16,20}။

\begin{figure}[b]

\begin{center}
% \fbox{\rule{0pt}{2in} \rule{0.9\linewidth}{0pt}}
   \includegraphics[width=1\linewidth]{penta.jpg}
\end{center}
   \caption{အဖြစ်တကယ် White House နဲ့ Pentagon အောက်မှာ ဘာတွေရှိလဲ? အထင်ကြီးစွာနဲ့, မြေညီအောက်မှာ တူးဖော်ထားတဲ့ တာနယ်တွေ ကွန်ယက်အနက်ကြီးရှိတယ် (ပုံ: \cite{31})။}
\label{fig:3}
\label{fig:onecol}
\end{figure}
\begin{figure*}[t]
\begin{center}
% \fbox{\rule{0pt}{2in} \rule{.9\linewidth}{0pt}}
\includegraphics[width=0.9\textwidth]{basescrop.png}
\end{center}
   \caption{Sauder ၏ သုတေသနအရ မြေအောက်၊ သမုဒ္ဒရာအောက်ရှိ စခန်းများ၊ ထို့အပြင် သမုဒ္ဒရာအောက်လမ်းကြောင်းများမှ အပြည့်အစုံတည်နေရာများကိုပြသသောမြေပုံ။ Sauder သည် \textit{"ဤစာရင်းထက် \textbf{အနည်းဆုံး ထပ်မံ၍} ဆိုဒ်တွေရှိနေသည်ဟု သေချာသည်"} \cite{22}.}
   \label{fig:4}
\end{figure*}

ဤစခန်းအချို့၏ အတိုင်းအတာတစ်ချို့ကို အသေးစိတ်ဖော်ပြသည့် Sauder ၏အရင်းအမြစ်များမှ သက်သေခံချက်အချို့မှာ\ :
\begin{flushleft}
\begin{enumerate}
    \item Camp David, Maryland: \textit{"ကျွန်ုပ်၏အရင်းအမြစ်က Camp David ၏ မြေအောက်ပိုင်းများသည် အလွန်ကျယ်ပြန့်ပြီး အံ့ဩစရာကြီးများသောအတွက်၊ လျှို့ဝှက်လမ်းကြောင်း မိုင်များစွာရှိနေပြီး၊ တစ်စုံတစ်ယောက်တင် အဲဒီဧရိယာကို အပြည့်အစုံ စိတ်ထဲမှာ ပြုလုပ်နိုင်မယ်ဆိုတာ မလွယ်နိုင်ဘူးလို့ ပြောပါတယ်"} \cite{22}။
    \item The White House, Washington DC: \textit{"ကျွန်ုပ်ဝင်ရောက်နီးစပ်သူ တစ်ဦးကို ၁၉၆၀ မှာ Lyndon B. Johnson အစိုးရသက်တမ်းအတွင်း ဒီဧရိယာထဲကို ခေါ်သွားခဲ့ကြတယ်။ သူမက White House ထဲက ဓာတ်လှေကားတစ်လုံးထဲကို ထည့်ပေးပြီး တိုက်ရိုက်အောက်သို့ ဆင်းအောင် လိုက်ကြားခဲ့တယ်။ သူမထင်တာက ဓာတ်လှေကားက အောက်အထပ် ၁၇ ထပ်လောက် ဆင်းသွားသလို ထင်တယ်။ မြေအောက်မှာ တံခါးဖွင့်လိုက်တော့ အရမ်းဝေးပျောက်သွားမယ့် ဆယ်လေးသန်းတစ်ခုကနေ လမ်းလျောက်ချဉ်းကပ်ခေါ်သွားမိတယ်။ အခြားတံခါးတွေနဲ့ လမ်းကြောင်းအသစ်တွေလည်း အဲဒီထဲမှာပါ ရှိနေခဲ့တယ်"} \cite{22}။ ပုံတွင် ပြထားသည် Figure \ref{fig:3}။
    \item Fort Meade, Maryland - ၁၉၇၀ ခုနှစ်များတွင် တစ်စွဲအမျှ ဝင်ရောက်သွားသူတစ်ဦး၏ အရင်းအမြစ်မှ: \textit{"ကျွန်ုပ်တံခါးဖွင့်လိုက်ရာမှာ လှေကားတစ်လုံးအောက်ကို ဆီခေါ်သွားတဲ့နေရာပါ။ ကျွန်ုပ်ကခြံစည်းပတ်ပတ်မှာ မျက်စိချိန်ကြည့်လိုက်ပါတယ်။ ဘယ်လောက်အလွှာ ဆင်းရမလဲ မတွက်နိုင်ဘူး၊ ဒါပေမဲ့ ၁၅-၂၀ ထပ်လောက်ရောက်မှာပဲဆိုတာ ထင်လာတယ်... လှေကားတစ်လှမ်း ဆင်းချိန်တုန်း တံခါးတစ်လုံးပေါ်လာတယ်... တံခါးဖွင့်ပြီး ဦးခေါင်းထိုးထုတ် လက်ယာ လက်ဝဲကြည့်ကြည့်တော့ နှစ်ဖက်စာလုံးယှဉ်နားနား ကြည့်ရတာ လမ်းကြောင်းတစ်ကြောင်းတွေ့တယ်၊ နှစ်ဖက်စလုံး မမြင်နိုင်အောင် တိုက်တိုက်ထွက်နေတယ်။ အဲဒီရာအနားမှာ တံခါးတွေကို အလျားကို ၃၀-၄၀ ပေခန့်ခန့် နေရအောင်ထားတယ်... ထပ်သွားလေ့လာချင်တာနဲ့ နောက်ထပ်အလွှာတစ်လွှာ ဆင်းတယ်... အဲဒီနေရာကလည်း တူညီတယ်... နောက်ထပ်အလွှာတစ်လွှာ ဆင်းပြီး အဖွင့်ကြည့်တော့ ပထမနှစ်ထပ်နဲ့ တူညီတယ်"} \cite{22}။
\end{enumerate}
\end{flushleft}

\begin{figure}[t]
\begin{center}
% \fbox{\rule{0pt}{2in} \rule{0.9\linewidth}{0pt}}
   \includegraphics[width=1\linewidth]{undersea.jpg}
\end{center}
   \caption{Walter Koerschner မှရေးဆွဲထားသော ပင်လယ်အောက်အခြေစိုက်စခန်း ပုံကြမ်း။ သူသည် ၁၉၆၀-ခုနှစ်များတွင် အမေရိကန်လေတပ် China Lake, California Weapons Center ရှိ US Navy ၏ Rock-Site ပင်လယ်အောက်အခြေစိုက်စခန်းအဖွဲ့၏ ပုံဆွဲသူတစ်ဦးဖြစ်သည်။ Sauder ၏ရင်းမြစ်တစ်ခုက China Lake တွင် မြေအောက်ပေအနက် တစ်မိုင်ရှိမြေအောက်အခြေစိုက်စခန်းရှိကြောင်းဖော်ပြသည် \cite{22,23}။}
\label{fig:5}
\label{fig:onecol}
\end{figure}

Sauder သည် တစ်မိုင်လိုင်းမြန်နှုန်း ၂,၀၀၀ mph အထိရောက်ရှိနိုင်သော မြေအောက်မှိုင်းလေ့ဘေးရှင်းရထားများ၊ ပင်လယ်ပြင်အောက်တွင် ဆောက်လုပ်ထားသော အခြေစိုက်စခန်းများ (ပုံ \ref{fig:5}) နှင့် ဧကန်ပြင်အောက်တစ်လျှောက် လမ်းကြောင်းတွင် ဆူမားရားရေအောက်နယ်မြေတိုးဝင်သော တူးလမ်းများအကြောင်း သက်သေခံချက်များလည်း လက်ခံရရှိခဲ့သည်။ မက္ကဆီကိုနယ်ချဲ့ရေအောက်အခြေစိုက်စခန်းတစ်ခုအကြောင်း သက်သေခံချက်တစ်ခုနှင့်ပတ်သက်၍ Sauder က \textit{"Underwater and Underground Bases စာအုပ်ထုတ်ဝေပြီး နှစ်ခွဲခန့်အကြာမှာ ထူးခြားသည့် ပင်လယ်အောက်စီမံကိန်းတစ်ခုအကြောင်း သိရှိသူတစ်ဦးက ကျွန်တော်ကို ဆက်သွယ်လာခဲ့ပါတယ်... သူက ဒီစီမံကိန်းကို မက္ကဆီကိုနယ်ချဲ့၏ ပင်လယ်အပ်အောက်မှာ ဖြစ်ပြီး Parsons က စီမံအပ်နှံသူဖြစ်သည်ဆိုပြီး ထောက်ပြခဲ့ပါတယ်။ Parsons က ပင်လယ်ပြင်အောက် မြေအောက် ၂,၈၀၀ ပေ အနက်တွင် အသုံးပြုရန် အထူးပစ္စည်းကိရိယာတချို့ ဝယ်ယူခဲ့တယ်လို့လည်း ပြောပါတယ်... ဒီပစ္စည်းကိရိယာတွေက ထူးခြားလောက်အောင် ဒီနေရာတွေမှာ လူသားများ တကယ်ရှိနေမှုကို မြဲမြံစွာ စဉ်းစားထားတာပါ"} \cite{22}။
\begin{figure}[t]
\begin{center}
% \fbox{\rule{0pt}{2in} \rule{0.9\linewidth}{0pt}}
   \includegraphics[width=1\linewidth]{sub.jpg}
\end{center}
   \caption{Walter Koerschner \cite{22,23} မှ ရေးဆွဲထားသော ရေအောက် စွန့်စူးသင်္ဘော တံတား လမ်းကြောင်း ပုံအလင်းဖြင့် ပြထားခြင်း။}
\label{fig:6}
\label{fig:onecol}
\end{figure}
\begin{figure}[t]
\begin{center}
% \fbox{\rule{0pt}{2in} \rule{0.9\linewidth}{0pt}}
   \includegraphics[width=1\linewidth]{iran.jpeg}
\end{center}
   \caption{အဆိုပါအီရန်အစိုးရ၏အာဏာပိုင်ဗီဒီယိုမှမြေညီအောက် "အမုံးမြို့" ကိုပြသနေသည့်ဗီဒီယိုဗျည်းတစ်ခု \cite{39,40}။}
\label{fig:12}
\label{fig:onecol}
\end{figure}
If there really is a vast secret transcontinental network of 170+ underground and undersea bases dug to depths of miles beneath the surface under our feet, connected by hypersonic vacuum-tube maglev trains, funded using the fruits of our labor, the masses of humanity today would be in a state of terminal and blissful ignorance, not only unaware of what is under them but what lies ahead of them in the near future, as they lap up the empty and coordinated statements of their politician handlers.

တကယ်လို့ မြေအောက်မှာ မိုင်နှစ်ရာကျော်နက်ရှိုင်းစွာတူးဖော်ထားတဲ့ ၁၇၀ကျော်သည့် မြေအောက်နဲ့ ပင်လယ်အောက်သိုလ်များ၊ ဟိုက်ပါစော်နစ်ဗ्याकျှုန်တူးမြေအောက်လမ်းရထားလမ်းတစ်လျှောက် ချိတ်ဆက်ထားပြီး၊ လူဦးရေဓားထုတ်ခွင့်ကြေးနဲ့ ဖြည့်စွက်ထားတာဆိုရင်၊ လူငယ်အများစုဟာ သူတို့ခြေထောက်အောက်မှာ ဘာရှိလဲဆိုတာ၊ နောက်လာမယ့်အနာဂတ်မှာ ဘာဖြစ်လာမလဲဆိုတာမသိပဲ၊ သူတို့ရဲ့နိုင်ငံရေးသမားတွေက ထုတ်လိုက်တဲ့ အချည်းနှီး သဘောတူမှုအကြိမ်ကြိမ်ကိုသာ လက်ခံစားသောက်နေကြမယ်။

An additional note - the existence of large underground tunnel networks has been revealed without a doubt in the ongoing conflict in the Middle East (Hamas tunnels under the Gaza Strip \cite{38}, and Iran's underground "missile city" (Figure \ref{fig:12}) \cite{39,40}). These should leave no doubt as to both the possibility of building, and actual existence of, such structures. They should also leave us wondering what structures other significantly better capitalized countries may have constructed during the same time.

ထပ်မံညွှန်ပြချင်တာက- မကြာသေးခင်က လတ်တလောမှာ ဖြစ်ပွားနေတဲ့ အနောက်အလယ်ပိုင်းပြဿနာအတွင်း (ဂါဇာကွိုက်ခေါ်ရဲတွင်ရှိတဲ့ ဟမတ်မြေအောက်လမ်းတန်းများ \cite{38}, နဲ့ အီရန်ရဲ့ မြေအောက် “ထောက်လှမ်းရန်မြို့” (ပုံ \ref{fig:12}) \cite{39,40}) ကြီးမားမှုမြောက်မားတဲ့ မြေအောက်လမ်းကြောင်း တွေဆီ လုံးဝသံသယတစ်စုံတစ်ရာမကျန် ထင်ရှားစွာ ထုတ်ဖော်ပြသပြီဖြစ်ပါတယ်။ ဒါက ခိုင်မာတဲ့ လုပ်ဆောင်နိုင်ရေးနဲ့ တကယ်ရှိရှိတည်ရှိနေပါတယ်လို့ ယုံကြည်မှုပေးစေပါတယ်။ အဲဒီလို အခြား နိုင်ငံကြီးအချို့က ထပ်မံအသုံးချပြီး ထူထောင်ထားနိုင်မယ့်ဒီထဲက တကယ်ရှိနိုင်မှုကို ကျွန်ုပ်တို့စဉ်းစားစေမှာပါ။

\subsection{အပိုတောင်ခန်း အညီအမျှအရေးကြီးသောနေရာများနှင့် သဘာဝဘေးအန္တရာယ် အတွက် ပြင်ဆင်မှုအထောက်အထားများ}

\begin{figure}[t]
\begin{center}
% \fbox{\rule{0pt}{2in} \rule{0.9\linewidth}{0pt}}

\includegraphics[width=1\linewidth]{tyrol.jpg}
\end{center}
   \caption{အနောက် တရီရောလ် တွင် တောင်တန်းဘေးဒုက္ခရောက်လိမ့်မည့် နေနိုင်သော ဘုံကာများ။ ဆွစ်ဇာလန်သည် ဥရောပအင်အားကြီး တောင်တန်းတစ်လျှောက် ကြီးပြီး တောင်တန်းဘုံကာများကို သြဇာလျောက်စွာ ဖုံးကွယ်ထားနိုင်တာကြောင့် ကျော်ကြားသည် \cite{32}။}
\label{fig:7}
\label{fig:onecol}
\end{figure}

\begin{figure}[t]
\begin{center}
% \fbox{\rule{0pt}{2in} \rule{0.9\linewidth}{0pt}}
\includegraphics[width=1\linewidth]{svalbard.jpg}
\end{center}
   \caption{နော်ဝေမိတီလာ ဆွဲဘာတ်နိုင်ငံရှိ Svalbard Global Seed Vault တွင် အစေ့အစံအမျိုးတစ်သိန်းကျော် ထားရှိထားသည်။ \cite{24} ၎င်းကိုအသုံးပြုရန်အတွက် ဘယ်လောက်ကပ်လန့်သည့်မတော်တဆမှုမျိုးတစ်ခုက လိုအပ်လာမလဲဆိုတာ စဉ်းစားမိတတ်တယ်။}
\label{fig:8}
\label{fig:onecol}
\end{figure}

အမေရိကန်မြေအောက်တွင် ပဋိညာဉ်ခန်းမကြီးများအပြင် ကမ္ဘာတစ်ဝှမ်းမှာလည်း ကပ်ဘေးဖြစ်နိုင်ခြေကို ပြင်ဆင်နေသည့် အခြားသောအချက်အလက်များစွာရှိသည်။ နော်ဝေ၊ ဆွစ်ဇာလန်၊ ဆွီဒင်နှင့် ဖင်လန်တို့သည် ထူးခြားသည့် ဥပမာကောင်းများဖြစ်သည်။

\begin{flushleft}
\begin{enumerate}
    \item Project Camelot မှ နော်ဝေနိုင်ငံရေးသမားတစ်ဦးထံမှ တင်ပြချက်တစ်ခုကို မျှဝေခဲ့သည် \cite{25,26}၊ သူ၏ ကိုယ်ရေးအချက်အလက်ကို အတည်ပြုခဲ့သော်လည်း သိမစေရန်ထားရှိခဲ့သည်။ သူက နော်ဝေးတွင် မြေအောက်အခြေစိုက်စခန်းကြီး ၁၈ ဆောင် ရှိကြောင်း၊ နော်ဝေ (နှင့် အစ္စရေးနှင့် "နိုင်ငံအများကြီး") က သဘာဝဘေးအန္တရာယ်တစ်ခုခုကို ကြိုတင်ပြင်ဆင်နိုင်ရန်အတွက် ဤအခြေစိုက်စခန်းများကို ဆောက်လျက်ရှိကြောင်း ဆိုသည်။ Richard Sauder သည်လည်း နော်ဝေရှိ တောင်တစ်လုံးကို ဝင်ဖောက်တည်ဆောက်ထားသော မြေအောက်စခန်းကြီးအတွင်းသို့ ဝင်ရောက်ခဲ့ဖူးသူတစ်ဦးထံမှ တင်ပြချက်တစ်ခု လက်ခံရရှိခဲ့သည် \cite{22}။
    \item ဆွစ်ဇာလန်သည် အယ်လ်ပစ်တောင်ကြီးများတွင် နျူကလီယာခိုလှုနယ်များစွာ တည်ဆောက်ထားသည့်အတွက် ကျယ်ပြန့်စွာ သိရှိကြသည် (ပုံ \ref{fig:7})။ ၎င်းတို့သည် ပျမ်းမျှ ၃၇၀,၀၀၀ ကျော်ရှိပြီး၊ နေထိုင်သူတိုင်းအတွက် လုံလောက်စွာ အမိုးအောက်ပေးနိုင်သည် \cite{27}။
    \item ဆွီဒင်နှင့် ဖင်လန်တို့တွင်လည်း နေအိမ်အကြီးအမြတ်ရှိသော မြို့ကြီးတိုင်းအတွက် နေရပ်လူများအားလုံးအတွက် ခိုလှုံနိုင်သည့် ခိုလှုံနယ်များလုံလောက်စွာ ရှိသည် \cite{27}။
\end{enumerate}
\end{flushleft}

Silicon Valley ရှိ စီးပွားရေးရှုံးရွေးထောက်ကြီးများလည်း ဤအကြောင်းကို သိရှိနေကြသည်။ သတင်းအရ၊ \textit{"LinkedIn တည်ထောင်သူနှင့် ရင်းနှီးမြှုပ်နှံသူအထင်ရှား Reid Hoffman သည် The New Yorker သို့ ပြောကြားခဲ့သည်မှာ ၂၀၂၄ ခုနှစ်အစောပိုင်း၌ Silicon Valley မှ ဒေါ်လာဘီလျံရှင်များ၏ ၅၀ ရာခိုင်နှုန်းကျော်က 'ဟာသဘီလေ့ဘေးကင်းအာမခံမှု' တစ်ခုခု၊ ဥပမာ မြေအောက်ခိုလှုနယ်များကို ဝယ်ယူထားကြောင်း ခန့်မှန်းကြောင်း ဆိုသည်... Forbes က တယောက်ဖြစ်သော Jim Dobson မပြောသည်မှာ ဒေါ်လာဘီလျံရှင်အတော်များများမှာ မည့်သည့်အချိန်မဆို ထွက်ခွာနိုင်ရန် ပုဂ္ဂလိကလေယာဉ်များ ကိုင်တွယ်ထားကြောင်း ဖြစ်သည်။ သူတို့တွင် စက်ဘီခ်ကားများ၊ လက်နက်များနှင့် မီးစက်များလည်း ရှိသည်"} \cite{28}။

၎င်းအပြင် Global Knowledge Vault (Arch Mission Foundation မှ ဦးစီးသည်) \cite{29} နှင့် Svalbard Global Seed Vault \cite{30} တို့ကဲ့သို့ လူသားမျိုးနွယ်ဖြတ်ပိုင်းပြသာနာတစ်ခု ဖြစ်ပွားသည့်အခါ လူသားတို့၏ အရေးပါသည့် အထွေထွေဓေလ့ပစ္စည်းများကို ထိန်းသိမ်းရန် ကြိုတင်ပြင်ဆင်နေသော အကြီးစားမှတ်တမ်းစုံစမ်းရေးပရောဂျက်အမျိုးမျိုးလည်း ရှိသည်။
\begin{figure*}[t]
\begin{center}
% \fbox{\rule{0pt}{2in} \rule{.9\linewidth}{0pt}}
\includegraphics[width=0.9\textwidth]{govcrop2.png}
\end{center}
   \caption{ယူအက်စ်အစိုးရဝင်ငွေ၊အသုံးစရိတ်နှင့်လျှို့ဝှက်မြေအောက်မြေခံအခြေစိုက်စခန်းအသုံးစရိတ်များကို 1998 မှ 2023 အထိ ပြသထားသည် \cite{19}.}
   \label{fig:9}
\end{figure*}
\section{ဒေမိုကရက်တစ် အုပ်ချုပ်မှုဖြင့် ကြီးမားသည့် မြေအောက်အခြေစိုက်စခန်းများအတွက် ငွေကြေးရယူမှုဆိုင်ရာ နည်းလမ်းများ}

သီလဝါမြေအောက်နှင့် သမုဒ္ဒရာအောက်တွင် တည်ဆောက်ထားသည့် အခြေစိုက်စခန်း ၁၇၀ ကျော် ပါဝင်သည့် အမျှော်အလှန်ကွန်ယက်ကြီးများသည် ချွေတာနေသူများကို မသိအောင် ဘယ်လို ငွေပေးအပ်ကြသည်လဲ? ဤလုပ်ငန်းစဉ်ကြီးသို့ စိုက်ထုတ်သည့် ငွေပမာဏအဆင့်အတန်းနှင့် ထွက်အလာကို ခန့်မှန်းနိုင်စေရန် စာရွက်စာတမ်းတစ်ခုရှိသည်။ ၂၀၁၇ တွင် အမေရိကန် ရင်းနှီးမြှုပ်နှံမှု ဗဟုသုတရှင်နှင့် ဘူရှ်အစိုးရ၌ အစိုးရအရာရှိဟောင်း Catherine Austin Fitts နှင့် မစ္စဂန် ပြည်နယ်တက္ကသိုလ် စီးပွားရေးပညာရှင် Mark Skidmore တို့သည် ၁၉၉၈-၂၀၁၅ ရေး အသုံးစရိတ်နှစ်များအတွင်း အမေရိကန်အစိုးရတွင် ခွင့်မပြုဘဲ အသုံးစရိတ် ၂၁ ထရီလျံဒေါ်လာ ရှာဖွေတွေ့ရှိခဲ့သည် \cite{11,12,13}။

သူတို့၏ အစီရင်ခံစာအရ၊ \textit{"၂၀၁၆ ခုနှစ် အောက်တိုဘာ ၇ ရက်နေ့တွင် Reuters သည် Scot Paltrow (2016) ၏ ဆောင်းပါးတစ်ပုဒ်ကို ထုတ်ဝေခဲ့ပြီး၊ ၂၀၁၅ အသုံးစရိတ်နှစ်တွင် လေတပ်သည် ၏စာရင်းများကိုချိတ်ဆွဲထားသည်ကို ပေါ်လွင်အောင်ဖြစ်စေရန် စာရင်းချုပ်ပြုပြင်ခြင်းများဖြင့် \$6.5 ထရီလျံ ဒေါ်လာကို ထည့်သွင်းခဲ့သည်ဟု ဖော်ပြခဲ့သည်။ ထိုနှစ်အတွက် လေတပ်အထွေထွေဖောင်စာရင်းသည် \$122 ဘီလျံ သာရှိသည်ဖြစ်သော်လည်း၊  ဤသည်မှာ အံ့အားသင့်ဖွယ် ရာဇဝတ်ဖြစ်သည်... DOD သည် ၂၀၀၁ စက်တင်ဘာ ၁၀ ရက်နေ့တွင် ကွန်ဂရက်အလွှာတွင် အမျိုးသားကာကွယ်ရေးဝန်ကြီး Donald Rumsfeld မှ DOD သည် \$2.3 ထရီလျံစာလည်အလဲလှယ်မှုတွင် ခေါင်းကို မပေးနိုင်ကြောင်း ထုတ်ဖေါ်ပြောကြားခဲ့တဲ့အကြောင်းကြောင့် သတင်းဌာနကြီးကြီးများတွင် ပထမဦးဆုံး နာမည်ကြီးခဲ့ပါသည်... ဤအထောက်အထားသည် နေ့အတိတစ်နေ့ ကမ္ဘာလုံးဆင်မြန်းရေးအာရုံကို ၉/၁၁ ဖြစ်စဉ်ကြီးကသန့်ရှင်းသွားစေသော်လည်း မြန်မြန်မေ့လျော့သွားခဲ့သည်... Professor Mark Skidmore သည် လေတပ်တွင်၎င်း \$6.5 ထရီလျံစာ အသုံးစရိတ်များကို မယူမှတ်နိုင်ကြောင်းသိရှိသောအခါတွင်၊ မစ္စစ် Fitts ကို ဆက်သွယ်သွားပြီး၊ ၂၀၁၇ နွေ ဦးတွင် HUD နှင့် DOD အတွက် ယင်းလိုမျိုး မသက်သေစိစစ်နိုင်သည့် လုပ်ငန်းဖော်ပြချက်ကြီးများကို ပြန်လည်ရှာဖွေရန် လက်တွဲလုပ်ရန် သဘောတူခဲ့သည်။ နောက် ၆ လအတွင်း Skidmore, Fitts နှင့် မိသားစုကျောင်းသားအနည်းငယ်တို့သည် ၁၉၉၈-၂၀၁၆ ကြာမြင့်သည့်ကာလအတွင်း USD\$21 ထရီလျံ မမှတ်တမ်းတင်နိုင်သည့် လုပ်ငန်းဖော်ပြချက်များကို စုစည်းနိုင်ခဲ့သည်"} \cite{12}။

၁၉၉၈-၂၀၁၅ ရီးနှစ် ၁၈ ကာလအတွင်း တရားဝင် အသိအမှတ်ပြုထားသည့် အမေရိကန်အစိုးရ ဝင်ငွေသည် \$40.8 ထရီလျံသာရှိသည် \cite{15}။ ၎င်းသည် အစိုးရဝင်ငွေအရွယ်အစား၏ တဝက်ကျော်ကို မြေအောက်အခြေစိုက်စခန်းများအတွက် လျှို့ဝှက်အသုံးပြုထားကြောင်း ပြသသည်။ ထို့အပြင် ဤလျှို့ဝှက်အသုံးစရိတ်သည် ဒီဖစ်စစ်စာရင်းကျရှုံးမှုကြီးအပေါ်တွင် ဆက်လက်ဖြစ်ပွားခဲ့သည်။ ယနေ့ထိ ဆက်လက်ဖြစ်နေသည်နှင့် ၁၉၉၈ မတိုင်မီလည်း ဖြစ်နေခဲ့ကြောင်း သုံးသပ်ရသည်။ ထို့ကြောင့် မြေအောက်အခြေစိုက်စခန်းများအတွက် ထုတ်လုပ်မှုစာရင်းစုစုပေါင်းသည် \$21 ထရီလျံထက် ပိုကြီးသည်။ ၂၀၁၆-၂၀၂၃ ကာလအပေါ်လည်း ယတစ်ခုတည်းနှုန်းဖြင့် အနှစ်ချုပ်ပြုလုပ်ပါက ၁၉၉၈ နောက်ပိုင်းတွင် \$36.6 ထရီလျံ အသုံးပြုကြောင်း တွက်ချက်နိုင်သည်။

၂၀၂၁ တွင် Mark Skidmore သည် Bloomberg မှ ပြောကြားခဲ့သည့် ၂၀၁၇-၁၉ အသုံးစရိတ်နှစ်များအတွင်း Pentagon သည် ခေါင်းမထည့်နိုင်စေရန် ဝင်ငွေစာရင်းပြုပြင်မှုစာရင်း \$94.7 ထရီလျံ ဖြစ်ခဲ့ကြောင်းကို ဆက်လက်သုတေသနထုတ်ပြန်ခဲ့သည် \cite{17,18}။ ၁၉၁၃ ခုနှစ်ရှိ Federal Reserve တည်ထောင်ပြီးနောက် တစ်ရာစုထဲအတွင်း အခြေစိုက်ခဲ့သော ဗဟိုဘဏ္ဍာစနစ်မှတစ်ဆင့် အမေရိကန်ဒေါ်လာကို ကူးယူမှုများကို ထည့်သွင်းစဉ်းစားသည့်အခါ \cite{37}၊ ဘဏ္ဍာရေးအများသုံးစာရင်းအားလုံးသည် အဓိပ္ပာယ်အလွန် အလှည့ံခံယူမှုဖြစ်ပြီး၊ အမေရိကန်ငွေကြေးနှင့် အစိုးရသည် ၎င်း၏ သခင်ကြီးတို့လိုလေ့လာသလို ချည်းကပ်ယူယူလို့ရနိုင်ကြသည့် သယောဇဉ်အရင်းအမြစ်ဝေမျှရေးစနစ်ဖြစ်သည်။
\section{Jove ၏ သားရဲတော်တို့ - အရိပ်အောက် အနောက်ဘက် ဘုရင်များ၏ ကိုယ်ပိုင်အမည်များ}

ဒါဆို အဲ့ဒီအပြိုင်အဆိုင်ကို တကယ်တိုက်ထုတ်နေရတဲ့သူကဘယ်သူလဲ။ ကျွန်ုပ်တို့သိနိုင်မှာမဟုတ်ပေ။ အကြောင်းမှာ၊ အနောက်ဘက် ငွေကြေးပိုင်ရှင်ဘုရင်တွေက သူတို့ကို အရိပ်ထဲမှာထားနေလို့ပဲဖြစ်သည်။ သီအိုရီ အနည်းအကျယ် မျိုးစုံရှိတယ်၊ အများသိပုဂ္ဂိုလ်တစ်ဦးတစ်ယောက်ကနေ ပင်လယ်နယ်ဘက်သားထိ ပြောဆိုကြသေးပေမယ့်၊ ကျွန်ုပ်မှာ အကောင်းဆုံးဖြေရှင်းချက်ကတော့ "Amallulla" ဆိုတဲ့ နာမည်မှောင်မှအမည်ခံ ဘလော့ဂါးတစ်ယောက်ရဲ့ ဘ ၀ တစ်လျှောက် လက်ရာ ဖြစ်သည်။ သူ့လက်ရာမှာ ရှေးကာလနဲ့ မျက်မှောက်လောကသမိုင်း၊ လျှို့ဝှက်အနက်ချက်များ၊ နဲ့ အနောက်နိုင်ငံရေးများအပေါ် တစုံတရာ မရှိမဖြစ်စာအုပ် ၅၀ နဲ့ စာရေးသူ ၂၀ ကျော်က အရေးပါဆုံး စုပေါင်းခြင်းတစ်ခု ဖြစ်သည် \cite{33,34}။ သူ့လက်ရာကို ယခု လာမယ့် ဘောင်သဘာဝဘေးအန္တရာယ်အပေါ်မှာတော့ "မေတ္တာတော်" ဟုသာ ဖော်ပြပေးနိုင်သည်။ ၎င်းဟာ ကျွန်ုပ်ရေးသားချက်ထက် \textit{တစ်ဆ ပိုပြီး} လုံးဝဝလုံးကူးအောင်၊ စုံလင်ပါဝင်သည်။

Amallulla က နိုင်ငံရေး အနောက်တိုင်း ဖက်ရှင်သုံးခုကို ရွေးချယ်သတ်မှတ်ခဲ့ပြီး၊ အတူတကွ သူတို့ကို "Jove ၏ သားရဲတော်များ" ဟု ခေါ်ခဲ့သည်။ သူတို့မှာ "နောက်ဆုံးခေတ်" ကို သိရှိထားကြသည် - မြေကြီး၏ နောက်နှုတ်သဘောဖြစ်အကြိမ်ကြိမ်ဖြစ်ပွားမှုများ။ သူ့အမြင်အရ ဒီဖက်ရှင်သုံးခုသည် ယနေ့ အနောက်နိုင်ငံများကိုတော်တော်ကောင်းကောင်း ထိန်းချုပ်နေကြပေမယ့်၊ မတူညီသော မူလနှင့် သမိုင်းဝင်အမည်များ၊ အတိတ် ရန်သူဖြစ်မှုများနှင့် တန်ဖိုးစနစ်၊ လုပ်ရပ်များ အကွာအဝေးများအပေါ် မူတည်၍ သုံးဖွဲ့ခြင်းကို ခွဲထုတ်ခဲ့သည်။

ဖက်ရှင်သုံးခုကို ခန့်မှန်းအားဖြင့် အောက်ပါအတိုင်း ခွဲခြားနိုင်သည်-

\begin{flushleft}
\begin{enumerate}
    \item \textbf{ဘဏ်လုပ်ငန်းရှင်များ}: ရှေးရိုးပခေတ် ရောမအရောင်ကျော်အထတ်၊ နောက်တစ်ကြိမ် Knights Templar ဖြစ်လာပြီး အမေရိကန်မှာ မြောက်ပိုင်းသံယောဇဥ် Freemasons ဖြစ်လာကြသူများ။
    \item \textbf{စဉ်းစားသူများ}: Rosicrucians များနှင့် တောင်ပိုင်းအမေရိကန် Freemasons များ။
    \item \textbf{Jesuits နှင့် Black Pope}: ရောမကက်သလစ်ဘုရားကျောင်းအတွင်း Jove ၏အမျိုးသားမျိုးနွယ်တန်းအုပ်စု။

ယနေ့မှာ ဤအုပ်စုသုံးခုသည် ဥရောပ Illuminati, Freemasons, နှင့် CIA တည်းဖြစ်ကြသည်။ Amallulla က ဖော်ပြခဲ့သည့်အတိုင်း - \textit{"အခုချိန်၊ အဆုံးသတ်အချိန်မှာပါပဲ၊ Jove ၏အမျိုးသမီးသားတို့သည် နိုင်ငံတော်သမ္မတနောက်လူပေါင်းများသည်ကိုပါ ပြီးပြည့်စုံအောင် ဝှက်ထားနိုင်သော need-to-know clearances နောက်က ခြုံငုံအောင် ဝှက်ထားလျက်ရှိကြသည်။ တစ်ခြားနည်းပြောရမယ်ဆိုရင် သူတို့သည် ကိုယ်တို့ကိုလူထုသည်ကြည့်ရှု့မရအောင် ဝှက်၍သားထုတ်နိုင်မှုအနုပညာကို အမြင့်ဆုံးအဆင့်အထိ ပြုလုပ်နိုင်လာပြီ။ \textbf{Jove ၏အမျိုးသမီးသားတို့သည် ယူအက်စ်အေ့၏ စစ်တပ်နှင့် အစိုးရတစ်လျှောက်လုံးကို ထိန်းချုပ်နိုင်သလို၊ ငွေကြေးစနစ်၊ ကြီးမားသည့်ကုမ္ပဏီများနှင့် သူတို့တည်ထောင်ခဲ့သော သမ္မတစနစ် (နိုင်ငံရေးသမားများအလွယ်တကူပျက်သပိတ်ကပ်နိုင်ကြောင်းသိနေပြီး ထိန်းချုပ်နိုင်သည့်အကြောင်း) အသုံးပြု၍ အနောက်တိုင်းကမ္ဘာတစ်လွှားကို ထိန်းချုပ်နေကြသည်။"} \cite{33,34}.

\begin{figure}[t]
\begin{center}
% \fbox{\rule{0pt}{2in} \rule{0.9\linewidth}{0pt}}
   \includegraphics[width=1\linewidth]{illuminati.jpg}
\end{center}
   \caption{Jove ၏ မျိုးဆက်များဆိုတာ တကယ် ဘယ်သူတွေပါလဲ? (ရုပ်ပုံ: \cite{35})}
\label{fig:10}
\label{fig:onecol}
\end{figure}

\begin{figure}[t]
\begin{center}
% \fbox{\rule{0pt}{2in} \rule{0.9\linewidth}{0pt}}
   \includegraphics[width=1\linewidth]{pike.jpg}
\end{center}
   \caption{အနီရောင်ဖြင့် အထူးဖော်ပြထားသော မွန်စာရေးသားသော ပိုင်ခ်ပိန်း ဘာသိုလစ်နှင့် အနောက်အမေရိက ပြည်ထောင်စု၏ ရှုခင်းများ \cite{36}။ အမေရိက ပြည်ထောင်စုသည် ဤနေရာကို ထိန်းချုပ်ရန်အတွက် တကယ်ပင် စဉ်းစားဖန်တီးခဲ့ခြင်းလား။}
\label{fig:11}
\label{fig:onecol}
\end{figure}

Amallulla ၏အဆိုအရ၊ ဒီလူတွေက သာသနာကို အနုညာတစွာ ဆူညံလေ့ရှိပြီး၊ ကမ္ဘာ့အဓိကဘာသာတရားဝင်စာအုပ်တွေကို ကိုယ်ပိုင်အကျိုးအတွက် သုံးတတ်ပါတယ်၊ အကြောင်းအရာသင်္ကေတကိုလည်း ရှုပ်ထွေးစွာ အသုံးပြုလေ့ရှိပါတယ်။ ထို့အပြင် သူတို့ရဲ့ ရန်သူတွေကိုလည်း မနားတစ်၊ မတရားတတ်ပါတယ်- \textit{"\textbf{နှစ်ပေါင်း ၂,၆၀၀ ကျော်အတွင်း၊ သူတို့နောက်ဆုံးအချိန်ဆိုင်ရာ သိရှိမှုတစ်ခုခုရှိသူအားလုံးကို စနစ်တကျ ဖျက်စီးခဲ့ကြသည်။ ဤအရာသည် ဥပမာအားဖြင့် druids, ဂျူးကာဘာလစ်များ, ဗဟုဗေဒ မြန်မာအီဂျစ်များ, အာရပ်များနှင့် အိန္ဒိယပြေဆရာကြီးများကိုသာ ဆိုလိုသည်မဟုတ်ဘဲ၊ တောင်အမေရိကရှိ ရှည်လျားသောခေါင်းဇီဝများနှင့် အလယ်အမေရိကရှိ မာယာဘုန်းတော်ရုပ်များကိုလည်း ဆိုလိုသည်။ ထိုသို့ရပ်တည်နေသော မြောက်အမေရိက လူဦးရေများကို ဖျက်စီးခဲ့ခြင်းသည် ဤနေရာကို နောက်ဆုံးအချိန်ဒေသအဖြစ် ထိန်းသိမ်းရန်အတွက် လုံးဝရုပ်သိမ်းထားသူတို့အုပ်စုဖြစ်ကြောင်း အထောက်အထားက ပြင်းပြင်းထန်ထန် ရှိနေသည်။ အမေရိကန် "အင်ဒီယန်" လူမျိုးစုကြီးအပျက်အလွန်သည် လက်မောင်းရှင်းမှုပြုလုပ်မှုသာ ဖြစ်သည်}"} \cite{33,34}
Amallulla သည် "ပြည်ထောင်စုအမေရိကန်ပြည်ထောင်စု" စီမံကိန်းတစ်ခုပြီးလုံးကို "Pike's Peak Batholith" ကို ထိန်းချုပ်နိုင်ရန် ရည်ရွယ်ချက်ဖြင့် ဆောင်ရွက်ခဲ့သည်ဟု ယုံကြည်ခဲ့သည်။ ၎င်းသည် Rocky မောင်တောင်တန်းရှိ granite မောင်တောင်တန်း စီးရီးတစ်ခုဖြစ်ပြီး ဘာသာရပ်ဆိုင်ရာဒုက္ခတွင် အလွန်ကောင်းမွန်သောကာကွယ်မှုကိုပေးစွမ်းနိုင်သည် (ပုံ \ref{fig:11})။ Amallulla ၏ထုတ်ဖော်ချက်အရ၊ \textit{"အမေရိကန်အပြည်ပြည်ဆိုင်ရာစစ်ပွဲမတိုင်မီ၊ အတွင်းအခန်းတွင်မူ၊ နှင့်ပြီးနောက်တွင်ပင် ဘဏ်လုပ်ငန်းရှင်များနှင့် စာအိုများသည် ပြည်ထောင်စုအမေရိကန်ပြည်ထောင်စုကို ထိန်းချုပ်ရန်မကပါဘဲ Pikes Peak batholith ကိုပင် တိုက်ယူခဲ့ကြသည်။ ဤ granite batholith သည် ကမ္ဘာပေါ်တွင် တွေ့ရှိရသောအထူးဆုံး granite batholith တွေထဲမှတစ်ခုဖြစ်သည်... ကမ္ဘာပေါ်တွင်မျှမရှိသည့် အမြင့်ဆုံးတောင်ပိုင်းမှာ၊ ပင်လယ်ကမ်းခြေမှဝေးလွင့်သည့် granite batholith တစ်ခုဖြစ်သည်။ ဒါဟာ ကမ္ဘာ့မူတည်ရာပြောင်းလဲမှုကို ရှင်သန်ချိန်သက်ရာတွင် အကောင်းဆုံးတည်နေရာ ဖြစ်သည်"} \cite{33,34} ။ Amallulla ၏သုတေသနတွင် ယနေ့တွင်ဤဒေသအောက်မြေ၊ နှင့် ပတ်ဝန်းကျင်မှာ မြေအောက်လမ်းကြောင်းစနစ်ကျယ်ပြန့်စွာ ဆောက်လုပ်ထာင်းမှုရှိသည်ဟု ဖော်ပြခဲ့သည် \cite{36}။

\section{နိဂုံးချုပ်}

ဤစာတမ်းတွင် အနောက်တိုင်းစိုးစံအထွားများသည် ကမ္ဘာပေါ်၌ ပြန်ထပ်တလဲလဲဖြစ်ပွားနေသောဘေးအန္တရာယ်များအား ထောင်နှန့်သောနှစ်ပေါင်းများစွာကြာထိန်းသိမ်းထားသည်၊ ထပ်မံဖြစ်ပွားလာမည်ဟုထပ်မံယုံကြည်သည်၊ ဤသို့ဖြစ်လာရင် စီစဉ်ပစ္စည်းများကောင်းစွာ စုပေါင်းပြီး မြေအောက်ပါလုံခြုံခိုလှုံရာများတည်ဆောက်ထားသည်နှင့် ပေါင်းစည်းတော်မူသည့်ဥပမာတချို့ကို ဆွေးနွေးဖော်ပြခဲ့သည်။ ထိုသို့ဖြစ်ပေါ်လာမည့်ဖြစ်ရပ်ကို နိုင်ငံရေးနှင့် စစ်ရေး အားဖြင့် အပိုင်အယားလုပ်နိုင်ရန် အစီအစဉ် ရှိနေသည်ဆိုသော သက်သေအထောက် အမျိုးမျိုးကိုလည်း ဖော်ပြတင်ပြခဲ့သည်။ ဤအရာကို အမေရိကန်ထဲတွင်ဘယ်လိုအသုံးစရိတ်ရရှိခဲ့သလဲဆိုသည်နှင့် ဒီအစီအစဉ်ကို ဦးဆောင်နေသော သွေးမျိုးချုပ်စိတ်ကူးအကြောင်း ထုံးစံမပြုတ်ဆုံးဖြတ်ချက်များကိုလည်း ကိုးကားဖော်ပြခဲ့သည်။ နောက်ထပ်အသေးစိတ်သိလိုသူများအတွက် ကိုးကားစာရင်းများထဲတွင် လေ့လာရှာဖွေရန် ရှိနေသည့် အချက်အလက်များအတော်များများရဲ႕တစ်စိတ်တစ်ပိုင်းသာ ဖော်ပြထားပါသည်။

မကြာခင်တွင် အသိဉာဏ်ပညာရှင်များ တိုက်တွန်းပြောဆိုနေသည့် သင်္ကေတအိမ်သာပြောင်းလဲလာသည့် ရာဇဝင်အဖြစ်အပျက်နှင့်ပတ်သက်ပြည့်စုံသော အကြောင်းအရာအားလုံးသည် ကမ္ဘာ့ပင်လယ်ထုအကြောင်းအရာ မော်ဂျူးနှစ်ခုအား ဝက်ဘ်ဆိုဒ်တွင် စာတမ်းနှစ်စောင်တိတိကို အပြည့်အစုံ ဆွေးနွေးဖော်ပြထားသည် \cite{3}။

\begin{figure}[t]
\begin{center}
% \fbox{\rule{0pt}{2in} \rule{0.9\linewidth}{0pt}}
   \includegraphics[width=1\linewidth]{npw.jpg}
\end{center}
   \caption{၁၅၉၀ မှ ၂၀၁၅ ထိ ဂျီဩမက်နက် မြောက်ဘက် pól ၏ တည်နေရာကို ၅ နှစ်ခြားပြသထားသည် \cite{41}။ ၁၉၇၅ တွင် ၎င်း၏ရွေ့လျားမှုမြန်မြန်ဆန်ဆန်တက်လာသည်။}
\label{fig:13}
\label{fig:onecol}
\end{figure}

နောက်ဆုံးတွင်၊ ဗေဒင်ကြိုတင်မြင်ဆိုခြင်းဖြင့် Amallulla မှ ပြောကြားခဲ့သည့်ဤမိန့်ခွန်းကို တင်ပြချင်ပါတယ်၊ \textit{"\textbf{အားလုံးသည် တစ်ချစ်တည်းဖြစ်ကြသည်}"} ဟု ရှင်းလင်းတင်ပြထားသည်။ \textit{"ဤနေရာတွင် မင်းရဲ့ စိတ်ကူးယဉ်ချက်ကို အလွန်အကန့်ထိ တိုးမြှင့်စေလိုပါတယ်။ မင်းလက်ရှိနေထိုင်လေ့လာခဲ့သော ကမ္ဘာကိုမေ့လိုက်၊ ပြန်ကြည့်စရာမလိုဘူး။ ဤကမ္ဘာသည် Matrix ရုပ်ရှင်ထဲကလို ပြုလုပ်ထားသော တုယ🌕နဲ့ ယုံကြည်မှုဖြစ်ပြီး နောက်ဆုံးအချိန်အထိ မီးပျောက်အိပ်စက်နေစေရန် ရည်ရွယ်မှုရှိသည်။ တစ်ခါတလေ ဘာသာပြန်ဇယားရေးသားနေတယ်လို့ ထင်မြင်ချင်တယ်၊ ဒါပေမယ့် ဤဝက်ဘ်ဆိုဒ်မှာ မျှဝေဖို့ ဆုံးဖြတ်သမျှဟာ တကယ်အဖြစ်အပျက်ပါပဲ။ “အားလုံးသည် တစ်ချစ်တည်းဖြစ်သည်” ကို သိမြင်ရန် ကြာမြင့်သြားခဲ့ရတယ်၊ နောက်ဆုံးမှာတော့ An Apocalyptic Synthesis အတွက် နာမည်တပ်လိုက်ခဲ့တယ်။ ဒီဟာကို ရှင်းပြရခက်ပါတယ်။ အခု Matrix ရုပ်ရှင်အနေနဲ့ စဉ်းစားကြရအောင်။ Analogy လည်း ရှင်းလင်းပါတယ်။ ဒီမှာ ခက်ခဲတာကတော့ ငါပြောမယ့်အချက်ဟာ မြှင့်တင်ဖေါ်ပြတာမဟုတ်ဘူးလို့ နားလည်စေချင်တာပါ။ ယခုအခါ Matrix ရုပ်ရှင်အစိတ်အပိုင်းက မင်းတွေ့နေရမှာနီးစပ်ဆုံး။ \textbf{မင်းဘဝထဲက အားလုံး၊ မှတ်တမ်းတင်သမိုင်းစဉ်အပါအဝင်၊ အစုလိုက်သဘောတူသိပ္ပံနဲ့ ပညာရေး၊ နိုင်ငံရေး၊ ဘာသာရေး၊ ဘာသာစကားတိုင်းဟာ လာမယ့် ကမ္ဘာသဲအလွှာလွှဲစေမှု သို့မဟုတ် ဝင်ရိုးမော်ဒ်လှည့်ထောင့်နဲ့ သက်ဆိုင်ပါတယ်။} ဒီအမှုကို အခုအခါ မမြင်ရသေးနိုင်ဘူး။ ဒါတောင် မင်းက မကောင်းသောအိပ်မက်ကနေ အလွယ်တကူနိုးထသလို ကြည့်လို့ မရဘူး။ အချိန်ယူရမယ်။ ဒါပေမယ့် ငါကအာမခံတယ်၊ ဒီလမ်းအဆုံးမှာ ကိုယ်တစ်လျောက်လုံးကို Matrix ကွန်ပျူတာထဲရှာဖွေရာ အဖြစ်မှန်ရုပ်ပြမှုအတုမှာ နေခဲ့တယ်ဆိုတာ သိလိုက်ရမှာပါ"} \cite{33,34}။
Good luck to all.

\section{Acknowledgments}

သိပ္သည့် ဒေသတြင် သိပ္ပံအတတ်အစွမ်းများဖြင့် ပါဝင်ကူညီခဲ့သော လူအပေါင်းတို့အား ကျေးဇူးအထူးတင်ရှိပါသည်။ သင်တို့ မရှိဘဲ ဤအလုပ်များ ဆောင်ရွက်ပိုင်နိင်မည် မဟုတ်ဘူး၊ လူ့အဖွဲ့အစည်းလည်း همچဉ်ဉာဏ် လှမ်းတောက်နေဆဲ ဖြစ်နေမည်။ သင်တို့၏ ရွေးချယ်မှုများသည် အမြဲတမ်းအောင်ပျော်သွားမည်။ ကျွန်ုပ်တို့သည် သင်တို့အား တန်ဖိုးထားရသည်၊ ကျွန်ုပ်အားလုံးအနေဖြင့် အထူးပင် ကျေးဇူးတင်ပါသည်။

\clearpage
\twocolumn

{\small
\bibliographystyle{ieee}
\bibliography{egbib}
}

\end{document}
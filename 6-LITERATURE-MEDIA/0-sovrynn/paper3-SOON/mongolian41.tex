\documentclass[10pt,twocolumn,letterpaper]{article}

\usepackage{booktabs}
% \usepackage{caption}
% \captionsetup[table]{skip=8pt}   % Зөвхөн хүснэгтэд нөлөөлнө
\usepackage{stfloats}  % Үүнийг өмнөх хэсэгт нэмнэ үү
\usepackage{float}

\usepackage{newunicodechar}
\usepackage{graphicx}

% Define a half-width em-dash
\newcommand{\shortemdash}{\scalebox{0.5}[1]{\textemdash}}

% Map U+2015 to this shorter dash
\newunicodechar{─}{\shortemdash}

\usepackage{fontspec}
\usepackage{ucharclasses}

%–– your fonts ––
\newfontfamily\latinfont{Latin Modern Roman}                % default (Latin)
\newfontfamily\mongscriptfont[Script=Mongolian]{Noto Serif}  
\newfontfamily\cyrillicfont[Script=Cyrillic]{Noto Serif}      % or Noto Serif, etc.

%–– detect and switch ––
\setDefaultTransitions{\latinfont}{}                        
\setTransitionsFor{Mongolian}{\mongscriptfont}{\latinfont} 
\setTransitionsFor{Cyrillic}{\cyrillicfont}{\latinfont}

\usepackage{cvpr}
\usepackage{times}
\usepackage{epsfig}
\usepackage{graphicx}
\usepackage{amsmath}
\usepackage{amssymb}

\usepackage[breaklinks=true,bookmarks=false]{hyperref}

\cvprfinalcopy % *** Энэ мөрийг эцсийн хувилбарт тайлбарлаарай

\def\cvprPaperID{****} % *** CVPR өгүүллийн ID-г энд оруулна уу
\def\httilde{\mbox{\tt\raisebox{-.5ex}{\symbol{126}}}}

\renewcommand{\refname}{Эшлэлүүд}
\renewcommand{\tablename}{Хүснэгт}
\renewcommand{\figurename}{Зураг}   % or whatever you like instead of "Hình"

\makeatletter
\def\abstract{%
  \centerline{\large\bf Хураангуй}% <-- your new label
  \vspace*{12pt}%
  \it%
}
\makeatother

% This makes the font slightly bigger than base (10) and bold in Subsection headings rather than using ptmb
\makeatletter
\def\cvprsubsection{%
  \@startsection{subsection}{2}{\z@}%
    {8pt plus 2pt minus 2pt}{6pt}%
    % {\normalfont\bfseries\selectfont}%
    {\normalfont\bfseries\fontsize{11}{13}\selectfont}%
}
\makeatother

% So this hardcodes the style for the numbers in the section/subsection headings so they're bold
\font\elvbf=ptmb scaled 1100
\font\elvbfs=ptmb scaled 1200
\makeatletter
% Section number: Large + bold
\renewcommand\thesection{%
  {\elvbfs\arabic{section}}%
}

% Subsection number: normalsize + bold + custom punctuation
\renewcommand\thesubsection{%
  {\elvbf
   \arabic{section}.\arabic{subsection}}%
}
\makeatother

% Хуудасны дугаарлалтыг илгээх хэлбэртээр дугаарлаж, хэвлэхэд дугааргүй болгодог
%\ifcvprfinal\pagestyle{empty}\fi
\setcounter{page}{1}
\begin{document}

\title{ECDO Илгээлт 3: Өнөө цагийн Барууны засаглалын төлөөлөгчид ойрын ирээдүйн геофизикийн их гамшигт бэлтгэж байгаа нотолгоо}

\author{Жунхо\\
Зарлагдсан огноо: 2025 оны 6 сар\\
Вэбсайт (Өгүүллүүдийг эндээс татаж авна уу): \href{https://sovrynn.github.io}{sovrynn.github.io}\\
ECDO судалгааны сан: \href{https://github.com/sovrynn/ecdo}{github.com/sovrynn/ecdo}\\
{\tt\small junhobtc@proton.me}
}

\maketitle
%\thispagestyle{empty}

\begin{abstract}

\begin{flushleft}
2024 оны тавдугаар сард “Ёс суртахууны скептик” \cite{0} нэртэй нууц нэртэй цахим зохиогч Exothermic Core-Mantle Decoupling Dzhanibekov Oscillation (ECDO) \cite{1} гэж нэрлэгдсэн шинийг санаачилсан онолыг хуваалцсан. Энэ онол нь дэлхий урьд өмнө огцом, сүйрлийн тэнхлэгийн шилжилтүүдийг туулж, эргэлтийн инерциэс шалтгаалан далай тэнгисүүд эх газрыг давахад хүргэсэн асар том дэлхийн үерүүдийг үүсгэсэн гэж үздэг. Мөн энэ онол нь тайлбарлах геофизикийн процесс болон дахин ийм эргэлт ойрхон болох магадлалтай гэдгийг харуулсан мэдээллийг танилцуулдаг. Иймэрхүү сүйрлийн үер, дэлхийн сүйрлийн таамаглал шинэ зүйл биш боловч ECDO онол нь шинжлэх ухаанч, орчин үеийн, олон салбарыг хамарсан, мэдээлэлд суурилсан хандлагаараа онцгой сонирхол татаж байна.

Энэхүү өгүүлэл нь энэ сэдвээр бичсэн миний гурав дахь ажил \cite{2,3} бөгөөд одоогийн улс төрийн талуудыг онцолсон болно:
\begin{enumerate}
    \item Барууны хүчнүүд геофизикийн сүйрэл ойртож байгаа гэж үздэг бөгөөд энэхүү үйл явдлыг улс төр, цэргийн давуу тал болгон ашиглахаар төлөвлөж буй талаар илчилсэн гэрчийн мэдүүлэг.
    \item Уг үйл явдалд бэлтгэх зорилгоор өргөн хүрээтэй, газрын дор болон далайн дор байр ангиудыг Батыг барьсан гэх нотолгоо.
    \item Эдгээр баазуудыг санхүүжүүлэхийн тулд Барууны санхүүгийн бүтцээс их хэмжээний мөнгө гадагш урсан гарч байгаагийн баримтууд.
\end{enumerate}
\end{flushleft}
Энэ баримт дэлхийн засаглалын эрх мэдэлтнүүд ойрын ирээдүйд тохиох геофизикийн гамшигт бэлтгэж буй өргөн хүрээтэй бэлтгэл ажлуудыг баримтжуулсан байна.
\end{abstract}

\section{Чөлөөт барилгачдын нууц холбоо ба "Англо-Саксоны ажиллагаа"}

2010 оны нэгдүгээр сард, шүгэл үлээгчдийн мэдүүлгийг цуглуулдаг өөр хэвлийн мэдээллийн байгууллага болох Project Camelot нь 2005 оны зургаадугаар сард Лондон хотын Ахмад Чөлөөт Барилгачдын уулзалтад биечлэн оролцсон мэдээлэгчийг \cite{4,6} ярилцсан юм. Уулзалтын үеэр хэлэлцсэн сэдвүүд нь цэргийн болон улс төрийн төлөвлөгөөнүүд бөгөөд ирэх \textbf{"геофизикийн үйл явдал"}, өөрөөр хэлбэл дэлхийн хэмжээний байгалийн гамшгийн хүрээнд төвлөрч байв.

\begin{figure}[b]
\begin{center}
\includegraphics[width=1\linewidth]{freemason.jpg}
\end{center}
   \caption{Британийн чөлөөт барилдлагууд өөрсдийн байгалийн төлөвт: чимээгүйхэн цөмийн бөмбөг хаяж, дэлхийг эзлэх талаар хуйвалдаж байна - Лондонгийн Earls Court-д, 1992 он \cite{5}.}
\label{fig:1}
\label{fig:onecol}
\end{figure}

\begin{figure*}[t]
\begin{center}
\includegraphics[width=1\textwidth]{british.jpg}
\end{center}
   \caption{1937 онд Британийн эзэнт гүрэн, Англо-саксон хүч чадлын сүр жавхлантай илэрхийлэл \cite{14}.}
   \label{fig:2}
\end{figure*}

Энэ мэдээлэгчийн хэлснээр, уг уулзалтад оролцсон 25-30 хүн \textit{"... бүгд Британичууд байсан бөгөөд тэдний зарим нь Их Британид олон нийтийн танил болсон, нэр хүндтэй хүмүүс байлаа... тэнд бага зэрэг язгууртны анги байсан, зарим нь ч нэлээд язгууртны гаралтай хүмүүс. Тэр уулзалтад би таньсан нэг хүн бол өндөр албан тушаалтай улс төрч байсан. Хоёр хүн бол цагдаа нараас байсан бөгөөд нэг нь цэргийн хүн байсан. Хоёул улсын хэмжээнд танигдсан бөгөөд аль аль нь одоогийн засгийн газарт зөвлөх гол хүмүүс юм — яг одоо"} \cite{4}. Мэдээлэгч хэлэхдээ, тэр энэ уулзалтанд оролцсон,\ \textit{"Зүгээр нэг тохиолдлоор! Би энэ уулзалтыг энгийн гурван сарын уулзалт гэж бодсон... Би энэ уулзалтад очсон, гэхдээ энэ нь миний хүлээж байсан уулзалт биш байсан. Намайг урьсан гэж боддог... ажлын байрны улмаас, мөн тэд намайг өөрсөдтэйгөө адил гэж итгэсэн учраас."} \cite{4}.

Уулзалтаар хэлэлцсэн үйл явдлын үндсэн цагийн хуваарь (2005 онд) дараах байдалтай байна:

\begin{flushleft}

\begin{enumerate}
    \item Ираныг эсвэл Хятадыг тактикийн цөмийн зэвсэг ашиглуулахад өдөөж, хязгаарлагдмал цөмийн солилцоонд хүргэн, дараа нь гал зогсоох тогтоох.
    \item Хятад руу биологийн зэвсэг тараах, 1970-аад оноос гол бай нь болсон гэж мэдээлэгдсэн.
    \item Айдас болон үймээнээс үүдэн хянах тоталитар цэргийн засгийн газруудыг гаргаж ирэх.
\end{enumerate}
\end{flushleft}

Гэвч хамгийн чухал зүйл нь эдгээр үйл явдлуудын дараа юу болохыг хүлээж байгаа явдал юм: \textit{"Бид энэ дайнд орж, тэгээд дараа нь... Дэлхий дээр геофизикийн нэгэн үйл явдал болох бөгөөд энэ нь хүн бүрт нөлөөлнө"} \cite{4}. Дотоод хүн хэлэхдээ энэ геофизикийн үйл явдлын үед, \textit{\textbf{"Дэлхийн царцдас ойролцоогоор 30 градус, 1700-2000 миль урагшаа шилжих бөгөөд энэ нь асар их өөрчлөлтийг үүсгэж, нөлөө нь маш удаан хугацаанд үргэлжлэх болно"}} \cite{4}.

Энэ бүх нууц төлөвлөгөөний шалтгаан нь мэдээж эрх мэдэл юм. Дотоод хүн тайлбарлахдаа, \textit{"Тухайн үед бид бүгд цөмийн болон биологийн дайныг даван гарсан байх болно. Хэрвээ энэ тохиолдвол дэлхийн хүн ам эрс буурна. Энэ геофизикийн үйл явдал болох үед үлдсэн хүмүүсийн тоо дахиад ойролцоогоор тэн хагас болж багасах байх. Үүнийг давж үлдсэн хүмүүс дэлхий болон үлдсэн хүн амыг дараачийн эринд удирдах хүмүүсийг тодорхойлно. Тэгэхээр бид катастрофын дараах эрин үед тухай ярьж байна. Хэн удирдах вэ? Хэн хяналтандаа байх вэ? Гол нь энэ л юм. Иймээс тэд эдгээр зүйлсийг тодорхой хугацааны дотор хүргэхээр үхэн хатан зүтгэж байгаа. [Замбараагүй байдал] болохоос өмнө зайлшгүй бүтэц бэлэн байх ёстой, ингэснээр ирээдүйд болох зүйлд амьд үлдээд, маргааш нь хоёр хөл дээрээ зогсож, эрх мэдлээ хадгалж, өмнөх эрх мэдлээ тогтоон барих боломжтой байх юм"} \cite{4}. Ярилцлагын үеэр энэ төлөвлөгөөний нэр болох "Англо-Саксоны Мисс" гэж нэрлэгддэг тухай мөн ярилцсан: \textit{[Ярилцагч]: "...Үүнийг Англо-Саксоны Мисс гэж нэрлэдэг шалтгаан нь үндсэндээ төлөвлөгөө нь Хятадуудыг устгаад, гамшгийн дараа, дахин сэргээгдэх үед Англо-Саксонууд шинэ дэлхийг дахин байгуулах, өвлөн эзэмших байр сууринд үлдэх, өөр ямар ч хүн байхгүй байх явдал уу. Зөв үү?" [Дотоод хүн]: "Зөв эсэхийг сайн мэдэхгүй ч, тантай санал нэг байна. Ядаж ХХ зуунаас хойш, бүр XIX, XVIII зуунд ч дэлхийн түүх голчлон Баруунаас болон дэлхийн умард бүсээс явуулж иржээ"} \cite{4}.
Regarding the exact timeframe of the expected geophysical event, the insider offers his best guess: \textit{"...мэдрэмж нь, маш зөн совингийн мэдрэмж л дээ, одоо тэд өөрсдийгөө цэгцлэх ёстой болсон... Миний бодлоор тэд хэзээ болохыг нь сайн мэдэж байгаа... \textbf{Миний энэ маш хүчтэй мэдрэмжээр, энэ миний насан дээр, гэхэд 20 жилийн дотор болох нь гарцаагүй}... бид одоо энэ үе рүү орчихсон байна, энэ геофизикийн үйл явдал болох гэж байгаа үе рүү, хамгийн сүүлд тохиолдсноос хойш 11,500 жил өнгөрсөн гэдгийг бодох юм бол, энэ нь ойролцоогоор 11,500 жил тутамд, циклийн дагуу болдог. Одоо дахин болох ёстой болчихсон... Тэд болох гэж байгааг ойлгож байгаа. Тэд болох нь тодорхой гэдгийг мэдэж байгаа... Дахиад, энэ бол тэд мэдэхгүй байна гэдэг нь байж боломгүй зүйл — хамгийн шилдэг ухаантнууд нь энэ тал дээр ажиллаж байгаа"} \cite{4}.

This is a powerful testimony for which we should be very grateful. In the interview, the author also discusses his belief that WWI and WWII were manufactured wars, and that the Anglo-Saxon Mission almost certainly dates back many, many generations. It has now been 15 years since the interview, which occurred in 2010. There are five years remaining until the insider's stated 20-year timeframe prediction for the geophysical event reaches its end.

\subsection{Катаклизмын тухай Друидын Нууцлаг Барууны Мэдлэг}

Давтагддаг катаклизмын талаар барууны мэдлэг маш сайн хадгалагдсан бөгөөд энэ нь зөвхөн Фриймейсонуудаар хязгаарлагдаагүй. Друидууд, хамгийн багадаа 2400 жилийн түүхтэй Кельтийн эртний соёлыг тун сайн баримтжуулж авсан \cite{7}, дэлхийн давтагддаг катаклизмын тухай мэдлэгийг дамжуулж ирсэн. Сүүлчийн Друид гэдгээрээ Бен МакБрэди нэрлэгддэг. "Сүүлчийн Друид" нэртэй, 1992 оны баримтат кинонд тэрээр Друидуудын мэдлэгийн тухай дараахи мэдээллийг хуваалцсан: \textit{"Уламжлалын дагуу би сүүлчийн гишүүн нь байж болох энэ бүлэглэл сүүлчийн их катаклизм, эсвэл дэлхийг хамарсан сүйрлийн дараа үүссэн юм. Одоо эдгээр их аймшигт цахилгаан шуурга, солирын сүүлэнд өртөх, эсвэл солирын борооны дунд орогнох үед соёл иргэншил бүрэн сүйдсэн байсан... Бүх мэдлэгийг энэхүү бүлгийн хүрээнд төвлөрүүлдэг байсан ч, тэд ялангуяа астрономид анхаарлаа хандуулж байжээ, учир нь маш олон ноцтой сүйрэл амссан учраас. Бүрэн хэмжээний астрономийн мэдлэгтэй байснаар ийм сүйрлүүд болж болзошгүй нөхцөл байдлыг урьдчилан таамаглах, өөрсдийгөө хамгаалах арга хэмжээ авах боломжтой гэж үздэг байсан. Ирландын агуу мегалит байгууламжуудыг ажиглавал 'ногоон оршуулгын' гэж нэрлэдэг тэр газрууд үнэндээ маш бүдүүн хүлэмж маягийн бөмбөгнөөс хамгаалах барилга байгууламжууд гэдгийг олж харна. Эдгээр нь ямар ч цүгийн түвшнээс өндөр цэг дээр байрладаг бөгөөд солирын борооноос ч бас хамгаалалт болдог"} \cite{8,9}.

% Мөн Фриймейсонууд өөрсдөө шууд Друидуудын гаралтай гэж үздэг \cite{10}.
\section{Өнөө үеийн өрнөдийн сүйрлийн бэлтгэлийн нотолгоо}

Үндсэндээ өрнөдийн эрх баригч хүчнүүд дэлхий нийтийн геофизикийн сүйрэл ойртоод байна гэж үзэж байгаа бол ийм үйл явдлаас өөрсдийгөө хамгаалахын тулд маш их бэлтгэл ажлууд хийгдэж байхыг бид хүлээх болно. Үнэхээр ч олон өрнөдийн орнуудад гүн суурьлагдсан нууц хонгилын баазуудын өргөн сүлжээ олон нийтэд ил байгаа нотолгоо бий. Ийм байгууламжууд нь цөмийн дайны үед оршин суугчдаа хамгаалж чадахаас гадна төрөл бүрийн байгалийн гамшгаас хамгаалах зориулалттай. Project Camelot\cite{4,6}-оос мэдээлэл өгсөн Их Британийн өндөр түвшний чөлөөт Mason-ийн мэдүүлгээр судлахад эдгээр нь зүгээр нэг боломжит хувилбарууд бус, харин урьдчилан боловсруулсан төлөвлөгөө юм шиг санагддаг. Мөн эдгээр хонгилын байгууламжуудыг барих, хүний нөөцөөр хангах, засвар үйлчилгээ хийхэд асар их мөнгө шаардагдах бөгөөд энэ нь АНУ-ын засгийн газарт 18 жилийн туршид алга болсон хэдэн арван их наяд ам.доллартай (дараагийн хэсэгт авч үзнэ) \cite{11,12,13} тохирч байна. Үхлийн хэмжээнд хүргэх эрсдэлтэй үйл явдалд бэлтгэх өөр жишээнүүдэд үрийн болон мэдлэгийн архивын төслүүд багтдаг.

\subsection{Америкийн газар доорх ба далай доорх баазууд}

Миний олсон газар доорх баазуудыг ил тод судалсан хамгийн өргөн судалгаа нь Ричард Саудерын хийсэн ажил бөгөөд тэрээр АНУ-ын бие даасан судлаач бөгөөд газар доорх баазын тухай хэд хэдэн ном хэвлүүлсэн байна \cite{22}. Саудерын ажилд засгийн газрын баримт бичиг, төлөвлөгөөг цуглуулж архивлах, түүхэн болон өнөөгийн мэдээ, технологиудыг нягтлан шалгах, эх сурвалжуудтай харилцах, дотоодын хүмүүсийн мэдүүлгийг цуглуулах багтдаг. Саудерын судалгааны дүнд АНУ болон түүний газар нутагт болон ойр орчимд орших гүнзгий газар доорх ба далай доорх баазуудын том сүлжээ (зураг \ref{fig:4}) байгаа нь илэрч, гүн нь хамгийн багадаа 3 мили хүрч болзошгүй бөгөөд газар доорх вакуум хоолой бүхий өндөр хурдны магнит хөвүүрийн галт тэрэгнүүдээр холбоотой байх магадлалтай. Эдгээр баазуудыг \textit{"өндөр санхүү, олон улсын, салбар хоорондын, мөнгө угаах далд тоглоом"}-оор санхүүжүүлдэг бөгөөд энэ нь Америкийн Нэгдсэн Улс нэртэй компанийг эзэмшдэг хүмүүсийн бүлэглэлийн бүрэн хяналтад байдаг \cite{22}. Кэтрин Остин Фиттс (түүний ажил дараагийн хэсэгт дурдагдана) болон түүний хамтрагч нэгэн судлаачийн хамт эдгээр баазуудын тоо хэмжээг тодорхойлох судалгаа явуулахад Америкт 170 газар доорх, далай доорх бааз байгааг тооцоолжээ \cite{16,20}.

\begin{figure}[b]
\begin{center}
% \fbox{\rule{0pt}{2in} \rule{0.9\linewidth}{0pt}}
   \includegraphics[width=1\linewidth]{penta.jpg}
\end{center}
   \caption{Цагаан ордон ба Пентагоны доор үнэндээ юу байдаг вэ? Магадгүй, гүн газар доорхи хонгилын сүлжээ байдаг бололтой (Зураг: \cite{31}).}
\label{fig:3}
\label{fig:onecol}
\end{figure}
\begin{figure*}[t]
\begin{center}
% \fbox{\rule{0pt}{2in} \rule{.9\linewidth}{0pt}}
\includegraphics[width=0.9\textwidth]{basescrop.png}
\end{center}
   \caption{Саудерын судалгаагаар газар доорх болон далайн доорх баазууд, мөн далай доорх шумбагч онгоцны хонгилоор эх газарт хүрэх замууд хамгийн магадлалтай байгаа нарийн байршлуудыг харуулсан газрын зураг. Саудер \textit{"итгэлтэйгээр [эдгээрээс] \textbf{илүү олон} байгууламж бий"} гэж хэлсэн байна \cite{22}.}
   \label{fig:4}
\end{figure*}

Энд Саудерын эх сурвалжуудаас эдгээр баазуудын заримын цар хүрээг тодорхойлсон гэрчийн мэдүүлгийн хэсгүүд байна:
\begin{flushleft}
\begin{enumerate}
    \item Кэмп Дэвид, Мэрилэнд: \textit{"Эх сурвалж минь надад Кэмп Дэвидын газрын доорх хэсгүүд маш өргөн цар хүрээтэй, нарийн төвөгтэй бөгөөд маш олон милийн нууц хонгилуудтай учраас яг энэ байгууламжийн бүтэн зургийг хэн нэг хүн толгойдоо бүрэн хадгалж чадах эсэх нь эргэлзээтэй гэж хэлсэн"} \cite{22}.
    \item Цагаан ордон, Вашингтон ДС: \textit{"Миний найзын нэгийг 1960-аад онд Линдон Б. Жонсон ерөнхийлөгчийн үед энэ байгууламж руу авч орсон. Тэрээр Цагаан ордонд нэгэн цахилгаан шатанд орж доош шууд буусан юм. Тэр цахилгаан шат 17 давхар доош явсан гэдэгтээ итгэдэг. Газрын доор хаалга онгойход түүнийг алсад тасарсан мэт санагдах, урт хонгилоор дагуулан явуулсан. Тэр хонгилоос өөр хаалга, хонгилууд салж байв"} \cite{22}. Зураг~\ref{fig:3}-д дүрсэлсэн.
    \item Форт Мид, Мэрилэнд - 1970-аад онд санамсаргүйгээр "подвал"-ууд руу орсон нэгэн эх сурвалжаас: \textit{"Би хаалгыг онгойлгоход доош буух шат харагдсан. Би очоод хашлага дагуу доош харахад хэдэн давхар доош буусан бэ гэдгийг тоолоогүй ч, ойролцоогоор 15-20 давхар юм шиг санагдсан... Би нэг давхарыг уруудаад гарсан хаалга руу орж толгойгоо цухуйлган зүүн, баруун тийш харвал хоёр чиглэлд бараа сураггүй урсах хонгил харагдсан. Энэ нь яг барилга болон гадна зогсоолын хамрах хүрээнээс хамаагүй илүү байсан. Эсрэг талын ханан дагуу 30-40 фут зайтай хаалганууд байсан... Би өөр давхар үзэхээр ахин нэг давхрыг уруудахад мөн л ижил зохион байгуулалттай байсан... Би ахиад нэг давхар уруудаж орвол эхний 2 давхартай ижил зураглалтай байсан"} \cite{22}.
\end{enumerate}
\end{flushleft}

\begin{figure}[t]
\begin{center}
% \fbox{\rule{0pt}{2in} \rule{0.9\linewidth}{0pt}}
   \includegraphics[width=1\linewidth]{undersea.jpg}
\end{center}
   \caption{Усан доорх баазын зураглал, Вальтер Кёршнерийн бүтээл. Тэрээр 1960-аад онд АНУ-ын Тэнгисийн цэргийн Хятадын Нуур, Калифорниа Зэвсгийн төвийн Тэнгисийн цэргийн Rock-Site усан доорх баазын багт зурагчаар ажиллаж байсан. Саудерын нэгэн эх сурвалж Хятадын Нуурт газрын гүн дэх нэг миль гүнд байрлах газар доорх бааз бий гэдгийг илчилсэн байдаг \cite{22,23}.}
\label{fig:5}
\label{fig:onecol}
\end{figure}

Саудер мөн 2,000 милийн хурдтай явдаг газар доорх соронзон левитацийн галт тэрэгнүүд, далайн ёроолд баригдсан баазууд (Зураг \ref{fig:5}), мөн хуурай газар руу чиглэсэн усан дорхи шумбагч онгоцны хонгилуудын талаар мэдүүлэг авсан. Мексикийн булан дахь усан дорхи баазын нэгэн мэдүүлгийн тухайд Саудер хэлэхдээ, \textit{"Усан дор болон газар доорх баазууд ном хэвлэгдсэний хагас жилийн дараа надад нэг хүн холбогдож, өөрөө ер бусын усан доорх төслийн талаар мэдлэгтэй гэж мэдэгдсэн... тэрээр уг төсөл Мексикийн булангийн ёроолд, Парсонс компани гүйцэтгэгчээр ажилласан гэдгийг тодорхойлсон. Цаашлаад, Парсонс компани далайн ёроолоос 2,800 фут гүнд ажиллах зориулалттай тусгай тоног төхөөрөмж худалдан авсан гэж хэлсэн... Тэр тоног төхөөрөмж нь суурилагдах газарт амьд хүмүүс байгаа гэсэн санааг давхар илтгэх онцгой зүйл байсан"} \cite{22}.
\begin{figure}[t]
\begin{center}
% \fbox{\rule{0pt}{2in} \rule{0.9\linewidth}{0pt}}
   \includegraphics[width=1\linewidth]{sub.jpg}
\end{center}
   \caption{Усан доорх шумбагч онгоцны хонгилын зургийг Walter Koerschner зурсан \cite{22,23}.}
\label{fig:6}
\label{fig:onecol}
\end{figure}
\begin{figure}[t]
\begin{center}
% \fbox{\rule{0pt}{2in} \rule{0.9\linewidth}{0pt}}
   \includegraphics[width=1\linewidth]{iran.jpeg}
\end{center}
   \caption{Ираны албан ёсны видеоноос авсан хүрээ, тэдний газар доорх "пуужингийн хот"-ыг харуулж байна \cite{39,40}.}
\label{fig:12}
\label{fig:onecol}
\end{figure}
Хэрвээ үнэхээр бидний хөл доор маш олон нууц тив дамнасан сүлжээ, гадаргуунаас милийн гүнд ухагдсан 170+ газар доорх болон далайн ёроолын бааз, тэдгээрийг холбосон гиперсоник вакуум хоолойн магнит левитэй галт тэрэг, бидний хөдөлмөрийн үр шимээр санхүүжүүлэгдсэн байгаа бол өнөөдөр хүн төрөлхтний ихэнх нь гүн бөгөөд жаргалтай мунхаг байдалд байх бөгөөд зөвхөн өөрсдийн доор юу байгааг мэдэхгүй төдийгүй, ойрын ирээдүйд өөрсдийг нь юу хүлээж байгааг ч мэдэлгүй, улстөрч "арчилж дэмжигчдийн" нь хоосон, уялдаа холбоотой мэдэгдлүүдэд автсаар байна.

Нэмэлт тайлбар - Ойрхи Дорнодод өрнөж буй зөрчилдөөний үед томоохон газар доорх хонгилын сүлжээнүүдийн оршин байгааг эргэлзээгүйгээр илчилсэн (Газар дундуур ухсан ХАМАС-ын хонгилууд \cite{38}, мөн Ираны газрын доорх "пуужингийн хот" (Зураг \ref{fig:12}) \cite{39,40}). Эдгээр нь ийм бүтцийг барих боломж болон бодитойгоор оршин байгааг эргэлзээгүй болгож байна. Мөн бусад илүү сайн санхүүжсэн улсууд ижил үед ямар байгууламжууд барьсан байж болох талаар бидэнд бодол төрүүлэх ёстой.

\subsection{Нэмэлт нөөц хонгил, гамшгийн бэлэн байдлын нотолгоо}

\begin{figure}[t]
\begin{center}
% \fbox{\rule{0pt}{2in} \rule{0.9\linewidth}{0pt}}
\includegraphics[width=1\linewidth]{tyrol.jpg}
\end{center}
\caption{Өмнөд Тирол дахь бункерууд, Швейцарь. Европын Альпын нуруу дамжсан Швейцарь нь уулын бункеруудаа ухаалгаар нууж чаддагаараа алдартай \cite{32}.}
\label{fig:7}
\label{fig:onecol}
\end{figure}

\begin{figure}[t]
\begin{center}
% \fbox{\rule{0pt}{2in} \rule{0.9\linewidth}{0pt}}
   \includegraphics[width=1\linewidth]{svalbard.jpg}
\end{center}
   \caption{Норвегид байрлах Свалбардын Дэлхийн Үрийн Сан нь нэг саяас илүү үр хадгалж байна \cite{24}. Үүнийг ашиглах ямар гамшиг тохиолдохыг гайхахгүй байхын аргагүй.}
\label{fig:8}
\label{fig:onecol}
\end{figure}

АНУ-ын газар доорх хааны баазуудаас гадна дэлхий даяар гамшгийн бэлтгэлтэй холбоотой олон нэмэлт ул мөр бий. Норвеги, Швейцар, Швед, Финлянд нь сайн жишээ юм:

\begin{flushleft}
\begin{enumerate}
    \item Project Camelot нь нэгэн Норвеги улс төрчийн холбогдох мэдүүлгийг хуваалцсан \cite{25,26}, хэн болохыг нь үнэлж, хувийн нууцад хадгалсан байна. Тэр Норвегид 18 өргөн хүрээтэй газрын доорх бааз байгааг, мөн Норвеги (Израиль болон "өөр олон улс орнууд"-ын хамт) эдгээр баазуудыг ямар нэгэн байгалийн гамшигт бэлтгэхээр барьж байгаа гэж мэдэгдсэн. Richard Sauder мөн Норвегид нэгэн ухсан уулын дотор баригдсан асар том газрын доорх бааз дотор орж үзсэн хүнээс мэдүүлэг авсан \cite{22}.
    \item Швейцарь улс Альпийн өндөрлөгт олон цөмийн хоргодох байрыг барьсан гэдгээрээ алдартай (Зураг \ref{fig:7}). Эдгээр нь гайхалтай 370,000 гаруй бааз бөгөөд энэ нь тус улсын бүх оршин суугчийг хоргодох байраар хангахад хангалттай юм \cite{27}.
    \item Швед болон Финлянд нь томоохон хот бүрийн оршин суугчдыг хоргодох байртай болгоход хангалттай хэмжээний хоргодох байртай \cite{27}. 
\end{enumerate}
\end{flushleft}

Силикон хөндийн бизнесийн тэрбумтнууд ч бас үүнд мэдлэгтэй бололтой. Үүний талаар, \textit{"LinkedIn-ний хамтран бүтээгч бөгөөд нэрт хөрөнгө оруулагч Рейд Хоффман, энэ жил The New Yorker сэтгүүлд хэлэхдээ Силикон хөндийн тэрбумтнуудын 50\%-иас илүү нь ямар нэг "апокалипсисын даатгал", жишээлбэл газрын доорх хоргодох байр худалдаж авсан хэмээн тооцоолж байгаагаа дурдсан... Forbes-ийн зохиолч Жим Добсон-ы хэлснээр олон тэрбумтан хувийн онгоцтой бөгөөд "ямар ч үед явахад бэлэн" байдаг. Тэд мөн мотоцикл, зэвсэг, цахилгаан үүсгүүр ч эзэмшдэг"} \cite{28}.

Мөн хүн төрөлхтөний гол нөөцийг мөхлийн гамшгийн үед хадгалахаар зорилготой Дэлхийн мэдлэгийн санг Arch Mission Foundation удирдаж байгаа \cite{29}, мөн Svalbard Дэлхийн үрийн сан \cite{30} зэрэг томоохон архивын төслүүд бий.
\begin{figure*}[t]
\begin{center}
% \fbox{\rule{0pt}{2in} \rule{.9\linewidth}{0pt}}
\includegraphics[width=0.9\textwidth]{govcrop2.png}
\end{center}
   \caption{АНУ-ын засгийн газрын орлого, зарлага, болон нууц газар доорх баазын зарлага 1998-2023 оны хооронд \cite{19}.}
   \label{fig:9}
\end{figure*}
\section{Асар том газар доорх баазуудыг ардчилсан санхүүжүүлэх механизм}

Тэгвэл тив дамнасан 170+ газар доорх болон далайн доорх асар том баазын сүлжээг хэрхэн санхүүжүүлдэг бөгөөд өрөнд байгаа боолуудыг хэрхэн харанхуйд байлгана вэ? Эдгээр төслүүдэд ямар хэмжээний мөнгө ордог, мөн хаанаас ирдгийг бидэнд харуулж чадах цаасны мөр байдаг. 2017 онд Америкийн хөрөнгө оруулалтын банкир, Бушийн засаг захиргааны хуучин албан тушаалтан Кэтрин Остин Фитс болон Мичиганы Улсын Их Сургуулийн эдийн засагч Марк Скидмор нар АНУ-ын засгийн газарт 1998-2015 санхүүгийн онуудад зөвшөөрөлгүйгээр 21 их наяд ам.долларын зарцуулалт хийсэн болохыг олж тогтоожээ \cite{11,12,13}.

Тэдний илтгэлд дурдсанаар, \textit{"2016 оны аравдугаар сарын 7-нд Ройтерс агентлагаас Скэт Палтроу (2016) нийтлэл нийтэлсэн бөгөөд түүнд 2015 санхүүгийн онд Армийн зүгээс \$6.5 их наяд ам.долларын нотолгоогүй нягтлан бодох бүртгэлийн тохируулга хийж, \"өрийн дэвтрүүдээ тэнцвэртэй харагдуулах хуурмаг сэтгэгдэл төрүүлсэн” гэжээ. Тухайн онд Армийн ерөнхий сангийн төсөв нь \$122 тэрбум байсан нь гайхалтай нээлт байв... Батлан хамгаалах яам олон жилийн өмнө 2001 оны есдүгээр сарын 10-нд Төрийн нарийн бичгийн дарга Доналд Рамсфелд Конгрессын сонсголын үеэр (C-SPAN, 2014) Батлан хамгаалах яам \$2.3 их наяд ам.долларын гүйлгээний баримтыг алдсан гэж хэлсэнээр нягтлан бодох бүртгэлийн асуудлаараа томоохон хэвлэлийн гарчиг болсон... Энэ мэдэгдэл тэр өдөр мэдээний гарчиг болсон ч маргааш нь 9/11-ийн гамшиг дэлхийн анхаарлыг татсанаар мартагдсан... Профессор Марк Скидмор Армийн нотлогдоогүй \$6.5 их наяд ам.долларын гүйлгээний талаарх мэдээг сонсоод хатагтай Фитстэй холбогдож, 2017 оны хавартаа хамтран ажиллаж, HUD болон Батлан хамгаалах яаманд баталгаажуулагдаагүй, их хэмжээний бусад ижил төстэй засгийн газрын тайланд дүн шинжилгээ хийхээр тохиролцсон. Дараагийн зургаан сарын турш Скидмор, Фитс болон цөөн оюутны баг хамтран албан ёсны засгийн газрын баримтыг цуглуулж, 1998-2016 оны хугацаанд нийт \$21 их наяд ам.долларын баримтжуулагдаагүй гүйлгээг илрүүлсэн"} \cite{12}.

1998-2015 оны 18 жилийн хугацаанд нийтэд зарласан АНУ-ын засгийн газрын орлого 40.8 их наяд ам.доллар байсан \cite{15}, үүнээс харвал АНУ-ын засгийн газрын орлогын талаас илүү хэсэг нь газар доорх баазуудад нууцаар зарцуулагдаж, олон нийтэд зарласан зарлагын дээрээс нэмэгдэн гарсан болохыг илтгэнэ. Мөн онцолмоор зүйл нь, энэ нууц зарлагад олон жилийн турш үргэлжилсэн төсвийн алдагдлын дээгүүр давхар явагдаж ирсэн бөгөөд өнөөдрийг хүртэл үргэлжилсээр байгаа, мөн 1998 оноос өмнө ч байсан байж болох учир нийт дүн нь 21 их наядаас ч хавьгүй их байна. Тухайн түвшинг 2016-2023 оны хугацаанд харьцуулж хэрэглэвэл 1998 оноос хойш нийт 36.6 их наяд ам.доллар зарцуулсан гэсэн үг.

2021 онд Марк Скидмор энэ судалгаагаа шинэчилж, Bloomberg 2017-19 санхүүгийн оны хугацаанд Пентагон бүртгэлдээ итгэмээргүй 94.7 их наяд ам.долларын нягтлан бодох бүртгэлийн тохируулга хийсэн тухай зарласантай холбогдуулан нийтэлсэн байна \cite{17,18}. Хэрвээ 1913 онд Холбооны Нөөцийн тогтолцоо байгуулагдсанаас хойш зуу гаруй жилийн турш төв банкны тогтолцоогоор АНУ-ын долларыг хуурамчаар бүтээгээд байгааг харгалзан үзвэл, нийтийн долларын тооцоо бол цэвэр хоёрдмол утгатай, утгагүй зүйл гэдэг тодорхой байна. АНУ-ын мөнгө болон засгийн газар нь зүгээр л нөөцийн хуваарилалтын тогтолцоо болж хувирсан бөгөөд үүний хаад эзэд дуртай мөнгөө чимээгүй хусаж (эсвэл бүр цутгаж) чаддаг байна.
\section{Йовын үр сад: Барууны сүүдрийн хаадын үнэн дүр төрх}

Тэгэхээр, хэн үнэндээ энэ тоглолтыг удирдаж байна вэ? Бид үүнийг яг мэдэх боломжгүй, учир нь капиталын барууны хаадууд өөрсдийгөө сүүдэрт нууж байдаг. Янз бүрийн онол таамаг, олны танил хүмүүсээс эхлээд харь гаригийн оршихуйнууд хүртэл байдаг ч, миний хамгийн сайн хариултыг "Amallulla" гэх нэрээр алдартай нэгэн нэргүй блогчин хүний амьдралын бүтээлээс олж болно. Түүний бүтээл нь 20 гаруй зохиолч, 50 "орлуулшгүй" баримтыг хамарсан өргөн хүрээний судалгаа бөгөөд эртний ба орчин цагийн түүх, нууцлаг бэлгэ тэмдэг, Барууны улс төрийн сэдвүүдийг хамарсан юм \cite{33,34}. Би түүний ажлыг удахгүй болох геофизикийн сүйрлийн талаарх "зөгнөлт" гэж итгэлтэйгээр хэлж чадна - энэ нь миний судалгаанаас \textit{мэдэгдэхүйц} өргөн хүрээтэй.

Amallulla нь Барууны гурван улс төрийн бүлгийг тодорхойлж, эдгээрийг нийлүүлэн "Йовын үр сад" гэж нэрлэсэн бөгөөд тэд дэлхийн "эцсийн цагийн" - давтагддаг катаклизмийн тухай мэдлэгтэй гэж үзжээ. Тэрээр эдгээр гурван бүлэг хамтдаа өнөөгийн Барууны улсуудыг удирддаг гэж итгэсэн ч, гарал үүсэл, түүхэн үнэн дүр төрх, өнгөрсөн маргаан зөрөлдөөн, үнэт зүйл, үйлдлийн ялгаа зэргийг үндэслэн гурван өөр бүлэгт хуваасан байна.

Гурван бүлгийг ерөнхийд нь дараах байдлаар ангилж болно:

\begin{flushleft}
\begin{enumerate}
\item \textbf{Банкчид}: Эртний Ромын элитүүд, хожим нь Америк дахь Тамплийн баатрууд болон Хойд нутгийн Чөлөөт барилдлагад хувирсан.
\item \textbf{Бодол санаатнууд}: Розенкройцчууд болон Өмнөд Америкийн чөлөөт барилдлагууд.
\item \textbf{Езуитүүд ба Хар Пап}: Ромын Католик сүм дэх Зевсийн үр удамын бүлэглэл.
\end{enumerate}
\end{flushleft}

Өнөө үед эдгээр гурван бүлэглэл нийлж Европийн Иллюминатууд, Чөлөөт барилдлагууд, мөн Тагнуулын төв газар (CIA)-г бүрдүүлдэг. Амаллуллагийн бичсэнээр, \textit{"Яг одоо, эцсийн цаг үед, Зевсийн үр удам нь АНУ-ын одоогийн Ерөнхийлөгч хүртэл мэдэх ёсгүй, тусгай зөвшөөрлийн ард сайнаар нуув. Өөрөөр хэлбэл, тэд өөрсдийгөө олон нийтийн хяналтаас төгс нуудаг болсон. \textbf{Зевсийн үр удам зөвхөн Америкийн Нэгдсэн Улсын цэргийн болон засгийн эрхийг төдийгүй, цаасан мөнгөний хүч, томоохон корпорациуд, өөрсдийн бүтээсэн Бүгд Найрамдах засаглалаар (улс төрчид амархан авлигад өртдөг, түүнд орж удирдах боломжтой гэдгийг мэдэж байсан учраас) бүх Баруун ертөнцийг хянадаг}"} \cite{33,34}.

\begin{figure}[t]
\begin{center}
% \fbox{\rule{0pt}{2in} \rule{0.9\linewidth}{0pt}}
   \includegraphics[width=1\linewidth]{illuminati.jpg}
\end{center}
   \caption{Йовегийн үр удам гэж яг хэн бэ? (Зураг: \cite{35})}
\label{fig:10}
\label{fig:onecol}
\end{figure}

\begin{figure}[t]
\begin{center}
% \fbox{\rule{0pt}{2in} \rule{0.9\linewidth}{0pt}}
   \includegraphics[width=1\linewidth]{pike.jpg}
\end{center}
   \caption{Алдарт Пайк Пийкийн батолит, улаанаар тодруулсан бөгөөд АНУ-ын өрнөд нутгийн байгаль орчин хамт байна \cite{36}. Америкийн Нэгдсэн Улс үнэхээр энэ газрыг хянахын тулд үүссэн байж болох уу?}
\label{fig:11}
\label{fig:onecol}
\end{figure}

Амаллуллагийн хэлснээр, эдгээр хүмүүс шашинг басамжилж, дэлхийн томоохон шашнуудын ариун судруудыг өөрсдийн ашиг сонирхлын үүднээс өөрчилдөг бөгөөд бэлгэдлийг ихээр ашигладаг. Мөн тэд дайснууддаа өршөөлгүй ханддаг: \textit{"\textbf{2600 гаруй жилийн хугацаанд тэд дэлхийн төгсгөлийн тухай онцгой мэдлэгтэй байсан хэнийг ч системтэйгээр устгаж ирсэн. Энэ нь зөвхөн друидууд, еврей каббалистууд, эртний египетчүүд, арабууд, энэтхэгийн мэргэдийг хэлээд зогсохгүй, Өмнөд Америкийн сунасан гавалтангууд болон Төв Америкийн Майя шүтээний лам нарыг ч хамруулж хэлж байгаа юм. Тэд Хойд Америкт цэцэглэн хөгжиж байсан хүн амыг ч мөн устган үүнийг Дэлхийн төгсгөлийн орон болгон хадгалсан нь маргаангүй нотолгоотой. Америкийн “Индианууд”-ыг устгасан нь зүгээр л үлдэгдлийг цэвэрлэх ажиллагаа байсан.}"} \cite{33,34}.
Амаллулла мөн "Америкийн Нэгдсэн Улс" төслийг бүхэлд нь "Пайкс Пик Батолит"-ийг хянах зорилгоор хэрэгжүүлсэн гэж үзсэн бөгөөд энэхүү боржин уулын нуруу нь Газрын ховор геофизикийн гамшгаас хамгаалах маш сайн байрлалтай юм (Зураг \ref{fig:11}). Амаллуллагийн хэлснээр, \textit{"АНУ-ын Иргэний дайн болсны өмнө, үеэр болон хойно санхүүчид ба сэтгэгчид Америкийн Нэгдсэн Улсыг хянах гэж бус, харин дэлхийн хамгийн өвөрмөц боржин батолит болох Пайкс Пик батолитийг хянахын төлөө тэмцсэн... Ийм өндөрт, далайн эргээс ийм хол байрладаг өөр ямар ч боржин батолит дэлхий дээр байхгүй. Энэ бол Газар дэлхийн цахилгаан хөдөлгөөний гажилтад тэсвэртэй үлдэхэд хамгийн тохиромжтой байрлал"} \cite{33,34}. Амаллуллагийн судалгаагаар өнөөдөр энэ бүс нутгийн доор болон эргэн тойронд өргөн хэмжээний газар доорх хонгилын систем баригдсан байна \cite{36}.

\section{Дүгнэлт}

Энэ өгүүлэлд би Өрнөдийн элитүүд хэдэн мянган жилийн турш дэлхийд давтагддаг сүйрлийн тухай мэдлэгээ болгоомжтой хадгалж ирсэн, тун удахгүй дахин тохиолдох гэж итгэдэг, энэ мэт үйл явдалд бэлдэж газар доор өргөн далайцтай хоргодох байгууламж барьсан, мөн ийм нөхцөлийг улс төрийн болон цэргийн хувьд ашиглан дэлхий ертөнцийн хяналтыг авахаар төлөвлөж буйг гэрчлэх янз бүрийн мэдүүлгийг дэлгэрэнгүй тайлбарласан. Энэ бүхнийг хэрхэн санхүүжүүлсэн тухай Америк дахь баримтуудыг мөн дурдсан бөгөөд, бүх зүйлийг удирдаж буй яг аль овгийн цусны удам болох талаар хамгийн бодитой онолыг иш татсан. Илүү ихийг мэдэхийг хүсэж байгаа хүмүүст, би дурддаггүй үлдээсэн олон нэмэлт мэдээлэл нь эшлэлүүдийг дэлгэн үзэхэд олдох боломжтой.

Ойртож буй геофизикийн үйл явдлын хамгийн хүчтэй хэмжигдэхүйц нотолгоо нь дэлхийн геомагнитын орон зай хурдацтай өөрчлөгдөж байгаа явдал юм. Энэ нь зөвхөн соронзон умард туйл түргэн шилжиж байгаа (Зураг \ref{fig:13}) болон Өмнөд Атлантын геомагнетик аномали өсөж байгаа төдийгүй, сүүлийн 400 жилийн турш ерөнхийдөө геомагнитын орон зайн суларч, гажиж буйг хэмжихэд илэрхий байгаагаар нотлогдоно \cite{3}. Ийм шинжлэх ухааны өгөгдлийг миний анхны хоёр ECDO өгүүлэлд нарийвчлан хэлэлцсэн бөгөөд тэдгээрийг миний вебсайтаас унших боломжтой \cite{3}.

\begin{figure}[t]
\begin{center}
% \fbox{\rule{0pt}{2in} \rule{0.9\linewidth}{0pt}}
   \includegraphics[width=1\linewidth]{npw.jpg}
\end{center}
   \caption{1590 оноос 2025 он хүртэлх геомагнитын хойд туйлын байрлалыг 5 жилийн давтамжтайгаар харуулсан байна \cite{41}. Түүний хөдөлгөөн 1975 оноос эрчимтэй хурдасжээ.}
\label{fig:13}
\label{fig:onecol}
\end{figure}

Эцэст нь би танд сургагч Амаллуулагийн энэхүү эшлэлтийг үлдээмээр байна. Тэрээр \textit{"\textbf{бүх юм ганц зүйл}"} гэж ингэж тайлбарласан: \textit{"Энд би таны төсөөллийг туйлын хил хүртэл нь хүргэхээс өөр аргагүйд хүрлээ. Та одоо амьдарч буй, бага наснаасаа мэддэг байсан хорвоог мартан орхих хэрэгтэй. Ардаа орхи. Энэ бол Matrix кинонд дүрслэгддэг шиг, хүмүүсийг сүүлийн мөч хүртэл унтуулж байлгахыг зорьсон, бүрэн зохиомол бодит байдал юм. Заримдаа би киноны зохиол бичиж байгаа юм болов уу гэж хүсдэг. Гэвч энэ вэбсайтаар танд хуваалцаж буй зүйл бол бодит үнэн. “Бүх юм ганц зүйл” гэдгийг ойлгоход хагас арван жил зарцуулсан бөгөөд үүнийг би “Сүйрлийн Нэгдмэл Үзэл”-ийн уриа болгон хурдан сонгосон. Үүнийг тайлбарлахад тун хэцүү ойлголт. Түр зуур Matrix киноны жишээгээр бодъё. Энэ бол маш сайн зүйрлэл. Харин надад хамгийн хэцүү нь хэлэх гэж буй маань дэгсдүүлэг биш гэдгийг ойлгуулах явдал. Яг одоогоор Matrix-тай зүйрлэх нь хэлэх гэж буй бодит байдалд хамгийн ойр сонголт байна. \textbf{Таны амьдрал дахь бүхий л зүйлс, бичигдсэн түүх, түгээмэл, зөвшөөрсөн шинжлэх ухаан болон академи, улс төр, шашин гээд бүгд нэг юман дээр л, газар дэлхийн хөрс шилжилт эсвэл тэнхлэгийн хазайлттай холбоотой юм.} Та үүнийг одоо харахгүй байна. Мөн энэ бодит байдлаас гэнэт сэрнэ гэж байхгүй. Үүнд цаг хугацаа шаардагдана. Гэхдээ замын төгсгөлд та бүхий л амьдралаа Matrix гэх компьютерийн симуляцлагдсан бодит байдалд амьдарч байсныгаа ойлгох болно гэдгийг би амлаж байна"} \cite{33,34}.
Амжилт хүсэж байна.

\section{Талархлын үгс}

Нийтийн хүртээл болгохын тулд мэдлэгээ хуваалцсан бүх хүмүүст баярлалаа. Таньгүйгээр энэ ажил боломжгүй байх байсан ба хүн төрөлхтөн харанхуйд үлдэх байлаа. Таны сонголт үүрд мөнхөд цэцэглэнэ. Бид бүгдийг таньд өртэй бөгөөд би хязгааргүй талархаж байна.

\clearpage
\twocolumn

{\small
\bibliographystyle{ieee}
\bibliography{egbib}
}

\end{document}
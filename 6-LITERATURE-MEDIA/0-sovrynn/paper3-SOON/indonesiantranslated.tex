\documentclass[10pt,twocolumn,letterpaper]{article}

% Barang-barang saya sendiri
\usepackage{booktabs}
% \usepackage{caption}
% \captionsetup[table]{skip=8pt}   % Hanya mempengaruhi tabel
\usepackage{stfloats}  % Tambahkan ini ke preamble
\usepackage{float}

\usepackage{cvpr}
\usepackage{times}
\usepackage{epsfig}
\usepackage{graphicx}
\usepackage{amsmath}
\usepackage{amssymb}

% Sertakan paket lain di sini, sebelum hyperref.

% Jika Anda mengomentari hyperref lalu menghapus komentarnya, Anda harus menghapus
% egpaper.aux sebelum menjalankan latex lagi.  (Atau cukup tekan 'q' pada latex pertama

% run, let it finish, and you should be clear).
\usepackage[breaklinks=true,bookmarks=false]{hyperref}

\cvprfinalcopy % *** Uncomment this line for the final submission

\def\cvprPaperID{****} % *** Enter the CVPR Paper ID here
\def\httilde{\mbox{\tt\raisebox{-.5ex}{\symbol{126}}}}

\renewcommand{\tablename}{Tabel}
\renewcommand{\figurename}{Gambar}   % or whatever you like instead of "Hình"
\renewcommand{\refname}{Referensi}

\makeatletter
\def\abstract{%
  \centerline{\large\bf Abstrak}% <-- your new label
  \vspace*{12pt}%
  \it%
}

% Pages are numbered in submission mode, and unnumbered in camera-ready
%\ifcvprfinal\pagestyle{empty}\fi
\setcounter{page}{1}
\begin{document}

%%%%%%%%% TITLE
\title{Makalah ECDO 3: Bukti Persiapan Penguasa Barat Saat Ini terhadap Bencana Geofisika Besar yang Akan Datang}

\author{Junho\\
Diterbitkan Juni 2025\\
Situs web (Unduh makalah di sini): \href{https://sovrynn.github.io}{sovrynn.github.io}\\
Repositori Riset ECDO: \href{https://github.com/sovrynn/ecdo}{github.com/sovrynn/ecdo}\\
{\tt\small junhobtc@proton.me}
% Untuk makalah yang penulisnya semua dari institusi yang sama,
% abaikan baris-baris berikut hingga tanda ``}'' penutup.
% Penulis dan alamat tambahan dapat ditambahkan dengan ``\and'',
% seperti penulis kedua.
% Untuk menghemat ruang, gunakan alamat email atau halaman rumah, bukan keduanya
% \and
% Author2\\
% Institution2\\
% Baris pertama alamat institution2\\
% {\tt\small secondauthor@i2.org}
}

\maketitle
%\thispagestyle{empty}

%%%%%%%%% ABSTRACT
\begin{abstract}
Pada Mei 2024, seorang penulis daring dengan nama samaran “The Ethical Skeptic” \cite{0} membagikan sebuah teori terobosan yang disebut Exothermic Core-Mantle Decoupling Dzhanibekov Oscillation (ECDO) \cite{1}. Teori ini mengemukakan bahwa Bumi pernah mengalami pergeseran katastrofik mendadak pada sumbu rotasinya yang memicu banjir besar di seluruh dunia saat lautan meluap ke kontinen akibat inersia rotasi. Selain itu, teori ini menyajikan penjelasan mengenai proses geofisika beserta data yang mengindikasikan bahwa bencana serupa bisa saja terjadi dalam waktu dekat. Meskipun ramalan banjir besar dan kiamat seperti ini bukan hal baru, namun teori ECDO sangat menarik karena pendekatannya yang ilmiah, modern, multidisipliner, dan berbasis data.


Makalah ini merupakan hasil karya ketiga saya \cite{2,3} terkait topik ini yang akan berfokus pada aspek politik masa kini:
\begin{flushleft}
\begin{enumerate}
    \item Kesaksian pelapor bahwa kekuatan Barat mempercayai adanya bencana geofisika yang akan segera terjadi serta berencana mengambil keuntungan politik dan militer dari peristiwa tersebut.
    \item Bukti adanya pangkalan bawah tanah dan bawah laut di wilayah Barat yang luas untuk mengantisipasi peristiwa tersebut.
    \item Bukti bahwa sejumlah besar mata uang Barat telah dialirkan untuk mendanai pangkalan-pangkalan ini.
\end{enumerate}
\end{flushleft}

Makalah ini mendokumentasikan persiapan ekstensif yang dilakukan oleh negara Barat berkuasa dalam menghadapi bencana geofisika yang mereka yakini akan segera terjadi.
\end{abstract}

%%%%%%%%% BODY TEXT
\section{Freemasonry dan "Misi Anglo-Saxon"}

Pada Januari 2010, Project Camelot, sebuah organisasi media dan jurnalisme alternatif yang mengumpulkan kesaksian pelapor, mewawancarai \cite{4,6} orang dalam yang hadir dalam sebuah pertemuan Mason Senior di Kota London pada Juni 2005. Topik yang dibahas dalam pertemuan tersebut adalah rencana militer dan politik yang berpusat pada latar belakang \textbf{"peristiwa geofisika"} yang akan datang, yaitu bencana alam global.

\begin{figure}[b]
\begin{center}
% \fbox{\rule{0pt}{2in} \rule{0.9\linewidth}{0pt}}

\includegraphics[width=1\linewidth]{freemason.jpg}
\end{center}
   \caption{Freemason Inggris yang diam-diam sedang merencanakan serangan bom nuklir untuk menguasai dunia - di Earls Court, London, 1992 \cite{5}.}
\label{fig:1}
\label{fig:onecol}
\end{figure}

\begin{figure*}[t]
\begin{center}
% \fbox{\rule{0pt}{2in} \rule{.9\linewidth}{0pt}}
\includegraphics[width=1\textwidth]{british.jpg}
\end{center}
   \caption{Kekaisaran Inggris pada tahun 1937, sebuah pertunjukan kekuatan Anglo-Saxon yang luar biasa \cite{14}.}
   \label{fig:2}
\end{figure*}

Menurut orang dalam ini, 25-30 orang yang hadir dalam pertemuan itu \textit{"...semuanya orang Inggris dan beberapa adalah tokoh yang sangat dikenal masyarakat Inggris... ada sedikit kalangan aristokrat juga di sana dan beberapa di antaranya berasal dari latar belakang yang sangat aristokrat. Ada seseorang yang saya kenali sebagai politisi senior. Dua lainnya adalah tokoh senior dari kepolisian dan militer. Keduanya dikenal secara nasional dan merupakan tokoh kunci penasehat pemerintahan saat ini — pada saat sekarang ini"} \cite{4}. Orang dalam itu mengatakan bahwa dia menghadiri pertemuan tersebut,\ \textit{"Secara tidak sengaja! Saya pikir itu adalah pertemuan tiga bulanan seperti biasanya... Saya hadir dan ternyata itu bukanlah pertemuan yang saya pikirkan. Alasan saya diundang... karena jabatan saya dan karena mereka percaya bahwa saya adalah salah satu dari mereka."} \cite{4}.

Linimasa mendasar dari peristiwa-peristiwa yang dibahas dalam pertemuan (pada 2005) adalah sebagai berikut:
\begin{flushleft}
\begin{enumerate}
    \item Memancing Iran atau Tiongkok untuk menggunakan senjata nuklir taktis yang menimbulkan pertukaran nuklir terbatas, lalu menetapkan gencatan senjata.
    \item Melepaskan senjata biologis kepada Tiongkok yang dilaporkan sebagai target utama "sejak tahun 70-an".
    \item Mendirikan pemerintahan militer totaliter yang dibenarkan oleh ketakutan dan kekacauan yang dihasilkan.
\end{enumerate}
\end{flushleft}

Namun hal terpenting adalah harapan mereka atas peristiwa-peristiwa tersebut: \textit{"Jadi kita akan menuju perang ini, lalu setelah itu... akan ada suatu peristiwa geofisika di Bumi yang akan berdampak pada semua orang"} \cite{4}. Orang dalam tersebut percaya bahwa selama peristiwa geofisika ini, \textit{\textbf{"kerak Bumi akan bergeser sekitar 30 derajat, sekitar 1700 hingga 2000 mil ke selatan, dan akan menyebabkan kekacauan besar, dampaknya akan bertahan sangat lama hingga masa mendatang"}} \cite{4}.

Alasan dari semua perencanaan rahasia ini, tentu saja, adalah kekuasaan. Orang dalam menjelaskan, \textit{"Pada saat itu kita semua akan mengalami perang nuklir dan biologis. Populasi Bumi, jika ini terjadi, akan berkurang drastis. Ketika peristiwa geofisika ini akan terjadi, maka mereka yang tersisa mungkin akan berkurang setengah lagi. Yang selamat menentukan siapa yang akan membawa dunia dan populasi yang tersisa ke era berikutnya. Jadi kami membahas tentang era paska bencana. Siapa yang akan memimpin? Siapa yang akan berkuasa? Semuanya membahas tentang itu. Mereka sangat berharap hal-hal ini bisa segera dilakukan dalam kerangka waktu tertentu... Sebuah struktur harus direncanakan sebelum [kekacauan] terjadi dengan kepastian bahwa mereka bisa bertahan terhadap apapun yang terjadi -- sehingga mereka dapat berdiri tegak setelahnya dan tetap berkuasa seperti sebelumnya"} \cite{4}. Selama wawancara, nama rencana ini, "Misi Anglo-Saxon", juga sempat dibahas: \textit{[Pewawancara]: "...sebutan Misi Anglo-Saxon diberikan karena pada dasarnya mereka ingin memusnahkan orang Cina sehingga paska bencana, Anglo-Saxon yang akan bertugas untuk pembangunan kembali dan mewarisi Bumi baru tanpa gangguan siapa pun. Apakah itu benar?" [Orang Dalam]: "Saya kurang tahu tentang itu, tapi saya setuju dengan Anda. Sepanjang abad ke-20 setidaknya, dan bahkan sebelum memasuki abad ke-19 dan ke-18, sejarah dunia ini memang sebagian besar dijalankan dari wilayah Barat dan Utara di planet ini"} \cite{4}.

Terkait kerangka waktu pasti dari peristiwa geofisika yang diharapkan, orang dalam tersebut memberikan tebakan terbaiknya: \textit{"...perasaan intuitif saya mengatakan bahwa mereka seharusnya sudah melakukan persiapan... Saya pikir mereka sudah memastikan kapan hal itu akan terjadi... \textbf{Saya punya perasaan yang sangat kuat bahwa itu akan terjadi di masa hidup saya, katakanlah dalam 20 tahun ke depan}... kita sekarang telah memasuki masa di mana peristiwa geofisika ini akan segera terjadi jika mempertimbangkan lamanya waktu yang telah berlalu sejak kejadian terakhir kali, sekitar 11.500 tahun lalu, dan itu terjadi sekitar setiap 11.500 tahun secara siklik. Ini sudah saatnya... Mereka tahu ini akan terjadi. Mereka punya bukti yang kuat bahwa ini akan terjadi... Inilah salah satu halnya -- sungguh mustahil jika mereka tidak tahu. Maksud saya, mereka pasti sudah mengerahkan orang terbaik di dunia untuk memprediksikan hal ini"} \cite{4}.

Ini adalah sebuah kesaksian penting yang patut kita syukuri. Dalam wawancara, penulis juga membahas keyakinannya bahwa Perang Dunia I dan II adalah perang yang dibuat-buat, dan bahwa Misi Anglo-Saxon sudah ada sejak beberapa generasi lalu. 15 tahun sudah berlalu sejak wawancara dilakukan di tahun 2010. Dan, kini masih tersisa lima tahun lagi hingga peristiwa geofisika itu terjadi sesuai kerangka waktu 20 tahun yang dinyatakan orang dalam tersebut.

\subsection{Pengetahuan Esoterik Druidik Barat Tentang Bencana}

Pengetahuan Barat terkait bencana berulang dirahasiakan tidak hanya oleh Freemason. Kaum Druid, sebuah budaya Celtic kuno yang telah didokumentasikan dan berasal dari setidaknya 2.400 tahun yang lalu \cite{7}, mewariskan pengetahuan tentang bencana berulang di Bumi. Druid terakhir dikenal dengan nama Ben McBrady. Dalam "The Last Druid", sebuah film dokumenter tahun 1992, ia membagikan informasi tentang pengetahuan para kaum Druid: \textit{"Saya adalah anggota terakhir dari Ordo yang terbentuk paska bencana atau malapetaka besar terakhir terjadi. Sekarang peradaban benar-benar hancur akibat dampak besar dan mengerikan dari badai listrik yang terperangkap di ekor meteor atau hujanan meteor di Bumi... Semua pengetahuan berada di bawah naungan ordo tetapi mereka sangat menitikberatkan astronomi karena mereka telah mengalami begitu banyak bencana krusial. Diyakini bahwa dengan adanya pengetahuan penuh tentang astronomi akan memungkinkan prediksi kondisi saat kemungkinan bencana terjadi dan mengambil tindakan untuk melindungi diri mereka sendiri. Jika Anda melihat kompleks megalitik besar di Irlandia yang digambarkan sebagai lorong kuburan, itu adalah tempat perlindungan bom yang sangat primitif. Letaknya jauh di atas tingkat gelombang pasang mana pun dan terlindung dari hujan meteor"} \cite{8,9}.
% Itu juga diyakini bahwa Freemasonry sendiri sebenarnya berasal dari para Druid \cite{10}.

\section{Bukti Persiapan Barat terhadap Bencana Alam di Masa Kini}

Mengingat bahwa penguasa Barat nampaknya meyakini bahwa bencana alam geofisika global segera terjadi, tentu saja kita menduga bahwa mereka sudah melakukan persiapan besar-besaran untuk melindungi diri dari peristiwa tersebut. Dan hal ini terbukti dari domain publik terkait jaringan ekstensif basis bawah tanah yang cukup dalam di berbagai negara Barat. Instalasi terkait akan melindungi penghuninya saat perang nuklir dan juga berfungsi sebagai perlindungan dari berbagai jenis bencana alam. Berdasarkan kesaksian Senior Freemason Inggris dari Project Camelot \cite{4,6}, skenario-skenario ini bukanlah kemungkinan, melainkan rencana yang sudah dipersiapkan. Perkiraan jumlah uang yang diperlukan untuk membangun, mempekerjakan staf, dan memelihara basis-basis ini juga cocok dengan nominal uang puluhan triliun dolar yang hilang dari pemerintahan AS selama 18 tahun (dibahas di bagian berikutnya) \cite{11,12,13}. Contoh lain persiapan mereka terhadap peristiwa kepunahan adalah beragam proyek yang diarsipkan seperti gudang benih dan pengetahuan.

\subsection{Basis Bawah Tanah dan Bawah Laut Amerika}

Penyelidikan publik paling ekstensif tentang basis bawah tanah yang saya temukan berasal dari Richard Sauder, seorang peneliti independen Amerika yang telah menerbitkan beberapa buku tentang basis bawah tanah yang dalam \cite{22}. Karya Sauder terdiri dari mengarsipkan dokumen dan rencana pemerintah, menelusuri kisah berita masa lampau maupun masa kini dan teknologi, membina sumber, dan mengumpulkan klaim orang dalam. Penelitian Sauder mengungkap bahwa ada jaringan besar basis bawah tanah dan bawah laut yang dalam di dan sekitar Amerika beserta wilayah-wilayahnya (Gambar \ref{fig:4}), mencapai kedalaman potensial setidaknya 3 mil, dan kemungkinan terhubung oleh kereta levitasi magnetik berkecepatan tinggi dalam tabung vakum bawah tanah. Basis-basis ini secara diam-diam dibiayai melalui \textit{"permainan cangkang pencucian uang tingkat tinggi, internasional, antar-lembaga"} yang dijalankan oleh pemilik perusahaan Amerika Serikat \cite{22}. Telaah lanjutan untuk memperkirakan jumlah basis ini oleh Catherine Austin Fitts (yang karyanya dibahas di bagian selanjutnya) dan salah satu kolaboratornya menghasilkan estimasi sebanyak 170 basis bawah tanah dan bawah laut Amerika \cite{16,20}.

\begin{figure}[b]
\begin{center}
% \fbox{\rule{0pt}{2in} \rule{0.9\linewidth}{0pt}}
   \includegraphics[width=1\linewidth]{penta.jpg}
\end{center}
   \caption{Apa sebenarnya yang ada di bawah Gedung Putih dan Pentagon? Ternyata, ada jaringan terowongan bawah tanah yang dalam (Gambar: \cite{31}).}
\label{fig:3}
\label{fig:onecol}
\end{figure}
\begin{figure*}[t]
\begin{center}
% \fbox{\rule{0pt}{2in} \rule{.9\linewidth}{0pt}}
\includegraphics[width=0.9\textwidth]{basescrop.png}
\end{center}
   \caption{Sebuah peta yang menunjukkan kemungkinan lokasi persis pangkalan bawah tanah dan bawah laut serta terowongan kapal selam bawah air yang menuju ke daratan yang terungkap dari penelitian Sauder. Ia \textit{"yakin bahwa ada \textbf{jauh lebih banyak} fasilitas daripada [ini]"} \cite{22}.}
   \label{fig:4}
\end{figure*}

Berikut adalah beberapa kutipan kesaksian dari sumber-sumber Sauder yang merincikan seberapa luas beberapa pangkalan ini:

\begin{flushleft}
\begin{enumerate}
    \item Camp David, Maryland: \textit{"Sumber saya memberitahukan saya bahwa bagian bawah tanah Camp David sangat luas dan rumit, serta ada bermil-mil terowongan rahasia sehingga diragukan jika ada yang bisa mengingat gambaran peta lengkap terkait fasilitas itu"} \cite{22}.
    \item The White House, Washington DC: \textit{"Salah satu teman dekat saya dibawa turun ke fasilitas ini selama pemerintahan Lyndon B. Johnson pada tahun 1960-an. Dia masuk ke sebuah lift di Gedung Putih dan dikawal langsung ke bawah. Lift itu turun sebanyak 17 tingkat. Ketika pintu terbuka di bawah tanah, dia dikawal menyusuri sebuah koridor yang tampak menghilang hingga titik lenyap di kejauhan. Ada pintu-pintu dan koridor lain yang terbuka dari koridor itu"} \cite{22}. Terlihat pada Gambar \ref{fig:3}.
    \item Fort Meade, Maryland - dari seorang sumber yang secara tidak sengaja masuk ke "ruang bawah tanah" pada tahun 1970-an: \textit{"Saya membuka pintu dan ada sebuah tangga yang mengarah ke bawah. Saya berjalan ke tepinya dan melihat ke bawah di antara pagar pembatas. Saya tidak menghitung jumlah lantai ke bawah, namun saya merasa ada sekitar 15-20 lantai... Saya menuruni satu anak tangga dan ada sebuah pintu... Saya membuka pintu untuk mengintip serta melihat ke kiri dan kanan lalu mendapati sebuah terowongan yang membentang sejauh mata memandang ke kedua arah. Luasnya jelas-jelas melebihi area cakupan gedung dan tempat parkir di lantai dasar. Ada pintu-pintu di sepanjang dinding seberang yang berjarak sekitar 30-40 kaki... Saya memutuskan untuk memeriksa beberapa lantai lagi, jadi saya menuruni satu tingkat... dan melihat tata letak yang sama... Saya turun satu lantai lagi dan tata letaknya masih persis seperti dua lantai pertama"} \cite{22}.
\end{enumerate}
\end{flushleft}
\begin{figure}[t]
\begin{center}
% \fbox{\rule{0pt}{2in} \rule{0.9\linewidth}{0pt}}
   \includegraphics[width=1\linewidth]{undersea.jpg}
\end{center}
   \caption{Ilustrasi pangkalan bawah laut oleh Walter Koerschner. Ia adalah ilustrator untuk tim pangkalan bawah laut Rock-Site Angkatan Laut AS di Pusat Persenjataan China Lake, California pada tahun 1960-an. Salah satu sumber Sauder mengungkapkan bahwa ada pangkalan bawah tanah sedalam satu mil di China Lake \cite{22,23}.}
\label{fig:5}
\label{fig:onecol}
\end{figure}

Sauder juga menerima kesaksian tentang kereta levitasi magnetik bawah tanah yang mencapai kecepatan 2.000 mph, pangkalan yang dibangun di bawah dasar laut (Gambar \ref{fig:5}), dan terowongan kapal selam bawah laut yang menuju ke daratan. Mengenai salah satu kesaksian tentang pangkalan bawah laut di Teluk Meksiko, Sauder mengatakan, \textit{"Sekitar setengah tahun setelah penerbitan Basis Bawah Tanah dan Bawah Laut, saya dihubungi oleh seorang pria yang mengatakan bahwa dia mengetahui tentang sebuah proyek bawah laut yang tidak biasa... dia menyebutkan bahwa proyek tersebut berada di bawah dasar laut Teluk Meksiko, dan bahwa Parsons adalah kontraktornya. Dia melanjutkan dengan mengatakan bahwa Parsons telah membeli beberapa peralatan khusus yang akan dioperasikan pada kedalaman 2.800 kaki di bawah dasar laut... Peralatan ini sangatlah khusus sehingga bisa dipastikan adanya manusia yang tinggal di tempat pemasangan alat tersebut"} \cite{22}.

\begin{figure}[t]
\begin{center}
% \fbox{\rule{0pt}{2in} \rule{0.9\linewidth}{0pt}}
   \includegraphics[width=1\linewidth]{sub.jpg}
\end{center}
   \caption{Ilustrasi terowongan kapal selam bawah laut oleh Walter Koerschner \cite{22,23}.}
\label{fig:6}
\label{fig:onecol}

\end{figure}

\begin{figure}[t]
\begin{center}
% \fbox{\rule{0pt}{2in} \rule{0.9\linewidth}{0pt}}
   \includegraphics[width=1\linewidth]{iran.jpeg}
\end{center}
   \caption{Sebuah cuplikan dari video resmi Iran yang menampilkan "kota rudal" bawah tanah mereka \cite{39,40}.}
\label{fig:12}
\label{fig:onecol}
\end{figure}

Jika memang benar ada jaringan rahasia transkontinental berupa 170+ pangkalan bawah tanah dan bawah laut yang digali hingga bermil-mil jauh di bawah permukaan tanah, terhubung oleh kereta maglev tabung vakum hipersonik, didanai dengan hasil kerja keras kita, massa umat manusia saat ini hidup dalam ketidaktahuan akan hal yang bersifat fatal, bukan hanya tidak menyadari apa yang ada di bawah mereka tetapi juga apa yang menanti mereka di masa depan, sementara mereka menelan mentah-mentah pernyataan kosong dan terkoordinasi dari para politisi pengendali.

Catatan tambahan - keberadaan jaringan terowongan bawah tanah yang besar telah terungkap dengan jelas dalam konflik yang sedang berlangsung di Timur Tengah (terowongan Hamas di bawah Jalur Gaza \cite{38}, dan "kota rudal" bawah tanah Iran (Gambar \ref{fig:12}) \cite{39,40}). Hal ini tidak menyisakan keraguan baik terhadap kemungkinan pembangunan maupun eksistensi nyata dari struktur seperti itu. Tentunya, ini juga membuat kita bertanya-tanya struktur seperti apa yang mungkin telah dibangun oleh negara-negara lain yang memiliki modal jauh lebih besar dalam waktu yang sama.

\subsection{Bukti Tambahan Persiapan Bunker dan Bencana}

\begin{figure}[t]
\begin{center}
% \fbox{\rule{0pt}{2in} \rule{0.9\linewidth}{0pt}}
   \includegraphics[width=1\linewidth]{tyrol.jpg}
\end{center}
   \caption{Bunker-bunker di Tyrol Selatan, Swiss. Swiss dikenal cerdik dalam menyamarkan bunker-bunker gunungnya yang membentang di Alpen Eropa \cite{32}.}
\label{fig:7}
\label{fig:onecol}
\end{figure}

\begin{figure}[t]
\begin{center}
% \fbox{\rule{0pt}{2in} \rule{0.9\linewidth}{0pt}}
   \includegraphics[width=1\linewidth]{svalbard.jpg}
\end{center}
   \caption{Gudang Benih Global Svalbard di Norwegia, berisi lebih dari satu juta benih \cite{24}. Kita tidak tahu pasti bencana seperti apa yang akan membutuhkan benih sebanyak itu.}
\label{fig:8}
\label{fig:onecol}
\end{figure}

Selain pangkalan kerajaan bawah tanah Amerika, ada banyak petunjuk tambahan terkait persiapan menghadapi bencana di seluruh dunia. Norwegia, Swiss, Swedia, dan Finlandia adalah contoh lainnya:
\begin{flushleft}
\begin{enumerate}
    \item Project Camelot membagikan sebuah kesaksian yang relevan dari seorang politisi Norwegia \cite{25,26}, yang identitasnya telah diverifikasi namun tetap dirahasiakan. Ia mengklaim bahwa Norwegia memiliki 18 pangkalan bawah tanah yang luas, dan bahwa Norwegia (bersama dengan Israel dan "banyak negara lain") sedang membangun pangkalan-pangkalan ini untuk mempersiapkan diri menghadapi semacam bencana alam. Richard Sauder juga menerima kesaksian dari seorang pria yang pernah berada di dalam pangkalan bawah tanah besar yang dibangun di dalam gunung Norwegia yang dilubangi \cite{22}.
    \item Swiss dikenal memiliki banyak bunker nuklir yang dibangun di pegunungan Alpen (Gambar \ref{fig:7}). Jumlahnya lebih dari 370.000—cukup untuk melindungi setiap penduduknya \cite{27}.
    \item Swedia dan Finlandia memiliki cukup banyak bunker untuk melindungi penduduk di setiap kota besar \cite{27}. 
\end{enumerate}
\end{flushleft}

Para taipan bisnis Silicon Valley tampaknya juga mengetahui hal ini. Dilaporkan, \textit{"Reid Hoffman, salah satu pendiri LinkedIn dan investor terkemuka, mengatakan kepada The New Yorker awal tahun ini bahwa ia memperkirakan lebih dari 50\% miliarder Silicon Valley telah membeli semacam 'asuransi kiamat,' seperti bunker bawah tanah... Menurut Jim Dobson, kontributor Forbes, banyak miliarder memiliki pesawat pribadi yang 'siap berangkat kapan saja.' Mereka juga memiliki sepeda motor, persenjataan, dan generator"} \cite{28}.

Ada juga berbagai proyek arsip besar seperti Global Knowledge Vault, yang dijalankan oleh Arch Mission Foundation, \cite{29} dan Svalbard Global Seed Vault \cite{30} yang tampaknya mempersiapkan diri untuk melestarikan aset vital umat manusia jika terjadi bencana tingkat kepunahan.

\begin{figure*}[t]
\begin{center}
% \fbox{\rule{0pt}{2in} \rule{.9\linewidth}{0pt}}
\includegraphics[width=0.9\textwidth]{govcrop2.png}
\end{center}
   \caption{Pendapatan, pengeluaran, dan pengeluaran rahasia pangkalan bawah tanah pemerintah AS dari tahun 1998 hingga 2023 \cite{19}.}
   \label{fig:9}
\end{figure*}
\section{Mekanisme Pendanaan Demokratis untuk Pangkalan Bawah Tanah Raksasa}

Jadi, bagaimana caranya jaringan transkontinental besar dengan jumlah lebih dari 170 pangkalan bawah tanah dan bawah laut ini didanai tanpa sepengetahuan para buruh utang? Ada satu jejak dokumen yang memberikan gambaran terkait skala uang yang masuk ke proyek-proyek ini dan dari mana asalnya. Pada tahun 2017, Catherine Austin Fitts, seorang bankir investasi Amerika dan mantan pejabat publik selama pemerintahan Bush, bersama Mark Skidmore, seorang ekonom dari Universitas Negeri Michigan, menemukan pengeluaran tidak resmi sebesar 21 triliun USD dalam pemerintahan AS selama tahun fiskal 1998-2015 \cite{11,12,13}.

Menurut laporan mereka, \textit{"Pada tanggal 7 Oktober 2016, Reuters menerbitkan sebuah artikel dari Scot Paltrow (2016), yang melaporkan bahwa pada tahun fiskal 2015 Angkatan Darat melakukan penyesuaian akuntansi sebesar \$6,5 triliun tanpa bukti yang valid “untuk menciptakan ilusi bahwa pembukuannya seimbang.” Mengingat anggaran dana umum Angkatan Darat tahun itu adalah \$122 miliar, ini adalah pengungkapan yang sangat mencengangkan... DOD telah menjadi berita utama di media beberapa tahun sebelumnya karena masalah akuntansi pada tanggal 10 September 2001 ketika Menteri Pertahanan Donald Rumsfeld menyatakan dalam sidang Kongres (C-SPAN, 2014) bahwa DOD telah kehilangan jejak transaksi sebesar \$2,3 triliun... Pengakuan ini menjadi berita utama pada hari itu, namun dilupakan sehari kemudian ketika tragedi 9/11 menarik perhatian seluruh dunia... Ketika Profesor Mark Skidmore mengetahui tentang transaksi Angkatan Darat yang tidak dapat diverifikasi sebesar \$6,5 triliun, dia menghubungi Ms. Fitts dan mereka sepakat pada musim semi 2017 untuk bekerja sama mengidentifikasi laporan pemerintah lain yang menunjukkan transaksi besar yang tidak dapat diverifikasi di HUD dan DOD. Selama enam bulan berikutnya, Skidmore, Fitts dan sekelompok kecil mahasiswa pascasarjana mengumpulkan dokumen resmi pemerintah yang menunjukkan total transaksi tak terdokumentasi sebesar \$21 triliun selama periode 1998-2016"} \cite{12}.

Pada periode yang sama selama 18 tahun antara 1998-2015, penerimaan pemerintah AS yang diakui secara publik hanya sebesar 40,8 triliun \cite{15}, menunjukkan bahwa lebih dari setengah penerimaan pemerintah AS secara diam-diam dihabiskan untuk pangkalan bawah tanah di samping pengeluaran pemerintah AS yang diakui secara publik. Perlu dicatat bahwa pengeluaran rahasia ini menyebabkan defisit anggaran yang telah berlangsung lama, dan kemungkinan diperkirakan sudah terjadi sebelum tahun 1998 hingga sekarang, yang menyiratkan bahwa jumlah nominal total yang dihabiskan untuk pangkalan-pangkalan ini jauh lebih besar dari 21 triliun dolar. Menggunakan rasio pengeluaran rahasia yang sama untuk periode 2016-2023 menghasilkan total sebesar 36,6 triliun USD yang dihabiskan sejak 1998.

Pada tahun 2021, Mark Skidmore menerbitkan pembaruan untuk riset ini terkait pengumuman dari Bloomberg bahwa selama tahun fiskal 2017-19, Pentagon mencatat penyesuaian akuntansi yang sangat besar sejumlah 94,7 triliun dolar \cite{17,18}. Jika kita memperhitungkan pemalsuan dolar AS melalui sistem perbankan sentral yang telah terjadi selama lebih dari satu abad sejak didirikannya Federal Reserve pada tahun 1913 \cite{37}, terbukti bahwa semua akuntansi dolar publik sepenuhnya tidak dapat dipercaya, dan bahwa mata uang serta pemerintahan AS hanyalah sistem alokasi sumber daya bagi para pemilik royalisnya untuk mengambil (atau lebih tepatnya, menguras) sebanyak yang mereka mau secara diam-diam.

\section{Keturunan Jupiter: Identitas Para Raja Bayangan Barat}

Jadi, siapa dalang di balik semuanya? Kita tidak bisa mengetahui secara pasti, karena para penguasa modal Barat selalu bersembunyi dalam bayangan. Meski muncul beragam spekulasi teori, mulai dari tokoh publik hingga makhluk ekstraterestrial, jawaban terbaik saya ada pada karya hidup seorang blogger anonim dengan nama samaran "Amallulla". Karyanya menggabungkan sintesis luas dari 20 penulis lebih dan 50 dokumen "tak tergantikan" yang membahas topik sejarah kuno dan modern, simbolisme okultisme, dan politik Barat \cite{33,34}. Saya hanya bisa menggambarkan karyanya sebagai "profetik" terkait bencana geofisika yang akan datang - dan itu \textit{jauh} lebih komprehensif daripada karya saya.

Amallulla mengidentifikasi tiga faksi politik Barat dengan sebutan kolektif "Keturunan Jupiter", yang memiliki pengetahuan tentang "akhir zaman" - bencana berulang Bumi. Ia percaya bahwa ketiga faksi ini bersama-sama mengontrol negara-negara Barat saat ini, namun ia membaginya menjadi tiga kelompok berbeda berdasarkan perbedaan asal usul dan identitas historis, kemungkinan perselisihan di masa lalu, serta perbedaan yang ia sadari dalam sistem nilai dan tindakannya.

Tiga faksi tersebut secara garis besar dapat dikategorikan sebagai berikut:

\begin{flushleft}
\begin{enumerate}
    \item \textbf{Para Bankir}: Elite Romawi Kuno, yang kemudian menjadi Ksatria Templar dan Yurisdiksi Utara Freemason di Amerika.
    \item \textbf{Para Pemikir}: Rosikrusian dan Freemason Amerika Selatan.
    \item \textbf{Para Jesuit dan Paus Hitam}: Faksi dari keturunan Jupiter di dalam Gereja Katolik Roma.
\end{enumerate}
\end{flushleft}

Saat ini, ketiga faksi tersebut bersama-sama membentuk Illuminati Eropa, Freemason, dan CIA. Seperti yang dijelaskan Amallulla, \textit{"Saat ini, di akhir zaman, keturunan Jupiter bersembunyi dengan sangat baik di balik izin akses informasi yang bahkan tidak dimiliki oleh Presiden Amerika Serikat yang sedang menjabat. Dengan kata lain, mereka merahasiakan keberadaan diri dari pengawasan publik. \textbf{Keturunan Jupiter tidak hanya mengendalikan militer dan pemerintahan Amerika Serikat, tetapi melalui kekuatan mata uang fiat, korporasi-korporasi besar, dan bentuk pemerintahan Republik yang mereka ciptakan (politisi yang mudah disuap dan dikendalikan), mereka menguasai seluruh dunia Barat}"} \cite{33,34}.

\begin{figure}[t]
\begin{center}
% \fbox{\rule{0pt}{2in} \rule{0.9\linewidth}{0pt}}
   \includegraphics[width=1\linewidth]{illuminati.jpg}
\end{center}
   \caption{Siapakah sebenarnya Keturunan Jupiter? (Gambar: \cite{35})}
\label{fig:10}
\label{fig:onecol}
\end{figure}

\begin{figure}[t]

\begin{center}
% \fbox{\rule{0pt}{2in} \rule{0.9\linewidth}{0pt}}
   \includegraphics[width=1\linewidth]{pike.jpg}
\end{center}
   \caption{Batolit Pike Peak yang terkenal, disorot dengan warna merah, bersamaan dengan lanskap Amerika Serikat bagian barat \cite{36}. Mungkinkah Amerika Serikat telah dipersiapkan untuk menguasai lokasi ini?}
\label{fig:11}
\label{fig:onecol}
\end{figure}

Menurut Amallulla, orang-orang ini meremehkan agama, memanipulasi kitab suci dari agama-agama besar dunia untuk keuntungan mereka, dan menggunakan simbolisme secara masif. Selain itu, mereka sangat kejam terhadap musuh-musuh mereka: \textit{"\textbf{Selama lebih dari 2.600 tahun, mereka secara sistematis mengeliminasi siapa pun yang memiliki pengetahuan khusus tentang akhir zaman. Dan saya tidak hanya mengacu pada para druid, kabbalah Yahudi, orang Mesir kuno, Arab, dan mistikus India, tapi juga Tengkorak Panjang di Amerika Selatan dan pendeta Maya di Amerika Tengah. Bukti bahwa mereka melenyapkan populasi yang dulu berkembang di Amerika Utara demi menjaga kawasan ini sebagai Tanah Akhir Zaman sangatlah ekstrim. Genosida terhadap “Indian” Amerika merupakan operasi pembersihan}"} \cite{33,34}.

Amallulla juga percaya bahwa seluruh proyek "Amerika Serikat" dilakukan untuk mengamankan kendali terhadap "Batolit Pikes Peak", barisan pegunungan granit di Gunung Rocky dengan perlindungan yang sangat baik terhadap bencana geofisika (Gambar \ref{fig:11}). Menurut Amallulla, \textit{"Sebelum, selama, dan setelah Perang Saudara, para bankir dan pemikir berjuang bukan untuk menguasai Amerika Serikat, tetapi demi batolit Pikes Peak, yang merupakan salah satu batolit granit paling unik di seluruh dunia... Tidak ada batolit granit lain di ketinggian tersebut dengan jarak sejauh itu dari garis pantai laut di mana pun di dunia. Ini adalah lokasi ideal untuk bertahan dari perpindahan kerak Bumi"} \cite{33,34}. Penelitian Amallulla mengungkapkan bahwa ada sistem terowongan bawah tanah yang luas dibangun di bawah dan sekitar area tersebut saat ini \cite{36}.

\section{Kesimpulan}

Dalam makalah ini saya telah merincikan berbagai kesaksian yang menunjukkan tindakan para elite Barat dalam menyimpan pengetahuan tentang bencana berulang di Bumi selama ribuan tahun secara seksama, dengan pemahaman bahwa hal tersebut akan terjadi dalam waktu dekat, mereka membangun tempat perlindungan bawah tanah yang luas untuk mempersiapkan diri dari peristiwa semacam itu, dan berencana memanfaatkannya secara politis dan militer untuk mencapai dominasi dunia. Saya juga menyebutkan petunjuk tentang bagaimana hal ini didanai di Amerika serta merujuk pada teori yang paling masuk akal mengenai garis keturunan yang menjalankan semua ini. Bagi mereka yang ingin mengetahui lebih banyak, masih banyak informasi tambahan yang belum saya sertakan namun dapat ditelusuri di referensi.

Data terukur paling akurat yang mengarah pada peristiwa geofisika mendatang adalah cepatnya pergeseran medan geomagnetik Bumi. Ini dapat diukur tidak hanya dari pergerakan kutub utara magnetik yang semakin cepat (Gambar \ref{fig:13}) dan tumbuhnya anomali geomagnetik Atlantik Selatan, tetapi juga dari percepatan pelemahan dan distorsi medan geomagnetik selama 400 tahun terakhir \cite{3}. Data ilmiah tersebut dijabarkan lebih jelas dalam dua makalah ECDO pertama saya yang dapat diakses di situs web saya \cite{3}.

\begin{figure}[t]
\begin{center}
% \fbox{\rule{0pt}{2in} \rule{0.9\linewidth}{0pt}}
   \includegraphics[width=1\linewidth]{npw.jpg}
\end{center}
   \caption{Posisi kutub utara geomagnetik dari tahun 1590 hingga 2025, digambarkan dalam interval 5 tahun \cite{41}. Pergerakannya mulai meningkat pesat pada tahun 1975.}
\label{fig:13}
\label{fig:onecol}
\end{figure}

Sebagai penutup, saya akan menuliskan sebuah kutipan dari Amallulla, sang prediktor masa depan, yang menjelaskan bagaimana \textit{"\textbf{segalanya adalah satu hal}"}: \textit{"Di sini saya benar-benar mendorong imajinasi Anda sampai batas terjauh. Anda harus melupakan dunia yang ditinggali sekarang yang tentunya sudah dikenali sejak kecil. Tinggalkanlah itu. Itu adalah realitas yang dibuat-buat serta berbeda dengan gambaran film Matrix yang akan membuat Anda tetap tertidur hingga saat-saat terakhir. Terkadang saya ingin menyangkal kenyataan ini, tetapi apa yang saya bagikan di situs web bukanlah naskah film. Dibutuhkan lebih dari setengah dekade bagi saya untuk menyadari “Segalanya adalah satu hal,” semboyan yang saya adaptasikan untuk An Apocalyptic Synthesis. Ini adalah konsep yang sulit dijelaskan. Untuk saat ini, mari kita berpikir dalam kerangka film Matrix. Analogi ini sangat sesuai. Saya kesulitan mengungkapkan hal ini namun yang ingin saya sampaikan berikutnya bukanlah sesuatu yang dikarang-karang. Untuk saat ini, analogi film Matrix adalah gambaran terbaik untuk memahami kenyataan menyakitkan yang ingin saya sampaikan. \textbf{Segalanya dalam hidup Anda, termasuk seluruh catatan sejarah, hal-hal yang dianggap umum, ilmu pengetahuan dan akademi konsensus, politik, agama, semuanya dalam satu atau cara lainnya berkaitan dengan perpindahan kerak Bumi atau kemiringan sumbu yang akan terjadi.} Anda hanya belum melihatnya sekarang. Ini bukanlah mimpi buruk semata. Semuanya butuh waktu. Tapi saya janji, penghujung jalan ini akan menuntun Anda pada kehidupan yang serupa dengan realitas simulasi komputer Matrix"} \cite{33,34}.

Semoga beruntung.

\section{Ucapan Terima Kasih}

Terima kasih kepada semua individu yang memilih untuk menyumbangkan pengetahuannya ke domain publik. Tanpa Anda, karya ini tidak mungkin ditulis dan umat manusia akan terus berada dalam kegelapan. Pilihan Anda akan memberikan dampak yang luar biasa bukan hanya bagi kita tapi juga generasi-generasi mendatang. Dukungan kalian sangat berarti dan saya sangat bersyukur tiada tara.

\clearpage
\twocolumn

{\small
\bibliographystyle{ieee}
\bibliography{egbib}
}

\end{document}
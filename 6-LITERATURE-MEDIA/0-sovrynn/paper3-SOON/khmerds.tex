\documentclass[10pt,twocolumn,letterpaper]{article}

\usepackage{booktabs}
% \usepackage{caption}
% \captionsetup[table]{skip=8pt}   % មានឥទ្ធិពលតែលើតារាងប៉ុណ្ណោះ
\usepackage{stfloats}  % បន្ថែមនេះទៅក្នុងការណែនាំ
\usepackage{float}

% \usepackage{fontspec}
\usepackage[english]{babel}

% load Lao via babelprovide, turn on "onchar=ids" for automatic shaping
\babelprovide[import,onchar=ids fonts]{khmer}

% main (rm) font for Latin
\babelfont{rm}{Noto Serif}

% Lao text in Noto Serif Lao at 1.2× scale
\babelfont[khmer]{rm}{Noto Serif Khmer}
\babelfont[khmer]{sf}{Noto Serif Khmer}

% alternate (sans-serif) font for Latin
\babelfont{alt}{Lato}

% Lao text in Noto Serif Lao for the alt family too
\babelfont[khmer]{alt}{Noto Serif Khmer}

\babelpatterns[khmer]{%
  % all Khmer consonants (U+1780–U+17A2)
  1ក 1ខ 1គ 1ឃ 1ង 1ច 1ឆ 1ជ 1ឈ 1ញ
  1ដ 1ឋ 1ឌ 1ឍ 1ណ 1ត 1ថ 1ទ 1ធ 1ន
  1ប 1ផ 1ព 1ភ 1ម 1យ 1រ 1ល 1វ 1ឝ
  1ឞ 1ស 1ហ 1ឡ 1អ
  % all modern independent vowels (U+17A5–U+17B3)
  1ឥ 1ឦ 1ឧ 1ឨ 1ឩ 1ឪ 1ឫ 1ឬ 1ឭ 1ឮ
  1ឯ 1ឰ 1ឱ 1ឲ 1ឳ%
}


\usepackage{cvpr}
\usepackage{times}
\usepackage{epsfig}
\usepackage{graphicx}
\usepackage{amsmath}
\usepackage{amssymb}
\usepackage[breaklinks=true,bookmarks=false]{hyperref}

\makeatletter
\def\cvprsubsection{\@startsection {subsection}{2}{\z@}
    {8pt plus 2pt minus 2pt}{6pt}{\bfseries\normalsize}}
\makeatother

\cvprfinalcopy % *** សូមដោះស្រាយបន្ទាត់នេះសម្រាប់ការដាក់ស្នើចុងក្រោយ

\def\cvprPaperID{****} % *** សូមបញ្ចូលលេខ ID របស់អត្ថបទ CVPR នៅទីនេះ
\def\httilde{\mbox{\tt\raisebox{-.5ex}{\symbol{126}}}}

% ទំព័រត្រូវបានរាប់លេខក្នុងរបៀបដាក់ស្នើ និងគ្មានលេខក្នុងរបៀបអានយកជាស្ថាពរ
%\ifcvprfinal\pagestyle{empty}\fi
\setcounter{page}{1}
\begin{document}

\title{ឯកសារ ECDO លេខ៣: ភស្តុតាងនៃការរៀបចំរបស់អំណាចគ្រប់គ្រងខាងលិចសម័យបច្ចុប្បន្នសម្រាប់គ្រោះមហន្តរាយភូគព្ភសាស្ត្រដែលកំពុងខិតខំចូលមកដល់}

\author{ជុនហូ\\
បោះពុម្ពផ្សាយ ខែមិថុនា ២០២៥\\
គេហទំព័រ (ទាញយកឯកសារនៅទីនេះ): \href{https://sovrynn.github.io}{sovrynn.github.io}\\
ឃ្លាំងស្រាវជ្រាវ ECDO: \href{https://github.com/sovrynn/ecdo}{github.com/sovrynn/ecdo}\\
{\tt\small junhobtc@proton.me}
}
\maketitle
%\thispagestyle{empty}

\begin{abstract}
នៅខែឧសភា ឆ្នាំ២០២៤ អ្នកនិពន្ធអនឡាញដែលមានឈ្មោះក្លែងក្លាយថា "The Ethical Skeptic" \cite{0} បានចែករំលែកទ្រឹស្ដីដ៏ច្នៃប្រឌិតមួយដែលហៅថា Exothermic Core-Mantle Decoupling Dzhanibekov Oscillation (ECDO) \cite{1}។ ទ្រឹស្ដីនេះស្នើថាផែនដីបានជួបប្រទះនឹងការផ្លាស់ប្តូរភ្លាមៗ និងអាក្រក់ដែលនៅក្នុងអ័ក្សវិលរបស់វា ដែលបណ្តាលឱ្យមានជំនន់ធំទូលាយទូទាំងពិភពលោកដោយសារសមុទ្រហូរលិចលើគោកដោយសារអ៊ីនឺរស៊ីវិល។ លើសពីនេះទៅទៀត វាបង្ហាញពីដំណើរការភូគព្ភសាស្ត្រពន្យល់ និងទិន្នន័យដែលបង្ហាញថាការបញ្ច្រាសបែបនេះអាចកើតឡើងជាបន្ទាន់។ ទោះបីជាការព្យាករណ៍អំពីជំនន់ដ៏អាក្រក់ និងថ្ងៃអន្តរាយបែបនេះមិនមែនជារឿងថ្មីក៏ដោយ ទ្រឹស្ដី ECDO គឺគួរឱ្យចាប់អារម្មណ៍យ៉ាងពិសេសដោយសារវិធីសាកសួរវិទ្យាសាស្ត្រ ទំនើប ពហុវិស័យ និងផ្អែកលើទិន្នន័យ។

អត្ថបទនេះគឺជាស្នាដៃទីបីរបស់ខ្ញុំ \cite{2,3} អំពីប្រធានបទនេះ ហើយផ្តោតលើផ្នែកនយោបាយសម័យបច្ចុប្បន្ននៃទ្រឹស្ដីនេះ៖
\begin{flushleft}
\begin{enumerate}
    \item សក្ខីកម្មរបស់អ្នករួមចំណែកដែលបានបញ្ចេញអំពើពុករលួយ ដែលអំណាចឡាទីនជឿថាគ្រោះមហន្តរាយភូមិសាស្ត្រអាចកើតឡើងគ្រប់ពេល និងមានគម្រោងទាក់ទាញអត្ថប្រយោជន៍នយោបាយ និងយោធាពីព្រឹត្តិការណ៍នេះ។
    \item ភស្តុតាងនៃមូលដ្ឋានក្រោមដី និងក្រោមសមុទ្រដ៏ធំទូលាយរបស់ឡាទីន ដែលត្រូវបានសាងសង់ឡើងដើម្បីរៀបចំសម្រាប់ព្រឹត្តិការណ៍នេះ។
    \item ភស្តុតាងនៃប្រាក់ច្រើនដែលត្រូវបានដកចេញពីរចនាសម្ព័ន្ធរូបិយប័ណ្ណឡាទីន ដើម្បីផ្តល់មូលនិធិដល់មូលដ្ឋានទាំងនេះ។
\end{enumerate}
\end{flushleft}
នេះជាការសិក្សាស្ដីអំពីការរៀបចំយ៉ាងទូលំទូលាយរបស់អំណាចដឹកនាំខាងលិច ដើម្បីត្រៀមខ្លួនសម្រាប់គ្រោះមហន្តរាយធរណីមាត្រដែលពួកគេជឿថានឹងកើតឡើងឆាប់ៗនេះ។
\end{abstract}

\section{សាណាហ្វ្រង់ស៊ីស្កូ និង "បេសកកម្មអង់គ្លេស-សាសន៍សាក់សុន"}
នៅខែមករា ឆ្នាំ ២០១០ គម្រោង Camelot ដែលជាអង្គការសារព័ត៌មាន និងកាសែតជំនួស ដែលចងក្រងសក្ខីកម្មអ្នកលាតត្រដាង បានសម្ភាសន៍ \cite{4,6} អ្នកស្និទ្ធស្នាលម្នាក់ដែលបានចូលរួមរាងកាយក្នុងកិច្ចប្រជុំរបស់សាម៉ុនជាន់ខ្ពស់នៅទីក្រុងឡុងដ៍ក្នុងខែមិថុនា ឆ្នាំ ២០០៥។ ប្រធានបទដែលបានពិភាក្សាក្នុងកិច្ចប្រជុំនោះគឺអំពីផែនការយោធា និងនយោបាយដែលផ្តោតលើផ្ទៃខាងក្រោយនៃ "ព្រឹត្តិការណ៍ភូគព្ភសាស្ត្រ" ដែលកំពុងខិតខំចូលមកដល់ ពោលគឺគ្រោះមហន្តរាយធម្មជាតិកម្រិតពិភពលោក។

\begin{figure}[b]
\begin{center}
   \includegraphics[width=1\linewidth]{freemason.jpg}
\end{center}
   \caption{អង់គ្លេសហ្វ្រីម៉ាសុននៅក្នុងសភាពធម្មជាតិរបស់ពួកគេ កំពុងរៀបគម្រោងដោយស្ងៀមស្ងាត់ដើម្បីទម្លាក់គ្រាប់បែកនុយក្លេអ៊ែរនិងគ្រប់គ្រងពិភពលោក - នៅអ៊ែលស៍ខតនៅទីក្រុងឡុងដ៍ ឆ្នាំ១៩៩២ \cite{5}.}
\label{fig:1}
\label{fig:onecol}
\end{figure}

\begin{figure*}[t]
\begin{center}
\includegraphics[width=1\textwidth]{british.jpg}
\end{center}
   \caption{ចក្រភពអង់គ្លេសនៅឆ្នាំ ១៩៣៧ ដែលបង្ហាញពីអំណាចដ៏រឹងមាំរបស់ជនជាតិអង់គ្លេស-សាសន៍ \cite{14}.}
   \label{fig:2}
\end{figure*}
បើតាមអ្នកស្និទ្ធស្នាលនេះ មនុស្សចំនួន ២៥-៣០ នាក់ដែលបានចូលរួមក្នុងកិច្ចប្រជុំនោះគឺ \textit{"...ទាំងអស់ជាជនជាតិអង់គ្លេស ហើយខ្លះក្នុងចំណោមពួកគេគឺជាតួអង្គល្បីៗដែលប្រជាជននៅចក្រភពអង់គ្លេសនឹងស្គាល់ភ្លាម... មានពួកអភិជនខ្លះ ហើយខ្លះក្នុងចំណោមពួកគេមកពីគ្រួសារអភិជន។ មានម្នាក់ដែលខ្ញុំបានស្គាល់នៅក្នុងកិច្ចប្រជុំនោះដែលជាអ្នកនយោបាយជាន់ខ្ពស់។ មានពីរនាក់ទៀតជាមន្ត្រីជាន់ខ្ពស់នៃក្រមរដ្ឋបាល និងម្នាក់ទៀតពីកងទ័ព។ ទាំងពីរនាក់នេះសុទ្ធតែល្បីល្បាញថ្នាក់ជាតិ ហើជាតួអង្គសំខាន់ក្នុងការផ្តល់យោបល់ដល់រដ្ឋាភិបាលបច្ចុប្បន្ន - នៅពេលនេះ"} \cite{4}។ អ្នកស្និទ្ធស្នាលនិយាយថាគាត់បានចូលរួមក្នុងកិច្ចប្រជុំនោះ \textit{"ដោយចៃដន្យ! ខ្ញុំគិតថាវាគ្រាន់តែជាកិច្ចប្រជុំធម្មតាដែលធ្វើឡើងរៀងរាល់៣ខែ... ខ្ញុំបានទៅកិច្ចប្រជុំនោះ ហើយវាមិនមែនជាកិច្ចប្រជុំដែលខ្ញុំរំពឹងទុកនោះទេ។ ខ្ញុំជឿថាខ្ញុំត្រូវបានអញ្ជើញ... ដោយសារតំណែងរបស់ខ្ញុំ និងដោយសារពួកគេគិតថាខ្ញុំគឺជាមនុស្សដូចគ្នានឹងពួកគេ"} \cite{4}។

បញ្ជីព្រឹត្តិការណ៍ជាមូលដ្ឋានដែលត្រូវបានពិភាក្សានៅក្នុងកិច្ចប្រជុំ (នៅឆ្នាំ ២០០៥) មានដូចខាងក្រោម៖

\begin{flushleft}
\begin{enumerate}
    \item បំផុសអ៊ីរ៉ង់ ឬចិនអោយប្រើអាវុធនុយក្លេអ៊ែរយុទ្ធសាស្ត្រ ហើយនាំអោយមានការប្រយុទ្ធគ្នាដោយប្រើអាវុធនុយក្លេអ៊ែរមានកម្រិត បន្ទាប់មកបង្កើតការឈប់បាញ់។
    \item បញ្ចេញអាវុធជីវសាស្ត្រលើចិន ដែលត្រូវបានគេរាយការណ៍ថាជាគោលដៅសំខាន់ "តាំងពីទសវត្សរ៍ឆ្នាំ៧០"។
    \item នាំអោយមានរដ្ឋាភិបាលយោធាសញ្ញាណតែ ដោយសារការភ័យខ្លាច និងភាពច្របូកច្របល់ដែលកើតឡើង។
\end{enumerate}
\end{flushleft}

ប៉ុន្តែអ្វីដែលសំខាន់បំផុតគឺអ្វីដែលគេរំពឹងថានឹងកើតឡើងបន្ទាប់ពីព្រឹត្តិការណ៍ទាំងនេះ៖ \textit{"ដូច្នេះយើងនឹងចូលទៅក្នុងសង្គ្រាមនេះ បន្ទាប់មក... នឹងមានព្រឹត្តិការណ៍ភូមិសាស្ត្រមួយកើតឡើងនៅលើផែនដី ដែលនឹងប៉ះពាល់ដល់មនុស្សគ្រប់គ្នា"} \cite{4}។ អ្នកដែលមានព័ត៌មានផ្ទៃក្នុងជឿថាក្នុងអំឡុងពេលព្រឹត្តិការណ៍ភូមិសាស្ត្រនេះ \textit{\textbf{"សំបកផែនដីនឹងរអិលប្រហែល ៣០ ដឺក្រេ ប្រហែល ១៧០០ ទៅ ២០០០ ម៉ាយខាងត្បូង ហើយវានឹងបណ្តាលឱ្យមានការរញ្ជួយយ៉ាងធំ ដែលផលប៉ះពាល់របស់វានឹងអស់រយៈពេលយូរអង្វែង"  }} \cite{4}។

មូលហេតុនៃការរៀបចំសម្ងាត់ទាំងអស់នេះគឺ អំណាច។ អ្នកដែលមានព័ត៌មានផ្ទៃក្នុងពន្យល់ថា \textit{"ឥឡូវនេះ នៅពេលនោះ យើងទាំងអស់គ្នានឹងបានឆ្លងកាត់សង្គ្រាមនុយក្លេអ៊ែរ និងជីវសាស្ត្រ។ ប្រជាជននៅលើផែនដី ប្រសិនបើវាកើតឡើង នឹងត្រូវបានកាត់បន្ថយយ៉ាងខ្លាំង។ នៅពេលដែលព្រឹត្តិការណ៍ភូមិសាស្ត្រនេះកើតឡើង នោះអ្នកដែលនៅសល់ប្រហែលជាត្រូវបានកាត់ច្រើនជាងមុនទៀត។ ហើយអ្នកណាដែលរស់រានមានជីវិតនឹងកំណត់ថាអ្នកណានឹងដឹកនាំពិភពលោក និងប្រជាជនដែលនៅសល់ចូលទៅក្នុងយុគសម័យបន្ទាប់។ ដូច្នេះយើងកំពុងនិយាយអំពីយុគសម័យបន្ទាប់ពីគ្រោះមហន្តរាយ។ អ្នកណានឹងគ្រប់គ្រង? អ្នកណានឹងកាន់កាប់? ដូច្នេះវាទាក់ទងនឹងរឿងនោះទាំងអស់។ ហើយនោះហើយជាមូលហេតុដែលពួកគេព្យាយាមយ៉ាងខ្លាំងដើម្បីឱ្យរឿងទាំងនេះកើតឡើងក្នុងរយៈពេលដែលបានកំណត់... រចនាសម្ព័ន្ធមួយត្រូវតែត្រៀមរួចជាស្រេចមុនពេល [ភាពច្របូកច្របល់] កើតឡើងជាមួយនឹងភាពច្បាស់លាស់ខ្លះថាវានឹងឆ្លងកាត់អ្វីដែលនឹងកើតឡើង — ដូច្នេះវាអាចឈរលើជើងទាំងពីរនៅថ្ងៃបន្ទាប់ ហើយបន្តកាន់អំណាច និងមានអំណាចដូចដែលវាធ្លាប់មាន"} \cite{4}។ ក្នុងអំឡុងពេលសម្ភាសន៍ ឈ្មោះនៃផែនការនេះដែលហៅថា "បេសកកម្មអង់គ្លេស-សាសន៍សាក់សុន" ក៏ត្រូវបានពិភាក្សាផងដែរ៖ \textit{[អ្នកសម្ភាសន៍]៖ "...មូលហេតុដែលវាត្រូវបានគេហៅថា បេសកកម្មអង់គ្លេស-សាសន៍សាក់សុន គឺដោយសារតែជាទូទៅផែនការគឺដើម្បីលុបបំបាត់ជនជាតិចិន ដើម្បីឱ្យបន្ទាប់ពីគ្រោះមហន្តរាយ និងនៅពេលដែលអ្វីៗត្រូវបានសាងសង់ឡើងវិញ វានឹងជាជនជាតិអង់គ្លេស-សាសន៍សាក់សុនដែលនៅក្នុងទីតាំងដើម្បីសាងសង់ និងទទួលមរតកផែនដីថ្មី ដោយគ្មាននរណាម្នាក់ផ្សេងទៀតនៅជិត។ តើវាត្រឹមត្រូវទេ?" [អ្នកដែលមានព័ត៌មានផ្ទៃក្នុង]៖ "ថាតើវាត្រឹមត្រូវឬអត់ ខ្ញុំពិតជាមិនដឹងទេ ប៉ុន្តែខ្ញុំយល់ស្របជាមួយអ្នក។ យ៉ាងហោចណាស់ក្នុងសតវត្សទី២០ និងសូម្បីតែមុននេះទៅក្នុងសតវត្សទី១៩ និងទី១៨ ប្រវត្តិសាស្ត្រនៃពិភពលោកនេះត្រូវបានដឹកនាំភាគច្រើនដោយលោកខាងលិច និងតំបន់ខាងជើងនៃភពផែនដី"} \cite{4}។
Regarding the exact timeframe of the expected geophysical event, the insider offers his best guess: \textit{"...អារម្មណ៍ ហើយវាជាអារម្មណ៍ដ៏ជ្រាលជ្រៅមួយ គឺថាពួកគេត្រូវតែរៀបចំខ្លួនឯងឥឡូវនេះ... ខ្ញុំគិតថាពួកគេមានគំនិតច្បាស់លាស់អំពីពេលវេលាដែលវានឹងកើតឡើង... \textbf{ខ្ញុំមានអារម្មណ៍ខ្លាំងណាស់ថាវានឹងកើតឡើងក្នុងអាយុកាលរបស់ខ្ញុំ ប្រហែលក្នុងរយៈពេល 20 ឆ្នាំខាងមុខ}... ឥឡូវនេះយើងបានចូលទៅក្នុងរយៈពេលនោះហើយ ដែលព្រឹត្តិការណ៍ភូគព្ភសាស្ត្រនេះនឹងកើតឡើង នៅពេលដែលយើងពិចារណាអំពីរយៈពេលដ៏យូរដែលបានកន្លងផុតទៅចាប់តាំងពីព្រឹត្តិការណ៍ចុងក្រោយដែលបានកើតឡើងប្រហែល 11,500 ឆ្នាំមុន ហើយវាកើតឡើងជាវដ្តប្រហែលរៀងរាល់ 11,500 ឆ្នាំ។ ឥឡូវវាត្រូវកើតឡើងម្តងទៀត... ពួកគេយល់ថាវានឹងកើតឡើង។ ពួកគេមានចំណេះដឹងដែលជាក់លាក់ថាវានឹងកើតឡើង... ជាថ្មីម្តងទៀត វាជារឿងមួយក្នុងចំណោមរឿងទាំងនេះ - វានឹងមិនអាចទៅរួចទេប្រសិនបើពួកគេមិនដឹង។ ខ្ញុំមានន័យថា អ្នកឆ្លាតវៃបំផុតនៅលើពិភពលោកនឹងកំពុងធ្វើការឱ្យពួកគេលើបញ្ហានេះ"} \cite{4}។

នេះគឺជាសក្ខីកម្មដ៏អស្ចារ្យមួយដែលយើងគួរតែដឹងគុណយ៉ាងខ្លាំង។ ក្នុងការសម្ភាសន៍ អ្នកនិពន្ធក៏បានពិភាក្សាអំពីជំនឿរបស់គាត់ថាសង្គ្រាមលោកលើកទី១ និងលើកទី២គឺជាសង្គ្រាមដែលត្រូវបានរៀបចំឡើង ហើយថាបេសកកម្មអង់គ្លេស-សាសន៍សាក់សុងប្រហែលជាមានអាយុកាលជាច្រើនជំនាន់មកហើយ។ ឥឡូវនេះ វាបានកន្លងមក 15 ឆ្នាំហើយចាប់តាំងពីការសម្ភាសន៍ ដែលបានកើតឡើងនៅឆ្នាំ 2010។ មានរយៈពេល 5 ឆ្នាំទៀតនៅសល់ ដល់ពេលដែលការព្យាករណ៍របស់អ្នកខាងក្នុងអំពីព្រឹត្តិការណ៍ភូគព្ភសាស្ត្រក្នុងរយៈពេល 20 ឆ្នាំនឹងឈានដល់ទីបញ្ចប់។
\subsection{ចំណេះដឹងអាថ៌កំបាំងរបស់ពួកឌ្រុយីដអំពីគ្រោះមហន្តរាយ}

ចំណេះដឹងខាងលិចអំពីគ្រោះមហន្តរាយដែលកើតឡើងដដែលៗត្រូវបានរក្សាទុកយ៉ាងល្អ ហើយមិនមែនតែដោយពួកហ្វ្រីម៉ាសុនទេ។ ពួកឌ្រុយីដ ដែលជាវប្បធម៌សេល្ត៍បុរាណដែលមានឯកសារយ៉ាងល្អ មានអាយុកាលយ៉ាងហោចណាស់ ២៤០០ ឆ្នាំ \cite{7} បានបញ្ជូនចំណេះដឹងអំពីគ្រោះមហន្តរាយដែលកើតឡើងដដែលៗនៅលើផែនដី។ ឌ្រុយីដដែលគេស្គាល់ចុងក្រោយត្រូវបានគេជឿថាគឺប៊ែន ម៉ាក់ប្រេឌី។ នៅក្នុងឯកសារយោង "The Last Druid" ឆ្នាំ ១៩៩២ គាត់បានចែករំលែកព័ត៌មានអំពីចំណេះដឹងរបស់ពួកឌ្រុយីដ៖ \textit{"សណ្តាប់ធ្នាប់ដែលខ្ញុំអាចជាសមាជិកចុងក្រោយតាមបែបបទ បានកើតឡើងបន្ទាប់ពីគ្រោះមហន្តរាយឬគ្រោះថ្នាក់ដ៏ធំចុងក្រោយ ដែលប៉ះពាល់ដល់ពិភពលោក។ ឥឡូវនេះជាមួយនឹងផលប៉ះពាល់ដ៏ធំនិងគួរឱ្យភ័យខ្លាចទាំងនេះលើផែនដីដោយព្យុះអគ្គីសនីដ៏ធំ ការជាប់នៅក្នុងកន្ទុយរបស់អាចមនផ្កាយ ឬភ្លៀងអាចមនផ្កាយ អរិយធម៌ដូចដែលយើងស្គាល់ត្រូវបានបំផ្លាញយ៉ាងអស់អាត្មា... ចំណេះដឹងទាំងអស់បានចូលមកក្នុងវិស័យនៃសណ្តាប់ធ្នាប់ ប៉ុន្តែពួកគេពិបាកជាពិសេសអំពីតារាសាស្ត្រ ពីព្រោះពួកគេបានជួបប្រទះនឹងគ្រោះថ្នាក់ដ៏សំខាន់ជាច្រើន។ វាត្រូវបានគេគិតថាចំណេះដឹងពេញលេញអំពីតារាសាស្ត្រនឹងអាចឱ្យពួកគេទស្សន៍ទាយលក្ខខណ្ឌនៅពេលដែលគ្រោះថ្នាក់ទាំងនេះអាចកើតឡើង និងធ្វើសកម្មភាពខ្លះដើម្បីការពារខ្លួនឯង។ ប្រសិនបើអ្នកមើលឃើញស្នាដៃមេហ្គាលីធិកដ៏ធំនៅអៀរឡង់ អ្នកនឹងឃើញថាអ្វីដែលត្រូវបានពិពណ៌នាថាជាផ្លូវផ្នូរគឺពិតជាជម្រកគ្រាប់បែកនៃប្រភេទដ៏បុរាណ។ ពួកវាស្ថិតនៅខ្ពស់ជាងកម្រិតរលកយ៉ាងណាក៏ដោយ ហើយពួកវាក៏ផ្តល់ការការពារពីភ្លៀងអាចមនផ្កាយផងដែរ"} \cite{8,9}។

% វាក៏ត្រូវបានគេជឿថាស្ថាប័នហ្វ្រីម៉ាសុនខ្លួនឯងពិតជាមកពីពួកឌ្រុយីដ \cite{10}។
\section{ភស្តុតាងនៃការរៀបចំសម្រាប់គ្រោះមហន្តរាយរបស់ខាងលិចនាពេលបច្ចុប្បន្ន}

ដោយសារអំណាចដឹកនាំខាងលិចហាក់ដូចជាជឿថាគ្រោះមហន្តរាយភូមិសាស្ត្រសាកលនឹងកើតឡើងឆាប់ៗ យើងអាចរំពឹងថានឹងមានការរៀបចំយ៉ាងសំខាន់ដើម្បីការពារខ្លួនពួកគេពីព្រឹត្តិការណ៍បែបនេះ។ ហើយពិតជាមានភស្តុតាងនៅក្នុងដែនសាធារណៈនៃបណ្តាញធំទូលាយនៃមូលដ្ឋានក្រោមដីជ្រៅនៅទូទាំងប្រទេសខាងលិចជាច្រើន។ ខណៈដែលការដាក់ដំឡើងបែបនេះពិតជាអាចការពារអ្នករស់នៅក្នុងសង្គ្រាមនុយក្លេអ៊ែរ វាក៏អាចការពារពីគ្រោះមហន្តរាយធម្មជាតិប្រភេទផ្សេងៗផងដែរ។ ដោយផ្អែកលើសក្ខីកម្មរបស់អ្នកធំនៃសាមនីយដ្ឋានអង់គ្លេសពីគម្រោង Camelot \cite{4,6} វាហាក់ដូចជាសេណារីយ៉ូទាំងនេះមិនមែនជាលទ្ធភាពទេ ប៉ុន្តែជាផែនការដែលបានគ្រោងទុកជាមុន។ ហើយក៏គួរកត់សម្គាល់ផងដែរនូវចំនួនប្រាក់ដ៏ច្រើនសន្ធឹកសន្ធាប់ដែលនឹងត្រូវការដើម្បីសាងសង់ ប្រើប្រាស់បុគ្គលិក និងថែទាំមូលដ្ឋានទាំងនេះ ដែលស្របគ្នាយ៉ាងល្អជាមួយនឹងចំនួនដូចជាប្រាក់រាប់សែនពាន់លានដុល្លារដែលបាត់ពីរដ្ឋាភិបាលសហរដ្ឋអាមេរិកក្នុងរយៈពេល ១៨ ឆ្នាំ (ដែលនឹងត្រូវបានគ្របដណ្តប់នៅផ្នែកបន្ទាប់) \cite{11,12,13}។ ឧទាហរណ៍ផ្សេងទៀតនៃការរៀបចំសម្រាប់ព្រឹត្តិការណ៍ដែលអាចនាំឱ្យវិនាសអស់ រួមមានគម្រោងប័ណ្ណសារផ្សេងៗដូចជាឃ្លាំងគ្រាប់ពូជ និងចំណេះដឹង។
\subsection{មូលដ្ឋានក្រោមដី និងក្រោមសមុទ្ររបស់អាមេរិក}

ការស៊ើបអង្កេតសាធារណៈដ៏ច្រើនបំផុតអំពីមូលដ្ឋានក្រោមដីដែលខ្ញុំបានរកឃើញគឺមកពី Richard Sauder ដែលជាអ្នកស្រាវជ្រាវឯករាជ្យអាមេរិកាន់ ដែលបានបោះពុម្ពសៀវភៅជាច្រើនអំពីមូលដ្ឋានក្រោមដីជ្រៅ \cite{22}។ ការងាររបស់ Sauder មានការប្រមូលឯកសាររាជរដ្ឋាភិបាល និងផែនការ ការសិក្សាព័ត៌មានប្រវត្តិសាស្ត្រ និងបច្ចុប្បន្ន បច្ចេកវិទ្យា ការបង្កើតប្រភពព័ត៌មាន និងការចងក្រងសំណើពីអ្នកខាងក្នុង។ ការស្រាវជ្រាវរបស់ Sauder បង្ហាញថាមានបណ្តាញធំនៃមូលដ្ឋានក្រោមដី និងក្រោមសមុទ្រជ្រៅនៅក្នុង និងជុំវិញអាមេរិក និងទឹកដីរបស់វា (រូបភាព \ref{fig:4}) ដែលអាចមានជម្រៅយ៉ាងហោចណាស់ ៣ ម៉ាយ ហើយអាចត្រូវបានភ្ជាប់ដោយរថភ្លើងម៉ាញេទិចលើខ្សែស្រឡាយអណ្តាតក្រោមដីល្បឿនលឿន។ មូលដ្ឋានទាំងនេះត្រូវបានផ្តល់មូលនិធិដោយស្ងាត់កំបាំងតាមរយៈ \textit{"ល្បែងហិរញ្ញវត្ថុខ្ពស់ អន្តរជាតិ អន្តរស្ថាប័ន ការសំអាតលុយ"} ដែលដំណើរការដោយមនុស្សដដែលដែលជាម្ចាស់ក្រុមហ៊ុនសហរដ្ឋអាមេរិក \cite{22}។ ការងារបន្តដែលធ្វើឡើងលើវិសាលភាពនៃមូលដ្ឋានទាំងនេះដោយ Catherine Austin Fitts (ដែលការងាររបស់នាងត្រូវបានគ្របដណ្តប់នៅក្នុងផ្នែកបន្ទាប់) និងអ្នកសហការម្នាក់របស់នាងបានផលិតការប៉ាន់ប្រមាណនៃមូលដ្ឋានក្រោមដី និងក្រោមសមុទ្រអាមេរិក ១៧០ កន្លែង \cite{16,20}។

\begin{figure}[b]
\begin{center}
% \fbox{\rule{0pt}{2in} \rule{0.9\linewidth}{0pt}}
   \includegraphics[width=1\linewidth]{penta.jpg}
\end{center}
   \caption{តើមានអ្វីស្ថិតនៅក្រោមវិមានស្ថានភាពស និងអគារប៉េនតាហ្គោនពិតប្រាកដ? ច្បាស់ណាស់ថាជាបណ្តាញផ្លូវក្រោមដីជ្រៅមួយ (រូបភាព៖ \cite{31})។}
\label{fig:3}
\label{fig:onecol}
\end{figure}
\begin{figure*}[t]
\begin{center}
% \fbox{\rule{0pt}{2in} \rule{.9\linewidth}{0pt}}
\includegraphics[width=0.9\textwidth]{basescrop.png}
\end{center}
\caption{ផែនទីបង្ហាញទីតាំងពិតប្រាកដដែលការស្រាវជ្រាវរបស់សូដែរបានបង្ហាញថាមានមូលដ្ឋានក្រោមដី និងក្រោមសមុទ្រពិតប្រាកដ ព្រមទាំងប្រឡាយយានដ្ឋានក្រោមទឹកដែលចូលទៅក្នុងដីគោក។ សូដែរបាននិយាយថា \textit{"ខ្ញុំប្រាកដថាមានមូលដ្ឋានច្រើនជាងនេះទៅទៀត"} \cite{22}។}
\label{fig:4}
\end{figure*}

ខាងក្រោមនេះគឺជាកំណាព្យខ្លះពីប្រភពរបស់សូដែរដែលពិពណ៌នាអំពីទំហំនៃមូលដ្ឋានមួយចំនួន៖
\begin{flushleft}
\begin{enumerate}
    \item ឃេមភ៍ដាវីដ រដ្ឋម៉ារីឡង់៖ \textit{"ប្រភពព័ត៌មានរបស់ខ្ញុំបានជំរាបខ្ញុំថាផ្នែកក្រោមដីនៃឃេមភ៍ដាវីដមានទំហំធំធេងនិងស្មុគស្មាញខ្លាំងណាស់ ហើយមានអន្លង់សម្ងាត់ជាច្រើនគីឡូម៉ែត្រ ដែលធ្វើឲ្យសង្ស័យថាមានមនុស្សណាម្នាក់អាចចងចាំផែនទីទាំងមូលនៃកន្លែងនេះក្នុងចិត្តរបស់គាត់បាន"} \cite{22}។
    \item វិមានស្ថានវរសិទ្ធិ វ៉ាស៊ីនតោន ឌីស៊ី៖ \textit{"មិត្តភក្តិជិតស្និទ្ធម្នាក់របស់ខ្ញុំត្រូវបានយកទៅកន្លែងនេះក្នុងអំឡុងរដ្ឋាភិបាលលីនដុន ប៊ី ចនសុន នាទសវត្សរ៍ឆ្នាំ១៩៦០។ គាត់បានចូលក្នុងជណ្តើរយន្តនៅវិមានស្ថានវរសិទ្ធិ ហើយត្រូវបាននាំចុះក្រោម។ គាត់ជឿថាជណ្តើរយន្តនោះចុះដល់១៧ជាន់ក្រោមដី។ នៅពេលដែលទ្វារបើក គាត់ត្រូវបាននាំតាមច្រកដែលហាក់ដូចជាបាត់អស់ទៅឆ្ងាយ។ មានទ្វារនិងច្រកផ្សេងៗទៀតបើកចេញពីច្រកនោះ"} \cite{22}។ បង្ហាញក្នុងរូបភាព \ref{fig:3}។
   \item ហ្វត៍ មីដ, ម៉ារីឡែន - ពីប្រភពមួយដែលបានជាប់ជំពប់ចូលទៅក្នុង "បាតផ្ទះ" ក្នុងទសវត្សរ៍ឆ្នាំ 1970៖ \textit{"ខ្ញុំបានបើកទ្វារហើយឃើញជណ្តើរធ្លាក់ចុះក្រោម។ ខ្ញុំបានដើរទៅដល់គែមហើយសម្លឹងមើលចុះក្នុងចន្លោះរ៉ាយ។ ខ្ញុំមិនបានរាប់ចំនួនជាន់ចុះក្រោមទេ ប៉ុន្តែខ្ញុំមានអារម្មណ៍ថាវាប្រហែល 15-20 ជាន់... ខ្ញុំបានចុះមកជាន់មួយហើយឃើញមានទ្វារមួយ... ខ្ញុំបានបើកទ្វារហើយលូកក្បាលចូលទៅមើលស្តាំឆ្វេង ឃើញអាងជ្រៅដែលលាតសន្ធឹងអស់ភ្នែកទាំងសងខាង។ វាពិតជាឆ្ងាយជាងតំបន់ដែលគ្របដណ្តប់ដោយអាគារនិងកន្លែងចតរថយន្តនៅផ្ទៃដី។ មានទ្វារតាមបណ្តោយជញ្ជាំងម្ខាងៗ ដែលឃ្លាតពីគ្នាប្រហែល 30-40 ហ្វីត... ខ្ញុំបានសម្រេចចិត្តពិនិត្យមើលជាន់ពីរបីទៀត ដូច្នេះខ្ញុំបានចុះមកមួយជាន់ទៀត... ហើយឃើញប្លង់ដូចគ្នា... ខ្ញុំបានចុះមកមួយជាន់ទៀតហើយមើលឃើញដូចគ្នានឹងជាន់ពីរដំបូង"} \cite{22}។
\end{enumerate}
\end{flushleft}

\begin{figure}[t]
\begin{center}
% \fbox{\rule{0pt}{2in} \rule{0.9\linewidth}{0pt}}
   \includegraphics[width=1\linewidth]{undersea.jpg}
\end{center}
   \caption{រូបភាពបង្ហាញអំពីមូលដ្ឋានក្រោមសមុទ្រ ដោយ Walter Koerschner។ គាត់ជាអ្នកគូររូបភាពសម្រាប់ក្រុមមូលដ្ឋានក្រោមសមុទ្រ Rock-Site របស់កងទ័ពសមុទ្រអាមេរិកនៅមជ្ឈមណ្ឌលអាវុធនៅ China Lake, California ក្នុងទសវត្សរ៍ឆ្នាំ 1960។ ប្រភពមួយរបស់ Sauder បានបង្ហាញថាមានមូលដ្ឋានក្រោមដីមួយជ្រៅមួយម៉ាយនៅ China Lake \cite{22,23}។}
\label{fig:5}
\label{fig:onecol}
\end{figure}

សូដារបានទទួលការបញ្ជាក់ពីរថភ្លើងលើកដោយម៉ាញេទិចក្រោមដីដែលអាចឈានដល់ល្បឿន ២,០០០ ម៉ាយក្នុងមួយម៉ោង មូលដ្ឋានដែលសាងសង់នៅក្រោមជាន់មហាសមុទ្រ (រូបភាព \ref{fig:5}) និងអូវែលក្រោមទឹកដែលនាំចូលទៅក្នុងដី។ ចំពោះការបញ្ជាក់មួយអំពីមូលដ្ឋានក្រោមទឹកនៅឈូងសមុទ្រម៉ិកស៊ិក សូដារបាននិយាយថា \textit{"ប្រហែលពាក់កណ្តាលឆ្នាំបន្ទាប់ពីការបោះពុម្ពផ្សាយរបស់ Underwater and Underground Bases ខ្ញុំត្រូវបានទាក់ទងដោយបុរសម្នាក់ដែលបាននិយាយថាគាត់មានចំណេះដឹងអំពីគម្រោងក្រោមទឹកដ៏ចម្លែកមួយ... គាត់បានបញ្ជាក់ថាគម្រោងនេះស្ថិតនៅក្រោមជាន់សមុទ្រនៃឈូងសមុទ្រម៉ិកស៊ិក ហើយ Parsons គឺជាអ្នកកាត់កង។ គាត់បានបន្តថា Parsons បានទិញឧបករណ៍ពិសេសមួយចំនួនដែលមានបំណងប្រើប្រាស់នៅក្រោមជាន់សមុទ្រជម្រៅ ២,៨០០ ហ្វីត... ឧបករណ៍នេះមានលក្ខណៈពិសេសដែលធ្វើឱ្យយើងអាចសន្និដ្ឋានបានថាមានមនុស្សរស់នៅក្នុងកន្លែងដែលវាត្រូវបានដំឡើង"} \cite{22}។
\begin{figure}[t]
\begin{center}
% \fbox{\rule{0pt}{2in} \rule{0.9\linewidth}{0pt}}
   \includegraphics[width=1\linewidth]{sub.jpg}
\end{center}
   \caption{រូបភាពបង្ហាញអំពីអាងជ្រៅសមុទ្រក្រោមទឹក ដោយ Walter Koerschner \cite{22,23}។}
\label{fig:6}
\label{fig:onecol}
\end{figure}
\begin{figure}[t]
\begin{center}
% \fbox{\rule{0pt}{2in} \rule{0.9\linewidth}{0pt}}
   \includegraphics[width=1\linewidth]{iran.jpeg}
\end{center}
   \caption{វីដេអូផ្លូវការមួយរបស់អ៊ីរ៉ង់បង្ហាញអំពី "ទីក្រុងមីស៊ីល" ក្រោមដីរបស់ពួកគេ \cite{39,40}។}
\label{fig:12}
\label{fig:onecol}
\end{figure}
ប្រសិនបើពិតជាមានបណ្តាញសម្ងាត់ធំទូលាយនៃមូលដ្ឋានក្រោមដី និងក្រោមសមុទ្រចំនួន ១៧០+ ដែលជីកជ្រៅរាប់គីឡូម៉ែត្រក្រោមផ្ទៃដីដែលយើងឈរនៅលើ ភ្ជាប់គ្នាដោយរថភ្លើងម៉ាញ៉េទិចល្បឿនខ្ពស់ក្នុងបំពង់សុញ្ញការ ហើយផ្តល់មូលនិធិដោយប្រាក់ឈ្នួលរបស់យើង មនុស្សជាតិសម័យនេះនឹងស្ថិតក្នុងសភាពភាពល្ងង់ខ្លៅយ៉ាងជ្រៅ មិនដឹងថាមានអ្វីក្រោមជើងពួកគេប៉ុណ្ណោះទេ ថែមទាំងមិនដឹងពីអ្វីដែលកំពុងរង់ចាំពួកគេនាពេលអនាគតដ៏ខ្លីខាងមុខផងដែរ ខណៈពួកគេជញ្ជក់យកសេចក្តីថ្លែងការណ៍ឥតប្រយោជន៍ពីអ្នកនយោបាយដែលគ្រប់គ្រងពួកគេ។

កំណត់សម្គាល់បន្ថែម - អត្ថិភាពនៃបណ្តាញអណ្តូងធំៗក្រោមដីត្រូវបានបង្ហាញយ៉ាងច្បាស់នៅក្នុងជម្លោះនៅដើម្បីកណ្តាល (អណ្តូងរបស់ Hamas ក្រោម Gaza Strip \cite{38} និង "ទីក្រុងមីស៊ីល" ក្រោមដីរបស់អ៊ីរ៉ង់ (រូបភាព \ref{fig:12}) \cite{39,40})។ ទាំងនេះគួរឱ្យយើងមិនសង្ស័យអំពីលទ្ធភាពនៃការសាងសង់ និងអត្ថិភាពពិតនៃស្ថាបត្យកម្មបែបនេះ។ វាក៏គួរឱ្យយើងឆ្ងល់ថា តើប្រទេសដែលមានលុយកាក់ច្រើនជាងនេះបានសាងសង់អ្វីខ្លះក្នុងអំឡុងពេលដូចគ្នានេះ។
\subsection{ភស្តុតាងបន្ថែមសម្រាប់ការរៀបចំប៊ុងកេរ និងគ្រោះមហន្តរាយ}

\begin{figure}[t]
\begin{center}
% \fbox{\rule{0pt}{2in} \rule{0.9\linewidth}{0pt}}
   \includegraphics[width=1\linewidth]{tyrol.jpg}
\end{center}
   \caption{ប៊ុងកឺនៅតំបន់ទីរ៉ុលខាងត្បូង ប្រទេសស្វ៊ីស។ ប្រទេសស្វ៊ីស ដែលគ្របដណ្តប់លើជួរភ្នំអាល់ប៊ែរអឺរ៉ុប ត្រូវបានគេស្គាល់ថាមានប៊ុងកឺលាក់ខ្លួនយ៉ាងឆ្លាតវៃនៅលើភ្នំ \cite{32}។}
\label{fig:7}
\label{fig:onecol}
\end{figure}

\begin{figure}[t]
\begin{center}
% \fbox{\rule{0pt}{2in} \rule{0.9\linewidth}{0pt}}
   \includegraphics[width=1\linewidth]{svalbard.jpg}
\end{center}
   \caption{ឃ្លាំងគ្រាប់ពូជសកលស្វ័យប្រវត្តិនៅស្វាលបាដ (Svalbard) ប្រទេសន័រវែស ដែលមានគ្រាប់ពូជច្រើនជាងមួយលាន \cite{24}។ យើងត្រូវតែឆ្ងល់ថា តើគ្រោះមហន្តរាយប្រភេទណាដែលនឹងត្រូវការប្រើប្រាស់វា។}
\label{fig:8}
\label{fig:onecol}
\end{figure}

មានភស្តុតាងជាច្រើនផ្សេងទៀតនៃការរៀបចំសម្រាប់គ្រោះមហន្តរាយនៅជុំវិញពិភពលោក ក្រៅពីមូលដ្ឋានរាជវង្សក្រោមដីរបស់អាមេរិក។ ន័រវែស ស្វ៊ីស ស៊ុយអែត និងហ្វាំងឡង់គឺជាឧទាហរណ៍ដ៏ល្អ៖

\begin{flushleft}
\begin{enumerate}
    \item គម្រោង Camelot បានចែករំលែកសក្ខីកម្មដែលពាក់ព័ន្ធពីអ្នកនយោបាយន។រូវែស ម្នាក់ \cite{25,26} ដែលអត្តសញ្ញាណរបស់គាត់ត្រូវបានផ្ទៀងផ្ទាត់ ប៉ុន្តែគេរក្សាជាការសម្ងាត់។ គាត់បានអះអាងថា ន័រវែសមានមូលដ្ឋានក្រោមដីចំនួន ១៨ ដែលធំទូលាយ ហើយន័រវែស (រួមជាមួយអ៊ីស្រាអែល និង "ប្រទេសជាច្រើនផ្សេងទៀត") កំពុងសាងសង់មូលដ្ឋានទាំងនេះដើម្បីរៀបចំសម្រាប់គ្រោះមហន្តរាយធម្មជាតិមួយចំនួន។ Richard Sauder ក៏បានទទួលសក្ខីកម្មមួយពីបុរសម្នាក់ដែលបានចូលទៅក្នុងមូលដ្ឋានក្រោមដីដ៏ធំមួយ ដែលសាងសង់នៅក្នុងភ្នំដែលបានជីករូងនៅន័រវែស \cite{22}។
    \item ប្រទេសស្វីស ត្រូវបានគេស្គាល់ថាមានប៉មសំរាប់ជ្រកកោនពីនុយក្លេអ៊ែរជាច្រើន ដែលសាងសង់នៅលើជួរភ្នំអាល់ប \ref{fig:7}។ ចំនួននេះមានច្រើនជាង ៣៧០,០០០ ដែលគ្រប់គ្រាន់ដើម្បីជ្រកកោនដល់ប្រជាពលរដ្ឋទាំងអស់ \cite{27}។
    \item ប្រទេសស៊ុយអែត និងហ្វាំងឡង់ មានប៉មជ្រកកោនគ្រប់គ្រាន់សម្រាប់ប្រជាពលរដ្ឋនៅក្នុងទីក្រុងធំៗគ្រប់ខេត្ត \cite{27}។
\end{enumerate}
\end{flushleft}

អ្នកជំនួញដ៏អស្ចារ្យនៅ Silicon Valley ក៏ហាក់ដូចជាដឹងដែរ។ តាមរយៈការរាយការណ៍ \textit{"Reid Hoffman អ្នកជំនួញរួមបង្កើត LinkedIn និងជាអ្នកវិនិយោគល្បីម្នាក់ បានប្រាប់ទៅកាសែត The New Yorker កាលពីដើមឆ្នាំនេះថា គាត់ប៉ាន់ប្រមាណថាមានអ្នកក្លាហានជាង ៥០\% នៅ Silicon Valley បានទិញធានារ៉ាប់រង "អាប៉ូកាលីប" មួយចំនួន ដូចជាបន្ទាយក្រោមដី... តាម Jim Dobson អ្នកសរសេរសម្រាប់ Forbes មានអ្នកក្លាហានជាច្រើនមានយន្តហោះឯកជន "ដែលត្រៀមរួចរាល់ដើម្បីចេញដំណើរភ្លាមៗ"។ ពួកគេក៏មានម៉ូតូ អាវុធ និងម៉ាស៊ីនភ្លើងផងដែរ"} \cite{28}។

មានគម្រោងអាគារសារពើរដ៏ធំៗជាច្រើនផងដែរ ដូចជា Global Knowledge Vault ដែលដំណើរការដោយ Arch Mission Foundation \cite{29} និង Svalbard Global Seed Vault \cite{30} ដែលហាក់ដូចជាកំពុងត្រៀមខ្លួនដើម្បីរក្សាទុកទ្រព្យសម្បត្តិសំខាន់ៗរបស់មនុស្សជាតិ ក្នុងករណីមានគ្រោះមហន្តរាយដែលអាចនាំអោយផុតពូជ។
\begin{figure*}[t]
\begin{center}
% \fbox{\rule{0pt}{2in} \rule{.9\linewidth}{0pt}}
\includegraphics[width=0.9\textwidth]{govcrop2.png}
\end{center}
   \caption{ចំណូល ការចំណាយ និងការចំណាយលើមូលដ្ឋានក្រោមដីសម្ងាត់របស់រដ្ឋាភិបាលសហរដ្ឋអាមេរិកពីឆ្នាំ ១៩៩៨ ដល់ ២០២៣ \cite{19}.}
   \label{fig:9}
\end{figure*}
\section{យន្តការផ្តល់មូលនិធិបែបប្រជាធិបតេយ្យសម្រាប់មូលដ្ឋានក្រោមដីដ៏ធំធេង}

ដូច្នេះតើរបៀបដែលបណ្តាញឆ្លងទ្វីបដ៏ធំធេងនៃមូលដ្ឋានក្រោមដី និងក្រោមសមុទ្រជាង ១៧០ ត្រូវបានផ្តល់មូលនិធិ ខណៈដែលរក្សាអ្នកបំរើឥណទាននៅក្នុងភាពងងឹត? មានដានក្រដាសមួយដែលអាចផ្តល់ឱ្យយើងនូវគំនិតនៃទំហំប្រាក់ដែលបានចំណាយលើគម្រោងទាំងនេះ និងប្រភពដែលវាមកពី។ នៅឆ្នាំ ២០១៧ កាតរីន អូស្ទីន ហ្វីតថ៍ ជាអ្នកធានាគារវិនិយោគអាមេរិកាំង និងជាមុនអ្នកមន្ត្រីសាធារណៈក្នុងអំឡុងពេលរដ្ឋាភិបាលប៊ូស្ស និងម៉ាក ស្គីដម័រ ជាអ្នកសេដ្ឋកិច្ចនៃសាកលវិទ្យាល័យរដ្ឋមីស៊ីហ្គែន បានរកឃើញការចំណាយដែលមិនបានអនុញ្ញាតចំនួន ២១ ទ្រីលានដុល្លារអាមេរិកក្នុងរដ្ឋាភិបាលសហរដ្ឋអាមេរិកក្នុងឆ្នាំហិរញ្ញវត្ថុ ១៩៩៨-២០១៥ \cite{11,12,13}។

យោងតាមរបាយការណ៍របស់ពួកគេ \textit{"នៅថ្ងៃទី ៧ ខែតុលា ឆ្នាំ ២០១៦ រ៉ូយ៉ឺតឺរបានបោះពុម្ពអត្ថបទមួយដោយស្កុត ផោលត្រូ (២០១៦) ដែលរាយការណ៍ថាក្នុងឆ្នាំហិរញ្ញវត្ថុ ២០១៥ កងទ័ពបានធ្វើការលៃតម្រូវគណនេយ្យដែលមិនមានការគាំទ្រចំនួន ៦.៥ ទ្រីលានដុល្លារ 'ដើម្បីបង្កើតឱ្យមានអារម្មណ៍ថាសៀវភៅរបស់ខ្លួនមានតុល្យភាព។' ដោយផ្អែកលើការថាថវិកាមូលដ្ឋានទូទៅរបស់កងទ័ពនៅឆ្នាំនោះគឺ ១២២ ពាន់លានដុល្លារ នេះគឺជាការបង្ហាញដ៏អស្ចារ្យ... ក្រសួងការពារជាតិបានធ្វើឱ្យមានចំណងជើងធំៗក្នុងព័ត៌មានពីច្រើនឆ្នាំមុននៅថ្ងៃទី ១០ ខែកញ្ញា ឆ្នាំ ២០០១ នៅពេលដែលរដ្ឋមន្ត្រីការពារជាតិដូណាល់ដ៍ រ៉ាមស្ហ៊្វែលដ៍បានថ្លែងក្នុងការប្រជុំសភា (ស៊ី-ស៊្ប៉ាន, ២០១៤) ថាក្រសួងការពារជាតិបានបាត់បង់ដាននៃប្រតិបត្តិការចំនួន ២.៣ ទ្រីលានដុល្លារ... ការទទួលស្គាល់នេះបានធ្វើឱ្យមានចំណងជើងព័ត៌មាននៅថ្ងៃនោះ ប៉ុន្តែត្រូវបានភ្លេចនៅថ្ងៃបន្ទាប់ នៅពេលដែលវិប្បដិសារីថ្ងៃទី ១១ កញ្ញាបានទាក់ទាញការយកចិត្តទុកដាក់ពីទូទាំងពិភពលោក... នៅពេលដែលអ្នកវិទ្យាសាស្ត្រម៉ាក ស្គីដម័របានដឹងអំពីប្រតិបត្តិការរបស់កងទ័ពដែលមិនអាចផ្ទៀងផ្ទាត់បានចំនួន ៦.៥ ទ្រីលានដុល្លារ គាត់បានទាក់ទងទៅកាន់មិss ហ្វីតថ៍ ហើយពួកគេបានយល់ព្រមនៅរដូវផ្ការីកឆ្នាំ ២០១៧ ដើម្បីធ្វើការជាមួយគ្នាដើម្បីកំណត់អត្ថបទរបាយការណ៍រដ្ឋាភិបាលផ្សេងទៀតដែលបង្ហាញពីប្រតិបត្តិការដែលមិនអាចផ្ទៀងផ្ទាត់បានចំនួនធំជាមួយនឹង HUD និងក្រសួងការពារជាតិ។ ក្នុងអំឡុងពេល ៦ ខែបន្ទាប់ ស្គីដម័រ ហ្វីតថ៍ និងក្រុមតូចមួយនៃនិស្សិតបញ្ចប់ការសិក្សាបានប្រមូលឯកសារផ្លូវការរបស់រដ្ឋាភិបាលដែលក្នុងនោះមានប្រតិបត្តិការដែលមិនអាចឯកសារបានសរុបចំនួន ២១ ទ្រីលានដុល្លារត្រូវបានកំណត់អំឡុងពេលឆ្នាំ ១៩៩៨-២០១៦"} \cite{12}។
ក្នុងអំឡុងពេល ១៨ ឆ្នាំពីឆ្នាំ ១៩៩៨ ដល់ ២០១៥ ប្រាក់ចំណូលរបស់រដ្ឋាភិបាលសហរដ្ឋអាមេរិកដែលបានប្រកាសជាសាធារណៈមានតែ ៤០,៨ ទ្រីលានដុល្លារ \cite{15}។ នេះបង្ហាញថាចំនួនលើសពីពាក់កណ្តាលនៃប្រាក់ចំណូលរដ្ឋាភិបាលត្រូវបានចំណាយសម្ងាត់លើមូលដ្ឋានក្រោមដី បន្ថែមលើការចំណាយសាធារណៈដែលបានទទួលស្គាល់។ ការចំណាយសម្ងាត់នេះកើតឡើងនៅលើភាពខ្វះខាតថវិកាដែលបានដំណើរការអស់រយៈពេលយូរ ហើយវាប្រហែលជាមិនត្រឹមតែបន្តរហូតមកដល់សព្វថ្ងៃនេះប៉ុណ្ណោះទេ ប៉ុន្តែក៏មានមុនឆ្នាំ ១៩៩៨ ផងដែរ។ នេះបញ្ជាក់ថាចំនួនសរុបដែលបានចំណាយលើមូលដ្ឋានទាំងនេះគឺច្រើនជាង ២១ ទ្រីលានដុល្លារ។ ការអនុវត្តសមាមាត្រចំណាយសម្ងាត់ដូចគ្នាទៅនឹងរយៈពេលពីឆ្នាំ ២០១៦ ដល់ ២០២៣ បង្ហាញពីចំនួនសរុប ៣៦,៦ ទ្រីលានដុល្លារដែលបានចំណាយចាប់តាំងពីឆ្នាំ ១៩៩៨។

នៅឆ្នាំ ២០២១ លោក Mark Skidmore បានបោះពុម្ពផ្សាយការអាប់ដេតលើការស្រាវជ្រាវនេះ ដែលទាក់ទងនឹងការប្រកាសរបស់ Bloomberg ថាក្នុងអំឡុងឆ្នាំហិរញ្ញវត្ថុ ២០១៧-១៩ ក្រសួងការពារជាតិបានកត់ត្រាការលៃតម្រូវគណនេយ្យដ៏អស្ចារ្យចំនួន ៩៤,៧ ទ្រីលានដុល្លារ \cite{17,18}។ បើយើងយកមកពិចារណានូវការបោកប្រាស់ដុល្លារសហរដ្ឋអាមេរិកតាមរយៈប្រព័ន្ធធនាគារកណ្តាលដែលបានកើតឡើងអស់ជាងមួយសតវត្សចាប់តាំងពីការបង្កើតធនាគារសហព័ន្ធនៅឆ្នាំ ១៩១៣ \cite{37} វាច្បាស់ណាស់ថាគណនេយ្យសាធារណៈទាំងអស់គ្រាន់តែជាពាក្យសំដីគ្មានន័យ ហើយរូបិយប័ណ្ណនិងរដ្ឋាភិបាលសហរដ្ឋអាមេរិកគ្រាន់តែជាប្រព័ន្ធចែកចាយធនធានដែលម្ចាស់រាជវង្សអាចយកចេញ (ឬផ្ទុយទៅវិញ ចាក់ចេញ) តាមដែលពួកគេចង់បាន។
\section{ពូជពង្សនៃយូវ៉េ៖ អត្តសញ្ញាណនៃព្រះមហាក្សត្រខាងលិចដែលស្ថិតនៅក្នុងស្រមោល}

ដូច្នេះ តើអ្នកណាដែលពិតជាកំពុងដឹកនាំ? យើងមិនអាចដឹងច្បាស់បានទេ ពីព្រោះព្រះមហាក្សត្រខាងលិចនៃដើមទុននៅតែលាក់ខ្លួននៅក្នុងស្រមោល។ ខណៈដែលមានទ្រឹស្ដីផ្សេងៗគ្នា ចាប់ពីតួអង្គសាធារណៈរហូតដល់អាយាដែលមកពីក្រៅផែនដី ចម្លើយល្អបំផុតដែលខ្ញុំមានសម្រាប់សំណួរនេះស្ថិតនៅក្នុងការងារជីវិតរបស់អ្នកសរសេរប្លក់អនាមិកម្នាក់ដែលប្រើឈ្មោះបំភ័ន្ត "អាម៉ាល់ឡូឡា"។ ការងាររបស់គាត់ជាការសំយោគដ៏ធំទូលាយនៃអ្នកនិពន្ធជាង ២០ នាក់ និងឯកសារ "ដែលមិនអាចជំនួសបាន" ជាង ៥០ ប្រកាសដែលគ្របដណ្ដប់លើប្រធានបទនៃប្រវត្តិសាស្ត្របុរាណ និងសម័យទំនើប និមិត្តសញ្ញាអាថ៌កំបាំង និងនយោបាយខាងលិច \cite{33,34}។ ខ្ញុំអាចពិពណ៌នាការងាររបស់គាត់ថាជា "ព្យាករណ៍" ទាក់ទងនឹងគ្រោះមហន្តរាយភូមិវិទ្យាដែលកំពុងខិតខំចូលមកដល់ - វាមានភាពគ្របដណ្ដប់ខ្លាំងជាងរបស់ខ្ញុំ។

អាម៉ាល់ឡូឡាបានកំណត់អត្តសញ្ញាណក្រុមនយោបាយខាងលិចបីក្រុម ដែលគាត់ហៅថា "ពូជពង្សនៃយូវ៉េ" ដែលមានចំណេះដឹងអំពី "យុគសម័យចុងក្រោយ" - គ្រោះមហន្តរាយដែលកើតឡើងឡើងវិញនៅលើផែនដី។ គាត់ជឿថាក្រុមទាំងបីនេះរួមគ្នាគ្រប់គ្រងប្រទេសខាងលិចនាពេលបច្ចុប្បន្ន ប៉ុន្តែបែងចែកពួកគេជាបីក្រុមផ្សេងគ្នាដោយផ្អែកលើប្រភពដើម និងអត្តសញ្ញាណប្រវត្តិសាស្ត្រផ្សេងៗគ្នា ការខ្វែងគំនិតដែលអាចមានក្នុងអតីតកាល និងភាពខុសគ្នាដែលអាចមើលឃើញនៅក្នុងប្រព័ន្ធតម្លៃ និងសកម្មភាពរបស់ពួកគេ។
ក្រុមទាំងបីអាចត្រូវបានចាត់ថ្នាក់យ៉ាងទូទៅដូចខាងក្រោម៖

\begin{flushleft}
\begin{enumerate}
    \item \textbf{អ្នកធានាគារ}: អភិជនរ៉ូមបុរាណ ដែលបានប្រែក្លាយជាគណៈកម្មាភិបាល Knights Templar និងអំណាចខាងជើងនៃអ្នកស្មោះស្ម័យ Freemasons នៅអាមេរិក។
    \item \textbf{អ្នកគិតគូរ}: ក្រុម Rosicrucians និងអ្នកស្មោះស្ម័យ Freemasons ខាងត្បូងនៃអាមេរិក។
    \item \textbf{ពួកយេស៊ូអ៊ីត និងព្រះអាទិទេពខ្មៅ}: ក្រុមដែលជាកូនចៅរបស់ Jove នៅក្នុងសាសនាកាតូលិករ៉ូម។
\end{enumerate}
\end{flushleft}
ថ្ងៃនេះ ក្រុមបក្សទាំងបីនេះរួមគ្នាបង្កើតជាអ៊ីលូមីណាទីអឺរ៉ុប ហ្វ្រីម៉ាសុន និងស៊ីអាយអេ។ ដូចដែលអាម៉ាល់ឡុឡាបានពិពណ៌នា \textit{"បច្ចុប្បន្ននេះ នៅចុងក្រោយនៃសម័យកាល កូនចៅរបស់ជូវបានលាក់ខ្លួនយ៉ាងល្អពីការត្រួតពិនិត្យរបស់សាធារណជន នៅពីក្រោយការអនុញ្ញាតិត្រឹមតែអ្នកដែលត្រូវដឹងប៉ុណ្ណោះ ដែលមិនរាប់បញ្ចូលសូម្បីតែប្រធានាធិបតីសហរដ្ឋអាមេរិកកំពុងកាន់តំណែង។ ម៉្យាងទៀត ពួកគេបានធ្វើឱ្យដូចគ្នានឹងសិល្បៈនៃការលាក់ខ្លួនពីការត្រួតពិនិត្យរបស់សាធារណជន។ \textbf{កូនចៅរបស់ជូវមិនត្រឹមតែគ្រប់គ្រងយោធា និងរដ្ឋាភិបាលសហរដ្ឋអាមេរិកប៉ុណ្ណោះទេ ប៉ុន្តែតាមរយៈអំណាចនៃរូបិយប័ណ្ណដែលបង្កើតឡើងដោយរដ្ឋាភិបាល ក្រុមហ៊ុនធំៗ និងទម្រង់រដ្ឋាភិបាលសាធារណរដ្ឋដែលពួកគេបានបង្កើត (ដោយដឹងថាអ្នកនយោបាយនឹងងាយនឹងខូចខាត ហើយដូច្នេះត្រូវបានគ្រប់គ្រង) ពួកគេគ្រប់គ្រងលើពិភពលោកខាងលិចទាំងមូល}"} \cite{33,34}។

\begin{figure}[t]
\begin{center}
% \fbox{\rule{0pt}{2in} \rule{0.9\linewidth}{0pt}}
   \includegraphics[width=1\linewidth]{illuminati.jpg}
\end{center}
   \caption{តើអ្នកណាជាកូនចៅរបស់យ៉ូវ? (រូបភាព៖ \cite{35})}
\label{fig:10}
\label{fig:onecol}
\end{figure}

\begin{figure}[t]
\begin{center}
% \fbox{\rule{0pt}{2in} \rule{0.9\linewidth}{0pt}}
   \includegraphics[width=1\linewidth]{pike.jpg}
\end{center}
   \caption{ថ្មភ្នំ Pike Peak ដ៏ល្បីល្បាញ ដែលត្រូវបានបន្លាយពណ៌ក្រហម រួមជាមួយនឹងរូបរាងដីធ្លានៃភាគខាងលិចសហរដ្ឋអាមេរិក \cite{36}។ តើសហរដ្ឋអាមេរិកពិតជាត្រូវបានបង្កើតឡើងដើម្បីគ្រប់គ្រងទីតាំងនេះមែនទេ?}
\label{fig:11}
\label{fig:onecol}
\end{figure}

តាមអាម៉ាល់ឡា បានចែងថាមនុស្សទាំងនេះធ្វើជាការប្រមាថសាសនា ប្រើប្រាស់គម្ពីរបរិសុទ្ធនៅក្នុងសាសនាធំៗរបស់ពិភពលោកដើម្បីអោយបានផលប្រយោជន៍ដល់ពួកគេ ហើយប្រើនិមិត្តសញ្ញាយ៉ាងខ្លាំងក្លា។ លើសពីនេះទៅទៀត ពួកគេគ្មានមេត្តាចិត្តចំពោះសត្រូវរបស់ពួកគេឡើយ៖ \textit{"\textbf{ក្នុងរយៈពេលជាង ២៦០០ឆ្នាំ ពួកគេបានលុបបំបាត់ជាប្រព័ន្ធដល់អ្នកដទៃទៀតដែលមានចំណេះដឹងជាក់លាក់អំពីពេលវេលាចុងក្រោយ។ ហើយក្នុងន័យនេះ ខ្ញុំមិនមែនសំដៅតែលើពួកឌ្រុយីដ ពួកគប្បីជ្វីហ្វ អេហ្ស៊ីបបុរាណ ពួកអារ៉ាប់ និងពួកឥណ្ឌាអាថ៌កំបាំងនោះទេ ប៉ុន្តែពួកអ៊ិលឡង់ហ្គេតខ្លួនវែងនៅអាមេរិកខាងត្បូង និងពួកបូជាចារ្យម៉ាយ៉ានៅអាមេរិកកណ្តាលផងដែរ។ ហើយភស្តុតាងដែលពួកគេបានផ្តួលចោលប្រជាជនមួយដែលធ្លាប់រីកចម្រើននៅអាមេរិកខាងជើងដើម្បីអោយរក្សាទឹកដីនេះជាទឹកដីនៃពេលវេលាចុងក្រោយនោះ ពិតជាមានភាពច្រើនលើសលប់។ ការបំផ្លាញជាតិសាសន៍អាមេរិកាំង "ឥណ្ឌា" គ្រាន់តែជាប្រតិបត្តិការសម្អាតបន្តិចបន្តួចប៉ុណ្ណោះ}"} \cite{33,34}។
អាម៉ាល់ឡា ក៏ជឿដែរថា គម្រោង "សហរដ្ឋអាមេរិក" ទាំងមូលត្រូវបានអនុវត្តឡើងដើម្បីធានាការគ្រប់គ្រងលើ "ភ្នំព្រះរាជទ្រព្យពីកស៊ីល" ដែលជាជួរភ្នំព្រះរាជទ្រព្យនៅក្នុងជួរភ្នំរូគីដែលផ្តល់នូវការការពារល្អបំផុតពីគ្រោះមហន្តរាយភូគព្ភសាស្ត្រ (រូបភាព \ref{fig:11})។ យោងតាមអាម៉ាល់ឡា \textit{"មុន ក្នុងអំឡុង និងបន្ទាប់ពីអ្វីដែលយើងគិតថាជាសង្គ្រាមស៊ីវិល អ្នកធនាគារ និងអ្នកគិតបានប្រយុទ្ធមិនមែនសម្រាប់ការគ្រប់គ្រងលើសហរដ្ឋអាមេរិកទេ ប៉ុន្តែសម្រាប់ភ្នំព្រះរាជទ្រព្យពីកស៊ីល ដែលជាភ្នំព្រះរាជទ្រព្យដ៏ប្លែកបំផុតមួយនៅលើពិភពលោក... មិនមានភ្នំព្រះរាជទ្រព្យពីកស៊ីលផ្សេងទៀតនៅកម្ពស់ខ្ពស់បែបនេះ និងនៅឆ្ងាយពីឆ្នេរសមុទ្រនៅកន្លែងណាផ្សេងទៀតនៅលើពិភពលោកឡើយ។ វាជាទីតាំងល្អបំផុតសម្រាប់រស់រានពីការផ្លាស់ប្តូរស្រទាប់ផែនដី"} \cite{33,34}។ ការស្រាវជ្រាវរបស់អាម៉ាល់ឡាបានបង្ហាញថាមានប្រព័ន្ធអណ្តូងក្រោមដីដ៏ធំទូលាយមួយត្រូវបានសាងសង់នៅក្រោម និងជុំវិញតំបន់នេះសព្វថ្ងៃ \cite{36}។

\section{សេចក្តីសន្និដ្ឋាន}

នៅក្នុងអត្ថបទនេះ ខ្ញុំបានរៀបរាប់យ៉ាងលម្អិតអំពីសក្ខីកម្មផ្សេងៗដែលបង្ហាញថាអភិជនខាងលិចបានរក្សាចំណេះដឹងអំពីគ្រោះមហន្តរាយដែលកើតឡើងដដែលៗរបស់ផែនដីអស់រយៈពេលរាប់ពាន់ឆ្នាំ ជឿថាគ្រោះមហន្តរាយមួយទៀតនឹងកើតឡើងឆាប់ៗ បានសាងសង់ជម្រកក្រោមដីដ៏ធំទូលាយដើម្បីរៀបចំសម្រាប់ព្រឹត្តិការណ៍បែបនេះ និងកំពុងធ្វើផែនការយកប្រយោជន៍ពីព្រឹត្តិការណ៍បែបនេះតាមរយៈនយោបាយ និងយោធាដើម្បីសម្រេចបាននូវការគ្រប់គ្រងពិភពលោក។ ខ្ញុំបានលើកឡើងអំពីគន្លឹះអំពីរបៀបដែលវាត្រូវបានផ្តល់មូលនិធិនៅអាមេរិក ព្រមទាំងបានយោងទៅលើទ្រឹស្តីដែលមិនឆ្ងាយពីការពិតបំផុតទាក់ទងនឹងខ្សែឈាមពិតប្រាកដដែលកំពុងដឹកនាំការសម្តែងនេះ។ ចំពោះអ្នកដែលចង់ដឹងបន្ថែម មានព័ត៌មានបន្ថែមជាច្រើនដែលខ្ញុំបានចោលដែលអាចរកបានដោយការជីករកតាមឯកសារយោង។
ចំណុចទិន្នន័យដែលអាចវាស់បានដ៏រឹងមាំបំផុតដែលបង្ហាញពីព្រឹត្តិការណ៍ភូមិវិទ្យាដែលកំពុងខិតជិតមកដល់គឺដែនម៉ាញេទិករបស់ផែនដីកំពុងផ្លាស់ប្តូរយ៉ាងឆាប់រហ័ស។ នេះអាចវាស់បានមិនត្រឹមតែដោយចលនាដែលកំពុងបង្កើនល្បឿននៃប៉ូលម៉ាញេទិកខាងជើង (រូបភាព \ref{fig:13}) និងការរីកធំឡើងនៃអាណូម៉ាលីដែនម៉ាញេទិកអាត្លង់ទិកខាងត្បូងប៉ុណ្ណោះទេ ប៉ុន្តែក៏មានការចុះខ្សោយ និងការខូចទ្រង់ទ្រាយជាទូទៅនៃដែនម៉ាញេទិកក្នុងរយៈពេល ៤០០ ឆ្នាំចុងក្រោយនេះផងដែរ \cite{3}។ ទិន្នន័យវិទ្យាសាស្ត្របែបនេះត្រូវបានពិភាក្សាយ៉ាងហ្មត់ចត់នៅក្នុងអត្ថបទ ECDO ពីរដំបូងរបស់ខ្ញុំ ដែលអាចចូលទៅមើលបានតាមរយៈគេហទំព័ររបស់ខ្ញុំ \cite{3}។

\begin{figure}[t]
\begin{center}
% \fbox{\rule{0pt}{2in} \rule{0.9\linewidth}{0pt}}
   \includegraphics[width=1\linewidth]{npw.jpg}
\end{center}
   \caption{ទីតាំងនៃប៉ូលម៉ាញេទិកខាងជើងពីឆ្នាំ ១៥៩០ ដល់ឆ្នាំ ២០២៥ ដែលបង្ហាញជាប្រាំឆ្នាំម្តង។ ចលនារបស់វាបានចាប់ផ្តើមលឿនយ៉ាងឆាប់រហ័សនៅឆ្នាំ ១៩៧៥។}
\label{fig:13}
\label{fig:onecol}
\end{figure}

នៅក្នុងការបញ្ចប់ ខ្ញុំនឹងទុកឱ្យអ្នកនូវសម្តីពីអាម៉ាលុល្លា ដែលពន្យល់អំពីរបៀបដែល \textit{"\textbf{អ្វីៗទាំងអស់គឺជារបស់មួយ}"}: \textit{"នៅទីនេះ ខ្ញុំត្រូវតែជំរុញការស្រមៃរបស់អ្នកដល់ដែនកំណត់ខ្ពស់បំផុត។ អ្នកត្រូវតែភ្លេចអំពីពិភពលោកដែលអ្នករស់នៅឥឡូវនេះ និងបានស្គាល់តាំងពីកុមារភាព។ ចាកចេញពីវា។ វាគឺជាការបង្កើតឡើងទាំងស្រុងមិនដូចគ្នានឹងអ្វីដែលត្រូវបានពិពណ៌នានៅក្នុងភាពយន្ត Matrix ហើយមានបំណងឱ្យអ្នកគេងរហូតដល់ពេលចុងក្រោយ។ ពេលខ្លះខ្ញុំសង្ឃឹមថាខ្ញុំកំពុងសរសេរស្គ្រីបសម្រាប់ភាពយន្តមួយ ប៉ុន្តែអ្វីដែលខ្ញុំកំពុងចែករំលែកជាមួយអ្នកនៅលើគេហទំព័រនេះគឺពិត។ វាបានយកពេលជាងពាក់កណ្តាលទសវត្សរ៍ដើម្បីយល់ឃើញថា «អ្វីៗទាំងអស់គឺជារបស់មួយ» ដែលខ្ញុំបានអនុវត្តយ៉ាងឆាប់រហ័សជាបាវចនាសម្រាប់ An Apocalyptic Synthesis។ នេះគឺជាគំនិតដែលពិបាកក្នុងការបញ្ចេញ។ សម្រាប់ពេលនេះ ចូរយើងគិតតាមរយៈភាពយន្ត Matrix។ វាជាការប្រៀបធៀបល្អ។ អ្វីដែលខ្ញុំពិបាកក្នុងការបញ្ចេញគឺថាអ្វីដែលខ្ញុំនឹងនិយាយនេះមិនមែនជាការព្យួរឡើយ។ សម្រាប់ពេលនេះ ការប្រៀបធៀបភាពយន្ត Matrix គឺជាអ្វីដែលខ្ញុំអាចធ្វើបានដើម្បីឱ្យអ្នកយល់ពីភាពពិតដ៏អាក្រក់នៃអ្វីដែលខ្ញុំនឹងនិយាយ។ \textbf{អ្វីៗទាំងអស់ក្នុងជីវិតរបស់អ្នក រួមទាំងប្រវត្តិសាស្ត្រទាំងមូល វិទ្យាសាស្ត្រសំខាន់ៗ និងសាកលវិទ្យាល័យ នយោបាយ សាសនា អ្វីៗទាំងអស់តាមវិធីមួយឬផ្សេងទៀតគឺទាក់ទងនឹងការផ្លាស់ប្តូរផ្ទៃរបស់ផែនដីឬការផ្លាស់ប្តូរអ័ក្សដែលកំពុងកើតឡើង។} អ្នកគ្រាន់តែមិនអាចមើលឃើញវានៅពេលនេះទេ។ ហើយអ្នកក៏មិនអាចភ្ញាក់ពីការពិតនេះបានដូចជាពីសុបិនអាក្រក់ដែរ។ វាត្រូវការពេលវេលា។ ប៉ុន្តែខ្ញុំសន្យាថា ទីបញ្ចប់នៃផ្លូវនេះគឺការយល់ដឹងថាអ្នកបានរស់នៅក្នុងភាពស្រដៀងគ្នានឹងការក្លែងធ្វើដោយកុំព្យូទ័រនៃភាពពិត Matrix ពេញមួយជីវិតរបស់អ្នក"} \cite{33,34}។
សូមជូនពរឲ្យជោគជ័យដល់មនុស្សគ្រប់គ្នា។

\section{សេចក្តីអំណរគុណ}

សូមអរគុណដល់បុគ្គលទាំងឡាយដែលបានជ្រើសរើសចែករំលែកចំណេះដឹងទៅឲ្យសាធារណៈ។ បើគ្មានអ្នកទេ ស្នាដៃនេះនឹងមិនអាចកើតមានបានឡើយ ហើយមនុស្សជាតិនឹងនៅតែក្នុងភាពងងឹត។ ជម្រើសរបស់អ្នកនឹងចេញផ្ការហូតដល់អនាគត។ យើងខ្ញុំជំពាក់អ្នកគ្រប់យ៉ាង ហើយខ្ញុំពិតជាមានអំណរគុណជាអនេក។
\clearpage
\twocolumn

{\small
\renewcommand{\refname}{ឯកសារយោង}
\bibliographystyle{ieee}
\bibliography{egbib}
}
\end{document}
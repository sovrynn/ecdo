\documentclass[10pt,twocolumn,letterpaper]{article}

\usepackage{booktabs}
% \usepackage{caption}
% \captionsetup[table]{skip=8pt}   % यो केवल तालिकाहरूमा लागू हुन्छ
\usepackage{stfloats}  % यसलाई प्रीएम्बलमा थप्नुहोस्
\usepackage{float}


\usepackage{fontspec}
\usepackage{ucharclasses}

%–– define your two fonts ––
\newfontfamily\latinfont{Latin Modern Roman}              % for all non-Devanagari text
% \newfontfamily\nepalifont[Script=Devanagari]{Noto Serif Devanagari}    % for all Nepali text; change “Phobikha” to your font’s exact name
\newfontfamily\nepalifont[Script=Devanagari]{Noto Sans Devanagari}    % for 

%–– ucharclasses auto-detects Unicode blocks ––
\setDefaultTransitions{\latinfont}{}                      
\setTransitionsForDevanagari{\nepalifont}{\latinfont}   % switch to Nepali font in Devanagari, then back
\setTransitionTo{DevanagariDanDa}{\nepalifont}

\usepackage{cvpr}
\usepackage{times}
\usepackage{epsfig}
\usepackage{graphicx}
\usepackage{amsmath}
\usepackage{amssymb}

\usepackage[breaklinks=true,bookmarks=false]{hyperref}

\cvprfinalcopy % *** अन्तिम पेशको लागि यो लाइन अनकमेन्ट गर्नुहोस्

\def\cvprPaperID{****} % *** यहाँ CVPR पेपर आईडी लेख्नुहोस्
\def\httilde{\mbox{\tt\raisebox{-.5ex}{\symbol{126}}}}

% पृष्ठहरू सबमिशन मोडमा नम्बर गरिन्छन्, र क्यामेरा-रेडीमा नम्बर रहित हुन्छन्
%\ifcvprfinal\pagestyle{empty}\fi
\setcounter{page}{1}
\begin{document}

\title{ईसीडीओ कागज ३: हालको पश्चिमी शासक शक्तिहरूले आसन्न भौगोलिक प्रलयको तयारी गरेको प्रमाण}

\author{जुनहो\\
प्रकाशित मिति: जुन २०२५\\
वेबसाइट (यहाँबाट पेपरहरू डाउनलोड गर्नुहोस्): \href{https://sovrynn.github.io}{sovrynn.github.io}\\
ECDO अनुसन्धान भण्डार: \href{https://github.com/sovrynn/ecdo}{github.com/sovrynn/ecdo}\\
{\tt\small junhobtc@proton.me}
}

\maketitle
%\thispagestyle{empty}

\begin{abstract}
मई २०२४ मा, "द एथिकल स्केप्टिक" नामक एक छद्मनामधारी अनलाइन लेखकले \cite{0} Exothermic Core-Mantle Decoupling Dzhanibekov Oscillation (ECDO) \cite{1} नामको एक नयाँ सिद्धान्त साझा गरे। यस सिद्धान्तले सुझाव दिन्छ कि पृथ्वीले अतीतमा अचानक र विनाशकारी तरिकाले आफ्नो घूर्णन अक्षमा ठूलो परिवर्तनको अनुभव गरिसकेको छ, जसले गर्दा समुद्रहरू घुमाउने जडत्वका कारण महादेशहरूमा पस्दै विशाल बाढी आएको थियो। साथै, यसमा एउटा स्पष्टीकरण दिने भू-भौतिक प्रक्रिया र तथ्याङ्कहरू प्रस्तुत गरिएको छ, जसले देखाउँछ कि यस्तै अर्को परिवर्तन नजिकै हुनसक्छ। यस्ता प्रलयकारी बाढी तथा संसार अन्त्यको भविष्यवाणीहरू नयाँ होइनन्, तर ECDO सिद्धान्त आधुनिक विज्ञान, बहु-क्षेत्र, तथा तथ्याङ्कमा आधारित दृष्टिकोणका कारण निकै आकर्षक छ।

यो लेख मेरो तेस्रो काम हो \cite{2,3} यस विषयमा, र यसले अहिलेको राजनीतिक पक्षहरूमा केन्द्रित गर्छ:
\begin{flushleft}
\begin{enumerate}
    \item ह्विसलब्लोअरहरूको बयान छ कि पश्चिमेली शक्तिहरूको विश्वास छ कि एक भू-प्राकृतिक प्रकोप नजिकै छ र तिनीहरूले यस घटनाबाट राजनीतिक तथा सैनिक फाइदा उठाउने योजना बनाइरहेका छन्।
    \item घटनाको तयारी गर्न निर्मित व्यापक पश्चिमेली भूमिगत तथा जलमुनि बेसहरूको प्रमाण।
    \item यी बेसहरूलाईको लागि पश्चिमेली मुद्रा संरचनाबाट ठूलो मात्रामा धन बाहिर पठाइएको प्रमाण।
\end{enumerate}
\end{flushleft}
यो पेपरले देखाउँछ कि पश्चिमी शासक शक्तिहरूले उनीहरू आसन्न छ भन्ने विश्वास गरेको भौगोलिक प्रलयको तयारीमा व्यापक तयारीहरू गरिरहेका छन्।
\end{abstract}

\section{फ्रीमेसनरी र "एङ्ग्लो-सेक्सन मिसन"}

जनवरी २०१० मा, प्रोजेक्ट क्यामेलट, वैकल्पिक मिडिया र पत्रकारिता संगठन जसले व्हिसलब्लोअरहरूको साक्षात्कार संकलन गर्छ, ले \cite{4,6} एउटा भित्री व्यक्तिलाई अन्तर्वार्ता गर्यो जो जुन २००५ मा लन्डन सहरमा सीनियर म्यासनहरूको बैठकमा शारीरिक रूपमा उपस्थित थिए। बैठकमा छलफल भएका विषयहरू सैन्य र राजनीतिक योजना थिए, जसको पृष्ठभूमिमा आउने \textbf{"भौगोलिक घटना"}, अर्थात्, विश्वव्यापी प्राकृतिक विपत्ति थियो।

\begin{figure}[b]
\begin{center}
\includegraphics[width=1\linewidth]{freemason.jpg}
\end{center}
   \caption{ब्रिटिश फ्रिमेसनहरू उनीहरूको प्राकृतिक अवस्थामा, शान्तिपूर्वक आणविक बम फ्याँक्ने र संसार कब्जा गर्ने योजना बनाउाउँदै - लण्डनको अर्ल्स कोर्टमा, १९९२ \cite{5}.}
\label{fig:1}
\label{fig:onecol}
\end{figure}

\begin{figure*}[t]
\begin{center}
\includegraphics[width=1\textwidth]{british.jpg}
\end{center}
   \caption{१९३७ मा बेलायती साम्राज्य, अङ्ग्रेज-सक्सन शक्तिको एक प्रभावशाली प्रदर्शन \cite{14}.}
   \label{fig:2}
\end{figure*}

यो भित्री स्रोतका अनुसार, बैठकमा उपस्थित २५-३० जना व्यक्ति \textit{"...सबैजना बेलायती थिए, र तिनीहरूमध्ये केही निकै प्रसिद्ध व्यक्ति थिए जसलाई संयुक्त अधिराज्यका मानिसहरूले तुरुन्तै चिनिहाल्छन्... त्यहाँ अलि-अलि अभिजात वर्ग पनि छ, र तीमध्ये केही निकै अभिजात पृष्ठभूमिबाट आएका हुन्। मैले उक्त बैठकमा चिनेको एक जना वरिष्ठ राजनीतिज्ञ थिए। दुई जना प्रहरीका वरिष्ठ पदाधिकारी थिए, र एक जना सैनिक क्षेत्रका थिए। दुबै राष्ट्रिय रुपमा चिनिएका छन् र दुबै वर्तमान सरकारलाई सल्लाह दिने प्रमुख व्यक्ति हुन् — अहिलेको समयमा"} \cite{4}. भित्रको स्रोत भन्छ कि उसले बैठकमा उपस्थित भयो,\ \textit{"शुद्ध संयोगले मात्र! मैले सोचें कि यो साधारण तीन महिनामा हुने बैठक हो... म त्यो बैठकमा गएँ र त्यो मैले अपेक्षा गरेको बैठक थिएन। मलाई विश्वास छ म आमन्त्रित भएको थिएँ... किनभने मैले राखेको पदका कारण र उनीहरूले विश्वास गरेका थिए, आफ्नै जस्तै म पनि उनीहरूमध्ये एक हुँ।"} \cite{4}.

बैठकमा (सन् २००५ मा) छलफल गरिएका घटनाहरूको मूल समयरेखा यसप्रकार छ:

\begin{flushleft}
\begin{enumerate}
    \item इरान वा चीनलाई सामरिक परमाणु हतियार प्रयोग गर्न उक्साउने र सीमित परमाणु युद्ध ल्याउने, त्यसपछि युद्धविराम स्थापन गर्नु।
    \item चीनमाथि जैविक हतियार प्रहार गर्ने, भनिएको अनुसार मुख्य लक्ष्य "७० को दशकदेखि"।
    \item परिणामी डर र अराजकताहरूको कारण देखाई पूरै अधिनायकवादी सैनिक सरकारहरू ल्याउने।
\end{enumerate}
\end{flushleft}

तर सबैभन्दा महत्वपूर्ण कुरा ती घटनाहरूपछि के हुने आशा गरिएको छ: \textit{"त्यसैले हामी यो युद्धमा जाने छौं, त्यसपछि... पृथ्वीमा एउटा भू-भौगोलिक घटना हुने छ जुन सबैलाई असर गर्नेछ"} \cite{4}। भित्री स्रोतले विश्वास गर्छ कि सो भू-भौगोलिक घटनाको समयमा, \textit{\textbf{"पृथ्वीको भूपर्पटी करिब ३० डिग्री, करिब १७०० देखि २००० माइल दक्षिणतिर सर्नेछ, र यसले विशाल उल्थान ल्याउने छ, जसका प्रभावहरू धेरै लामो समयसम्म रहनेछ"}} \cite{4}।

यी सबै गोप्य योजनाको कारण भनेको, निश्चित रूपमा, शक्ति हो। भित्री स्रोतले स्पष्ट पार्छ, \textit{"अब, त्यो समयसम्म हामी सबैले परमाणु र जैविक युद्ध भोगिसकेका हुने छौं। यदि त्यो भयो भने, पृथ्वीको जनसंख्या उल्लेखनीय रूपले घट्नेछ। जब त्यो भू-भौगोलिक घटना हुने छ, तब बाँकी रहेका जनसंख्यामा फेरि करिब आधा कम हुन सक्छ। र त्यो बाँच्ने को होला भन्ने कुराले नै तोक्छ कोले संसार र बाँकी रहेका जनसंख्यालाई नयाँ युगमा लैजान्छ। त्यसैले हामी एक महाविनाशकारी घटनापछिको युगबारे कुरा गर्दैछौं। को जिम्मामा रहने? को नियन्त्रणमा रहने? सबै कुरा त्यही हो। र त्यसैले उनीहरू यस्ता कुरा निश्चित समयसीमामा घटोस् भनेर त्यसमा यति धेरै बौलाएका छन्... संरचना पहिले नै तयार हुनुपर्छ [अराजकता] हुनु भन्दा अगाडि, केही निश्चितताका साथ कि त्यो आउने कुरा जोगिन सक्छ -- ताकि अर्को दिन त्यो आफ्नो ठाउँमा उभिन सकोस्, अनि त्यही शक्ति कायम रहोस् जसको मजा उसले अघिल्लो पल्ट लिएको थियो"} \cite{4}। अन्तर्वार्ताको क्रममा, यस योजनाको नाम, "एंग्लो-सैक्सन मिशन", पनि छलफल हुन्छ: \textit{[अन्तर्वार्ता लिनेजन]: "...यसलाई एंग्लो-सैक्सन मिशन भनिने कारण हो मुख्य रूपमा योजना चिनियाँहरूलाई समाप्त गर्ने हो, ताकि विपत्तिपछि र जब कुरा फेरि बनाइन्छ, एंग्लो-सैक्सनहरूं नै नयाँ पृथ्वीलाई फेरी बनाउन र उत्तराधिकारी बन्न सक्षम हुनेछन्, अरु कोही नहोस। के त्यो ठीक हो?" [भित्री स्रोत]: "त्यो ठीक हो कि होइन मलाई साँच्चै थाहा छैन, तर म तपाईंको कुरा सहमत छु। कम्तिमा २०औं शताब्दीमा, र अझ अघि १९औं र १८औं शताब्दीमा समेत, संसारको इतिहास प्रायः पश्चिम र पृथ्वीको उत्तर क्षेत्रबाट नै चलेको हो"} \cite{4}।
Regarding the exact timeframe of the expected geophysical event, the insider offers his best guess: \textit{"...अनुमान, र यो धेरै अन्तरदृष्टिपूर्ण छ, उनीहरूले अबै आफ्ना काम मिलाउनुपर्छ भन्ने हो...मलाई लाग्छ उनीहरूलाई यो कहिले हुन्छ भन्ने राम्रो विचार छ... \textbf{मसँग यो धेरै बलियो अनुभूति छ कि यो मेरो जीवनकालमा हुन्छ, भनौं २० वर्षभित्र}...अब हामी त्यो अवधिमा प्रवेश गरिसकेका छौं जहाँ यो भूवैज्ञानिक घटना हुन लागेको छ, जब हामी अघिल्लो घटना भएर गएको समयको लम्बाइलाई विचार गर्छौं जुन करिब ११,५०० वर्षअघि भएको थियो, र यो करिब ११,५०० वर्षमा, चक्रिक रुपमा हुँदै आउँछ। अब फेरि हुन लागेको छ...उनीहरूले यो हुन गइरहेको छ भन्ने बुझ्छन्। उनीहरूलाई निश्चित ज्ञान छ कि यो हुने छ... फेरी, यो ती कुराहरु मध्ये एक हो -- यदि उनीहरूलाई थाहा छैन भने त्यो कल्पना गर्न नसकिने हुन्छ। मेरो मतलब, विश्वका सबैभन्दा बुद्धिमान दिमागहरुले योका लागि काम गरिरहेका छन्"} \cite{4}.

This is a powerful testimony for which we should be very grateful. In the interview, the author also discusses his belief that WWI and WWII were manufactured wars, and that the Anglo-Saxon Mission almost certainly dates back many, many generations. It has now been 15 years since the interview, which occurred in 2010. There are five years remaining until the insider's stated 20-year timeframe prediction for the geophysical event reaches its end.

\subsection{ड्रुइडहरूको गूढ पश्चिमा विपत्तिको ज्ञान}

पुनरावृत्त हुने विपत्तिहरूको पश्चिमा ज्ञान राम्ररी सुरक्षित गरिएको छ, र केवल फ्रिमेसनहरूले मात्र होइन। ड्रुइडहरू, राम्ररी अभिलेखित प्राचीन सेल्टिक संस्कृति जसको कम्तीमा २४०० वर्षको इतिहास छ \cite{7}, पृथ्वीमा पुनरावृत्त हुने विपत्तिहरूको ज्ञान पास गर्ने गर्थे। अन्तिम ज्ञात ड्रुइड बेन म्याकब्रेडी भएको विश्वास गरिन्छ। "द लास्ट ड्रुइड" नामक १९९२ को वृत्तचित्रमा, उनले ड्रुइडहरूको ज्ञानबारे जानकारी साझा गरे: \textit{"परम्पराद्वारा म अन्तिम सदस्य हुन सक्छु भन्ने आदेश अघिल्लो ठूलो विपत्ति, वा प्रलय, जसले संसारलाई असर गरेको थियो, पछि अस्तित्वमा आयो। यी ठूला र भयानक असरहरू पृथ्वीमा ठूला विद्युत् आँधीहरूले गर्दा, धूमकेतुहरूको पुच्छरमा समातिनु, वा उल्कावृष्टिको झरीमा परिनु, हामीले चिनेको सभ्यता पूर्ण रूपमा नष्ट भएको थियो... सबै ज्ञान आदेशको क्षेत्रभित्र आइरहेको थियो, तर उनीहरू विशेष गरी खगोलशास्त्रप्रति चिन्तित थिए किनभने उनीहरूले धेरै महत्वपूर्ण विपत्तिहरू भोगेका थिए। यस्तो सोचिएको थियो कि खगोलशास्त्रको पूर्ण ज्ञानले उनीहरूलाई ती विपत्तिहरू कहिले सम्भावित छन् भन्ने भविष्यवाणी गर्न र आफैंलाई जोगाउने उपाय अपनाउन मद्दत पुर्‍याउछ। यदि तपाईं आयरल्याण्डका ठूला मेगालिथिक संरचनाहरू हेर्नुहुन्छ भने तपाईंले देख्नुहुन्छ कि जुन चिहानको बाटोको रूपमा वर्णन गरिएको छ ती वास्तवमा धेरै प्राथमिक बम सेल्टरहरू हुन्। ती कुनै पनि ज्वारभाटादेखि माथि छन् र तिनीहरूले उल्का-वृष्टिबाट पनि सुरक्षा दिन्छन्"} \cite{8,9}.

% It is also believed that Freemasonry itself actually originates from the Druids \cite{10}.
\section{हालको पश्चिमी विनाशको तयारीको प्रमाण}

जसरी सत्ताधारी पश्चिमी शक्तिहरूले विश्वव्यापी भू-भौतिकिक विनाश नजिकिएको विश्वास गर्छन् जस्तो देखिन्छ, हामीले यस्ता घटनाबाट बच्नको लागि तीव्र तयारी भइरहेको अपेक्षा गर्न सक्छौं। वास्तवमा, विभिन्न पश्चिमी देशहरूमा फैलिएको गहिरो भूमिगत आधारहरूको ठूलो जालको सार्वजनिक प्रमाण छ। यस्ता संरचनाहरूले निश्चित रूपमा आणविक युद्धमा बासिन्दाहरूलाई सुरक्षा दिने थिए, यी विभिन्न प्राकृति प्रकोपहरूबाट पनि सुरक्षा स्वरूप काम गर्ने थिए। प्रोजेक्ट क्यामेलट \cite{4,6} का ब्रिटिश वरिष्ठ फ्रिमेसनको साक्षात्कारले देखाउँछ कि यी सम्भावनाहरू मात्र होइनन्, बरु पूर्वनियोजित योजनाहरू हुन्। यसका साथसाथै यस्ता आधारहरू निर्माण, जनशक्ति व्यवस्थापन, र मर्मतसम्भार गर्नका लागि चाहिने विशाल रकम उल्लेखनीय छ, जुन अमेरिकी सरकारबाट १८ वर्षमा हराएको दसौं ट्रिलियन डलरजस्तै रकमसँग मेल खान्छ (अर्को खण्डमा उल्लेख गरिएको) \cite{11,12,13}। विलुप्ति-स्तरको घटनाका लागि तयारीका अन्य उदाहरणहरूमा बीउ र ज्ञानको संग्रहका विभिन्न परियोजनाहरू पर्छन्।

\subsection{अमेरिकी भूमिगत र समुद्रमुनिको आधारहरू}

मैले पत्ता लगाएको भूमिगत आधारमाथिको सबैभन्दा ठूलो सार्वजनिक अनुसन्धान स्वतन्त्र अमेरिकी अनुसन्धानकर्ता रिचर्ड सउडरबाट आएको छ, जसले गहिरो भूमिगत आधारहरूको विषयमा धेरै पुस्तकहरू प्रकाशित गरेका छन् \cite{22}। सउडरको काम सरकारका कागजात र योजनाहरू सङ्कलन गर्नु, ऐतिहासिक तथा हालका समाचार, प्रविधिहरूको अवलोकन गर्नु, स्रोतहरू सँग संवाद गर्नु, र भित्री दावीहरूको सङकलनमा केन्द्रित छ। सउडरको अध्यनले देखाउँछ कि अमेरिका र यसको अधीनस्थ क्षेत्रमा ठूलो गहिरो भूमिगत तथा समुद्रमुनिको आधारहरूको जालछ (चित्र \ref{fig:4}), जसको गहिराइ कम्तिमा ३ माइलसम्म पुग्न सक्छ, र सम्भवतः भूमिगत भ्याक्युम ट्युब उच्चगतिको चुम्बकीय रेलमार्गद्वारा आपसमा जोडिएको हुनसक्छ। यी आधारहरू \textit{"उच्च वित्त, अन्तर्राष्ट्रिय, अन्तरनिकाय, मुद्रा सफा गर्ने खोल खेल"} मार्फत गोप्य रूपमा वित्त पोषित छन्, जसलाई संयुक्त राज्य अमेरिकाको कम्पनीका मालिकहरूले सञ्चालन गर्छन् \cite{22}। क्याथरिन औस्टिन फिट्स (जसको काम अर्को खण्डमा उल्लेख छ) र उनको एक सहकर्मीले यी आधारहरूको मापदण्ड पत्ता लगाउन गरेका अनुसन्धानहरूले अमेरिकी भूमिगत तथा समुद्रमुनि १७० आधारित आधारहरूको अनुमान प्रस्तुत गरेको छ \cite{16,20}।

\begin{figure}[b]

\begin{center}
% \fbox{\rule{0pt}{2in} \rule{0.9\linewidth}{0pt}}
   \includegraphics[width=1\linewidth]{penta.jpg}
\end{center}
   \caption{श्वेत सदन र पेन्टा्गनको तल वास्तवमै के छ? देखिन्छ, तिनीहरू मुनि गहिरो भूमिगत सुरुङहरूको नेटवर्क छ (तस्बिर: \cite{31})।}
\label{fig:3}
\label{fig:onecol}
\end{figure}
\begin{figure*}[t]
\begin{center}
% \fbox{\rule{0pt}{2in} \rule{.9\linewidth}{0pt}}
\includegraphics[width=0.9\textwidth]{basescrop.png}
\end{center}
   \caption{साउडरको अनुसन्धानले पत्ता लगाएका भूमिगत र समुद्रमुनी आधारहरुका निश्चित स्थानहरू देखाउने नक्सा, साथै भित्री भूभागतिर जाने जलमुनि पनडुब्बी सुरुङहरू पनि। साउडर \textit{"निश्चित छ कि \textbf{यी भन्दा धेरै} संरचनाहरू छन्"} \cite{22}.}
   \label{fig:4}
\end{figure*}

यहाँ साउडरका स्रोतहरुले यी आधारहरुको विस्तार उल्लेख गरेका केही बयानका अंशहरू छन्:
\begin{flushleft}
\begin{enumerate}
    \item क्याम्प डेभिड, म्यारील्याण्ड: \textit{"मेरो स्रोतले मलाई जानकारी दियो कि क्याम्प डेभिडको भुईँमुनि भागहरू अत्यन्तै ठूलो र जटिल छन्, र त्यहाँ यति धेरै माइल लामो गोप्य सुरुङहरू छन् कि सायदै कुनै एक व्यक्तिले सम्पूर्ण ठाउँको पूर्ण नक्सा आफ्नो दिमागमा राख्न सक्छ"} \cite{22}।
    \item ह्वाइट हाउस, वाशिंगटन डीसी: \textit{"मेरो एक नजिकको साथीलाई १९६० को दशकमा लिंडन बी. जोनसन प्रशासनको समयमा यो सुविधा भित्र ल्याइएको थियो। उनी ह्वाइट हाउसमा रहेको एलिभेटरमा छिर्नु भयो र सिधै तल लैजानुभयो। उहाँलाई विश्वास छ कि एलिभेटर १७ तलासम्म झर्‍यो। जब ढोका भुईँमुनि खुल्यो, उनलाई यस्तो सुरुङ हुँदै लैजाँदै थिए जुन टाढासम्म हराउँदै गएको देखिन्थ्यो। त्यो सुरुङबाट अरू ढोका र बाटाहरू खुलिरहेका थिए"} \cite{22}। चित्र \ref{fig:3} मा देखाइएको छ।
    \item फोर्ट मीड, म्यारील्याण्ड - १९७० को दशकमा संयोगले "तलभित्र" पसेका स्रोतबाट: \textit{"मैले ढोका खोलें र त्यो तल झर्ने सिँढी रहेछ। म किनारामा गएर रेलिङ बिच हेरेँ। तल कहिलेसम्म तल थियो भन्ने मैले गनिन, तर लगभग १५-२० तलाजति होला जस्तो लाग्यो... म एक तल झरेँ र त्यहाँ ढोका थियो... मैले ढोका खोलेर टाउको बाहिर निकालेर दायाँ-बायाँ हेरेँ, जहाँ दुवैतर्फ टाढासम्म फैलिएको सुरुङ देखियो। त्यो निश्चित रूपमा भुईँ र पार्किङ क्षेत्रभन्दा परसम्म फैलिएको थियो। अर्को पर्खालतिर ३०-४० फिटको दुरीमा ढोकाहरू थिए... मैले थप केही तल जाँच गर्ने निधो गरेँ, अर्को तल झरेँ... र उस्तै संरचना देखेँ... म एक तल झरेँ र हेर्दा पहिलो २ तलाजस्तै थियो"} \cite{22}।
\end{enumerate}
\end{flushleft}

\begin{figure}[t]
\begin{center}
% \fbox{\rule{0pt}{2in} \rule{0.9\linewidth}{0pt}}
   \includegraphics[width=1\linewidth]{undersea.jpg}
\end{center}
   \caption{वाल्टर कोएर्श्नरद्वारा बनाइएको समुद्र मुनिको आधारको चित्रण। उनी १९६० को दशकमा अमेरिकी नौसेनाको चीन लेक, क्यालिफोर्निया वेपन्स सेन्टरमा नौसेनाको रक-साइट समुद्र मुनि आधार टोलीका चित्रकार थिए। साउडरका स्रोतहरू मध्ये एकले खुलासा गर्छ कि चीन लेकमा एक माइल गहिराइमा भूमिगत आधार छ \cite{22,23}।}
\label{fig:5}
\label{fig:onecol}
\end{figure}

साउडरले भूमिगत चुम्बकीय लेभिटेशन ट्रेनहरूको बारेमा पनि प्रमाणहरू प्राप्त गरे, जुन २,००० माइल प्रतिघण्टा सम्मको गतिमा पुग्छन्, समुद्रको भुईँमुनि बनाइएका आधारहरू (चित्र \ref{fig:5}), र समुद्र मुनिका पनडुब्बी सुरुङ्हरू जसले भित्री भूमिमा लैजान्छन्। मेक्सिकोको खाडीमा रहेको समुद्र मुनिको आधारको बारेमा आएका मध्येकाअन्य प्रमाणको सन्दर्भमा, साउडर भन्छन्, \textit{"Underwater and Underground Bases को प्रकाशन भएको लगभग आधा वर्षपछि, मलाई एकजना व्यक्तिले सम्पर्क गरे जसले अनौठो समुद्र मुनिको परियोजना को बारेमा ज्ञान भएको बताए... उनले यो परियोजना मेक्सिकोको खाडीको समुद्री भुइँमुनि रहेको र पर्सन्स ठेकेदार रहेको उल्लेख गरे। उनले थपे कि पर्सन्सले समुद्रको भुइँमुनि २,८०० फिट तल सञ्चालनका लागि विशेष उपकरण किनेका थिए... यो उपकरण यति अनौठो छ कि यसले जहाँ स्थापना गरिएको छ त्यहाँ प्रत्यक्ष मानव उपस्थिति भएको स्पष्ट अनुमान लगाउँछ"} \cite{22}।
\begin{figure}[t]
\begin{center}
% \fbox{\rule{0pt}{2in} \rule{0.9\linewidth}{0pt}}
   \includegraphics[width=1\linewidth]{sub.jpg}
\end{center}
   \caption{वाल्टर कोएर्श्नरद्वारा बनाइएको एक पानीमुनि पनडुब्बी सुरुङ्गको चित्रण \cite{22,23}।}
\label{fig:6}
\label{fig:onecol}
\end{figure}
\begin{figure}[t]
\begin{center}
% \fbox{\rule{0pt}{2in} \rule{0.9\linewidth}{0pt}}
   \includegraphics[width=1\linewidth]{iran.jpeg}
\end{center}
   \caption{सरकारी ईरानी भिडियोबाट लिइएको दृश्य जसमा उनीहरूको भूमिगत "मिसाइल सिटी" देखाइएको छ \cite{39,40}।}
\label{fig:12}
\label{fig:onecol}
\end{figure}
यदि साँच्चिकै हाम्रो खुट्टाको तलका सतहको माइलहरूसम्म खनेका १७०+ भूमिगत र भूमिसागर मुनिका आधारशिलाका विशाल गोप्य अन्तरमहादेशीय सञ्जालहरू छन् भने, ती हाइपरसोनिक भ्याकुम-ट्युब म्याग्लेव ट्रेनहरूले जडित छन् भने, र हाम्रा श्रमका फलहरूद्वारा वित्तीय संरक्षण गरिएको छ भने, आजको मानवता ठूलो सुखद अज्ञानतामा रहनेथिइन, केवल तल के छ भन्नेमा होइन, तर निकट भविष्यमा के आउँदैछ भन्ने थाहा नपाउने थिए, किनभने उनीहरूले आफ्ना राजनीतिज्ञहरूको खाली र समन्वित बयानहरू आत्मसात गरिरहेका छन्।

एक थप नोट - ठूलो भूमिगत सुरुङ्ग सञ्जालको अस्तित्व बिना कुनै शंका मध्यपूर्वको जारी संघर्षमा खुलिसकेको छ (गाजा पट्टिमा हामासका सुरुङ्गहरू \cite{38}, र इरानको भूमिगत "मिसाइल शहर" (चित्र \ref{fig:12}) \cite{39,40})। यिनीहरूले त्यस्ता संरचनाहरू बनाउने सम्भावना मात्र होइन, तिनको वास्तविक अस्तित्व पनि छ भन्नेमा कुनै शंका नरहनु पर्छ। यसले हामीलाई अरु छवि राम्रोलगानी गरेका देशहरूले सोही समयमा कस्ता संरचनाहरू निर्माण गरेका होलान् भन्ने प्रश्न उठाउन पनि बाध्य बनाउँछ।

\subsection{थप बंकर र महाविपत्ति तयारी प्रमाणहरू}

\begin{figure}[t]
\begin{center}
% \fbox{\rule{0pt}{2in} \rule{0.9\linewidth}{0pt}}

\includegraphics[width=1\linewidth]{tyrol.jpg}
\end{center}
   \caption{दक्षिण टिरोल, स्विट्जरल्याण्ड मा बंकरहरू। स्विट्जरल्याण्ड, जुन युरोपेली अल्प्स पर्वत श्रृंखलामा फैलिएको छ, यसको पहाडी बंकरहरूलाई चतुर तरिकाले लुकाउनेका लागि परिचित छ \cite{32}.}
\label{fig:7}
\label{fig:onecol}
\end{figure}

\begin{figure}[t]
\begin{center}
% \fbox{\rule{0pt}{2in} \rule{0.9\linewidth}{0pt}}
   \includegraphics[width=1\linewidth]{svalbard.jpg}
\end{center}
   \caption{नर्वेमा रहेको स्वालबार्ड ग्लोबल सिड भल्ट, जसमा एक मिलियन भन्दा बढी बिउहरु छन् \cite{24}। कस्तो विपत्तिले यसको प्रयोग आवश्यक पार्ला भन्ने जिज्ञासा उठ्छ।}
\label{fig:8}
\label{fig:onecol}
\end{figure}

अमेरिकाका भूमिगत शाही आधारहरू बाहेक विश्वभर विपत्तिको तयारीका थप थुप्रै संकेतहरू छन्। नर्वे, स्विट्जरल्याण्ड, स्वीडेन, र फिनल्याण्ड उत्कृष्ट उदाहरण हुन्:

\begin{flushleft}
\begin{enumerate}
    \item प्रोजेक्ट क्यामेलोटले एक जना नर्वेजियन राजनीतिज्ञको सान्दर्भिक बयान साझा गर्यो \cite{25,26}, जसको पहिचान उनीहरूले प्रमाणित गरे तापनि गोप्य राखियो। उनले दाबी गर्छन् कि नर्वेमा १८ वटा विशाल भूमिगत आधारहरू छन्, र नर्वे (इजरायल तथा "धेरै अन्य देशहरू" सहित) ले यी आधारहरू कुनै प्रकारको प्राकृतिक प्रकोपको तयारीका लागि निर्माण गरिरहेको छ। रिचर्ड साउडरले पनि एक जना व्यक्तिबाट बयान पाए जसले नर्वेको एउटा ठूला पहाडको भित्र निर्माण गरिएको विशाल भूमिगत आधारमा प्रवेश गरेका थिए \cite{22}।
    \item स्वीट्जरल्यान्ड उच्च हिमाली क्षेत्रहरूमा बनाइएका धेरै आणविक बंकरका लागि प्रख्यात छ (चित्र \ref{fig:7})। यी बंकरको संख्या रोमांचक ३७०,००० भन्दा बढी छ - जुन प्रत्येक बासिन्दालाई आश्रय दिन पर्याप्त छ \cite{27}।
    \item स्वीडेन र फिनल्याण्डमा प्रत्येक प्रमुख सहरका बासिन्दालाई आश्रय दिनका लागि पर्याप्त बंकर छन् \cite{27}।
\end{enumerate}
\end{flushleft}

सिलिकन भ्यालीका व्यवसायिक धनाढ्यहरूलाई पनि यसबारे थाहा छ जस्तो देखिन्छ। भनिन्छ, \textit{"लिंक्डइनका सह-संस्थापक तथा ख्यातिप्राप्त लगानीकर्ता रीड हफम्यानले यस वर्षको सुरुमा 'द न्युयोर्कर' लाई भनेका थिए कि सिलिकन भ्यालीका ५०\% भन्दा बढी अर्बपतिहरूले कुनै न कुनै तहको "प्रलय बीमा", जस्तै भूमिगत बंकर किनिएका छन्... फोर्ब्सका योगदानकर्ता जिम डोब्सनका अनुसार, धेरै अर्बपतिका निजी विमानहरू 'एकैछिनमा प्रस्थान गर्न तयार' छन्। उनीहरूसँग मोटरसाइकल, हातहतियार, र जेनेरेटरहरू पनि छन्"} \cite{28}।

साथै, ग्लोबल नलेज भल्ट जसलाई आर्च मिसन फाउन्डेसनद्वारा संचालन गरिएको छ \cite{29} र स्वालबार्ड ग्लोबल सिड भल्ट \cite{30} जस्ता विभिन्न विशाल अभिलेख परियोजनाहरू पनि छन्, जसले विनाशकारी प्रकोपको अवस्थामा मानव जातिका जीवन्त सम्पत्ति सुरक्षित गर्नको लागि तयारी गरिरहेको देखिन्छ।
\begin{figure*}[t]
\begin{center}
% \fbox{\rule{0pt}{2in} \rule{.9\linewidth}{0pt}}
\includegraphics[width=0.9\textwidth]{govcrop2.png}
\end{center}
   \caption{१९९८ देखि २०२३ सम्म अमेरिकी सरकारको राजस्व, खर्च, र गुप्त भूमिगत आधार खर्च \cite{19}।}
   \label{fig:9}
\end{figure*}
\section{ठूला भूमिगत बेसहरूका लागि लोकतान्त्रिक वित्तीय संयन्त्रहरू}

त्यसो भए कसरी १७०+ भूमिगत र समुद्रमूनी बेसहरूको ठूलो महादेशीय सञ्जाललाई वित्त पोषण गरिन्छ जब ऋणदासहरूलाई अन्धकारमा राखिएको हुन्छ? त्यस्ता योजनाहरूमा कति पैसा जान्छ र कहाँबाट आउँछ भन्ने कुरा पत्ता लगाउन एउटा कागजी प्रमाण छ। सन् २०१७ मा, क्याथरिन अस्टिन फिट्स, एक अमेरिकी लगानी बैंकर र बुश प्रशासनको समयको पूर्व सार्वजनिक पदाधिकारी, र मार्क स्किडमोर, मिशिगन स्टेट विश्वविद्यालयका अर्थशास्त्री,ले १९९८-२०१५ आर्थिक वर्षहरूमा अमेरिकी सरकारमा २१ ट्रिलियन अमेरिकी डलरको अनधिकृत खर्च भेट्टाए \cite{11,12,13}।

उनीहरूको प्रतिवेदन अनुसार, \textit{"२०१६ अक्टोबर ७ मा रोयटर्सले स्कट पाल्ट्रो (२०१६) को लेख प्रकाशित गर्यो जसमा २०१५ आर्थिक वर्षमा सेना (आर्मी)ले \$६.५ ट्रिलियनको आधारहीन लेखापरीक्षण मिलान “आफ्नो किताबहरू सन्तुलित देखाउनको लागि” गरेको उल्लेख छ। त्यो वर्ष सेनाको साधारण कोषको बजेट \$१२२ अर्ब भएकाले, यो एकदमै आश्चर्यजनक कुरा हो... डिओडी (DOD) ले यहाँभन्दा धेरै वर्ष अघि अर्थात् २००१ सेप्टेम्बर १० मा आफ्नो लेखा समस्याकै कारण सचिव डोनाल्ड रम्सफेल्डले कांग्रेसको सुनुवाइमा (C-SPAN, २०१४) डिओडीले \$२.३ ट्रिलियन कारोबार हराएको कुरा बताए... यो स्वीकारोक्ति त्यहि दिन संचार माध्यमहरूमा आएको थियो, तर भोलिपल्टै ९/११ को दुःखद घटनाले सारा विश्वको ध्यान तानेपछि बिर्सियो... जब प्रोफेसर मार्क स्किडमोरलाई सेनाका रूपमा \$६.५ ट्रिलियन अप्रमाणिक कारोबार भएको थाहा भयो, उनले मिस फिट्सलाई सम्पर्क गरे र उनीहरू २०१७ को वसन्तमा सँगै अन्य त्यस्तो सरकारी प्रतिवेदन खोज्ने निर्णयमा पुगे जसमा HUD र DOD मा असामान्य रूपमा ठूलो अप्रमाण्य कारोबार देखिएको थियो। आगामी छ महिनामा, स्किडमोर, फिट्स र केही स्नातकोत्तर विद्यार्थीहरूले सरकारी आधिकारिक कागजातहरू संकलन गरे जसमा १९९८-२०१६ अवधिमा जम्मा \$२१ ट्रिलियन अप्रमाण्य कारोबार भेटियो"} \cite{12}।

उही १८ वर्ष, सन् १९९८-२०१५ भित्र, सार्वजनिक रूपमा स्वीकृत अमेरिकी सरकारका आम्दानी जम्मा ४०.८ ट्रिलियन मात्र थियो \cite{15}, जसले देखाउँछ कि अमेरिकी सरकारका आधिकारिक खर्चभन्दा आधाभन्दा धेरै रकम भूमिगत बेसहरूमा गोप्य रूपमा खर्च गरियो। अरू उल्लेखनीय कुरा के छ भने यो गोप्य खर्च दीर्घकालिन बजेट घाटामा पनि थपिएको छ, र सायद यो आजसम्म निरन्तर भइरहेको छ र १९९८ भन्दा पहिलेदेखि पनि थियो, जसले देखाउँछ कि यी बेसहरूमा खर्च भएको कुल रकम २१ ट्रिलियन डलरभन्दा धेरै नै छ। त्यहि अनुपातको गोप्य खर्चलाई २०१६-२०२३ को अवधिमा लगाउँदा सन् १९९८ पछि हालसम्म जम्मा ३६.६ ट्रिलियन अमेरिकी डलर खर्च भएको देखिन्छ।

सन् २०२१ मा, मार्क स्किडमोरले यसै अनुसन्धानको अद्यावधिक सार्वजनिक गरे जसमा ब्लूमवर्गले २०१७-१९ का आर्थिक वर्षहरूमा पेण्टागनले ९४.७ ट्रिलियन डलरका लेखा परिमार्जन गरेको घोषणा गरेको थियो \cite{17,18}। यदि हामी सन् १९१३ मा फेडरल रिजर्भ स्थापना भएदेखि सय वर्षभन्दा बढी समयदेखि भएको अमेरिकी डलरको केन्द्रीय बैंकिंग प्रणालीमार्फत भएको नक्कलीकरणलाई पनि ध्यानमा राख्छौं भने \cite{37}, यो स्पष्ट हुन्छ कि सबै सार्वजनिक डलर लेखा केवल दोहोरो बोलीको हास्यास्पद कुरा हो, र अमेरिकी मुद्रा तथा सरकार यसको शाही मालिकहरूका लागि स्रोत व्यवस्थापन प्रणाली मात्र हो, जहाँबाट उनीहरूले चाहेको जति सजिलैसँग लुकाएर निकाल्न सक्छन् (वा वास्तवमा, खन्ति निकाल्न सक्छन्)।
\section{Progeny of Jove: छायाँमा रहेका पश्चिमी राजाहरूको पहिचान}

त्यसो भए, को हो वास्तवमा सबैलाई चलाउने? हामी पक्का जान्न सक्दैनौं, किनकि पूँजीका पश्चिमी राजा आफैलाई छायाँमा राख्छन्। त्यहाँ धेरै प्रकारका सिद्धान्तहरू छन्, सार्वजनिक व्यक्तित्वदेखि लिएर बाह्यग्रही प्राणीहरू सम्म, तर मेरो विचारमा यसको उत्कृष्ट उत्तर "Amallulla" उपनाम लिएका एक गुमनाम ब्लगरको जीवन कार्यमा पाइन्छ। उनको कार्यमा २० भन्दा बढी लेखकहरू र ५० "अपूरणीय" दस्तावेजहरूको विशाल संश्लेषण थियो जसले प्राचीन र आधुनिक इतिहास, गोप्य प्रतीकवाद, र पश्चिमी राजनीति जस्ता विषयहरू समेटेको छ \cite{33,34}। उनको कार्यलाई म आउने भौगोलिक प्रलय सम्बन्धमा "द्रष्टव्य" भन्नसक्छु - यो \textit{महत्वपूर्ण रूपमा} मेरो भन्दा धेरै व्यापक छ।

Amallulla ले तीनवटा पश्चिमी राजनीतिक गुटहरू पहिचान गरे, जसलाई उनले सँगै "Progeny of Jove" भने, जसले "अन्त समय" - पृथ्वीका आवर्ती प्रलयहरूको ज्ञान राख्छन्। उनी विश्वास गर्थे कि यी तीन गुटहरूले आज पश्चिमी मुलुकहरूलाई सँगै नियन्त्रण गर्छन्, तर भिन्न उद्गम, ऐतिहासिक पहिचान, सम्भावित विगतका असमझदारीहरू, र उनीहरूको मूल्य प्रणाली र कार्यहरूमा देखिन्छ भिन्नताका आधारमा तीन फरक समूहमा विभाजित गरे।

यी तीनवटा गुटहरूलाई लगभग निम्न अनुसार वर्गीकृत गर्न सकिन्छ:

\begin{flushleft}
\begin{enumerate}
    \item \textbf{ब्याङ्करहरू}: प्राचीन रोमन उच्च वर्ग, जसले अमेरिका मा नाइट्स टेम्प्लर र नर्दन जुरिस्डिक्शन फ्रिमेसनमा रुपान्तरण गरे।
    \item \textbf{विद्वान्हरू}: रोजिकु्रसियन्स र दक्षिण अमेरिकी फ्रिमेसनहरू।
    \item \textbf{जेसुइट्स र कालो पोप}: रोमन क्याथोलिक चर्च भित्र जोभको सन्तानको एउटा गुट।
\end{enumerate}
\end{flushleft}

आज, यी तीन गुटहरू मिलेर युरोपेली इलुमिनाटी, फ्रिमेसन, र सीआईए बनेका छन्। अमल्लुलाले वर्णन गरे झैँ, \textit{"हाल समयको अन्त्यमा, जोभका सन्तान 'निड टु नो' क्लियरन्स पछाडि राम्ररी लुकेका छन् जसले गर्दा संयुक्त राज्य अमेरिकाका वर्तमान राष्ट्रपतिलाई समेत बाहिर राख्दछ। त्यो भन्नुको अर्थ, उनीहरूले आफूलाई सार्वजनिक अनुगमनबाट लुकाउने कला सिद्ध गरेका छन्। \textbf{जोभका सन्तानले केवल संयुक्त राज्य अमेरिकाको सेना र सरकारलाई मात्र नियन्त्रण गर्दैनन्, तर फिएट मुद्रा, ठूला कम्पनीहरू, र उनीहरूले आविष्कार गरेको गणराज्य सरकारको माध्यमबाट (जहाँ नेताहरू सजिलै भ्रष्ट हुने थाहा पाएर, जसले गर्दा नियन्त्रण गर्न सजिलो हुन्छ), उनीहरूले सम्पूर्ण पश्चिमी संसारलाई नियन्त्रण गर्छन्}"} \cite{33,34}.

\begin{figure}[t]
\begin{center}
% \fbox{\rule{0pt}{2in} \rule{0.9\linewidth}{0pt}}
   \includegraphics[width=1\linewidth]{illuminati.jpg}
\end{center}
   \caption{जवका सन्तान को हुन् त? (चित्र: \cite{35})}
\label{fig:10}
\label{fig:onecol}
\end{figure}

\begin{figure}[t]
\begin{center}
% \fbox{\rule{0pt}{2in} \rule{0.9\linewidth}{0pt}}
   \includegraphics[width=1\linewidth]{pike.jpg}
\end{center}
   \caption{प्रसिद्ध पाइक पीक बाथोलिथ, रातो रंगमा देखाइएको, र संयुक्त राज्य अमेरिकाको पश्चिमी भूभाग \cite{36}। के संयुक्त राज्य अमेरिका साँच्चै यस स्थानलाई नियन्त्रण गर्न सिर्जना गरिएको थियो?}
\label{fig:11}
\label{fig:onecol}
\end{figure}

अमालुल्लाका अनुसार, यी मानिसहरूले धर्मलाई तिरस्कार गर्छन्, विश्वका प्रमुख धर्मका पवित्र ग्रन्थहरूलाई आफ्नै फाइदाका लागि चलाउँछन्, र प्रतीकात्मकता अत्यधिक प्रयोग गर्छन्। अतिरिक्त रूपमा, उनीहरू आफ्ना शत्रुप्रति निर्दयी छन्: \textit{"\textbf{२,६०० वर्षभन्दा बढी समयको अवधिमा, उनीहरूले अन्त्य समयको विशेष ज्ञान भएका अरू सबैलाई योजनाबद्ध रूपमा समाप्त गरे। र यसबाट मेरो मतलब केवल ड्रुइड, यहूदी काब्बालिस्ट, प्राचीन इजिप्टियन, अरबी, र भारतीय योगीहरू मात्र होइन, तर दक्षिण अमेरिकाका लम्बोतास भएका खोपडीहरू र मध्य अमेरिकाका माया पुजारीहरू पनि हुन्। र प्रमाण छ कि उनीहरूले केही समय पहिलेसम्म फस्टाएको उत्तर अमेरिकाको जनसंख्यालाई पूर्ण रूपमा नष्ट गरे ताकि यसलाई अन्त्य समयको भूमिको रुपमा सुरक्षित गर्न सकियोस् भन्ने, अत्यन्तै आश्चर्यजनक छ। अमेरिकी “इन्डियन”हरूको जातीय संहार त केवल सफा गर्ने अभियान मात्र थियो}"} \cite{33,34}.
Amallulla ले पनि विश्वास गर्थिन् कि सम्पूर्ण "संयुक्त राज्य अमेरिका" परियोजना "पाइक्स पिक बाथोलिथ" को नियन्त्रण सुरक्षित गर्ने उद्देश्यका लागि गरिएको थियो, जुन रकी पर्वत शृंखलामा पर्ने एउटा ग्रानाइट पर्वत शृंखला हो जसले भू-भौतिक प्रकोपबाट उत्कृष्ट सुरक्षा दिन्छ (Figure \ref{fig:11})। Amallulla का अनुसार, \textit{"हामीले गृह युद्धका रूपमा चिन्ने घटनाको अघि, समयमा, र पछि, बैंकर्स र विचारकहरूले संयुक्त राज्य अमेरिकाको नियन्त्रणको लागि भन्दा बढी पाइक्स पिक बाथोलिथको नियन्त्रणको लागि लडेका छन्, जुन संसारकै सबैभन्दा अद्वितीय ग्रानाइट बाथोलिथहरूमध्ये एक हो... संसारमा अन्य कुनै पनि यति उच्च उचाइमा तथा समुद्र किनारदेखि यति टाढा यस्तो ग्रानाइट बाथोलिथ छैन। यो पृथ्वीको सतह विस्थापनबाट बाँच्नको लागि आदर्श स्थान हो"} \cite{33,34}। Amallulla को अनुसन्धानले देखायो कि हाल यो क्षेत्रमा मुनि र वरिपरि विस्तृत भूमिगत सुरुङ प्रणाली बनाइएको छ \cite{36}।

\section{निष्कर्ष}

यस पत्रमा मैले विभिन्न साक्ष्यहरूको विवरण दिएको छु जसले पश्चिमी कुलीनहरूले हजारौं वर्षदेखि पृथ्वीका आवृत्त प्रकोपहरूको ज्ञानलाई सावधानीपूर्वक जोगाइराखेको, अर्को प्रकोप नजिकै आएको विश्वास गरेको, यस्ता घटनाका लागि व्यापक भूमिगत आश्रयहरू निर्माण गरेको, र यो अवसरलाई राजनीतिक तथा सैनिक रूपमा प्रयोग गरी विश्व नियन्त्रणमा लिन योजना बनाएको संकेत गर्छन्। मैले अमेरिकामा यो कसरी वित्त पोषित भयो भन्ने केही संकेतहरू उल्लेख गरेको छु, साथै यो सब चलाइरहेका विशिष्ट रगतको नशा को हुन् भन्ने विषयमा सबैभन्दा कम असम्भव सिद्धान्तको पनि उल्लेख गरेको छु। थप जान्न चाहनेको लागि, मैले उल्लेख नगरेको थुप्रै थप जानकारी सन्दर्भहरूमा खोजेर पाउन सकिन्छ।

आउँदो भू-भौतिक घटनाको सबैभन्दा बलियो मापन गर्न सकिने डाटा बिन्दु पृथ्वीको छिटो सर्दै गएको भू-चुम्बकीय क्षेत्र हो। यो केवल चुम्बकीय उत्तर ध्रुवको तेज गतिको सर्नु (Figure \ref{fig:13}) र दक्षिण एटलान्टिक भू-चुम्बकीय असामान्यताको वृद्धि भएमा मात्र होइन, पछिल्लो ४०० वर्षमा भू-चुम्बकीय क्षेत्रको तीव्र रुपमा कमजोरी र विकृति बढेको तथ्यमा पनि देखिन्छ \cite{3}। यस्ता वैज्ञानिक तथ्यहरू मेरो पहिलो दुई ECDO पत्रहरूमा विस्तारमा छलफल गरिएको छ, जुन मेरो वेबसाइटमा उपलब्ध छन् \cite{3}।

\begin{figure}[t]
\begin{center}
% \fbox{\rule{0pt}{2in} \rule{0.9\linewidth}{0pt}}
   \includegraphics[width=1\linewidth]{npw.jpg}
\end{center}
   \caption{सन १५९० देखि २०२५ सम्मको भू-चुम्बकीय उत्तर ध्रुवको स्थिति, ५-५ वर्षको अन्तरमा देखाइएको छ \cite{41}। यसको गति सन् १९७५ मा तीव्र गतिमा बढ्न थाल्यो।}
\label{fig:13}
\label{fig:onecol}
\end{figure}

अन्त्यमा, म तपाईंलाई भविष्यवाणी गर्ने अमलुलाबाट यो उद्धरण छोड्न चाहन्छु, जसले कसरी \textit{"\textbf{सबै एकै कुरा हुन्}"} भनेर व्याख्या गर्छ: \textit{"यहाँ मलाई तपाईंको कल्पनालाई चरम सीमासम्म पु-याउनु आवश्यक छ। तपाईंले अहिले बाँचिरहनुभएको, बचपनदेखि चिनेको संसारलाई बिर्सनैपर्छ। त्यसलाई छोडिदिनुहोस्। त्यो पूर्ण रूपमा बनावटी वास्तविकता हो, जुन म्याट्रिक्स फिल्ममा देखाइएको जत्तिकै छ र तपाईंलाई अन्तिम क्षणसम्म निद्रामा राख्न बनाइएको हो। कहिलेकाँही मलाई लाग्छ, म कुनै चलचित्रको लागि स्क्रिप्ट लेख्दै छु, तर जुन कुरा म तपाईंहरूसँग यो वेबसाइटमा बाँड्दै छु, त्यो वास्तविक हो। मलाई “सबै एकै कुरा हो” भन्ने बुझ्न आधा दशकभन्दा बढी लाग्यो, जसलाई मैले छिटै An Apocalyptic Synthesis को नारा बनाएको थिएँ। यो बुझाउन सजिलो कुरा होइन। अहिलेका लागि, हामी म्याट्रिक्स फिल्मको सन्दर्भमा सोचौं। यो राम्रो उपमा हो। मलाई बुझाउन गाह्रो लागेको कुरा के हो भने म भन्न लागेको कुरा कुनै अतिशयोक्ति होइन। अहिलेका लागि, म्याट्रिक्स फिल्मको उपमा नै तपाईंलाई मैले भन्न लागेको कुराको डरलाग्दो वास्तविकता बुझाउन सबैभन्दा नजिकको माध्यम हो। \textbf{तपाईंको जीवनमा भएका सबै कुरा, पूरै लिखित इतिहास, मुख्यधारा, सहमति विज्ञान र शैक्षिक जगत, राजनीति, धर्म, सबै कुनै न कुनै रूपमा पृथ्वीको सतह सर्ने वा अक्षीय झुकाव आउन लागेको विषयसँग सम्बन्धित छ।} अहिले तपाईंले देख्नुहुन्न। नत तपाईं एक नराम्रो सपनाबाट ब्युँझिएझैँ यो वास्तविकतामा सजिलै जाग्न सक्ने अवस्था छ। यसलाई बुझ्न समय लाग्छ। तर म तपाईंलाई वाचा गर्छु, यो यात्राको अन्त्यमा तपाईंले बुझ्नुहुनेछ कि तपाईं पूरै जीवन म्याट्रिक्स कम्प्युटर-सिमुलेटेड वास्तविकतामा बाँचिरहनुभएको रहेछ"} \cite{33,34}.
Good luck to all.

\section{धन्यवाद}

सार्वजनिक क्षेत्रमा ज्ञान योगदान गर्न रोज्ने सबै व्यक्तिहरूलाई धन्यवाद। तपाईंहरू बिना, यो कार्य सम्भव हुने थिएन र मानवता अझै अन्धकारमै रहने थियो। तपाईंहरूको चुनावले सदा लागि फुल्नेछ। हामी तपाईंहरूलाई सबै कुराको ऋणी छौं, र म अनन्त कृतज्ञ छु।

\clearpage
\twocolumn

{\small
\bibliographystyle{ieee}
\bibliography{egbib}
}

\end{document}
\documentclass[10pt,twocolumn,letterpaper]{article}

\usepackage{booktabs}
% \usepackage{caption}
% \captionsetup[table]{skip=8pt}   % Only affects tables
\usepackage{stfloats}  % Add this to the preamble
\usepackage{float}

%–– line-breaks for Lao instead of Thai ––
\XeTeXlinebreaklocale "lo"
\XeTeXlinebreakskip = 0pt plus 1pt

\usepackage{fontspec}
\usepackage{ucharclasses}

%–– define your two fonts ––
\newfontfamily\latinfont{Latin Modern Roman}         % for all non-Lao (e.g. Latin) text
\newfontfamily\laofont[Script=Lao]{Noto Serif Lao}   % for all Lao text
% \newfontfamily\laofont[Script=Lao]{FreeSerif}   % for all Lao text

%–– ucharclasses auto-detects Unicode blocks ––
\setDefaultTransitions{\latinfont}{}                   % outside Lao → Latin
\setTransitionsFor{Lao}{\laofont}{\latinfont}          % Lao block → Lao font, then back

\usepackage{cvpr}
\usepackage{times}
\usepackage{epsfig}
\usepackage{graphicx}
\usepackage{amsmath}
\usepackage{amssymb}

\usepackage[breaklinks=true,bookmarks=false]{hyperref}

\cvprfinalcopy % *** Uncomment this line for the final submission

\def\cvprPaperID{****} % *** Enter the CVPR Paper ID here
\def\httilde{\mbox{\tt\raisebox{-.5ex}{\symbol{126}}}}

% 1) Choose your desired fixed leading:
\renewcommand\baselinestretch{1.2}  % or 1.3, 1.1…  adjust to taste

% 2) Force TeX to *always* use \baselineskip, never fall back to \lineskip:
\makeatletter
  \setlength\lineskiplimit{-\maxdimen} % always allow baselineskip
  \setlength\lineskip{0pt}             % no extra glue ever
\makeatother

% \renewcommand{\tablename}{ตาราง}
\renewcommand{\figurename}{ຮູບພາບ}   % or whatever you like instead of "Hình"
\renewcommand{\refname}{ເອກະສານອ້າງອີງ}

\makeatletter
\def\abstract{%
  \centerline{\large\bf ບົດສະຫຼຸບຫຍໍ້}% <-- your new label
  \vspace*{12pt}%
  \it%
}
\makeatother

% This makes the font slightly bigger than base (10) and bold in Subsection headings rather than using ptmb
\makeatletter
\def\cvprsubsection{%
  \@startsection{subsection}{2}{\z@}%
    {8pt plus 2pt minus 2pt}{6pt}%
    % {\normalfont\bfseries\selectfont}%
    {\normalfont\bfseries\fontsize{11}{13}\selectfont}%
}
\makeatother

% So this hardcodes the style for the numbers in the section/subsection headings so they're bold
\font\elvbf=ptmb scaled 1100
\font\elvbfs=ptmb scaled 1200
\makeatletter
% Section number: Large + bold
\renewcommand\thesection{%
  {\elvbfs\arabic{section}}%
}

% Subsection number: normalsize + bold + custom punctuation
\renewcommand\thesubsection{%
  {\elvbf
   \arabic{section}.\arabic{subsection}}%
}
\makeatother

% Pages are numbered in submission mode, and unnumbered in camera-ready
%\ifcvprfinal\pagestyle{empty}\fi
\setcounter{page}{1}
\begin{document}

\title{ເອກະສານ ECDO ສະບັບທີ 3: ຫຼັກຖານການກະກຽມຂອງມະຫາອຳນາດປົກຄອງຕາເວັນຕົກໃນປັດຈຸບັນ ສໍາລັບໄພພິບັດທາງທໍລະນີຟີຊິກທີ່ຈະມາເຖິງໃນໄວໆນີ້}

\author{Junho\\
ເຜີຍແຜ່ເດືອນມິຖຸນາ 2025\\
ເວັບໄຊ (ດາວໂຫລດເອກະສານໄດ້ທີ່ນີ້): \href{https://sovrynn.github.io}{sovrynn.github.io}\\
ຄັງເກັບກຳຂໍ້ມູນວິໄຈ: \href{https://github.com/sovrynn/ecdo}{github.com/sovrynn/ecdo}\\
{\tt\small junhobtc@proton.me}
}

\maketitle
%\thispagestyle{empty}

\begin{abstract}
ໃນເດືອນພຶດສະພາ 2024, ນັກຂຽນອອນລາຍນິລະນາມຜູ້ໜຶ່ງທີ່ຮູ້ຈັກກັນໃນນາມ ”The Ethical Skeptic” \cite{0} ໄດ້ແບ່ງປັນທິດສະດີໃໝ່ທີ່ກ້າວໜ້າ ເຊິ່ງເອີ້ນວ່າ ການສັ່ນສະເທືອນແບບ Dzhanibekov ຂອງການແຍກຕົວຂອງແກນໂລກ ແລະ ເປືອກໂລກທີ່ປ່ອຍຄວາມຮ້ອນອອກມາ (ECDO) \cite{1}.
ທິດສະດີນີ້ຊີ້ໃຫ້ເຫັນວ່າ ໂລກເຄີຍປະສົບກັບການປ່ຽນແປງກະທັນຫັນມາແລ້ວ, ການປ່ຽນແປງທີ່ຮ້າຍແຮງຂອງແກນໝູນວຽນຂອງມັນ, ເຊິ່ງກໍ່ໃຫ້ເກີດນໍ້າຖ້ວມໃຫຍ່ທົ່ວໂລກ ເນື່ອງຈາກມະຫາສະໝຸດໄດ້ໄຫຼທະລັກເຂົ້າຖ້ວມທະວີບຕ່າງໆ ຍ້ອນແຮງເໜັງຕີງຂອງການໝູນວຽນ. ນອກຈາກນີ້, ມັນຍັງນຳສະເໜີຂະບວນການທາງທໍລະນີຟີຊິກທີ່ອະທິບາຍໄດ້ ແລະ ຂໍ້ມູນທີ່ຊີ້ບອກວ່າ ການປ່ຽນແປງແບບນີ້ອີກຄັ້ງອາດຈະເກີດຂຶ້ນໃນໄວໆນີ້. 
ເຖິງແມ່ນວ່າຄຳທຳນາຍກ່ຽວກັບນໍ້າຖ້ວມໃຫຍ່ ແລະ ວັນສິ້ນໂລກແບບນີ້ບໍ່ແມ່ນເລື່ອງໃໝ່ກໍຕາມ, ແຕ່ທິດສະດີ ECDO ມີຄວາມໜ້າສົນໃຈເປັນພິເສດ ເນື່ອງຈາກລັກສະນະທາງວິທະຍາສາດຂອງມັນ, ທີ່ທັນສະໄໝ, ມີຫຼາຍສາຂາວິຊາ ແລະ ໃຊ້ຂໍ້ມູນເປັນຫຼັກ.
ເອກະສານສະບັບນີ້ແມ່ນຜົນງານທີສາມຂອງຂ້າພະເຈົ້າ \cite{2,3} ໃນຫົວຂໍ້ນີ້, ແລະ ເນັ້ນໜັກໃສ່ດ້ານການເມືອງໃນປັດຈຸບັນຂອງທິດສະດີນີ້: 
\begin{flushleft}
\begin{enumerate}
    \item ຄໍາໃຫ້ການຂອງຜູ້ເປີດເຜີຍຂໍ້ມູນຊີ້ວ່າ ບັນດາອໍານາດຕາເວັນຕົກເຊື່ອວ່າໄພພິບັດທາງທໍລະນີຟີຊິກກຳລັງຈະເກີດຂຶ້ນໃນໄວໆນີ້ ແລະ ພວກເຂົາມີແຜນທີ່ຈະໃຊ້ປະໂຫຍດທາງດ້ານການເມືອງ ແລະ ການທະຫານຈາກເຫດການດັ່ງກ່າວ.
    \item ຫຼັກຖານຂອງຖານທັບໃຕ້ດິນ ແລະ ໃຕ້ນໍ້າຂະໜາດໃຫຍ່ຂອງຝ່າຍຕາເວັນຕົກ ທີ່ຖືກສ້າງຂຶ້ນເພື່ອກະກຽມຮັບມືກັບເຫດການດັ່ງກ່າວ.
    \item ຫຼັກຖານຂອງເງິນຈໍານວນມະຫາສານທີ່ຖືກດຶງອອກຈາກລະບົບເງິນຕາຂອງຝ່າຍຕາເວັນຕົກ ເພື່ອໃຊ້ໃນການກໍ່ສ້າງຖານທັບເຫຼົ່ານີ້. 
\end{enumerate}
\end{flushleft}

ເອກະສານສະບັບນີ້ໄດ້ບັນທຶກການກະກຽມຢ່າງກວ້າງຂວາງທີ່ບັນດາອຳນາດປົກຄອງຕາເວັນຕົກກຳລັງດໍາເນີນການ ເພື່ອກະກຽມຮັບມືກັບໄພພິບັດທາງທໍລະນີຟີຊິກທີ່ພວກເຂົາເຊື່ອວ່າກຳລັງຈະເກີດຂຶ້ນໃນໄວໆນີ້.
\end{abstract}

\section{ສະມາຄົມຟຣີເມສັນ ແລະ "ພາລະກິດແອງໂກຼ-ແຊັກຊົງ"}

ໃນເດືອນມັງກອນ 2010, ໂຄງການ Camelot, ເຊິ່ງເປັນອົງກອນສື່ທາງເລືອກ ແລະ ໜັງສືພິມທີ່ລວບລວມຄຳໃຫ້ການຂອງຜູ້ເປີດເຜີຍຄວາມລັບ, ໄດ້ສໍາພາດ \cite{4,6} ຜູ້ຢູ່ວົງໃນຄົນໜຶ່ງ ຜູ້ທີ່ໄດ້ເຂົ້າຮ່ວມກອງປະຊຸມຂອງບັນດາ Senior Masons ໃນ ເມືອງລອນດອນ ດ້ວຍຕົນເອງ ໃນເດືອນມິຖຸນາ ປີ 2005.
ຫົວຂໍ້ທີ່ປຶກສາຫາລືໃນກອງປະຊຸມແມ່ນແຜນການທາງທະຫານ ແລະ ການເມືອງ ໂດຍມີພື້ນຖານຢູ່ເທິງ \textbf{"ເຫດການທາງທໍລະນີຟີຊິກ"}, ທີ່ຈະມາເຖິງ, ເຊິ່ງກໍຄືໄພພິບັດທໍາມະຊາດທົ່ວໂລກ. 

\begin{figure}[b]
\begin{center}
   \includegraphics[width=1\linewidth]{freemason.jpg}
\end{center}
   \caption{ກຸ່ມຟຣີເມສັນຊາວອັງກິດໃນສະພາບປົກກະຕິຂອງພວກເຂົາ, ກຳລັງວາງແຜນຢ່າງລັບໆທີ່ຈະຖິ້ມລະເບີດນິວເຄລຍ ແລະ ຍຶດຄອງໂລກ - ທີ່ Earls Court ໃນລອນດອນ, ປີ 1992 \cite{5}.}
\label{fig:1}
\label{fig:onecol}
\end{figure}

\begin{figure*}[t]
\begin{center}
\includegraphics[width=1\textwidth]{british.jpg}
\end{center}
   \caption{ອານາຈັກອັງກິດໃນປີ 1937, ເປັນການສະແດງອອກອັນຍິ່ງໃຫຍ່ຂອງອຳນາດຊາວແອງໂກຼ-ແຊັກຊົງ \cite{14}.}
   \label{fig:2}
\end{figure*}

ອີງຕາມຜູ້ຢູ່ວົງໃນຄົນນີ້, ຜູ້ເຂົ້າຮ່ວມກອງປະຊຸມ 25-30 ຄົນແມ່ນ \textit{"...ລ້ວນແຕ່ເປັນຄົນອັງກິດ, ແລະ ບາງຄົນໃນນັ້ນແມ່ນບຸກຄົນທີ່ມີຊື່ສຽງທີ່ຄົນໃນສະຫະລາດຊະອານາຈັກຈະຮູ້ຈັກທັນທີ... ມີຊົນຊັ້ນສູງຢູ່ບາງສ່ວນ, ແລະ ບາງຄົນກໍມາຈາກພື້ນຖານຄອບຄົວຊົນຊັ້ນສູງພໍສົມຄວນ. ມີຄົນໜຶ່ງທີ່ຂ້າພະເຈົ້າຈຳແນກໄດ້ໃນກອງປະຊຸມນັ້ນ ເຊິ່ງເປັນນັກການເມືອງອາວຸໂສ. ອີກສອງຄົນເປັນບຸກຄົນສຳຄັນຈາກຕຳຫຼວດ, ແລະ ອີກຄົນໜຶ່ງແມ່ນມາຈາກກອງທັບ. ທັງສອງຄົນເປັນທີ່ຮູ້ຈັກທົ່ວປະເທດ ແລະ ທັງສອງແມ່ນບຸກຄົນສຳຄັນໃນການໃຫ້ຄຳປຶກສາແກ່ລັດຖະບານຊຸດປັດຈຸບັນ — ໃນເວລານີ້"} \cite{4}.
ຜູ້ຢູ່ວົງໃນໄດ້ກ່າວວ່າ ລາວໄດ້ເຂົ້າຮ່ວມກອງປະຊຸມ,\ \textit{"ໂດຍບັງເອີນຢ່າງສິ້ນເຊີງ! ຂ້າພະເຈົ້າຄິດວ່າມັນເປັນກອງປະຊຸມປົກກະຕິທີ່ຈັດຂຶ້ນທຸກໆສາມເດືອນ... ຂ້າພະເຈົ້າໄດ້ໄປຮ່ວມກອງປະຊຸມນີ້ ແລະ ມັນບໍ່ແມ່ນກອງປະຊຸມທີ່ຂ້າພະເຈົ້າຄາດຫວັງໄວ້. ຂ້າພະເຈົ້າເຊື່ອວ່າຂ້າພະເຈົ້າຖືກເຊີນ... ຍ້ອນຕໍາແໜ່ງທີ່ຂ້າພະເຈົ້າເຄີຍດໍາລົງ ແລະ ຍ້ອນວ່າພວກເຂົາເຊື່ອວ່າ, ເປັນເຫມືອນພວກເຂົາ, ຂ້າພະເຈົ້າແມ່ນຄົນໜຶ່ງໃນກຸ່ມຂອງພວກເຂົາ."} \cite{4}.
ລໍາດັບເຫດການພື້ນຖານທີ່ໄດ້ປຶກສາຫາລືໃນກອງປະຊຸມ (ໃນປີ 2005) ມີດັ່ງນີ້: 

\begin{flushleft}
\begin{enumerate}
    \item ຍຸຍົງໃຫ້ອີຣ່ານ ຫຼື ຈີນ ໃຊ້ອາວຸດນິວເຄລຍທາງຍຸດທະວິທີ ແລະ ນໍາໄປສູ່ການແລກປ່ຽນການໂຈມຕີດ້ວຍນິວເຄລຍແບບຈໍາກັດ, ແລ້ວຈຶ່ງສະຖາປານາຂໍ້ຕກລງການຢຸດຍິງ.
    \item ການປ່ອຍອາວຸດຊີວະພາບໃສ່ຈີນ, ເຊິ່ງມີລາຍງານວ່າເປັນເປົ້າໝາຍຫຼັກ "ຕັ້ງແຕ່ທົດສະວັດປີ 1970".
    \item ການນໍາພາລັດຖະບານທະຫານຜະເດັດການເຂົ້າສູ່ອໍານາດ ໂດຍອ້າງເຖິງຄວາມຢ້ານກົວ ແລະ ຄວາມວຸ້ນວາຍທີ່ເກີດຂຶ້ນ.
\end{enumerate}
\end{flushleft}

ແຕ່ສິ່ງທີ່ສຳຄັນທີ່ສຸດແມ່ນສິ່ງທີ່ຄາດວ່າຈະເກີດຂຶ້ນຫຼັງຈາກເຫດການເຫຼົ່ານີ້: \textit{"ສະນັ້ນ ພວກເຮົາຈະກຳລັງຈະເຂົ້າສູ່ສົງຄາມນີ້, ແລ້ວຫຼັງຈາກນັ້ນ... ຈະມີເຫດການທາງທໍລະນີຟີຊິກເກີດຂຶ້ນເທິງໂລກ ເຊິ່ງຈະສົ່ງຜົນກະທົບຕໍ່ທຸກຄົນ"} \cite{4}.
ຜູ້ຢູ່ວົງໃນເຊື່ອວ່າ ໃນລະຫວ່າງເຫດການທາງທໍລະນີຟີຊິກນີ້, \textit{\textbf{"ເປືອກໂລກຈະເລື່ອນປະມານ 30 ອົງສາ, ປະມານ 1700 ຫາ 2000 ໄມລ໌ ລົງໄປທາງໃຕ້, ແລະ ມັນຈະເຮັດໃຫ້ເກີດຄວາມປັ່ນປ່ວນຢ່າງຫຼວງຫຼາຍ, ເຊິ່ງຜົນກະທົບຂອງມັນຈະຄົງຢູ່ເປັນເວລາດົນນານ"}} \cite{4}. 

ເຫດຜົນສໍາລັບການວາງແຜນລັບໆທັງໝົດນີ້, ແນ່ນອນ, ແມ່ນອຳນາດ.
ຜູ້ຢູ່ວົງໃນໄດ້ອະທິບາຍວ່າ, \textit{"ໃນເວລານັ້ນ, ພວກເຮົາທຸກຄົນຈະໄດ້ຜ່ານພົ້ນສົງຄາມນິວເຄລຍ ແລະ ຊີວະພາບມາແລ້ວ. ຖ້າເຫດການນີ້ເກີດຂຶ້ນ, ປະຊາກອນໂລກຈະຫຼຸດລົງຢ່າງຫຼວງຫຼາຍ. ເມື່ອເຫດການທາງທໍລະນີຟີຊິກນີ້ຈະເກີດຂຶ້ນ, ແລ້ວຜູ້ທີ່ຍັງເຫຼືອຢູ່ກໍອາດຈະຫຼຸດລົງອີກເຄິ່ງໜຶ່ງ. ແລະ ໃຜທີ່ລອດຊີວິດຈາກເຫດການນັ້ນ ຈະເປັນຜູ້ກໍານົດວ່າໃຜຈະນຳພາໂລກ ແລະ ປະຊາກອນທີ່ຍັງເຫຼືອເຂົ້າສູ່ຍຸກຕໍ່ໄປ. ສະນັ້ນ, ພວກເຮົາກຳລັງເວົ້າເຖິງຍຸກຫຼັງເຫດການໄພພິບັດ. ໃຜຈະເປັນຜູ້ນຳ? ໃຜຈະເປັນຜູ້ຄວບຄຸມ? ສະນັ້ນ, ມັນແມ່ນທັງໝົດກ່ຽວກັບເລື່ອງນັ້ນ. ແລະ ນັ້ນຄືເຫດຜົນທີ່ພວກເຂົາຮີບຮ້ອນຢ່າງຍິ່ງ ທີ່ຈະໃຫ້ສິ່ງເຫຼົ່ານີ້ເກີດຂຶ້ນພາຍໃນຂອບເຂດເວລາທີ່ກຳນົດໄວ້... ໂຄງສ້າງຈຳເປັນຕ້ອງຖືກຈັດຕັ້ງໄວ້ກ່ອນທີ່ [ຄວາມວຸ້ນວາຍ] ຈະເກີດຂຶ້ນ ເພື່ອໃຫ້ໝັ່ນໃຈວ່າໂຄງສ້າງຈະສາມາດຢູ່ລອດຈາກສິ່ງທີ້ກຳລັງຈະເກີດຂື້ນ - ທັງສາມາດຟື້ນຕົວໄດ້ຢ່າງໝັ້ນຄົງໃນມື້ຕໍ່ມາ, ແລະ ຈາກນັ້ນກໍຍັງຄົງຢູ່ໃນອຳນາດ ແລະ ມີອຳນາດທີ່ເຄີຍມີມາກ່ອນ"} \cite{4}.
ໃນລະຫວ່າງການສຳພາດ, ໄດ້ມີການກ່າວເຖິງຊື່ "ພາລະກິດແອງໂກຼ-ແຊັກຊົງ", ກໍໄດ້ຖືກປຶກສາຫາລືຄືກັນ: \textit{[ຜູ້ສຳພາດໄດ້ກ່າວວ່າ]: "...ເຫດຜົນທີ່ມັນຖືກເອີ້ນວ່າ "ພາລະກິດແອງໂກຼ-ແຊັກຊົງ" ນັ້ນກໍຍ້ອນວ່າແຜນການພື້ນຖານແມ່ນການກຳຈັດຊາວຈີນໃຫ້ຫມົດໄປ ເພື່ອວ່າຫຼັງຈາກໄພພິບັດ ແລະ ເມື່ອສິ່ງຕ່າງໆຖືກສ້າງຂຶ້ນໃໝ່, ມັນຈະເປັນຊາວແອງໂກຼ-ແຊັກຊົງ ຜູ້ທີ່ຈະຢູ່ໃນຖານະສ້າງສາຄືນໃໝ່ ແລະ ສືບທອດໂລກໃໝ່, ໂດຍບໍ່ມີຄົນອື່ນໆເຫຼືອຢູ່ເລຍ. ນັ້ນແມ່ນແທ້ບໍ?"}
\textit{[ຜູ້ຢູ່ວົງໃນ]: "ວ່ານັ້ນຖືກຕ້ອງຫຼືບໍ ຂ້າພະເຈົ້າບໍ່ຮູ້ແທ້ໆ, ແຕ່ຂ້າພະເຈົ້າເຫັນດີກັບທ່ານ. ຢ່າງໜ້ອຍກໍຕະຫຼອດສະຕະວັດທີ 20, ແລະ ແມ້ແຕ່ກ່ອນໜ້ານັ້ນໃນສະຕະວັດທີ 19 ແລະ 18, ປະຫວັດສາດຂອງໂລກນີ້ສ່ວນໃຫຍ່ແມ່ນຖືກຄຸ້ມຄອງຈາກຝັ່ງຕາເວັນຕົກ ແລະ ພາກເໜືອຂອງໂລກ"} \cite{4}. 
ກ່ຽວກັບກຳນົດເວລາທີ່ແນ່ນອນຂອງເຫດການທໍລະນີຟີຊິກທີ່ຄາດວ່າຈະເກີດຂຶ້ນ, ຜູ້ຢູ່ວົງໃນໄດ້ໃຫ້ການຄາດຄະເນທີ່ດີທີ່ສຸດຂອງລາວວ່າ: \textit{"...ຄວາມຮູ້ສຶກ, ແລະ ມັນເປັນສັນຊາດຕະຍານທີ່ດີ, ຄືວ່າພວກເຂົາຕ້ອງຈັດການສິ່ງຕ່າງໆໃຫ້ຮຽບຮ້ອຍໃນຕອນນີ້... ຂ້າພະເຈົ້າຄິດວ່າພວກເຂົາຮູ້ດີວ່າເຫດການນັ້ນຈະເກີດຂຶ້ນເມື່ອໃດ... \textbf{ຂ້າພະເຈົ້າມີຄວາມຮູ້ສຶກທີ່ແຮງກ້າຫຼາຍວ່າ ມັນຈະເກີດຂຶ້ນໃນຊ່ວງຊີວິດຂອງຂ້າພະເຈົ້າ, ເວົ້າອີກຢ່າງໜຶ່ງຄືພາຍໃນ 20 ປີ}... ຕອນນີ້ພວກເຮົາໄດ້ເຂົ້າສູ່ຊ່ວງເວລາທີ່ເຫດການທໍລະນີຟີຊິກນີ້ກຳລັງຈະເກີດຂຶ້ນແລ້ວ, ເມື່ອເຮົາພິຈາລະນາເຖິງໄລຍະເວລາທີ່ຜ່ານມາຕັ້ງແຕ່ຄັ້ງສຸດທ້າຍເຊິ່ງເກີດຂຶ້ນປະມານ 11,500 ປີກ່ອນ, ແລະ ມັນເກີດຂຶ້ນປະມານທຸກໆ 11,500 ປີ, ເປັນວົງຈອນ. ຕອນນີ້ມັນເຖິງກຳນົດທີ່ຈະເກີດຂຶ້ນອີກແລ້ວ... ພວກເຂົາເຂົ້າໃຈວ່າມັນຈະເກີດຂຶ້ນ. ພວກເຂົາມີຄວາມຮູ້ທີ່ແນ່ນອນວ່າມັນຈະເກີດຂຶ້ນ... ອີກເທື່ອໜຶ່ງ, ມັນເປັນສິ່ງໜຶ່ງໃນບັນດາສິ່ງເຫຼົ່ານີ້—ມັນຄົງເປັນສິ່ງທີ່ຄິດບໍ່ເຖິງເລີຍ ຖ້າພວກເຂົາບໍ່ຮູ້. ຂ້າພະເຈົ້າໝາຍຄວາມວ່າ, ສະໝອງແນວຫນ້າຂອງໂລກກຳລັງເຮັດວຽກໃຫ້ພວກເຂົາໃນເລື່ອງນີ້"} \cite{4}. 
ນີ້ແມ່ນຄຳໃຫ້ການທີ່ຊົງພະລັງ ເຊິ່ງພວກເຮົາຄວນຈະຮູ້ສຶກຂອບໃຈເປັນຢ່າງຍິ່ງ. ໃນບົດສໍາພາດ, ຜູ້ຂຽນຍັງໄດ້ສົນທະນາກ່ຽວກັບຄວາມເຊື່ອຂອງລາວວ່າສົງຄາມໂລກຄັ້ງທີ 1 ແລະ ສົງຄາມໂລກຄັ້ງທີ 2 ເປັນສົງຄາມທີ່ຖືກສ້າງຂຶ້ນ, ແລະ ພາລະກິດແອງໂກຼ-ແຊັກຊົງ ເກືອບແນ່ນອນວ່າມີມາແຕ່ຫຼາຍໆລຸ້ນຄົນແລ້ວ. ປັດຈຸບັນນີ້ເປັນເວລາ 15 ປີແລ້ວນັບຕັ້ງແຕ່ການສໍາພາດ, ເຊິ່ງເກີດຂຶ້ນໃນປີ 2010. ຍັງເຫຼືອອີກຫ້າປີຈຶ່ງຈະຮອດກຳນົດເວລາ 20 ປີທີ່ຜູ້ຢູ່ວົງໃນໄດ້ຄາດຄະເນໄວ້ສຳລັບເຫດການທໍລະນີຟີຊິກ.
\subsection{ຄວາມຮູ້ລຶກລັບຂອງຊາວດຣູອິດຝ່າຍຕາເວັນຕົກກ່ຽວກັບໄພພິບັດຄັ້ງໃຫຍ່}

ຄວາມຮູ້ຂອງຊາວຕາເວັນຕົກກ່ຽວກັບໄພພິບັດທີ່ເກີດຂຶ້ນຊ້ຳໆ ໄດ້ຖືກເກັບຮັກສາໄວ້ເປັນຢ່າງດີ, ແລະ ບໍ່ພຽງແຕ່ຊາວຟຣີເມສັນເທົ່ານັ້ນ. ຊາວດຣູອິດ, ເຊິ່ງເປັນວັດທະນະທໍາຊາວແຄລຕິກບູຮານທີ່ມີການບັນທຶກໄວ້ເປັນຢ່າງດີ ທີ່ມີມາຢ່າງໜ້ອຍ 2400 ປີ  \cite{7}, ໄດ້ສືບທອດຄວາມຮູ້ກ່ຽວກັບໄພພິບັດທີ່ເກີດຂຶ້ນຊ້ຳໆ ຂອງໂລກ. ເຊື່ອກັນວ່າຊາວດຣູອິດຄົນສຸດທ້າຍທີ່ຮູ້ຈັກກັນແມ່ນ Ben McBrady.
ໃນສາລະຄະດີປີ 1992 ເລື່ອງ  "The Last Druid", ລາວໄດ້ແບ່ງປັນຂໍ້ມູນກ່ຽວກັບຄວາມຮູ້ຂອງຊາວດຣູອິດວ່າ: \textit{"ອົງກອນທີ່ຂ້າພະເຈົ້າອາດຈະເປັນສະມາຊິກຄົນສຸດທ້າຍຕາມທໍານຽມປະເພນີ, ໄດ້ເກີດຂຶ້ນຫຼັງຈາກໄພພິບັດຄັ້ງໃຫຍ່ສຸດ, ຫຼື ຫາຍະນະ, ເຊິ່ງສົ່ງຜົນກະທົບຕໍ່ໂລກ. ເມື່ອເກີດຜົນກະທົບອັນໃຫຍ່ຫຼວງ ແລະ ຮ້າຍແຮງເຫຼົ່ານີ້ຕໍ່ໂລກ ຈາກພາຍຸຝົນຟ້າຄະນອງຂະຫນາດໃຫຍ່, ຖືກດືງດູດເຂົ້າຫາງຂອງດາວຕົກ, ຫຼື ຝົນດາວຕົກ, ອະລິຍະທໍາທີ່ເຮົາຮູ້ຈັກໄດ້ຖືກທໍາລາຍລົງຢ່າງສິ້ນເຊີງ... ຄວາມຮູ້ທັງໝົດລ້ວນແຕ່ຢູ່ໃນຂອບເຂດຂອງກຸ່ມນີ້, ແຕ່ພວກເຂົາໄດ້ເອົາໃຈໃສ່ເປັນພິເສດກັບດາລາສາດ ເພາະວ່າພວກເຂົາໄດ້ປະສົບກັບໄພພິບັດທີ່ສໍາຄັນຫຼາຍຢ່າງ. ເປັນທີ່ເຊື່ອກັນວ່າ ຄວາມຮູ້ທາງດາລາສາດຢ່າງຄົບຖ້ວນ ຈະຊ່ວຍໃຫ້ພວກເຂົາສາມາດຄາດຄະເນສະພາບການເມື່ອໄພພິບັດເຫຼົ່ານີ້ມີແນວໂນ້ມທີ່ຈະເກີດຂຶ້ນ ແລະ ດໍາເນີນການບາງຢ່າງເພື່ອປົກປ້ອງຕົນເອງ. ຖ້າທ່ານເບິ່ງສະຖານທີ່ຫີນໃຫຍ່ບູຮານ (megalithic complexes) ອັນຍິ່ງໃຫຍ່ໃນປະເທດໄອແລນ, ທ່ານຈະເຫັນວ່າສິ່ງທີ່ຖືກອະທິບາຍວ່າເປັນສຸສານທາງເຂົ້າ (passage graves) ນັ້ນ ແທ້ຈິງແລ້ວແມ່ນບ່ອນຫຼົບໄພລະເບີດແບບດັ້ງເດີມໃນຍຸກບູຮານ. ພວກມັນຢູ່ສູງກວ່າລະດັບຄື້ນສຶນາມິທຸກລະດັບ ແລະ ພວກມັນຍັງໃຫ້ການປົກປ້ອງຈາກຝົນດາວຕົກອີກດ້ວຍ"} \cite{8,9}.
% ຍັງເຊື່ອກັນວ່າຟຣີເມສັນຣີເອງກໍ່ມີຕົ້ນກຳເນີດມາຈາກຊາວດຣູອິດ \cite{10}.

\section{ຫຼັກຖານການກະກຽມຮັບມືໄພພິບັດຄັ້ງໃຫຍ່ຂອງຝ່າຍຕາເວັນຕົກໃນຍຸກປັດຈຸບັນ}

ເນື່ອງຈາກວ່າອຳນາດປົກຄອງຢູ່ຂອງຝ່າຍຕາເວັນຕົກເບິ່ງຄືຈະເຊື່ອວ່າ ໄພພິບັດທໍລະນີຟີຊິກທົ່ວໂລກກຳລັງຈະເກີດຂຶ້ນໃນອີກບໍ່ດົນ, ພວກເຮົາຄາດຫວັງວ່າຈະມີການກະກຽມທີ່ສຳຄັນເກີດຂຶ້ນ ເພື່ອປົກປ້ອງຕົນເອງຈາກເຫດການດັ່ງກ່າວ. ແລະແທ້ຈິງແລ້ວ, ມີຫຼັກຖານເຜີຍແຜ່ສູ່ສາທາລະນະກ່ຽວກັບເຄືອຂ່າຍອຸໂມງໃຕ້ດິນເລິກທີ່ກວ້າງຂວາງໃນຫຼາຍປະເທດຕາເວັນຕົກ. ໃນຂະນະທີ່ສະຖານທີ່ດັ່ງກ່າວຈະສາມາດປົກປ້ອງຜູ້ຄົນໃນກໍລະນີສົງຄາມນິວເຄລຍ, ພວກມັນຍັງຈະເປັນບ່ອນປ້ອງກັນຈາກໄພພິບັດທໍາມະຊາດຫຼາຍຮູບແບບອີກດ້ວຍ.
ອີງຕາມຄຳໃຫ້ການຂອງຟຣີເມສັນອາວຸໂສຊາວອັງກິດຈາກໂຄງການ Camelot \cite{4,6}, ເບິ່ງຄືວ່າສະຖານະການເຫຼົ່ານີ້ບໍ່ແມ່ນພຽງແຕ່ຄວາມເປັນໄປໄດ້, ແຕ່ແມ່ນແຜນການທີ່ຖືກວາງໄວ້ລ່ວງໜ້າແລ້ວ. ສິ່ງທີ່ຄວນສັງເກດອີກຢ່າງໜຶ່ງຄືຈໍານວນເງິນມະຫາສານທີ່ຈະຕ້ອງໃຊ້ໃນການກໍ່ສ້າງ, ຈັດຫາບຸກຄະລາກອນ, ແລະ ບໍາລຸງຮັກສາຖານທັບເຫຼົ່ານີ້, ເຊິ່ງສອດຄ້ອງກັນດີກັບຈໍານວນເງິນຈໍານວນຫຼວງຫຼາຍເຖິງຫຼາຍສິບລ້ານລ້ານໂດລາທີ່ຫາຍໄປຈາກລັດຖະບານສະຫະລັດໃນໄລຍະ 18 ປີ (ຈະໄດ້ກ່າວເຖິງໃນພາກຕໍ່ໄປ) \cite{11,12,13}. 
ຕົວຢ່າງອື່ນໆຂອງການກະກຽມສຳລັບເຫດການລະດັບການສູນພັນ ລວມມີບັນດາໂຄງການເກັບຮັກສາຕ່າງໆ ເຊັ່ນ: ຄັງເກັບເມັດພັນ ແລະ ຄັງເກັບຄວາມຮູ້. 

\subsection{ຖານທັບໃຕ້ດິນ ແລະ ໃຕ້ທະເລຂອງອາເມຣິກາ}

ການສືບສວນສາທາລະນະທີ່ກວ້າງຂວາງທີ່ສຸດກ່ຽວກັບຖານທັບໃຕ້ດິນທີ່ຂ້ອຍໄດ້ພົບເຫັນແມ່ນມາຈາກ Richard Sauder, ນັກຄົ້ນຄວ້າເອກະລາດຊາວອາເມຣິກັນ ຜູ້ທີ່ໄດ້ພິມເຜີຍແຜ່ປຶ້ມຫຼາຍຫົວກ່ຽວກັບຖານທັບໃຕ້ດິນເລິກ \cite{22}. 
ຜົນງານຂອງ Sauder ປະກອບດ້ວຍການເກັບຮັກສາເອກະສານ ແລະ ແຜນການຂອງລັດຖະບານ, ການພິຈາລະນາເລື່ອງລາວຂ່າວ ແລະ ເທັກໂນໂລຢີທາງປະຫວັດສາດ ແລະ ປັດຈຸບັນ, ການພັດທະນາແຫຼ່ງຂໍ້ມູນ, ແລະ ການຮວບຮວມຂໍ້ມູນຈາກຄຳກ່າວອ້າງຂອງຄົນວົງໃນ.
ການຄົ້ນຄວ້າຂອງ Sauder ເປີດເຜີຍວ່າ ມີເຄືອຂ່າຍຂະໜາດໃຫຍ່ຂອງຖານທັບໃຕ້ດິນເລິກ ແລະ ຖານທັບໃຕ້ືທະເລ ຢູ່ໃນ ແລະ ອ້ອມຮອບອາເມຣິກາ ແລະ ດິນແດນຂອງມັນ (Figure \ref{fig:4}), ເຊິ່ງອາດຈະມີຄວາມເລິກຢ່າງໜ້ອຍ 3 ໄມລ໌, ແລະ ອາດຈະເຊື່ອມຕໍ່ກັນດ້ວຍລົດໄຟແມ່ເຫຼັກຄວາມໄວສູງແບບສູນຍາກາດໃຕ້ດິນ. 
ຖານທັບເຫຼົ່ານີ້ໄດ້ຮັບການສະໜອງທຶນແບບລັບໆ \textit{"ຜ່ານເກມຫຼອກລວງການຟອກເງິນທີ່ມີລັກສະນະການເງິນຂະໜາດໃຫຍ່, ລະຫວ່າງປະເທດ ແລະ ຫຼາຍອົງການ."} ດໍາເນີນງານໂດຍຄົນກຸ່ມດຽວກັນທີ່ເປັນເຈົ້າຂອງບໍລິສັດສະຫະລັດອາເມຣິກາ \cite{22}. ຜົນງານຕິດຕາມກວດກາຂອບເຂດຂອງຖານທັບເຫຼົ່ານີ້ໂດຍ Catherine Austin Fitts  (ເຊິ່ງຜົນງານຂອງລາວຈະຖືກກ່າວເຖິງໃນພາກຕໍ່ໄປ) ແລະ ໜຶ່ງໃນຜູ້ຮ່ວມມືຂອງລາວໄດ້ຄາດຄະເນວ່າ ມີຖານທັບໃຕ້ດິນ ແລະ ໃຕ້ທະເລຂອງອາເມຣິກາປະມານ 170 ແຫ່ງ \cite{16,20}. 
\begin{figure}[b]
\begin{center}
% \fbox{\rule{0pt}{2in} \rule{0.9\linewidth}{0pt}}
   \includegraphics[width=1\linewidth]{penta.jpg}
\end{center}
   \caption{ແທ້ຈິງແລ້ວມີຫຍັງຢູ່ໃຕ້ທຳນຽບຂາວ ແລະ ອາຄານ Pentagon? ເຫັນໄດ້ຢ່າງຊັດເຈນວ່າ, ມີເຄືອຂ່າຍອຸໂມງໃຕ້ດິນເລິກ  (Picture: \cite{31}).}
\label{fig:3}
\label{fig:onecol}
\end{figure}


\begin{figure*}[t]
\begin{center}
% \fbox{\rule{0pt}{2in} \rule{.9\linewidth}{0pt}}
\includegraphics[width=0.9\textwidth]{basescrop.png}
\end{center}
   \caption{ແຜນທີ່ທີ່ສະແດງສະຖານທີ່ແນ່ນອນຂອງຜົນການວິໄຈຂອງ Sauder ເປີດເຜີຍ ວ່າມີຖານທັບໃຕ້ດິນ ແລະ ໃຕ້ທະເລຢ່າງແນ່ນອນ, ພ້ອມທັງອຸໂມງໃຕ້ນ້ຳສຳລັບເຮືອດຳນ້ຳທີ່ເຊື່ອມຕໍ່ໄປສູ່ແຜ່ນດິນ.
Sauder ແມ່ນ \textit{"ແນ່ໃຈວ່າ ມີສະຖານທີ່ \textbf{ຫຼາຍກວ່ານີ້ອີກ} (facilities) ຫຼາຍບ່ອນ [ນອກເໜືອຈາກເຫຼົ່ານີ້]"} \cite{22}.}
   \label{fig:4}
\end{figure*}

ນີ້ແມ່ນບາງສ່ວນຂອງຄຳໃຫ້ການຈາກແຫຼ່ງຂໍ້ມູນຂອງ Sauder ທີ່ໄດ້ອະທິບາຍລາຍລະອຽດກ່ຽວກັບຂອບເຂດຂອງຖານທັບເຫຼົ່ານີ້:

\begin{flushleft}
\begin{enumerate}
    \item Camp David, Maryland: \textit{"ແຫຼ່ງຂໍ້ມູນຂອງຂ້ອຍໄດ້ແຈ້ງໃຫ້ຮູ້ວ່າ ພາກສ່ວນໃຕ້ດິນຂອງ Camp David ນັ້ນກວ້າງຂວາງ ແລະ ສະລັບຊັບຊ້ອນຫຼາຍ, ແລະ ມີອຸໂມງລັບຫຼາຍໄມລ໌, ຈົນວ່າເປັນທີ່ສົງໄສວ່າຈະມີບຸກຄົນໃດຜູ້ໜຶ່ງຈະມີແຜນທີ່ຂອງສະຖານທີ່ນັ້ນຢູ່ໃນຄວາມຈຳຂອງຕົນເອງໄດ້."} \cite{22}. 
    \item The White House (ທຳນຽບຂາວ), Washington DC: \textit{"ໝູ່ສະໜິດຄົນໜຶ່ງຂອງຂ້າພະເຈົ້າຖືກພາລົງໄປໃນສະຖານທີ່ແຫ່ງນີ້ໃນຊ່ວງການບໍລິຫານຂອງປະທານາທິບໍດີ Lyndon B. Johnson ໃນຊ່ວງປີ 1960. ລາວໄດ້ເຂົ້າໄປໃນລິຟຢູ່ທຳນຽບຂາວ ແລະ ຖືກນຳພາລົງໄປຊັ້ນລຸ່ມໂດຍກົງ. ລາວເຊື່ອວ່າລິຟໄດ້ລົງໄປ 17 ຊັ້ນ. ເມື່ອປະຕູເປີດອອກຢູ່ໃຕ້ດິນ, ລາວຖືກນຳພາໄປຕາມລະບຽງທາງເດິນທີ່ເບິ່ງຄືວ່າຫາຍລັບໄປສຸດສາຍຕາ. ປະຕູ ແລະ ລະບຽງທາງເດີນອື່ນໆກໍໄດ້ເປີດອອກຈາກລະບຽງທາງເດີນນັ້ນ"} \cite{22}. 
ດັ່ງທີ່ສະແດງຢູ່ໃນຮູບ \ref{fig:3}.
    \item Fort Meade, Maryland - ຈາກແຫຼ່ງຂໍ້ມູນຜູ້ທີ່ບັງເອີນເຂົ້າໄປໃນ "ຫ້ອງໃຕ້ດິນ" ໃນຊ່ວງປີ 1970: \textit{"ຂ້າພະເຈົ້າໄດ້ເປີດປະຕູອອກ ແລະ ມັນເປັນຂັ້ນໄດທີ່ນໍາລົງໄປລຸ່ມ. ຂ້າພະເຈົ້າເດີນໄປທີ່ຂອບແລ້ວເບິ່ງລົງໄປທີ່ຊ່ອງວ່າງລະຫວ່າງຂັນໄດ ຂ້າພະເຈົ້າບໍ່ໄດ້ນັບຈຳນວນຊັ້ນທີ່ລົງໄປ, ແຕ່ຂ້າພະເຈົ້າຮູ້ສຶກວ່າມັນປະມານ 15-20 ຊັ້ນ... ຂ້າພະເຈົ້າເດີນລົງໄປຊັ້ນໜຶ່ງ ແລ້ວກໍມີປະຕູ... ຂ້າພະເຈົ້າເປີດປະຕູອອກ ແລະ ຢຽດຫົວເຂົ້າໄປເບິ່ງຊ້າຍຂວາ ກໍເຫັນອຸໂມງທີ່ທອດຍາວຈົນສຸດສາຍຕາທັງສອງທິດທາງ. ມັນແນ່ນອນວ່າເລິກກວ່າພື້ນທີ່ຕຶກອາຄານ ແລະ ລານຈອດລົດຢູ່ຊັ້ນພື້ນດິນຫຼາຍ. ມີປະຕູຕາມຝາຜະໜັງກົງກັນຂ້າມ ໂດຍມີໄລຍະຫ່າງກັນປະມານ 30-40 ຟຸດ... ຂ້າພະເຈົ້າຕັດສິນໃຈກວດເບິ່ງອີກສອງສາມຊັ້ນ, ສະນັ້ນຂ້າພະເຈົ້າຈຶ່ງເດີນລົງໄປອີກຊັ້ນໜຶ່ງ... ແລະກໍເຫັນການຈັດວາງແບບດຽວກັນ... ຂ້ອຍເດີນລົງໄປອີກຊັ້ນໜຶ່ງ ແລ້ວແນມເບິ່ງກໍເຫັນແບບດຽວກັນກັບສອງຊັ້ນທຳອິດ"} \cite{22}.
\end{enumerate}
\end{flushleft}

\begin{figure}[t]
\begin{center}
% \fbox{\rule{0pt}{2in} \rule{0.9\linewidth}{0pt}}
   \includegraphics[width=1\linewidth]{undersea.jpg}
\end{center}
   \caption{ຮູບປະກອບຂອງຖານທັບໃຕ້ທະເລ, ໂດຍ Walter Koerschner.
ລາວເຄີຍເປັນນັກແຕ້ມຮູບປະກອບໃຫ້ກັບທີມຖານທັບໃຕ້ທະເລ Rock-Site ຂອງກອງທັບເຮືອສະຫະລັດ ທີ່ສູນອາວຸດ China Lake, ລັດຄາລິຟໍເນຍ ຂອງກອງທັບເຮືອ ໃນຊ່ວງປີ 1960. ໜຶ່ງໃນແຫຼ່ງຂໍ້ມູນຂອງ Sauder ເປີດເຜີຍວ່າ ມີຖານທັບໃຕ້ດິນເລິກໜຶ່ງໄມລ໌ຢູ່ China Lake \cite{22,23}.}
\label{fig:5}
\label{fig:onecol}
\end{figure}

Sauder ຍັງໄດ້ຮັບຄໍາໃຫ້ການກ່ຽວກັບລົດໄຟແມ່ເຫຼັກລອຍທີ່ສາມາດບັນລຸຄວາມໄວເຖິງ 2,000 ໄມລ໌ຕໍ່ຊົ່ວໂມງ, ຖານທັບທີ່ສ້າງຂຶ້ນໃຕ້ພື້ນມະຫາສະໝຸດ (ຮູບພາບ \ref{fig:5}), ແລະ ອຸໂມງໃຕ້ທະເລສຳລັບເຮືອດຳນ້ຳທີ່ເຊື່ອມຕໍ່ໄປຫາພື້ນທີ່ໃນແຜ່ນດິນ.
ກ່ຽວກັບຄຳໃຫ້ການອັນໜຶ່ງທີ່ກ່າວເຖິງຖານທັບໃຕ້ທະເລໃນອ່າວເມັກຊິໂກ, Sauder ກ່າວວ່າ, \textit{"ປະມານເຄິ່ງປີຫຼັງຈາກການຕີພິມປຶ້ມ Underwater and Underground Bases, ຂ້າພະເຈົ້າໄດ້ຖືກຕິດຕໍ່ໂດຍຊາຍຄົນໜຶ່ງທີ່ບອກວ່າ ລາວມີຄວາມຮູ້ກ່ຽວກັບໂຄງການໃຕ້ທະເລທີ່ຜິດປົກກະຕິ... ລາວໄດ້ລະບຸວ່າໂຄງການດັ່ງກ່າວແມ່ນຢູ່ໃຕ້ພື້ນທະເລຂອງອ່າວເມັກຊິໂກ, ແລະ Parson's ແມ່ນຜູ້ຮັບເໝົາ. ລາວໄດ້ເວົ້າຕໍ່ໄປວ່າ Parsons ໄດ້ຊື້ເຄື່ອງມືພິເສດບາງຢ່າງທີ່ຕັ້ງໃຈໄວ້ສໍາລັບການປະຕິບັດງານ 2,800 ຟຸດໃຕ້ພື້ນທະເລ... ອຸປະກອນດັ່ງກ່າວມີລັກສະນະພິເສດພໍທີ່ຈະສະແດງໃຫ້ເຫັນຢ່າງຊັດເຈນວ່າ ຕ້ອງມີການປະກົດຕົວຂອງມະນຸດທີ່ມີຊີວິດຢູ່ໃນສະຖານທີ່ທີ່ມີການຕິດຕັ້ງ"} \cite{22}.

\begin{figure}[t]
\begin{center}
% \fbox{\rule{0pt}{2in} \rule{0.9\linewidth}{0pt}}
   \includegraphics[width=1\linewidth]{sub.jpg}
\end{center}
   \caption{ຮູບປະກອບຂອງອຸໂມງໃຕ້ນ້ຳສຳລັບເຮືອດຳນ້ຳ, ໂດຍ Walter Koerschner \cite{22,23}.}
\label{fig:6}
\label{fig:onecol}
\end{figure}

\begin{figure}[t]
\begin{center}
% \fbox{\rule{0pt}{2in} \rule{0.9\linewidth}{0pt}}
   \includegraphics[width=1\linewidth]{iran.jpeg}
\end{center}
   \caption{ວິດີໂອຈາກທາງການອີຣ່ານທີ່ສະແດງໃຫ້ເຫັນ "ເມືອງຂີປະນາວຸດ" ໃຕ້ດິນຂອງພວກເຂົາ \cite{39,40}.}
\label{fig:12}
\label{fig:onecol}
\end{figure}

ຖ້າຫາກວ່າມີເຄືອຂ່າຍລັບທີ່ກວ້າງໃຫຍ່ໄພສານລະຫວ່າງທະວີບຂອງຖານທັບໃຕ້ດິນ ແລະ ໃຕ້ທະເລຫຼາຍກວ່າ 170 ແຫ່ງ ທີ່ຖືກຂຸດລົງໄປເລິກຫຼາຍໄມລ໌ຢູ່ໃຕ້ພື້ນດິນທີ່ພວກເຮົາຢືນຢູ່, ເຊື່ອມຕໍ່ກັນດ້ວຍລົດໄຟແມ່ເຫຼັກຄວາມໄວສູງແບບສູນຍາກາດ, ໂດຍໄດ້ຮັບການສະໜອງທຶນຈາກຜົນຂອງແຮງງານຂອງພວກເຮົາ, ມວນມະນຸດໃນທຸກວັນນີ້ຈະຢູ່ໃນສະພາວະທີ່ບໍ່ຮູ້ເລື່ອງລາວຢ່າງທີ່ສູດ ແລະ ມີຄວາມສຸກທີ່ສຸດ, ບໍ່ພຽງແຕ່ບໍ່ຮູ້ວ່າມີຫຍັງຢູ່ໃຕ້ພວກເຂົາ, ແຕ່ຍັງບໍ່ຮູ້ວ່າຈະມີຫຍັງເກີດຂຶ້ນກັບພວກເຂົາໃນອະນາຄົດອັນໃກ້ນີ້, ໃນຂະນະທີ່ພວກເຂົາກໍ່ເຊື່ອຟັງຖ້ອຍຄຳທີ່ວ່າງເປົ່າ ແລະ ຖືກຈັດຕັ້ງໂດຍພວກນັກການເມືອງທີ່ຄຸມພວກເຂົາ.
ຂໍ້ສັງເກດເພີ່ມເຕີມ - ການມີຢູ່ຂອງເຄືອຂ່າຍອຸໂມງໃຕ້ດິນຂະໜາດໃຫຍ່ໄດ້ຖືກເປີດເຜີຍຢ່າງບໍ່ຕ້ອງສົງໄສໃນຄວາມຂັດແຍ້ງທີ່ກຳລັງດຳເນີນຢູ່ໃນຕາເວັນອອກກາງ (ອຸໂມງ Hamas ໃຕ້ເຂດ Gaza Strip \cite{38}, ແລະ "ເມືອງຂີປະນາວຸດ" ໃຕ້ດິນຂອງອີຣ່ານ
 (figure \ref{fig:12}) \cite{39,40}).  
ສິ່ງເຫຼົ່ານີ້ບໍ່ຄວນຈະເຮັດໃຫ້ມີຂໍ້ສົງໄສທັງໃນຄວາມເປັນໄປໄດ້ໃນການກໍ່ສ້າງ ແລະ ການມີຢູ່ແທ້ຈິງຂອງໂຄງສ້າງດັ່ງກ່າວ. 
ພວກເຂົາຍັງເຮັດໃຫ້ພວກເຮົາສົງໄສອີກວ່າ ບັນດາປະເທດອື່ນໆ ທີ່ມີທຶນສູງກວ່າຢ່າງຫຼວງຫຼາຍ ອາດຈະໄດ້ສ້າງໂຄງສ້າງຫຍັງແດ່ໃນຊ່ວງເວລາຽດຽວກັນນັ້ນ.
\subsection{ຫຼັກຖານເພີ່ມເຕີມກ່ຽວກັບການກະກຽມທີ່ຫຼົບໄພໃຕ້ດິນ ແລະ ໄພພິບັດ}

\begin{figure}[t]
\begin{center}
% \fbox{\rule{0pt}{2in} \rule{0.9\linewidth}{0pt}}
   \includegraphics[width=1\linewidth]{tyrol.jpg}
\end{center}
   \caption{ທີ່ຫຼົບໄພໃຕ້ດິນໃນ South Tyrol, Switzerland.
ສະວິດເຊີແລນ, ເຊິ່ງກວມເອົາພູຜາປ່າດົງ Alps ຂອງເອີຣົບ, ມີຊື່ສຽງໃນການປິດບັງ ທີ ຫຼົບໄພໃຕ້ດິນ ທີ່ຢູ່ໃນພູຜາຂອງຕົນຢ່າງສະຫຼາດ \cite{32}.}
\label{fig:7}
\label{fig:onecol}
\end{figure}

\begin{figure}[t]
\begin{center}
% \fbox{\rule{0pt}{2in} \rule{0.9\linewidth}{0pt}}
   \includegraphics[width=1\linewidth]{svalbard.jpg}
\end{center}
   \caption{ຄັງເມັດພັນໂລກ Svalbard (ສະຟາລບາດ) ທີ່ປະເທດນໍເວ, ເຊິ່ງບັນຈຸເມັດພັນຫຼາຍກວ່າໜຶ່ງລ້ານເມັດ \cite{24}.
ເຮົາຕ້ອງສົງໄສວ່າ ໄພພິບັດແບບໃດທີ່ຈະເຮັດໃຫ້ຕ້ອງນຳໃຊ້ມັນ.}
\label{fig:8}
\label{fig:onecol}
\end{figure}

ມີຂໍ້ຄິດເພີ່ມເຕີມຫຼາຍຢ່າງກ່ຽວກັບການກະກຽມຮັບມືໄພພິບັດທົ່ວໂລກ ນອກເໜືອໄປຈາກຖານທັບໃຕ້ດິນຂອງຊາວອາເມຣິກັນ.
ນໍເວ, ສະວິດເຊີແລນ, ສະວີເດັນ ແລະ ຟິນແລນ ແມ່ນຕົວຢ່າງທີ່ດີ:

\begin{flushleft}
\begin{enumerate}
    \item ໂຄງການ Camelot ໄດ້ເຜີຍແຜ່ຄຳໃຫ້ການທີ່ກ່ຽວຂ້ອງຈາກນັກການເມືອງຊາວນໍເວຄົນໜຶ່ງ \cite{25,26}, ເຊິ່ງພວກເຂົາໄດ້ຢັ້ງຢືນຕົວຕົນແລ້ວ ແຕ່ເກັບໄວ້ເປັນສ່ວນຕົວ. 
ລາວກ່າວຫາວ່ານໍເວມີຖານທັບໃຕ້ດິນຂະໜາດໃຫຍ່ 18 ແຫ່ງ, ແລະ ນໍເວ (ພ້ອມກັບອິດສະຣາແອລ ແລະ "ອີກຫຼາຍປະເທດ") ກໍາລັງສ້າງຖານທັບເຫຼົ່ານີ້ເພື່ອກະກຽມຮັບມືກັບໄພພິບັດທໍາມະຊາດບາງຢ່າງ. 
Richard Sauder ຍັງໄດ້ຮັບຄຳໃຫ້ການຈາກຊາຍຄົນໜຶ່ງ ຜູ້ທີ່ເຄີຍເຂົ້າໄປໃນຖານທັບໃຕ້ດິນຂະໜາດໃຫຍ່ ເຊິ່ງຖືກສ້າງຂຶ້ນພາຍໃນພູທີ່ຖືກເຈາະໃຫ້ກວ້າງອອກໃນປະເທດນໍເວ \cite{22}. \
    \item ສະວິດເຊີແລນເປັນທີ່ຮູ້ຈັກກັນດີວ່າມີທີ່ຫຼົບໄພໃຕ້ດິນນິວເຄລຍຈໍານວນຫຼາຍທີ່ສ້າງຂຶ້ນໃນເຂດພູສູງຂອງ Alps (Figure \ref{fig:7}).
ຈຳນວນເຫຼົ່ານີ້ມີຫຼາຍກວ່າ 370,000 ແຫ່ງຢ່າງໜ້າອັດສະຈັນ – ພຽງພໍທີ່ຈະໃຫ້ທີ່ພັກພິງແກ່ພົນລະເມືອງທຸກໆຄົນ \cite{27}. 
    \item ສະວີເດັນ ແລະ ຟິນແລນ ມີທີ່ຫຼົບໄພໃຕ້ດິນພຽງພໍທີ່ຈະໃຫ້ທີ່ພັກພິງແກ່ຊາວເມືອງໃນທຸກໆເມືອງໃຫຍ່ \cite{27}. 
\end{enumerate}
\end{flushleft}

ເຫັນໄດ້ຊັດເຈນວ່າ ມະຫາເສດຖີທຸລະກິດຈາກ Silicon Valley ກໍຮູ້ເລື່ອງນີ້ເຊັ່ນກັນ. 
ມີລາຍງານວ່າ, \textit{"Reid Hoffman, ຜູ້ຮ່ວມກໍ່ຕັ້ງ LinkedIn ແລະ ນັກລົງທຶນທີ່ມີຊື່ສຽງ, ໄດ້ບອກກັບ The New Yorker ເມື່ອຕົ້ນປີນີ້ວ່າ ລາວຄາດຄະເນວ່າ ຫຼາຍກວ່າ 50\% ຂອງມະຫາເສດຖີ Silicon Valley ໄດ້ຊື້ "ປະກັນໄພວັນສິ້ນໂລກ" ໃນລະດັບໃດໜຶ່ງ ເຊັ່ນ: ທີ່ຫຼົບໄພໃຕ້ດິນ... ອີງຕາມ Jim Dobson, ນັກຂຽນຂອງ Forbes, ມະຫາເສດຖີຈໍານວນຫຼາຍມີຍົນບິນສ່ວນຕົວ "ພ້ອມທີ່ຈະອອກເດີນທາງໄດ້ທັນທີທັນໃດ." ພວກເຂົາເຈົ້າຍັງເປັນເຈົ້າຂອງລົດຈັກ, ອາວຸດ ແລະ ເຄື່ອງປັ່ນໄຟ"} \cite{28}. 
ນອກຈາກນີ້ ຍັງມີໂຄງການເກັບຮັກສາຂໍ້ມູນຂະໜາດໃຫຍ່ຕ່າງໆ ເຊັ່ນ: Global Knowledge Vault, ທີ່ດໍາເນີນການໂດຍມູນນິທິ Arch Mission Foundation, \cite{29} ແລະ Svalbard Global Seed Vault \cite{30} ເຊິ່ງເບິ່ງຄືວ່າກໍາລັງກະກຽມທີ່ຈະຮັກສາຊັບສິນອັນສໍາຄັນຂອງມະນຸດຊາດ ໃນກໍລະນີທີ່ເກີດໄພພິບັດໃນລະດັບສູນພັນ. 
\begin{figure*}[t]
\begin{center}
% \fbox{\rule{0pt}{2in} \rule{.9\linewidth}{0pt}}
\includegraphics[width=0.9\textwidth]{govcrop2.png}
\end{center}
   \caption{ລາຍຮັບ, ລາຍຈ່າຍ, ແລະ ລາຍຈ່າຍສໍາລັບຖານລັບໃຕ້ດິນຂອງລັດຖະບານສະຫະລັດ ຕັ້ງແຕ່ປີ 1998 ຫາ 2023 \cite{19}.}
   \label{fig:9}
\end{figure*}

\section{ກົນໄກການສະໜອງທຶນແບບປະຊາທິປະໄຕ ສຳລັບຖານທັບໃຕ້ດິນຂະໜາດໃຫຍ່.}

ດັ່ງນັ້ນ, ເຄືອຂ່າຍຖານທັບໃຕ້ດິນ ແລະ ໃຕ້ທະເລຂະໜາດໃຫຍ່ຂ້າມທະວີບຫຼາຍກວ່າ 170 ແຫ່ງນັ້ນ ໄດ້ຮັບການສະໜອງທຶນແນວໃດ ໃນຂະນະທີ່ປິດບັງບໍ່ໃຫ້ປະຊາຊົນຜູ້ເປັນໜີ້ສິນຮູ້ເລີຍ? 
ມີຮ່ອງຮອຍເອກະສານອັນໜຶ່ງທີ່ສາມາດເຮັດໃຫ້ພວກເຮົາພໍຮູ້ໄດ້ເຖິງຂະໜາດຂອງເງິນທີ່ໃຊ້ເຂົ້າໃນໂຄງການເຫຼົ່ານີ້ ແລະ ທີ່ມາຂອງມັນ. 
ໃນປີ 2017, Catherine Austin Fitts, ນັກລົງທຶນດ້ານການທະນາຄານຊາວອາເມຣິກັນ ແລະ ອະດີດເຈົ້າໜ້າທີ່ລັດຖະການໃນໄລຍະການບໍລິຫານຂອງປະທານາທິບໍດີ Bush, ແລະ Mark Skidmore, ນັກເສດຖະສາດຈາກມະຫາວິທະຍາໄລ Michigan State, ໄດ້ພົບເຫັນການໃຊ້ຈ່າຍທີ່ບໍ່ໄດ້ຮັບອະນຸຍາດເຖິງ 21 ລ້ານລ້ານໂດລາສະຫະລັດ (USD) ໃນລັດຖະບານສະຫະລັດອາເມຣິກາ ໃນຊ່ວງປີປະມານ 1998-2015 \cite{11,12,13}.
ຕາມລາຍງານຂອງພວກເຂົາ, \textit{"ໃນວັນທີ 7 ຕຸລາ 2016, ສຳນັກຂ່າວ Reuters ໄດ້ເຜີຍແຜ່ບົດຄວາມຂອງ Scot Paltrow (2016), ເຊິ່ງລາຍງານວ່າໃນປີປະມານ 2015, ກອງທັບໄດ້ປັບປຸງບັນຊີທີ່ບໍ່ສາມາດກວດສອບໄດ້ເຖິງ 6.5 ລ້ານລ້ານໂດລາສະຫະລັດ "ເພື່ອສ້າງພາບລວງຕາວ່າບັນຊີເງີນຂອງພວກເຂົາມີສົມດຸນ.” 
ພິຈາລະນາວ່າງົບປະມານກອງທຶນທົ່ວໄປຂອງກອງທັບໃນປີດຽວກັນນັ້ນແມ່ນ 122 ຕື້ໂດລາສະຫະລັດ, ນີ້ຖືເປັນການເປີດເຜີຍທີ່ໜ້າຕົກໃຈຢ່າງຍິ່ງ... ກະຊວງປ້ອງກັນປະເທດ (DOD) ໄດ້ກາຍເປັນຂ່າວໃຫຍ່ໃນສື່ຕ່າງໆຫຼາຍປີກ່ອນໜ້ານັ້ນ ສຳລັບບັນຫາການບັນຊີຂອງຕົນ ໃນວັນທີ 10 ກັນຍາ 2001, ເມື່ອລັດຖະມົນຕີກະຊວງປ້ອງກັນປະເທດ Donald Rumsfeld ໄດ້ຖະແຫຼງໃນກອງປະຊຸມສະພາ (C-SPAN, 2014) ວ່າ ກະຊວງປ້ອງກັນປະເທດ (DOD) ໄດ້ສູນເສຍບັນທຶກທຸລະກຳມູນຄ່າ 2.3 ລ້ານລ້ານໂດລາສະຫະລັດ... ການຍອມຮັບນີ້ໄດ້ກາຍເປັນຂ່າວໃຫຍ່ໃນວັນນັ້ນ, ແຕ່ຖືກລືມໄປໃນມື້ຕໍ່ມາ ເມື່ອເຫດການໂສກນາດຕະກໍາ 9/11 ໄດ້ດຶງດູດຄວາມສົນໃຈຂອງຄົນທົ່ວໂລກ... ເມື່ອສາດສະດາຈານ Mark Skidmore ໄດ້ຮູ້ກ່ຽວກັບທຸລະກໍາທີ່ກອງທັບບໍ່ສາມາດກວດສອບໄດ້ເຖິງ 6.5 ລ້ານລ້ານໂດລາສະຫະລັດ, ລາວໄດ້ຕິດຕໍ່ຫາທ່ານນາງ Fitts ແລະ ພວກເຂົາໄດ້ຕົກລົງກັນໃນລະດູບໃບໄມ້ປົ່ງຂອງປີ 2017 ທີ່ຈະເຮັດວຽກຮ່ວມກັນເພື່ອກໍານົດລາຍງານຂອງລັດຖະບານອື່ນໆທີ່ຄ້າຍຄືກັນ ເຊິ່ງຊີ້ໃຫ້ເຫັນເຖິງທຸລະກໍາທີ່ບໍ່ສາມາດກວດສອບໄດ້ຂະໜາດໃຫຍ່ຜິດປົກກະຕິພາຍໃນ ກະຊວງການເຄຫາ ແລະ ພັດທະນາຕົວເມືອງ (HUD) ແລະ ກະຊວງປ້ອງກັນປະເທດ (DOD). ໃນໄລຍະຫົກເດືອນຕໍ່ມາ, Skidmore, Fitts ແລະ ທີມງານນັກສຶກສາຈົບໃໝ່ກຸ່ມນ້ອຍໆໄດ້ລວບລວມເອກະສານທາງການຂອງລັດຖະບານ ເຊິ່ງໄດ້ລະບຸທຸລະກຳທີ່ບໍ່ສາມາດຢັ້ງຢືນໄດ້ລວມທັງໝົດ 21 ລ້ານລ້ານໂດລາສະຫະລັດ ໃນຊ່ວງປີ 1998-2016"} \cite{12}. 
ຕະຫຼອດໄລຍະ 18 ປີດຽວກັນ ຕັ້ງແຕ່ປີ 1998 ຫາ 2015, ລາຍຮັບຂອງລັດຖະບານສະຫະລັດທີ່ຖືກເປີດເຜີຍຕໍ່ສາທາລະນະມີພຽງແຕ່ 40.8 ລ້ານລ້ານໂດລາສະຫະລັດ \cite{15}, ຊີ້ໃຫ້ເຫັນວ່າຫຼາຍກວ່າເຄິ່ງໜຶ່ງຂອງລາຍຮັບລັດຖະບານສະຫະລັດອາເມຣິກາ ຖືກໃຊ້ຈ່າຍຢ່າງລັບໆໄປກັບຖານທັບໃຕ້ດິນ, ນອກເໜືອຈາກງົບປະມານທີ່ເປີດເຜີຍຢ່າງເປັນທາງການຂອງລັດຖະບານ
ສິ່ງທີ່ໜ້າສັງເກດອີກຢ່າງໜຶ່ງຄື ການໃຊ້ຈ່າຍແບບລັບໆນີ້ໄດ້ເກີດຂຶ້ນນອກເໜືອຈາກການຂາດດຸນງົບປະມານທີ່ມີມາຢ່າງຍາວນານ, ແລະ ຄາດວ່າບໍ່ພຽງແຕ່ສືບຕໍ່ມາຈົນເຖິງທຸກມື້ນີ້ເທົ່ານັ້ນ ແຕ່ຍັງມີມາກ່ອນປີ 1998 ອີກດ້ວຍ, ເຊິ່ງໝາຍຄວາມວ່າຈຳນວນເງິນລວມທີ່ໃຊ້ຈ່າຍໄປກັບຖານທັບເຫຼົ່ານີ້ແມ່ນຫຼາຍກວ່າ 21 ລ້ານລ້ານໂດລາສະຫະລັດ. 
ການນຳໃຊ້ສັດສ່ວນຂອງການໃຊ້ຈ່າຍແບບລັບໆອັນດຽວກັນກັບຊ່ວງປີ 2016-2023 ຈະເຮັດໃຫ້ໄດ້ຈຳນວນເງິນລວມທັງໝົດ 36.6 ພັນຕື້ໂດລາສະຫະລັດ ທີ່ຖືກໃຊ້ຈ່າຍນັບຕັ້ງແຕ່ປີ 1998 ເປັນຕົ້ນມາ.

ໃນປີ 2021, Mark Skidmore ໄດ້ເຜີຍແຜ່ບົດສະຫຼຸບການປັບປຸງການວິໄຈນີ້ ກ່ຽວກັບການປະກາດຂອງ Bloomberg ທີ່ວ່າໃນຊ່ວງປີປະມານ 2017-19, Pentagon ໄດ້ບັນທຶກການປັບປຸງບັນຊີເປັນຈຳນວນເງິນ 94.7 ພັນຕື້ໂດລາສະຫະລັດ ຢ່າງໜ້າຕົກໃຈ \cite{17,18}.
ຖ້າພວກເຮົາຄໍານຶງເຖິງການປອມແປງເງິນໂດລາສະຫະລັດຜ່ານລະບົບທະນາຄານກາງທີ່ເກີດຂຶ້ນມາເປັນເວລາກວ່າໜຶ່ງສະຕະວັດ ນັບຕັ້ງແຕ່ການສ້າງຕັ້ງທະນາຄານກາງສະຫະລັດ (Federal Reserve) ໃນປີ 1913 ເປັນຕົ້ນມາ \cite{37}, ມັນຈຶ່ງກາຍເປັນທີ່ຈະແຈ້ງວ່າການບັນຊີເງິນໂດລາສາທາລະນະທັງໝົດແມ່ນການເວົ້າທີ່ຫຼອກລວງຢ່າງສິ້ນເຊີງ, ແລະ ສະກຸນເງິນຂອງສະຫະລັດ ແລະ ລັດຖະບານແມ່ນເປັນພຽງລະບົບການຈັດສັນຊັບພະຍາກອນ ທີ່ເຈົ້າຂອງຜູ້ຊົງອຳນາດສາມາດກອບໂກຍເອົາ (ຫຼືເວົ້າອີກຢ່າງໜຶ່ງ, ຖອກເທອອກ) ໄດ້ຫຼາຍເທົ່າທີ່ພວກເຂົາຕ້ອງການຢ່າງງຽບໆ. 
\section{ລູກຫຼານຂອງ Jove: ກຸ່ມອຳນາດລັບຜູ້ຄວບຄຸ່ມໂລກຕາເວັນຕົກ}

ສະນັ້ນ, ໃຜກັນແທ້ທີ່ເປັນຜູ້ກຳອຳນາດ? 
ພວກເຮົາບໍ່ສາມາດຮູ້ໄດ້ຢ່າງແນ່ນອນ, ເພາະວ່າບັນດາກະສັດແຫ່ງທຶນຂອງໂລກຕາເວັນຕົກ ຍັງຄົງປິດບັງຕົວເອງຢູ່ໃນເງົາມືດ. ເຖິງແມ່ນວ່າຈະມີທິດສະດີຫລາຍຢ່າງ, ຕັ້ງແຕ່ບຸກຄົນສາທາລະນະຈົນເຖິງສິ່ງມີຊີວິດນອກໂລກ, ຄໍາຕອບທີ່ດີທີ່ສຸດທີ່ຂ້າພະເຈົ້າມີຕໍ່ຄໍາຖາມນີ້ແມ່ນມາຈາກຜົນງານຕະຫຼອດຊີວິດຂອງບລັອກເກີທີ່ບໍ່ເປີດເຜີຍຊື່ ຜູ້ທີ່ໃຊ້ຊື່ແຝງວ່າ "Amallulla". ຜົນງານຂອງເຂົາແມ່ນການລວມປະສົມປະສານຢ່າງກວ້າງຂວາງຂອງນັກຂຽນຫຼາຍກວ່າ 20 ຄົນ ແລະເອກະສານ “ທີ່ບໍ່ສາມາດຫາສິ່ງໃດມາແທນໄດ້” ກວ່າ 50 ສະບັບ ເຊິ່ງກວມເອົາຫົວຂໍ້ປະຫວັດສາດບູຮານ ແລະ ສະໄໝໃໝ່, ສັນຍະລັກລຶກລັບ, ແລະ ການເມືອງຕາເວັນຕົກ. \cite{33,34}. ຂ້າພະເຈົ້າສາມາດອະທິບາຍຜົນງານຂອງລາວວ່າ "ເປັນຄໍາທໍານາຍ" ກ່ຽວກັບໄພພິບັດທາງທໍລະນີຟີຊິກທີ່ກຳລັງຈະເກີດຂຶ້ນ — ມັນແມ່ນ \textit{ຢ່າງເຫັນໄດ້ຊັດ} ຄົບຖ້ວນສົມບູນກວ່າຂອງຂ້າພະເຈົ້າຫຼາຍ. 
Amallulla ໄດ້ລະບຸສາມກຸ່ມການເມືອງຂອງຝ່າຍຕາເວັນຕົກ, ເຊິ່ງລາວເອີ້ນລວມກັນວ່າ "ລູກຫລານຂອງ Jove", ຜູ້ທີ່ມີຄວາມຮູ້ກ່ຽວກັບ "ວັນສິ້ນໂລກ" – ເຊິ່ງກໍຄືໄພພິບັດທີ່ເກີດຂຶ້ນຊ້ຳໆຂອງໂລກ. 
ລາວເຊື່ອວ່າສາມກຸ່ມນີ້ຮ່ວມກັນຄວບຄຸມປະເທດຊາດຝ່າຍຕາເວັນຕົກໃນປັດຈຸບັນ, ແຕ່ໄດ້ແບ່ງພວກເຂົາອອກເປັນສາມກຸ່ມທີ່ແຕກຕ່າງກັນໂດຍອີງໃສ່ຕົ້ນກຳເນີດ ແລະ ເອກະລັກທາງປະຫວັດສາດທີ່ແຕກຕ່າງກັນ, ຄວາມຂັດແຍ້ງທີ່ອາດຈະເຄີຍເກີດຂຶ້ນໃນອະດີດ, ແລະ ຄວາມແຕກຕ່າງທີ່ຮັບຮູ້ໄດ້ໃນລະບົບຄ່ານິຍົມ ແລະ ການກະທຳຂອງພວກເຂົາ. 
ສາມກຸ່ມດັ່ງກ່າວສາມາດຈັດປະເພດແບບຄາວໆໄດ້ດັ່ງນີ້:

\begin{flushleft}
\begin{enumerate}
    \item \textbf{ນັກທະນາຄານ}: ຊົນຊັ້ນສູງໂຣມັນບູຮານ, ຜູ້ທີ່ກາຍມາເປັນກຸ່ມອັດສະວິນວິຫານ (Knights Templar) ແລະ ພັດທະນາເປັນກຸ່ມຟຣີເມສັນພາກເໜືອໃນອາເມຣິກາ.
    \item \textbf{ນັກຄິດ}: ກຸ່ມ Rosicrucians ແລະ ຟຣີເມສັນພາກໃຕ້ຂອງອາເມຣິກາ. 
    \item \textbf{The Jesuits and Black Pope}: ກຸ່ມຂອງລູກຫລານຂອງ Jove ທີ່ຝັງຕົວຢູ່ພາຍໃນໂບດ Roman Catholic. 
\end{enumerate}
\end{flushleft}

ທຸກມື້ນີ້, ສາມກຸ່ມເຫຼົ່ານີ້ລວມກັນປະກອບເປັນກຸ່ມ European Illuminati, Freemasons, ແລະ CIA. 
ຕາມທີ່ Amallulla ໄດ້ພັນລະນາໄວ້, \textit{" ໃນຕອນນີ້, ໃນຍຸກສຸດທ້າຍ, ລູກຫລານຂອງ Jove ໄດ້ຖືກປິດບັງໄວ້ຢ່າງດີພາຍໃຕ້ລະບົບການອະນຸຍາດແບບ "need-to-know" ເຊິ່ງຍົກເວັ້ນທັງປະທານາທິບໍດີຄົນປັດຈຸບັນຂອງສະຫະລັດອາເມລິກາກໍຕາມ. ເວົ້າອີກຢ່າງໜຶ່ງ, ພວກເຂົາໄດ້ພັດທະນາສິລະປະໃນການປິດບັງຕົວເອງຈາກການກວດສອບຂອງສາທາລະນະຊົນຢ່າງສົມບູນແບບແລ້ວ. \textbf{ລູກຫລານຂອງ Jove ບໍ່ພຽງແຕ່ຄວບຄຸມກອງທັບ ແລະ ລັດຖະບານຂອງສະຫະລັດອາເມລິກາເທົ່ານັ້ນ, ແຕ່ຍັງຜ່ານອຳນາດຂອງ ເງິນຕາທີ່ບໍ່ມີສິ່ງຄໍ້າປະກັນ, ບໍລິສັດໃຫຍ່ໆ, ແລະ ຮູບແບບການປົກຄອງແບບສາທາລະນະລັດທີ່ພວກເຂົາໄດ້ສ້າງຂຶ້ນມາ (ໂດຍຮູ້ວ່ານັກການເມືອງຈະຖືກສໍ້ລາດບັງຫຼວງໄດ້ຢ່າງງ່າຍ ແລະ ດ້ວຍເຫດນັ້ນຈຶ່ງຖືກຄວບຄຸມ.), ພວກເຂົາຄວບຄຸມໂລກຕາເວັນຕົກທັງໝົດ}"} \cite{33,34}. 
\begin{figure}[t]
\begin{center}
% \fbox{\rule{0pt}{2in} \rule{0.9\linewidth}{0pt}}
   \includegraphics[width=1\linewidth]{illuminati.jpg}
\end{center}
   \caption{ລູກຫລານຂອງ Jove ແມ່ນໃຜກັນແນ່?
(Image: \cite{35})}
\label{fig:10}
\label{fig:onecol}
\end{figure}

\begin{figure}[t]
\begin{center}
% \fbox{\rule{0pt}{2in} \rule{0.9\linewidth}{0pt}}
   \includegraphics[width=1\linewidth]{pike.jpg}
\end{center}
   \caption{ຫີນແກຼນິດທີ່ມີຊື່ສຽງ, ທີ່ຖືກເນັ້ນເປັນສີແດງ, ພ້ອມກັບພູມສັນຖານຂອງພາກຕາເວັນຕົກຂອງສະຫະລັດອາເມລິກາ. \cite{36}. ສະຫະລັດອາເມຣິກາ ອາດຖືກສ້າງຕັ້ງຂຶ້ນເພື່ອຄວບຄຸມສະຖານທີ່ແຫ່ງນີ້ແທ້ບໍ່?}
\label{fig:11}
\label{fig:onecol}
\end{figure}

ຕາມ Amallullaໄດ້ກ່າວໄວ້, ຄົນເຫຼົ່ານີ້ດູໝິ່ນສາສະໜາ, ປັບປ່ຽນໜັງສືສັກສິດໃນສາສະໜາຫຼັກຕ່າງໆຂອງໂລກເພື່ອຜົນປະໂຫຍດຂອງພວກເຂົາ, ແລະ ໃຊ້ສັນຍາລັກຢ່າງຫຼວງຫຼາຍ.
ນອກຈາກນັ້ນ, ພວກເຂົາກໍໂຫດຮ້າຍເມື່ອເວົ້າເຖິງສັດຕູຂອງພວກເຂົາ: \textit{"\textbf{ຕະຫຼອດໄລຍະເວລາຫຼາຍກວ່າ 2,600 ປີ, ພວກເຂົາໄດ້ກຳຈັດຜູ້ອື່ນໆທີ່ມີຄວາມຮູ້ສະເພາະກ່ຽວກັບວັນສິ້ນໂລກຢ່າງເປັນລະບົບ. ແລະໂດຍສິ່ງນີ້, ຂ້າພະເຈົ້າບໍ່ໄດ້ໝາຍເຖິງພຽງແຕ່ຊາວ Druids, ນັກຄິດ Kabbalah ຊາວຢິວ, ຊາວອີຢິບບູຮານ, ຊາວອາຣັບ, ແລະ ຜູ້ນິຍົມລັດທິລັບຂອງອິນເດຍເທົ່ານັ້ນ, ແຕ່ຍັງລວມເຖິງພວກກະໂຫຼກຫົວທີ່ຍາວ (Elongated Skulls) ໃນອາເມລິກາໃຕ້ ແລະ ພວກນັກບວດ Maya ໃນອາເມລິກາກາງອີກດ້ວຍ. ແລະ ຫຼັກຖານທີ່ວ່າພວກເຂົາໄດ້ທຳລາຍລ້າງປະຊາກອນທີ່ເຄີຍຈະເລີນຮຸ່ງເຮືອງໃນອາເມລິກາເໜືອ ເພື່ອຮັກສາດິນແດນແຫ່ງຍຸກສຸດທ້າຍນັ້ນແມ່ນມີຢ່າງຫຼວງຫຼາຍ. ການຂ້າລ້າງເຜົ່າພັນ "ຊາວອິນເດຍ" ອາເມຣິກັນ ແມ່ນເປັນພຽງການປະຕິບັດກວາດລ້າງຂັ້ນສຸດທ້າຍເທົ່ານັ້ນ}"} \cite{33,34}.

Amallulla ຍັງເຊື່ອວ່າ ໂຄງການ "ສະຫະລັດອາເມຣິກາ" ທັງໝົດແມ່ນຖືກດໍາເນີນເພື່ອຈຸດປະສົງເພື່ອໃນການຄວບຄຸມ "ຫີນແກຼນິດທີ່ມີຊື່ສຽງ" ເຊິ່ງເປັນທິວເຂົາຫີນແກຼນິດໃນເຂດພູ Rocky Mountains ທີ່ໃຫ້ການປ້ອງກັນທີ່ດີເລີດຈາກໄພພິບັດທາງທໍລະນີຟີຊິກ (ຮູບພາບ \ref{fig:11}). 
ຕາມ Amallullaໄດ້ກ່າວໄວ້, \textit{"ກ່ອນ, ໃນລະຫວ່າງ, ແລະ ຫຼັງຈາກສິ່ງທີ່ພວກເຮົາຄິດວ່າເປັນສົງຄາມກາງເມືອງ, ກຸ່ມນັກທະນາຄານ ແລະ ນັກຄິດໄດ້ຕໍ່ສູ້ກັນບໍ່ແມ່ນເພື່ອຄວບຄຸມສະຫະລັດອາເມຣິກາເທົ່ານັ້ນ, ແຕ່ເພື່ອຍາດແຍ່ງເອົາຫີນແກຼນິດທີ່ມີຊື່ສຽງ, ເຊິ່ງເປັນໜຶ່ງໃນບັນດາແຫຼ່ງຫີນແກຣນິດທີ່ເປັນເອກະລັກທີ່ສຸດໃນໂລກ... ບໍ່ມີແຫຼ່ງຫີນແກຼນິດອື່ນໃດທີ່ຕັ້ງຢູ່ສູງປານນີ້ ແລະ ຢູ່ໄກຈາກຝັ່ງມະຫາສະໝຸດປານນີ້ ຢູ່ບ່ອນໃດໃນໂລກອີກແລ້ວ. ມັນເປັນສະຖານທີ່ທີ່ເໝາະສົມທີ່ສຸດສຳລັບການເອົາຊີວິດລອດຈາກການເຄື່ອນຍ້າຍຂອງເປືອກໂລກ."} \cite{33,34}. 
ຜົນງານວິໄຈຂອງ Amallulla ໄດ້ເປີດເຜີຍວ່າ ປັດຈຸບັນມີລະບົບອຸໂມງໃຕ້ດິນຂະໜາດໃຫຍ່ທີ່ຖືກສ້າງຂຶ້ນພາຍໃຕ້ ແລະ ອ້ອມຮອບພື້ນທີ່ນີ້ \cite{36}. 
\section{ບົດສະຫຼຸບ}

ໃນເອກະສານສະບັບນີ້, ຂ້າພະເຈົ້າໄດ້ລົງລາຍລະອຽດບັນດາຄຳໃຫ້ການຕ່າງໆ ທີ່ຊີ້ໃຫ້ເຫັນວ່າ ຊົນຊັ້ນສູງຝ່າຍຕາເວັນຕົກໄດ້ຮັກສາຄວາມຮູ້ກ່ຽວກັບໄພພິບັດຂອງໂລກທີ່ເກີດຂຶ້ນຊ້ຳໆ ໄວ້ຢ່າງລະມັດລະວັງເປັນເວລາຫຼາຍພັນປີ, ເຊື່ອວ່າໄພພິບັດອີກຄັ້ງໜຶ່ງກຳລັງຈະເກີດຂຶ້ນ, ໄດ້ສ້າງທີ່ຫຼົບໄພໃຕ້ດິນຂະໜາດໃຫຍ່ ເພື່ອກະກຽມຮັບມືກັບເຫດການດັ່ງກ່າວ, ແລະ ກຳລັງວາງແຜນທີ່ຈະໃຊ້ປະໂຫຍດຈາກເຫດການດັ່ງກ່າວທາງດ້ານການເມືອງ ແລະ ທາງດ້ານການທະຫານ ເພື່ອບັນລຸການຄອບຄອງໂລກ. 
ຂ້າພະເຈົ້າໄດ້ກ່າວເຖິງຂໍ້ຄິດກ່ຽວກັບວິທີການສະໜອງທຶນໃນອາເມຣິກາ, ພ້ອມທັງໄດ້ອ້າງອີງເຖິງທິດສະດີທີ່ບໍ່ໄກເກີນຄວາມຈິງທີ່ສຸດ ກ່ຽວກັບສາຍເລືອດທີ່ແນ່ນອນທີ່ກຳລັງຄວບຄຸມເລື່ອງນີ້. 
ສຳລັບຜູ້ທີ່ຕ້ອງການຮູ້ເພີ່ມເຕີມ, ຍັງມີຂໍ້ມູນເພີ່ມເຕີມອີກຫຼວງຫຼາຍທີ່ຂ້າພະເຈົ້າບໍ່ໄດ້ກ່າວເຖິງ ເຊິ່ງສາມາດຄົ້ນພົບໄດ້ໂດຍການສຶກສາຈາກເອກະສານອ້າງອີງ.
ຂໍ້ມູນທີ່ສາມາດວັດແທກໄດ້ຢ່າງຊັດເຈນທີ່ສຸດທີ່ຊີ້ໃຫ້ເຫັນເຖິງເຫດການທາງທໍລະນີຟີຊິກທີ່ກຳລັງຈະເກີດຂຶ້ນ ແມ່ນການປ່ຽນແປງຢ່າງວ່ອງໄວຂອງສະໜາມແມ່ເຫຼັກໂລກ.
ສິ່ງນີ້ສາມາດວັດແທກໄດ້ບໍ່ພຽງແຕ່ຈາກການເຄື່ອນທີ່ຢ່າງວ່ອງໄວຂອງຂົ້ວແມ່ເຫຼັກເໜືອເທົ່ານັ້ນ (ຮູບພາບ \ref{fig:13}) ແລະ ການເພິ່ມຂື້ນຂອງຄວາມຜິດປົກກະຕິຂອງສະໜາມແມ່ເຫຼັກໂລກຢູ່ມະຫາສະໝຸດອັດລັງຕິກໃຕ້, ແຕ່ຍັງສາມາດວັດແທກໄດ້ຈາກການອ່ອນແອລົງ ແລະ ການບິດເບືອນຂອງສະໜາມແມ່ເຫຼັກໂລກເລັ່ງຕົວຂຶ້ນໃນຊ່ວງ 400 ປີທີ່ຜ່ານມາ \cite{3}. 
ຂໍ້ມູນວິທະຍາສາດດັ່ງກ່າວໄດ້ຖືກປຶກສາຫາລືຢ່າງລະອຽດຢູ່ໃນເອກະສານ ECDO ສອງສະບັບທຳອິດຂອງຂ້າພະເຈົ້າ, ເຊິ່ງສາມາດເຂົ້າເຖິງໄດ້ໃນເວັບໄຊທ໌ຂອງຂ້າພະເຈົ້າ \cite{3}. 
\begin{figure}[t]
\begin{center}
% \fbox{\rule{0pt}{2in} \rule{0.9\linewidth}{0pt}}
   \includegraphics[width=1\linewidth]{npw.jpg}
\end{center}
   \caption{ຕໍາແໜ່ງຂອງຂົ້ວແມ່ເຫຼັກເໜືອຂອງໂລກ ແຕ່ປີ 1590 ຫາ 2025, ສະແດງໃຫ້ເຫັນໃນແຕ່ລະ 5 ປີ \cite{41}.
ການເຄື່ອນທີ່ຂອງມັນໄດ້ເລີ່ມເລັ່ງຕົວຢ່າງວ່ອງໄວໃນປີ 1975.}
\label{fig:13}
\label{fig:onecol}
\end{figure}

ສະຫຼຸບທ້າຍນີ້, ຂ້າພະເຈົ້າຂໍຝາກຄຳເວົ້າຂອງ Amallulla ຜູ້ພະຍາກອນຄົນນີ້ໄວ້, ເຊິ່ງອະທິບາຍວ່າ \textit{"\textbf{ທຸກສິ່ງທຸກຢ່າງແມ່ນສິ່ງດຽວກັນ}"}: \textit{"ໃນທີ່ນີ້, ຂ້າພະເຈົ້າຈຳເປັນຕ້ອງຜັກດັນຈິນຕະນາການຂອງເຈົ້າໃຫ້ໄປສູ່ຂີດຈຳກັດທີ່ສຸດ. ທ່ານຕ້ອງລືມໂລກທີ່ທ່ານອາໄສຢູ່ ແລະ ໄດ້ຮູ້ຈັກມານັບຕັ້ງແຕ່ເດັກນ້ອຍ. ຖິ້ມມັນໄວ້ເບື້ອງຫຼັງຂອງທ່ານ. ມັນເປັນຄວາມຈິງທີ່ຖືກສ້າງຂຶ້ນທັງໝົດ ບໍ່ຕ່າງຫຍັງກັບສິ່ງທີ່ສະແດງຢູ່ໃນຮູບເງົາ The Matrix ແລະ ມີຈຸດປະສົງເພື່ອໃຫ້ທ່ານນອນຫຼັບຈົນເຖິງວິນາທີສຸດທ້າຍ. ບາງຄັ້ງຂ້າພະເຈົ້າກໍຄິດວ່າຂ້າພະເຈົ້າກໍາລັງຂຽນບົດໜັງ, ແຕ່ສິ່ງທີ່ຂ້າພະເຈົ້າກໍາລັງແບ່ງປັນກັບທ່ານຢູ່ໃນເວັບໄຊທ໌ນີ້ແມ່ນຄວາມຈິງ. ມັນໃຊ້ເວລາຂ້າພະເຈົ້າຫຼາຍກວ່າເຄິ່ງທົດສະວັດ ເພື່ອທີ່ຈະຮັບຮູ້ວ່າ “ທຸກສິ່ງທຸກຢ່າງແມ່ນສິ່ງດຽວກັນ“, ເຊິ່ງຂ້າພະເຈົ້າໄດ້ນຳໃຊ້ຢ່າງວ່ອງໄວ ເປັນຄໍາຂວັນສໍາລັບ An Apocalyptic Synthesis. ນີ້ແມ່ນແນວຄິດທີ່ຍາກທີ່ຈະສື່ສານໃຫ້ເຂົ້າໃຈ. ສຳລັບຕອນນີ້, ຂໍໃຫ້ເຮົາມາຄິດເຖິງພາບໃນຮູບເງົາ The Matrix. ມັນເປັນການປຽບທຽບທີ່ດີ. ສິ່ງທີ່ຂ້າພະເຈົ້າຮູ້ສຶກວ່າສື່ສານຍາກ ຄືສິ່ງທີ່ຂ້າພະເຈົ້າກໍາລັງຈະເວົ້າຕໍ່ໄປນີ້ບໍ່ແມ່ນການເວົ້າເກີນຈິງ. 
ສຳລັບຕອນນີ້, ການປຽບທຽບກັບຮູບເງົາ The Matrix ແມ່ນໃກ້ຄຽງທີ່ສຸດເທົ່າທີ່ຂ້າພະເຈົ້າສາມາດເຮັດໄດ້ ເພື່ອເຮັດໃຫ້ທ່ານເຂົ້າໃຈຄວາມເປັນຈິງອັນແທ້ຈິງຂອງສິ່ງທີ່ຂ້າພະເຈົ້າກຳລັງຈະເວົ້າ. 
\textbf{ທຸກສິ່ງໃນຊີວິດຂອງທ່ານ, ລວມທັງປະຫວັດສາດທີ່ຖືກບັນທຶກໄວ້ທັງໝົດ, ວິທະຍາສາດກະແສຫຼັກ, ວິທະຍາສາດທີ່ໄດ້ຮັບການຍອມຮັບ ແລະ ວິຊາການ, ການເມືອງ, ສາສະໜາ, ທຸກສິ່ງທຸກຢ່າງໃນທາງໃດທາງໜຶ່ງ ແມ່ນກ່ຽວກັບການເຄື່ອນຍ້າຍຂອງເປືອກໂລກ ຫຼື ການປ່ຽນແປງແກນໂລກທີ່ຈະມາເຖິງ.} ທ່ານພຽງແຕ່ຍັງບໍ່ເຫັນມັນໃນຕອນນີ້. 
ທ່ານກໍບໍ່ສາມາດຕື່ນຂຶ້ນສູ່ຄວາມເປັນຈິງນີ້ໄດ້ ເໝືອນກັບການຕື່ນຈາກຝັນຮ້າຍ. ມັນຕ້ອງໃຊ້ເວລາ. 
ແຕ່ຂ້າພະເຈົ້າສັນຍາກັບທ່ານ, ປາຍທາງຂອງເສັ້ນທາງສາຍນີ້ຄືການຮັບຮູ້ວ່າທ່ານໄດ້ດໍາລົງຊີວິດຢູ່ໃນໂລກທີ່ຄ້າຍຄືກັບຄວາມເປັນຈິງທີ່ຈຳລອງດ້ວຍຄອມພິວເຕີໃນຮູບເງົາ The Matrix ຕະຫຼອດຊີວິດຂອງທ່ານ"} \cite{33,34}.

ຂໍໃຫ້ທຸກຄົນໂຊກດີ.
\section{ຄຳຂອບໃຈ}

ຂອບໃຈທຸກໆບຸກຄົນທີ່ເລືອກປະກອບສ່ວນຄວາມຮູ້ສູ່ສາທາລະນະ. ຫາກປາດສະຈາກພວກທ່ານແລ້ວ, ຜົນງານນີ້ກໍຈະບໍ່ສາມາດເປັນໄປໄດ້ ແລະ ມະນຸດຊາດກໍຈະຍັງຄົງຢູ່ໃນຄວາມມືດມົນຕໍ່ໄປ. ການເລືອກຂອງທ່ານຈະເບັ່ງບານຕະຫຼອດໄປ. ພວກຂ້າພະເຈົ້າເປັນໜີ້ບຸນຄຸນທ່ານທຸກສິ່ງທຸກຢ່າງ, ແລະ ຂ້າພະເຈົ້າຮູ້ສຶກຂອບໃຈຢ່າງບໍ່ມີທີ່ສິ້ນສຸດ.

\clearpage
\twocolumn

{\small
% \renewcommand{\refname}{References}
\bibliographystyle{ieee}
\bibliography{egbib}
}

\end{document}

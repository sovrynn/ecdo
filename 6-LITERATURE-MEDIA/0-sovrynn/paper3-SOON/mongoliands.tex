\documentclass[10pt,twocolumn,letterpaper]{article}

\usepackage{booktabs}
% \usepackage{caption}
% \captionsetup[table]{skip=8pt}   % Зөвхөн хүснэгтэд нөлөөлнө
\usepackage{stfloats}  % Үүнийг урьдчилсан хэсэгт нэмнэ үү
\usepackage{float}

\usepackage[english]{babel}

% load Lao via babelprovide, turn on "onchar=ids" for automatic shaping
\babelprovide[import,onchar=ids fonts]{mongolian}

% main (rm) font for Latin
\babelfont{rm}{Noto Serif}
% alternate (sans-serif) font for Latin
\babelfont{alt}{Lato}

% Lao text in Noto Serif Lao at 1.2× scale
\babelfont[mongolian]{rm}{Noto Serif}
\babelfont[mongolian]{sf}{Noto Serif}
% Lao text in Noto Serif Lao for the alt family too
\babelfont[mongolian]{alt}{Noto Serif}

\usepackage{cvpr}
\usepackage{times}
\usepackage{epsfig}
\usepackage{graphicx}
\usepackage{amsmath}
\usepackage{amssymb}
\usepackage[breaklinks=true,bookmarks=false]{hyperref}

\cvprfinalcopy % *** Эцсийн илгээлтдээ энэ мөрийг uncomment хийх

\makeatletter
\def\cvprsubsection{\@startsection {subsection}{2}{\z@}
    {8pt plus 2pt minus 2pt}{6pt}{\bfseries\normalsize}}
\makeatother

\def\cvprPaperID{****} % *** CVPR Paper ID-г энд оруулна уу
\def\httilde{\mbox{\tt\raisebox{-.5ex}{\symbol{126}}}}

% Илгээлтийн горимд хуудасны дугаарлагдсан, камер-бэлэн горимд дугаарлагдаагүй байна
%\ifcvprfinal\pagestyle{empty}\fi
\setcounter{page}{1}
\begin{document}

\title{ECDO-ийн 3-р илтгэл: Өнөөгийн Барууны засаглалуудын Геофизикийн гамшгийн өмнөх бэлтгэлийн нотолгоо}

\author{Junho\\
2025 оны 6-р сард хэвлэгдсэн\\
Вэбсайт (Эндээс өгүүллүүд татаж авах): \href{https://sovrynn.github.io}{sovrynn.github.io}\\
ECDO Судалгааны Репозитори: \href{https://github.com/sovrynn/ecdo}{github.com/sovrynn/ecdo}\\
{\tt\small junhobtc@proton.me}
}
\maketitle
%\thispagestyle{empty}

\begin{abstract}
2024 оны 5-р сард "The Ethical Skeptic" \cite{0} хэмээх нэрээ нууцласан онлайн зохиолч "Exothermic Core-Mantle Decoupling Dzhanibekov Oscillation" (ECDO) \cite{1} гэдэг шинэлэг онолыг дэлгэсэн. Энэ онол нь Дэлхий өмнө нь эргэлтийн инерцийн улмаас далайн ус тивийн гадаргуу дээр урсан гарч, дэлхий даяар томоохон үерүүд үүсгэсэн тэнхлэгийн огцом, сүйрэлтэй өөрчлөлтүүдээр дамжин өнгөрсөн гэж үздэг. Үүнээс гадна энэ нь ийм эргэлтийн өөрчлөлт дахин ойртож болзошгүй гэдгийг харуулсан геофизикийн үйл явц болон өгөгдлийг танилцуулж байна. Ийм сүйрлийн үер, төгсгөлийн өдрийн таамаглал шинэ зүйл биш ч ECDO онол нь шинжлэх ухааны, орчин үеийн, олон салбарын болон өгөгдөлд тулгуурласан арга барилаараа онцгой сонирхол татахуйц юм.

Энэхүү өгүүлэл миний энэ сэдвээрх гурав дахь бүтээл \cite{2,3} бөгөөд энэ онолын орчин үеийн улс төрийн асуудлуудад тусгагдсан болно:
\begin{flushleft}
\begin{enumerate}
\item Барууны эрх мэдэлтнүүд геофизикийн гамшиг дөхөж байгаа гэдэгт итгэдэг бөгөөд энэ үйл явдлыг ашиглан улс төр, цэргийн давуу тал авахаар төлөвлөж байгаа гэх хулгайлагчдын гэрчлэл.
\item Энэ үйл явдалд бэлтгэхээр барууны нууц газар доорх, далайн ёроолын баазуудыг барьсан байгаа нотолгоо.
\item Эдгээр баазуудыг санхүүжүүлэхийн тулд барууны валютын бүтцээс асар их хэмжээний мөнгө урсгаж байгаа нотолгоо.
\end{enumerate}
\end{flushleft}
Энэхүү илтгэлд Барууны засаглалын эрх мэдэлтнүүд гүйцэтгэж буй өргөн хүрээний бэлтгэл ажлыг тэд ойролцоо гэж үзэж буй геофизикийн гамшгийн төлөө баримталж байгааг баримтжуулсан байна.
\end{abstract}

\section{Фримасоны ба "Англо-Саксон Мисс"}
2010 оны 1-р сард, биднийгээс өөр сэтгүүл зүйн байгууллага болох "Project Camelot" Лондон хотын Ахмад Масонуудын 2005 оны 6-р сард болсон хуралд биечлэн оролцсон дотроос нэгэн хүнийг \cite{4,6} ярилцсан. Энэ хуралд хэлэлцсэн сэдвүүд нь \textbf{"газарзүйн үйл явдал"} буюу дэлхийн хэмжээний байгалийн гамшигийн дэвсгэрт төвлөрсөн цэрэг, улс төрийн төлөвлөгөөнүүд байв.

\begin{figure}[b]
\begin{center}
\includegraphics[width=1\linewidth]{freemason.jpg}
\end{center}
   \caption{Британийн Фримасоны өөрсдийн байгалийн байдалд дэлхийг эзлэх, цөмийн бөмбөг хаях нууц төлөвлөгөө гаргаж байгаа нь - Лондон хотын Эрлс Корт дээр, 1992 он \cite{5}.}
\label{fig:1}
\label{fig:onecol}
\end{figure}

\begin{figure*}[t]
\begin{center}
\includegraphics[width=1\textwidth]{british.jpg}
\end{center}
   \caption{1937 оны Британийн Эзэнт Гүрэн, Англо-Саксон хүчний гайхамшигт илрэл \cite{14}.}
   \label{fig:2}
\end{figure*}
Энэхүү дотоодын мэдэгчийн хэлснээр, уулзалтад оролцсон 25-30 хүн бүгд \textit{"...Англичууд байсан бөгөөд тэдний зарим нь Их Британид хүмүүс шууд таних маш алдартай хүмүүс байсан... тэнд язгууртны жагсаалт бага зэрэг байсан бөгөөд зарим нь нэлээд язгууртан гаралтай. Би тэр уулзалтад нэг ахмад улс төрчийг танилцуулсан. Нөгөө хоёр нь цагдаагийн томоохон албан тушаалтан, нэг нь цэргийнхэн байсан. Хоёулаа үндэсний хэмжээнд алдартай бөгөөд хоёулаа одоогийн засгийн газарт зөвлөх гол хүмүүс юм — яг одоо"} \cite{4}. Мэдэгч хэлэхдээ тэр уг уулзалтад \textit{"Ямар ч төлөвлөгөөнгүйгээр оролцсон! Би энэ нь ердийн улирлын уулзалт гэж бодсон... Би энэ уулзалтад очоод миний хүлээж байсан уулзалт биш байсан. Намайг урьсан гэж би бодож байна... миний эзэлж байсан албан тушаал болон тэднийх шиг би ч тэдний нэг гэж тэд итгэсэн учраас."} \cite{4}.

Уулзалтад хэлэлцсэн үндсэн үйл явдлын цаг хугацааны дараалал (2005 онд) дараах байдалтай байна:

\begin{flushleft}
\begin{enumerate}
    \item Иран эсвэл Хятадыг тактик цөмийн зэвсэг хэрэглэхэд хүргэж, хязгаарлагдмал цөмийн мөргөлдөөн үүсгэх, дараа нь гал зогсоохыг бий болгох.
    \item Хятад дээр биологийн зэвсэг хэрэглэх, "70-аад оноос хойш" гол бай онош болсон.
    \item Үүний улмаас үүссэн айдас, эмх замбараагүй байдлыг шалтаг болгон тоталитар цэргийн засаглалыг бий болгох.
\end{enumerate}
\end{flushleft}

Гэхдээ хамгийн чухал зүйл бол эдгээр үйл явдлуудын дараа юу болох вэ гэдэгт оршино: \textit{"Тэгэхээр бид энэ дайнд орж, дараа нь... Дэлхий дээр бүх хүнд нөлөөлөх геофизикийн үйл явдал болно"} \cite{4}. Дотоод мэдээлэгчийн хэлснээр энэ геофизикийн үйл явдалд \textit{\textbf{"Дэлхийн царцдас ойролцоогоор 30 градусаар, 1700-2000 миль уртрагаар өмнө зүг рүү шилжих бөгөөд энэ нь маш удаан үргэлжлэх асар том эмх замбараагүй байдлыг үүсгэх болно"}} \cite{4}.

Энэ бүх нууц төлөвлөгөөний шалтгаан нь мэдээж хүч юм. Дотоод мэдээлэгч тайлбарлаж байна, \textit{"Тэр үед бид бүгд цөмийн болон биологийн дайнаар дамжин өнгөрсөн байх болно. Хэрэв ийм зүйл болвол Дэлхийн хүн ам эрс багасах болно. Энэ геофизикийн үйл явдал болох үед үлдсэн хүмүүс магадгүй дахин хагас болно. Тэгээд хэн амьд үлдэх нь хэн дэлхий болон үлдсэн хүн амыг дараагийн эрин рүү хөтлөхийг шийдэх болно. Тиймээс бид асар том гамшгийн дараах эриний тухай ярьж байна. Хэн удирдах вэ? Хэн хяналт тавих вэ? Тэгэхээр энэ бүхэн үүний тухай юм. Тийм ч учраас тэд эдгээр зүйлсийг тодорхой хугацаанд болохыг хүсч байна... [Эмх замбараагүй байдал] болохоос өмнө ямар нэгэн тодорхой байдлаар амьд үлдэх бүтэц байх ёстой - ингэснээр энэ нь дараагийн өдөр хоёр хөл дээрээ зогсох, дараа нь хүчирхэг хэвээр байх, өмнө нь хүртсэн хүчийг хадгалах боломжтой болно"} \cite{4}. Ярилцлагын үеэр энэ төлөвлөгөөний нэр болох "Англо-Саксон Мисс"-ийг бас хэлэлцсэн: \textit{[Ярилцлагын явуулдагч]: "...үүнийг Англо-Саксон Мисс гэж нэрлэсэн шалтгаан нь үндсэндээ Хятадуудыг устгах төлөвлөгөө юм. Ингэснээр гамшгийн дараа бүх зүйлийг дахин бүтээх үед шинэ Дэлхийг өвлөн авах, дахин бүтээх байр суурийг Англо-Саксончууд эзэлнэ, өөр хэн ч үлдэхгүй. Энэ үү?" [Дотоод мэдээлэгч]: "Энэ зөв эсэхийг би үнэхээр мэдэхгүй, гэхдээ би таны хэлсэнтэй санал нийлнэ. Наад зах нь 20-р зуунд, тэр ч байтугай 19, 18-р зуунуудад энэ дэлхийн түүхийг голчлон Баруун болон Хойд бүсээс удирдсан"} \cite{4}.
Хүлээгдэж буй геофизикийн үзэгдлийн яг хугацааны талаар дотогч хүн өөрийн таамаглалаа дурьдсан: \textit{"...мэдрэмж, энэ бол маш зөн совингийнх, тэд одоо үйл ажиллагаагаа зохион байгуулах ёстой гэсэн... Тэд энэ хэзээ болохыг сайн мэдэж байгаа гэж бодож байна... \textbf{Надад энэ миний амьдралын хугацаанд, жишээ нь 20 жилийн дотор болно гэдэг маш хүчтэй мэдрэмж төрж байна}... бид одоо энэ геофизикийн үзэгдэл болох гэж буй үед орж ирлээ, сүүлчийн удаа 11,500 жилийн өмнө болсон гэж үзвэл, энэ нь ойролцоогоор 11,500 жил тутам давтагддаг циклтэй. Одоо дахин болох үе ирлээ... Тэд энэ болохыг ойлгож байна. Тэдэнд энэ болно гэдэгт итгэлтэй мэдлэг бий... Дахин хэлэхэд, энэ тэд мэдэхгүй байхын аргагүй зүйлсийн нэг юм. Дэлхийн хамгийн авьяаслаг олон талын эрдэмтэд энэ талаар тэдний төлөө ажиллаж байгаа"} \cite{4}.

Энэ бол бидний маш их талархдаг хүчирхэг баримт юм. Ярилцлагадаа зохиолч Дэлхийн I ба II дайныг зохиомол дайн байсан гэдэгт итгэдэг бөгөөд Англо-Саксон Эрхэм Зорилго нь олон, олон үеийн өмнөөс эхэлсэн гэдгийг хэлжээ. Ярилцлага 2010 онд болсон бөгөөд одоо 15 жил өнгөрсөн. Дотогч хүний геофизикийн үзэгдлийн таамагласан 20 жилийн хугацаа дуусахад 5 жил үлдсэн.
\subsection{Катастрофын Друдын Нууцлаг Барууны Мэдлэг}

Катастрофын давтамжтай тохиолдлын тухай Барууны мэдлэг нь зөвхөн Фримасонуудаар төдийлөн хадгалагдаагүй. Друдууд, дор хаяж 2400 жилийн түүхтэй кельтүүдийн эртний соёл \cite{7}, Дэлхийн давтагддаг гамшгийн талаарх мэдлэгээ дамжуулж ирсэн. Сүүлчийн мэдэгдэж буй друд нь Бен МакБради гэдэг. 1992 онд хийгдсэн "Сүүлчийн Друд" баримтат кинонд тэрээр друдуудын мэдлэгийн талаарх мэдээллийг хуваалцсан: \textit{"Миний уламжлал ёсоор сүүлчийн гишүүн байж болох энэ багц нь сүүлчийн агуу гамшиг, эсвэл дэлхийд нөлөөлсөн сүйрлийн дараа үүссэн. Одоо цахилгаантай агуу шуурга, солирын сүүл эсвэл солирын бороонд орсноор бидний мэдэх соёл иргэншил бүрэн сүйдэв... Бүх мэдлэг энэ багцын хүрээнд багтдаг байсан, гэхдээ тэд одон орон судлалд онцгой анхаардаг байсан, учир нь тэд маш олон чухал гамшгийг туулсан. Одон орон судлалын бүрэн мэдлэг нь эдгээр гамшгийн тохиолдох нөхцөлийг урьдчилан таамаглаж, өөрсдийгөө хамгаалах арга хэмжээ авах боломжийг олгоно гэж үздэг байсан. Ирланд дахь агуу мегалит цогцолборуудыг харвал гарцын булш гэж тодорхойлдог зүйлс нь үнэндээ махбодийн төрлийн бөмбөгний хамгаалалт юм. Тэд нь ямар ч далайн долгионы түвшнээс өндөр, мөн солирын борооноос хамгаалах боломжтой"} \cite{8,9}.

% Фримасоны үүсэл нь друдуудаас гаралтай гэдэгт итгэдэг \cite{10}.
\section{Орчин үеийн Барууны Аюулд Бэлтгэх Үйл Явдлын Нотолгоо}

Дэлхийн геофизикийн аюул нь ойртсон гэж Барууны ноёрхогч хүчнүүд итгэдэг бол тэд ийм үйл явдлаас хамгаалахын тулд ихээхэн бэлтгэл хийж байх хэрэгтэй. Үнэхээр ч олон Барууны орнуудад гүн газрын өргөн сүлжээний суурингууд байгаа гэдгийн нотолгоо нийтлэг хэрэглээнд байдаг. Ийм байгууламжууд цөмийн дайнаас оршин суугчдыг хамгаалахаас гадна олон төрлийн байгалийн гамшигаас хамгаалах боломжтой. Project Camelot-ын Британийн Ахмад Фримасонын гэрчлэлээр \cite{4,6} эдгээр хувилбарууд бол боломжит хувилбарууд биш, харин урьдчилан бэлтгэсэн төлөвлөгөө юм. Мөн эдгээр суурингуудыг барих, ажиллуулах, засварлахын тулд шаардагдах асар их хэмжээний мөнгө, жишээ нь 18 жилийн хугацаанд АНУ-ын засгийн газраас алга болсон хэдэн арван их наяд доллар \cite{11,12,13} зэрэгтэй тохирч байгаа нь анхаарал татахуйц. Устгалын түвшний үйл явдлын бэлтгэлийн бусад жишээнд үрийн болон мэдлэгийн бэхэлгээ зэрэг төрөл бүрийн архивчлалын төслүүд орно.
\subsection{Америкийн газар доорх ба далайн ёроолын баазууд}

Газар доорх баазуудын талаарх хамгийн өргөн цар хүрээтэй нийтлэг судалгааг би Ричард Саудер гэх бие даасан америк судлаачаас олсон бөгөөд тэрээр гүн газар доорх баазуудын талаар хэд хэдэн ном хэвлүүлсэн \cite{22}. Саудерыйн бүтээлүүд нь засгийн газрын баримт бичиг, төлөвлөгөөг архивлаж, түүхэн болон одоогийн мэдээ, технологийг судалж, эх сурвалжуудыг боловсруулж, дотогчдын мэдээллийг цуглуулах ажлаас бүрддэг. Саудерыйн судалгаагаар Америк болон түүний нутаг дэвсгэрт хамгийн багадаа 3 миль гүн хүртэлх гүнзгий газар доорх ба далайн ёроолын баазуудын том сүлжээ байдаг бөгөөд (Зураг \ref{fig:4}), магадгүй вакуум хоолойтой гүн газрын хурдан галт тэрэгний сүлжээгээр холбогдсон байж болох юм \cite{22}. Эдгээр баазууд нь Америкийн Нэгдсэн Улсын компанийг эзэмшдэг хүмүүсийн удирдсан \textit{"өндөр санхүү, олон улсын, агентлагуудын хоорондын мөнгө угаах бүрхэг тоглоом"} гэх зохицуулалтаар далд санхүүждэг \cite{22}. Катерин Остин Фиттс (түүний ажлыг дараагийн хэсэгт авч үзнэ) болон түүний хамтрагчдын нэгээр хийсэн дараагийн ажиллагаагаар эдгээр баазуудын хэмжээ 170 газар доорх ба далайн ёроолын америк бааз хүртэл гэж тооцоолсон \cite{16,20}.

\begin{figure}[b]
\begin{center}
% \fbox{\rule{0pt}{2in} \rule{0.9\linewidth}{0pt}}
   \includegraphics[width=1\linewidth]{penta.jpg}
\end{center}
   \caption{Цагаан ордон, Пентагоны доор юу нуугдаж байна вэ? Гүн газрын дамжуулах хоолойн сүлжээ нь илт харагдаж байна (Зураг: \cite{31}).}
\label{fig:3}
\label{fig:onecol}
\end{figure}
\begin{figure*}[t]
\begin{center}
% \fbox{\rule{0pt}{2in} \rule{.9\linewidth}{0pt}}
\includegraphics[width=0.9\textwidth]{basescrop.png}
\end{center}
\caption{Содэрийн судалгаагаар илрүүлсэн газар доорхи, далайн ёроолын бааз, түүнчлэн далайн гатлагаар эх газар руу чиглэсэн шуудангийн хонгилуудын яг байршлыг харуулсан газрын зураг. Содэр \textit{"энэ [эдгээрээс] \textbf{олон дахин илүү} байгууламж байгаа гэдэгт итгэлтэй байна"} \cite{22}.}
\label{fig:4}
\end{figure*}

Эдгээр баазуудын зарим нь хэрхэн тэлсэн тухай Содэрийн эх сурвалжаас авсан зарим баримтууд энд оруулав:
\begin{flushleft}
\begin{enumerate}
    \item Кэмп Дэвид, Мэрилэнд: \textit{"Миний мэдэгдэж буй эх сурвалжийн мэдээлснээр Кэмп Дэвидийн газар доорхи хэсгүүд нь маш өргөн цар хүрээтэй, нарийн зохион байгуулалттай бөгөөд нууц гарцууд нь хэдэн мянган километртэй тул энэ байгууламжийн бүрэн гүйцэд газрын зургийг нэг хүний толгойд багтаах нь эргэлзээтэй"} \cite{22}.
    \item Цагаан ордон, Вашингтон DC: \textit{"Миний ойр дотны нэг найз 1960-аад оны Линдон Б. Жонсоны засаглалын үед энэ байгууламжийн газар доорхи хэсэгт авч явсан. Тэрээр Цагаан ордны цахилгаан шатаар шууд доош бууж, 17 давхар доош буужээ. Хаалга нээгдэхэд газар доор урт коридор түүнийг угтсан бөгөөд тэр коридор нь холын зайд алга болж байгаа мэт харагдаж, түүнээс бусад хаалга, коридорууд салбарласан"} \cite{22}. \ref{fig:3} зурагт үзүүлэв.
    \item Форт Мид, Мэрилэнд - 1970-аад оны үед "подвалд" санамсаргүй орж ирсэн нэгэн эх сурвалжаас: \textit{"Би хаалгыг нээгээд доошоо хүрэх шат гарч ирэв. Би ирмэг рүү очоо хашлаганы хоорондоор доош харлаа. Би давхрын тоог тоолоогүй ч 15-20 давхар орчим гэсэн мэдрэмж төрсөн... Би нэг давхар доошоо бууж нэг хаалга оллоо... Би хаалгыг нээгээд толгойгоо дотогш сугаран зүүн, баруун тийш харахад хоёр чиглэлд харагдахгүйтлээ үргэлжилсэн хонгил харсан. Энэ нь гадаргуу дээрх барилга, зогсоолын талбайгаас хамаагүй илүү том хэмжээтэй байлаа. Эсрэг хананууд дээр 30-40 футын зайтай хаалганууд байрласан байв... Би хэдэн давхар дахиад нэг үзээд үзье гэж өөр нэг давхар доош буув... мөн адилхан төлөвтэй байлаа... Би дахиад нэг давхар доош бууж хартал эхний 2 давхартай ижилхэн зүйл харсан"} \cite{22}.
\end{enumerate}
\end{flushleft}

\begin{figure}[t]
\begin{center}
% \fbox{\rule{0pt}{2in} \rule{0.9\linewidth}{0pt}}
   \includegraphics[width=1\linewidth]{undersea.jpg}
\end{center}
   \caption{Далайн ёроолын баазын дүрслэл, Уолтер Көршнерын бүтээл. Тэр 1960-аад оны үед АНУ-ын Цэргийн Далайн Чина Лэйк, Калифорни дахь Зэвсгийн Төвд Rock-Site далайн ёроолын баазын багийн зураач байсан. Саудерыйн эх сурвалжуудын нэг нь Чина Лэйкт нэг милийн гүнд газар доорх бааз байдаг гэдгийг илчилсэн \cite{22,23}.}
\label{fig:5}
\label{fig:onecol}
\end{figure}

Содер мөн цагт 2000 миль хурдтай газар доорх соронзон дээшлүүлэх галт тэрэг, далайн ёроолд баригдсан бааз (Зураг \ref{fig:5}), болон эх газар руу хүрдэг усан доорхи шумбагч хонгилуудын талаар гэрчлэлүүд хүлээн авсан. Мексикийн буланд байрлах усан доорх баазын тухай нэг гэрчлэлийн тухай Содер хэлэхдээ, \textit{"Усан доор болон газар доорх баазууд ном хэвлэгдсэний хагас жилийн дараа надад нэгэн залуу холбогдон, тэрээр усан доорх ер бусын төслийн талаар мэдлэгтэй гэж хэлсэн... тэрээр Мексикийн булангийн ёроолд энэхүү төсөл байгаа бөгөөд Парсонс компани гүйцэтгэгч нь гэж тодорхойлсон. Тэрээр цааш нь Парсонс компани далайн ёроолоос 2800 фут гүнд ажиллахад зориулсан тусгай тоног төхөөрөмж худалдаж авсан гэж хэлсэн... Энэ тоног төхөөрөмж нь онцгой шинж чанартай тул суурилуулсан газруудад хүмүүс амьд байгааг тодорхой харуулж байна"} \cite{22}.
\begin{figure}[t]
\begin{center}
% \fbox{\rule{0pt}{2in} \rule{0.9\linewidth}{0pt}}
   \includegraphics[width=1\linewidth]{sub.jpg}
\end{center}
   \caption{Усны доорхи шумбагч онгоцны хонгилын дүрслэл, Валтер Көршнер \cite{22,23}.}
\label{fig:6}
\label{fig:onecol}
\end{figure}
\begin{figure}[t]
\begin{center}
% \fbox{\rule{0pt}{2in} \rule{0.9\linewidth}{0pt}}
   \includegraphics[width=1\linewidth]{iran.jpeg}
\end{center}
   \caption{Иран албан ёсны видеон дээрээс "пуужингийн хот" гэх газар доорхи цэргийн бааз-ыг харуулсан клип \cite{39,40}.}
\label{fig:12}
\label{fig:onecol}
\end{figure}
Хэрэв үнэхээр манай хөл доор хэдэн милийн гүнд ухагдсан, 170 гаруй газар доорх болон далайн ёроолын баазуудыг гиперсоник вакуум хоолойт соронзон төмөр замаар холбосон, бидний хөдөлмөрийн үр дүнгээр санхүүждэг нууц дамнасан сүлжээ байдаг бол өнөөгийн хүн төрөлхтөн зөвхөн өөрсдийн доор юу байгааг мэдэхгүй байхаас гадна ойрын ирээдүйд тэднийг юу хүлээж байгааг ч мэдэхгүй, улс төрчийнхөө зохицуулсан хоосон мэдэгдлүүдийг шуналтайгаар дагах terminal тайван мунхаг байдалд байх болно.

Нэмэлт тэмдэглэл - томоохон газар доорхи хонгилын сүлжээний оршин тогтнох нь Ойрхи Дорнодод үргэлжлэх зэвсэгт мөргөлдөөнөэр (Газын туулан дорх Hamas-ын хонгилууд \cite{38}, Ираны газар доорхи "пуужин хот" (Зураг \ref{fig:12}) \cite{39,40}) эргэлзээгүй илэрсэн. Эдгээр нь ийм бүтэц барих боломж болон бодит оршин тогтнох тухай аливаа эргэлзээг арилгах ёстой. Мөн эдгээр нь бусад хамаагүй илүү хөрөнгөтэй орнууд ижил хугацаанд ямар бүтэц бариулсан болох талаар биднийг гайхшруулах ёстой.
\subsection{Нэмэлт Бункер ба Катаклизм бэлтгэлийн Нотолгоо}

\begin{figure}[t]
\begin{center}
% \fbox{\rule{0pt}{2in} \rule{0.9\linewidth}{0pt}}
\includegraphics[width=1\linewidth]{tyrol.jpg}
\end{center}
   \caption{Швейцарийн Өмнөд Тирольд байрлах бункерууд. Европын Альпын нурууг хамарсан Швейцар улс өөрийн уулын бункеруудыг ухаалаг байдлаар далдлуулснаараа алдартай \cite{32}.}
\label{fig:7}
\label{fig:onecol}
\end{figure}

\begin{figure}[t]
\begin{center}
% \fbox{\rule{0pt}{2in} \rule{0.9\linewidth}{0pt}}
\includegraphics[width=1\linewidth]{svalbard.jpg}
\end{center}
   \caption{Норвегийн Свалбардын Дэлхийн Үрийн Сан, нэг сая гаруй үрийг агуулдаг \cite{24}. Ямар аюул осол гарвал үүнийг ашиглах шаардлагатай болохыг бодоход гайхалтай.}
\label{fig:8}
\label{fig:onecol}
\end{figure}

Дэлхийн өөр олон газарт гамшгийн бэлтгэлийн нэмэлт шинж тэмдгүүд байдаг, Америкийн газар доорх хааны баазуудаас гадна. Норвеги, Швейцарь, Швед, Финланд нь үүний гайхалтай жишээ юм:

\begin{flushleft}
\begin{enumerate}
    \item Project Camelot нь Норвегийн улс төрчээс холбогдох гэрчлэлтэй танилцсан \cite{25,26}, түүний үнэн зөвийг баталсан ч нууцлахыг сонгосон. Тэрээр Норвегид 18 өргөн хэмжээний газар доорх бааз байдаг бөгөөд Норвеги (Израиль болон "бусад олон орнуудтай" хамт) эдгээр баазуудыг ямар нэг байгалийн гамшгад бэлдэж байгаа гэж мэдүүлсэн. Ричард Саудер ч мөн Норвегийн нэг ууланд баригдсан асар том газар доорх бааз дотор байсан нэгэн хүнээс гэрчлэл авсан \cite{22}.
    \item Швейцарь Альпын өндөрлөгт баригдсан олон цөмийн бункертэй гэдгээрээ алдартай (Зураг \ref{fig:7}). Эдгээр нь 370,000-аас дээш тоогоор оршин суугч бүрт зориулсан хангалттай хамгаалалттай \cite{27}.
    \item Швед, Финланд нь томоохон хот бүрт оршин суугчдад зориулсан хангалттай бункертэй \cite{27}.
\end{enumerate}
\end{flushleft}

Силикон Хөндийн бизнесийн магнатууд мэдээж мэдэж байгаа. \textit{"LinkedIn-ийн хамтран үндэслэгч, алдартай хөрөнгө оруулагч Рейд Хоффман энэ жил The New Yorker-т хэлснээр Силикон Хөндийн тэрбумтнуудын 50\%-с дээш нь газар доорх бункер гэх мэт ямар нэг түвшний "апокалипсийн даатгал" худалдаж авсан байна... Forbes-ийн хувь нэмэр оруулагч Жим Добсоны хэлснээр олон тэрбумтнууд хувийн онгоцтой бөгөөд "ямар ч мөчид нисэхэд бэлэн" байдаг. Тэд мотортой дугуй, зэвсэг, цахилгаан үүсгүүр ч эзэмшдэг"} \cite{28}.

Мөн хүн төрөлхтний чухал хөрөнгийн санг устах түвшний аюулаас хамгаалж бэлтгэхээр Arch Mission Foundation-ийн удирдлага дор ажилладаг Global Knowledge Vault \cite{29}, Svalbard Global Seed Vault \cite{30} зэрэг олон томоохон архивчилсан төслүүд бий.
\begin{figure*}[t]
\begin{center}
% \fbox{\rule{0pt}{2in} \rule{.9\linewidth}{0pt}}
\includegraphics[width=0.9\textwidth]{govcrop2.png}
\end{center}
   \caption{АНУ-ын засгийн газрын орлого, зарлага, нууц газар доорхи баазын зардлын мэдээлэл 1998-2023 он \cite{19}.}
   \label{fig:9}
\end{figure*}
\section{Даяарчлагдсан Газар Доорхи Баазуудыг Ардчилсан Санхүүжүүлэх Механизмууд}

Тэгвэл 170 гаруй газар доорхи, далайн ёроолын баазуудыг асар том трансконтинентал сүлжээг хэрхэн өрөнд орсон боолуудад мэдэгдэлгүйгээр санхүүжүүлдэг вэ? Эдгээр төслүүдэд орох мөнгөн хэмжээ, гарал үүслийг ойлгоход тусалдаг ганц бичиг баримт бий. 2017 онд Бушын засгийн үеийн АНУ-ын хөрөнгө оруулалтын банкны ажилтан, төрийн албан хаагч байсан Катерин Остин Фиттс, Мичиганы Их Сургуулийн эдийн засагч Марк Скидмор нар 1998-2015 оны санхүүгийн жилүүдэд АНУ-ын засгийн газар 21 их наяд ам.долларыг зөвшөөрөлгүй зарцуулсан гэдгийг тогтоожээ \cite{11,12,13}.

Тэдний тайланд: \textit{"2016 оны 10-р сарын 7-нд Рейтер агентлаг Скот Палтроугийн (2016) нийтлэлийг гаргасан бөгөөд 2015 оны санхүүгийн жилд Арми нь өөрийн дансыг тэнцвэртэй гэж үзүүлэхийн тулд 6.5 их наяд ам.долларын баталгаажуулалтгүй дансны тохируулга хийсэн гэж мэдэгдсэн. Тэр жилийн Армийн ерөнхий санхүүжилт 122 тэрбум ам.доллар байсан тул энэ нь гайхалтай илрэл байв... 2001 оны 9-р сарын 10-нд Батлан Хамгаалах Яамны дансны асуудал томоохон хэвлэл мэдээллийн анхаарлыг татаж, Батлан Хамгаалахын сайд Доналд Рамсфельд Конгрессийн хуралдаан дээр (C-SPAN, 2014) БХЯ 2.3 их наяд ам.долларын гүйлгээгээ алдсан гэдгээ хүлээн зөвшөөрсөн... Энэ хүлээн зөвшөөрөл тухайн өдөр мэдээллийн төлөөллийг эзэмшиж байсан ч дараагийн өдөр 9/11-ийн гамшиг дэлхий дахины анхаарлыг татсан тул мартагдсан... Профессор Марк Скидмор Армийн 6.5 их наяд ам.долларын баталгаажуулалтгүй гүйлгээний тухай мэдээлэл сонсоод Фиттс хатагтайтай холбогдож, 2017 оны хавар ХХБО, БХЯ-д ер бусын том хэмжээний баталгаажуулалтгүй гүйлгээг илрүүлэхээр тохиролцсон. Дараагийн зургаан сарын турш Скидмор, Фиттс нарын багахан оюутнуудын баг албан ёсны засгийн газрын баримт бичигтэй ажиллаж, 1998-2016 оны хооронд нийт 21 их наяд ам.долларын баримтжуулалтгүй гүйлгээг тогтоосон"} \cite{12}.
1998-2015 оны хооронд 18 жилийн хугацаанд АНУ-ын засгийн газрын албан ёсоар хүлээн зөвшөөрөгдсөн орлого ердөө 40.8 их наяд ам.доллар байсан \cite{15}. Энэ нь АНУ-ын засгийн газрын албан ёсны зардлаас гадна нууц далд бааз барихад нийт орлогын хагасаас илүү хувийг нууцаар зарцуулсан гэсэн үг юм. Мөн энэхүү нууц зардал нь удаан хугацааны төсвийн алдагдал дээр нэмэгдсэн бөгөөд зөвхөн өнөөдөр хүртэл үргэлжилсэн төдийгүй 1998 оноос өмнө ч байсан байх магадлалтай. Иймд эдгээр баазуудад зарцуулсан нийт дүн 21 их наяд ам.доллараас хамаагүй их байна гэж үзэж болно. Нууц зардлын ижил харьцааг 2016-2023 оны хооронд хэрэглэвэл 1998 оноос хойш нийт 36.6 их наяд ам.доллар зарцуулсан гэж тооцоолж болно.

2021 онд Марк Скидмор Bloomberg-ийн мэдээлснээр 2017-19 оны санхүүгийн жилүүдэд Пентагон 94.7 их наяд ам.долларын нягтлан бодох бүртгэлийн тохируулга хийсэн тухай судалгааныхаа шинэчилсэн хувилбарыг нийтэлсэн \cite{17,18}. 1913 онд Холбооны Нөөцийн Систем байгуулагдсанаас хойш нэг зуун гаруй жилийн турш АНУ-ын төв банкны системээр долларыг хуурамчаар үйлдсэн явдлыг \cite{37} харгалзан үзвэл, нийтийн долларын бүртгэл бүхэн хоосон яриа болох нь илт харагдаж байна. АНУ-ын валют болон засгийн газар нь зөвхөн түүний хаадын эзэд хүссэн хэмжээгээ чимээгүйхэн хураан авах (эсвэл илүү нарийн хэлбэл, асар их хэмжээгээр зарцуулах) нөөцийн хуваарилалтын систем юм.
\section{Жовийн Үр: Сүүдрийн Барууны Ноёдын Үндэс}

Тэгэхээр, хэн үнэхээр удирдлага гаргаж байна вэ? Бид үүнийг баттай мэдэх боломжгүй, учир нь капиталын Барууны ноёд өөрсдийгөө сүүдэрт нуудаг. Нийтийн нэр хүндтэй хүмүүсээс гадна гаригийн хачин биет хүртэл олон төрлийн онолууд байдаг ч, миний хамгийн сайн хариулт нь "Амаллула" гэх нэрээр алдаршсан нэрээ нууцласан блогчийн амьдралын бүтээлд оршдог. Түүний бүтээл нь эртний болон орчин үеийн түүх, нууцлаг бэлгэдэл, Барууны улс төрийн сэдвээр 20 гаруй зохиолч, 50 гаруй "орлуулшгүй" баримт бичгийг нэгтгэсэн томоохон бүтээл байсан \cite{33,34}. Би түүний бүтээлийг дайрахад бэлэн геофизикийн гамшгийн талаар "шүтээнчлэлтэй" гэж тодорхойлж болно - энэ нь минийхээс \textit{илүү} өргөн хүрээтэй.

Амаллула "сүүлчийн цаг" буюу Дэлхийн давтагддаг гамшгийн талаар мэдлэгтэй гэж үздэг "Жовийн Үр" гэж нэрлэсэн гурван Барууны улс төрийн бүлгийг тодорхойлсон. Тэрээр эдгээр гурван бүлэг хамтдаа өнөөдөр Барууны орнуудыг удирдаж байгаа гэдэгт итгэдэг ч, тэдний гарал үүсэл, түүхэн үндэс, өмнөх санал зөрчил, үнэт зүйлсийн систем болон үйл ажиллагааны ялгааг үндэслэн гурван өөр бүлэгт хуваадаг.
Гурван фракцийг дараах байдлаар тодорхой бус ангилж болно:

\begin{flushleft}
\begin{enumerate}
    \item \textbf{Банкчид}: Эртний Ромын элитүүд, тэд Америк дахь Knights Templar болон Freemasons-ийн Northern Jurisdiction болон хувирсан.
    \item \textbf{Бодогчид}: Rosicrucians болон Өмнөд Америкийн Freemasons.
    \item \textbf{Езуитчүүд ба Хар Пап}: Ромын Католик сүмийн дотор Jove-ийн үр удам фракц.
\end{enumerate}
\end{flushleft}
Өнөөдөр эдгээр гурван бүлэг нь Европын Иллюминати, Фримасоны, болон ТХК-г бүрдүүлдэг. Амаллугийн тайлбарласанчлан, \textit{"Одоогийн байдлаар, эцсийн цаг үед, Жовеийн удам үр хойчис АНУ-ын одоогийн Ерөнхийлөгчийг ч хамарсан нууцлалын түвшний ард нуугдаж байна. Өөрөөр хэлбэл, тэд нийтэд харагдахаас нуугдахуйц ур чадвараа төгөлдөржүүлсэн. \textbf{Жовеийн удам хойчис зөвхөн АНУ-ын цэрэг, засгийн газрыг удирддаг төдийгүй, хуурамч валют, томоохон корпораци, өөрсдийн зохион бүтээсэн Бүгд Найрамдах засаглалын хэлбэрийн (улс төрийн хүмүүсийг хялбархан залилан мөхөөлдөг, улмаар хяналтандаа оруулах боломжтой гэдгийг мэдэж байсан) хүчээр бүх Барууны ертөнцийг удирддаг}"} \cite{33,34}.

\begin{figure}[t]
\begin{center}
% \fbox{\rule{0pt}{2in} \rule{0.9\linewidth}{0pt}}
   \includegraphics[width=1\linewidth]{illuminati.jpg}
\end{center}
   \caption{Жовын үр хүүхдүүд хэн бэ? (Зураг: \cite{35})}
\label{fig:10}
\label{fig:onecol}
\end{figure}

\begin{figure}[t]
\begin{center}
% \fbox{\rule{0pt}{2in} \rule{0.9\linewidth}{0pt}}
   \includegraphics[width=1\linewidth]{pike.jpg}
\end{center}
   \caption{Улаанаар тодруулсан алдарт Пайк Пик батолит ба АНУ-ын баруун бүсийн гадаргуу \cite{36}. Америкийн Нэгдсэн Улс үнэхээр энэ байршлыг хянахын тулд бүтээгдсэн гэж үү?}
\label{fig:11}
\label{fig:onecol}
\end{figure}

Амаллулагийн хэлснээр, эдгээр хүмүүс шашинг үл тоомсорлож, дэлхийн томоохон шашнуудын ариун номнуудыг өөрсдийн ашиг тусын тулд ашигладаг бөгөөд бэлгэдлийг өргөн ашигладаг. Түүнчлэн, тэд өрсөлдөгчдөөтэй холбоотойгоор харгис хэрцгий байдаг: \textit{"\textbf{2600 гаруй жилийн хугацаанд тэд эцсийн цаг үеийн тодорхой мэдлэгтэй байсан бусад хэн бүхнийг системтэйгээр устгасан. Үүнд би зөвхөн друид, еврей каббалист, эртний египетчүүд, араб, энэтхэгийн мөргөлчдийг хэлэхгүй, харин Өмнөд Америкийн Урт Толгойтнууд болон Төв Америкийн Маяа лам нар ч бас багтана. Тэдэн Хойд Америкт эцсийн цаг үеийн Газрыг хадгалахын тулд нэгэн удаа цэцэглэн хөгжиж байсан хүн амыг устгасан гэдгийн баримт нь бүрэн давамгайлсан. Америкийн "индиан" үндэстнийг устгах нь зөвхөн үлдэгдлийг цэвэрлэх ажиллагаа байсан}"} \cite{33,34}.
Амалла мөн "Америкийн Нэгдсэн Улс" төсөл бүхэлдээ "Пайкс Пийн батолит"-ийг хяналтандаа авах зорилгоор хэрэгжүүлсэн гэдэгт итгэдэг. Энэ нь Рокки нурууны боржин чулуулагтай уулын нутаг бөгөөд газарзүйн гамшгаас маш сайн хамгаалалт өгдөг (Зураг \ref{fig:11}). Амаллагийн хэлснээр, \textit{"Бидний Иргэний дайн гэж үздэг үйл явдлын өмнө, үеэр, дараа ч банкир болон бодлогчид Америкийн Нэгдсэн Улсыг хянахын тулд төдийлөн тэмцээгүй, харин дэлхий дээрх хамгийн өвөрмөц боржин батолит болох Пайкс Пийн батолитыг хянахын тулд тэмцсэн... Дэлхийн хаана ч далайн эргээс хол, ийм өндөрт орших өөр боржин батолит байхгүй. Энэ нь газрын гадаргын шилжилтээс амьд үлдэх хамгийн тохиромжтой газар юм"} \cite{33,34}. Амаллагийн судалгаагаар өнөөдөр энэ нутаг дэвсгэрийн доор болон эргэн тойронд өргөн хэрэглээний хонгилын систем баригдсан байна \cite{36}.

\section{Дүгнэлт}

Энэхүү өгүүлэлд Барууны элитүүд дэлхийн давтагддаг сүйрлүүдийн талаархи мэдлэгээ мянгаад жилийн турш анхааралтай хадгалж ирсэн, дахин ийм сүйрэл дөхөж байгаа гэдэгт итгэдэг, ийм үйл явдлыг даван туулахын тулд өргөн хэрэглээний газар доорхи оромж бариулсан, мөн ийм үйл явдлыг дэлхийн ноёрхолд хүрэх улс төр, цэргийн ашиг сонирхолд ашиглахаар төлөвлөж байгаа тухай олон төрлийн баримтуудыг дэлгэрэнгүй танилцууллаа. Америкт үүнийг хэрхэн санхүүжүүлсэн тухай товч мэдээлэл, мөн энэ бүхний ард хэн байгаа тухай хамгийн бодитой онолуудыг дурьдсан. Илүү ихийг мэдэхийг хүсвэл миний дурдаагүй нэмэлт мэдээллийг лавлахаас олж болно.
Газарын геофизикийн үйл явдалд хүргэх хамгийн хүчтэй хэмжигдэхүүн цэг бол Дэлхийн геомагнет талбарын хурдан шилжилт юм. Энэ нь зөвхөн соронзон хойд туйлын хурдацтай хөдөлгөөн (Зураг \ref{fig:13}), Өмнөд Атлантын геомагнет гажилтын тэлэлтээр төдийгүй сүүлийн 400 жилийн туршид геомагнет талбарын ерөнхий суларч, гажих хурдацтай явцаар хэмжигдэж байна \cite{3}. Иймэрхүү шинжлэх ухааны өгөгдлийг миний вэбсайтад байгаа анхны хоёр ECDO илтгэлд дэлгэрэнгүй хэлэлцсэн байдаг \cite{3}.

\begin{figure}[t]
\begin{center}
% \fbox{\rule{0pt}{2in} \rule{0.9\linewidth}{0pt}}
   \includegraphics[width=1\linewidth]{npw.jpg}
\end{center}
   \caption{1590-2025 оны хооронд соронзон хойд туйлын байрлалыг 5 жилийн алхамаар харуулсан \cite{41}. Түүний хөдөлгөөн 1975 оноос хурдацтай нэмэгдсэн.}
\label{fig:13}
\label{fig:onecol}
\end{figure}

Эцэст нь хэлэхэд, би та бүхэнд Амаллулагийн "бүх зүйл бол нэг зүйл" гэдэг утга учиртай ишлэлийг үлдээе: \textit{"Энд би таны төсөөллийг хамгийн хязгаарт хүртэл шахах шаардлагатай. Та одоо амьдарч байгаа, багаасаа мэддэг байсан ертөнцийг март. Үүнийг ардаа үлдээгээрэй. Энэ бол Матриц кинонд дүрслэгдсэнтэй адил бүтээгдсэн бодит бус ертөнц бөгөөд сүүлчийн мөч хүртэл таныг унтуулах зорилготой. Заримдаа би киноны зохиол бичиж байгаа болоосой гэж хүсдэг, гэхдээ энэ вэбсайтаар хуваалцаж буй зүйл бол бодит зүйл юм. "Бүх зүйл бол нэг зүйл" гэдгийг ойлгоход би хагас зуунаас илүү хугацаа зарцуулсан бөгөөд үүнийг "Апокалиптик синтез"-ийн уриа болгон хурдан хүлээн зөвшөөрсөн. Энэ бол дамжуулахад хэцүү ойлголт юм. Одоохондоо Матриц киноны жишээг авч үзье. Энэ бол сайн зүйрлэл юм. Миний хэлэх гэсэн зүйл бол хэтрүүлэл биш гэдгийг тайлбарлахад хэцүү байна. Одоохондоо Матриц киноны зүйрлэл бол миний хэлэх гэж буй бодит байдлыг ойлгуулах хамгийн ойрхон арга юм. \textbf{Таны амьдрал дахь бүх зүйл, түүхэн тэмдэглэл, ердийн шинжлэх ухаан ба академи, улс төр, шашин, бүх зүйл нь нэг талаараа дэлхийн гадаргуугийн шилжилт эсвэл тэнхлэгийн хазайлттай холбоотой.} Та үүнийг яг одоо харж чадахгүй байна. Та мөн муу зүүднээс сэрсэн мэт энэ бодит байдлыг анзаарч чадахгүй. Энэ нь цаг хугацаа шаарддаг. Гэхдээ би танд амлаж байна, энэ замын төгсгөл нь та бүхэн амьдралынхаа туршид Матриц компьютерийн симуляцлагдсан бодит байдалд амьдарч байсныг ойлгох болно"} \cite{33,34}.
Бүх хүмүүст амжилт хүсье.

\section{Талархал}

Нийтийн мэдлэгийн санд мэдлэгээ хуваалцсан бүх хүмүүст талархал илэрхийлье. Таныгүйгээр энэ бүтээл биелэхгүй, хүн төрөлхтөн харанхуйд үлдэх байсан. Таны сонголт мөнхөд цэцэглэнэ. Бид танд бүхнээ өртэй бөгөөд би танд төгс талархаж байна.
\clearpage
\twocolumn

{\small
\renewcommand{\refname}{Лавлагаа}
\bibliographystyle{ieee}
\bibliography{egbib}
}
\end{document}
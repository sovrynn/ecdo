\documentclass[10pt,twocolumn,letterpaper]{article}

% My own stuff
\usepackage{booktabs}
% \usepackage{caption}
% \captionsetup[table]{skip=8pt}   % Only affects tables
\usepackage{stfloats}  % Add this to the preamble

\usepackage[utf8]{inputenc}
\usepackage[LAE]{fontenc}
\usepackage[farsi]{babel}
\TOCLanguage{farsi}

\usepackage{amsmath}




\usepackage{cvpr}
\usepackage{times}
\usepackage{epsfig}
\usepackage{graphicx}
\usepackage{amssymb}

% Include other packages here, before hyperref.

% If you comment hyperref and then uncomment it, you should delete
% egpaper.aux before re-running latex.  (Or just hit 'q' on the first latex
% run, let it finish, and you should be clear).
\usepackage[breaklinks=true,bookmarks=false]{hyperref}

\cvprfinalcopy % *** Uncomment this line for the final submission

\def\cvprPaperID{****} % *** Enter the CVPR Paper ID here
\def\httilde{\mbox{\tt\raisebox{-.5ex}{\symbol{126}}}}

% Pages are numbered in submission mode, and unnumbered in camera-ready
%\ifcvprfinal\pagestyle{empty}\fi
\setcounter{page}{1}
\begin{document}

%%%%%%%%% TITLE
\title{ECDO داده‌محور قسمت ۲/۲: بررسی ناهنجاری‌های علمی و تاریخی که بهترین توضیح برای "چرخش زمین" ECDO است}

\author{جونهو\\
وبسایت: \href{https://sovrynn.github.io}{sovrynn.github.io}\\
مخزن تحقیقاتی ECDO: \href{https://github.com/sovrynn/ecdo}{github.com/sovrynn/ecdo}\\
{\tt\small firstauthor@i1.org}
}

\maketitle
%\thispagestyle{empty}

%%%%%%%%% ABSTRACT
\begin{abstract}
در ماه می ۲۰۲۴، یک نویسنده آنلاین با نام مستعار "شهرآزمای اخلاقی" \cite{0} نظریه‌ای بنیادین بنام جدایش هسته-گوشته اگزوترمیک به همراه نوسان ژانیبکوف (ECDO) منتشر کرد \cite{1}. این نظریه نه تنها پیشنهاد می‌کند که زمین دچار تغییرات ناگهانی و فاجعه‌بار در محور چرخشی خود شده است و باعث ایجاد سیل عظیمی در سراسر جهان با رانش اقیانوس‌ها به روی قاره‌ها به دلیل اینرسی چرخشی می‌شود، بلکه فرایند فیزیکی علّی توضیحی را نیز پیشنهاد می‌کند که داده‌هایی ارائه می‌دهد که نشان می‌دهد ممکن است چرخش دیگری نیز قریب‌الوقوع باشد. در حالی که پیش‌بینی‌های سیل و قیامت فاجعه‌بار جدید نیستند، نظریه ECDO به خاطر رویکرد علمی، مدرن، چند‌رشته‌ای و مبتنی بر داده‌های خود به‌ویژه جذاب است.

این مقاله تحقیقی تشکیل‌دهنده بخش دوم یک خلاصه فشرده دو بخشی از ۶ ماه تحقیق مستقل \cite{2,20} در نظریه ECDO است، که تمرکز ویژه‌ای بر ناهنجاری‌های علمی و تاریخی دارد که بهترین توضیح برای "چرخش زمین" فاجعه‌بار ECDO است.

\end{abstract}

%%%%%%%%% BODY TEXT

\section{مقدمه}

زمین‌شناسی و تاریخ یکنواخت‌گرا مدرن ادعا می‌کنند که چشم‌اندازهای زمین‌شناسی اصلی seperti دره گرند کنیون در طی میلیون‌ها سال تشکیل شده‌اند \cite{143}; که نمک در دره مرگ (کالیفرنیا) وجود دارد چون صدها میلیون سال پیش زیر اقیانوس بوده است \cite{144}; که اجداد ما از ۱۵۰ نسل پیش تمام عمر خود را صرف ساختن قبرهای عظیم کردند \cite{29,70}; و اینکه سوخت‌های فسیلی به‌اصطلاح صدها میلیون سال قدمت دارند \cite{104}. شاید شگفت‌انگیزترین نکته این باشد که انسان‌ها ۳۰۰,۰۰۰ سال قدمت دارند \cite{145}، با این حال تاریخ و تمدن ثبت‌شده تنها به حدود ۵,۰۰۰ سال - معادل ۱۵۰ نسل انسانی - بازمی‌گردد.

این ناهنجاری‌ها، همانطور که خواهیم دید، بهترین توضیح با نیروهای زمین‌شناسی فاجعه‌بار است.

\section{ماموت‌های فوری منجمد شده و دفن شده در گل و لای}

\begin{figure}[t]
\begin{center}
   \includegraphics[width=1\linewidth]{jarkov-mammoth.jpg}
\end{center}
   \caption{ماموت جارکوف، یک ماموت سیبریایی ۲۰,۰۰۰ ساله که به طور کامل در گل و لای منجمد شده یافت شده است \cite{51}.}
\label{fig:1}
\label{fig:onecol}
\end{figure}

یکی از این دسته ناهنجاری‌ها، ماموت‌های کاملاً سالم و منجمد در گل است که معمولاً در مناطق قطبی یافت می‌شوند (شکل \ref{fig:1}). ماموت برزوسکا، که در سیبری در شن‌ و ماسه‌های گلی دفن شده بود، به قدری سالم بود که گوشت آن هزاران سال پس از مرگش قابل خوردن بود. در دهان و معده‌اش نیز غذای گیاهی داشت که دانشمندان را به این مسئله معما وا‌داشت که چگونه ممکن است به سرعت منجمد شده باشد اگر تا لحظات مرگش در حال چریدن بر گیاهان گل‌دار بوده است \cite{17}. گفته شده است، \textit{"در سال ۱۹۰۱ پس از کشف یک جسد کامل ماموت در نزدیکی رودخانه برزوسکا، نوعی جنجال ایجاد شد، زیرا به نظر می‌رسید این حیوان در میانه‌ی تابستان از سرما مرده باشد. محتویات معده‌اش به خوبی حفظ شده و شامل کره‌های گل‌دار و لوبیای وحشی گل‌دار بوده است: این به این معناست که باید آنها را حدود پایان ژوئیه یا ابتدای اوت ببلعیده باشد. موجود به قدری ناگهانی مرده بود که هنوز دسته‌ای از علف‌ها و گل‌ها را در فک تاش نگه‌ داشته بود. واضح بود که با یک نیروی عظیم گرفته و چند مایل از مکان چراگاهش به دور پرتاب شده است. لگن و یک پایش شکسته شده بود - این موجود بزرگ بر زانوهایش افتاده و سپس در گرم‌ترین زمان سال از سرما مرده بود"} \cite{18}. علاوه بر این، \textit{"[دانشمندان روسی] ثبت کرده‌اند که حتی لایه داخلی معده حیوان نیز دارای ساختار فیبری به طور عالی حفظ شده‌ است، که نشان‌دهنده این است که گرمای بدن او توسط فرآیندی فوق‌العاده قدرتمند در طبیعت از میان رفته است. سندرسون، با توجه خاص به این نکته، این مسئله را به موسسه غذاهای منجمد آمریکا برد: برای یخ‌زدگی کامل یک ماموت چه نیازی هست تا محتوای رطوبت حتی درونی‌ترین قسمت‌های بدن او، حتی تا لایه داخلی معده او، فرصت کافی برای تشکیل کریستال‌های بزرگی نداشته باشد که ساختار فیبری گوشت را نابود کند؟... چند هفته بعد، موسسه پاسخ را به سندرسون بازگرداند: این امر کاملاً غیرممکن است. با تمام دانش علمی و مهندسی ما، هیچ راه شناخته شده‌ای وجود ندارد که گرمای بدن را از یک جسد به بزرگی یک ماموت به سرعت کافی برای یخ‌زدگی برداریم بدون اینکه کریستال‌های بزرگ رطوبتی در گوشت شکل بگیرد. علاوه بر این، بعد از اینکه تمامی تکنیک‌های علمی و مهندسی را امتحان کردند، به طبیعت نگاه کردند و نتیجه‌گیری کردند که هیچ فرآیند شناخته شده‌ای در طبیعت وجود ندارد که بتواند این کار را انجام دهد"} \cite{19}.

\section{گراند کنیون}

گراند کنیون، بخشی از حوضه بزرگ در جنوب غربی آمریکای شمالی، یکی از پدیده‌های طبیعی دیگر است که منشأ فاجعه‌آمیزی را پیشنهاد می‌دهد (شکل \ref{fig:2}). برای شروع، لایه‌های سنگ ماسه و سنگ آهک رسوبی که گراند کنیون را تشکیل می‌دهند، مناطق وسیعی به وسعت ۲.۴ میلیون کیلومتر مربع را پوشش می‌دهند \cite{21}. شکل \ref{fig:3} گستردگی لایه سنگ ماسه کوکونینو را در سراسر غرب ایالات متحده نشان می‌دهد. چنین لایه‌های افقی عظیم و یکنواختی از مواد فقط می‌تواند همزمان گذاشته شده باشند.

\begin{figure}[b]
\begin{center}
   \includegraphics[width=1\linewidth]{grand-canyon.jpg}
\end{center}
   \caption{گراند کنیون، در آریزونا، ایالات متحده \cite{49}.}
\label{fig:2}
\label{fig:onecol}
\end{figure}

\begin{figure}[t]
\begin{center}
   \includegraphics[width=1\linewidth]{coconino.jpg}
\end{center}
   \caption{اندازه لایه سنگ ماسه کوکونینو در غرب ایالات متحده \cite{21}.}
\label{fig:3}
\label{fig:onecol}
\end{figure}

یک نگاه دقیق‌تر به گرند کنیون به ما می‌گوید که رسوب‌گذاری این لایه‌های گسترده‌ی رسوبات نیز به طور همزمان با نیروهای تکتونیکی قابل توجهی رخ داده است. برای درک این موضوع، باید به مناطقی خاص در کنیون نگاهی بیندازیم که در آن لایه‌های رسوبات خم شده و مکشوف شده‌اند. محققانی از Answers in Genesis \cite{42} به نمونه‌های سنگی از برخی از این خم‌ها، مانند Monument Fold، به صورت میکروسکوپی نگاه کردند و با توجه به نبود ویژگی‌هایی که اگر این خم‌ها طی زمان‌های طولانی تحت فشار و حرارت تشکیل شده بودند باید دیده می‌شد، نتیجه‌گیری کردند که لایه‌های رسوبات توسط نیروهای تکتونیکی در حالی که هنوز نرم بودند، یعنی به زودی پس از رسوب‌گذاری آن‌ها، خم شده‌اند \cite{43}.

\begin{figure*}
\begin{center}
\includegraphics[width=1\textwidth]{Grand_Staircase-big.jpg}
\end{center}
   \caption{لایه‌های رسوبی که گرند کنیون (سمت راست تصویر) را تشکیل می‌دهند مستقیماً به سمت شمال تا Cedar Breaks در یوتا (سمت چپ تصویر) کشیده می‌شوند، جایی که آن‌ها به سمت بالا خم شده‌اند \cite{50}.}
\label{fig:4}
\end{figure*}

با یک نگاه وسیع‌تر، متوجه می‌شویم که لایه‌هایی که گرند کنیون را تشکیل می‌دهند، فقط داخل کنیون خم نشده‌اند. لایه‌ها در شرق در East Kaibab Monocline \cite{46} و همچنین به سمت شمال در Cedar Breaks، یوتا خم شده‌اند (شکل \ref{fig:4}). این موضوع نشان می‌دهد که این لایه‌ها ممکن است روی هم به صورت سریع و پیاپی رسوب‌گذاری و پس از آن همزمان خم شده باشند. برای مرجع، لایه‌های افقی گرند کنیون حدود ۱۷۰۰ متر ضخامت دارند. مقیاس فرآیند زمین‌شناسی مورد نیاز برای ایجاد لایه‌های رسوبی با این ضخامت عظیم بسیار بزرگ است.

تشکیل واقعی گرند کنیون نیز یک مسئله اختلافی در زمین‌شناسی مدرن است. زمین‌شناسی یکنواخت‌گرا پیشنهاد می‌کند که گرند کنیون توسط رودخانه کلرادو طی میلیون‌ها سال شکل گرفته است \cite{47}. با این حال، تیم تحقیقاتی Answers in Genesis معتقد است که گرند کنیون احتمالاً در عرض چند هفته به دلیل فرسایش خروجی از یک دریاچه باستانی که مرزهای خود را شکست و جرمی از رسوبات را از بین برد، شکل گرفت. شواهدی از یک دریاچه با ارتفاع بالا در شرق گرند کنیون در رسوبات دریاچه‌ای و فسیل‌های دریایی وجود دارد. مقایسه گرند کنیون با نمونه‌های وسیع از فرسایش خروجی، مانند Afton Canyon و Mount St. Helens، نشان می‌دهد که توپولوژی مشابهی دارد و نشان می‌دهد که دره‌های بزرگی می‌توانند به سرعت از طریق جریان زیاد آب سرازیر ایجاد شوند \cite{48}.

با توجه به مقیاس فرآیندهای زمین‌شناسی مورد نیاز برای رسوب‌گذاری این لایه‌های عظیم خاک، همزمانی نیروهای بزرگ تکتونیکی بلافاصله پس از رسوب‌گذاری لایه‌های رسوبی، و اندازه بسیار کوچک رودخانه کلرادو در مقایسه با مقیاس عظیم گرند کنیون، به نظر می‌رسد هیچ چیز تدریجی در مورد تشکیل آن وجود نداشته باشد.

\section{شهر زیرزمینی درینکویو}

علاوه بر اهرام، یک نمونه عالی از مهندسی باستانی شهر زیرزمینی درینکویو (شکل \ref{fig:5}) است که در کاپادوکیا، ترکیه قرار دارد. این بزرگترین پناهگاه زیرزمینی در میان بیش از ۲۰۰ پناهگاه زیرزمینی در این منطقه است \cite{54}. این شهر زیرزمینی تخمین زده می‌شود تا ۲۰,۰۰۰ نفر را در خود جای دهد و شامل ۱۸ طبقه بوده و به عمق ۸۵ متر می‌رسد. با اینکه سن آن قطعی نیست، تخمین زده می‌شود که حداقل ۲۸۰۰ سال قدمت داشته باشد. این شهر از سنگ‌های نرم آتش‌فشانی تراشیده شده است \cite{52, 53}.

\begin{figure}[b]
\begin{center}
   \includegraphics[width=1\linewidth]{derinkuyu.jpeg}
\end{center}
   \caption{نمودار شهر زیرزمینی درینکویو \cite{56}.}
\label{fig:5}
\label{fig:onecol}
\end{figure}
دلیل جالب بودن درینکویو این است که مشخص نیست چرا یک جامعه تصمیم گرفته که کل شهر را زیر زمین بسازد. برای ایجاد فضای زندگی زیر زمین، هر حفره‌ای باید از سنگ تراشیده شود. شکل‌ها و بافت‌های خشن تونل‌های زیرزمینی به وضوح نشان می‌دهد که این‌ها با نیروی کار دستی حک شده‌اند، نه با ابزارهای قدرتی، که از نظر میزان سختی چندین برابر ساخت پناهگاه‌ها بر روی زمین بوده است. در واقع، مشخص نیست چرا انسان می‌خواهد در طول زندگی محدود خود به طور دائمی زیر زمین زندگی کند، زمانی که کشاورزی، نور خورشید، طبیعت و اکتشاف فقط بر روی زمین فراهم است. "تاریخ" سنتی پیشنهاد می‌کند که درینکویو توسط مسیحیانی که نیاز به مکانی خلوت برای تمرین دین خود داشتند، ایجاد شده است \cite{53}. اما عقل سلیم نتیجه می‌گیرد که راه مستقیم برای مقابله با دشمن "جنگ یا فرار" است، نه "حفر یک شهر زیرزمینی از سنگ".

مقیاس، عمق و دقت طراحی شهر زیرزمینی به وضوح نشان می‌دهد که به عنوان یک ساختار دفاعی نظامی موقت برای بهتر مبارزه با مهاجمین در زمان تنگنا طراحی نشده است، بلکه یک پناهگاه بلند مدت برای محافظت در برابر نیروهای کشنده بر روی سطح است. درینکویو فقط دارای اتاق‌های خواب، آشپزخانه‌ها و حمام‌های اساسی نبود، بلکه اصطبل‌ها برای حیوانات، مخازن آب، انبارهای غذا، دستگاه‌های فشار شراب و روغن، مدارس، نمازخانه‌ها، مقبره‌ها و چاه‌های تهویه عظیم نیز تجهیز شده بود (شکل \ref{fig:6}). چرا یک پناهگاه نظامی به یک دستگاه فشار شراب نیاز دارد و باید به عمق ۸۵ متر با این پیچیدگی کنده شود؟

محتمل‌ترین توضیح برای ایجاد درینکویو می‌توانسته نیازی فوری برای آماده‌سازی یک پناهگاه بلند مدت و خودپایدار بوده باشد تا در برابر نیروهای ژئوفیزیکی فاجعه‌باری که روی سطح زمین رخ می‌دهد محافظت کند.

\begin{figure}[t]
\begin{center}
   \includegraphics[width=1\linewidth]{derinkuyu-air.jpg}
\end{center}
   \caption{یک چاه تهویه عمیق در درینکویو \cite{53}.}
\label{fig:6}
\label{fig:onecol}
\end{figure}

\section{انباشت‌های زیست‌توده}

مخلوط‌های زیست‌توده از انواع مختلف حیوانات و گیاهان، که اغلب به شکل فسیل در لایه‌های رسوبی یافت می‌شود، یک ناهنجاری معماگونه دیگر هستند. در "Reliquoæ Diluvianæ"، کشیش ویلیام باکلند یافته‌های گونه‌های متعدد حیات وحش را شرح می‌دهد که بدون دلیل قابل توضیحی در کنار هم پیدا شده‌اند، پراکنده در سراسر بریتانیا و اروپا، دفن شده در لایه‌های رسوبی 'دیلوویوم' \cite{58}. چنین مخلوط‌هایی از باقی‌مانده‌های حیوانی نیز در غار Skjonghelleren در جزیره والدروی نروژ یافت شد. در این غار، بیش از ۷۰۰۰ استخوان از پستانداران، پرندگان و ماهی‌ها در چندین لایه رسوبی مخلوط یافت شد \cite{59}. مثال دیگری غار سن سیرو، غار "غول‌ها"، در ایتالیا است. در این غار، چندین تن استخوان پستانداران، عمدتاً اسب آبی، در حالتی تازه یافت شد که به زیورآلات تبدیل و برای ساخت زغال لامپ ارسال شد. استخوان‌های حیوانات مختلف به هم مخلوط شده، شکسته، خرد و به قطعات پراکنده گزارش شده‌اند \cite{60,61}. در مندس باستانی در مصر، مخلوطی از گونه‌های مختلف استخوان حیوان با گل شیشه‌ای (ویتریفیفید) یافت شد \cite{57}. چنین یافته‌هایی ممکن است معماگونه به نظر برسند، اما به سادگی توسط سیل‌های عظیم که ترکیبی از حیوانات مرده را در لایه‌های رسوبی می‌گذارند، توضیح داده می‌شود، حیوانات را داخل یا دفن شده در غارها می‌کند و در مورد زیست‌توده شیشه‌ای در مصر، تخلیه‌های الکتریکی عظیم پس از سیل ناشی از جابجایی هسته-پوسته. شکل \ref{fig:7} یک مواجهه‌ معمول از 'ماده لزج' زیست‌توده آلاسک را نشان می‌دهد \cite{56}.

\begin{figure}[t]
\begin{center}
   \includegraphics[width=1\linewidth]{muck-crop.jpeg}
\end{center}
   \caption{'لزج' آلاسک، متشکل از قطعات پراکنده و نابسامان درختان، گیاهان و حیوانات در لای فریز شده و یخ \cite{146}.}
\label{fig:7}
\label{fig:onecol}
\end{figure}

\section{پناهگاه‌های باستانی}

نیاکان ما ساختارهای باستانی مهندسی شده بسیاری به جا گذاشته‌اند که در آن بقایای انسانی یافت شده است. این‌ها معمولاً به عنوان مقبره‌های پرزرق و برق تعبیر می‌شوند، اما نگاهی دقیق‌تر نشان می‌دهد که ممکن است در واقع پناهگاه‌های باستانی بوده باشند.

\begin{figure}[b]
\begin{center}
   \includegraphics[width=1\linewidth]{ww19.jpg}
\end{center}
   \caption{نیوبرینج، ایرلند - برای مقیاس به بازدیدکنندگان در ورودی توجه کنید.}
\label{fig:8}
\label{fig:onecol}
\end{figure}

یکی از نمونه‌های برجسته نیوبرینج (شکل \ref{fig:8})، بنای اصلی در مجموعه برو نا بویینه، مجموعه‌ای از ساختارهای باستانی شامل مواردی است که به عنوان مقبره‌های راهرو نامیده می‌شوند. این مقبره‌ها شامل یک یا چند اتاقک دفن شده پوشیده از خاک یا سنگ هستند و دارای راهروی باریکی هستند که از سنگ‌های بزرگ ساخته شده است \cite{70}. این یک نمونه از مهندسی گسترده یک ساختار محافظت شده پیچیده است که در طی نسل‌های متعدد ساخته شده است، به این منظور که تعداد کمی از مردم را دفن کند، که حتی زنده نبودند زمانی که ساخت مقبره شروع شد. زمانی که در سال ۱۶۹۹ توسط یک مالک محلی دوباره کشف شد، در خاک دفن شده بود.

نگاهی سرسری به ساختار نشان می‌دهد که تلاش‌های بسیار زیادی برای ساخت آن صرف شده است - نیوبرینج شامل حدود ۲۰۰,۰۰۰ تن مواد است. درون آن، \textit{"...یک راهروی دارای اتاقک است که می‌شود از ورودی در سمت جنوب‌شرقی بنای دسترسی پیدا کرد. راهرو به طول ۱۹ متر (۶۰ فوت)، یا حدود یک سوم راه به مرکز ساختاری کشیده می‌شود. در انتهای راهرو سه اتاقک کوچک از اتاقک مرکزی بزرگتر با سقف طاقی بلند منشعب شده‌است... دیوارهای این راهرو از تخته‌سنگ‌های بزرگ به نام ارتوستات ساخته شده‌اند، بیست و دو تا از آنها در سمت غربی و بیست و یک تا در سمت شرقی قرار دارند. ارتفاع متوسط آنها ۱.۵ متر است"} \cite{70}. همچنین جزئیات پیچیده مهندسی ضدآب‌دهی وجود دارد. به عنوان مثال، در سقف، \textit{"فواصل بین سنگ‌های سقف با مخلوطی از خاک سوخته و ماسه دریایی به منظور ضدآب کردن مهروموم شدند و از این مخلوط دو تاریخ رادیوکربن با مرکزیت در حدود سال ۲۵۰۰ قبل از میلاد برای ساختار مقبره به دست آمد"} \cite{71}. علاوه بر این، افزایش ارتفاع منجر به اتاقک داخلی ممکن است برای اهداف مشابهی ایجاد شده باشد: \textit{"از آنجا که کف راهرو و اتاقک مقبره از شیب زمین روی تپه‌ای که بنا روی آن ساخته شده پیروی می‌کند، بین سطح کف ورودی و داخل اتاقک تفاوتی نزدیک به ۲ متر وجود دارد"} \cite{71}.

\begin{figure}[b]
\begin{center}
   \includegraphics[width=1\linewidth]{dolmen.jpg}
\end{center}
   \caption{دولمن د سوتو، اسپانیا \cite{53}.}
\label{fig:9}
\label{fig:onecol}
\end{figure}

نبود بقایای انسانی در داخل نیز نکته‌ای کنجکاوشده است. کاوش‌ها نشان دادند که تکه‌های استخوان سوخته و نسوخته نمایانگر تعداد کمی از انسان‌ها، در اطراف راهرو پخش شده‌اند. ساخت نیوبرینج بر اساس تاریخ‌های رادیوکربن مواد داخلی حداقل چندین نسل به طول انجامیده است. چرا یک جامعه باستانی این همه تلاش برای ساختن یک مقبره عظیم و مهندسی‌شده با دقت صرف کند فقط برای این که استخوان‌های تعدادی از متوفیان را در راهرو پراکنده کند؟ بسیار منطقی‌تر است که این ساختارهای مگالیتی باستانی و دقیقاً ضدآب‌شده به عنوان پناهگاه‌های انسانی برای محافظت از مردم در برابر فاجعه‌های بازگشتی زمین ساخته شده باشند.

در اوئلبا، جنوب اسپانیا، یک نمونه مشابه دولمن د سوتو (شکل \ref{fig:9}) است، یکی از حدود ۲۰۰ چنین سایت در منطقه \cite{72,32}. این یک سازه ساده و مهندسی‌شده بالا است که با استفاده از سنگ‌های مگالیتی ساخته شده و قطر آن ۷۵ متر است. به طور گزارش شده، فقط هشت بدن پس از حفاری یافت شده است که همگی در وضعیتی جنینی دفن شده‌اند.

\section{ذکر ناهنجاری‌های قابل توجه}

در این بخش، به طور مختصر برخی از ناهنجاری‌های قابل توجه‌تر را ذکر می‌کنم که همگی به خوبی با یک فاجعه مانند ECDO توضیح داده شده‌اند.

\subsection{ناهنجاری‌های زیستی}

\begin{figure}[b]
\begin{center}
   \includegraphics[width=1\linewidth]{bottleneck.jpg}
\end{center}
   \caption{یک گردنه تکاملی که نشان دهنده کاهش ۹۵\٪ی نرها حدود ۶۰۰۰ سال پیش است \cite{62}.}
\label{fig:10}
\label{fig:onecol}
\end{figure}

برخی از ناهنجاری‌های زیستی قابل توجه گردنه‌های ژنتیکی و فسیل‌های نهنگ درون‌زمینی هستند. زن et al. (2018) تعداد ۱۲۵ توالی کروموزوم Y از انسان‌های مدرن را مدل سازی کردند و بر اساس شباهت‌ها و جهش‌ها در DNA، یک گردنه کاهش جمعیتی ۹۵\٪ی در جمعیت مردان را حدود ۵۰۰۰ تا ۷۰۰۰ سال پیش شناسایی کردند (شکل \ref{fig:10}) \cite{62}. فسیل‌های نهنگ در چندین صد متری بالای سطح دریا، در سوئدنبورگ، میشیگان، ورمونت، کانادا، شیلی و مصر یافت شده‌اند \cite{63,64,65,66}. این نهنگ‌ها در وضعیت‌های مختلفی یافت شده‌اند: کاملاً حفظ شده، در باتلاق‌هایی که بالای رسوبات یخبندان قرار دارند، یا زیر رسوب دفن شده. تعداد نمونه‌ها در این مکان‌ها از چند عدد تا بیش از صد عدد متغیر است. نهنگ‌ها جانداران اعماق دریا هستند و به ندرت به ساحل‌ها نزدیک می‌شوند. چگونه این نهنگ‌ها در چنین ارتفاعات بالایی، اغلب در فواصل بسیار دور از خشکی‌ها، به پایان رسیدند؟

انقراض‌های دسته جمعی متعددی در گذشته زمین رخ داده‌اند، که به طور کامل مورد مطالعه قرار گرفته‌اند، شامل "پنج بزرگ" رویدادهای فَنروزوئيك: انقراض اواخر اوردویسی (LOME)، انقراض اواخر دونین (LDME)، انقراض پایان-پرمین (EPME)، انقراض پایان-تریسیک (ETME) و انقراض پایان-کرتاسه (ECME) \cite{88,89}. جالب است که چندین از این انقراض‌ها در همان دوره‌های تاریخی که بسیاری از لایه‌های گرند کانیون رخ داده‌اند، به عنوان طبقه‌بندی می‌شوند؛ یعنی، لایه‌های پرمین و دونین.

\subsection{ناهنجاری‌های فیزیکی}

\begin{figure}[b]
\begin{center}
   \includegraphics[width=1\linewidth]{columbia.jpg}
\end{center}
   \caption{ریپل‌های جریان عظیم در دریاچه یخچالی کلمبیا، ایالت واشینگتن \cite{80}.}
\label{fig:11}
\label{fig:onecol}
\end{figure}

مناظر بسیاری غیر از گرند کانیون وجود دارند که احتمالاً از طریق نیروهای کاتاکلیسمیک شکل گرفته‌اند. شواهدی از جریان بزرگ قاره‌ای آب را می‌توان در ریپل‌های جریان عظیم در سراسر جهان یافت. یکی از این نمونه‌ها، کانال‌های اسکابلندز در شمال غربی آرام است. در اینجا، نه تنها مناظر رسوبی و صخره‌های غیرقانونی را می‌بینیم بلکه بیش از صد توالی از ریپل‌های وسیع تشکیل شده از جریان‌های مگا جریان نیز وجود دارد \cite{78,79}. این‌ها نسخه‌های بزرگتر ریپل‌هایی هستند که در بسترهای شنی جریان‌ها شکل می‌گیرند. این‌ها را می‌توان در سراسر جهان در فرانسه، آرژانتین، روسیه و آمریکای شمالی یافت \cite{81}. شکل \ref{fig:11} برخی از این ریپل‌ها را در ایالت واشینگتن در ایالات متحده نشان می‌دهد \cite{80}.

\begin{figure}[b]
\begin{center}
% \fbox{\rule{0pt}{2in} \rule{0.9\linewidth}{0pt}}
   \includegraphics[width=1\linewidth]{zhangjiajie.jpg}
\end{center}
   \caption{ستون‌های سنگی عظیم در جنگل ملی ژانگ‌ژیاجی، جنوب چین.}
\label{fig:12}
\label{fig:onecol}
\end{figure}

\begin{figure}[b]
\begin{center}
   \includegraphics[width=1\linewidth]{hoy.jpg}
\end{center}
   \caption{پیلار دریایی اولد من آف هوی، اسکاتلند \cite{83}.}
\label{fig:13}
\label{fig:onecol}
\end{figure}

ساختارهای فرسایش داخلی نیز به خوبی توسط چرخش زمین مشابه ECDO توضیح داده می‌شوند. جنوب چین نمونه‌ی خوبی از مناظر کارست وسیع است که از طریق فرسایش آبی تشکیل شده‌اند \cite{82}. این مناظر شامل کارست برج، کارست قله، کارست مخروطی، پل‌های طبیعی، دره‌ها، سیستم‌های بزرگ غار و فروچاله‌ها می‌شوند. یکی از چشمگیرترین این‌ها پارک ملی ژانگجیاجی است که دارای ستون‌های ماسه‌سنگ کوارتز وسیع است (شکل \ref{fig:12}) \cite{84}. این ستون‌ها در ارتفاع متوسط بیش از ۱,۰۰۰ متر قرار دارند و تعدادشان بیش از ۳,۱۰۰ است. بیش از ۱,۰۰۰ تا از آن‌ها بیش از ۱۲۰ متر ارتفاع دارند و ۴۵ تا از آن‌ها به بیش از ۳۰۰ متر می‌رسند \cite{85}. این ستون‌ها به ستون‌های فرسایش دریایی شباهت دارند (شکل \ref{fig:13}) که ستون‌های سنگی ساحلی هستند که توسط فروریختن ماده‌ی اطراف به دلیل امواج اقیانوسی تشکیل شده‌اند. مناظر فرسایش مشابهی را می‌توان در مخروط‌های سنگی اورگوپ، ترکیه، و همچنین سیوداد انکانتادا، اسپانیا، یافت، که هر دو بیش از ۱,۰۰۰ متر بالاتر از سطح دریا قرار دارند. همه‌ی این مکان‌ها ترکیبی از نمک و فسیل‌های دریایی اقیانوسی در نزدیکی خود دارند که نشان‌دهنده‌ی هجمه‌های دریایی گذشته است \cite{15,86,87}. البته داستان‌های سیل \cite{3} به اقیانوس اشاره می‌کنند که بسیار بالاتر از ۱,۰۰۰ متر می‌رفت، و این توسط حضور آب شور و نمک‌زارهای وسیع در آند و هیمالیا که چند کیلومتر بالاتر از سطح دریا هستند تأیید می‌شود. به عنوان مثال، نمک‌زار اویونی در بولیوی به ۳۶۵۳ متر بالاتر از سطح دریا می‌رسد \cite{94}.

\subsection{رویدادهای تغییر سریع اقلیم}

ادبیات علمی مدرن به وجود رویدادهای تغییر سریع اقلیم جهانی در تاریخ اخیر زمین اذعان می‌کند. دو نمونه‌ی قابل توجه، رویدادهای ۴٫۲ کیلو سال و ۸٫۲ کیلو سال هستند که هر دو با کاهش جمعیت و اختلال در اسکان اجتماعی در مناطق جغرافیایی گسترده همزمانی دارند. این رویدادها به عنوان ناهنجاری‌هایی در رسوبات و هسته‌های یخی، فسیل‌های مرجان، مقادیر ایزوتوپ O18، سوابق گرده و استالاگمیت، و داده‌های سطح دریا حفظ می‌شوند. تغییرات اقلیمی استنباط شده شامل کاهش سریع دماهای جهانی، خشکی، اختلال در جریان معکوس عرضی اقیانوس اطلس و پیشروی یخبندان‌ها است \cite{90,91,92}. رویداد ۸٫۲ کیلو سال به خصوص همزمان با احتمال ورود شور آب دراماتیک دریای سیاه در حدود ۶۴۰۰ قبل از میلاد است \cite{93}.

\subsection{ناهنجاری‌های باستان‌شناسی}

شواهد باستان‌شناسی از برخی شهرهای باستانی نشان می‌دهد که چندین لایه شامل دفن و تخریب دارند که سوابق رخدادهای ویرانگر گذشته را ایجاد می‌کند. شهر باستانی اریحا یکی از این شهرهاست که در فلسطین امروزی واقع شده است. این شهر دارای لایه‌های تخریبی متعددی است که شامل فروپاشی سازه‌های سنگی و آتش‌سوزی شدید می‌شود \cite{96,97}. زمان‌نگاری ثبت شده در لایه‌های آن حدود ۹۰۰۰ قبل از میلاد تا ۲۰۰۰ قبل از میلاد را نشان می‌دهد. برجی که به نظر می‌رسد بریده و در رسوب دفن شده است به ویژه مورد توجه است که به حدود ۷۴۰۰ قبل از میلاد برمی‌گردد (شکل \ref{fig:14}) \cite{95}. چتال هویوک \cite{99}، گرامالوت \cite{98}، و کاخ مینویی کنوسوس در کرت \cite{100,101} همه نمونه‌های مشابهی از سایت‌های باستان‌شناسی هستند که حاوی لایه‌های متعدد هستند و اغلب شواهدی از تخریب دارند.

\begin{figure}[t]
\begin{center}
   \includegraphics[width=1\linewidth]{jericho.jpg}
\end{center}
   \caption{بازسازی باستان‌شناسانه دفن برج اریحا در حدود ۷۴۰۰ قبل از میلاد \cite{95}.}
\label{fig:14}
\label{fig:onecol}
\end{figure}

قطعه‌ای دیگر از شواهد برای فاجعه‌های عمده که تمدن انسانی را مختل کرده‌اند، تصویر نامپا است، یک عروسک سفالی که در زیر حدود ۱۰۰ متر گدازه در آیداهو پیدا شده است \cite{102,103}. جریان گدازه‌ای که زیر آن پیکره یافت شد، تخمین زده می‌شود که در دورهٔ ترشیاری پایانی یا چهارگانهٔ اولیه گذاشته شده باشد و به نظر می‌رسد ۲ میلیون ساله باشد. با این حال، جریان گدازه در منطقه به نظر می‌رسد نسبتاً تازه باشد. چنین یافته‌هایی نه تنها به فجایعی که تمدن را از بین برده‌اند اشاره می‌کنند، بلکه کرونولوژی‌های دوخت زمانی مدرن را نیز به سوال می‌کشند.

\section{دربارهٔ روش‌های سن‌سنجی مدرن}

دلیل قابل توجهی برای شک و تردید به زمان‌بندی‌های مدرن وجود دارد که سن‌های بسیار طولانی به میزان میلیون‌ها یا حتی تا صدها میلیون سال را به مواد فیزیکی مختلف اختصاص می‌دهند.

داستان متعارف بیان می‌کند که آنچه به‌اصطلاح «سوخت فسیلی» نامیده می‌شود، مانند زغال‌سنگ، نفت و گاز طبیعی صدها میلیون سال قدمت دارند \cite{104}. با این حال، تاریخ‌نگاری کربنی واقعی از نفت در خلیج مکزیک سنی در حدود ۱۳۰۰۰ سال برای نفت نشان داد \cite{105}. کربن-۱۴ دارای نیمه‌عمر بسیار کوتاهی (۵۷۳۰ سال) است که انتظار می‌رود پس از چند صد هزار سال به‌طور کامل تجزیه شود. با این حال، در زغال‌سنگ و فسیل‌هایی که گفته می‌شود هزار برابر قدیمی‌تر هستند، یافت شده است \cite{106}. در واقع، زغال‌سنگ مصنوعی در شرایط کنترل شده در آزمایشگاه، عمدتاً در گرمای بالا، تنها در ۲ تا ۸ ماه تولید شده است \cite{107}.

روش‌های تاریخ‌نگاری رادیوایزوتوپی غیر از تاریخ‌نگاری کربنی نیز ممکن است دقیق نباشند. گروه تحقیقاتی Answers in Genesis تناقضاتی در تاریخ‌های به‌دست‌آمده از این روش‌ها یافته‌اند که صحت آن‌ها را مورد تردید قرار می‌دهد \cite{108}. بافت نرم حاوی سلول‌های خونی، عروق و کلاژن حتی در بقایای دایناسورهایی که گفته می‌شود صد میلیون سال قدمت دارند نیز یافت شده است \cite{109,110}. بر اساس آنچه می‌دانیم، ممکن است سن‌های پذیرفته‌شده متعارف از زمان‌بندی زمین‌شناسی زمین و مواد فیزیکی مانند سنگ‌ها و سوخت‌های فسیلی با اختلافات بسیاری از درجات نادرست باشند.

\section{نتیجه‌گیری}

در این مقاله، من به متقاعدکننده‌ترین ناهنجاری‌ها پرداختم که منشأ فاجعه‌آمیز را پیشنهاد می‌دهند و بهترین توضیح را با مدل ECDO یک چرخش زمین می‌دهند. در حالی که متنوع‌اند، مجموعه ارائه‌شده کامل نیست - ناهنجاری‌های بیشتری گردآوری شده و در مخزن گیت‌هاب تحقیقاتی من به‌صورت عمومی در دسترس است \cite{2}.

\section{سپاسگزاری‌ها}

از «Ethical Skeptic»، نویسنده اصلی پایان‌نامه ECDO، برای تکمیل پایان‌نامه بینش‌آمیز و بنیادین خود و به اشتراک‌گذاری آن با جهان سپاس‌گزارم. پایان‌نامه سه‌گانه او \cite{1} به عنوان کار مرجع برای نظریه جدایی هسته-ماده داغ نوسان ژان‌بیکوف باقی می‌ماند و اطلاعات بسیار بیشتری درباره موضوع دارد نسبت به آنچه من در اینجا به اختصار خلاصه کرده‌ام.

و البته، از غول‌هایی که بر شانه‌های آنها ایستاده‌ایم سپاس‌گزارم؛ آنهایی که تمامی تحقیقات و پژوهش‌هایی را انجام دادند که این کار را ممکن کرد و تلاش کردند تا به بشریت روشنایی بخشند.

{\small
\bibliographystyle{ieee}
\bibliography{egbib}
}

\end{document}

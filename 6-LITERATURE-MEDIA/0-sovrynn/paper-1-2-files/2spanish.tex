\documentclass[10pt,twocolumn,letterpaper]{article}

% My own stuff
\usepackage{booktabs}
% \usepackage{caption}
% \captionsetup[table]{skip=8pt}   % Only affects tables
\usepackage{stfloats}  % Add this to the preamble

\usepackage{cvpr}
\usepackage{times}
\usepackage{epsfig}
\usepackage{graphicx}
\usepackage{amsmath}
\usepackage{amssymb}

% Include other packages here, before hyperref.

% If you comment hyperref and then uncomment it, you should delete
% egpaper.aux before re-running latex.  (Or just hit 'q' on the first latex
% run, let it finish, and you should be clear).
\usepackage[breaklinks=true,bookmarks=false]{hyperref}

\cvprfinalcopy % *** Uncomment this line for the final submission

\def\cvprPaperID{****} % *** Enter the CVPR Paper ID here
\def\httilde{\mbox{\tt\raisebox{-.5ex}{\symbol{126}}}}

% Pages are numbered in submission mode, and unnumbered in camera-ready
%\ifcvprfinal\pagestyle{empty}\fi
\setcounter{page}{1}
\begin{document}

%%%%%%%%% TITLE
\title{Parte 2/2 del Resumen Basado en Datos del ECDO: Una Investigación de Anomalías Científicas e Históricas Mejor Explicadas por un "Giro de la Tierra" ECDO}

\author{Junho\\
Sitio web: \href{https://sovrynn.github.io}{sovrynn.github.io}\\
Repositorio de Investigación ECDO: \href{https://github.com/sovrynn/ecdo}{github.com/sovrynn/ecdo}\\
{\tt\small junhobtc@proton.me}
}

\maketitle
%\thispagestyle{empty}

%%%%%%%%% ABSTRACT
\begin{abstract}
En mayo de 2024, un autor pseudónimo en línea llamado “El Escéptico Ético” \cite{0} publicó una teoría revolucionaria llamada Desacoplamiento Exotérmico Núcleo-Manto Oscilación Dzhanibekov (ECDO) \cite{1}. Esta teoría no solo propone que la Tierra ha experimentado anteriormente cambios catastróficos repentinos en el eje de rotación, creando una inundación mundial masiva al hacer que los océanos derramen sobre los continentes debido a la inercia rotacional, sino que también propone un proceso geofísico causal explicativo junto con datos que sugieren que otro giro similar podría ser inminente. Si bien tales predicciones cataclísmicas de inundaciones y el fin del mundo no son nuevas, la teoría del ECDO es singularmente convincente debido a su enfoque científico, moderno, multidisciplinario y basado en datos.

Este artículo de investigación constituye la segunda parte de un resumen condensado de dos partes de 6 meses de investigación independiente \cite{2,20} sobre la teoría del ECDO, enfocándose específicamente en las anomalías científicas e históricas que son mejor explicadas por un catastrófico "giro de la Tierra" ECDO.

\end{abstract}

%%%%%%%%% BODY TEXT

\section{Introducción}

La geología y la historia modernistas uniformitarias afirman que los principales paisajes geológicos como el Gran Cañón se formaron durante millones de años \cite{143}; que la sal existe en el Valle de la Muerte (California) porque solía estar bajo el océano hace cientos de millones de años \cite{144}; que nuestros ancestros de hace 150 generaciones pasaron toda su vida construyendo tumbas gigantes \cite{29,70}; y que los llamados "combustibles fósiles" tienen cientos de millones de años \cite{104}. Quizás lo más fascinante es que se cree que los humanos tienen 300,000 años \cite{145}, sin embargo, la historia y la civilización registradas solo datan de hace unos 5,000 años, el equivalente a 150 generaciones humanas.

Como veremos, tales anomalías se explican mejor por fuerzas geológicas catastróficas.

\section{Mamut Congelado Rápidamente Enterrado en Barro}

\begin{figure}[t]
\begin{center}
   \includegraphics[width=1\linewidth]{jarkov-mammoth.jpg}
\end{center}
   \caption{El Mamut de Jarkov, un mamut siberiano perfectamente conservado de 20,000 años encontrado en barro congelado \cite{51}.}
\label{fig:1}
\label{fig:onecol}
\end{figure}

Una de esas categorías de anomalías son los mamuts perfectamente conservados y congelados encontrados comúnmente en las regiones del Ártico (Figura \ref{fig:1}). El mamut de Beresovka, descubierto en Siberia enterrado en grava limosa, estaba tan perfectamente conservado que su carne seguía siendo comestible miles de años después de su muerte. También tenía alimento vegetal en su boca y estómago, desconcertando a los científicos sobre cómo pudo haber sido congelado tan rápidamente si estaba pastando en plantas con flores justo antes de su muerte \cite{17}. Según informes, \textit{"En 1901 se causó una sensación por el descubrimiento de un cadáver completo de mamut cerca del río Berezovka, ya que este animal parecía haber muerto de frío a mediados de verano. El contenido de su estómago estaba bien conservado e incluía ranúnculos y frijoles silvestres en flor: esto significaba que debían haber sido tragados hacia finales de julio o principios de agosto. La criatura había muerto tan repentinamente que aún sostenía en sus mandíbulas un bocado de hierbas y flores. Claramente había sido atrapado por una fuerza tremenda y lanzado varios kilómetros desde su lugar de pastoreo. La pelvis y una pierna estaban fracturadas: el enorme animal había sido derribado de rodillas y luego congelado hasta morir, en lo que normalmente es el momento más caluroso del año"} \cite{18}. Además, \textit{"[Científicos rusos] registraron que incluso el revestimiento más interno del estómago del animal tenía una estructura fibrosa perfectamente conservada, lo que indica que su calor corporal había sido eliminado por algún proceso súper prodigioso de la naturaleza. Sanderson, prestando especial atención a este punto, llevó el problema al Instituto Americano de Alimentos Congelados: ¿Qué se necesita para congelar un mamut entero de tal manera que el contenido de humedad incluso de las partes más internas de su cuerpo, incluso del revestimiento interno de su estómago, no tenga tiempo suficiente para formar cristales lo suficientemente grandes como para destruir la estructura fibrosa de la carne?... Algunas semanas después, el Instituto regresó a Sanderson con la respuesta: Es absolutamente imposible. Con todos nuestros conocimientos científicos y de ingeniería, no existe absolutamente ninguna forma conocida de eliminar el calor corporal de un cadáver tan grande como un mamut lo suficientemente rápido como para congelarlo sin que se formen grandes cristales de humedad en la carne. Además, después de agotar las técnicas científicas y de ingeniería, buscaron en la naturaleza y concluyeron que no existe ningún proceso conocido en la naturaleza que pudiera lograr tal hazaña"} \cite{19}.

\section{El Gran Cañón}

El Gran Cañón, parte de la Gran Cuenca en el suroeste de América del Norte, es otro fenómeno natural que sugiere orígenes catastróficos (Figura \ref{fig:2}). Para empezar, las capas de arenisca y caliza sedimentarias que componen el Gran Cañón abarcan áreas enormes de hasta 2.4 millones de km$^2$ \cite{21}. La Figura \ref{fig:3} muestra la extensión de la capa de Arenisca Coconino a través del oeste de los Estados Unidos. Dichas capas horizontales masivas de material uniforme solo podrían haber sido depositadas de una vez.

\begin{figure}[t]
\begin{center}
   \includegraphics[width=1\linewidth]{grand-canyon.jpg}
\end{center}
   \caption{El Gran Cañón, en Arizona, EE.UU \cite{49}.}
\label{fig:2}
\label{fig:onecol}
\end{figure}

\begin{figure}[t]
\begin{center}
   \includegraphics[width=1\linewidth]{coconino.jpg}
\end{center}
   \caption{Tamaño de la capa de Arenisca en Coconino en el oeste de los Estados Unidos \cite{21}.}
\label{fig:3}
\label{fig:onecol}
\end{figure}

Un vistazo más de cerca al Gran Cañón nos dice que la deposición de estas extensas capas de sedimentos también ocurrió conjuntamente con fuerzas tectónicas significativas. Para entender esto, debemos observar de cerca a ciertas áreas en el cañón donde las capas de sedimento han sido plegadas y expuestas. Investigadores de Respuestas en Génesis \cite{42} examinaron microscópicamente muestras de rocas de algunos de estos pliegues, como el Monument Fold, y basándose en la falta de características que deberían haber estado presentes si los pliegues se hubieran formado durante largos períodos bajo calor y presión, concluyeron que las capas de sedimento fueron plegadas por fuerzas tectónicas mientras aún estaban blandas, es decir, poco después de su deposición \cite{43}.

\begin{figure*}
\begin{center}
\includegraphics[width=1\textwidth]{Grand_Staircase-big.jpg}
\end{center}
   \caption{Las capas de sedimento que conforman el Gran Cañón (lado derecho de la imagen) se extienden directamente hacia el norte hasta Cedar Breaks, Utah (lado izquierdo de la imagen), donde todas se doblan hacia arriba \cite{50}.}
\label{fig:4}
\end{figure*}

Al ampliar la vista, encontramos que las capas que conforman el Gran Cañón no solo han sido plegadas dentro del cañón. Las capas han sido plegadas hacia el este en la Monoclina Kaibab Este \cite{46}, pero también hacia el norte en Cedar Breaks, Utah (Figura \ref{fig:4}). Esto sugiere que estas capas pueden haber sido plegadas todas juntas después de haber sido depositadas una encima de la otra en rápida sucesión. Como referencia, las capas horizontales del Gran Cañón tienen aproximadamente 1700 metros de espesor. La magnitud del proceso geológico necesario para depositar capas de sedimento de un kilómetro y medio de grosor es enorme.

La formación actual del Gran Cañón es otro tema de debate en la geología moderna. La geología uniformista propone que el Gran Cañón fue tallado por el río Colorado durante millones de años \cite{47}. Sin embargo, el equipo de investigación de Respuestas en Génesis cree que el Gran Cañón probablemente se formó en cuestión de semanas debido a la erosión de un canal de desbordamiento cuando un lago antiguo rompió sus límites, lo que eliminó grandes cantidades de sedimento al formar el cañón. Hay evidencia de un lago de gran altitud al este del Gran Cañón en depósitos de sedimentos lacustres y fósiles marinos. Comparar el Gran Cañón con otros ejemplos a gran escala de erosión por desbordamiento, como el Cañón de Afton y el Monte Santa Helena, revela una topografía similar, y muestra que los grandes cañones pueden ser creados rápidamente a través de grandes cantidades de agua corriente \cite{48}.

Considerando la magnitud de los procesos geológicos necesarios para depositar sedimento sobre áreas tan vastas de tierra, la concurrencia de fuerzas tectónicas masivas ocurriendo poco después de que las capas de sedimento fueran depositadas, y el tamaño minúsculo del río Colorado en comparación con la enorme escala del Gran Cañón, parece que puede no haber habido nada gradual en su formación.

\section{Ciudad Subterránea de Derinkuyu}

Aparte de las pirámides, un gran ejemplo de ingeniería antigua es la ciudad subterránea de Derinkuyu (Figura \ref{fig:5}), ubicada en Capadocia, Turquía. Es la más grande de más de 200 refugios subterráneos en la región \cite{54}. Se estima que esta ciudad subterránea albergó hasta 20,000 personas y abarca 18 pisos, alcanzando profundidades de 85 metros. Aunque su edad no es segura, se estima que tiene al menos 2800 años. La ciudad fue tallada en roca volcánica blanda \cite{52, 53}.

\begin{figure}[b]
\begin{center}
   \includegraphics[width=1\linewidth]{derinkuyu.jpeg}
\end{center}
   \caption{Diagrama de la ciudad subterránea de Derinkuyu \cite{56}.}
\label{fig:5}
\label{fig:onecol}
\end{figure}
La razón por la que Derinkuyu es interesante es porque no está claro por qué alguna comunidad decidiría construir una ciudad entera bajo tierra. Para crear espacio habitable bajo tierra, cada cavidad debe ser tallada en la roca. Las formas y texturas toscas de los túneles subterráneos dejan claro que fueron tallados con mano de obra, en lugar de con herramientas eléctricas, lo cual habría sido órdenes de magnitud más difícil que construir refugios sobre la superficie. De hecho, no es aparente por qué cualquier ser humano querría vivir permanentemente bajo tierra durante los confines de su vida terrenal, cuando la agricultura, la luz del sol, la naturaleza y la exploración están disponibles solo sobre la superficie. La "historia" convencional propone que Derinkuyu fue creada por cristianos que necesitaban un lugar apartado para practicar su religión \cite{53}. Pero el sentido común concluiría que la forma más directa de lidiar con los enemigos es "luchar o huir", no "tallar una ciudad subterránea en la roca".

La escala, profundidad y el esmero en el diseño de la ciudad subterránea dejan claro que no fue diseñada como una estructura defensiva militar temporal para luchar mejor contra los invasores en tiempos de estrés, sino más bien, como un refugio a largo plazo para protegerse contra fuerzas fatales en la superficie. Derinkuyu estaba equipada no solo con dormitorios básicos, cocinas y baños, sino también con establos para animales, tanques de agua, despensas, prensas de vino y aceite, escuelas, capillas, tumbas y enormes pozos de ventilación (Figura \ref{fig:6}). ¿Por qué un refugio militar requeriría una prensa de vino y necesitaría ser excavado 85 metros de profundidad con tanta complejidad?

La explicación más plausible para la creación de Derinkuyu habría sido una necesidad apremiante de preparar un refugio autosuficiente y a largo plazo para protegerse contra fuerzas geofísicas catastróficas en la superficie de la Tierra.

\begin{figure}[t]
\begin{center}
   \includegraphics[width=1\linewidth]{derinkuyu-air.jpg}
\end{center}
   \caption{Un pozo de ventilación profundo en Derinkuyu \cite{53}.}
\label{fig:6}
\label{fig:onecol}
\end{figure}

\section{Acumulaciones de Biomasa}

Las mezclas de biomasa de varios tipos de animales y plantas, a menudo encontradas fosilizadas en capas de sedimentos, son otra anomalía desconcertante. En "Reliquoæ Diluvianæ", el Rev. William Buckland detalla hallazgos de numerosas especies de fauna que no tenían una razón explicable para encontrarse juntas, dispersas por Gran Bretaña y Europa, enterradas en capas de 'diluvio' sedimentario \cite{58}. Tales mezclas de restos de animales también se encontraron en la Cueva de Skjonghelleren en la isla de Valdroy, Noruega. En esta cueva, se encontraron más de 7,000 huesos de mamíferos, aves y peces mezclados a través de múltiples capas de sedimentos \cite{59}. Otro ejemplo es San Ciro, la "Cueva de los Gigantes", en Italia. En esta cueva, se encontraron varias toneladas de huesos de mamíferos, principalmente de hipopótamos, en un estado tan fresco que fueron cortados en adornos y enviados para la fabricación de negro de humo. Los huesos de los diferentes animales se mezclaron entre sí, rotos, destrozados y dispersos en fragmentos \cite{60,61}. En Mendes, Egipto, se encontró una mezcla de diversas especies de huesos de animales mezclados con arcilla vitrificada (vidriosa) \cite{57}. Tales hallazgos pueden parecer desconcertantes, pero se explican fácilmente por inundaciones masivas que depositaron mezclas de animales muertos en capas de sedimentos, depositando animales en cuevas o enterrándolos vivos, y en el caso de biomasa vitrificada en Egipto, descargas eléctricas masivas post-inundación por un desplazamiento del núcleo-manto. La Figura \ref{fig:7} representa una exposición típica de 'lodo' de biomasa de Alaska \cite{56}.

\begin{figure}[t]
\begin{center}
   \includegraphics[width=1\linewidth]{muck-crop.jpeg}
\end{center}
   \caption{'Lodo' de Alaska, compuesto de fragmentos dispersos caóticamente de árboles, plantas y animales en limo congelado y hielo \cite{146}.}
\label{fig:7}
\label{fig:onecol}
\end{figure}

\section{Bunkers Antiguos}
Nuestros antepasados dejaron muchas estructuras antiguas altamente ingenierizadas donde se han encontrado restos humanos. Estos generalmente son interpretados como tumbas elaboradas, pero una mirada más cercana sugiere que en realidad podrían haber sido antiguos refugios.

\begin{figure}[b]
\begin{center}
   \includegraphics[width=1\linewidth]{ww19.jpg}
\end{center}
   \caption{Newgrange, Irlanda - vean a los visitantes en la entrada para la escala.}
\label{fig:8}
\label{fig:onecol}
\end{figure}

Un excelente ejemplo es Newgrange (Figura \ref{fig:8}), el monumento principal en el complejo Brú na Bóinne, una colección de estructuras antiguas incluyendo las llamadas tumbas de corredor. Estas tumbas consisten en una o más cámaras funerarias cubiertas de tierra o piedra y tienen un estrecho pasillo de acceso hecho de grandes piedras \cite{70}. Es un ejemplo de ingeniería extensiva de una estructura protegida compleja, construida durante múltiples generaciones, supuestamente para enterrar a un puñado de personas que ni siquiera estaban vivas cuando comenzó la construcción de la tumba. Cuando fue redescubierta por un terrateniente local en 1699, estaba cubierta de tierra.

Una mirada superficial a la estructura revela el inmenso esfuerzo empleado en construirla - Newgrange consta de unas 200,000 toneladas de material. En su interior, \textit{"...hay un pasaje con cámaras, al que se puede acceder por una entrada en el lado sureste del monumento. El pasaje se extiende por 19 metros (60 pies), o aproximadamente un tercio del camino hacia el centro de la estructura. Al final del pasaje hay tres pequeñas cámaras fuera de una cámara central más grande con un techo abovedado con cornisas altas... Las paredes de este pasaje están formadas por grandes losas de piedra llamadas ortostatos, veintidós de las cuales están en el lado occidental y veintiuna en el lado oriental. Promedian 1½ metros de altura”} \cite{70}. También hay detalles intrincados de ingeniería impermeable. Por ejemplo, en el techo, \textit{“Las intersticios del techo fueron rellenados con una mezcla de suelo quemado y arena de mar para hacerlos impermeables y de esta mezcla se obtuvieron dos fechas de radiocarbono centradas en 2500 a.C. para la estructura de la tumba"} \cite{71}. Además, un aumento de nivel que conduce a la cámara interior puede haber sido implementado para propósitos similares: \textit{"Dado que el suelo del pasaje y de la cámara de la tumba sigue la elevación del terreno de la colina sobre la cual está construido el monumento, hay una diferencia de casi 2 metros en el nivel del suelo entre la entrada y el interior de la cámara"} \cite{71}.

\begin{figure}[t]
\begin{center}
   \includegraphics[width=1\linewidth]{dolmen.jpg}
\end{center}
   \caption{El Dolmen de Soto, España \cite{53}.}
\label{fig:9}
\label{fig:onecol}
\end{figure}

La falta de restos humanos en el interior es también un punto curioso. Las excavaciones revelaron fragmentos de huesos quemados y no quemados representando a un puñado de personas, esparcidos por el pasaje. Se estima que la construcción de Newgrange tomó al menos varias generaciones basándose en fechas de carbono de los materiales en su interior. ¿Por qué una comunidad antigua gastaría tanto esfuerzo para construir una tumba masiva y altamente ingenierizada solo para esparcir los fragmentos óseos de unos pocos fallecidos en su pasadizo? Es mucho más plausible que estas estructuras megalíticas antiguas y cuidadosamente impermeabilizadas fueran construidas como refugios humanos para proteger a las personas durante los cataclismos recurrentes de la Tierra.

En Huelva, en el sur de España, un ejemplo similar es el Dolmen de Soto (Figura \ref{fig:9}), uno de unos 200 sitios similares en el área \cite{72,32}. Es una estructura aerodinámica y altamente ingenierizada construida utilizando piedras megalíticas y tiene un diámetro de 75 metros. Según se informa, solo se encontraron ocho cuerpos durante la excavación, todos enterrados en posición fetal.

\section{Menciones de Anomalías Notables}

En esta sección, menciono brevemente algunas anomalías más notables, todas las cuales están bien explicadas por un cataclismo tipo ECDO.

\subsection{Anomalías Biológicas}
\begin{figure}[b]
\begin{center}
   \includegraphics[width=1\linewidth]{bottleneck.jpg}
\end{center}
   \caption{Un cuello de botella genético que representa una reducción del 95\% de los machos hace aproximadamente 6,000 años \cite{62}.}
\label{fig:10}
\label{fig:onecol}
\end{figure}

Algunas anomalías biológicas notables son los cuellos de botella genéticos y los fósiles de ballenas en el interior. Zeng et al. (2018) modelaron 125 secuencias del cromosoma Y de humanos modernos, y basados en las similitudes y mutaciones en el ADN, identificaron un cuello de botella de reducción de población del 95\% en la población masculina hace alrededor de 5,000 a 7,000 años (Figura \ref{fig:10}) \cite{62}. Se han encontrado fósiles de ballenas a cientos de metros sobre el nivel del mar, en Suecia, Michigan, Vermont, Canadá, Chile y Egipto \cite{63,64,65,66}. Estas ballenas se encontraron en estados variados: perfectamente conservadas, en pantanos sobre depósitos glaciares, o enterradas en sedimentos. El número de especímenes en estos sitios varía desde unos pocos hasta más de cien. Las ballenas son criaturas de aguas profundas y rara vez se aventuran cerca de las costas. ¿Cómo llegaron estas ballenas a estar a tales altitudes, a menudo a distancias extremas tierra adentro?

Se han producido numerosas extinciones masivas en el pasado de la Tierra, siendo las más estudiadas las "Cinco Grandes" del Fanerozoico: las extinciones masivas del Ordovícico tardío (LOME), Devónico tardío (LDME), final del Pérmico (EPME), final del Triásico (ETME) y final del Cretácico (ECME) \cite{88,89}. Curiosamente, varias de estas extinciones están clasificadas como ocurridas en los mismos períodos históricos que muchas de las capas del Gran Cañón, a saber, las capas Pérmica y Devónica.

\subsection{Anomalías Físicas}

\begin{figure}[b]
\begin{center}
   \includegraphics[width=1\linewidth]{columbia.jpg}
\end{center}
   \caption{Rizaduras de corriente masiva en el Lago Glacial Columbia, estado de Washington \cite{80}.}
\label{fig:11}
\label{fig:onecol}
\end{figure}

Hay muchos paisajes aparte del Gran Cañón que probablemente se formaron a través de fuerzas cataclísmicas. La evidencia de un flujo de agua continental masivo se puede encontrar en enormes rizaduras de corriente en todo el mundo. Un ejemplo de ello son las Channeled Scablands en el Noroeste del Pacífico. Aquí, no solo vemos paisajes de depósitos sedimentarios y bloques erráticos, sino también más de cien secuencias de grandes rizaduras formadas por mega flujos de corriente \cite{78,79}. Estas son versiones a mayor escala de rizaduras formadas en los lechos de arena de los arroyos. Se pueden encontrar en todo el mundo en Francia, Argentina, Rusia y América del Norte \cite{81}. La Figura \ref{fig:11} muestra algunas de estas rizaduras en el estado de Washington en los Estados Unidos \cite{80}.

\begin{figure}[t]
\begin{center}
% \fbox{\rule{0pt}{2in} \rule{0.9\linewidth}{0pt}}
   \includegraphics[width=1\linewidth]{zhangjiajie.jpg}
\end{center}
   \caption{Pilares de piedra masivos en el Bosque Nacional de Zhangjiajie, sur de China.}
\label{fig:12}
\label{fig:onecol}
\end{figure}

\begin{figure}[t]
\begin{center}
   \includegraphics[width=1\linewidth]{hoy.jpg}
\end{center}
   \caption{Pilar de mar Old Man of Hoy, Escocia \cite{83}.}
\label{fig:13}
\label{fig:onecol}
\end{figure}

Las estructuras de erosión en el interior también están bien explicadas por un vuelco de la Tierra similar al ECDO. El sur de China es un gran ejemplo de paisajes kársticos masivos, formados a través de la erosión del agua \cite{82}. Estos paisajes incluyen karst de torre, karst de pináculo, karst de cono, puentes naturales, gargantas, grandes sistemas de cuevas y dolinas. Uno de los más impresionantes de estos es el Bosque Nacional de Zhangjiajie, que contiene enormes pilares de arenisca de cuarzo (Figura \ref{fig:12}) \cite{84}. Estos pilares se alzan a una elevación promedio de más de 1,000 metros y suman más de 3,100. Más de 1,000 de ellos se elevan por encima de los 120 metros de altura, y 45 alcanzan más de 300 metros \cite{85}. Estos pilares se asemejan a los pilares de erosión marina (Figura \ref{fig:13}), que son pilares rocosos costeros formados por el colapso del material circundante debido a las olas del océano. Paisajes de erosión similares se pueden encontrar en los conos rocosos de Urgup, Turquía, así como en la Ciudad Encantada, España, que se encuentran ambos a más de 1,000 metros sobre el nivel del mar. Todas estas ubicaciones tienen alguna combinación de fósiles marinos oceánicos y sal cerca, lo que sugiere incursiones marinas pasadas \cite{15,86,87}. Por supuesto, las historias del diluvio \cite{3} mencionan que el océano alcanzó mucho más de 1,000 metros, y esto se verifica por la presencia de agua salada y vastas salinas en los Andes y el Himalaya, varios kilómetros sobre el nivel del mar. La salina de Uyuni en Bolivia, por ejemplo, alcanza los 3653 metros sobre el nivel del mar \cite{94}.

\subsection{Eventos de Cambio Climático Rápidos}

La literatura científica moderna reconoce la existencia de eventos de cambio climático global rápido en la historia reciente de la Tierra. Dos ejemplos notables son los eventos de 4.2 y 8.2 mil años, ambos coincidiendo con la reducción de la población y la interrupción del asentamiento social sobre grandes áreas geográficas. Estos eventos se conservan como anomalías en sedimentos y núcleos de hielo, corales fósiles, valores de isótopos O18, registros de polen y espeleotemas, y datos de nivel del mar. Los cambios climáticos inferidos incluyen una rápida caída en las temperaturas globales, aridificación, interrupción del corriente de inversión meridional del Atlántico, y avances glaciales \cite{90,91,92}. El evento de 8.2 mil años en particular es concurrente con una posible dramática inundación de agua salada del Mar Negro alrededor de 6400 a.C. \cite{93}.

\subsection{Anomalías Arqueológicas}

La evidencia arqueológica de algunas ciudades antiguas muestra múltiples capas que implican entierro y destrucción, creando registros de eventos cataclísmicos pasados. La antigua ciudad de Jericó es una de estas, ubicada en la actual Palestina. Contiene múltiples capas de destrucción, con colapso de estructuras de piedra y fuego intenso \cite{96,97}. La cronología registrada en sus capas data aproximadamente desde el 9000 a.C. hasta el 2000 a.C. Cabe destacar su torre, que parece haber sido cortada y enterrada en sedimento alrededor del 7400 a.C. (Figura \ref{fig:14}) \cite{95}. Çatal Höyük \cite{99}, Gramalote \cite{98}, y el palacio minoico de Cnossos en Creta \cite{100,101} son todos ejemplos similares de sitios arqueológicos que contienen múltiples capas, a menudo con evidencia de destrucción.

\begin{figure}[b]
\begin{center}
   \includegraphics[width=1\linewidth]{jericho.jpg}
\end{center}
   \caption{Reconstrucción arqueológica del entierro de la Torre de Jericó hacia el 7400 a.C. \cite{95}.}
\label{fig:14}
\label{fig:onecol}
\end{figure}

Otra prueba de grandes cataclismos que interrumpieron la civilización humana es la Imagen de Nampa, una muñeca de arcilla encontrada bajo aproximadamente 100 metros de lava en Idaho \cite{102,103}. Se estimó que el flujo de lava bajo el cual se encontró la figurilla fue depositado durante el período Terciario Tardío o Cuaternario temprano, suponiéndose que tiene 2 millones de años. Sin embargo, el flujo de lava en la región parece ser relativamente fresco. Tales hallazgos no solo apuntan a cataclismos importantes que destruyen civilizaciones, sino que también cuestionan las cronologías modernas de datación.

\section{En Relación con los Métodos Modernos de Datación}
Hay una razón significativa para ser escéptico de las cronologías modernas, que asignan edades extremadamente largas de millones, o incluso hasta cientos de millones de años a diversos materiales físicos.

La narrativa convencional afirma que los llamados "combustibles fósiles" tales como el carbón, el petróleo y el gas natural tienen cientos de millones de años \cite{104}. Sin embargo, una datación real de carbono del petróleo en el Golfo de México encontró una edad de aproximadamente 13,000 años para el petróleo \cite{105}. El Carbono-14 tiene una vida media tan corta (5,730 años) que se supone que se descompone completamente después de unos pocos cientos de miles de años. No obstante, se ha encontrado en carbón y fósiles que supuestamente son mil veces más antiguos \cite{106}. De hecho, se ha producido carbón artificial en un laboratorio bajo condiciones controladas, principalmente altas temperaturas, en solo 2-8 meses \cite{107}.

Los métodos de datación por radioisótopos distintos a la datación por carbono pueden tampoco ser precisos. El grupo de investigación Answers in Genesis encontró inconsistencias en fechas derivadas de tales métodos que ponen en cuestión su veracidad \cite{108}. Se han encontrado incluso tejidos blandos que contienen células sanguíneas, vasos y colágeno en restos de dinosaurios que supuestamente tienen cien millones de años \cite{109,110}. Basado en lo que sabemos, es posible que las edades convencionalmente aceptadas de la escala de tiempo geológico de la Tierra y materiales físicos como rocas y combustibles fósiles sean incorrectas por muchos órdenes de magnitud.

\section{Conclusión}

En este artículo, he cubierto las anomalías más convincentes que sugieren orígenes catastróficos y que se explican mejor por un vuelco de la Tierra ECDO. Aunque diverso, el conjunto presentado es incompleto - más anomalías han sido compiladas y están disponibles públicamente en mi repositorio de investigación en GitHub \cite{2}.

\section{Agradecimientos}

Gracias a Ethical Skeptic, el autor original de la tesis ECDO, por completar su tesis perspicaz e innovadora y compartirla con el mundo. Su tesis en tres partes \cite{1} sigue siendo el trabajo autoritativo para la teoría de Oscilación del Desacoplamiento Núcleo-Manto Exotérmico Dzhanibekov (ECDO), y contiene mucha más información sobre el tema de la que he resumido brevemente aquí.

Muchas gracias a PS por revisar este documento.

Y por supuesto, gracias a los gigantes en cuyos hombros nos apoyamos; aquellos que han hecho toda la investigación e investigación que hicieron posible este trabajo y trabajaron para traer luz a la humanidad.

{\small
\bibliographystyle{ieee}
\bibliography{egbib}
}

\end{document}
